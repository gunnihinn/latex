\documentclass[10pt,a4paper]{amsart}

\usepackage{lmodern}
\linespread{1.1}
\usepackage[utf8]{inputenc}
\usepackage[T1]{fontenc}

\usepackage{fancyref}
\usepackage[colorlinks=true]{hyperref}

\usepackage{amsmath}
\usepackage{amssymb}
\usepackage{amsthm}

\newtheorem{theo}{Theorem}
\newtheorem{prop}[theo]{Proposition}
\newtheorem{lemm}[theo]{Lemma}
\newtheorem*{maintheo}{Theorem}
\newtheorem*{claim}{Claim}

\allowdisplaybreaks

\newcommand{\kk}[1]{\mathbb{#1}}
\newcommand{\cc}[1]{\mathcal{#1}}

\def\<{\langle}
\def\>{\rangle}

\def\qandq{\quad\text{and}\quad}
\def\ov#1{\overline{#1}}

\def\p{\partial}
\def\bp{\bar\partial}

\DeclareMathOperator{\Span}{span}
\DeclareMathOperator{\Ric}{Ric}
\DeclareMathOperator{\Gr}{Gr}
\DeclareMathOperator{\GL}{GL}
\DeclareMathOperator{\im}{Im}
\DeclareMathOperator{\Vol}{Vol}
\DeclareMathOperator{\Ker}{Ker}
\DeclareMathOperator{\End}{End}
\DeclareMathOperator{\Aut}{Aut}
\DeclareMathOperator{\Hom}{Hom}
\DeclareMathOperator{\Adj}{Adj}
\DeclareMathOperator{\id}{id}
\DeclareMathOperator{\tr}{tr}

\def\hsc{holomorphic sectional curvature}
\def\bl#1{\widehat{#1}}
\def\blX{\bl{X}}
\def\eps{\varepsilon}

\author{Gunnar \TH\'or Magn\'usson}
\address{Hafnarfj\"or\dh{}ur, Iceland}
\email{gunnar@magnusson.io}
\date{\today}
\title[Failing at blowups]
{How to fail to prove that blowups have\\positive holomorphic sectional curvature}

\hypersetup{
 pdfauthor={Gunnar Þór Magnússon},
 pdftitle={Positive holomorphic sectional curvature of blowups of points},
 pdfkeywords={},
 pdfsubject={},
 pdflang={English}}


\begin{document}


\begin{abstract}
We show that a K\"ahler metric of the form $\mu^*h + e^{-t} b$ on the blowup
of a point of a K\"ahler manifold of positive holomorphic sectional curvature
is not positively curved on the exceptional divisor.
\end{abstract}

\maketitle


\section*{Introduction}

Let $X$ be a compact K\"ahler manifold of dimension $n$.
Let $h$ be a K\"ahler metric on $X$.
We write $D$ for its Chern connection and
$R(\alpha,\ov\beta,\gamma,\ov\delta) = h(\frac i2
D^2_{\smash{\alpha,\ov\beta}}\gamma, \ov\delta)$ for its curvature tensor. The
holomorphic sectional curvature of $h$ is
$$
H(\xi)
= \frac{R(\xi, \ov\xi, \xi, \ov\xi)}{|\xi|^4},
$$
where $\xi$ is a nonzero tangent field.
We say that $h$ has positive holomorphic sectional curvature if $H > 0$ for all
tangent fields.
Examples of manifolds that carry such metrics are the complex projective space
with its Fubini--Study metric, the standard metric on a Grassmannian manifold,
and projective bundles over bases of positive holomorphic sectional curvature
~\cite{alvarez2018projectivized}
(and therefore also bundles of Grassmannians and flag manifolds over similar
bases).
Tsukamoto~\cite{tsukamoto1957kahlerian} showed that manifolds that admit such
metrics are simply connected, and recently Xiaokui Yang~\cite{yang2017rc}
proved that such manifolds are projective and rationally connected, answering a
question of Yau~\cite[Problem~67]{yau1993open}.

A related question of Yau (again Problem~67) is whether the blowup of a compact
K\"ahler manifold of positive \hsc{} along a smooth submanifold again has
positive \hsc.
This is open; even for complex surfaces it is not known whether del~Pezzo
surfaces admit positively curved metrics (aside from the del~Pezzo surfaces of
degrees 8 and 9, where the nontrivial one admits such a metric because it's
also a Hirzebruch surface and thus a projectivized bundle). In this note we
explain that the brute force approach to this problem does not work.

Prior art in this direction has focused on showing that projective bundles
over manifolds with positive \hsc{} are also positive, along with more
general fibrations with positively curved fibers and base.
The common thread in the papers of
Hitchin~\cite{hitchin1975curvature},
\'Alvarez~\cite{alvarez2016positive},
\'Alvarez,
Heier and Zheng~\cite{alvarez2018projectivized} or Chaturvedi and
Heier~\cite{chaturvedi2020hermitian}
is to consider metrics of the form $\mu^* h + e^{-t} b$, where $h$ is a metric
on the base and $b$ a fiberwise metric on the fibers and show they have
positive curvature for large enough $t$.
In this note we try this approach for the blowup of a point and show that
the resulting metric does not have positive \hsc{} on the exceptional divisor.

This of course doesn't mean the blowup doesn't have positive \hsc{}, only that
the most straightforward method of constructing a metric on it fails to show that.




\section{Our wrongs}

Let $X$ be a compact K\"ahler manifold of dimension $\dim_{\kk C} X = n$ that
admits a metric $h$ of positive \hsc.



\begin{prop}
\label{prop:fs}
Let $\mu : \bl X \to X$ be the blowup of $X$ at a point $p$.
There exists a closed $(1,1)$-form $\beta$ on $\blX$ whose restriction to
the exceptional divisor is the Fubini--Study metric.
\end{prop}

\begin{proof}
This is proved for blowups of smooth submanifolds in
Voisin's textbook~\cite{voisin2002theorie}. We can simplify that proof a little
since we're only blowing up points.

Locally around $p$, which may assume is the origin in a complex vector space
$V$, the blowup is
$$
\bl V
= \{ (v,[w]) \in V \times \kk P(V) \mid v \in \kk C w \}
$$
and $\mu$ is the projection onto the first factor.
Pick an inner product on $V$ and let $B(r) \subset V$ be a ball of radius $r$
centered at $0$.
We pick $r$ so that $B(r)$ fits into the coordinate chart that implicitly lurks
in the background.
Let $\psi$ be a bump function supported on that chart that is identically $1$
on $B(r)$.
The $(1,1)$-form $\frac i2 \partial\bar\partial \log |v|^2$ on $V \setminus
\{0\}$ descends to $\kk P(V)$ and defines the Fubini--Study metric.
If $p_j : V \times V \setminus \{0\} \to V$ for $j = 1,2$ are the projections
onto the first and second factors then
$\frac i2 \partial \bar\partial (p_1^*\psi \log |p_2^*v|^2)$
defines a closed $(1,1)$-form on $V \times \kk P(V)$ that restricts to the
pullback of the Fubini--Study metric by $p_2$ on $B(r) \times \kk P(V)$.
It also extends to the rest of $X$ by zero.
Its restriction to $\bl V$ is the form we want.
\end{proof}




Let $p \in X$ be a point and blow it up to obtain $\mu : \bl X \to X$.
Let $b$ be the Hermitian form associated to the $(1,1)$-form $\beta$ on $\bl X$
we constructed in Proposition~\ref{prop:fs}.
Then $h_t = \mu^*h + e^{-t} b$ is a K\"ahler metric on $\bl X$ for all $t$ large
enough.
If we can show that it has positive \hsc{} on a neighborhood around the
exceptional divisor for all $t \geq t_0$ then we can conclude that it eventually
has positive \hsc{} on all of $\blX$ by using Wu's~\cite{wu1973remark}
characterization of the \hsc{} to show positivity outside of that neighborhood
for $t$ large enough.


Recall that locally around $p$ the blowup is
$$
\bl V
= \{ (v,[w]) \in V \times \kk P(V) \mid v \in \kk C w \}.
$$
If $f \in \GL V$ then $f$ acts on the blowup by $f(v, [w]) = (f(v), [f(w)])$.
This is an isomorphism that maps the exceptional divisor to itself, and we can
map any point on the divisor to any other point on it.
Let $(0, [w])$ be a point on $E$.
Let's choose normal coordinates $(x_1,\ldots,x_n)$ centered at $p$.
There exists $f \in U(n)$ so that $f(w) = (0 \ldots, 0, 1)$, and the
coordinates obtained by applying $f$ to the old ones are still centered at $p$
and are normal there because $f \in U(n)$.
Picking the chart $\{(y_1, \ldots, y_n) \in \kk C^n \mid y_n \not= 0 \}$ for
$\kk P(V)$
we realize the blowup as
$$
\bl X
= \{ (x,y) \in \kk C^n \times \kk C^{n-1}
\mid x_j y_k = x_k y_j \text{ for $j,k = 1,\ldots,n$, where $y_n = 1$}  \}.
$$
In these coordinates the point we want to calculate the \hsc{} at is $(0,0)$
and we have $D_{h,\xi} \partial / \partial x_j = 0$ at $0$ for $j = 1, \ldots, n$.

We also note that close to the exceptional divisor, $h_t = \mu^* h + e^{-t} b$ is
just the restriction of the product metric $p_1^* h \oplus e^{-t} p_2^* g$ on
$V \times \kk P(V)$ to $\bl X$, where $p_j$ are the projections onto the
factors and $g$ is the Fubini--Study metric.

The only equations that give us any information at $(0,0)$ are
$$
x_j - y_j x_n = 0, \quad j = 1, \ldots, n-1.
$$
Their differentials are
$$
dx_j - y_j dx_n - x_n dy_j = 0, \quad j=1,\ldots,n-1
$$
and so the tangent fields
$$
\xi_j = x_n e_j + f_j,
\quad j=1,\ldots,n-1,
\qandq
\xi_n = \sum_{j=1}^{n-1} y_j e_j + e_n
$$
are a basis for
the intersection of all the kernels, that is, $T_{\blX}$.
Here $e_j$ is the tangent field corresponding to the coordinate $x_j$
and $f_k$ the one corresponding to $y_k$.
Note that the orthogonal complement of $T_{\blX}$ at $(0,0)$ is spanned by
$(e_1, \ldots, e_{n-1})$.

Any holomorphic tangent field $\xi$ near $(0,0)$ can be written as
$$
\xi = \sum_{j=1}^n a_j \xi_j
= \sum_{j=1}^{n-1} (a_j x_n + a_n y_j) e_j + a_j f_j
+ a_n e_n,
$$
where the $a_j$ are holomorphic functions.
We want to check the positivity of the curvature at the origin, which we can do
on the unit sphere, so we may assume that $\sum_{j=1}^n |a_j|^2 = 1$ there.

Recall that the curvature of $h_t$ is
$$
R_{h_t}(\xi, \ov\xi, \xi, \ov\xi)
= p_1^* R_h(\xi, \ov\xi, \xi, \ov\xi)
+ e^{-t} p_2^* R_g(\xi, \ov\xi, \xi, \ov\xi)
- |\sigma(\xi)\xi|^2,
$$
where $\sigma(\xi)\xi = \pi_N(D_{h \oplus e^{-t} g,\xi} \xi)$ is the second
fundamental form, and the norm is on the quotient bundle.

First,
$$
p_1^*R_h(\xi, \ov\xi, \xi, \ov\xi)
= |a_n|^4 H_h(e_n)
$$
at the origin. Second,
$$
p_2^*R_g(\xi, \ov\xi, \xi, \ov\xi)
= 2 \biggl(\sum_{j=1}^{n-1} |a_j|^2\biggr)^2
= 2(1 - |a_n|^2)^2
$$
at the origin
because the Fubini--Study metric has constant \hsc{} 2.

Let's now write $\pi_N$ for the projection onto the orthogonal complement of
$T_{\smash{\blX}}$ in $T_{\smash{\kk C^n \times \kk C^{n-1}|\blX}}$.
At the origin it is just the projection onto the first $n-1$ coordinates.
Third, then,
\begin{align*}
\pi_N(p_1^*D_{h,\xi} \xi)
= \sum_{j=1}^{n-1} d_{\xi}(a_j x_n + a_n y_j) \, e_j
&= \sum_{j=1}^{n-1} (a_j d_{\xi}x_n + a_n d_{\xi} y_j) \, e_j
\\
&= \sum_{j=1}^{n-1} (a_j a_n + a_n a_j ) \, e_j
= 2 a_n \sum_{j=1}^{n-1} a_j \, e_j
\end{align*}
at the origin
because $D_{h,\xi} e_j = 0$ there by our choice of coordinates.
And fourth,
$$
\pi_N(p_2^*D_{g,\xi} \xi)
= \pi_N \biggl( \sum_{j=1}^{n-1} d_\xi y_j \, f_j + y_j p_2^*D_{g,\xi} f_j \biggr) = 0
$$
at the origin.
The second fundamental form then contributes
$$
|\sigma(\xi)\xi|^2
= 4 |a_n|^2 \sum_{j=1}^{n-1} |a_j|^2 |e_j|^2
= 4 |a_n|^2(1 - |a_n|^2)
$$
to the curvature
so finally
\begin{align*}
R_{h_t}(\xi, \ov\xi, \xi, \ov\xi)
= |a_n|^4 H_h(e_n)
+ e^{-t} 2(1 - |a_n|^2)^2
- 4 |a_n|^2(1 - |a_n|^2).
\end{align*}
at the origin.
We would like this to be positive for all $a_n$ with $|a_n| \leq 1$ for all
$t$ large enough.
But note that the limit as $t \to \infty$ is
$$
|a_n|^2 (|a_n|^2 (H_h(e_n) + 4) - 4)
$$
which is negative for $|a_n|^2 < 4/(4 + H_h(e_N))$, so for $t$ large enough we
get negative \hsc{} in some directions.\footnote{%
We can be more precise: Consider the polynomial $p_t(x) = H_h(e_n) x^2
+ 2e^{-t}(1-x)^2 - 4x(1-x)$ so that
$R_{h_t}(\xi,\ov\xi,\xi,\ov\xi) = p_t(|a_n|^2)$.
Its global minimum is
$2e^{-t} - 2(1+e^{-t})^2/(H_h(e_n)+4+2e^{-t})$ which is negative for all large enough $t$.
}%




\bibliographystyle{plainurl}
\bibliography{main}

\end{document}
