\documentclass[10pt,a4paper]{amsart}

\usepackage{lmodern}
\linespread{1.1}
\usepackage[utf8]{inputenc}
\usepackage[T1]{fontenc}

\usepackage{fancyref}
\usepackage[colorlinks=true]{hyperref}

\usepackage{amsmath}
\usepackage{amssymb}
\usepackage{amsthm}

\newtheorem{theo}{Theorem}
\newtheorem{prop}[theo]{Proposition}
\newtheorem*{maintheo}{Theorem}

\newcommand{\kk}[1]{\mathbb{#1}}
\newcommand{\cc}[1]{\mathcal{#1}}

\def\<{\langle}
\def\>{\rangle}

\def\qandq{\quad\text{and}\quad}
\def\ov#1{\overline{#1}}

\DeclareMathOperator{\Span}{span}
\DeclareMathOperator{\Ric}{Ric}
\DeclareMathOperator{\Gr}{Gr}
\DeclareMathOperator{\GL}{GL}
\DeclareMathOperator{\im}{Im}
\DeclareMathOperator{\Vol}{Vol}
\DeclareMathOperator{\Ker}{Ker}
\DeclareMathOperator{\End}{End}
\DeclareMathOperator{\Aut}{Aut}
\DeclareMathOperator{\Hom}{Hom}
\DeclareMathOperator{\id}{id}
\DeclareMathOperator{\tr}{tr}

\def\hsc{holomorphic sectional curvature}
\def\bl#1{\widehat{#1}}
\def\blX{\bl{X}}

\author{Gunnar \TH\'or Magn\'usson}
\address{Hafnarfj\"or\dh{}ur, Iceland}
\email{gunnar@magnusson.io}
\date{\today}
\title[Positive holomorphic sectional curvature of blowups of points]
{Positive holomorphic sectional\\curvature of blowups of points}

\hypersetup{
 pdfauthor={Gunnar Þór Magnússon},
 pdftitle={Positive holomorphic sectional curvature of blowups of points},
 pdfkeywords={},
 pdfsubject={},
 pdflang={English}}


\begin{document}


\begin{abstract}
We show that the blowup of a point on a compact K\"ahler manifold of positive
holomorphic sectional curvature has positive holomorphic sectional curvature.
\end{abstract}

\maketitle


\section*{Introduction}

Let $X$ be a compact K\"ahler manifold of dimension $n$.
Let $h$ be a K\"ahler metric on $X$.
We write $D$ for its Chern connection and
$R(\alpha,\ov\beta,\gamma,\ov\delta) = h(\frac i2
D^2_{\smash{\alpha,\ov\beta}}\gamma, \ov\delta)$ for its curvature tensor. The
holomorphic sectional curvature of $h$ is
$$
H(\xi)
= \frac{R(\xi, \ov\xi, \xi, \ov\xi)}{|\xi|^4},
$$
where $\xi$ is a nonzero tangent field.
We say that $h$ has positive holomorphic sectional curvature if $H > 0$ for all
tangent fields.
Examples of manifolds that carry such metrics are the complex projective space
with its Fubini--Study metric, the standard metric on a Grassmannian manifold,
and projective bundles over bases of positive holomorphic sectional curvature
~\cite{alvarez2018projectivized}
(and therefore also bundles of Grassmannians and flag manifolds over similar
bases).
Tsukamoto~\cite{tsukamoto1957kahlerian} showed that manifolds that admit such
metrics are simply connected, and recently Xiaokui Yang~\cite{yang2017rc}
proved that such manifolds are projective and rationally connected, answering a
question of Yau~\cite[Problem~67]{yau1993open}.

A related question of Yau (again Problem~67) is whether the blowup of a compact
K\"ahler manifold of positive \hsc{} along a smooth submanifold again has
positive \hsc. In this note we claim to prove this for the blowup of a point:

\begin{maintheo}
Let $X$ be a compact K\"ahler manifold of dimension $n > 1$ that admits a
K\"ahler metric of positive \hsc.
If $\mu : \blX \to X$ is the blowup of $X$ at a point, then $\blX$ also admits
a K\"ahler metric of positive \hsc.
\end{maintheo}

In particular, this shows that del~Pezzo surfaces, which are blowups of the
complex projective plane in at most 9 points, admit K\"ahler metrics of
positive \hsc.

Our approach is one of brute force, essentially the same as that of
\'Alvarez~\cite{alvarez2016positive},
\'Alvarez,
Heier and Zheng~\cite{alvarez2018projectivized} or Chaturvedi and
Heier~\cite{chaturvedi2020hermitian} in the cases of projective bundles and
fiber bundles. It is known that there exists a closed $(1,1)$-form on the
blowup that restricts to the Fubini--Study metric on the exceptional divisor.
We choose local coordinates carefully, grit our teeth, and calculate the
curvature of that form plus a multiple of the pullback of the original metric
on the base using the classical Codazzi--Griffiths equations for the curvature
of a subbundle.



\section{Preliminaries}

\begin{prop}
\label{prop:positive}
Let $X$ be a compact K\"ahler manifold and let $h$ be a K\"ahler metric of
positive \hsc. Let $b$ be a Hermitian form on $X$.
Then $e^t h + b$ is a metric of positive \hsc{} for all $t \gg 0$.
\end{prop}

\begin{proof}
Chaturvedi and Heier~\cite{chaturvedi2020hermitian} prove this when $h$ and $b$
are both metrics.
We will basically reproduce their proof, which relies on
Wu's~\cite{wu1973remark} characterization of \hsc:
$$
H(x, \xi) = \sup_{f : D \to X} R_{f^*h}(0, \partial/\partial z),
$$
where $f : D \to X$ is an embedding of the unit disk that maps
$0$ to $x$ and $\partial/\partial z$ to $\xi$ at $0$.
The supremum can always be achieved; for K\"ahler metrics we do this by
choosing normal coordinates centered at a point such that $\xi$ is a multiple
of one of the coordinate fields at the center. (Wu handles general Hermitian
metrics.) When the supremum is achieved, we have $D_\xi \xi = 0$ at $x$ (by the
Codazzi--Griffiths equations for the curvature of a subbundle).

Let then $f : D \to X$ be an embedding that realizes the \hsc{} of $h$ at
$(x,\xi)$.
The curvature of the pullback of $h_t = e^t h + b$ to $D$ is
\begin{align*}
R_{f^*h}
= \partial_z \bar\partial_z \log(e^t |\xi|^2_h + |\xi|^2_b)
&= \partial_z \frac{e^t \<\xi, \ov{D_\xi\xi}\>_h + \bar\partial_\xi |\xi|^2_b}{e^t |\xi|^2_h + |\xi|^2_b}
\\
&= \frac{e^t H_h(\xi)|\xi|^4_h + e^t \<D_\xi \xi, \ov{D_\xi\xi}\>_h + \partial_\xi\bar\partial_\xi |\xi|^2_b}{e^t |\xi|^2_h + |\xi|^2_b}
\\
&\qquad
- \frac{e^t \<D_\xi \xi, \ov{\xi}\>_h + \partial_\xi |\xi|^2_b}{e^t |\xi|^2_h + |\xi|^2_b}
\frac{e^t \<\xi, \ov{D_\xi\xi}\>_h + \bar\partial_\xi |\xi|^2_b}{e^t |\xi|^2_h + |\xi|^2_b}.
\end{align*}
We have $D_\xi \xi = 0$ at $x$, so at the origin this simplifies to
\begin{align*}
R_{f^*h}
&= \frac{e^t H_h(\xi)|\xi|^4_h + \partial_\xi\bar\partial_\xi |\xi|^2_b}{e^t |\xi|^2_h + |\xi|^2_b}
- \frac{|\, \partial_\xi |\xi|^2_b \,|^2}{(e^t |\xi|^2_h + |\xi|^2_b)^2}
\\
&= H_h(\xi) |\xi|^2_h + O(e^{-t})
\end{align*}
which is positive for all $t$ large enough since $H_h$ is positive.

The \hsc{} is a smooth function on $\kk P(T_X)$, so if it is positive at a
point it is positive on a neighborhood of that point.
As $X$ is compact, we conclude that $h_t$ has positive \hsc{} for all $t \gg 0$.
\end{proof}


When $h$ and $b$ are both metrics there is a quicker proof:
Using the Codazzi--Griffits equations and the short exact sequence $0 \to T_X
\to T_X \oplus T_X \to T_X \to 0$ we calculate the curvature of $e^t h+b$ to be
$$
R(\xi, \ov\xi, \xi, \ov\xi)
= e^t R_h(\xi, \ov\xi, \xi, \ov\xi) + R_b(\xi, \ov\xi, \xi, \ov\xi)
- |\sigma(\xi)\xi|^2_{Q,t},
$$
where $\sigma(\xi)\xi = D_{h,\xi}\xi - D_{b,\xi} \xi$ is the second fundamental
form and the norm is on the ``quotient'' bundle.
If we simultaneously diagonalize the Hermitian forms at a point we see that
$|\eta|^2_{Q,t} \to |\eta|^2_b$ when $t \to \infty$ for any fixed tangent field
$\eta$. Then the \hsc{} of $e^t h + b$ is positive for all $t$ large enough (or
negative, if $h$ has negative curvature).



\begin{prop}
\label{prop:fs}
Let $\mu : \bl X \to X$ be the blowup of $X$ at a point $p$.
There exists a closed $(1,1)$-form $\beta$ on $X$ whose restriction to
the exceptional divisor is the Fubini--Study metric.
\end{prop}

\begin{proof}
This is proved for blowups of smooth submanifolds in
Voisin's textbook~\cite{voisin2002theorie}. We can simplify that proof a little
since we're only blowing up points.

Locally around $p$, which may assume is the origin in a complex vector space
$V$, the blowup is
$$
\bl V
= \{ (v,[w]) \in V \times \kk P(V) \mid v \in \kk C w \}
$$
and $\mu$ is the projection onto the first factor.
Pick an inner product on $V$ and let $B(r) \subset V$ be a ball of radius $r$
centered at $0$.
We pick $r$ so that $B(r)$ fits into the coordinate chart that implicitly lurks
in the background.
Let $\psi$ be a bump function supported on that chart that is identically $1$
on $B(r)$.
The $(1,1)$-form $\frac i2 \partial\bar\partial \log \|v\|^2$ on $V \setminus
\{0\}$ descends to $\kk P(V)$ and defines the Fubini--Study metric.
If $p_j : V \times V \setminus \{0\} \to V$ for $j = 1,2$ are the projections
onto the first and second factors then
$\frac i2 \partial \bar\partial (p_1^*\psi \log \|p_2^*v\|^2)$
defines a closed $(1,1)$-form on $V \times \kk P(V)$ that restricts to the
pullback of the Fubini--Study metric by $p_2$ on $B(r) \times \kk P(V)$.
It also extends to the rest of $X$ by zero.
Its restriction to $\bl V$ is the form we want.
\end{proof}



\section{The proof}

Let $X$ be a compact K\"ahler manifold of dimension $\dim_{\kk C} X = n$ that
admits a metric $h$ of positive \hsc.
Let $p \in X$ be a point and blow it up to obtain $\mu : \bl X \to X$.
Let $b$ be the Hermitian form associated to the $(1,1)$-form $\beta$ on $\bl X$
we constructed in Proposition~\ref{prop:fs}.
Then $h_t = e^t \mu^*h + b$ is a K\"ahler metric on $\bl X$ for all $t$ large
enough.
We are going to show it also eventually has positive \hsc.

It is actually enough to show that $h_t$ has positive \hsc{} on the exceptional
divisor.
If it does, then $h_t$ has positive \hsc{} on a neighborhood $U$ around the
divisor for all $t$ large enough.
Then Proposition~\ref{prop:positive} shows that $h_t$ also has positive \hsc{}
on $\bl X \setminus U$ for large enough $t$.

We're going to argue that we can calculate the \hsc{} on the exceptional divisor
at the center of a well-chosen coordinate chart.
Recall that locally around $p$ the blowup is
$$
\bl V
= \{ (v,[w]) \in V \times \kk P(V) \mid v \in \kk C w \}.
$$
If $f \in \GL V$ then $f$ acts on the blowup by $f(v, [w]) = (f(v), [f(w)])$.
This is an isomorphism that maps the exceptional divisor to itself, and we can
map any point on the divisor to any other point.

Let $(0, [w])$ be a point on $E$.
Let's choose normal coordinates $(x_1,\ldots,x_n)$ centered at $p$.
There exists $f \in U(n)$ so that $f(w) = (0 \ldots, 0, 1)$, and the
coordinates obtained by applying $f$ to the old ones are still centered at $p$
and are normal there because $f \in U(n)$.
Picking the chart $\{(y_1, \ldots, y_n) \in \kk C^n \mid y_n \not= 0 \}$ for
$\kk P(V)$
we realize the blowup as
$$
\bl X
= \{ (x,y) \in \kk C^n \times \kk C^{n-1}
\mid x_j y_k = x_k y_j \text{ for $j,k = 1,\ldots,n$, where $y_n = 1$}  \}.
$$
In these coordinates the point we want to calculate the \hsc{} at is $(0,0)$
and we have $D_{h,\xi} \partial / \partial x_j = 0$ at $0$ for $j = 1, \ldots, n$.

We also note that close to the exceptional divisor, $h_t = e^t \mu^* h + b$ is
just the restriction of the product metric $e^t p_1^* h \oplus p_2^* g$ on
$V \times \kk P(V)$ to $\bl X$, where $p_j$ are the projections onto the
factors and $g$ is the Fubini--Study metric.


\begin{proof}
The only equations that give us any information at $(0,0)$ are
$$
x_j - y_j x_n = 0, \quad j = 1, \ldots, n-1.
$$
Their differentials are
$$
\alpha_j := dx_j - y_j dx_n - x_n dy_j = 0, \quad j=1,\ldots,n-1
$$
and so the tangent fields
$$
\xi_j = x_n e_j + f_j,
\quad j=1,\ldots,n-1,
\qandq
\xi_n = \sum_{j=1}^{n-1} y_j e_j + e_n
$$
span the intersection of all the kernels, that is, $T_X$.
Here $e_j$ is the tangent field corresponding to the coordinate $x_j$
and $f_k$ the one corresponding to $y_k$.
Note that the normal bundle at $(0,0)$ is spanned by $(e_1, \ldots, e_{n-1})$.

Any holomorphic tangent field $\xi$ near $(0,0)$ can be written as
$$
\xi = \sum_{j=1}^n a_j \xi_j
= \sum_{j=1}^{n-1} (a_j x_n + a_n y_j) e_j + a_j f_j
+ a_n e_n,
$$
where the $a_j$ are holomorphic functions.
We want to check the positivity of the curvature at the origin, which we can do
on the unit sphere, so we may assume that $\sum_{j=1}^n |a_j|^2 = 1$ there.

Recall that the curvature of $h_t$ is
$$
R_{h_t}(\xi, \ov\xi, \xi, \ov\xi)
= e^t p_1^* R_h(\xi, \ov\xi, \xi, \ov\xi)
+ p_2^* R_g(\xi, \ov\xi, \xi, \ov\xi)
- |\sigma(\xi)\xi|^2,
$$
where $\sigma(\xi)\xi = \pi_N(D_{e^th \oplus g,\xi} \xi)$ is the second
fundamental form, and the norm is on the quotient bundle.

First,
$$
e^t p_1^*R_h(\xi, \ov\xi, \xi, \ov\xi)
= e^{-t} |a_n|^4 H_h(e_n)
$$
at the origin. Second,
$$
p_2^*R_g(\xi, \ov\xi, \xi, \ov\xi)
= 2 \biggl(\sum_{j=1}^{n-1} |a_j|^2\biggr)^2
= 2(1 - |a_n|^2)^2
$$
at the origin
because the Fubini--Study metric has constant \hsc{} 2.

Let's now write $\pi_N$ for the projection onto the orthogonal complement of
$T_{\blX}$ in $T_{\kk C^n \times \kk C^{n-1}|\blX}$.
At the origin it is just the projection onto the first $n-1$ coordinates.

Third, then,
\begin{align*}
\pi_N(p_1^*D_{h,\xi} \xi)
&= \sum_{j=1}^{n-1} d_{\xi}(a_j x_n + a_n y_j) \, e_j
\\
&= \sum_{j=1}^{n-1} (a_j d_{\xi}x_n + a_n d_{\xi} y_j) \, e_j
\\
&= \sum_{j=1}^{n-1} (a_j a_n + a_n a_j ) \, e_j
= 2 a_n \sum_{j=1}^{n-1} a_j \, e_j
\end{align*}
at the origin
because $D_{h,\xi} e_j = 0$ there by our choice of coordinates.

And fourth,
$$
\pi_N(p_2^*D_{g,\xi} \xi)
= \pi_N \biggl( \sum_{j=1}^{n-1} d_\xi y_j \, f_j + y_j p_2^*D_{g,\xi} f_j \biggr) = 0
$$
at the origin.
The contribution of the second fundamental form to the curvature is then
$$
|\sigma(\xi)\xi|^2
= 4 |a_n|^2 \sum_{j=1}^{n-1} |a_j|^2 |e_j|^2
= 4 e^{-t} |a_n|^2 \sum_{j=1}^{n-1} |a_j|^2
= 4 e^{-t} |a_n|^2(1 - |a_n|^2)
$$
so finally
$$
R_{h_t}(\xi, \ov\xi, \xi, \ov\xi)
= e^{-t} |a_n|^4 H_h(e_n)
+ 2(1 - |a_n|^2)^2
- 4 e^{-t} |a_n|^2(1 - |a_n|^2)
$$
at the origin.
We have to show that this is positive for all $a_n$ with $|a_n| \leq 1$ for $t$
large enough.

Set $b = \inf_{\kk P(T_X)} H_h > 0$ and let
$$
f(x) = e^{-t} b x^2 + 2(1-x)^2 - 4e^{-t} x(1-x).
$$
Then $R_{h_t}(\xi, \ov\xi, \xi, \ov\xi) \geq f(|a_n|^2)$.
We have
$$
f(x) = e^{-t}(b x^2 - 4 x(1-x)) + 2(1-x)^2
=: e^{-t} p(x) + q(x).
$$
Note that $p(1) = b > 0$.
If $p$ is positive on $[0,1]$ we are done, so we may assume it has a root. Let
$m = p(\frac12) = b/4-1$ be the minimum of $p$ on $[0,1]$ and let $0 \leq x_0 <
1$ be the larger root of $p$ on $[0,1]$.
Then $f(x) > 0$ for $x > x_0$ and $f(x) \geq e^{-t} m + q(x_0)$ for $0 \leq x
\leq x_0$, as $q$ achieves its maximum at $0$.
But then $f(x) \geq q(x_0)/2 > 0$ for all $t$ larger than $t_0$ such that
$e^{t_0}m = -q(x_0)/2$, and as both $m$ and $x_0$ only depend on $b$ this $t_0$
can be chosen uniformely over the manifold.
\end{proof}



It might be possible to extend this approach to cover the blowup of a smooth
submanifold of a compact K\"ahler manifold of positive \hsc.
The problem I see is that the blowup along a submanifold $Y \subset X$ is
constructed by picking local coordinates $(z_1,\ldots,z_n)$ such that $Y = \{z
\mid z_{k+1} = \cdots = z_n = 0\}$,
and the standard method of producing normal coordinates centered at a point
(see for example Zheng~\cite{zheng2000complex})
does not preserve these equations.
Our trick here thus does not carry through verbatim and there might be extra
difficulties involved in the curvature calculations.
If any enterprising soul takes this on I'd be happy to hear about it.






\bibliographystyle{plain}
\bibliography{main}

\end{document}
