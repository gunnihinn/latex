\documentclass[10pt,a4paper]{amsart}

\usepackage{lmodern}
\linespread{1.1}
\usepackage[utf8]{inputenc}
\usepackage[T1]{fontenc}

\usepackage{fancyref}
\usepackage[colorlinks=true]{hyperref}

\usepackage{amsmath}
\usepackage{amssymb}
\usepackage{amsthm}

\newtheorem{theo}{Theorem}
\newtheorem{prop}[theo]{Proposition}
\newtheorem*{maintheo}{Theorem}

\newcommand{\kk}[1]{\mathbb{#1}}
\newcommand{\cc}[1]{\mathcal{#1}}

\def\<{\langle}
\def\>{\rangle}

\def\qandq{\quad\text{and}\quad}
\def\ov#1{\overline{#1}}

\DeclareMathOperator{\Span}{span}
\DeclareMathOperator{\Ric}{Ric}
\DeclareMathOperator{\Gr}{Gr}
\DeclareMathOperator{\GL}{GL}
\DeclareMathOperator{\im}{Im}
\DeclareMathOperator{\Vol}{Vol}
\DeclareMathOperator{\Ker}{Ker}
\DeclareMathOperator{\End}{End}
\DeclareMathOperator{\Aut}{Aut}
\DeclareMathOperator{\Hom}{Hom}
\DeclareMathOperator{\id}{id}
\DeclareMathOperator{\tr}{tr}

\def\hsc{holomorphic sectional curvature}
\def\bl#1{\widehat{#1}}
\def\blX{\bl{X}}

\author{Gunnar \TH\'or Magn\'usson}
\address{Hafnarfj\"or\dh{}ur, Iceland}
\email{gunnar@magnusson.io}
\date{\today}
\title[Positive holomorphic sectional curvature of blowups of points]
{Positive holomorphic sectional\\curvature of blowups of points}

\hypersetup{
 pdfauthor={Gunnar Þór Magnússon},
 pdftitle={Positive holomorphic sectional curvature of blowups of points},
 pdfkeywords={},
 pdfsubject={},
 pdflang={English}}


\begin{document}


\begin{abstract}
We show that the blowup of a point on a compact K\"ahler manifold of positive
holomorphic sectional curvature has positive holomorphic sectional curvature.
\end{abstract}

\maketitle


\section*{Introduction}

Let $X$ be a compact K\"ahler manifold of dimension $n$.
Let $h$ be a K\"ahler metric on $X$ and denote its curvature tensor by $R$.
The holomorphic sectional curvature of $h$ is
$$
H(\xi)
= \frac{R(\xi, \ov\xi, \xi, \ov\xi)}{|\xi|^4},
$$
where $\xi$ is a nonzero tangent field.
We say that $h$ has positive holomorphic sectional curvature if $H > 0$ for all
tangent fields.
Examples of such metrics are the Fubini--Study metric on projective space,
the standard metric on a Grassmannian manifold,
and projective bundles over bases of positive holomorphic sectional curvature
~\cite{alvarez2018projectivized}
(and therefore also bundles of Grassmannians and flag manifolds over similar
bases).
Recently Xiaokui Yang~\cite{yang2017rc} proved that such manifolds are
projective and rationally connected, confirming a conjecture of
Yau~\cite{yau1993open}.

A related question of Yau is whether the blowup of a compact K\"ahler manifold
of positive \hsc{} again has positive \hsc.
In this note we claim to prove this for the blowup of a point:

\begin{maintheo}
Let $X$ be a compact K\"ahler manifold of dimension $n > 1$ that admits a
K\"ahler metric of positive \hsc.
If $\mu : \blX \to X$ is the blowup of $X$ at a point, then $\blX$ also admits
a K\"ahler metric of positive \hsc.
\end{maintheo}

Our approach is one of brute force, essentially the same as that of \'Alvarez
and her collaborators~\cite{alvarez2018projectivized}. It is known that there
exists a closed $(1,1)$-form on the blowup that restricts to the Fubini--Study
metric on the exceptional divisor. We choose local coordinates carefully and
calculate the curvature of that form plus a multiple of the pullback of the
original metric on the base.



\section{Preliminaries}

\begin{prop}
\label{prop:positive}
Let $X$ be a compact K\"ahler manifold and let $h$ be a K\"ahler metric of
positive \hsc. Let $b$ be a Hermitian form on $X$.
Then $e^t h + b$ is a metric of positive \hsc{} for all $t \gg 0$.
\end{prop}

\begin{proof}
Chaturvedi and Heier~\cite{chaturvedi2020hermitian} prove this when $h$ and $b$
are both metrics.
We will basically just reproduce their proof, which relies on
Wu's~\cite{wu1973remark} characterization of \hsc:
$$
H(x, \xi) = \sup_{f : D \to X} R_{f^*h}(0, \partial/\partial z),
$$
where $f : D \to X$ is an embedding of the unit disk that maps
$0$ to $x$ and $\partial/\partial z$ to $\xi$ at $0$.
The supremum can always be achieved; for K\"ahler metrics by choosing normal
coordinates centered at a point such that $\xi$ is a multiple of one of the
coordinate fields at the center.
When the supremum is achieved, we have $D_\xi \xi = 0$ at $x$ (by the
Codazzi--Griffiths equations for the curvature of a subbundle).

Let then $f : D \to X$ be an embedding that realizes the \hsc{} of $h$ at
$(x,\xi)$.
The curvature of the pullback of $h_t = e^t h + b$ to $D$ is
\begin{align*}
R_{f^*h}
= \partial_z \bar\partial_z \log(e^t |\xi|^2_h + |\xi|^2_b)
&= \partial_z \frac{e^t \<\xi, \ov{D_\xi\xi}\>_h + \bar\partial_\xi |\xi|^2_b}{e^t |\xi|^2_h + |\xi|^2_b}
\\
&= \frac{e^t H_h(\xi)|\xi|^4_h + e^t \<D_\xi \xi, \ov{D_\xi\xi}\>_h + \partial_\xi\bar\partial_\xi |\xi|^2_b}{e^t |\xi|^2_h + |\xi|^2_b}
\\
&\qquad
- \frac{e^t \<D_\xi \xi, \ov{\xi}\>_h + \partial_\xi |\xi|^2_b}{e^t |\xi|^2_h + |\xi|^2_b}
\frac{e^t \<\xi, \ov{D_\xi\xi}\>_h + \bar\partial_\xi |\xi|^2_b}{e^t |\xi|^2_h + |\xi|^2_b}.
\end{align*}
We have $D_\xi \xi = 0$ at $x$, so at the origin this simplifies to
\begin{align*}
R_{f^*h}
&= \frac{e^t H_h(\xi)|\xi|^4_h + \partial_\xi\bar\partial_\xi |\xi|^2_b}{e^t |\xi|^2_h + |\xi|^2_b}
- \frac{|\, \partial_\xi |\xi|^2_b \,|^2}{(e^t |\xi|^2_h + |\xi|^2_b)^2}
\\
&= H_h(\xi) |\xi|^2_h + O(e^{-t})
\end{align*}
which is positive for all $t$ large enough since $H_h$ is positive.

The \hsc{} is a smooth function on $\kk P(T_X)$, so if it is positive at a
point it is positive on a neighborhood of that point.
As $X$ is compact, we conclude that $h_t$ has positive \hsc{} for all $t \gg 0$.
\end{proof}


\begin{prop}
\label{prop:fs}
Let $\mu : \bl X \to X$ be the blowup of $X$ at a point $p$.
There exists a closed $(1,1)$-form $\beta$ on $X$ whose restriction to
the exceptional divisor is the Fubini--Study metric.
\end{prop}

\begin{proof}
This is proved for blowups of smooth submanifolds in
Voisin's textbook~\cite{voisin2002theorie}. We can simplify that proof a little
since we're only blowing up points.

Locally around $p$, which may assume is the origin in a complex vector space
$V$, the blowup is
$$
\bl V
= \{ (v,[w]) \in V \times \kk P(V) \mid v \in \kk C w \}
$$
and $\mu$ is the projection onto the first factor.
Pick an inner product on $V$ and let $B(r) \subset V$ be a ball of radius $r$
centered at $0$.
We pick $r$ so that $B(r)$ fits into the coordinate chart that implicitly lurks
in the background.
Let $\psi$ be a bump function supported on that chart that is identically $1$
on $B(r)$.
The $(1,1)$-form $\frac i2 \partial\bar\partial \log \|v\|^2$ on $V \setminus
\{0\}$ descends to $\kk P(V)$ and defines the Fubini--Study metric.
If $p_j : V \times V \setminus \{0\} \to V$ for $j = 1,2$ are the projections
onto the first and second factors then
$\frac i2 \partial \bar\partial (p_1^*\psi \log \|p_2^*v\|^2)$
defines a closed $(1,1)$-form on $V \times \kk P(V)$ that restricts to the
pullback of the Fubini--Study metric by $p_2$ on $B(r) \times \kk P(V)$.
It also extends to the rest of $X$ by zero.
Its restriction to $\bl V$ is the form we want.
\end{proof}



\section{The proof}

Let $X$ be a compact K\"ahler manifold of dimension $\dim_{\kk C} X = n$ that
admits a metric $h$ of positive \hsc.
Let $p \in X$ be a point and blow it up to obtain $\mu : \bl X \to X$.
Let $b$ be the Hermitian form associated to the $(1,1)$-form $\beta$ on $\bl X$
we constructed in Proposition~\ref{prop:fs}.
Then $h_t = e^t \mu^*h + b$ is a K\"ahler metric on $\bl X$ for all $t$ large
enough.
We are going to show it also eventually has positive \hsc.

It is actually enough to show that $h_t$ has positive \hsc{} on the exceptional
divisor.
If it does, then $h_t$ has positive \hsc{} on a neighborhood $U$ around the
divisor for all $t$ large enough.
Then Proposition~\ref{prop:positive} shows that $h_t$ also has positive \hsc{}
on $\bl X \setminus U$ for large enough $t$.

We're going to argue that we can calculate the \hsc{} on the exceptional divisor
at the center of a well-chosen coordinate chart.
Recall that locally around $p$ the blowup is
$$
\bl V
= \{ (v,[w]) \in V \times \kk P(V) \mid v \in \kk C w \}.
$$
If $f \in \GL V$ then $f$ acts on the blowup by $f(v, [w]) = (f(v), [f(w)])$.
This is an isomorphism that maps the exceptional divisor to itself, and we can
map any point on the divisor to any other point.

Let $(0, [w])$ be a point on $E$.
Let's choose normal coordinates $(x_1,\ldots,x_n)$ centered at $p$.
There exists $f \in U(n)$ so that $f(w) = (0 \ldots, 0, 1)$, and the
coordinates obtained by applying $f$ to the old ones are still centered at $p$
and normal there because $f \in U(n)$.
Picking the chart $\{(y_1, \ldots, y_n) \in \kk C^n \mid y_n \not= 0 \}$ for
$\kk P(V)$
we realize the blowup as
$$
\bl X
= \{ (x,y) \in \kk C^n \times \kk C^{n-1}
\mid x_j y_k = x_k y_j \text{ for $j,k = 1,\ldots,n$, where $y_n = 1$}  \}.
$$
In these coordinates the point we want to calculate the \hsc{} at is $(0,0)$
and we have $D_{h,\xi} \partial / \partial x_j = 0$ at $0$ for $j = 1, \ldots, n$.

We also note that close to the exceptional divisor, $h_t = e^t \mu^* h + b$ is
just the restriction of the product metric $e^t p_1^* h \oplus p_2^* g$ on
$V \times \kk P(V)$ to $\bl X$, where $p_j$ are the projections onto the
factors and $g$ is the Fubini--Study metric.


\begin{proof}[Proof for surfaces]
Let's consider surfaces first to simplify things a little.
Our local picture is then
$$
X
= \{ (x, y, z) \in \kk C^3
\mid x = yz \}.
$$
The tangent space of $X$ is
$$
T_{X} = \ker(dx - z dy - y dz) \subset T_{\kk C^3|X}.
$$
We write $\alpha = dx - z dy - y dz$ on $T_{\kk C^3}$.
Note that $T_X$ is spanned by $u = z e_1 + e_2$ and $v = y e_1 + e_3$ in a
neighborhood around $E$, where $e_j$ are the coordinate tangent fields on $\kk
C^3$.

Recall that the curvature of $h_t$ is
$$
R_{h_t}(\xi, \ov\xi, \xi, \ov\xi)
= e^t p_1^* R_h(\xi, \ov\xi, \xi, \ov\xi)
+ p_2^* R_g(\xi, \ov\xi, \xi, \ov\xi)
- |\sigma(\xi)\xi|^2,
$$
where $\sigma(\xi)\xi = q(D_{e^th \oplus g,\xi} \xi)$ is the second fundamental
form and $q$ is the projection to the normal bundle of $X$, and the norm is on
the quotient bundle.

On $E$, the line $\kk C e_1 \subset T_{\kk C^3}$ is orthogonal to $T_X =
\ker dx$ with respect to the product metric $e^t h \oplus g$, so it represents
the normal bundle there.
We have $D_{e^th \oplus g} = D_h \oplus D_g$.
If $\xi$ is tangent to $X$ around $E$ we can write $\xi = a u + bv = (az + by)
e_1 + a e_2 + b e_3$ for holomorphic functions $a$ and $b$. We are only
interested in checking positivity on a compact set so we may assume that $|a|^2
+ |b|^2 = 1$ at $(0,0,0)$.

We have
$$
p_3^* D_{g,\xi} \xi
= D_{g,\xi} (be_3)
= d_\xi b \, e_3 + b D_{g,\xi} e_3,
$$
which is a multiple of $e_3$ and thus contributes nothing to the second
fundamental form.
We also have
$$
p_{12}^* D_{h,\xi}\xi
= D_{h,\xi}((az + by) e_1 + ae_2)
= 2ab \, e_1 + d_\xi a \, e_2
$$
at $(0,0,0)$ because $D_{h,\xi} e_1 = D_{h,\xi} e_2 = 0$ and $y = z = 0$ there.
The second fundamental form is then
$$
\sigma(\xi)\xi = 2 ab
$$
at the origin, and its norm is
$$
|\sigma(\xi)\xi|^2
= 4 |ab|^2 / |e_1|^2
= e^{-t} 4 |ab|^2.
$$
Recall that the Fubini--Study metric has constant \hsc{} 2 and that the norm of
$e_3$ at the origin is $1$. Then
\begin{align*}
R_{h_t}(\xi, \ov\xi, \xi, \ov\xi)
&= e^t p_1^* R_h(\xi, \ov\xi, \xi, \ov\xi)
+ p_2^* R_g(\xi, \ov\xi, \xi, \ov\xi)
- |\sigma(\xi)\xi|^2
\\
&= e^{-t} |a|^4 H_h(e_2)
+ 2 |b|^4
- 4 e^{-t} |a|^2|b|^2.
\end{align*}
If $b = 0$ at the origin then $a = 1$ so $e^t H_h(e_2) > 0$ for any $t$.
Otherwise
$$
R_{h_t}(\xi, \ov\xi, \xi, \ov\xi) \to 2|b|^4 > 0
$$
as $t \to \infty$ so $h_t$ has positive \hsc{} at $(0,0,0)$ for all $t$ large
enough.
\end{proof}


\begin{proof}
In general the only equations that give us any information at $(0,0)$ are
$$
x_j - y_j x_n = 0, \quad j = 1, \ldots, n-1.
$$
Their differentials are
$$
\alpha_j := dx_j - y_j dx_n - x_n dy_j = 0, \quad j=1,\ldots,n-1
$$
and so the tangent fields
$$
\xi_j = x_n e_j + f_j,
\quad j=1,\ldots,n-1,
\qandq
\xi_n = \sum_{j=1}^{n-1} y_j e_j + e_n
$$
span the intersection of all the kernels, that is, $T_X$.
Here $e_j$ is the field corresponding to the coordinate $x_j$
and $f_k$ the one corresponding to $y_k$.
Note that the normal bundle at $(0,0)$ is spanned by $(e_1, \ldots, e_{n-1})$.

Any holomorphic tangent field $\xi$ near $(0,0)$ can be written as
$$
\xi = \sum_{j=1}^n a_j \xi_j
= \sum_{j=1}^{n-1} (a_j x_n + a_n y_j) e_j + a_j f_j
+ a_n e_n,
$$
where the $a_j$ are holomorphic functions.
We want to check the positivity of the curvature at the origin, which we can do
on the unit sphere, so we may assume that $\sum_{j=1}^n |a_j|^2 = 1$ there.

Recall that the curvature of $h_t$ is
$$
R_{h_t}(\xi, \ov\xi, \xi, \ov\xi)
= e^t p_1^* R_h(\xi, \ov\xi, \xi, \ov\xi)
+ p_2^* R_g(\xi, \ov\xi, \xi, \ov\xi)
- |\sigma(\xi)\xi|^2,
$$
where $\sigma(\xi)\xi = \pi_N(D_{e^th \oplus g,\xi} \xi)$ is the second
fundamental form, and the norm is on the quotient bundle.

First,
$$
e^t p_1^*R_h(\xi, \ov\xi, \xi, \ov\xi)
= e^{-t} |a_n|^4 H_h(e_n)
$$
at the origin. Second,
$$
p_2^*R_g(\xi, \ov\xi, \xi, \ov\xi)
= 2 \biggl(\sum_{j=1}^{n-1} |a_j|^2\biggr)^2
= 2(1 - |a_n|^2)^2
$$
at the origin
because the Fubini--Study metric has constant \hsc{} 2.

Let's now write $\pi_N$ for the projection onto the orthogonal complement of
$T_{\blX}$ in $T_{\kk C^n \times \kk C^{n-1}|\blX}$.
At the origin it is just the projection onto the first $n-1$ coordinates.

Third, then,
\begin{align*}
\pi_N(p_1^*D_{h,\xi} \xi)
&= \sum_{j=1}^{n-1} d_{\xi}(a_j x_n + a_n y_j) \, e_j
\\
&= \sum_{j=1}^{n-1} (a_j d_{\xi}x_n + a_n d_{\xi} y_j) \, e_j
\\
&= \sum_{j=1}^{n-1} (a_j a_n + a_n a_j ) \, e_j
= 2 a_n \sum_{j=1}^{n-1} a_j \, e_j
\end{align*}
at the origin
because $D_{h,\xi} e_j = 0$ there by our choice of coordinates.

And fourth,
$$
\pi_N(p_2^*D_{g,\xi} \xi)
= \pi_N \biggl( \sum_{j=1}^{n-1} d_\xi y_j \, f_j + y_j p_2^*D_{g,\xi} f_j \biggr) = 0
$$
at the origin.
The contribution of the second fundamental form to the curvature is then
$$
|\sigma(\xi)\xi|^2
= 4 |a_n|^2 \sum_{j=1}^{n-1} |a_j|^2 |e_j|^2
= 4 e^{-t} |a_n|^2 \sum_{j=1}^{n-1} |a_j|^2
= 4 e^{-t} |a_n|^2(1 - |a_n|^2)
$$
so finally
$$
R_{h_t}(\xi, \ov\xi, \xi, \ov\xi)
= e^{-t} |a_n|^4 H_h(e_n)
+ 2(1 - |a_n|^2)^2
- 4 e^{-t} |a_n|^2(1 - |a_n|^2)
$$
at the origin.
If $|a_n| = 1$ we have just $e^{-t} H_h(e_n) > 0$;
otherwise
$$
R_{h_t}(\xi, \ov\xi, \xi, \ov\xi) \to 2(1 - |a_n|^2)^2 > 0
$$
as $t \to \infty$,
so $h_t$ has positive \hsc{} at $(0,0)$ for all $t$ large enough.
\end{proof}







\bibliographystyle{plain}
\bibliography{main}

\end{document}
