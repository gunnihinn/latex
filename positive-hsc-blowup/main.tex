\documentclass[10pt,a4paper]{amsart}

\usepackage{lmodern}
\linespread{1.1}
\usepackage[utf8]{inputenc}
\usepackage[T1]{fontenc}

\usepackage{fancyref}
\usepackage[colorlinks=true]{hyperref}

\usepackage{amsmath}
\usepackage{amssymb}
\usepackage{amsthm}

\newtheorem{theo}{Theorem}
\newtheorem{prop}[theo]{Proposition}
\newtheorem{lemm}[theo]{Lemma}
\newtheorem*{maintheo}{Theorem}
\newtheorem*{claim}{Claim}

\allowdisplaybreaks

\newcommand{\kk}[1]{\mathbb{#1}}
\newcommand{\cc}[1]{\mathcal{#1}}

\def\<{\langle}
\def\>{\rangle}

\def\qandq{\quad\text{and}\quad}
\def\ov#1{\overline{#1}}

\def\p{\partial}
\def\bp{\bar\partial}

\DeclareMathOperator{\Span}{span}
\DeclareMathOperator{\Ric}{Ric}
\DeclareMathOperator{\Gr}{Gr}
\DeclareMathOperator{\GL}{GL}
\DeclareMathOperator{\im}{Im}
\DeclareMathOperator{\Vol}{Vol}
\DeclareMathOperator{\Ker}{Ker}
\DeclareMathOperator{\End}{End}
\DeclareMathOperator{\Aut}{Aut}
\DeclareMathOperator{\Hom}{Hom}
\DeclareMathOperator{\Adj}{Adj}
\DeclareMathOperator{\id}{id}
\DeclareMathOperator{\tr}{tr}

\def\hsc{holomorphic sectional curvature}
\def\bl#1{\widehat{#1}}
\def\blX{\bl{X}}
\def\eps{\varepsilon}

\author{Gunnar \TH\'or Magn\'usson}
\address{Hafnarfj\"or\dh{}ur, Iceland}
\email{gunnar@magnusson.io}
\date{\today}
\title[Positive holomorphic sectional curvature of blowups of points]
{Positive holomorphic sectional\\curvature of blowups of points}

\hypersetup{
 pdfauthor={Gunnar Þór Magnússon},
 pdftitle={Positive holomorphic sectional curvature of blowups of points},
 pdfkeywords={},
 pdfsubject={},
 pdflang={English}}


\begin{document}


\begin{abstract}
We show that the blowup of a point on a compact K\"ahler manifold of positive
holomorphic sectional curvature has positive holomorphic sectional curvature.
\end{abstract}

\maketitle


\section*{Introduction}

Let $X$ be a compact K\"ahler manifold of dimension $n$.
Let $h$ be a K\"ahler metric on $X$.
We write $D$ for its Chern connection and
$R(\alpha,\ov\beta,\gamma,\ov\delta) = h(\frac i2
D^2_{\smash{\alpha,\ov\beta}}\gamma, \ov\delta)$ for its curvature tensor. The
holomorphic sectional curvature of $h$ is
$$
H(\xi)
= \frac{R(\xi, \ov\xi, \xi, \ov\xi)}{|\xi|^4},
$$
where $\xi$ is a nonzero tangent field.
We say that $h$ has positive holomorphic sectional curvature if $H > 0$ for all
tangent fields.
Examples of manifolds that carry such metrics are the complex projective space
with its Fubini--Study metric, the standard metric on a Grassmannian manifold,
and projective bundles over bases of positive holomorphic sectional curvature
~\cite{alvarez2018projectivized}
(and therefore also bundles of Grassmannians and flag manifolds over similar
bases).
Tsukamoto~\cite{tsukamoto1957kahlerian} showed that manifolds that admit such
metrics are simply connected, and recently Xiaokui Yang~\cite{yang2017rc}
proved that such manifolds are projective and rationally connected, answering a
question of Yau~\cite[Problem~67]{yau1993open}.

A related question of Yau (again Problem~67) is whether the blowup of a compact
K\"ahler manifold of positive \hsc{} along a smooth submanifold again has
positive \hsc. In this note we claim to prove this for the blowup of a point:

\begin{maintheo}
Let $X$ be a compact K\"ahler manifold of dimension $n > 1$ that admits a
K\"ahler metric of positive \hsc.
If $\mu : \blX \to X$ is the blowup of $X$ at a point, then $\blX$ also admits
a K\"ahler metric of positive \hsc.
\end{maintheo}

In particular, this shows that del~Pezzo surfaces, which are blowups of the
complex projective plane in at most 9 points, admit K\"ahler metrics of
positive \hsc.

Our approach is one of brute force, essentially the same as that of
\'Alvarez~\cite{alvarez2016positive},
\'Alvarez,
Heier and Zheng~\cite{alvarez2018projectivized} or Chaturvedi and
Heier~\cite{chaturvedi2020hermitian} in the cases of projective bundles and
fiber bundles. It is known that there exists a closed $(1,1)$-form on the
blowup that restricts to the Fubini--Study metric on the exceptional divisor.
We choose local coordinates carefully, grit our teeth, and calculate the
curvature of that form plus a multiple of the pullback of the original metric
on the base using the classical Codazzi--Griffiths equations for the curvature
of a subbundle.



\section{Preliminaries}

\begin{prop}
\label{prop:positive}
Let $X$ be a compact K\"ahler manifold and let $h$ be a K\"ahler metric of
positive \hsc. Let $b$ be a Hermitian form on $X$.
Then $h + e^{-t} b$ is a metric of positive \hsc{} for all $t \gg 0$.
\end{prop}

\begin{proof}
Chaturvedi and Heier~\cite{chaturvedi2020hermitian} prove this when $h$ and $b$
are both metrics.
We will basically reproduce their proof, which relies on
Wu's~\cite{wu1973remark} characterization of \hsc:
$$
H(x, \xi) = \sup_{f : D \to X} R_{f^*h}(0, \partial/\partial z),
$$
where $f : D \to X$ is an embedding of the unit disk that maps
$0$ to $x$ and $\partial/\partial z$ to $\xi$ at $0$.
The supremum can always be achieved; for K\"ahler metrics we do this by
choosing normal coordinates centered at a point such that $\xi$ is a multiple
of one of the coordinate fields at the center. (Wu handles general Hermitian
metrics.) When the supremum is achieved, we have $D_\xi \xi = 0$ at $x$ (by the
Codazzi--Griffiths equations for the curvature of a subbundle).

Let then $f : D \to X$ be an embedding that realizes the \hsc{} of $h$ at
$(x,\xi)$.
The curvature of the pullback of $h_t = e^t h + b$ to $D$ is
\begin{align*}
R_{f^*h}
= \partial_z \bar\partial_z \log(|\xi|^2_h + e^{-t} |\xi|^2_b)
&= \partial_z \frac{\<\xi, \ov{D_\xi\xi}\>_h + e^{-t} \bar\partial_\xi |\xi|^2_b}{|\xi|^2_h + e^{-t} |\xi|^2_b}
\\
&= \frac{H_h(\xi)|\xi|^4_h + \<D_\xi \xi, \ov{D_\xi\xi}\>_h + e^{-t} \partial_\xi\bar\partial_\xi |\xi|^2_b}{|\xi|^2_h + e^{-t} |\xi|^2_b}
\\
&\qquad
- \frac{\<D_\xi \xi, \ov{\xi}\>_h + e^{-t} \partial_\xi |\xi|^2_b}{|\xi|^2_h + e^{-t} |\xi|^2_b}
\frac{\<\xi, \ov{D_\xi\xi}\>_h + e^{-t} \bar\partial_\xi |\xi|^2_b}{|\xi|^2_h + e^{-t} |\xi|^2_b}.
\end{align*}
We have $D_\xi \xi = 0$ at $x$, so at the origin this simplifies to
\begin{align*}
R_{f^*h}
&= \frac{H_h(\xi)|\xi|^4_h + e^{-t} \partial_\xi\bar\partial_\xi |\xi|^2_b}{|\xi|^2_h + e^{-t} |\xi|^2_b}
- \frac{|\, \partial_\xi |\xi|^2_b \,|^2}{(|\xi|^2_h + e^{-t} |\xi|^2_b)^2}
\\
&= H_h(\xi) |\xi|^2_h + O(e^{-t})
\end{align*}
which is positive for all $t$ large enough since $H_h$ is positive.

The \hsc{} is a smooth function on $\kk P(T_X)$, so if it is positive at a
point it is positive on a neighborhood of that point.
As $X$ is compact, we conclude that $h_t$ has positive \hsc{} for all $t \gg 0$.
\end{proof}


When $h$ and $b$ are both metrics there is a quicker proof:
Using the Codazzi--Griffits equations and the short exact sequence $0 \to T_X
\to T_X \oplus T_X \to T_X \to 0$ we calculate the curvature of $h+e^{-t} b$ to be
$$
R(\xi, \ov\xi, \xi, \ov\xi)
= R_h(\xi, \ov\xi, \xi, \ov\xi)
+ e^{-t} R_b(\xi, \ov\xi, \xi, \ov\xi)
- |\sigma(\xi)\xi|^2_{Q,t},
$$
where $\sigma(\xi)\xi = D_{h,\xi}\xi - D_{b,\xi} \xi$ is the second fundamental
form and the norm is on the ``quotient'' bundle.
If we simultaneously diagonalize the Hermitian forms at a point we see that
$|\eta|^2_{Q,t} \to 0$ when $t \to \infty$ for any fixed tangent field
$\eta$. Then the \hsc{} of $h + e^{-t} b$ is positive for all $t$ large enough.



\begin{prop}
\label{prop:fs}
Let $\mu : \bl X \to X$ be the blowup of $X$ at a point $p$.
There exists a closed $(1,1)$-form $\beta$ on $X$ whose restriction to
the exceptional divisor is the Fubini--Study metric.
\end{prop}

\begin{proof}
This is proved for blowups of smooth submanifolds in
Voisin's textbook~\cite{voisin2002theorie}. We can simplify that proof a little
since we're only blowing up points.

Locally around $p$, which may assume is the origin in a complex vector space
$V$, the blowup is
$$
\bl V
= \{ (v,[w]) \in V \times \kk P(V) \mid v \in \kk C w \}
$$
and $\mu$ is the projection onto the first factor.
Pick an inner product on $V$ and let $B(r) \subset V$ be a ball of radius $r$
centered at $0$.
We pick $r$ so that $B(r)$ fits into the coordinate chart that implicitly lurks
in the background.
Let $\psi$ be a bump function supported on that chart that is identically $1$
on $B(r)$.
The $(1,1)$-form $\frac i2 \partial\bar\partial \log |v|^2$ on $V \setminus
\{0\}$ descends to $\kk P(V)$ and defines the Fubini--Study metric.
If $p_j : V \times V \setminus \{0\} \to V$ for $j = 1,2$ are the projections
onto the first and second factors then
$\frac i2 \partial \bar\partial (p_1^*\psi \log |p_2^*v|^2)$
defines a closed $(1,1)$-form on $V \times \kk P(V)$ that restricts to the
pullback of the Fubini--Study metric by $p_2$ on $B(r) \times \kk P(V)$.
It also extends to the rest of $X$ by zero.
Its restriction to $\bl V$ is the form we want.
\end{proof}



\section{The proof}

Let $X$ be a compact K\"ahler manifold of dimension $\dim_{\kk C} X = n$ that
admits a metric $h$ of positive \hsc.
Let $p \in X$ be a point and blow it up to obtain $\mu : \bl X \to X$.
Let $b$ be the Hermitian form associated to the $(1,1)$-form $\beta$ on $\bl X$
we constructed in Proposition~\ref{prop:fs}.
Then $h_t = \mu^*h + e^{-t} b$ is a K\"ahler metric on $\bl X$ for all $t$ large
enough.
We are going to show it also eventually has positive \hsc{} by showing
that it has positive \hsc{} on a neighborhood $U$ around the exceptional
divisor for all $t$ large enough.
Then Proposition~\ref{prop:positive} shows that $h_t$ also has positive \hsc{}
on $\bl X \setminus U$ for large enough $t$.

For this we're going to show that $h_t$ has positive \hsc{} on a neighborhood
around any point on $E$ for all large enough $t$.
The compactness of the exceptional divisor then lets us conclude.

We're going to calculate the \hsc{} on the exceptional divisor
at the center of a well-chosen coordinate chart.
Recall that locally around $p$ the blowup is
$$
\bl V
= \{ (v,[w]) \in V \times \kk P(V) \mid v \in \kk C w \}.
$$
If $f \in \GL V$ then $f$ acts on the blowup by $f(v, [w]) = (f(v), [f(w)])$.
This is an isomorphism that maps the exceptional divisor to itself, and we can
map any point on the divisor to any other point.

Let $(0, [w])$ be a point on $E$.
Let's choose normal coordinates $(x_1,\ldots,x_n)$ centered at $p$.
There exists $f \in U(n)$ so that $f(w) = (0 \ldots, 0, 1)$, and the
coordinates obtained by applying $f$ to the old ones are still centered at $p$
and are normal there because $f \in U(n)$.
Picking the chart $\{(y_1, \ldots, y_n) \in \kk C^n \mid y_n \not= 0 \}$ for
$\kk P(V)$
we realize the blowup as
$$
\bl X
= \{ (x,y) \in \kk C^n \times \kk C^{n-1}
\mid x_j y_k = x_k y_j \text{ for $j,k = 1,\ldots,n$, where $y_n = 1$}  \}.
$$
In these coordinates the point we want to calculate the \hsc{} at is $(0,0)$
and we have $D_{h,\xi} \partial / \partial x_j = 0$ at $0$ for $j = 1, \ldots, n$.

We also note that close to the exceptional divisor, $h_t = \mu^* h + e^{-t} b$ is
just the restriction of the product metric $p_1^* h \oplus e^{-t} p_2^* g$ on
$V \times \kk P(V)$ to $\bl X$, where $p_j$ are the projections onto the
factors and $g$ is the Fubini--Study metric.


\begin{lemm}
The metric $h_t = \mu^*h + e^{-t} b$ has positive \hsc{} on $E$ for all $t$.
\end{lemm}

\begin{proof}
The only equations that give us any information at $(0,0)$ are
$$
x_j - y_j x_n = 0, \quad j = 1, \ldots, n-1.
$$
Their differentials are
$$
dx_j - y_j dx_n - x_n dy_j = 0, \quad j=1,\ldots,n-1
$$
and so the tangent fields
$$
\xi_j = x_n e_j + f_j,
\quad j=1,\ldots,n-1,
\qandq
\xi_n = \sum_{j=1}^{n-1} y_j e_j + e_n
$$
are a basis for
the intersection of all the kernels, that is, $T_X$.
Here $e_j$ is the tangent field corresponding to the coordinate $x_j$
and $f_k$ the one corresponding to $y_k$.
Note that the normal bundle at $(0,0)$ is spanned by $(e_1, \ldots, e_{n-1})$.

Any holomorphic tangent field $\xi$ near $(0,0)$ can be written as
$$
\xi = \sum_{j=1}^n a_j \xi_j
= \sum_{j=1}^{n-1} (a_j x_n + a_n y_j) e_j + a_j f_j
+ a_n e_n,
$$
where the $a_j$ are holomorphic functions.
We want to check the positivity of the curvature at the origin, which we can do
on the unit sphere, so we may assume that $\sum_{j=1}^n |a_j|^2 = 1$ there.

Recall that the curvature of $h_t$ is
$$
R_{h_t}(\xi, \ov\xi, \xi, \ov\xi)
= p_1^* R_h(\xi, \ov\xi, \xi, \ov\xi)
+ e^{-t} p_2^* R_g(\xi, \ov\xi, \xi, \ov\xi)
- |\sigma(\xi)\xi|^2,
$$
where $\sigma(\xi)\xi = \pi_N(D_{h \oplus e^{-t} g,\xi} \xi)$ is the second
fundamental form, and the norm is on the quotient bundle.

First,
$$
p_1^*R_h(\xi, \ov\xi, \xi, \ov\xi)
= |a_n|^4 H_h(e_n)
$$
at the origin. Second,
$$
p_2^*R_g(\xi, \ov\xi, \xi, \ov\xi)
= 2 \biggl(\sum_{j=1}^{n-1} |a_j|^2\biggr)^2
= 2(1 - |a_n|^2)^2
$$
at the origin
because the Fubini--Study metric has constant \hsc{} 2.

Let's now write $\pi_N$ for the projection onto the orthogonal complement of
$T_{\blX}$ in $T_{\kk C^n \times \kk C^{n-1}|\blX}$.
At the origin it is just the projection onto the first $n-1$ coordinates.

Third, then,
\begin{align*}
\pi_N(p_1^*D_{h,\xi} \xi)
&= \sum_{j=1}^{n-1} d_{\xi}(a_j x_n + a_n y_j) \, e_j
\\
&= \sum_{j=1}^{n-1} (a_j d_{\xi}x_n + a_n d_{\xi} y_j) \, e_j
\\
&= \sum_{j=1}^{n-1} (a_j a_n + a_n a_j ) \, e_j
= 2 a_n \sum_{j=1}^{n-1} a_j \, e_j
\end{align*}
at the origin
because $D_{h,\xi} e_j = 0$ there by our choice of coordinates.

And fourth,
$$
\pi_N(p_2^*D_{g,\xi} \xi)
= \pi_N \biggl( \sum_{j=1}^{n-1} d_\xi y_j \, f_j + y_j p_2^*D_{g,\xi} f_j \biggr) = 0
$$
at the origin.
The contribution of the second fundamental form to the curvature is then
$$
|\sigma(\xi)\xi|^2
= 4 |a_n|^2 \sum_{j=1}^{n-1} |a_j|^2 |e_j|^2
= 4 |a_n|^2 \sum_{j=1}^{n-1} |a_j|^2
= 4 |a_n|^2(1 - |a_n|^2)
$$
so finally
\begin{align*}
R_{h_t}(\xi, \ov\xi, \xi, \ov\xi)
= |a_n|^4 H_h(e_n)
+ e^{-t} 2(1 - |a_n|^2)^2
- 4 |a_n|^2(1 - |a_n|^2).
\end{align*}
at the origin.
We want to show that this is positive for all $a_n$ with $|a_n| \leq 1$.

Set $m = \inf_{\kk P(T_X)} H_h > 0$ and let
$$
f_t(x) = m x^2 + 2 e^{-t} (1-x)^2 - 4x(1-x).
$$
Then $R_{h_t}(\xi, \ov\xi, \xi, \ov\xi) \geq f(|a_n|^2)$.
We have
$$
f_t(x)
= (m + 4 + 2e^{-t})x^2 - 4(e^{-t} + 1)x + 2e^{-t}.
$$
As a polynomial in $x$ the coefficient of $x^2$ of $f_t$ is positive.
Now
$$
f_t'(x)
= 2(m + 4 + 2e^{-t})x - 4(e^{-t} + 1)
$$
so $f_t'(x_0(t)) = 0$ when
$$
x_0(t) = \frac{2(e^{-t} + 1)}{m + 4 + 2e^{-t}}.
$$
At $x_0(t)$, where $f_t$ achieves its global minimum, we have
$$
f_t(x_0(t))
=
4 \frac{(e^{-t} + 1)^2}{m + 4 + 2e^{-t}}
- 4 \frac{(e^{-t} + 1)^2}{m + 4 + 2e^{-t}}
+ 2e^{-t}
= 2e^{-t} > 0
$$
so $f_t(x_0(t))$ is strictly decreasing in $t$ and tends to $0$ as $t \to
\infty$, so $f_t(x) \geq f_t(x_0(t)) > 0$ for all $x \in [0,1]$ and all $t \in
\kk R$.
\end{proof}




It might be possible to extend this approach to cover the blowup of a smooth
submanifold of a compact K\"ahler manifold of positive \hsc.
One problem
is that the blowup along a submanifold $Y \subset X$ is
constructed by picking local coordinates $(z_1,\ldots,z_n)$ such that $Y = \{z
\mid z_{k+1} = \cdots = z_n = 0\}$,
and the standard method of producing normal coordinates centered at a point
(see for example Zheng~\cite{zheng2000complex})
does not preserve these equations.
Our trick here thus does not carry through verbatim and there might be extra
difficulties involved in the curvature calculations, comparable to working
with general Hermitian metrics.
If any enterprising soul takes this on I'd be happy to hear about it.





\subsection*{A surface, local description}

Have $\blX = \{(x,y,z) \mid x - y z = 0\}$.
The tangent space of $\blX$ is spanned by
$$
\xi_1 = z e_x + e_y,
\quad
\xi_2 = y e_x + e_z.
$$
The FS metric is
$$
g(z)
= \frac{\xi_3 \ov\eta_3}{1 + |z|^2}
- \frac{\xi_3 \ov \eta_3 |z|^2}{(1 + |z|^2)^2}
= \frac{\xi_3 \ov\eta_3}{(1 + |z|^2)^2}.
$$
Its Chern connection is
$$
D_\xi \eta_3
= d_\xi \eta_3 - \frac{\xi_3 \ov z}{1 + |z|^2} \eta_3
- \frac{\eta_3 \ov z}{1 + |z|^2} \xi_3.
$$
Let $D_h = d + A$.
The second fundamental form is
$$
\sigma(\xi) \eta
= \pi(D_{h,\xi} \eta \oplus D_{g,\xi} \eta),
$$
where $\pi$ is the orthogonal projection onto the normal bundle.
The tangent bundle of $\blX$ is $\ker \alpha$, where
$\alpha = dx - z dy - y dz$.

Let $u = (z, 1, 0)$ and $v = (y, 0, 1)$ and let $S = \kk C u + \kk C v \subset
\kk C^3$.
We have $S = \ker \alpha$, where $\alpha \in (\kk C^3)^*$, so
$$
0 \longrightarrow S \longrightarrow \kk C^3
\stackrel{\alpha}{\longrightarrow} \kk C \longrightarrow 0
$$
is exact.
This gives an isomorphism $\kk C^3 / S \cong \kk C$.
We have $\alpha = dx - z dy - y dz$, so the
quotient map is given by $Q = (1, -z, -y)$.
We should probably just write this as
$$
0 \longrightarrow
\kk C^2 \stackrel{M}{\longrightarrow }
\kk C^3 \stackrel{Q}{\longrightarrow}
\kk C \longrightarrow 0
$$
where
$$
M = \begin{pmatrix}
z & y
\\
1 & 0
\\
0 & 1
\end{pmatrix}
\qandq
Q = \begin{pmatrix}
1 & -z & -y
\end{pmatrix}.
$$
If $\xi = a u + b v = (az + by, a, b)$ then $\pi_{12}(\xi) = (az + by, a)$ and
$\pi_3(\xi) = b$. The curvature tensor is then
$$
\displaylines{
R_t(\xi,\ov\xi,\xi,\ov\xi)
\geq m |(az + by, a)|^4_h
+ \frac{e^{-t} |b|^4}{(1+|z|^2)^2}
\hfill\cr\noalign{\vskip-6pt}\hfill{}
- |a^2 \sigma(u,u) + 2ab \sigma(u, v) + b^2 \sigma(v,v)|_q^2.
}
$$
The last term is the norm of a quadratic form
$$
\tau(a,b)
=
\begin{pmatrix} a & b \end{pmatrix}
\begin{pmatrix}
\sigma(u,u) & \sigma(u,v)
\\
\sigma(u,v) & \sigma(v,v)
\end{pmatrix}
\begin{pmatrix} a \\ b \end{pmatrix}
$$
whose coefficients are $\cc C^\infty$ functions.
Now
\begin{align*}
\sigma(u,u)
= q(D_u u)
= q(d_u z \, e_1 + z D_u e_1 + D_u e_2)
= q(z D_u e_1 + D_u e_2)
\end{align*}
so we get $\sigma(u,u)(0) = 0$ because we chose normal coordinates.
Similarly we have $\sigma(v,v)(0) = 0$ and
$$
\sigma(u,v)(0)
= q(d_u y \, e_1 + y D_u e_1 + D_u e_3)
= q(z d_{e_1} y \, e_1 + d_{e_2} y \, e_1)
= 1.
$$
Expanding we get
\begin{align*}
|(az + by, a)|^4_h
&= (|(az + by, a)|_h^2)^2
\\
&= (|az+by|^2 h_{11} + (az+by)\bar a h_{12} + a\ov{(az+by)}h_{21} + |a|^2 h_{22})^2
\end{align*}

We have a Hermitian metric on $\kk C^3$ given by
$$
H = \begin{pmatrix}
	J & 0
	\\
	0 & e^{-t} \theta
\end{pmatrix},
$$
where $\theta(x,y,z) = 1 / (1+|z|^2)^2$.
The quotient metric is then the inverse of
\begin{align*}
Q H^{-1} \ov{Q^t}
&= \begin{pmatrix} 1 & -z & -y \end{pmatrix}
\begin{pmatrix}
J^{-1} & 0
\\
0 & e^t/\theta
\end{pmatrix}
\begin{pmatrix} 1 \\ -\bar z \\ -\bar y \end{pmatrix}
\\
&= \begin{pmatrix} 1 & -z & -y \end{pmatrix}
\begin{pmatrix}
h^{11} - \bar z h^{12}
\\
h^{21} - \bar z h^{22}
\\
-e^t \bar y/\theta
\end{pmatrix}
= h^{11} - \bar z h^{12} - z h^{21} + |z|^2 h^{22} + e^t |y|^2 / \theta,
\end{align*}
or
$$
h_Q
= \frac{1}{h^{11} - \bar z h^{12} - z h^{21} + |z|^2 h^{22} + e^t |y|^2 / \theta}.
$$
This is actually just $1 / |\ov Q^t|^2_{H^{-1}}$.
What's hard is to see that this behaves OK in the limit without going through
all the calculations.
For $z = 0$ the $h$ terms are $h^{11}$, which is $1$ at $y = 0$.
For $(y,z)$ in a small ball $B(r)$ we then have
$$
h^{11} - \bar z h^{12} - z h^{21} + |z|^2 h^{22} + e^t |y|^2 / \theta
\geq 1 - \varepsilon + e^t |y|^2,
$$
so
$$
h_Q \leq \frac{1}{1 - \epsilon + e^t|y|^2}.
$$
We can also pick $r$ so that
$$
|(az + by, a)|_h^2 \geq |a|^2 - \varepsilon
$$
on $B(r)$, as $h(0,0)$ is the standard inner product. We're now at
\begin{align*}
R_t &\geq
m (|a|^4 - 2\varepsilon|a|^2 + \varepsilon^2)
+ \frac{e^{-t}|b|^4}{(1-|z|^2)^2}
- 2 \frac{2 |a|^2|b|^2}{1 - \varepsilon + e^t |y|^2}
\\
&\geq
m (|a|^4 - 2\varepsilon|a|^2 + \varepsilon^2)
+ \frac{e^{-t}|b|^4}{(1-r^2)^2}
- \frac{4 |a|^2|b|^2}{1 - \varepsilon}
\end{align*}
Let
$$
p(x,y)
= m(x^2 - 2\varepsilon x + \varepsilon^2)
+ \frac{e^{-t}y^2}{(1-r^2)^2}
- \frac{4xy}{1 - \varepsilon}.
$$
The gradient of $p$ is
$$
\nabla p
= \begin{pmatrix}
2mx - 2 m \varepsilon - \frac{4y}{1-\varepsilon}
\\
\frac{2e^{-t}y}{(1-r^2)^2} - \frac{4x}{1-\varepsilon}
\end{pmatrix},
\quad
H =
\begin{pmatrix}
2m & - \frac{4}{1-\eps}
\\
-\frac{4}{1-\eps} & \frac{2e^{-t}}{(1-r^2)^2}
\end{pmatrix}.
$$
The Hessian is constant; its determinant is
$$
\det H
= \frac{2me^{-t}}{(1-r^2)^2} - \frac{16}{(1-\eps)^2}
$$
which is negative for $t$ large enough, so the extremal points become saddle points.
We want to consider positive $x,y$, so may want to look at the Lagrange function
$$
p(x,y) - \lambda xy.
$$
Its gradient is
$$
\begin{pmatrix}
2mx - 2 m \varepsilon - \frac{4y}{1-\varepsilon} - \lambda y
\\
\frac{2e^{-t}y}{(1-r^2)^2} - \frac{4x}{1-\varepsilon} - \lambda x
\\
xy
\end{pmatrix}.
$$
If $x = 0$ we get $y = 0$ (from the second equation) and $2m\eps = 0$, so
we don't have an extremal point.
If $y = 0$ we likewise get $x = 0$.
We have $p(0,y) = e^{-t}y^2 / (1-r^2)^2 > 0$ and $p(x,0) = m(x-\eps)^2 \geq 0$.


This is zero when
\begin{align*}
2mx - 2 m\varepsilon - \frac{4y}{1-\varepsilon} &= 0,
\\
\frac{2e^{-t}y}{(1-r^2)^2} - \frac{4x}{1-\varepsilon} &= 0,
\end{align*}
that is when
\begin{align*}
mx - \frac{2y}{1-\varepsilon} &= m\varepsilon,
\\
\frac{e^{-t}y}{(1-r^2)^2} &= \frac{2x}{1-\varepsilon},
\end{align*}
or
$$
x = \frac{(1-\epsilon)e^{-t}}{2(1-r^2)^2} y
\qandq
\biggl(\frac{m(1-\epsilon)e^{-t}}{2(1-r^2)^2}
- \frac{2}{1-\epsilon}\biggr)y = m\epsilon.
$$
In any case this fixes $x$ and $y$, and since $p$ is a parabola this point is a
global minimum.
The negative term in our setup is
$$
\frac{4xy}{1-\epsilon}
= \frac{2 e^{-t}}{(1-r^2)^2}y^2
$$
which gives
$$
R_t \geq
m\biggl(
\frac{(1-\epsilon)e^{-t}}{2(1-r^2)^2} y
- \varepsilon
\biggr)^2
- \frac{e^{-t}y^2}{(1-r^2)^2}.
$$
We have
$$
\frac{m(1-\epsilon)e^{-t}}{2(1-r^2)^2}
- \frac{2}{1-\epsilon}
= \frac{me^{-t}(1-\eps)^2 - 4(1-r^2)^2}{2(1-r^2)^2(1-\eps)}
$$
so
$$
y = \frac{2 m \eps (1-r^2)^2 (1-\eps)}{m e^{-t} (1 - \eps)^2 - 4(1-r^2)^2}.
$$
Then
$$
\frac{(1-\epsilon)e^{-t}}{2(1-r^2)^2} y
=
\frac{m e^{-t} \eps (1-\eps)^2}{m e^{-t} (1 - \eps)^2 - 4(1-r^2)^2}
$$
so
$$
\frac{(1-\epsilon)e^{-t}}{2(1-r^2)^2} y - \eps
=
\frac{4(1-r^2)^2 \eps}{m e^{-t} (1 - \eps)^2 - 4(1-r^2)^2}.
$$
Now
\begin{align*}
\frac{e^{-t}y^2}{(1-r^2)^2}
&= \frac{e^{-t}}{(1-r^2)^2}
\biggl(
\frac{2 m \eps (1-r^2)^2 (1-\eps)}{m e^{-t} (1 - \eps)^2 - 4(1-r^2)^2}
\biggr)^2
\\
&=
\frac{4 m^2 e^{-t} \eps^2 (1-r^2)^2 (1-\eps)^2}
{(m e^{-t} (1 - \eps)^2 - 4(1-r^2)^2)^2}
\end{align*}



When $y \not= 0$ then $h_Q \to 0$ as $t \to \infty$, but when $y = 0$ it doesn't.
So it actually gets better the farther away from $E$ we get?
Note that $h^{11} = h_{22} / \det J$, $h^{22} = h_{11} / \det J$,
$h^{12} = -h_{12} / \det J$ and $h^{21} = -h_{21} / \det J$.

The metric on $S$ is
\begin{align*}
H_S
&=
\begin{pmatrix}
\bar z & 1 & 0
\\
\bar y & 0 & 1
\end{pmatrix}
\begin{pmatrix}
J & 0
\\
0 & e^{-t}\theta
\end{pmatrix}
\begin{pmatrix}
z & y
\\
1 & 0
\\
0 & 1
\end{pmatrix}
=
\begin{pmatrix}
\bar z & 1 & 0
\\
\bar y & 0 & 1
\end{pmatrix}
\begin{pmatrix}
z h_{11} + h_{12} & y h_{11}
\\
z h_{21} + h_{22} & y h_{21}
\\
0 & e^{-t} \theta
\end{pmatrix}
\\
&=
\begin{pmatrix}
|z|^2 h_{11} + \bar z h_{12} + z h_{21} + h_{22} &
\bar z y h_{11} + y h_{21}
\\
\bar y z h_{11} + \bar y h_{12} &
|y|^2 h_{11} + e^{-t} \theta
\end{pmatrix}.
\end{align*}
Take $\xi = a u + bv$, where $u = ze_x + e_y$ and $v = ye_x + e_z$.
Then
\begin{align*}
\sigma(\xi)\xi =
q(D_\xi\xi)
&= q(d_\xi a \, u + a D_\xi u + d_\xi b \, v + b D_\xi v)
\\
&= q(a D_\xi u + b D_\xi v)
\\
&= q(a d_\xi z \, e_x + az D_\xi  e_x + D_\xi  e_y + b d_\xi y \, e_x + by D_\xi  e_x + D_\xi  e_z)
\\
&= 2ab
+ \alpha( (az + by) D_\xi e_x + D_\xi e_y + D_\xi e_z).
\end{align*}
We can calculate this some more.
Need to write out the connection forms of $h_X$ and Fubini--Study.
The one of $h_X$ is
$$
J^{-1} \partial J
=
\begin{pmatrix}
h^{11} & h^{12}
\\
h^{21} & h^{22}
\end{pmatrix}
\begin{pmatrix}
\partial h_{11} & \partial h_{12}
\\
\partial h_{21} & \partial h_{22}
\end{pmatrix}
=
\begin{pmatrix}
h^{11} \partial h_{11} {+} h^{12} \partial h_{21} &
h^{11} \partial h_{12} {+} h^{12} \partial h_{22}
\\
h^{21} \partial h_{11} {+} h^{22} \partial h_{21} &
h^{21} \partial h_{12} {+} h^{22} \partial h_{22}
\end{pmatrix}
$$
so
\begin{align*}
\alpha((az+by) D_\xi e_x)
&= (az + by) (
h^{11} \partial_\xi h_{11} {+} h^{12} \partial_\xi h_{21}
- z  (h^{21} \partial_\xi h_{11} {+} h^{22} \partial_\xi h_{21} )
),
\\
\alpha(D_\xi e_y)
&=
h^{11} \partial_\xi h_{12} {+} h^{12} \partial_\xi h_{22}
- z (h^{21} \partial_\xi h_{12} {+} h^{22} \partial_\xi h_{22}).
\end{align*}
I think that what matters here is that
$$
\sigma(\xi)\xi
= p(a,b),
$$
where $p$ is some degree-two polynomial in $a,b$ (whose coefficients are smooth
functions of $x,y,z$ and $h_X$ and the Fubini--Study metric).
Then
$$
|\sigma(\xi)\xi|^2_Q
= \frac{|p(a,b)|^2}{h^{11} - \bar z h^{12} - z h^{21} + |z|^2 h^{22} + e^t |y|^2 \theta}
= |p(a,b)|^2 + O(e^{-t} |v|^2).
$$
At $0$ we have $\sigma(\xi)\xi = 2ab$.
This should let us extend the result on the exceptional divisor to a
neighborhood of it for all large $t$.



If $\sigma = q \circ D$ is the second fundamental form of any embedding we have
$$
\sigma(a \xi, b \eta)
= q(D_{a\xi}(b \eta))
= q(a d_\xi b \, \eta + ab D_\xi \eta)
= ab \sigma(\xi, \eta)
$$
so it is a tensor.
If $(\xi_j)$ is a holomorphic frame for $T_Z$ and $(a_j)$ are holomorphic
functions and $\xi = \sum_j a_j \xi_j$ we get
$$
\sigma(\xi, \xi)
= \sum_{j,k} a_j a_k \, \sigma(\xi_j, \xi_k).
$$
Then
$$
|\sigma(\xi, \xi)|_{Q,t}
\leq \sum_{j,k} |a_j||a_k| \, |\sigma(\xi_j, \xi_k)|_{Q,t}.
$$
We claim that $f_{jk}(v, t) := |\sigma(\xi_j, \xi_k)|_{Q,t}$ is a smooth
nonnegative function such that $f(0, t) = 1$ and $f(v, t) \to 0$ as $t \to
\infty$ for $v$ not in $E$.
Then $f_{jk} \leq f := \max(f_{jk})$ so
$$
|\sigma(\xi, \xi)|_{Q,t} \leq |a|^2 f(v,t) .
$$
Recall that $\xi_j = x_n e_j + f_j$ for $j = 1, \ldots, n-1$ and
$\xi_n = \sum_{j=1}^{n-1} y_j e_j + e_n$.
Then
\begin{align*}
\pi_V(\xi)
&= \sum_{j=1}^{n-1} a_j x_n e_j + \sum_{k=1}^{n-1} a_n y_k e_k + a_n e_n
= \sum_{j=1}^{n-1} (a_j x_n + a_n y_j) e_j + a_n e_n,
\\
\pi_{\kk P(V)}(\xi)
&= \sum_{j=1}^{n-1} a_j f_j.
\end{align*}
I don't think we can (want to?) say much more than that $|\pi_V(\xi)|^2$ and
$|\pi_{\kk P(V)}(\xi)|^2$ are degree-two homogeneous polynomials in $a$
(over $\cc C^\infty$).

We have to prove our estimate for the quotient norm because what I wrote is true
for a sum of metrics on the same space, which is not what we have.
If
$$
0 \to S \to V \to Q \to 0
$$
is a short exact sequence and $h$ is an inner product on $V$, the inner product
on the quotient is
$$
h_Q = ((q^\vee)^{*} h^\vee)^\vee.
$$
If $H$ and $Q$ are the matrices of the inner product and quotient map we get
$$
H_Q = \ov{(\ov Q \ov{H^{-1}} Q^t)^{-1}}.
$$
Our $H = J \oplus e^{-t} K$ so $H^{-1} = J^{-1} \oplus e^{t} K^{-1}$.
Write $Q = ( Q_1 \ Q_2)$, where $Q_1$ is a $n \times (n - 1)$ matrix and $Q_2$ is
a $(n-1) \times (n-1)$ matrix and get
$$
\ov{Q} \ov{H^{-1}} Q^t
=
\begin{pmatrix}
\ov{Q_1} & \ov{Q_2}
\end{pmatrix}
\begin{pmatrix}
\ov{J^{-1}} & 0
\\
0 & e^{t} \ov{K^{-1}}
\end{pmatrix}
\begin{pmatrix}
Q_1^t \\ Q_2^t
\end{pmatrix}
=
\ov{Q_1} \ov{J^{-1}} Q_1^t
+ e^{t} \ov{Q_2} \ov{K^{-1}} Q_2^t
$$
and so
$$
H_Q
=
(Q_1 J^{-1} \ov{Q_1^t}
+ e^{t} Q_2 K^{-1} \ov{Q_2^t})^{-1}.
$$

\begin{claim}
On $x_n \not= 0$ we have $|\xi|_{H_Q} \to 0$ as $t \to \infty$ for any $\xi$.
\end{claim}

\begin{proof}
We can prove the statement pointwise, so pick a point.
The matrices $J^{-1}$ and $K^{-1}$ are Hermitian positive definite, so
$Q_1J^{-1} \ov{Q_1^t}$ and $Q_2 K^{-1} \ov{Q_2^t}$ are Hermitian positive
semidefinite.
By hypothesis, the matrix $Q_1J^{-1} \ov{Q_1^t} + e^{t} Q_2 K^{-1} \ov{Q_2^t}$ is
also invertible for any $t$.
If we fix a $t_0$, we can simultaneously diagonalize that matrix and, say,
$Q_1J^{-1} \ov{Q_1^t}$. Then the third matrix, the difference of the other two,
is also diagonal.
We may thus assume that $H_Q$ is a diagonal matrix whose entries are of the form
$1/(a_j + e^t b_j)$, where $a_j + e^t b_j > 0$ for all $j$.
The claim follows if $b_j > 0$ for all $j$.
By inspection we in fact have $Q_2 = -x_n I_{n-1}$.
Then $Q_2 K^{-1} \ov{Q_2^t} = |x_n|^2 K^{-1}$ is positive definite if $x_n
\not= 0$, in which case $b_j > 0$ for all $j$.
\end{proof}

We have
$$
Q_1 =
\begin{pmatrix}
1 & 0 & \cdots & -y_1
\\
0 & 1 & \cdots & -y_2
\\
  & & \ddots &
\\
0 & \cdots & 1 & -y_{n-1}
\end{pmatrix},
\quad
Q_2 = -x_n I_{n-1}.
$$
Then at least $Q_2 K^{-1} \ov{Q_2^t} = |x_n|^2 K^{-1}$.
For a matrix $A = (a_{jk})$ we have
$$
Q_1 A \ov{Q_1^t}
= (a_{jk} - y_j a_{nk} - \bar y_k a_{jn} + y_j \bar y_k a_{nn})_{1 \leq j,k \leq n-1}.
$$
That's not fun but in any case it is enough to see that $h_Q$ is $O(e^{-t})$.
Then the second fundamental form is only $O(e^{-t})$.

We have
$$
0 \to T_Z \to T_V \oplus T_{\kk P(V)} \to N_{Z/X} \to 0
$$
and $D_{\Hom} q(\xi) + q(D\xi) = 0$ for all sections
$\xi$ of $T_Z$.
We're going to use $h_t = h + e^{-t} b$.
Both $D$ and $D_{\Hom}$ are independent of $t$.
We get
\begin{align*}
R_t(\xi,\ov\xi,\xi,\ov\xi)
&= \pi_V^* R_h(\xi,\ov\xi,\xi,\ov\xi)
+ e^{-t} \pi_{\kk P(V)}^* R_{FS}(\xi,\ov\xi,\xi,\ov\xi)
- |D_{\Hom,\xi}q(\xi)|^2
\\
&\geq
m |\pi_V(\xi)|^4_h
+ 2 e^{-t} |\pi_{\kk P(V)}(\xi)|^4_{FS}
- |D_{\Hom}q|^2 |\xi|^4_t,
\end{align*}
where $0 < m \leq H_h$.
We can calculate that $|D_{\Hom}q|^2$ is $O(e^{-t})$
and have
$$
|\xi|^2_t
= |\pi_V(\xi)|^2_h + e^{-t} |\pi_{\kk P(V)}(\xi)|^2_{FS}
$$
so we get
$$
\displaylines{
R_t(\xi,\ov\xi,\xi,\ov\xi)
\geq
(m - |D_{\Hom}q|^2) |\pi_V(\xi)|^4_h
+ (2 - e^{-2t} |D_{\Hom}q|^2) |\pi_{\kk P(V)}(\xi)|^4_{FS}
\hfill\cr\hfill{}
- 2 e^{-t} |D_{\Hom}q|^2 |\pi_V(\xi)|^2_h  |\pi_{\kk P(V)}(\xi)|^2_{FS}.
}
$$
As $t \to \infty$ then $|D_{\Hom}q|^2 \to 0$ so the middle term is eventually
positive.
We have to figure out what $e^t | \, \cdot \, |^2_{\Hom}$ tends to as $t \to
\infty$.
We should have
\begin{align*}
H_{\Hom}
= H_t^\vee \otimes H_t^\vee \otimes H_Q.
\end{align*}
Again diagonalizing $J$ and $K$ at a point, this becomes
$$
H_{\Hom}
= (A^{-1} + e^{t} B^{-1}) \otimes
(A^{-1} + e^{t} B^{-1}) \otimes
(A \otimes e^t B)
$$


\section{Let's just blow up projective space}

In local coordinates the Chern connection of the Fubini--Study metric is
$$
D_\xi \eta
= d_\xi \eta
- \frac{\<\xi, \bar z\>}{1+|z|^2} \eta
- \frac{\<\eta, \bar z\>}{1+|z|^2}.
$$
We have
$$
\xi_j = x_ne_j + f_j, \quad j=1,\ldots,n-1
\qandq
\xi_n = \sum_{j=1}^{n-1} y_j e_j + e_n,
$$
and these are all in the kernel of $\alpha_j = dx_j - y_j dx_n - x_n dy_j$.

Then
$$
D_{\xi_j} \xi_k
= x_n^2 D_{e_j} e_k + D_{f_j} f_k
= -x_n^2
\biggl(
\frac{\bar x_j}{1+|x|^2} e_k
+ \frac{\bar x_k}{1+|x|^2} e_j
\biggr)
+ \frac{\bar y_j}{1+|y|^2} f_k + \frac{\bar y_k}{1+|y|^2} f_j
$$
and
$$
\alpha_l(D_{\xi_j}\xi_k)
= -x_n^2
\frac{\bar x_j \delta_{kl} + \bar x_k \delta_{jl}}{1+|x|^2}
- \frac{\bar y_j \delta_{kl} + \bar y_k \delta_{jl}}{1+|y|^2}
$$
for $j,k,l = 1, \ldots, n-1$.
We have $x_j = y_j x_n$ for $j=1,\ldots,n-1$, which gives
\begin{align*}
\alpha_l(D_{\xi_j}\xi_k)
&= -x_n^2
\frac{\bar y_j \bar x_n \delta_{kl} + \bar y_k \bar x_n \delta_{jl}}{1+|y|^2+|x_n|^2}
- \frac{\bar y_j \delta_{kl} + \bar y_k \delta_{jl}}{1+|y|^2}
\\
&= -
\biggl(
\frac{x_n |x_n|^2}{1+|y|^2+|x_n|^2}
+ \frac{1}{1+|y|^2}
\biggr)(\bar y_j \delta_{kl} + \bar y_k \delta_{jl})
\end{align*}
If we try to estimate the norm on the quotient by the one upstairs, we
should look at
$$
D_{\xi_j} \xi_k
=  -x_n^2
\biggl(
\frac{\bar y_j \bar x_n}{1+|y|^2+|x_n|^2} e_k
+ \frac{\bar y_k \bar x_n}{1+|y|^2+|x_n|^2} e_j
\biggr)
+ \frac{\bar y_j}{1+|y|^2} f_k + \frac{\bar y_k}{1+|y|^2} f_j.
$$
We could get rid of the $f_j$ and $f_k$ terms by subtracting $\bar y_j / (1+|y|^2) \xi_k$.
But if anything we should subtract $-y_j|x_n|^2/(1+|y|^2+|x_n|^2) \xi_k$ to get
rid of the $e_k$ terms to end up only with things whose norm goes to zero.
We get
\begin{align*}
D_{\xi_j} \xi_k
&\cong
\biggl(
\frac{1}{1+|y|^2}
+ \frac{|x_n|^2}{1+|y|^2+|x_n|^2}
\biggr)
\bar y_j f_k
+ \biggl(
\frac{1}{1+|y|^2}
+ \frac{|x_n|^2}{1+|y|^2+|x_n|^2}
\biggr)
\bar y_k
f_j
\\
&= T(y,x_n) (\bar y_j f_k + \bar y_k f_j),
\end{align*}
where
$$
T(y,x_n) =
\frac{1}{1+|y|^2}
+ \frac{|x_n|^2}{1+|y|^2+|x_n|^2}.
$$
Then
\begin{align*}
|\sigma(\xi_j,\xi_k)|
&\leq Te^{-t}\biggl(
\frac{|y_j|^2}{1+|y|^2}
- \frac{|y_j|^2|y_k|^2}{(1{+}|y|^2)^2}
+ 2\frac{|y_j|^2 |y_k|^2}{(1{+}|y|^2)^2}
+ \frac{|y_k|^2}{1{+}|y|^2}
- \frac{|y_j|^2|y_k|^2}{(1{+}|y|^2)^2}
\biggr)^{\mkern-6mu 1/2}
\\
&=
Te^{-t}\biggl(
\frac{|y_j|^2 + |y_k|^2}{1+|y|^2}
\biggr)^{\mkern-6mu 1/2}
\end{align*}
if I've calculated the FS metric correctly.
Also
\begin{align*}
D_{\xi_j} \xi_n
&= e_j
+ \sum_{k=1}^{n-1} y_k D_{\xi_j} e_k
+ D_{\xi_j} e_n
\\
&= e_j
+ \sum_{k=1}^{n-1} x_n y_k D_{e_j} e_k
+ x_n D_{e_j} e_n
\\
&= e_j
- \sum_{k=1}^{n-1}
\frac{x_n y_k \bar x_j}{1+|x|^2} e_k
- \sum_{k=1}^{n-1}
\frac{x_n y_k \bar x_k}{1+|x|^2} e_j
- \frac{x_n \bar x_j}{1+|x|^2} e_n
- \frac{|x_n|^2}{1+|x|^2} e_j
\\
&= e_j
- \frac{x_n\bar x_j}{1+|x|^2} \xi_n
- \frac{|y|^2 \bar x_n + \bar x_n}{1+|x|^2}
x_n e_j
\\
&\cong
e_j + \frac{(1 + |y|^2)\bar x_n}{1+|y|^2+|x_n|^2} f_j
=: e_j + \bar x_n S(y,x_n) f_j.
\end{align*}
If we instead try to absorb the $f_j$ into the $e_j$ we get
\begin{align*}
D_{\xi_j} \xi_k
&\cong
-x_n \biggl(
\frac{|x_n|^2}{1+|y|^2+|x_n|^2}
+ \frac{1}{1+|y|^2}
\biggr)
\bigl(
\bar y_k e_j
+ \bar y_j e_k
\bigr),
& j&=1,\ldots,n-1.
\\
D_{\xi_j} \xi_n
&\cong
\frac{|x_n|^4}{1+|y|^2+|x_n|^2} e_j,
& j&=1,\ldots,n-1.
\\
D_{\xi_n} \xi_n &\cong 0.
\end{align*}

Let $\xi = \sum_{j=1}^n a_j \xi_j$. Then
\begin{align*}
D_\xi \xi
&= \sum_k d_\xi a_k \xi_k + a_k D_\xi \xi_k
\\
&\cong \sum_{j,k} a_j a_k D_{\xi_j} \xi_k
\\
&\cong \sum_{j,k=1}^{n-1} a_j a_k T (\bar y_j f_k + \bar y_k f_j)
+ 2 \sum_{j=1}^{n-1} a_j a_n (e_j + \bar x_n S f_j)
\\
&= 2 a_n \sum_{j=1}^{n-1} a_j e_j
+ 2 a_n \bar x_n \sum_{j=1}^{n-1} a_j S f_j
+ 2 T \<a, \bar y\> \sum_{j=1}^{n-1} a_j f_j
\\
&= 2 a_n \sum_{j=1}^{n-1} a_j e_j
+ 2 (a_n \bar x_n S + T \<a, \bar y\>) \sum_{j=1}^{n-1} a_j f_j
\end{align*}
so
$$
|\sigma(\xi,\xi)|^2
\leq 4 |a_n|^2 |(a', 0)|^2_{e}
+ 4e^{-t} |a_n \bar x_n S + T \<a', \bar y\>|^2 |a'|^2_f.
$$
where $a' := (a_1, \ldots, a_{n-1})$.
We could write that inner product as $|\<a, \ov{(Ty, Sx_n)}\>|^2$.
Recall that $\xi_j = x_n e_j + f_j$ and $\xi_n = \sum_j y_j e_j + e_n$.
Then
$$
|\xi|^2_e
= \biggl|
\sum_{j=1}^{n-1} (a_j x_n + a_n y_j) e_j
+ a_n e_n
\biggr|^2_e
= \bigl| x_n (a',0) + a_n (y,1) \bigr|^2_e
$$
and
$|\xi|^2_f = |a'|^2_f$. The curvature tensor is then at least
$$
2 |\xi|_e^4
- 4 |a_n|^2 |(a', 0)|^2_{e}
+ 2 e^{-t} (|a'|^2_f
- 2 |a_n \bar x_n S + T \<a', \bar y\>|^2) |a'|^2_f.
$$
We have $T\<a', \bar y\>(0) = 0$ and $\bar x_n S(0) = 0$.
Also $|\xi|^4_e(0) = |a_n|^4$.
Now
$$
\displaylines{
|\xi|^2_e
= \sum_{j,k}^{n-1} (a_j x_n + a_n y_j)\ov{(a_k x_n + a_n y_k)}
\<e_j, \bar e_k\>_e
+ \sum_{j}^{n-1} (a_j x_n + a_n y_j)\bar a_n \<e_j, \bar e_n\>_e
\hfill\cr\hfill{}
+ \sum_{j}^{n-1} a_n \ov{(a_j x_n + a_n y_j)} \<e_n, \bar e_j\>_e
+ |a_n|^2 |e_n|^2_e.
}
$$
Meanwhile
$$
|(a',0)|^2_e
= \sum_{j,k=1}^{n-1} a_j \bar a_k \<e_j, \bar e_k\>_e.
$$




And
\begin{align*}
\alpha_l(D_{\xi_j} \xi_n)
&= \delta_{jl}
- \frac{x_n y_l \bar x_j}{1+|x|^2}
- \sum_{k=1}^{n-1} \frac{x_n y_k \bar x_k}{1+|x|^2} \delta_{jl}
+ \frac{y_l x_n \bar x_j}{1+|x|^2}
- \frac{|x_n|^2}{1+|x|^2} \delta_{jl}
\\
&=
\biggl(
1 - \frac{|x_n|^2}{1+|x|^2}
\biggr)
\delta_{jl}
- \frac{x_n}{1+|x|^2}
\sum_{k=1}^{n-1} y_k \bar x_k
\,
\delta_{jl}
\\
&=
\biggl(
1 - \frac{|x_n|^2 + |x_n|^2 |y|^2}{1+|y|^2+|x_n|^2}
\biggr)
\delta_{jl}
=
\frac{1 + |y|^2 - |x_n|^2 |y|^2}{1+|y|^2+|x_n|^2}
\delta_{jl}
\end{align*}
for $j,l = 1,\ldots,n-1$.
I am compelled to also calculate
\begin{align*}
D_{\xi_n} \xi_j
&= e_j + x_n D_{\xi_n} e_j
= e_j + x_n \sum_{k=1}^{n-1} y_k D_{e_k} e_j + D_{e_n} e_j
\\
&=
e_j - \sum_{k=1}^{n-1}
\frac{x_n y_k \bar x_k}{1+|x|^2} e_j
+ \frac{x_n y_k\bar x_j}{1+|x|^2} e_k
- \frac{|x_n|^2}{1+|x|^2} e_j
- \frac{x_n \bar x_j}{1+|x|^2} e_n
\end{align*}
and note that
\begin{align*}
\alpha_l(D_{\xi_n} \xi_j)
&= \delta_{jl}
- \sum_{k=1}^{n-1}
\frac{x_n y_k \bar x_k}{1+|x|^2} \delta_{jl}
- \frac{x_n y_l\bar x_j}{1+|x|^2}
- \frac{|x_n|^2}{1+|x|^2} \delta_{jl}
+ \frac{y_l x_n \bar x_j}{1+|x|^2}
= \alpha_l(D_{\xi_j} \xi_n)
\end{align*}
as it should, which is a nice sanity check.
Finally
write $\pi_e$ for the projection onto $\kk C^n$
and $\pi_f$ for the projection onto $\kk C^{n-1}$.
Because $\pi_e \xi_n = \xi_n$ we get
$$
D_{\xi_n} \xi_n
= d_{\xi_n} \xi_n - 2 \frac{\<\xi_n, \bar x\>}{1+|x|^2} \xi_n
= - 2 \frac{\<\xi_n, \bar x\>}{1+|x|^2} \xi_n
$$
so $\alpha_l(D_{\xi_n} \xi_n) = 0$ for all $l$.

Now let $\xi = \sum_{j=1}^n a_j \xi_j$. Then
$$
\sigma(\xi,\xi)
= \sum_{j,k=1}^n a_j a_k \sigma(\xi_j, \xi_k)
= \biggl(
\sum_{j,k=1}^n a_j a_k \, \alpha_l(D_{\xi_j} \xi_k)
\biggr)_{l=1,\ldots,n-1}.
$$
For a fixed $l$ we have
$$
\displaylines{
\sum_{j,k=1}^n a_j a_k \, \alpha_l(D_{\xi_j} \xi_k)
=
\sum_{j,k=1}^{n-1} a_j a_k \biggl(
\frac{x_n |x_n|^2}{1+|y|^2+|x_n|^2}
+ \frac{1}{1+|y|^2}
\biggr)(\bar y_j \delta_{kl} + \bar y_k \delta_{jl})
\hfill\cr\hfill{}
+ 2 \sum_{j=1}^{n-1} a_j a_n
\frac{1 + |y|^2 - |x_n|^2 |y|^2}{1+|y|^2+|x_n|^2}
\delta_{jl}
\cr\hfill{}
=
2 \biggl(
\frac{x_n |x_n|^2}{1+|y|^2+|x_n|^2}
+ \frac{1}{1+|y|^2}
\biggr)
\sum_{j=1}^{n-1} a_l a_j \bar y_j
+ 2
\frac{1 + |y|^2 - |x_n|^2 |y|^2}{1+|y|^2+|x_n|^2}
a_l a_n.
}
$$
If we want to try big-O things again this is just $2 a_l a_n + O(|z|)$.

Our problem is to calculate the inner product on the quotient space,
which involves inverting the Fubini--Study matrix.
If not we're left with calculating
$$
|\sigma(\xi,\xi)|^2
= \sum_{l,m=1}^{n-1}
\sum_{j,k=1}^n
\sum_{p,q=1}^n
a_j a_k \bar a_p \bar a_q
\alpha_l(D_{\xi_j}\xi_k)
\ov{\alpha_m(D_{\xi_p} \xi_q)}
h_{Q,lm}
$$


\subsection*{Bravely retreating to the projective plane}

\def\foo#1{\frac{#1}{1+|x|^2}}

Set $n = 2$ so we can invert everything by hand.
The Fubini--Study matrix is
$$
\frac{1}{1+|x|^2}
\begin{pmatrix}
1 - \foo{|x_1|^2} & -\foo{x_2 \bar x_1}
\\
- \foo{x_1 \bar x_2} & 1 - \foo{|x_2|^2}
\end{pmatrix}.
$$
Its determinant is
$$
\frac{1}{(1+|x|^2)^2}
\det(I_2 - \bar x x^t / (1+|x|^2))
= \frac{1}{(1+|x|^2)^2}
(1 - |x|^2 / (1+|x|^2))
= \frac{1}{(1+|x|^2)^3}
$$
and its inverse is
$$
(1+|x|^2)^3
\begin{pmatrix}
1 - \foo{|x_2|^2} & \foo{x_2 \bar x_1}
\\
\foo{x_1 \bar x_2} & 1 - \foo{|x_1|^2}
\end{pmatrix}.
$$
The metric on the quotient is then
the inverse of
$$
\displaylines{
(1+|x|^2)^3
\begin{pmatrix}
1 & -y & -x_2
\end{pmatrix}
\begin{pmatrix}
1 - \foo{|x_2|^2} & \foo{x_2 \bar x_1} & 0
\\
\foo{x_1 \bar x_2} & 1 - \foo{|x_1|^2} & 0
\\
0 & 0 & e^{t} \frac{(1+|y|^2)^2}{(1+|x|^2)^3}
\end{pmatrix}
\begin{pmatrix}
1 \\ -\bar y \\ -\bar x_2
\end{pmatrix}
\hfill\cr\hfill{}
=
(1+|x|^2)^3
\begin{pmatrix}
1 & -y & -x_2
\end{pmatrix}
\begin{pmatrix}
1 - \foo{|x_2|^2} - \foo{\bar y x_2 \bar x_1}
\\
\foo{x_1 \bar x_2} - \bar y + \foo{\bar y |x_1|^2}
\\
-e^{t} \frac{\bar x_2(1+|y|^2)^2}{(1+|x|^2)^3}
\end{pmatrix}
\cr\hfill{}
=
(1+|x|^2)^3
\!
\biggl(
1
- \foo{|x_2|^2}
- \foo{\bar y x_2 \bar x_1}
- \foo{y x_1 \bar x_2}
+ |y|^2
- \foo{|y|^2 |x_1|^2}
+ e^{t} \frac{|x_2|^2(1+|y|^2)^2}{(1+|x|^2)^3}
\biggr)
\cr\hfill{}
=
(1+|x|^2)^3\biggl(
\frac{1 + |x_1|^2}{1+|x|^2}
+ |y|^2 \frac{1 + |x_2|^2}{1+|x|^2}
- \foo{\bar y x_2 \bar x_1}
- \foo{y x_1 \bar x_2}
+ e^{t} \frac{|x_2|^2(1+|y|^2)^2}{(1+|x|^2)^3}
\biggr)
\cr\hfill{}
=
(1+|x|^2)^2\biggl(
1 + |x_1|^2
+ |y|^2(1 + |x_2|^2)
- \bar y x_2 \bar x_1
- y x_1 \bar x_2
+ e^{t} \frac{|x_2|^2(1+|y|^2)^2}{(1+|x|^2)^2}
\biggr).
}
$$
Using $x_1 = x_2 y$ we get
$$
\displaylines{
(1+|x|^2)^2\biggl(
1
+ |y|^2 (1 + 2 |x_2|^2)
- (y^2 + \bar y^2)|x_2|^2
+ e^{t} \frac{|x_2|^2(1+|y|^2)^2}{(1+|x|^2)^2}
\biggr).}
$$

We only have $\alpha = dx_1 - y dx_2 - x_2 dy$ so
\begin{align*}
\sigma(\xi,\xi)
&= a_1^2 \sigma(\xi_1, \xi_1)
+ 2 a_1 a_2 \sigma(\xi_1, \xi_2)
\\
&= - 2 a_1^2 \biggl(
\frac{x_2^2 \bar x_1}{1+|x|^2}
+ \frac{\bar y}{1+|y|^2}
\biggr)
+ 2 a_1 a_2 \biggl(
1 - \frac{|x_2|^2}{1+|x|^2}
- \frac{\bar x_1 x_2 y}{1+|x|^2}
\biggr).
\end{align*}
We should use that $x_1 = x_2 y$ somewhere.
Then $|x|^2 = |x_2|^2(1 + |y|^2)$
and
$$
\sigma(\xi,\xi)
= - 2 a_1^2 \biggl(
\frac{|x_2|^2 x_2}{1+|x_2|^2(1+|y|^2)}
+ \frac{1}{1+|y|^2}
\biggr) \bar y
+
\frac{2 a_1 a_2 }{1+|x_2|^2(1+|y|^2)}
.
$$
We also have $\pi_e(\xi) = (a_1 x_2 + a_2 y) e_1 + a_2 e_2$
and $\pi_f(\xi) = a_1$.

We'll switch to $x = x_2$ now.
The curvature tensor is
\begin{align*}
R_t(\xi,\ov\xi,\xi,\ov\xi)
&= 2 |\pi_e(\xi)|^4_e
+ 2 |\pi_f(\xi)|^4_e
- |\sigma(\xi,\xi)|^2_{q,t}
\\
&= 2|(a_1 x + a_2 y) e_1 + a_2 e_2|^4_e
+ 2 e^{-t} \frac{|a_1|^4}{(1+|y|^2)^2}
\cr
&\qquad
- \biggl|2 a_1^2 \biggl(
\frac{|x|^2 x}{1+|x|^2(1+|y|^2)}
+ \frac{1}{1+|y|^2}
\biggr) \bar y
-
\frac{2 a_1 a_2 }{1+|x|^2(1+|y|^2)}
\biggr|^2
\!\!
/ h_{q,t}.
\end{align*}
Suppose $a_2 = 0$, we get
\begin{align*}
R_t(\xi,\ov\xi,\xi,\ov\xi)
&= \frac{2|x|^2 (1 + |y|^2)^2}{(1+|x|^2+|y|^2)^4} |a_1|^4
+  \frac{2 e^{-t}}{(1+|y|^2)^2} |a_1|^4
\cr
&\qquad
- 4 \biggl|
\frac{|x|^2 x}{1+|x|^2(1+|y|^2)}
+ \frac{1}{1+|y|^2}
\biggr|^2 |y|^2 |a_1|^4
/ h_{q,t}.
\end{align*}
The function $h_{q}(x,y,t)$ is of the form $h_q(x,y,t) = p(x,y) + e^t r(x,y)$,
where $p$ and $r$ are polynomials where $p(0,0) = 1$ and $r(0,0) = 0$.
The whole expression is then of the form
$$
R_1(x,y) + e^{-t} R_2(y) - \frac{P(x,y)}{Q(x,y) + e^t S(x,y)}
$$
for rational functions $R_1$, $R_2$ and polynomials $P,Q,S$ such that
$R_1(0,0) = 0$, $R_2(0) = 1$, $P(0,0) = 0$, $Q(0,0) = 1$ and $S(0,y) = 0$.
All of these are positive around $(0,0)$. Multiplying out we have to check the
positivity of
$$
R_1 Q + e^t R_1 S + e^{-t} R_2 Q + R_2 S - P
$$
around $0$ for all $t$.
At $0$ this equals $e^{-t}$.



\bibliographystyle{plainurl}
\bibliography{main}

\end{document}
