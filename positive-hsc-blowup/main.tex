\documentclass[10pt,a4paper]{amsart}

\usepackage{lmodern}
\linespread{1.1}
\usepackage[utf8]{inputenc}
\usepackage[T1]{fontenc}

\usepackage{fancyref}
\usepackage[colorlinks=true]{hyperref}

\usepackage{amsmath}
\usepackage{amssymb}
\usepackage{amsthm}

\newtheorem{theo}{Theorem}
\newtheorem{prop}[theo]{Proposition}
\newtheorem{lemm}[theo]{Lemma}
\newtheorem*{maintheo}{Theorem}

\allowdisplaybreaks

\newcommand{\kk}[1]{\mathbb{#1}}
\newcommand{\cc}[1]{\mathcal{#1}}

\def\<{\langle}
\def\>{\rangle}

\def\qandq{\quad\text{and}\quad}
\def\ov#1{\overline{#1}}

\def\p{\partial}
\def\bp{\bar\partial}

\DeclareMathOperator{\Span}{span}
\DeclareMathOperator{\Ric}{Ric}
\DeclareMathOperator{\Gr}{Gr}
\DeclareMathOperator{\GL}{GL}
\DeclareMathOperator{\im}{Im}
\DeclareMathOperator{\Vol}{Vol}
\DeclareMathOperator{\Ker}{Ker}
\DeclareMathOperator{\End}{End}
\DeclareMathOperator{\Aut}{Aut}
\DeclareMathOperator{\Hom}{Hom}
\DeclareMathOperator{\Adj}{Adj}
\DeclareMathOperator{\id}{id}
\DeclareMathOperator{\tr}{tr}

\def\hsc{holomorphic sectional curvature}
\def\bl#1{\widehat{#1}}
\def\blX{\bl{X}}

\author{Gunnar \TH\'or Magn\'usson}
\address{Hafnarfj\"or\dh{}ur, Iceland}
\email{gunnar@magnusson.io}
\date{\today}
\title[Positive holomorphic sectional curvature of blowups of points]
{Positive holomorphic sectional\\curvature of blowups of points}

\hypersetup{
 pdfauthor={Gunnar Þór Magnússon},
 pdftitle={Positive holomorphic sectional curvature of blowups of points},
 pdfkeywords={},
 pdfsubject={},
 pdflang={English}}


\begin{document}


\begin{abstract}
We show that the blowup of a point on a compact K\"ahler manifold of positive
holomorphic sectional curvature has positive holomorphic sectional curvature.
\end{abstract}

\maketitle


\section*{Introduction}

Let $X$ be a compact K\"ahler manifold of dimension $n$.
Let $h$ be a K\"ahler metric on $X$.
We write $D$ for its Chern connection and
$R(\alpha,\ov\beta,\gamma,\ov\delta) = h(\frac i2
D^2_{\smash{\alpha,\ov\beta}}\gamma, \ov\delta)$ for its curvature tensor. The
holomorphic sectional curvature of $h$ is
$$
H(\xi)
= \frac{R(\xi, \ov\xi, \xi, \ov\xi)}{|\xi|^4},
$$
where $\xi$ is a nonzero tangent field.
We say that $h$ has positive holomorphic sectional curvature if $H > 0$ for all
tangent fields.
Examples of manifolds that carry such metrics are the complex projective space
with its Fubini--Study metric, the standard metric on a Grassmannian manifold,
and projective bundles over bases of positive holomorphic sectional curvature
~\cite{alvarez2018projectivized}
(and therefore also bundles of Grassmannians and flag manifolds over similar
bases).
Tsukamoto~\cite{tsukamoto1957kahlerian} showed that manifolds that admit such
metrics are simply connected, and recently Xiaokui Yang~\cite{yang2017rc}
proved that such manifolds are projective and rationally connected, answering a
question of Yau~\cite[Problem~67]{yau1993open}.

A related question of Yau (again Problem~67) is whether the blowup of a compact
K\"ahler manifold of positive \hsc{} along a smooth submanifold again has
positive \hsc. In this note we claim to prove this for the blowup of a point:

\begin{maintheo}
Let $X$ be a compact K\"ahler manifold of dimension $n > 1$ that admits a
K\"ahler metric of positive \hsc.
If $\mu : \blX \to X$ is the blowup of $X$ at a point, then $\blX$ also admits
a K\"ahler metric of positive \hsc.
\end{maintheo}

In particular, this shows that del~Pezzo surfaces, which are blowups of the
complex projective plane in at most 9 points, admit K\"ahler metrics of
positive \hsc.

Our approach is one of brute force, essentially the same as that of
\'Alvarez~\cite{alvarez2016positive},
\'Alvarez,
Heier and Zheng~\cite{alvarez2018projectivized} or Chaturvedi and
Heier~\cite{chaturvedi2020hermitian} in the cases of projective bundles and
fiber bundles. It is known that there exists a closed $(1,1)$-form on the
blowup that restricts to the Fubini--Study metric on the exceptional divisor.
We choose local coordinates carefully, grit our teeth, and calculate the
curvature of that form plus a multiple of the pullback of the original metric
on the base using the classical Codazzi--Griffiths equations for the curvature
of a subbundle.



\section{Preliminaries}

\begin{prop}
\label{prop:positive}
Let $X$ be a compact K\"ahler manifold and let $h$ be a K\"ahler metric of
positive \hsc. Let $b$ be a Hermitian form on $X$.
Then $e^t h + b$ is a metric of positive \hsc{} for all $t \gg 0$.
\end{prop}

\begin{proof}
Chaturvedi and Heier~\cite{chaturvedi2020hermitian} prove this when $h$ and $b$
are both metrics.
We will basically reproduce their proof, which relies on
Wu's~\cite{wu1973remark} characterization of \hsc:
$$
H(x, \xi) = \sup_{f : D \to X} R_{f^*h}(0, \partial/\partial z),
$$
where $f : D \to X$ is an embedding of the unit disk that maps
$0$ to $x$ and $\partial/\partial z$ to $\xi$ at $0$.
The supremum can always be achieved; for K\"ahler metrics we do this by
choosing normal coordinates centered at a point such that $\xi$ is a multiple
of one of the coordinate fields at the center. (Wu handles general Hermitian
metrics.) When the supremum is achieved, we have $D_\xi \xi = 0$ at $x$ (by the
Codazzi--Griffiths equations for the curvature of a subbundle).

Let then $f : D \to X$ be an embedding that realizes the \hsc{} of $h$ at
$(x,\xi)$.
The curvature of the pullback of $h_t = e^t h + b$ to $D$ is
\begin{align*}
R_{f^*h}
= \partial_z \bar\partial_z \log(e^t |\xi|^2_h + |\xi|^2_b)
&= \partial_z \frac{e^t \<\xi, \ov{D_\xi\xi}\>_h + \bar\partial_\xi |\xi|^2_b}{e^t |\xi|^2_h + |\xi|^2_b}
\\
&= \frac{e^t H_h(\xi)|\xi|^4_h + e^t \<D_\xi \xi, \ov{D_\xi\xi}\>_h + \partial_\xi\bar\partial_\xi |\xi|^2_b}{e^t |\xi|^2_h + |\xi|^2_b}
\\
&\qquad
- \frac{e^t \<D_\xi \xi, \ov{\xi}\>_h + \partial_\xi |\xi|^2_b}{e^t |\xi|^2_h + |\xi|^2_b}
\frac{e^t \<\xi, \ov{D_\xi\xi}\>_h + \bar\partial_\xi |\xi|^2_b}{e^t |\xi|^2_h + |\xi|^2_b}.
\end{align*}
We have $D_\xi \xi = 0$ at $x$, so at the origin this simplifies to
\begin{align*}
R_{f^*h}
&= \frac{e^t H_h(\xi)|\xi|^4_h + \partial_\xi\bar\partial_\xi |\xi|^2_b}{e^t |\xi|^2_h + |\xi|^2_b}
- \frac{|\, \partial_\xi |\xi|^2_b \,|^2}{(e^t |\xi|^2_h + |\xi|^2_b)^2}
\\
&= H_h(\xi) |\xi|^2_h + O(e^{-t})
\end{align*}
which is positive for all $t$ large enough since $H_h$ is positive.

The \hsc{} is a smooth function on $\kk P(T_X)$, so if it is positive at a
point it is positive on a neighborhood of that point.
As $X$ is compact, we conclude that $h_t$ has positive \hsc{} for all $t \gg 0$.
\end{proof}


When $h$ and $b$ are both metrics there is a quicker proof:
Using the Codazzi--Griffits equations and the short exact sequence $0 \to T_X
\to T_X \oplus T_X \to T_X \to 0$ we calculate the curvature of $e^t h+b$ to be
$$
R(\xi, \ov\xi, \xi, \ov\xi)
= e^t R_h(\xi, \ov\xi, \xi, \ov\xi) + R_b(\xi, \ov\xi, \xi, \ov\xi)
- |\sigma(\xi)\xi|^2_{Q,t},
$$
where $\sigma(\xi)\xi = D_{h,\xi}\xi - D_{b,\xi} \xi$ is the second fundamental
form and the norm is on the ``quotient'' bundle.
If we simultaneously diagonalize the Hermitian forms at a point we see that
$|\eta|^2_{Q,t} \to |\eta|^2_b$ when $t \to \infty$ for any fixed tangent field
$\eta$. Then the \hsc{} of $e^t h + b$ is positive for all $t$ large enough (or
negative, if $h$ has negative curvature).



\begin{prop}
\label{prop:fs}
Let $\mu : \bl X \to X$ be the blowup of $X$ at a point $p$.
There exists a closed $(1,1)$-form $\beta$ on $X$ whose restriction to
the exceptional divisor is the Fubini--Study metric.
\end{prop}

\begin{proof}
This is proved for blowups of smooth submanifolds in
Voisin's textbook~\cite{voisin2002theorie}. We can simplify that proof a little
since we're only blowing up points.

Locally around $p$, which may assume is the origin in a complex vector space
$V$, the blowup is
$$
\bl V
= \{ (v,[w]) \in V \times \kk P(V) \mid v \in \kk C w \}
$$
and $\mu$ is the projection onto the first factor.
Pick an inner product on $V$ and let $B(r) \subset V$ be a ball of radius $r$
centered at $0$.
We pick $r$ so that $B(r)$ fits into the coordinate chart that implicitly lurks
in the background.
Let $\psi$ be a bump function supported on that chart that is identically $1$
on $B(r)$.
The $(1,1)$-form $\frac i2 \partial\bar\partial \log |v|^2$ on $V \setminus
\{0\}$ descends to $\kk P(V)$ and defines the Fubini--Study metric.
If $p_j : V \times V \setminus \{0\} \to V$ for $j = 1,2$ are the projections
onto the first and second factors then
$\frac i2 \partial \bar\partial (p_1^*\psi \log |p_2^*v|^2)$
defines a closed $(1,1)$-form on $V \times \kk P(V)$ that restricts to the
pullback of the Fubini--Study metric by $p_2$ on $B(r) \times \kk P(V)$.
It also extends to the rest of $X$ by zero.
Its restriction to $\bl V$ is the form we want.
\end{proof}



\section{The proof}

Let $X$ be a compact K\"ahler manifold of dimension $\dim_{\kk C} X = n$ that
admits a metric $h$ of positive \hsc.
Let $p \in X$ be a point and blow it up to obtain $\mu : \bl X \to X$.
Let $b$ be the Hermitian form associated to the $(1,1)$-form $\beta$ on $\bl X$
we constructed in Proposition~\ref{prop:fs}.
Then $h_t = e^t \mu^*h + b$ is a K\"ahler metric on $\bl X$ for all $t$ large
enough.
We are going to show it also eventually has positive \hsc.

It is actually enough to show that $h_t$ has positive \hsc{} on the exceptional
divisor.
(FIXME: Need more than positive; need bounded away from $0$.)
If it does, then $h_t$ has positive \hsc{} on a neighborhood $U$ around the
divisor for all $t$ large enough.
Then Proposition~\ref{prop:positive} shows that $h_t$ also has positive \hsc{}
on $\bl X \setminus U$ for large enough $t$.

We're going to argue that we can calculate the \hsc{} on the exceptional divisor
at the center of a well-chosen coordinate chart.
Recall that locally around $p$ the blowup is
$$
\bl V
= \{ (v,[w]) \in V \times \kk P(V) \mid v \in \kk C w \}.
$$
If $f \in \GL V$ then $f$ acts on the blowup by $f(v, [w]) = (f(v), [f(w)])$.
This is an isomorphism that maps the exceptional divisor to itself, and we can
map any point on the divisor to any other point.

Let $(0, [w])$ be a point on $E$.
Let's choose normal coordinates $(x_1,\ldots,x_n)$ centered at $p$.
There exists $f \in U(n)$ so that $f(w) = (0 \ldots, 0, 1)$, and the
coordinates obtained by applying $f$ to the old ones are still centered at $p$
and are normal there because $f \in U(n)$.
Picking the chart $\{(y_1, \ldots, y_n) \in \kk C^n \mid y_n \not= 0 \}$ for
$\kk P(V)$
we realize the blowup as
$$
\bl X
= \{ (x,y) \in \kk C^n \times \kk C^{n-1}
\mid x_j y_k = x_k y_j \text{ for $j,k = 1,\ldots,n$, where $y_n = 1$}  \}.
$$
In these coordinates the point we want to calculate the \hsc{} at is $(0,0)$
and we have $D_{h,\xi} \partial / \partial x_j = 0$ at $0$ for $j = 1, \ldots, n$.

We also note that close to the exceptional divisor, $h_t = e^t \mu^* h + b$ is
just the restriction of the product metric $e^t p_1^* h \oplus p_2^* g$ on
$V \times \kk P(V)$ to $\bl X$, where $p_j$ are the projections onto the
factors and $g$ is the Fubini--Study metric.


\begin{lemm}
The metric $h_t = e^t \mu^*h + b$ has positive \hsc{} on $E$ for all $t$.
\end{lemm}

\begin{proof}
The only equations that give us any information at $(0,0)$ are
$$
x_j - y_j x_n = 0, \quad j = 1, \ldots, n-1.
$$
Their differentials are
$$
dx_j - y_j dx_n - x_n dy_j = 0, \quad j=1,\ldots,n-1
$$
and so the tangent fields
$$
\xi_j = x_n e_j + f_j,
\quad j=1,\ldots,n-1,
\qandq
\xi_n = \sum_{j=1}^{n-1} y_j e_j + e_n
$$
are a basis for
the intersection of all the kernels, that is, $T_X$.
Here $e_j$ is the tangent field corresponding to the coordinate $x_j$
and $f_k$ the one corresponding to $y_k$.
Note that the normal bundle at $(0,0)$ is spanned by $(e_1, \ldots, e_{n-1})$.

Any holomorphic tangent field $\xi$ near $(0,0)$ can be written as
$$
\xi = \sum_{j=1}^n a_j \xi_j
= \sum_{j=1}^{n-1} (a_j x_n + a_n y_j) e_j + a_j f_j
+ a_n e_n,
$$
where the $a_j$ are holomorphic functions.
We want to check the positivity of the curvature at the origin, which we can do
on the unit sphere, so we may assume that $\sum_{j=1}^n |a_j|^2 = 1$ there.

Recall that the curvature of $h_t$ is
$$
R_{h_t}(\xi, \ov\xi, \xi, \ov\xi)
= e^t p_1^* R_h(\xi, \ov\xi, \xi, \ov\xi)
+ p_2^* R_g(\xi, \ov\xi, \xi, \ov\xi)
- |\sigma(\xi)\xi|^2,
$$
where $\sigma(\xi)\xi = \pi_N(D_{e^th \oplus g,\xi} \xi)$ is the second
fundamental form, and the norm is on the quotient bundle.

First,
$$
e^t p_1^*R_h(\xi, \ov\xi, \xi, \ov\xi)
= e^{-t} |a_n|^4 H_h(e_n)
$$
at the origin. Second,
$$
p_2^*R_g(\xi, \ov\xi, \xi, \ov\xi)
= 2 \biggl(\sum_{j=1}^{n-1} |a_j|^2\biggr)^2
= 2(1 - |a_n|^2)^2
$$
at the origin
because the Fubini--Study metric has constant \hsc{} 2.

Let's now write $\pi_N$ for the projection onto the orthogonal complement of
$T_{\blX}$ in $T_{\kk C^n \times \kk C^{n-1}|\blX}$.
At the origin it is just the projection onto the first $n-1$ coordinates.

Third, then,
\begin{align*}
\pi_N(p_1^*D_{h,\xi} \xi)
&= \sum_{j=1}^{n-1} d_{\xi}(a_j x_n + a_n y_j) \, e_j
\\
&= \sum_{j=1}^{n-1} (a_j d_{\xi}x_n + a_n d_{\xi} y_j) \, e_j
\\
&= \sum_{j=1}^{n-1} (a_j a_n + a_n a_j ) \, e_j
= 2 a_n \sum_{j=1}^{n-1} a_j \, e_j
\end{align*}
at the origin
because $D_{h,\xi} e_j = 0$ there by our choice of coordinates.

And fourth,
$$
\pi_N(p_2^*D_{g,\xi} \xi)
= \pi_N \biggl( \sum_{j=1}^{n-1} d_\xi y_j \, f_j + y_j p_2^*D_{g,\xi} f_j \biggr) = 0
$$
at the origin.
The contribution of the second fundamental form to the curvature is then
$$
|\sigma(\xi)\xi|^2
= 4 |a_n|^2 \sum_{j=1}^{n-1} |a_j|^2 |e_j|^2
= 4 e^{-t} |a_n|^2 \sum_{j=1}^{n-1} |a_j|^2
= 4 e^{-t} |a_n|^2(1 - |a_n|^2)
$$
so finally
$$
R_{h_t}(\xi, \ov\xi, \xi, \ov\xi)
= e^{-t} |a_n|^4 H_h(e_n)
+ 2(1 - |a_n|^2)^2
- 4 e^{-t} |a_n|^2(1 - |a_n|^2)
$$
at the origin.
We want to show that this is positive for all $a_n$ with $|a_n| \leq 1$.

Set $m = \inf_{\kk P(T_X)} H_h > 0$ and let
$$
f_t(x) = e^{-t} m x^2 + 2(1-x)^2 - 4e^{-t} x(1-x).
$$
Then $R_{h_t}(\xi, \ov\xi, \xi, \ov\xi) \geq f(|a_n|^2)$.
We have
$$
f_t(x)
= (m e^{-t} + 4e^{-t} + 2)x^2 - 4(e^{-t} + 1)x + 2.
$$
As a polynomial in $x$ the coefficient of $x^2$ of $f_t$ is positive.
Now
$$
f_t'(x)
= 2(m e^{-t} + 4e^{-t} + 2)x - 4(e^{-t} + 1)
$$
so $f_t'(x_0(t)) = 0$ when
$$
x_0(t) = \frac{2(e^{-t} + 1)}{m e^{-t} + 4e^{-t} + 2}.
$$
At $x_0(t)$, where $f_t$ achieves its global minimum, we have
$$
f_t(x_0(t))
=
2 - 2 \frac{(e^{-t} + 1)^2}{m e^{-t} + 4e^{-t} + 2}.
$$
At last
$$
\frac{\partial}{\partial t} f_t(x_0(t))
= -2 \frac{2e^{-t}(e^{-t}+1)(m e^{-t} + 4e^{-t} + 2) + (e^{-t}+1)^2(me^{-t}+4e^{-t})}{(m e^{-t} + 4e^{-t} + 2)^2}
< 0,
$$
so $f_t(x_0(t))$ is strictly decreasing in $t$ and tends to $0$ as $t \to
\infty$, so $f_t(x) \geq f_t(x_0(t)) > 0$ for all $x \in [0,1]$ and all $t \in
\kk R$.
\end{proof}

In fact, as long as $m + 4 > 0$ the \hsc{} of $h_t$ will be positive on $E$ for
all $t$.



It might be possible to extend this approach to cover the blowup of a smooth
submanifold of a compact K\"ahler manifold of positive \hsc.
One problem
is that the blowup along a submanifold $Y \subset X$ is
constructed by picking local coordinates $(z_1,\ldots,z_n)$ such that $Y = \{z
\mid z_{k+1} = \cdots = z_n = 0\}$,
and the standard method of producing normal coordinates centered at a point
(see for example Zheng~\cite{zheng2000complex})
does not preserve these equations.
Our trick here thus does not carry through verbatim and there might be extra
difficulties involved in the curvature calculations, comparable to working
with general Hermitian metrics.
If any enterprising soul takes this on I'd be happy to hear about it.


For any Hermitian metric $h$ and holomorphic tangent field $\xi$ we have
$$
- \frac{1}{|\xi|^2} \log |\xi|^2
= H(\xi)
- \frac{1}{|\xi|^2} \biggl(
\frac{|D_\xi \xi|^2}{|\xi|^2}
- \frac{\<D_\xi \xi, \ov \xi\>}{|\xi|^4}
\biggr)
\leq H(\xi)
$$
by Cauchy--Schwarz.


\begin{prop}
Let $h$ be a K\"ahler metric of positive \hsc{} on $B(r)$.
Then
$$
h_t(\xi, \ov\eta)
= e^t h(\xi, \ov\eta) + \tfrac i2 \partial \bar\partial \log |z|^2(\xi, \ov\eta)
$$
has positive \hsc{} on $B(r) \setminus\{0\}$ for $t \geq t_0$.
\end{prop}

Let $B(r)$ be a small ball around $p$.
On $B(r) \setminus \{0\} =: B^*(r)$
the metric $h_t$ identifies with
\begin{align*}
h_t(\xi, \ov\eta)
&= e^t h(\xi, \ov\eta) + \tfrac i2 \partial \bar\partial \log |z|^2(\xi, \ov\eta)
\\
&= e^t h(\xi, \ov\eta)
+ \frac{\<\xi, \ov\eta\>}{|z|^2}
- \frac{\<\xi, \ov z\>}{|z|^2}
\frac{\<z, \ov\eta\>}{|z|^2}.
\end{align*}
If we denote by $b$ the form on the right-hand side we have $b(\xi, \ov z) = 0$
for all $\xi$ and $b \geq 0$ by Cauchy--Schwarz.
The form $h_t$ is then a K\"ahler metric for all $t$.


Actually, note that
\begin{align*}
\partial b(\xi, \ov\eta)
&= \frac{\<\partial \xi, \ov\eta\>}{|z|^2}
{-} \frac{\<dz, \ov z\>}{|z|^2} \frac{\<\xi, \ov\eta\>}{|z|^2}
{-} \frac{\<\partial \xi, \ov z\>}{|z|^2} \frac{\<z, \ov\eta\>}{|z|^2}
{-} \frac{\<\xi, \ov z\>}{|z|^2} \frac{\<dz, \ov\eta\>}{|z|^2}
{+} 2\frac{\<dz, \ov z\>}{|z|^2} \frac{\<\xi, \ov z\>}{|z|^2} \frac{\<z, \ov\eta\>}{|z|^2}
\\
&= b(\partial \xi, \ov\eta)
- b\biggl(\frac{\<dz, \ov z\>}{|z|^2} \xi, \ov\eta\biggr)
- b\biggl(\frac{\<\xi, \ov z\>}{|z|^2} dz, \ov\eta\biggr)
\end{align*}
so
$$
D\xi := d \xi
- \frac{\<dz, \ov z\>}{|z|^2} \xi
- \frac{\<\xi, \ov z\>}{|z|^2} dz
$$
is \emph{a} Chern connection for $b$.
Then
\begin{align*}
\bar\partial D \xi
&= \bar\partial \biggl(
d \xi
- \frac{\<dz, \ov z\>}{|z|^2} \xi
- \frac{\<\xi, \ov z\>}{|z|^2} dz
\biggr)
\\
&= \frac{\<dz, \ov{d\bar z}\>}{|z|^2} \xi
+ \frac{\<z, \ov{d\bar z}\>}{|z|^2} \frac{\<dz, \ov z\>}{|z|^2} \xi
- \frac{\<\xi, \ov{d\bar z}\>}{|z|^2} dz
+ \frac{\<z, \ov{d\bar z}\>}{|z|^2} \frac{\<\xi, \ov z\>}{|z|^2} dz
\\
&= b(dz, \ov{d\bar z}) \xi - b(\xi, \ov{d\bar z}) dz
\end{align*}
is \emph{a} curvature form for $b$, and
$$
R(\alpha,\ov\beta,\gamma,\ov\delta)
= b(D^2_{\alpha,\ov\beta} \gamma, \ov\delta)
= b(\alpha, \ov\beta) b(\gamma, \ov\delta)
+ b(\gamma, \ov\beta) b(\alpha, \ov\delta)
$$
is the curvature tensor of any Chern connection of $b$.
Note that $R(\alpha,\ov\alpha,\alpha,\ov\alpha) = 2 b(\alpha,\ov\alpha)^2 \geq 0$
and explodes as $z \to 0$.
Degenerate geometry then gives
$$
R_t = e^t R_h + R_b - \sigma^* q_t,
$$
where $\sigma = D_h - D_b$ and $q_t$ is the quotient form.
For $x > 0$ and $y \geq 0$ we have
$$
\frac{e^t x y}{e^t x + y} \to y,
\quad
t \to \infty,
$$
so by simultaneously diagonalizing $h$ and $b$ we see that $q_t \to b$ as $t
\to \infty$.
We have $\sigma(\xi, \xi) = A_h(\xi) \xi + 2\frac{\<\xi, \ov z\>}{|z|^2} \xi$,
where $A_h$ is a local matrix for $D_h$.
Then
$$
\frac{|\sigma(\xi, \xi)|_{q_t}^2}{|\xi|_t^4}
\leq \frac{|\sigma(\xi, \xi)|_{t}^2}{|\xi|_t^4}
\leq
\frac{|A_h(\xi) \xi|^2_{q_t}}{(e^t |\xi|^2_h + |\xi|^2_b)^2}
+ 4\frac{|\xi|^2}{|z|^2}
\frac{|\xi|^2_{q_t}}{(e^t |\xi|^2_h + |\xi|^2_b)^2}
$$
for all $\xi$ and $t$.

Take $t_0$ large enough so that $|q_t - b| \leq 1$ for $t \geq t_0$.
Then the first term is bounded by some constant $C$ for all $\xi$ and $t \geq
t_0$. The term $|\xi|^2_{q_t} / (e^t |\xi|_h^2 + |\xi|^2_b)$ is then also
bounded by a constant (say $C$) for all $\xi$ and $t \geq t_0$, and
so is $1/(|z|^2 |\xi|^2_b)$, (say $C$ again). Then
$$
\frac{|\sigma(\xi, \xi)|_{q_t}^2}{|\xi|_t^4}
\leq
\frac{K}{e^t}
+ \frac{M |\xi|^2}{|z|^2(e^t|\xi|_h^2 + |\xi|_b^2)}
$$
for all $t \geq t_0$ and some constants $K$ and $M$.
We need to analyze the other terms and see if they can beat the one
that explodes at $0$.
The only one that might is
$$
\frac{R_b(\xi,\ov\xi,\xi,\ov\xi)}{|\xi|_t^4}
= \frac{2 |\xi|_b^4}{|\xi|_t^4}
= \frac{2}{(1 + e^t |\xi|_h^2/|\xi|_b^2)^2}
\to 2
$$
as $z \to 0$ for fixed $t$. So that doesn't help

Write $z = r \theta$ with $|\theta| = 1$. Then
$$
|\xi|^2_b = \frac{|\xi|^2}{r^2} - \frac{|\<\xi, \ov\theta\>|^2}{r^2}.
$$

OK, so these estimates.
We have
$$
b(\xi,\ov\xi) \leq \frac{2|\xi|^2}{|z|^2}
$$
by Cauchy--Schwarz. Then $|z|^2 b$ is bounded.
I don't think that $|z|^2 b(\xi, \ov\xi)$ has a limit as $z \to 0$:
Take $\xi = (1, 0, \ldots, 0)$ and $z = (z, 0, \ldots, 0)$ on the one hand and
$z = (0, z, 0, \ldots, 0)$ on the other.
Along one direction we get the limit $0$ and along the other we get $1$.
This could be a problem.
Is
$$
\lim_{z \to 0} \frac{|\<\xi, \ov z\>|^2}{|z|^2}
$$
anything?
If it is shouldn't it just be the derivative of $|\<\xi, \ov z\>|^2$?
You know, the complex linear one it doesn't have?
OK. So. Let
$$
f(r,\theta) = r^2|\<\xi, \ov{\theta}\>|^2
$$
for $\theta \in S(V)$. Then
$$
\frac{f(r, \theta)}{r^2} \to |\<\xi, \ov{\theta}\>|^2
$$
as $r \to 0$, and the RHS can be anything between $0$ and $|\xi|$.

Have to justify the limit $q_t \to b$ more, show that it can be done uniformely.
Probably test the difference on the unit sphere.


Suppose we picked $U$ so that the coordinates are normal with respect to $h$ at
$0$.
Then
$$
h(\xi,\ov\eta)
= \<\xi, \ov\eta\>
- \sum_{j,k,l,m} R_{jklm} \xi_j \ov{\eta}_k z_l \ov z_m
+ O(|z|^3)
$$
for $\xi = \sum_j \xi_j e_j$ and $\eta = \sum_k \eta_k e_k$.

Let
$$
p :=
\p \bp \log(|z|^2 + \epsilon)
= \p \frac{\<z, \ov{dz} \>}{|z|^2 + \epsilon}
= \frac{\<dz, \ov{dz} \>}{|z|^2 + \epsilon}
- \frac{\<dz, z \>}{|z|^2 + \epsilon}
\frac{\<z, \ov{dz} \>}{|z|^2 + \epsilon}
$$
Then
$$
p(\xi,\ov\xi)
= \frac{1}{|z|^2 + \epsilon}
\biggl(
|\xi|^2
- \frac{|\<\xi, \ov z\>|^2}{|z|^2 + \epsilon}
\biggr)
\geq
\frac{1}{|z|^2 + \epsilon}
\biggl(
|\xi|^2
- \frac{|\xi|^2 |z|^2}{|z|^2 + \epsilon}
\biggr)
=
\frac{\epsilon |\xi|^2}{(|z|^2 + \epsilon)^2}
$$
so $p$ is a Hermitian metric on $V$.
It doesn't give a metric on $\kk P(V)$ though because it's not invariant
under scaling.
We only have $p_\epsilon(\lambda z) = p_{\epsilon/\lambda}(z)$.
Any homogeneous function is zero at $0$, so we can't get a metric on $\kk P(V)$
this way.

Have
$$
\log |\xi|^2_t
= t + \log\biggl(
|\xi|^2_h
+ e^{-t} \frac{|\xi|^2}{|z|^2}
- e^{-t} \frac{\<\xi, \ov z\>}{|z|^2}
\frac{\<z, \ov\xi\>}{|z|^2}
\biggr).
$$
We have
$$
\frac{\<\xi, \ov\eta\>}{|z|^2}
- \frac{\<\xi, \ov z\>}{|z|^2}
\frac{\<z, \ov \eta\>}{|z|^2}
= \frac{1}{|z|^2} \< P_z\xi, \ov{P_z\eta} \>,
$$
where $P_z : \kk C^n \to \kk C^n$ is the orthogonal projection onto $(\kk
C z)^\perp$.
Can also write $P_z(\xi) = \xi - \frac{\<\xi, \ov z\>}{|z|^2} z$,
which leads back where we were.
Then
$$
\<\xi, \ov\eta\>_t
= \<\xi, \ov\eta\>_h
+ \frac{e^{-t}}{|z|^2} \< P_z \xi, \ov{P_z \eta} \>.
$$

Morally, if we look at the matrices:
$H_t = O(1/|z|^2)$, so $H_t^{-1} = O(|z|^2)$.
$\partial H_t = O(1/|z|^3)$ and $\partial\bar\partial H_t = O(1/|z|^4)$
so $R_t=O(1/|z|^2)$.
I don't get how Hitchin makes this work.


\subsection*{Degenerate}

Hello, darkness, my old friend:
\begin{align*}
b(\xi,\ov\eta)
&= \frac{\<\xi, \ov\eta\>}{|z|^2}
- \frac{\<\xi, \ov z\>}{|z|^2}
\frac{\<z, \ov\eta\>}{|z|^2}.
\\
\partial_\nu b(\xi, \ov\eta)
&= \frac{\<d_\nu \xi, \ov\eta\>}{|z|^2}
- \frac{\<\xi, \ov\eta\>}{|z|^2}
\frac{\<\nu, \ov z\>}{|z|^2}
- \frac{\<d_\nu \xi, \ov z\>}{|z|^2}
\frac{\<z, \ov\eta\>}{|z|^2}
\\
&\quad
+
\frac{\<\nu, \ov z\>}{|z|^2}
\frac{\<\xi, \ov z\>}{|z|^2}
\frac{\<z, \ov\eta\>}{|z|^2}
+ \frac{\<\xi, \ov z\>}{|z|^2}
\frac{\<z, \ov\eta\>}{|z|^2}
\frac{\<\nu, \ov z\>}{|z|^2}
- \frac{\<\xi, \ov z\>}{|z|^2}
\frac{\<\nu, \ov\eta\>}{|z|^2}
\\
&= b(d_\nu \xi, \ov\eta)
- b\biggl(\frac{\<\nu, \ov z\>}{|z|^2} \xi , \ov\eta\biggr)
- b\biggl(\frac{\<\xi, \ov z\>}{|z|^2} \nu , \ov\eta\biggr)
\\
&= b\biggl(d_\nu \xi
- \frac{\<\nu, \ov z\>}{|z|^2} \xi
- \frac{\<\xi, \ov z\>}{|z|^2} \nu
, \ov\eta
\biggr)
\end{align*}
for all holomorphic $\xi, \eta, \nu$, so
$$
D_\nu \xi := d_\nu \xi
- \frac{\<\nu, \ov z\>}{|z|^2} \xi
- \frac{\<\xi, \ov z\>}{|z|^2} \nu
$$
is a ``Chern connection'' for $b$.
We can add $\alpha(\nu) z$ to it for any 1-form $\alpha$ and get another
``Chern connection''.
We then also get a ``curvature tensor'' (this time in quotes because we don't
have a metric, but this tensor is unique)
$$
R(\alpha,\ov\beta,\gamma,\ov\delta)
= b(\alpha,\ov\beta) b(\gamma, \ov\delta)
+ b(\alpha,\ov\delta) b(\gamma, \ov\beta)
$$
that satisfies $R(\alpha,\ov\alpha,\alpha,\ov\alpha) = 2b(\alpha,\ov\alpha)^2$.
If $\pi : V \setminus \{ 0 \} \to \kk P(V)$ is the projection then $b$ is just
$\pi^*h_{FS}$, and we have $R_b = \pi^* R_{FS}$.
But even if this works you're no better off because you get terms involving $b$
and the second fundamental form, both of which blow up at $0$.


\subsection*{Maybe just calculate things?}

Let $(z,w)$ be the coordinates on the product.
Near $z = 0$ we have $w = (z_1, \ldots, z_{n-1})$.
If $\xi$ is as before we write $\xi = \xi' + \xi''$ and get
$$
\<\xi, \ov\eta\>_t
= \<\xi', \ov{\eta'} \>_h
+ e^{-t} \frac{\<\xi'', \ov{\eta''} \>}{1 + |z|^2 - |z_n|^2}
- e^{-t} \frac{\<\xi'', \ov{z'} \>}{1 + |z|^2 - |z_n|^2}
\frac{\<z', \ov{\eta''} \>}{1 + |z|^2 - |z_n|^2}.
$$
But remember that our frame is
$$
\xi_j = z_n e_j + f_j,
\quad j=1,\ldots,n-1,
\qandq
\xi_n = \sum_{j=1}^{n-1} z_j e_j + e_n
$$
outside of $z = 0$.
For $j, k < n$ we have
\begin{align*}
\< \xi_j, \ov{\xi_k} \>_t
&= |z_n|^2 h_{jk}
+ e^{-t} \frac{\delta_{jk}}{1 + |z|^2 - |z_n|^2}
- e^{-t} \frac{\ov z_j z_k}{(1 + |z|^2 - |z_n|^2)^2},
\quad
j, k < n
\\
\< \xi_j, \ov{\xi_n} \>_t
&= z_n \biggl(
\sum_{k=1}^{n-1} \ov z_k h_{jk} + h_{jn}
\biggr),
\quad j < n
\\
\< \xi_n, \ov{\xi_n} \>_t
&= \biggl(
\sum_{j,k=1}^{n-1} z_j \ov z_k h_{jk}
+ \sum_{j}^{n-1} z_j h_{jn}
+ \sum_{k}^{n-1} \ov z_j h_{nk}
+ h_{nn}
\biggr).
\end{align*}
At least this frame is orthogonal at $z = 0$.

Note that $z$ are normal coordinates for $h$ at $0$, so
$$
h_{jk}
= \delta_{jk} - R_{jklm} z_l \ov z_m + O(|z|^3).
$$
Up to order 2 we then have
\begin{align*}
\< \xi_j, \ov{\xi_k} \>_t
&= |z_n|^2 \delta_{jk}
+ e^{-t}(1 - |z|^2 + |z_n|^2) \delta_{jk}
- e^{-t}\ov z_j z_k ,
\quad
j, k < n
\\
\< \xi_j, \ov{\xi_n} \>_t
&= z_n \ov z_j,
\quad j < n
\\
\< \xi_n, \ov{\xi_n} \>_t
&=
1 + |z|^2 - |z_n|^2 - \sum_{lm} R_{nnlm} z_l \ov z_m
.
\end{align*}
because
$$
\frac{1}{(1+|z|^2-|z_n|^2)}
= \sum_{k \geq 0} (-1)^k (|z|^2 - |z_n|^2)^k
= 1 - |z|^2 + |z_n|^2 + O(|z|^3)
$$
for $|z|^2 - |z_n|^2 < 1$.
This is almost diagonal.
It gets us the curvature tensor in a neighborhood of $0$.

We have to try to do like Hitchin does to invert this.
Set $z' = (z_1,\ldots,z_{n-1},0)$. Let
$$
P(z) = z - \frac{\< z, \ov{z'} \>}{1 + |z'|^2} z'.
$$
Then
\begin{align*}
P^2(z)
&= z - \frac{\< z, \ov{z'} \>}{1 + |z'|^2} z'
- \frac{\< z - \frac{\< z, \ov{z'} \>}{1 + |z'|^2} z', \ov{z'} \>}{1 + |z'|^2} z'
\\
&= z - \frac{\< z, \ov{z'} \>}{1 + |z'|^2} z'
- \frac{\< z, \ov{z'} \>}{1 + |z'|^2} z'
+ \frac{\< z, \ov{z'} \>}{1 + |z'|^2} \frac{\< z', \ov{z'} \>}{1 + |z'|^2} z'
\\
&= z - 2 \frac{\< z, \ov{z'} \>}{1 + |z'|^2} z'
+ \frac{\< z, \ov{z'} \>}{1 + |z'|^2} \frac{|z'|^2}{1 + |z'|^2} z'
= P(z)
- \frac{\< z, \ov{z'} \>}{(1 + |z'|^2)^2} z'
\end{align*}
and $\ov{P^t} = P$.
So $P$ is not quite a projection, but it is invertible in places because
$$
v = \frac{(1 + |z'|^2)^2}{\< v, \ov{z'} \>}
(Pv - P^2 v)
$$
for $v$ such that $v_n = 0$ and $\<v, \ov{z'}\> \not= 0$.
We have
$
b(\xi, \ov\eta) = \< P_z \xi, \ov{P_z \eta} \> / (1+|z'|^2)
$
so the matrix of $h_t$ in our frame is
$$
H_t = A + \frac{e^{-t}}{1 + |z'|^2} P_z
$$
where
$$
A = \begin{pmatrix}
a_{11} & 0  & \cdots & z_1 \ov z_n
\\
0 & a_{22} & \cdots & z_2 \ov z_n
\\
\vdots & \ddots & \cdots & \vdots
\\
z_n \ov z_1 & \cdots & z_n \ov z_{n-1} & a_{nn}
\end{pmatrix},
$$
where $a_{jj} = |z_n|^2$ and
$a_{nn} = 1 + |z|^2 - |z_n|^2 - \sum_{lm} R_{nnlm} z_l \ov z_m$.
Can we invert $A$ (when that's possible)?
After one row operation we get
$$
A_1 =
\left(
\begin{array}{cccc|cccc}
a_{11} & 0 & \cdots & z_1 \ov z_n
&
1 & 0 & \cdots & 0
\\
0 & a_{22} & \cdots & z_2 \ov z_n
&
0 & 1 & \cdots & 0
\\
\vdots & \ddots & \cdots & \vdots
\\
0 & \cdots & z_{n} \ov z_{n-1} & a_{nn} - \frac{|z_1 z_n|^2}{a_{11}}
&
- \frac{z_n \ov z_1}{a_{11}} & 0 & \cdots & 1
\end{array}
\right)
$$
so after $n-1$ row operations we have
$$
A' =
\left(
\begin{array}{cccc|cccc}
a_{11} & 0 & \cdots & z_1 \ov z_n
&
1 & 0 & \cdots & 0
\\
0 & a_{22} & \cdots & z_2 \ov z_n
&
0 & 1 & \cdots & 0
\\
\vdots & \ddots & \cdots & \vdots
\\
0 & \cdots & 0 & a_{nn} - |z_n|^2 \sum_{j=1}^{n-1} \frac{|z_j|^2}{a_{jj}}
&
-\frac{z_n \ov z_1}{a_{11}} & \cdots & -\frac{z_{n} \ov z_{n-1}}{a_{n-1,n-1}} & 1
\end{array}
\right).
$$
Let's set $b = a_{nn} - |z_n|^2 \sum_{j=1}^{n-1} \frac{|z_j|^2}{a_{jj}}$.
Continuing on we get
$$
A'' =
\left(
\begin{array}{cccc|cccc}
a_{11} &  & &
&
1 + \frac{|z_1z_n|^2}{a_{11} b} &
\frac{|z_n|^2 z_2 \ov z_1}{a_{22} b} &
\cdots &
- \frac{z_1 \ov z_n}{b}
\\
 & a_{22} & &
&
\frac{|z_n|^2 z_2 \ov z_1}{a_{11}b} &
1 + \frac{|z_n|^2|z_2|^2}{a_{22} b} &
\cdots &
-\frac{z_2 \ov z_n}{b}
\\
  & & \ddots & &
\\
 & &  & b
&
-\frac{z_n \ov z_1}{a_{11}} &
\cdots &
-\frac{z_{n} \ov z_{n-1}}{a_{n-1,n-1}} &
1
\end{array}
\right).
$$
Finally
$$
A^{-1} =
\left(
\begin{array}{cccc}
\frac{1}{a_{11}} + \frac{|z_1z_n|^2}{a_{11}^2 b} &
\frac{|z_n|^2 z_1 \ov z_2}{a_{11} a_{22} b} & \cdots &
- \frac{z_1 \ov z_n}{a_{11} b}
\\
\frac{|z_n|^2 z_2 \ov z_1}{a_{11} a_{22} b} &
\frac{1}{a_{22}} + \frac{|z_n|^2|z_2|^2}{a_{22}^2 b} & \cdots &
-\frac{z_2 \ov z_n}{a_{22} b}
\\
& & \ddots &
\\
-\frac{z_n \ov z_1}{a_{11}b} & \cdots &
-\frac{z_{n} \ov z_{n-1}}{a_{n-1,n-1}b} &
\frac 1b
\end{array}
\right).
$$
Thankfully we should only need to write this as $I + B z + O(|z^2|)$
if we're going to calculate $R_h = R_0 + O(|z|)$.
We have $a_{jj} = e^{-t} + O(|z|^2)$ for $j < n$ and $a_{nn} = 1 + O(|z|^2)$,
so $b = 1 + e^{t}|z|^2 + O(|z|^3)$.
Let's write
$q(z) = |z|^2 - |z_n|^2$,
$p_j(z) = 1 - q(z) - |z_j|^2$
and $p_n(z) = 1 + q(z)$.
Then
\begin{align*}
H_t
&= \begin{pmatrix}
e^{-t} p_1(z) + |z_n|^2 &
0 &
\cdots &
z_1 \ov z_n
\\
0 &
e^{-t} p_2(z) + |z_n|^2 &
\cdots &
z_2 \ov z_n
\\
\vdots & \cdots & \ddots & \vdots
\\
\ov z_1 z_n &
\cdots &
\ov z_{n-1} z_{n} &
p_n(z) - \sum\limits_{lm} R_{nnlm} z_l \ov z_m
\end{pmatrix}
\\
H_t^{-1} &=
\begin{pmatrix}
e^t & & &
\\
& e^t & &
\\
& & \ddots &
\\
& & & 1
\end{pmatrix}
+ O(|z|^2)
=: C + O(|z|^2).
\end{align*}
Can write
$$
\displaylines{
H_t =
A + e^{-t} B
=
\begin{pmatrix}
|z_n|^2 & 0 & \cdots & z_n \ov z_1
\\
0 & |z_n|^2 & \cdots & z_n \ov z_2
\\
& & \ddots &
\\
\ov z_n z_1 & \cdots & \ov z_n z_{n-1} &
1 + |z|^2 - |z_n|^2 - \sum\limits_{lm} R_{nnlm} z_l \ov z_m
\end{pmatrix}
\hfill\cr\hfill{}
+ e^{-t}
\begin{pmatrix}
p_1 & & &
\\
& p_2 & &
\\
& & \ddots &
\\
& & & 0
\end{pmatrix}.
}
$$
Then
\begin{align*}
\Theta_t
&= H^{-1} \partial H \wedge H^{-1} \bar\partial H
- H^{-1} \partial \bar\partial H
\\
&= C(\partial A + e^{-t} \partial B) \wedge
C (\bar\partial A + e^{-t} \bar\partial B)
- C (\partial\bar\partial A + e^{-t} \partial\bar\partial B)
\\
&= C\partial A \wedge C \bar\partial A
- C \partial\bar\partial A
+ e^{-t}(
C \partial A \wedge C \bar\partial B
+ C \partial B \wedge C \bar\partial A
)
\\
&\qquad
+ e^{-2t} C\partial B \wedge C \bar\partial B
- e^{-t} C\partial\bar\partial B
\\
&= C\partial A \wedge C \bar\partial A
- C \partial\bar\partial A
+ C \partial A \wedge \bar\partial B
+ \partial B \wedge C \bar\partial A
\\
&\qquad
+ \partial B \wedge \bar\partial B
- \partial\bar\partial B.
\end{align*}
The form $e^t C \partial B$ is fixed and $O(|z|)$, so we can control it near $0$.
However we need to look at $C \partial A \wedge C \bar\partial A$.
We also have
$$
- C \partial\bar\partial A - e^{-t} C \partial\bar\partial B
= - C \partial\bar\partial A - \partial\bar\partial B
$$
because $e^{-t} C B = B$.
Have
$$
\partial\bar\partial A
=
\begin{pmatrix}
dz_n d\bar z_n & 0 & \cdots & dz_n d\bar z_1
\\
0 & dz_n d\bar z_n & \cdots & dz_n d\bar z_2
\\
& & \ddots &
\\
dz_1 d\bar z_n & \cdots & d z_{n-1} d\bar z_n &
dzd\bar z - dz_n d\bar z_n - \sum\limits_{lm} R_{nnlm} d z_l d\bar z_m
\end{pmatrix}
$$
so for $\xi = \sum_j a_j \partial/\partial z_j$ we have
$$
\tfrac i2 \partial\bar\partial A(\xi,\ov\xi,\xi,\ov\xi)
=
\ov\xi^t \begin{pmatrix}
|a_n|^2 & 0 & \cdots & a_n \bar a_1
\\
0 & |a_n|^2 & \cdots & a_n \bar a_2
\\
& & \ddots &
\\
a_1 \bar a_n & \cdots & a_{n-1} \bar a_n &
|a|^2 - |a_n|^2 - \sum\limits_{lm} R_{nnlm} a_l \bar a_m
\end{pmatrix}
\xi.
$$
We forgot to multiply by $C$. Adding that we get
$e^{t}$ terms everywhere except in the bottom row
(where we should get $1/b$; have to check that better).
Denote the matrix by $D$.
Then
$$
D \xi
= \begin{pmatrix}
2 e^t |a_n|^2 a_1
\\
2 e^t |a_n|^2 a_2
\\
\vdots
\\
a_n \Bigl(
2(|a|^2 - |a_n|^2)
- \sum_{lm} R_{nnlm} a_l \bar a_m
\Bigr)
\end{pmatrix}
$$
because your matrix should be transposed,
so
$$
\tfrac i2 C \partial\bar\partial A(\xi,\ov\xi,\xi,\ov\xi)
= |a_n|^2\Bigl(
2(1+e^{t})(|a|^2 - |a_n|^2)
- \sum_{lm} R_{nnlm} a_l \bar a_m
\Bigr).
$$
Still need to calculate the contributions of
$\partial B \wedge \bar\partial B - \partial\bar\partial B$
and $C\partial A \wedge C \bar\partial A$.

I think we need to pay a little more attention to the exponential terms to
be able to argue they behave as we want in the limit.
We have
$$
a_j = |z_n|^2 + e^{-t} f_j(z),
$$
where
$$
f_j(z) = \frac{1}{1 + |z|^2 - |z_n|^2} - \frac{|z_j|^2}{(1+|z|^2-|z_n|^2)^2}.
$$
Note that $a_j(0) = e^{-t} > 0$ for all $t$.
We set $b = a_{nn} - |z_n|^2 \sum_{j=1}^{n-1} |z_j|^2/a_{jj}$.
I don't think that simplifies much.

Try inverting
$$
B = \begin{pmatrix}
g_1 & \frac{z_2 \bar z_1}{1+|z'|^2} & \cdots & 0
\\
\frac{z_1 \bar z_2}{1+|z'|^2} & g_2 & \cdots & 0
\\
\vdots & \ddots & \cdots & \vdots
\\
0 & \cdots & 0 & e^t(1+|z'|^2)
\end{pmatrix}
$$
where $g_j(z) = 1 - \frac{|z_j|^2}{1+|z'|^2}$
near $0$.
(We take the constant $1$ from the matrix of $h$ and add it to the other.)
Write $B$ as
$$
\def\foo#1{#1}
\begin{pmatrix}
1 & & &
\\
& 1 & &
\\
& & \ddots &
\\
& & & e^t(1+|z'|^2)
\end{pmatrix}
-
\frac{1}{1+|z'|^2}
\begin{pmatrix}
\foo{|z_1|^2} & \foo{z_2 \bar z_1} & \cdots & 0
\\
\foo{z_1 \bar z_2} & \foo{|z_2|^2} & \cdots & 0
\\
\vdots & \cdots & \ddots & \vdots
\\
0 & 0 & 0 & 0
\end{pmatrix}
=: J -
\frac{C}{1+|z'|^2} .
$$
For $|z| < 1$ we can invert this as
$$
B^{-1}
= \sum_{k \geq 0} \frac{1}{(1+|z'|^2)^k} (J^{-1} C)^k J^{-1}
= \sum_{k \geq 0} \frac{1}{(1+|z'|^2)^k} C^k J^{-1}.
$$
To the second order
$$
\frac{1}{1+|z'|^2} = 1 - |z'|^2
$$
so if we want we can approximate $B^{-1} = J^{-1} + C$.
Let's set
$$
A =
\begin{pmatrix}
|z_n|^2 & 0 & \cdots & \bar z_1 z_n
\\
0 & |z_n|^2 & \cdots & \bar z_2 z_n
\\
\vdots & \cdots & \ddots & \vdots
\\
z_1 \bar z_n & z_2 \bar z_n & \cdots & |z'|^2
- \sum\limits_{lm} R_{nnlm} z_l \bar z_m
\end{pmatrix}
$$
so $H_t = A + \frac{e^{-t}}{1+|z'|^2} B$.
We want to invert this near $0$.
Let's write
$$
H_t = \frac{e^{-t}}{1+|z'|^2} B (I + e^t(1+|z'|^2) B^{-1}A)
$$
and get for $z$ in some ball that depends on $t$
$$
H_t^{-1}
= e^t(1+|z'|^2)
\sum_{k \geq 0} (-1)^k e^{tk} (1+|z'|^2)^k (B^{-1}A)^k B^{-1}.
$$
But as $t\to\infty$ this ball gets smaller and the limit is just the origin.
Moving the exponential factor around doesn't solve the problem because
we'll still have $e^t B^{-1}A$.
Maybe don't try to write the series?
Set $H_t = \frac{e^{-t}}{1+|z'|^2} B K_t$
with $K_t = I + e^t(1+|z'|^2) B^{-1}A$
and get $H_t^{-1} = e^t(1+|z'|^2) K_t^{-1} B^{-1}$.
The inverse of $J$ involves an $e^{-t}$ term which might help with $K_t$:
$$
K_t = I + e^t(1+|z'|^2) B^{-1}A
= I + e^t(1+|z'|^2) \sum_{k \geq 0} \frac{1}{(1+|z'|^2)^k} C^k J^{-1}.
$$
But no, that term only affects the bottom row of $C^k$; the others pick up an
$e^t$ term.
Try to calculate through and see if everything ends up OK?



\subsection*{Surfaces}

Set $n = 2$ and get
$$
H_t =
\begin{pmatrix}
|w|^2 h_{11} + e^{-t} f(z)&
w(\bar z h_{11} + h_{12})
\\
\bar w(z h_{11} + h_{21})
&
|z|^2 h_{11} + z h_{12} + \bar z h_{21} + h_{22}
\end{pmatrix},
$$
where $f(z) = \frac{1}{1+|z|^2} - \frac{|z|^2}{(1+|z|^2)^2}$.
Then
\begin{align*}
\det H_t
&=
(|w|^2 h_{11} + e^{-t} f(z))
(|z|^2 h_{11} + z h_{12} + \bar z h_{21} + h_{22})
\\
&\qquad
- w(\bar z h_{11} + h_{12})
\bar w(z h_{11} + h_{21})
\\
&=
|w|^2 h_{11}
(z h_{12} + \bar z h_{21} + h_{22})
+ e^{-t} f(z)
(|z|^2 h_{11} + z h_{12} + \bar z h_{21} + h_{22})
\\
&\qquad
- |w|^2
(
\bar z h_{11} h_{21}
+ z h_{11} h_{12}
+ h_{12} h_{21}
)
\\
&=
|w|^2 \det H
+ e^{-t} f(z)
(|z|^2 h_{11} + z h_{12} + \bar z h_{21} + h_{22})
\end{align*}
so
\begin{align*}
H_t^{-1}
&= \frac{1}{\det H_t}
\begin{pmatrix}
|z|^2 h_{11} + z h_{12} + \bar z h_{21} + h_{22}
&
- w(\bar z h_{11} + h_{12})
\\
- \bar w(z h_{11} + h_{21})
&
|w|^2 h_{11} + e^{-t} f(z)
\end{pmatrix}
\\
&= \frac{1}{\det H_t} (\operatorname{Adj} H' + e^{-t} \operatorname{Adj} D).
\end{align*}
We should be able to write $H_t = H' + e^{-t} D = \ov M^t H M + e^{-t} D$.
Then
$$
\displaylines{
H_t^{-1} \partial H_t \wedge H_t^{-1} \bar\partial H_t
- H_t^{-1} \partial\bar\partial H_t
\hfill\cr\qquad{}
= \frac{1}{\det H_t}
(\Adj H' + e^{-t} \Adj D)
(\p H' + e^{-t} \p D)
\hfill\cr\hfill{}
\wedge
\frac{1}{\det H_t}
(\Adj H' + e^{-t} \Adj D)
(\bp H' + e^{-t} \bp D)
\cr\hfill{}
- \frac{1}{\det H_t}
(\Adj H' + e^{-t} \Adj D)
(\p\bp H' + e^{-t} \p\bp D)
\cr\qquad{}
= \frac{1}{\det H_t}
(e^{-t} \Adj H' \p D + \Adj H' \p H' + e^{-t} \Adj D \p H')
\hfill\cr\hfill{}
\wedge
\frac{1}{\det H_t}
(e^{-t} \Adj H' \bp D + \Adj H' \bp H' + e^{-t} \Adj D \bp H')
\cr\hfill{}
- \frac{1}{\det H_t}
(\Adj H' \p\bp H' + \Adj H' \p\bp D + e^{-t} \Adj D \p\bp H')
}
$$
because $\operatorname{Adj} D D = 0$.
We have $\det H_t = \det H(|w|^2 + \text{stuff})$
and
$$
\Adj H' = \Adj M \Adj H \Adj \ov M^t.
$$
Then
\begin{align*}
\Adj \ov M^t
\p D
&= \begin{pmatrix}
e^{-t} \p f & 0
\\
0 & 0
\end{pmatrix}
\\
\Adj H \Adj \ov M^t \p D
&=
\begin{pmatrix}
(|z|^2 h_{11} + z h_{12} + \bar z h_{21} + h_{22})
e^{-t} \p f & 0
\\
0 & 0
\end{pmatrix}
\\
\Adj H' \p D
&=
\begin{pmatrix}
(|z|^2 h_{11} + z h_{12} + \bar z h_{21} + h_{22})
e^{-t} \p f & 0
\\
0 & 0
\end{pmatrix}.
\end{align*}
I'm not sure this was any quicker than just calculating directly.
For small enough $z,w$ the function
$g := |z|^2 h_{11} + z h_{12} + \bar z h_{21} + h_{22} \not= 0$
because $h_{22} \not= 0$ so we get
\begin{align*}
\frac{1}{\det H_t} \Adj H' \p D
&= \frac{1}{ge^{-t}f (1 + |w|^2 \det H/(gfe^t))} \begin{pmatrix}
g e^{-t} \p f & 0
\\
0 & 0
\end{pmatrix}
\\
&= \frac{1}{(1 + |w|^2 \det H/(gfe^t))} \begin{pmatrix}
\p f / f & 0
\\
0 & 0
\end{pmatrix}.
\end{align*}
Similarly we get
$$
\frac{1}{\det H_t} \Adj H' \bp D
= \frac{1}{(1 + |w|^2 \det H/(gfe^t))} \begin{pmatrix}
\bp f / f & 0
\\
0 & 0
\end{pmatrix}.
$$

We have
$$
H_t = \ov M^t H M
+ \begin{pmatrix} e^{-t} f(z) \\ 0 \end{pmatrix}
\begin{pmatrix} 1 & 0 \end{pmatrix}
=: \ov M^t H M + u v^t
$$
with $M = \begin{pmatrix} w & 0 \\ z & 1 \end{pmatrix}$
so the matrix determinant lemma says that
\begin{align*}
\det H_t
&= \det(\ov M^t H M)
+ v^t \operatorname{Adj} (\ov M^t H M) u
\\
&= |w|^2 \det H
+ e^{-t} f(z)
(|z|^2 h_{11} + z h_{12} + \bar z h_{21} + h_{22}).
\end{align*}


\subsection*{Decreasing Wu}

Maybe try Wu and show that what we get is strictly decreasing in $t$ for any
small $z$ and tends to $0$?

If $\xi = (a\ b)$ then
$$
|
\xi
|^2_t
= |M \xi|^2 + e^{-t} |a|^2 f(z)
$$
so
$$
\bar\partial_{\bar \xi} \log |\xi|^2_t
= \frac{\< M\xi, \ov{D_{\xi}(M\xi)}\>}{|\xi|^2_t}
+ e^{-t} \frac{a \ov{\partial_\xi a} f + |a|^2 \bar\partial_{\bar\xi} f}{|\xi|^2_t}
$$
and
\begin{align*}
-\frac{1}{|\xi|^2_t}
\partial_\xi \bar\partial_{\bar \xi} \log |\xi|^2_t
&=
\frac{R(\xi, \ov\xi, M\xi, \ov{M\xi})}{|\xi|^4_t}
\\
&\qquad
- \frac{\< D_\xi(M\xi), \ov{D_{\xi}(M\xi)}\>}{|\xi|^4_t}
+ \frac{\partial_\xi |\xi|^2_t}{|\xi|^2_t}
\frac{\< M\xi, \ov{D_{\xi}(M\xi)}\>}{|\xi|^4_t}
\\
&\qquad
- e^{-t} \frac{|\partial_\xi a|^2 f}{|\xi|^4_t}
+ e^{-t} \frac{a \ov{\partial_\xi a} f}{|\xi|^4_t}
\frac{\partial_\xi |\xi|^2_t}{|\xi|^2_t}
\\
&\qquad
- e^{-t} \frac{\partial_\xi a \, \bar a \bar\partial_{\bar\xi} f}{|\xi|^4_t}
- e^{-t} \frac{|a|^2 \partial_\xi \bar\partial_{\bar\xi} f}{|\xi|^4_t}
+ e^{-t} \frac{|a|^2 \bar\partial_{\bar\xi} f}{|\xi|^4_t}
\frac{\partial_\xi |\xi|^2_t}{|\xi|^2_t}
\\
&=
\frac{R(\xi, \ov\xi, M\xi, \ov{M\xi})}{|\xi|^4_t}
\\
&\qquad
- \frac{\< D_\xi(M\xi), \ov{D_{\xi}(M\xi)}\>}{|\xi|^4_t}
+ \frac{\< D_\xi(M\xi), \ov{M\xi}\>}{|\xi|^4_t}
\frac{\< M\xi, \ov{D_{\xi}(M\xi)}\>}{|\xi|^2_t}
\\
&\qquad
+
%\frac{\partial_\xi |\xi|^2_t}{|\xi|^2_t}
e^{-t} \frac{\p_\xi a \, \bar a f}{|\xi|_t^2}
\frac{\< M\xi, \ov{D_{\xi}(M\xi)}\>}{|\xi|^4_t}
+ e^{-t} \frac{|a|^2 \p_\xi f}{|\xi|_t^2}
\frac{\< M\xi, \ov{D_{\xi}(M\xi)}\>}{|\xi|^4_t}
\\
&\qquad
- e^{-t} \frac{|\partial_\xi a|^2 f}{|\xi|^4_t}
+ e^{-t} \frac{a \ov{\partial_\xi a} f}{|\xi|^4_t}
\frac{\partial_\xi |\xi|^2_t}{|\xi|^2_t}
\\
&\qquad
- e^{-t} \frac{\partial_\xi a \, \bar a \bar\partial_{\bar\xi} f}{|\xi|^4_t}
- e^{-t} \frac{|a|^2 \partial_\xi \bar\partial_{\bar\xi} f}{|\xi|^4_t}
+ e^{-t} \frac{|a|^2 \bar\partial_{\bar\xi} f}{|\xi|^4_t}
\frac{\partial_\xi |\xi|^2_t}{|\xi|^2_t}
\end{align*}
so I think we get
$$
\displaylines{
-\frac{1}{|\xi|^2_t}
\partial_\xi \bar\partial_{\bar \xi} \log |\xi|^2_t
= \frac{R(\xi, \ov\xi, M\xi, \ov{M\xi})}{|\xi|^4_t}
\hfill\cr\hfill{}
- \frac{1}{|\xi|^4_t}
\biggl(
|D_\xi(M\xi)|^2
- \frac{|\< D_\xi(M\xi), \ov{M\xi}\>|^2}{|\xi|^2_t}
\biggr)
+ e^{-t} K
}
$$
over some compact set around $0$ that only depends on the local coordinates.
If $M\xi \not= 0$ we have
\begin{align*}
q_t(\xi)
:= |D_\xi(M\xi)|^2
- \frac{|\< D_\xi(M\xi), \ov{M\xi}\>|^2}{|\xi|^2_t}
&\geq
|D_\xi(M\xi)|^2
- \frac{|D_\xi(M\xi)|^2 |M\xi|^2}{|\xi|^2_t}
\\
&=
|D_\xi(M\xi)|^2
- \frac{|D_\xi(M\xi)|^2}{1 + e^{-t}|a|^2f/|M\xi|^2}
\\
&= |D_\xi(M\xi)|^2 \frac{|a|^2f}{e^t |M\xi|^2 + |a|^2f}
\geq 0
\end{align*}
for all $t$, and if $M\xi = 0$ at a point we just get $q_t(\xi) = |D_\xi(M\xi)|^2
\geq 0$. Note that $M$ is invertible for $w \not= 0$, but we can't control
$q_t(\xi)$ uniformely as $w \to 0$.
This doesn't doom us because we want to take sup of the LHS above, but it's
not great.



\subsection*{A surface, local description}

Have $\blX = \{(z_1,z_2,z_3) \mid z_1 - z_2 z_3 = 0\}$.
The tangent space of $\blX$ is spanned by
$$
\xi_1 = z_3 e_1 + e_2,
\quad
\xi_2 = z_2 e_1 + e_3.
$$
The FS metric is
$$
g(z_3)
= \frac{\xi_3 \ov\eta_3}{1 + |z_3|^2}
- \frac{\xi_3 \ov \eta_3 |z_3|^2}{(1 + |z_3|^2)^2}
= \frac{\xi_3 \ov\eta_3}{(1 + |z_3|^2)^2}.
$$
Its Chern connection is
$$
D_\xi \eta_3
= d_\xi \eta_3 - \frac{\xi_3 \ov z_3}{1 + |z_3|^2} \eta_3
- \frac{\eta_3 \ov z_3}{1 + |z_3|^2} \xi_3.
$$
Let $D_h = d + A$.
The second fundamental form is
$$
\sigma(\xi) \eta
= \pi(D_{h,\xi} \eta \oplus D_{g,\xi} \eta),
$$
where $\pi$ is the orthogonal projection onto the normal bundle.
The tangent bundle of $\blX$ is $\ker \lambda$, where
$\lambda = dz_1 - z_3 dz_2 - z_2 dz_3$.
We note that $\xi_3 = e_1$ is never in $\ker \lambda$.
The orthogonal projection onto the complementary bundle is then
$$
\xi \mapsto \frac{\<\xi, e_1\>_t}{|e_1|^2_t} e_1.
$$
Setting $\xi = a\xi_1 + b\xi_2$ the nonzero contribution comes from
$$
D_{h} \xi
= D ( (a z_3 + b z_2) e_1 + a e_2)
= d(a z_3 + b z_2) e_1 + (a z_3 + b z_2) D e_1 + a D e_2.
$$

The nonzero contribution is
$$
\sigma(\xi_2, \xi_2)
= \pi(D_{g,\xi_2} \xi_2)
= \pi(D_{g,e_3} e_3)
= - \frac{2 \ov z_3}{1 + |z_3|^2} e_3
$$
so
$$
|\sigma(\xi_2, \xi_2)|^2
= \frac{4 |z_3|^2}{(1+|z_3|^2)^4}.
$$
For $\xi = a \xi_1 + b \xi_2$ we get
$$
H_t(\xi)
= \frac{e^t R_h(\xi')}{|\xi|_t^4}
+ \frac{2 |b|^4 |e_3|^4}{|\xi|^4_t}
- \frac{4 |b|^4 |z_3|^2}{(1+|z_3|^2)^4|\xi|_t^4}
$$
where $\xi' = (bz_2 + az_3) e_1 + be_2$.
Now,
$$
2|e_3|^4 - 4 \frac{|z_3|^2}{(1+|z_3|^2)^2}|e_3|^2
= 2|e_3|^2
\frac{1 - 2|z_3|^2}{(1+|z_3|^2)^2}
$$
is positive for all $|z_3| < 1/\sqrt 2$.
Because $h$ has positive \hsc{}, we find that $H_t > 0$ for all $t$ and
small enough $z$.



\subsection*{Yet another curvature calculation}

Let $u = (z, 1, 0)$ and $v = (y, 0, 1)$ and let $S = \kk C u + \kk C v \subset
\kk C^3$.
We have $S = \ker \alpha$, where $\alpha \in (\kk C^3)^*$, so
$$
0 \longrightarrow S \longrightarrow \kk C^3
\stackrel{\alpha}{\longrightarrow} \kk C \longrightarrow 0
$$
is exact.
This gives an isomorphism $\kk C^3 / S \cong \kk C$.
We have $\alpha = dx - z dy - y dz$, so the
quotient map is given by $Q = (1, -z, -y)$.

We have a Hermitian metric on $\kk C^3$ given by
$$
H = \begin{pmatrix}
	J & 0
	\\
	0 & e^{-t} \theta
\end{pmatrix},
$$
where $\theta(x,y,z) = 1 / (1+|z|^2)^2$.
The quotient metric is then the inverse of
\begin{align*}
Q H^{-1} \ov{Q^t}
&= \begin{pmatrix} 1 & -z & -y \end{pmatrix}
\begin{pmatrix}
J^{-1} & 0
\\
0 & e^t/\theta
\end{pmatrix}
\begin{pmatrix} 1 \\ -\bar z \\ -\bar y \end{pmatrix}
\\
&= \begin{pmatrix} 1 & -z & -y \end{pmatrix}
\begin{pmatrix}
h^{11} - \bar z h^{12}
\\
h^{21} - \bar z h^{22}
\\
-e^t \bar y/\theta
\end{pmatrix}
= h^{11} - \bar z h^{12} - z h^{21} + |z|^2 h^{22} + e^t |y|^2 \theta,
\end{align*}
or
$$
h_Q
= \frac{1}{h^{11} - \bar z h^{12} - z h^{21} + |z|^2 h^{22} + e^t |y|^2 \theta}.
$$
Note that $h^{11} = h_{22} / \det J$, $h^{22} = h_{11} / \det J$,
$h^{12} = -h_{12} / \det J$ and $h^{21} = -h_{21} / \det J$.
The metric on $S$ is
\begin{align*}
H_S
&=
\begin{pmatrix}
\bar z & 1 & 0
\\
\bar y & 0 & 1
\end{pmatrix}
\begin{pmatrix}
J & 0
\\
0 & e^{-t}\theta
\end{pmatrix}
\begin{pmatrix}
z & y
\\
1 & 0
\\
0 & 1
\end{pmatrix}
=
\begin{pmatrix}
\bar z & 1 & 0
\\
\bar y & 0 & 1
\end{pmatrix}
\begin{pmatrix}
z h_{11} + h_{12} & y h_{11}
\\
z h_{21} + h_{22} & y h_{21}
\\
0 & e^{-t} \theta
\end{pmatrix}
\\
&=
\begin{pmatrix}
|z|^2 h_{11} + \bar z h_{12} + z h_{21} + h_{22} &
\bar z y h_{11} + y h_{21}
\\
\bar y z h_{11} + \bar y h_{12} &
|y|^2 h_{11} + e^{-t} \theta
\end{pmatrix}.
\end{align*}
Take $\xi = a u + bv$, where $u = ze_x + e_y$ and $v = ye_x + e_z$.
Then
\begin{align*}
\sigma(\xi)\xi =
q(D_\xi\xi)
&= q(d_\xi a \, u + a D_\xi u + d_\xi b \, v + b D_\xi v)
\\
&= q(a D_\xi u + b D_\xi v)
\\
&= q(a d_\xi z \, e_x + az D_\xi  e_x + D_\xi  e_y + b d_\xi y \, e_x + by D_\xi  e_x + D_\xi  e_z)
\\
&= 2ab
+ \alpha( (az + by) D_\xi e_x + D_\xi e_y + D_\xi e_z).
\end{align*}
We can calculate this some more.
Need to write out the connection forms of $h_X$ and Fubini--Study.
The one of $h_X$ is
$$
J^{-1} \partial J
=
\begin{pmatrix}
h^{11} & h^{12}
\\
h^{21} & h^{22}
\end{pmatrix}
\begin{pmatrix}
\partial h_{11} & \partial h_{12}
\\
\partial h_{21} & \partial h_{22}
\end{pmatrix}
=
\begin{pmatrix}
h^{11} \partial h_{11} {+} h^{12} \partial h_{21} &
h^{11} \partial h_{12} {+} h^{12} \partial h_{22}
\\
h^{21} \partial h_{11} {+} h^{22} \partial h_{21} &
h^{21} \partial h_{12} {+} h^{22} \partial h_{22}
\end{pmatrix}
$$
so
\begin{align*}
\alpha((az+by) D_\xi e_x)
&= (az + by) (
h^{11} \partial_\xi h_{11} {+} h^{12} \partial_\xi h_{21}
- z  (h^{21} \partial_\xi h_{11} {+} h^{22} \partial_\xi h_{21} )
),
\\
\alpha(D_\xi e_y)
&=
h^{11} \partial_\xi h_{12} {+} h^{12} \partial_\xi h_{22}
- z (h^{21} \partial_\xi h_{12} {+} h^{22} \partial_\xi h_{22}).
\end{align*}
I think that what matters here is that
$$
\sigma(\xi)\xi
= p(a,b),
$$
where $p$ is some degree-two polynomial in $a,b$ (whose coefficients are smooth
functions of $x,y,z$ and $h_X$ and the Fubini--Study metric).
Then
$$
|\sigma(\xi)\xi|^2_Q
= \frac{|p(a,b)|^2}{h^{11} - \bar z h^{12} - z h^{21} + |z|^2 h^{22} + e^t |y|^2 \theta}
= |p(a,b)|^2 + O(e^{-t} |v|^2).
$$
At $0$ we have $\sigma(\xi)\xi = 2ab$.
This should let us extend the result on the exceptional divisor to a
neighborhood of it for all large $t$.



In general, if $Z = \{x \in X \mid f(x) = 0\}$ for a holomorphic $f : X \to \kk
C^m$, then $df : T_X \to f^* T_{\kk C^m}$ descends to an isomorphism $N_{Z/X} \cong
T_{\kk C^m}$.
Given a metric on $X$
and a section $\xi$ of $T_Z$ we have $df(\xi) = 0$ so
$$
0 = D(df(\xi))
= D(df)(\xi) + df(D\xi).
$$
For the second fundamental form we then have
$$
|\sigma(\xi)\eta|^2
= |df(D_\xi \eta)|^2
= |D_\xi(df)(\eta)|^2.
$$
Now $D(df)$ is a perfectly good, fixed element of $\Hom(T_Z^{\otimes 2},
N_{Z/X})$.
Then
$$
|D_\xi(df)(\eta)|^2
\leq |D(df)|^2 |\xi|^2 |\eta|^2
$$
for all $\xi$ and $\eta$ over some compact ball.
In our case, $D$ does not depend on $t$, but the norm of $D(df)$ does.
That norm is a function of the metric on $T_Z^{\otimes 2}$ and $N_{Z/X}$.
I think the former should be $O(e^{2t})$ and the latter converges to some fixed
Hermitian form as $t \to \infty$. Then
$$
|D(df)|^2
= \tr\bigl(
h_{T^{\otimes 2}}^{-1} \circ
\ov{(D(df))^*} \circ
h_{N} \circ D(df)
\bigr)
= O(e^{-2t})
$$
so we finally get
$$
\frac{|\sigma(\xi)\xi|^2}{|\xi|^4} = O(e^{-2t})
$$
which converges faster to $0$ than our other positive terms, and we win.

We don't really need the gymnastics at the beginning.
If $\sigma = q \circ D$ is the second fundamental form of any embedding we have
$$
\sigma(a \xi, b \eta)
= q(D_{a\xi}(b \eta))
= q(a d_\xi b \, \eta + ab D_\xi \eta)
= ab \sigma(\xi, \eta)
$$
so it is a tensor.

We use that $|f\xi| \leq |f||\xi|$, which is true for the $\ell_2$-operator norm,
and then we want to conclude by knowing things about the Hilbert--Schmidt norm.
But we have
$$
|f|_2 \leq |f|_{HS}
$$
so that's OK.

We have to prove our estimate for the quotient norm because what I wrote is true
for a sum of metrics on the same space, which is not what we have.
If
$$
0 \to S \to V \to Q \to 0
$$
is a short exact sequence and $h$ is an inner product on $V$, the inner product
on the quotient is
$$
h_Q = ((q^\vee)^{*} h^\vee)^\vee.
$$
If $H$ and $Q$ are the matrices of the inner product and quotient map we get
$$
H_Q = \ov{(\ov Q \ov{H^{-1}} Q^t)^{-1}}.
$$
Our $H = J \oplus e^{-t} K$ so $H^{-1} = J^{-1} \oplus e^{t} K^{-1}$.
Write $Q = Q_1 \oplus Q_2$ and get
$$
\ov{Q} \ov{H^{-1}} Q^t
=
\begin{pmatrix}
\ov{Q_1} & \ov{Q_2}
\end{pmatrix}
\begin{pmatrix}
\ov{J^{-1}} & 0
\\
0 & e^{t} \ov{K^{-1}}
\end{pmatrix}
\begin{pmatrix}
Q_1^t \\ Q_2^t
\end{pmatrix}
=
\ov{Q_1} \ov{J^{-1}} Q_1^t
+ e^{t} \ov{Q_2} \ov{K^{-1}} Q_2^t
$$
and so
$$
H_Q
=
(Q_1 J^{-1} \ov{Q_1^t}
+ e^{t} Q_2 K^{-1} \ov{Q_2^t})^{-1}.
$$
Say $V$ has dimension $n + m$ and $S$ has dimension $n$.
Then $Q$ is an $m \times (n+m)$ matrix that we write as a direct sum of
the $m \times n$ matrix $Q_1$ and the $m \times m$ matrix $Q_2$.
The dimension at least check out.
In our case we have
$$
Q_1 =
\begin{pmatrix}
1 & 0 & \cdots & -y_1
\\
0 & 1 & \cdots & -y_2
\\
  & & \ddots &
\\
0 & \cdots & 1 & -y_{n-1}
\end{pmatrix},
\quad
Q_2 = -x_n I_{n-1}.
$$
Then at least $Q_2 K^{-1} \ov{Q_2^t} = |x_n|^2 K^{-1}$.
For a matrix $A = (a_{jk})$ we have
$$
Q_1 A \ov{Q_1^t}
= (a_{jk} - y_j a_{nk} - \bar y_k a_{jn} + y_j \bar y_k a_{nn})_{1 \leq j,k \leq n-1}.
$$
That's not fun but in any case it is enough to see that $h_Q$ is $O(e^{-t})$.
Then the second fundamental form is only $O(e^{-t})$.

We have
$$
0 \to T_Z \to T_V \oplus T_{\kk P(V)} \to N_{Z/X} \to 0
$$
and $D_{\Hom} q(\xi) + q(D\xi) = 0$ for all sections
$\xi$ of $T_Z$.
We're going to use $h_t = h + e^{-t} b$.
Both $D$ and $D_{\Hom}$ are independent of $t$.
We get
\begin{align*}
R_t(\xi,\ov\xi,\xi,\ov\xi)
&= \pi_V^* R_h(\xi,\ov\xi,\xi,\ov\xi)
+ e^{-t} \pi_{\kk P(V)}^* R_{FS}(\xi,\ov\xi,\xi,\ov\xi)
- |D_{\Hom,\xi}q(\xi)|^2
\\
&\geq
m |\pi_V(\xi)|^4_h
+ 2 e^{-t} |\pi_{\kk P(V)}(\xi)|^4_{FS}
- |D_{\Hom}q|^2 |\xi|^4_t,
\end{align*}
where $0 < m \leq H_h$.
We can calculate that $|D_{\Hom}q|^2$ is $O(e^{-t})$
and have
$$
|\xi|^2_t
= |\pi_V(\xi)|^2_h + e^{-t} |\pi_{\kk P(V)}(\xi)|^2_{FS}
$$
so we get
$$
\displaylines{
R_t(\xi,\ov\xi,\xi,\ov\xi)
\geq
(m - |D_{\Hom}q|^2) |\pi_V(\xi)|^4_h
+ (2 - e^{-2t} |D_{\Hom}q|^2) |\pi_{\kk P(V)}(\xi)|^4_{FS}
\hfill\cr\hfill{}
- 2 e^{-t} |D_{\Hom}q|^2 |\pi_V(\xi)|^2_h  |\pi_{\kk P(V)}(\xi)|^2_{FS}.
}
$$
As $t \to \infty$ then $|D_{\Hom}q|^2 \to 0$ so the middle term is eventually
positive.
We have to figure out what $e^t | \, \cdot \, |^2_{\Hom}$ tends to as $t \to
\infty$.
We should have
\begin{align*}
H_{\Hom}
&= H_t^\vee \otimes H_t^\vee \otimes H_Q
\\
&= (J^\vee \oplus e^{t} K^\vee) \otimes
(J^\vee \oplus e^{t} K^\vee) \otimes
(Q_1 J^{-1} \ov{Q_1^t} + e^{t} Q_2 K^{-1} \ov{Q_2^t})^{-1}
\end{align*}
so when $t \to \infty$ this tends to
$$
\frac{1}{|x_n|^2} K^\vee \otimes K^\vee \otimes K
$$
which, once again, blows up as $x \to 0$.



\bibliographystyle{plainurl}
\bibliography{main}

\end{document}
