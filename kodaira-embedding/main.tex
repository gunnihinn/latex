\documentclass{article}

\usepackage[utf8]{inputenc}
\usepackage[T1]{fontenc}
\usepackage{amsmath}
\usepackage{amssymb}
\usepackage{amsthm}
\usepackage{lmodern}

\begin{document}

\title{The Kodaira embedding theorem}
\author{Gunnar \TH\'or Magn\'usson}

\maketitle

Let's read Kodaira's 1954 paper ``On K\"ahler varieties of restricted type (an intrinsic characterization of algebraic varieties)''. It's the paper that first proved the Kodaira embedding theorem. We'll go step by step and take notes as we go along.

The language and some of the methods can be updated with 70 years of hindsight. We'll do so.

A compact complex manifold $X$ is called a \emph{Hodge manifold} if $X$ carries a K\"ahler metric whose exterior form belongs to an integral cohomology class. Such a metric will be called a \emph{Hodge metric}. The main purpose of Kodaira's paper is to prove that every Hodge manifold is isomorphic to a projective manifold; that is, a smooth submanifold of a projective space.


\section{Preliminaries}

We talk about some basics of complex geometry; line bundles, transition functions, Chern classes. We recall the Kodaira vanishing theorem.

\section{Quadratic transformations}

A.K.A.~blow-ups.

It seems like Kodaira only has to blow up points. We gather some facts about their blow-ups. In particular, we pull back line bundles to the blow-up, and prove some lemmas about what happens to pull-backs of positive bundles.


\end{document}
