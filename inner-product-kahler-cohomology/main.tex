\documentclass[11pt]{article}

\usepackage{lmodern}
%\linespread{1.1}
\usepackage[utf8]{inputenc}
\usepackage[T1]{fontenc}

\usepackage[normalem]{ulem}
\usepackage{textcomp}
\usepackage{hyperref}
\usepackage{tikz-cd}

\usepackage{amsmath}
\usepackage{amssymb}
\usepackage{amsthm}

\newtheorem{theo}{Theorem}[section]
\newtheorem{prop}[theo]{Proposition}
\newtheorem*{lemm}{Lemma}
\newtheorem{coro}[theo]{Corollary}
\newtheorem*{coro*}{Corollary}
\theoremstyle{definition}
\newtheorem{defi}[theo]{Definition}
\newtheorem{exam}[theo]{Example}
\newtheorem{exer}[theo]{Exercise}
\newtheorem*{rema}{Remark}

\newcommand{\kk}[1]{\mathbb{#1}}
\newcommand{\cc}[1]{\mathcal{#1}}

\def\eps{\varepsilon}
\def\empty{\varnothing}

\def\^#1{^{[#1]}}
\def\ov#1{\overline{#1}}
\def\<{\langle}
\def\>{\rangle}
\def\bw#1{\bigwedge\!\!{}^{#1}\, T_X^*}

\def\CC{\mathbf{C}}
\def\EE{\mathcal{E}}
\def\FF{\mathcal{F}}
\def\NN{\mathbf{N}}
\def\RR{\mathbf{R}}
\def\QQ{\mathbf{Q}}
\def\ZZ{\mathbf{Z}}
\def\PP{\mathbf{P}}
\def\HH{\mathcal{H}}

\def\dV{\mathrm{d}\mkern-1mu V}

\DeclareMathOperator{\Span}{Span}
\DeclareMathOperator{\Hom}{Hom}
\DeclareMathOperator{\End}{End}
\DeclareMathOperator{\Aut}{Aut}
\DeclareMathOperator{\Ker}{Ker}
\DeclareMathOperator{\Img}{Im}
\DeclareMathOperator{\sgn}{sgn}
\DeclareMathOperator{\GL}{GL}
\DeclareMathOperator{\ord}{ord}
\DeclareMathOperator{\len}{len}
\DeclareMathOperator{\id}{id}

\author{Gunnar Þór Magnússon}
\date{\today}
\title{A primer on the inner product in the\\cohomology ring of compact K\"ahler manifolds}


\begin{document}

\maketitle

\begin{abstract}
We review the basic properties of the inner product on the cohomology of a compact K\"ahler manifold that the choice of a K\"ahler metric induces.
\end{abstract}


\section*{Introduction}

A friend of mine is a Riemannian geometer.
One day he came to me -- big guy, strong guy, tears in his eyes -- and asked a question that came down to some linear algebra in the cohomology ring of a compact K\"ahler manifold.
We figured out the question and he asked for a reference for the background.
As far as I know there is none; textbooks cover the basics but leave the connecting of the dots to PhD students.
This little note is for my friend.

If we're going to prove anything here, it is the following.
Let $X$ be a compact K\"ahler manifold equipped with a K\"ahler metric $\omega$.
The space of smooth differential forms on $X$ has the $L^2$ inner product
\[
\< u, \bar v \>
= \int_X \< u(x), \ov{v(x)} \> \, dV_\omega,
\]
where $\< u(x), \ov{v(x)} \>$ is the inner product $\omega$ induces on $\bigwedge^\bullet T_X^*$.
If $\HH^k(X, \CC)$ is the space of harmonic $k$-forms Hodge theory gives an isomorphism
\[
H^k(X,\CC)
\cong \HH^k(X,\CC) 
\subset \cc C^\infty\bigl(X, \bigwedge\!\!{}^k\, T_X^*\bigr).
\]
Thus we get an inner product on $H^k(X,\CC)$, defined by taking the $L^2$-inner product of the harmonic representatives of classes.
A priori it depends on the harmonic representatives and thus the metric chosen.

\begin{prop}
The inner product on $H^k(X,\CC)$ only depends on the cohomology class of $\omega$.
\end{prop}

It is also possible to define the inner product directly in the cohomology ring without using a K\"ahler metric, though actually proving that the result is positive definite requires a metric.
Sometimes the inner product can also be effectively calculated in the ring.




\section{The setting}

We will work on a compact K\"ahler manifold $X$ of complex dimension $n$, equipped with a K\"ahler metric $\omega$.


\subsection{Hodge theory}

The smooth closed differential forms on $X$ along with the exterior derivative form a resolution of the constant sheaf $\CC$, from which we get the de~Rham cohomology groups $H^k(X, \CC)$.
The metric on $X$ yields a Laplace operator, and we have an isomorphism
\[
H^k(X, \CC) \cong \HH^k(X, \CC),
\]
where $\HH^k(X, \CC)$ is the space of harmonic $k$-forms.

Each differential $k$-form can be written as a sum of $(p,q)$-forms.
The $\bar\partial$ operator acts on these, and they give a resolution of the sheaf $\Omega^p$ of sections of holomorphic $p$-forms on $X$.
These give the Dolbeault cohomology groups $H^{p,q}(X, \CC)$.

We can also define a Laplace operator for the $\bar\partial$ (and $\partial$) operators.
Since the metric is K\"ahler we have $\frac12 \Delta_d = \Delta_\partial = \Delta_{\bar\partial}$.
It follows that if we take a harmonic $k$-form and write it as a sum of $(p,q)$-forms with $p+q = k$, the components of the sum are harmonic as well.
This leads to the decomposition
\[
\HH^k(X, \CC) = \bigoplus_{p+q=k} \HH^{p,q}(X, \CC),
\]
and since we also have isomorphisms
\[
\HH^{p,q}(X,\CC) \cong H^{p,q}(X,\CC)
\]
we get a splitting 
\[
H^k(X, \CC) \cong \bigoplus_{p+q=k} H^{p,q}(X, \CC)
\]
and by conjugating harmonic $(p,q)$-forms we get isomorphisms
\[
H^{p,q}(X,\CC) \cong \ov{H^{q,p}(X,\CC)}.
\]
This splitting and isomorphisms are obtained by passing through harmonic forms and splitting those into their component parts, so a priori they depend on the metric chosen.
Luckily for us they are in fact independent of the metric; see \cite[Corollary~VI.8.7]{demailly-complex}.


\subsection{Operators of Hodge theory}


\paragraph{Lefschetz operator}

The K\"ahler metric $\omega$ defines the Lefschetz operator on differential forms by
\[
L : \cc C^\infty\bigl(X, \bw {p,q}\bigr) \to \cc C^\infty\bigl(X, \bw{p+1,q+1}\bigr),
\quad
u \mapsto u \wedge \omega.
\]
Since $d\omega = 0$ this operator descends to cohomology and gives an operator
\[
L_c : H^{p,q}(X,\CC) \to H^{p+1,q+1}(X,\CC),
\quad
[u] \mapsto [u] \cup [\omega].
\]
That operator clearly only depends on the cohomology class of the metric $\omega$.

A basic fact in K\"ahler geometry is that $L$ commutes with the Laplacian $\Delta$, so $L\Delta = \Delta L$.
Therefore $L$ sends harmonic forms to harmonic forms.
It follows that $L_c$ is the same operator as we get by going through harmonic forms.
In other words, the diagram
\[
\begin{tikzcd}
\HH^{p,q}(X,\CC) \ar[r,"L"] \ar[d] & \HH^{p+1,q+1}(X,\CC) \ar[d]
\\
H^{p,q}(X,\CC) \ar[r,"L_c"] & H^{p+1,q+1}(X,\CC)
\end{tikzcd}
\]
commutes.


\paragraph{Hodge star}

The metric also defines the Hodge star operator
\[
* : \cc C^\infty\bigl(X, \bw {p,q}\bigr) \to \cc C^\infty\bigl(X, \bw{n-q,n-p}\bigr)
\]
defined by requiring that
\[
u(x) \wedge *v(x)
= \< u(x), \ov{v(x)} \> \, \dV_\omega(x)
\]
for all smooth $(p,q)$-forms $u$.
It is again a basic fact that $* \Delta = \Delta *$, so $*$ sends harmonic forms to harmonic forms.
We can use this to define an operator $*_c$ in cohomology by completing the diagram
\[
\begin{tikzcd}
\HH^{p,q}(X,\CC) \ar[r,"*"] \ar[d] & \HH^{n-q,n-p}(X,\CC) \ar[d]
\\
H^{p,q}(X,\CC) \ar[r,"*_c"] & H^{n-q,n-p}(X,\CC).
\end{tikzcd}
\]


\paragraph{Primitive forms and classes}

One can check with linear algebra that the Lefschetz operator $L : \cc C^\infty(X, \bigwedge^{p,q}T_X^*) \to \cc C^\infty(X, \bigwedge^{p+1,q+1} T_X^*)$ is injective when $p+q < n$.
A little more linear algebra shows that
\[
L^{k} : \cc C^\infty\bigl(X, \bw {n-k}\bigr) \to \cc C^\infty\bigl(X, \bw{n+k}\bigr)
\]
is an isomorphism.
Pushing this one step futher, to $L^{k+1}$, then results in an operator that has a nontrivial kernel.
We define \emph{primitive forms} as this kernel,
\[
\cc C^\infty\bigl(X, \bw {n-k}\bigr)_p := \Ker\Bigl( 
L^{k+1} : \cc C^\infty\bigl(X, \bw {n-k}\bigr) \to \cc C^\infty\bigl(X, \bw{n+k+2}\bigr)
\Bigr).
\]
Induction on the degree $k$ shows that $k$-forms split into primitive $k$-forms and the images of lower-degree primitive forms under powers of $L$, or
\[
\cc C^\infty\bigl(X, \bw {k}\bigr)
= \bigoplus_{j} L^j \cc C^\infty\bigl(X, \bw {k-2j}\bigr)_p.
\]
As above we see that we can define primitive harmonic forms, and that if we decompose a harmonic form into its primitive parts, those parts are again harmonic.

Using harmonic forms we prove the hard Lefschetz theorem, which says that we have the same isomorphisms in cohomology:
\[
L_c^k : H^{n-k}(X, \CC) \to H^{n+k}(X,\CC).
\]
We can now define primitive classes $H^{n-k}_p$ as the kernel of $L_c^{k+1}$.
By the same induction process as before we prove that
\[
H^k(X,\CC) = \bigoplus_j L_c^j H^{k-2j}_p.
\]
Since the primitive decomposition of a harmonic form is harmonic, this decomposition is the same as the one we get through harmonic forms:
The diagram
\[
\begin{tikzcd}
\HH^k(X,\CC) \ar[r] \ar[d] &
\displaystyle\bigoplus_j L^j \HH^{k-2j}_p \ar[d]
\\
H^k(X,\CC) \ar[r] &
\displaystyle\bigoplus_j L_c^j H^{k-2j}_p
\end{tikzcd}
\]
commutes.


\paragraph{Hodge star and primitive forms}

In general it is difficult to calculate $*v$ explicitly for a given form.
If $v$ is a primitive $k$-form, however, there's a very useful formula by Andr\'e Weil that calculates it.

We adopt the notation that $L\^k := L^k / k!$.
It really helps.

\begin{prop}
Let $v$ be a primitive $(p,q)$-form and set $k = p+q$.
Then
\[
* L\^j v
= (-1)^{\binom{k+1}{2}} i^{p-q} L\^{n-k-j} \bar v.
\]
\end{prop}

We can now explain the proof of the main result of this note.

\begin{proof}
Let $[u]$ and $[v]$ be $(p,q)$-classes on $X$ and let $u$ and $v$ be their harmonic representatives.
By definition we have
\[
\< [u], \ov{[v]} \>
= \< u, \bar v \>
= \int_X \< u(x), \ov{v(x)} \> \, \dV
= \int_X u(x) \wedge * v(x).
\]
Write
\[
u = \sum_j L\^j u_{k-2j}
\quad\text{and}\quad
v = \sum_j L\^j v_{k-2j},
\]
where $u_{k-2j}$ and $v_{k-2j}$ are harmonic primitive $(p-j,q-j)$-forms.
Then
\[
*v = \sum_j (-1)^{\binom{k-2j}{2}} i^{p-q} L\^{n-k+2j} \ov{v_{k-2j}},
\]
so
\begin{align*}
\< [u], \ov{[v]} \>
&= \sum_{j,l} (-1)^{\binom{k-2l}{2}} i^{p-q} \int_X L\^j u_{k-2j} \wedge L\^{n-k+2l} \ov{v_{k-2j}}
\\
&= \sum_{j,l} (-1)^{\binom{k-2l}{2}} i^{p-q} \int_X L_c\^j [u_{k-2j}] \cup L_c\^{n-k+2l} [\ov{v_{k-2j}}]
\end{align*}
and the right-hand side only depends on the cohomology class of $\omega$.
\end{proof}


It's common to use $L^2$ metrics to construct Hermitian metrics on the base of deformations of a manifold.
For example, consider a family $\pi : \cc X \to S$ of manifolds with trivial canonical bundle, and suppose we have a smoothly varying family $\omega_s$ of Ricci-flat K\"ahler metrics on each manifold $X_s$.
Using a local nonzero holomorphic section $\sigma : S \to \cc R^0 \pi_* K_{\cc X / S}$ we get a morphism
\[
\begin{tikzcd}
T_S \ar[r,"\phi"] &
\cc R^1 \pi_* T_{\cc X/S} \ar[r,"\bullet \cup \sigma"] &
\cc R^1 \pi_* \Omega_{\cc X/S}^{n-1}
\end{tikzcd}
\]
that is an isomorphism when the family is effective.
The $L^2$ metrics induced by $\omega_s$ define a metric on the rightmost bundle, which turns out to only depend on the cohomology classes $[\omega_s]$.

Let $\pi : \cc X \to S$ be a family of manifolds with ample canonical bundle.
The Kodaira--Spencer morphism is
\[
\begin{tikzcd}
T_S \ar[r,"\phi"] &
\cc R^1 \pi_* T_{\cc X/S}.
\end{tikzcd}
\]
We have the sequence
\[
\begin{tikzcd}
0 \ar[r] & 
T_{\cc X/S} \ar[r] &
T_{\cc X} \ar[r] &
\pi^* T_S \ar[r] &
0
\end{tikzcd}
\]
and thus
\[
K_{\cc X} = K_{\cc X / S} \otimes \pi^* K_S.
\]
Also
\[
\begin{tikzcd}
0 \ar[r] & 
\cc R^0 \pi_* T_{\cc X/S} \ar[r] &
\cc R^0 \pi_* T_{\cc X} \ar[r] &
T_{S} \ar[r] &
\phantom{.}
\\
\ar[r] &
\cc R^1 \pi_* T_{\cc X/S} \ar[r] &
\cc R^1 \pi_* T_{\cc X} \ar[r] &
\cc R^1 \pi_* \cc O_{\cc X} \otimes T_{S} \ar[r] &
\phantom{.}
\end{tikzcd}
\]


We can look at the intersection form
\[
b(u,\bar v) = \int_X u \cup \bar v \cup \omega\^{n-k}
\]
on $(p,q)$-classes with $k = p+q$ as a Hermitian form over the open set of the complexified K\"ahler cone.
It is non-degenerate by the Hodge index theorem.
We arrange things so that $\omega = \Re \sigma$.
We have $\partial_\tau \omega = \frac12 \tau$, so
\[
\partial_\tau b(u, \bar v)(\sigma)
= \frac12 \int_X u \cup \bar v \cup \tau \cup \omega\^{n-k-1}
= b(D_\tau u, \bar v)(\sigma).
\]
Now
\begin{align*}
\bar\partial_\psi b(D_\tau u, \bar v)(\sigma)
&= \frac12 \int_X u \cup \bar\partial_\psi \bar v \cup \tau \cup \omega\^{n-k-1}
+ \frac14 \int_X u \cup \bar v \cup \tau \cup \bar\psi \omega\^{n-k-2}
\\
&= b(D_\tau u, \ov{\partial_\psi v})
+ b(F_{\tau\bar\psi} u, \bar v)
\end{align*}
so
\[
b(F_{\tau\bar\psi} u, \bar v)
= R(\tau, \bar\psi, u, \bar v)
= \frac14 \int_X u \cup \bar v \cup \tau \cup \bar\psi \cup \omega\^{n-k-2}
= \frac14 b(u \cup \tau, \ov{v \cup \psi}).
\]



\bibliographystyle{plain}
\bibliography{main}


\end{document}
