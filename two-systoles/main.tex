\documentclass{amsart}
\usepackage[utf8]{inputenc}

\usepackage{amsmath}
\usepackage{amssymb}
\usepackage{enumerate}
\usepackage{setspace}
\usepackage{caption}
\usepackage{mathrsfs}
\usepackage{hyperref}
\usepackage{esint}
\usepackage{amssymb}
\usepackage{hyperref}
\usepackage{comment}
\usepackage[normalem]{ulem}
\usepackage[all]{xy}

\usepackage{color}
\usepackage{lipsum}
\newcommand\blfootnote[1]{%
  \begingroup
  \renewcommand\thefootnote{}\footnote{#1}%
  \addtocounter{footnote}{-1}%
  \endgroup
}
\newtheorem{theorem}{Theorem}[section]
\newtheorem{lemma}[theorem]{Lemma}
\newtheorem{corollary}[theorem]{Corollary}
\newtheorem{proposition}[theorem]{Proposition}
\newtheorem{example}[theorem]{Example}
\newtheorem{conjecture}[theorem]{Conjecture}
\newtheorem{question}[theorem]{Question}
\numberwithin{equation}{section}

\theoremstyle {definition}
\newtheorem{definition}[theorem]{Definition}
\newtheorem{remark}[theorem]{Remark}

\DeclareMathOperator{\scal}{scal}

\DeclareMathOperator{\area}{area}
\DeclareMathOperator{\Aop}{A}
\DeclareMathOperator{\Hess}{Hess}
\DeclareMathOperator{\vol}{vol}
\DeclareMathOperator{\Ric}{Ric}
\DeclareMathOperator{\Rm}{Rm}
\DeclareMathOperator{\tr}{tr}
\DeclareMathOperator{\Div}{div}
\DeclareMathOperator{\Sec}{sec}
\DeclareMathOperator{\width}{width}
\DeclareMathOperator{\lip}{Lip}
\DeclareMathOperator{\dist}{dist}
\DeclareMathOperator{\pt}{pt}
\DeclareMathOperator{\inj}{inj}
\DeclareMathOperator{\sys}{sys}
\DeclareMathOperator{\stre}{str}
\DeclareMathOperator{\id}{id}

\title{2-systoles of some Kaehler manifolds with positive scalar curvature}
\author{Hugues Auvray, Thomas Richard}
\date{}

\begin{document}

\maketitle
\begin{abstract}
    Recent work on Riemannian manifolds with positive scalar curvature suggests that under appropriate topological assumptions a positive lower bound on the scalar curvature can lead to the existence of a topologically non trivial 2-sphere of small area.

    We show here how this can be shown for some Kaehler manifolds.
\end{abstract}

In recent years, the study of positive scalar curvature metrics has been revived with a new focus on quantitative results (in constrast with the topological results obtained in the 80's). For instance Bray-Brendle-Neves in dimension 3 and Zhu in dimension at most 7 have shown the following result:

\begin{theorem}
    Let $2\leq n\leq 7$ and let $(\mathbb{S}^2\times\mathbb{T}^{n-2},g)$ have $\scal_g\geq 2$. Then there exists a 2-sphere $\Sigma^2\in [\mathbb{S}^2\times\{\ast\}]$ whose area is at most $4\pi$. Moreover if any $2$-sphere in $[\mathbb{S}^2\times\{\ast\}]$ has area at least $4\pi$ then $g$ is the product of the round metric on $\mathbb{S}^2$ and a flat metric on $\mathbb{T}^{n-2}$.
\end{theorem}

The proof of this result uses stable minimal hypersurfaces. The theorem is actually slightly stronger as it applies to Riemannian manifolds $(M^n,g)$ with a non-zero degree map $M^n\to \mathbb{S}^2\times\mathbb{T}^{n-2}$. If we set the systole $\sys_h(g)$ of a homology class $h\in H_k(M)$ with $(M^n,g)$ a Riemannian manifold to be the infimum the area of smooth representatives of $h$, this suggest that lower bound on the scalar curvature can under appropriate topological assumptions provide an upper bound on the systole for some $h\in H_2(M)$.

It is at present very unclear how one can generalise the estimate above to other topological contexts: in particular it is currently unknown if a positive scalar curvature metric on $\mathbb{S}^2\times\mathbb{S}^2$ has an upper bound on the systole of one of its factors though something can be said under an extra metric hypothesis (see ...).

Here we show how restricting the investigation to Kaehler manifolds with positive scalar curvature allows to give answers to the question in various topological settings.

We will show for instance:

\begin{theorem}\label{thm_prod_proj}
    Let $X_0$ be a compact Kaehler $n_0$-manifold
    \marginpar{\footnotesize What's a $n_0$-manifold?}
    with $c_1(X_0)=0$ %and let $\tilde\omega_0$ be a Kaehler ricci flat metric on $X_0$.
    and $X=X_0\times\Pi_{k=1}^N\mathbb{CP}^{n_k}$. %and let $\tilde\omega=\tilde\omega_0+\sum_{k=1}^N \omega_{k}$ where $\omega_k$ is the Fubini-Study metric\footnote{normalised to have sectionnal curvature between $1/4$ and $1$.} on $\mathbb{CP}^{n_k}$.

    Let $\omega$ be any Kaehler metric on $X$ such that :
    \[\scal_\omega\geq \sum_{k=1}^N \frac{n_k(n_k+1)}{2}.\]
    Then there exists an homologically non trivial $\mathbb{CP}^1\subset X$ whose area is at most $4\pi$.

    Moreover, if any homologically non trivial $\mathbb{CP}^1\subset X$ has area at least $4\pi$ then $\omega$ is isometric to $\omega_0+\sum_{k=1}^N\omega_k$ for some Ricci flat Kaehler metric $\omega_0$ on $X_0$ and $\omega_k$ the Fubini-Study metric\footnote{normalised to have sectionnal curvature between $1/4$ and $1$.} on $\mathbb{CP}^{n_k}$.
\end{theorem}

Note that the constant $\sum_{k=1}^N \frac{n_k(n_k+1)}{2}$ is the scalar curvature of $\Pi_{k=1}^N\mathbb{CP}^{n_k}$ endowed with the product of the Fubini-Study metrics on each factor.
If one denotes by $C_k$ a projective line in the $\mathbb{CP}^{n_k}$ factor, this shows that $\min_k\sys_{[C_k]}(\omega)\leq 4\pi$ with equality if $\omega$ is a product of constant scalar curvature metrics on each of it factors. The statement can actually be made a bit more precise, during the proof we will get:
\[\sum_{k=1}^N\frac{n_k(n_k+1)}{\sys_{[C_k]}(\omega)}\geq \sum_{k=1}^N \frac{n_k(n_k+1)}{4\pi}\]

The key observation is that the integral of the scalar curvature and the systoles of complex submanifolds can be recovered from cohomological data for Kaehler manifolds.


\section{Systoles of Kaehler manifolds}

The following observation is classic and follows from Wirtinger's inequality and the Kaehler condition:
\begin{proposition}
    Les $(X,\omega)$ be a Kaehler manifold and $Y\subset X$ be a compact complex submanifold. Then for any smooth $\Sigma\in [Y]$, $\mathcal{A}(Y)\leq\mathcal{A}(\Sigma)$.
\end{proposition}
In particular this implies that $\sys_{[Y]}(\omega)=\mathcal{A}(Y)$ for any compact complex submanifold $Y\subset X$.


\section{Products of projective spaces (and manifolds with nonnpositive first Chern class)}

In this section we prove Theorem \ref{thm_prod_proj}.
Set $S=\sum_{k=1}^N n_k(n_k+1)$.

We first prove the inequality under a weaker hypothesis on the scalar curvature: a lower bound in average is enough to get the upper bound on one of the systole. We will prove the following:
\begin{proposition}
    \label{thm_prod_proj_avg}
    Let $X_0$ be a compact Kaehler $n_0$-manifold with $c_1(X_0)\leq 0$, %and let $\tilde\omega_0$ be a Kaehler ricci flat metric on $X_0$.
    Let $X=X_0\times\Pi_{k=1}^N\mathbb{CP}^{n_k}$ %and let $\tilde\omega=\tilde\omega_0+\sum_{k=1}^N \omega_{k}$ where $\omega_k$ is the Fubini-Study metric on $\mathbb{CP}^{n_k}$.

    Let $\omega$ be any Kaehler metric on $X$ such that:
    \[\int_X(\scal_\omega-S)\omega^n\geq 0.\]
    Then:
    \[\sum_{k=1}^N\frac{n_k(n_k+1)}{\sys_{[C_k]}(\omega)}\geq \sum_{k=1}^N \frac{n_k(n_k+1)}{4\pi}\]
    In particular $\min_k \sys_{[C_k]}(\omega) \leq 4\pi$.

    Moreover, if $\sys_{[C_1]}(\omega)=\dots= \sys_{[C_N]}(\omega)=4\pi$ then $c_1(X_0)=0$ and $\omega$ is in the Kaehler class of $\omega_0+\sum_{k=1}^N\omega_k$ for some Ricci flat Kaehler metric $\omega_0$ on $X_0$.
\end{proposition}

\begin{proof}
Let $\rho$ (resp. $\rho_i$) denote the Ricci form of $\omega$ (resp. $\omega_i$). We know that in $H_{\mathbb{R}}^{1,1}(X)=\bigoplus_k H^{1,1}_\mathbb{R}(X_k)$ we have:
\begin{align*}
    [\rho]&=2\pi c_1(X)\\
    &=2\pi \sum_{k=0}^N c_1(X_k)\\
    &=-[\kappa_0]+\frac{n_1+1}{2}[\omega_1]+\dots+\frac{n_N+1}{2}[\omega_N]
\end{align*}
where $\kappa_0$ is a non-negative (1,1)-form such that $[-\kappa_0]=2\pi c_1(X_0)$.

Let $\eta_0\in H^{1,1}(X_0)$ denote the component of $[\omega]$ in $H^{1,1}(X_0)$. We can then write the cohomology class of the Kaehler form $\omega$ in $H_{\mathbb{R}}^{1,1}(X)=\bigoplus_k H^{1,1}_\mathbb{R}(X_k)$ as  $[\omega]=\eta_0+\sum_{k=1}^Na_k[\omega_k]$ for some reals $a_k>0$. %Since $c_1(X_0)=0$,  Yau's solution to the Calabi conjecture gives us a Ricci flat metric $\omega_0\in\kappa_0$.
Let $n_0=\dim_\mathbb{C}X_0$ and $n=n_0+n_1+\dots+n_k$. We have:
\[\rho\wedge\omega^{n-1}=\frac{1}{2n}\scal_\omega\omega^n%\geq \frac{n}{2}\scal_{\tilde\omega}\omega^n.
\]

We integrate this inequality over $X$ and use that $\int_X\scal_\omega\geq\int_X\scal_{\tilde\omega}$
 to get :
\[\int_{X}\rho\wedge\omega^{n-1}=\frac{1}{2n}\int_X\scal_\omega\omega^n\geq\frac{1}{2n}\int_X S\omega^n =\frac{S}{2n}\int_{X}\omega^n.\]
%since $\scal_{\tilde\omega}$ is constant

On the one hand:
\begin{align*}
    \int_{X}\omega^{n}&=\int_{\tilde X}\left(\eta_0+\sum_{k=1}^Na_k\omega_k\right)^{n} \\
    &=\frac{n!}{n_0!n_1!\cdots n_N!}a_1^{n_1}\cdots a_N^{n_N}\int_{\tilde X}\eta_0^{n_0}\wedge\omega_1^{n_1}\wedge\dots\wedge\omega_N^{n_N}
\end{align*}
by expanding the $n-$th power and keeping only the multiples of the volume form.

On the other hand:
\begin{align*}
    \int_{X}&\rho\wedge\omega^{n-1}=
    \int_{X}\left(-\kappa_0+\frac{n_1+1}{2}\omega_1+\dots+\frac{n_N+1}{2}\omega_N\right)\wedge\left(\eta_0+\sum_{k=1}^Na_k\omega_k\right)^{n-1}\\
    \leq &\frac{1}{2}
    \begin{pmatrix}n-1 \\n_0\end{pmatrix}\int_{ X}\left((n_1+1)\omega_1+\dots+(n_N+1)\omega_N\right)\wedge\eta_0^{n_0}\wedge\left(\sum_{k=1}^Na_k\omega_k\right)^{n-n_0-1}\\
    \leq &\frac{1}{2}\begin{pmatrix}n-1 \\n_0\end{pmatrix}\left(\sum_{k=1}^N\frac{(n_k+1)(n-n_0-1)!}{n_1!\dots (n_k-1)!\cdots n_N!}a_1^{n_1}\cdots a_{k}^{n_k-1}\cdots a_N^{n_N}\right)\int_{ X}\eta_0^{n_0}\wedge\dots\wedge\omega_N^{n_N}\\
	\leq &\frac{1}{2}\left(\sum_{k=1}^N\frac{(n_k+1)(n-1)!}{n_0!n_1!\dots (n_k-1)!\cdots n_N!}a_1^{n_1}\cdots a_{k}^{n_k-1}\cdots a_N^{n_N}\right)\int_{ X}\eta_0^{n_0}\wedge\dots\wedge\omega_N^{n_N}
\end{align*}

Hence:
\begin{multline*}
    \frac{1}{2}\left(\sum_{k=1}^N\frac{(n_k+1)(n-1)!}{n_0!n_1!\cdots (n_k-1)!\cdots n_N!}a_1^{n_1}\cdots a_{k}^{n_k-1}\cdots a_N^{n_N}\right)\\ \geq \frac{S}{2n}\frac{n!}{n_0!n_1!\cdots n_N!}a_1^{n_1}\cdots a_N^{n_N}
\end{multline*}
and:
\[\sum_{k=1}^N\frac{n_k(n_k+1)}{a_k}\geq S\]

With the expression of $S$ we get:
\[\sum_{k=1}^N\frac{n_k(n_k+1)}{a_k}\geq \sum_{k=1}^N n_k(n_k+1)\]

Writing this as $\sum_{k=1}^N\left(\frac{n_k(n_k+1)}{a_k}-n_k(n_k+1)\right)\geq 0$ we see that one of the $a_k$, say $a_{k_0}=\min_k a_k$, is at most 1.

Let $\Sigma_k\subset \tilde{X}$ be an holomorphic $\mathbb{CP}^1$ in the $k$-th $\mathbb{CP}^{n_k}$ factor in $\tilde X$. Since $\Sigma_k$ is a complex submanifold, then the area of $\Sigma_k$ with respect to $\omega$ (resp. $\tilde\omega$) is $\mathcal{A}_\omega(\Sigma_k)=\int_{\Sigma_k}\omega=a_k\int_{\Sigma_k}\omega_k$ (resp. $\mathcal{A}_{\tilde\omega}(\Sigma_k)=\int_{\Sigma_k}\omega_k$). Hence:
\[\mathcal{A}_{\omega}(\Sigma_{k_0})=\int_{\Sigma_{k_0}}\omega=a_{k_0}\int_{\Sigma_{k_0}}\omega_{k_0}\leq \int_{\Sigma_{k_0}}\omega_{k_0}
=\mathcal{A}_{\tilde\omega}(\Sigma_{k_0}).\]

For the last part of the proposition, assume that $\min_k a_k=1$, then since :
\[\sum_{k=1}^N\left(\frac{n_k(n_k+1)}{a_k}-n_k(n_k+1)\right)\geq 0\] we get that $a_1=\dots=a_N=1$, and thus
\[[\omega]=[\eta_0]+[\omega_1]+\dots+[\omega_N]=[\omega_0+\omega_1+\dots+\omega_N].\]

Moreover, we also get that $\int_X\kappa_0\wedge\eta_0^{n_0-1}\wedge\omega_1^{n_1}\wedge\dots\wedge\omega_N^{n_N}=0$. Hence $[\kappa_0]=2\pi c_1(X_0)=0$ and by Yau's solution to the Calabi Problem there is a Ricci flat Kaehler metric $\omega_0\in[\eta_0]$. Hence:
\[[\omega]=[\omega_0]+[\omega_1]+\dots+[\omega_N]=[\omega_0+\omega_1+\dots+\omega_N].\]

\end{proof}

All that remains to be shown is the rigidity.
\begin{proof}[Proof of Theorem \ref{thm_prod_proj}, rigidity part.]
    Using the same notations as in the previous proof, let $\hat\omega$ be the product metric $\omega_0+\omega_1+\dots+\omega_N$.

    Under the integral bound $\int_X(\scal_\omega-S)\omega^n\geq 0$, we already know that if any non homologically trivial 2-sphere has area at least $4\pi$ then $\omega$ is in the Kaehler class of the product metric $\hat\omega$ and \[\int_X\scal_\omega\omega^n=\int_X S\omega^n=\int_X\scal_{\hat\omega}\omega^n.\]

    If we furthermore assume the pointwise bound $\scal_\omega\geq\scal_{\hat\omega}$, we get that $\scal_\omega=\scal_{\hat\omega}$. Hence $\omega$ and $\hat\omega$ are two constant scalar curvature metric in the same Kaehler class, by Chen-Tian (?), $\omega$ and the product metric $\hat\omega$ are isometric.
\end{proof}



\begin{itemize}
\item Other $\tilde X$ would be interesting, in particular I would like to know what's going on with compact Kaehler Einstein 4-mfd with positive first Chern class ($\mathbb{CP}^2$ blown up at up to 8 points if I remember correctly). The computation doesn't seem to give anything for $\mathbb{CP}^2$ blown up at a point, if I understand its cohomology and Kaehler cone enough (see next section). The question of of which model metric on $\mathbb{CP}^2$ blown up at a point to compare seems delicate (no KE or cscK metric)...
\item The few cases known in the Riemannian case suggest that rather than a total control of the topology, the existence of a non-zero degree map from $\tilde X$ to the model space $X$ together with the lower bound on the scalar curvature should be enough. Would this work with a nonzero degree holomorphic map  $\tilde X\to X_0\times\prod_{k=1}^N\mathbb{CP}^{n_k}$ ? (This would handle blow-ups I guess, as long as they have a positive scalar curvature Kaehler metric.)
\end{itemize}
\section{Product of $\mathbb{CP}^n$ and a projective curve of genus at least 2.}

\begin{proposition}
    Let $X=\mathbb{CP}^n\times \Sigma_g$ where $\Sigma_g$ is a smooth projective curve of genus $g\geq 2$, and let $\omega$ be a Kahler metric on $X$ such that $\int_X\scal_\omega\omega^n\geq 0$, then \[\sys_{[C]}(\omega)\leq \frac{n(n+1)}{2(g-1)}\sys_{[\Sigma_g]}(\omega)\] where $C$ is the image of the standard embedding $\mathbb{CP}^1\to\mathbb{CP}^n$.

    Equality is achieved if $\omega$ is in a multiple the Kaehler class of the product metric $\frac{n(n+1)}{2}\omega_{FS}+\omega_h$ where $\omega_{FS}$ is the Fubini-Study metric on $\mathbb{CP}^n$ and $\omega_h$ is the hyperbolic metric on $\Sigma_g$.

    If equality is achieved and moreover $\scal_X\geq 0$, then $\scal_X=0$ and $\omega$ is homothetic to $\frac{n(n+1)}{2}\omega_{FS}+\omega_h$.
\end{proposition}
\begin{proof}
    Let $\omega_{FS}$ be the Kaehler form of the Fubini-Study metric on $\mathbb{CP^n}$ and $\omega_h$ be the Kaehler form coming from the hyperbolic metric on $\Sigma_g$.
    Then there exists positive reals $a$ and $b$ such that $[\omega]=a[\omega_{FS}]+b[\omega_h]$ in $H^{1,1}(X)$.

    The cohomology class of the ricci form of $\omega$ is given by
    $[\rho]=\frac{n+1}{2}[\omega_{FS}]-[\omega_h]$.

    Then since $\rho\wedge\omega^{n-1}=\frac{1}{2(n+1)}\scal_\omega\omega^n$ the hypothesis $\int_X\scal_\omega\omega^n\geq 0$ gives:
    \[\int_X \left (\frac{n+1}{2}[\omega_{FS}]-[\omega_h]\right)\wedge \left(a[\omega_{FS}]+b[\omega_h]\right)^n\geq 0.\]

    We compute :
    \begin{align*}
        \left (\frac{n+1}{2}\omega_{FS}-\omega_h\right)\wedge & \left(a[\omega_{FS}]+b[\omega_h]\right)^n\\
        &=\left(\frac{n+1}{2}\omega_{FS}-\omega_h\right)\wedge(a^n\omega_{FS}^n+na^{n-1}b\omega_{FS}^{n-1}\omega_h)\\
        &=\left(\frac{n(n+1)}{2}a^{n-1}b-a^n\right)\omega_h\wedge\omega_{FS}^n\\
        &=a^{n-1}\left(\frac{n(n+1)}{2}b-a\right)\omega_h\wedge\omega_{FS}^n.
    \end{align*}
    Hence $\int_X\scal_{\omega}\omega^n\geq 0$ is equivalent to:
    \[a\leq\frac{n(n+1)}{2}b.\]
    Moreover, since $C$ and $\Sigma_g$ are complex curves in $X$, \[\sys_{[C]}(\omega)=\int_C\omega=a\int_C\omega_{FS}=4\pi a\]
    and
    \[\sys_{[\Sigma_g]}(\omega)=\int_{\Sigma_g}\omega=b\int_{\Sigma_g}\omega_{h}=4(g-1)\pi b\] by Gauss-Bonnet.

    Thus $\sys_{[C]}(\omega)\leq\frac{n(n+1)}{2(g-1)}\sys_{[\Sigma_g]}(\omega)$.

    If one has equality, we get that $a=\frac{n(n+1)}{2}b$ which shows that $[\omega]$ is a multiple of $[\frac{n(n+1)}{2}\omega_{FS}+\omega_h]$ in $H^{1,1}(X)$. If the pointwise condition $\scal_\omega\geq 0$ is satisfied, then we get that $\scal_\omega=0$ and the result follows from uniqueness of cscK metrics in a given Kaehler class.
\end{proof}

\section{$\mathbb{CP}^2$ blown up at a point.}

Let $X$ be the blow up of $\mathbb{CP}^2$ at a point $p$. Let $\pi:X\to\mathbb{CP}^2$ be the blow up map. Denote by $[\omega_0]$ the pull back of the tautological class of $\mathbb{CP}^2$ and by $[e]$ the class of the exceptional divisor. Note that in $H^2(X)$, $[e]^2=-[\omega_0]^2$ and $[e][\omega_0]=[\omega_0][e]=0$.

Then, by Demailly-Paun I guess ?, the Kaehler cone of $X$ is :
\[\mathcal{K}(X)=\{a[\omega_0]-b[e]|a>b>0\}\]
see \url{https://mathoverflow.net/q/96118/8887}, \url{https://mathoverflow.net/q/92660/8887}, \url{https://mathoverflow.net/q/394161/8887}
Moreover, the first Chern class of $X$ is $c_1(X)=3[\omega_0]-[e]$.

Assume $\omega$ is a Kaelher metric on $X$ with Kae0hler class $a[\omega_0]-b[e]$ and scalar curvature bigger than $s_0$. Let $\rho$ is its Ricci form, then:
\[[\rho\wedge\omega]=(3[\omega_0]-[e])(a[\omega_0]-b[e])=(3a-b)[\omega_0]^2.\]
Moreover $\rho\wedge\omega=\tfrac{1}{4}\scal_\omega \omega\wedge\omega\geq\tfrac{1}{4} s_0\omega\wedge\omega$ and:
\[[\omega\wedge\omega]=(a[\omega_0]-b[e])^2=(a^2-b^2)[\omega_0]^2\]

The inequality $\scal_\omega\geq s_0$ then gives that:
\[3a-b\geq \frac{s_0}{4}(a^2-b^2)\]
This is equivalent to:
\[\left(b-\frac{2}{s_0}\right)^2-\left(a+\frac{6}{s_0}\right)^2\geq \frac{32}{s_0^2}\]
Which gives:
\[\left(b+a+\frac{4}{s_0}\right)\left(b-a-\frac{8}{s_0}\right)\geq\frac{32}{s_0^2} \]

This bounds a bounded region of the Kaehler cone (double check the computation), need to work explicit bounds for $a$ and $b$...
%The curve with equation $\left(b+a-\tfrac{1}{s_0}\right)\left(b-a+\tfrac{2}{s_0}\right)=\tfrac{5}{4s_0^2}$ is an hyperbola with center $\left(\tfrac{3}{2s_0},-\tfrac{1}{2s_0}\right)$ and asymptotes the lines $b=-a+\tfrac{1}{s_0}$ and $b=a-\tfrac{2}{s_0}$. We see that for $b=a-\tfrac{1}{s_0}$ (which corresponds to a form in the Kaehler cone if $a>\tfrac{1}{s_0}$), the inequality is satisfied for any $a$ big enough. Hence we have no upper bound on $a$ or $b$.

\section{What would a general condition be ?}

Let $X^n$ be a compact Kaehler manifold and $\mathcal{K}(X)\subset H^{1,1}(X,\mathbb{R})$ be its Kaehler cone. Let $c_1$ be its first Chern class. Let \[\mathcal{K}_+(X)=\left\{[\omega]\in\mathcal{K}(X)\middle|\int_X c_1(X)\wedge \omega^{n-1}\geq \int_X\omega^n\right\}.\]
Let $\mathcal{P}\subset H_2(X)$ be the set of the homology classes of  all $\mathbb{P}^1 \subset X$.

Set $\sys_2^+(X)=\sup_{[\omega]\in \mathcal{K}_+(X)}\inf_{P\in\mathcal{P}}\int_P\omega$. The question becomes : for which $X$ is $\sys_2^+(X)$ finite ?

Here we asked the question for positive scalar curvature in average. We can ask the same question for pointwise positive scalar curvature, by replacing $\mathcal{K}_+(X)$ be the set $\mathcal{K}_{++}(X)$ of Kaehler classes which can be represented by a metric $\omega$ such that $\rho_\omega\wedge \omega^{n-1}\geq \omega^n$ pointwise. It is unclear to me whether $\mathcal{K}_{++}(X)\neq \mathcal{K}_{+}(X)$.

\appendix
\section{Ricci form and scalar curvature.}

Some facts. Le $g_{FS}$ be the Fubini-Study metric on $\mathbb{CP}^n$ with sectionnal curvature between $1/4$ and $1$, then $\Ric_{g_{FS}}=\frac{n+1}{2}g_{FS}$ and the scalar curvature is thus $n(n+1)$.


\begin{lemma}
    Let $(X^n,\omega)$ be a Kaehler manifold with ricci form $\rho$, then $\rho\wedge \omega^{n-1}=\frac{1}{2n}\scal_\omega\omega^n$
\end{lemma}
\begin{proof}
    We will prove that for any real $(1,1)$-form $\alpha$,
    \marginpar{\footnotesize GM: I mean, you can prove that if you want, but complex geometers know this. A short proof notes that $\alpha \wedge \omega^{n-1} = \Lambda \alpha$, where $\Lambda$ is the adjoint of the Lefschetz operator, and $\Lambda \alpha$ is known to be the trace of $\alpha$ wrt $\omega$.}
    $\alpha\wedge \omega^{n-1}= \tfrac{2}{n}a\omega^n$ where $a$ is the real trace of the symmetric form associated to $\alpha$.

    Since $\alpha\mapsto \alpha\wedge\omega^{n-1}$ is linear and $\Lambda^{n,n}X$ is generated by $\omega^n$, there is a linear form $\ell:\Lambda^{1,1}X\to C^{\infty}(X)$ such that $\alpha\wedge\omega^{n-1}=\ell(\alpha)\omega^n$. The symmetries of the problem give that $\ell$ is a multiple of the trace. Evaluating $\ell$ on $\omega$ gives the right factor.

    %This is a purely pointwise statement, thus we can work in $\mathbb{C}^n$ with its standard Kaehler form $\omega=\frac{i}{2}\sum_{k=1}^n dz_k\wedge d\bar z_k$ and write $\alpha=\frac{i}{2}\sum_{k,l}\alpha_{k\bar l}dz^k\wedge d\bar{z}^l$ for some reals $a_{kl}$, recall that $\omega^n=\frac{i^{n}}{2^{n}} n!\bigwedge_kdz^k\wedge d\bar{z}^k$. We also have:
    %\[\omega^{n-1}=\frac{i^{n-1}}{2^{n-1}}(n-1)!\sum_{k=1}^n\left(\bigwedge_{l\neq k}dz^l\wedge d\bar{z}^l\right).\]
    %Hence:
    %\[\alpha\wedge\omega^{n-1}=\frac{i^{n}}{2^{n}}(n-1)!\left(\sum_{k=1}^n\alpha_{k\bar k}\right)\left(\bigwedge_{l}dz^l\wedge d\bar{z}^l\right)=\frac{1}{n}\left(\sum_k\alpha_{k\bar k}\right)\omega^n.\]
    %Since $\sum_k\alpha_{k\bar k}$ is the complex trace of $\alpha$, it is half of its real trace and the proof is complete.

\end{proof}
\end{document}
