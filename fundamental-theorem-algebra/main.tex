\documentclass[11pt,draft]{article}

\usepackage{tgpagella}
\linespread{1.1}
\usepackage[utf8]{inputenc}
\usepackage[T1]{fontenc}

\usepackage[normalem]{ulem}
\usepackage{textcomp}
\usepackage{hyperref}

\usepackage{amsmath}
\usepackage{amssymb}
\usepackage{amsthm}

\newtheorem*{theo}{Theorem}
\newtheorem*{fundamental}{Fundamental theorem of algebra}

\newcommand{\kk}[1]{\mathbb{#1}}
\newcommand{\cc}[1]{\mathcal{#1}}

\def\ov#1{\overline{#1}}

\author{Gunnar Þór Magnússon}
\date{\today}
\title{The fundamental theorem of algebra}

\begin{document}

\maketitle



\begin{fundamental}
Every nonconstant polynomial over the complex numbers has a root.
\end{fundamental}



\section{The roots}


Charles Fefferman~\cite{fefferman} offers a rather simple approach to the
fundamental theorem of algebra.
Our outline has some quality of life improvements due to Kevin McGerty~\cite{kevin}.
The same idea is also used in Rudin~\cite[Chapter 8]{rudin1976principles}

\begin{proof}
Let $p(z) = a_0 + a_1 z + \cdots + a_n z^n$ be a polynomial of degree $n > 1$.
If we write
\[
    p(z) = a_n z^n (1 + a_{n-1}/z + \cdots + a_0/z^n)
\]
then we see that $1 + a_{n-1}/z + \cdots + a_0/z^n \to 1$ as $|z| \to \infty$,
so $|p|$ tends to the same limit as $|a_n z^n|$, which is infinity.
Thus $|p|$ attains a minimum value $m$, as it will be arbitrarily big outside
a very large disk but attain a minimum on the inside of that compact set.

Suppose that $m > 0$. If $|p|$ attains $m$ at $z_0$, then $q(z) := p(z-z_0)/m$
attains its minimum at $0$ and its minimum is $1$, so we may assume the same is
true for $p$. We then have
\[
    p(z) = 1 + b_k z^k + \cdots + b_n z^n,
\]
where $b_k \not= 0$.
Note that $|p|^2$ also attains its minimum at $0$ and that its minimum is $1$.
Now let $w \in \kk C$ be such that $w^k = -1/b_k$.
We have
\begin{align*}
    |p(tw)|^2
    &= (1 - t^k + \cdots + b_n w^n t^n)
    (1 - t^k + \cdots + \ov{b_n}\ov w^n t^n)
    \\
    &= 1 - t^k(2 + t q(t))
\end{align*}
for some real-valued polynomial $q$ and for real $t$.
The function $t \mapsto 2 + tq(t)$ is continuous and takes the value $2$ at $0$,
so it is positive for small $t > 0$.
But then $|p(tw)|^2 < 1$ for those small $t$, which contradicts $1$ being its
minimum.
\end{proof}


The main idea of the proof actually goes back to d'Alembert's original
proof, though it is somewhat hard to follow.

The first point is that $|p|$ must attain its minimum on $\kk C$. Today this
follows easily from facts about compactness and continuity. This is not limited
to complex-valued polynomials, as real polynomials also have this property.

The second ingredient is the existence of roots for arbitrary complex numbers,
which lets us make a change of variables to transform our polynomial into one of
the form $1 - t^k r(t)$ for a continuous function $r$ that is positive at $0$.
This is the step that fails for real polynomials; indeed there is no change of
variables that will achieve this for our best friend $p(x) = 1 + x^2$, which
\emph{does} attain its global minimum of $1$ at $0$.



\section{Liouville}

This is likely the proof people have seen of the fundamental theorem of algebra
if they've seen one at all. It's a staple of introductory complex analysis
courses. A professor of mine used to say that a good benchmark for whether a
result in complex analysis was useful was to see whether it could prove the
fundamental theorem of algebra.

To get here, simply develop the theory of holomorphic functions to the point
where we get the Cauchy integral formula, prove that holomorphic functions are
analytic, and prove Liouville's theorem:

\begin{theo}
A bounded entire function is constant.
\end{theo}

From there the proof of the fundamental theorem is short.

\begin{proof}
Assume $p$ is a nonconstant polynomial that has no root.
Then $1/p$ is a well-defined entire function.
We know as before that $|p(z)| \to \infty$ as $|z| \to \infty$, so $1/p$ is bounded.
But then $1/p$ is constant, so $p$ is constant, which is a contradiction.
\end{proof}

It's perhaps less clear where to pick something to point out that fails in the
real case here than before.
The real equivalent of a holomorphic function is perhaps a smooth function,
which can famously fail to be analytic (hello, partitions of unity!).
Liouville's theorem also fails for smooth functions, as something like $x
\mapsto 1/(1+x^2)$ shows.

Obviously we can also prove the fundamental theorem of algebra using results
that are proved using Liouville's theorem, like the Picard theorems. This feels
less sporting somehow, so we won't do it.

This proof is by contradiction, but it feels like it should be an example of
such a proof that goes ``Assume not $X$, then do all the proof and arrive at
$X$, which is a contradiction'' and could thus be turned into a direct proof.
The part where that doesn't work is in the growth estimate on $|p(z)|$, which
goes nowhere if we don't assume $p$ is nonconstant.




\section{Open mapping}

One of the main theorems on holomorphic functions is the open mapping theorem:

\begin{theo}
Let $U \subset \kk C$ be open and let $f : U \to \kk C$ be holomorphic.
Assume $f$ is nonconstant.
Then $f$ is an open map, that is, the image of any open subset under $f$ is open.
\end{theo}

By my old teacher's remark, this should be enough to prove the fundamental theorem of algebra.
And along with a helping of topology, it is.

\begin{proof}
Recall the Riemann sphere $\widehat{\kk C} = \kk C \cup \{\infty\}$.
Any polynomial $p$ induces a map $p : \widehat{\kk C} \to \widehat{\kk C}$ by setting $p(\infty) = \infty$,
and this induced map is holomorphic.

Assume $p$ is nonconstant.
Then the induced map $p$ is open, so the image $p(\widehat{\kk C}) \subset \widehat{\kk C}$ is open.
That image is also closed, because the source space $\widehat{\kk C}$ is compact and $p$ is
continuous, and the target space $\widehat{\kk C}$ is Hausdorff.
The image $p(\widehat{\kk C})$ is thus both open and closed, which means it is
all of $\widehat{\kk C}$ because that space is connected.
In particular, $p$ must map some point to zero.
\end{proof}

We've skipped over many of the details involved, like extending the open mapping
theorem to complex curves, proving that the induced map on the Riemann sphere is
holomorphic, and all the topology of the Riemann sphere. In our defence they're
all standard results in a first course on complex geometry.

This is the first direct proof of the fundamental theorem of algebra we see,
which is maybe good news for the more logically conservative reader.

The topological results we cite remain true for the extended real line $\kk R \cup \{\infty\}$,
as does the fact that a polynomial extends to a map of the extended line.
However, a real polynomial may not be an open map -- the function $x \mapsto x^2$
maps the open interval $(-1,1)$ to the half-open interval $[0,1)$ -- so the
proof breaks down there for real polynomials.



\section{Differential topology}


Here's a neat proof that seems to use little else from complex geometry than
that holomorphic maps are orientation-preserving.

\begin{proof}
Let $\kk C_n[X]$ be the complex vector space of polynomials of degree $\leq n$.
There is a holomorphic map
\[
    f : \prod_{j=1}^n \kk P(\kk C_1[X])
    \to \kk P(\kk C_n[X]),
    \quad
    ([p_1], \ldots, [p_n]) \mapsto [p_1 \cdots p_n].
\]
If $[x_j:y_j]$ are the coordinates of the $j$-th $\kk P(\kk C_1[X])$ factor,
they correspond to the polynomial $x_j + y_j X$. In the chart where $y_j \not=
0$ the image of this map thus lands in the chart of $\kk P(\kk C_n[X])$ where
the coefficient of $X^n$ is nonzero.
Let $z_j = x_j / y_j$ be the homogeneous coordinates; then
the differential of this map in those charts is
\[
    df_{(z_1, \ldots, z_n)}
    = \sum_{j=1}^n (z_1 + X) \cdots \widehat{(z_j+X)} \cdots (z_n+X) d z_j.
\]
If all the $z_j$ are distinct the differential is injective, and thus
surjective as the tangent spaces have the same dimensions. The cardinality of
the fiber at such a point is $|S_n|$,
where $S_n$ is the symmetric group on $n$ letters.

Now any holomorphic map of positive degree between compact connected complex
manifolds of the same dimension is surjective: A holomorphic map is
orientation-preserving, so its degree is exactly equal to the number of points
in the preimage of a regular value; see Milnor~\cite[Section~5]{milnor}.
The image therefore contains the set of
regular values, which is dense by Sard's theorem; again Milnor~\cite[Section~2]{milnor}.
But the image is also closed as it is the image of a compact set, so it must be
all of the target manifold.
\end{proof}

This actually proves directly that any complex polynomial splits into linear
factors, while most other proofs have to add an extra induction step to conclude
that.

I don't have a counterexample that shows where this breaks down in the
real case, but it must be related to orientability. In the real case the
manifold $\kk P(\kk R_2[X])$ isn't orientable, and neither is its product with
copies of itself. Therefore we aren't able to conclude the image contains the
set of regular values of the map, so we don't get surjectivity.


It's fun to see that this proof fails in a specific real case.
(We don't see \emph{why} it fails though.)
For degree-two real polynomials we have
\[
    f : \kk P(\kk R_1[X]) \times \kk P(\kk R_1[X])
    \to \kk P(\kk R_2[X]),
    \quad
    ([p], [q]) \mapsto [p q].
\]
Let's write $p = aX + b$ and $q = cX + d$, then
\[
    pq = ac X^2 + (ad + bc) X + bd
\]
and the discriminant of $pq$ is
\[
\Delta(pq)
= (ad + bc)^2 - 4 ac bd
= (ad - bc)^2.
\]
We can define a kind of discriminant of an element of $\kk P(\kk R_2[X])$ by
noting that for any degree-two polynomial $p$ we have $\Delta(\lambda p) =
\lambda^2 \Delta(p)$. Therefore we get a well-defined discriminant up to sign
$[\Delta(p)]$ of elements in the projective space whose value is either
negative, zero, or positive.

We conclude from the above that any element in the image of $f$ has zero or
positive projectivized discriminant.
However, there are elements in $\kk P(\kk R_2[X])$ whose projectivized
discriminant is negative, like our best friend $[X^2 + 1]$ whose projectivized
discriminant is $[\Delta(X^2 + 1)] = [-4] < 0$.
The map in this case is therefore not surjective.

It's possible to give a much more overkill proof along the same lines by
invoking Grauert and Remmert's open mapping theorem for finite holomorphic
morphisms between same-dimensional complex spaces. The proof then follows the
same lines as the open mapping proof in the one-dimen\-sion\-al case.



\bibliographystyle{plain}
\bibliography{main}

\end{document}
