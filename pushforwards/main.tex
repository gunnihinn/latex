\documentclass[11pt]{amsart}

\usepackage{tgpagella}
\linespread{1.1}
\usepackage[utf8]{inputenc}
\usepackage[T1]{fontenc}

\usepackage[normalem]{ulem}
\usepackage{textcomp}
\usepackage{hyperref}

\usepackage{amsmath}
\usepackage{amssymb}
\usepackage{amsthm}

\usepackage{tikz-cd}
\usepackage{color}

\newtheorem{theo}{Theorem}
\newtheorem{prop}[theo]{Proposition}
\newtheorem{lemm}[theo]{Lemma}
\newtheorem{coro}[theo]{Corollary}
\theoremstyle{definition}
\newtheorem{defi}[theo]{Definition}
\newtheorem{ques}[theo]{Question}
\newtheorem{exam}[theo]{Example}
\newtheorem{cexam}[theo]{Counterexample}

\newcommand{\kk}[1]{\mathbb{#1}}
\newcommand{\cc}[1]{\mathcal{#1}}

\def\lra{\longrightarrow}
\def\^#1{^{[#1]}}
\def\qandq{\quad\text{and}\quad}
\def\ov#1{\overline{#1}}
\def\rank{\operatorname{rank}}

\def\reg#1{#1_{\text{reg}}}
\def\sing#1{#1_{\text{sing}}}


\def\lit#1#2{\bgroup\color{#1} #2\egroup}

\def\eps{\varepsilon}

\def\pd{\operatorname{pd}}
\def\NN{\mathbf{N}}
\def\ZZ{\mathbf{Z}}
\def\QQ{\mathbf{Q}}
\def\RR{\mathbf{R}}
\def\CC{\mathbf{C}}
\def\PP{\mathbf{P}}

\DeclareMathOperator{\re}{Re}
\DeclareMathOperator{\Hess}{Hess}
\DeclareMathOperator{\pr}{pr}
\DeclareMathOperator{\Span}{span}
\DeclareMathOperator{\Gr}{Gr}
\DeclareMathOperator{\GL}{GL}
\DeclareMathOperator{\im}{Im}
\DeclareMathOperator{\Vol}{Vol}
\DeclareMathOperator{\Ker}{Ker}
\DeclareMathOperator{\Img}{Im}
\DeclareMathOperator{\End}{End}
\DeclareMathOperator{\Aut}{Aut}
\DeclareMathOperator{\Hom}{Hom}
\DeclareMathOperator{\Sym}{Sym}
\DeclareMathOperator{\id}{id}
\DeclareMathOperator{\tr}{tr}
\DeclareMathOperator{\adj}{adj}
\DeclareMathOperator{\Rm}{Rm}

\newcommand{\ext}[1]{\bigwedge{}^{\!\!#1}\,}

% Projective spaces
\newcommand{\PC}[1]{\PP(\CC^{#1})}
\newcommand{\PV}[1]{\PP(#1)}

\def\<{\langle}
\def\>{\rangle}
\def\d{\mathrm{d}}

\def\curv{\frac i{2\pi}\Theta}

\newtheorem{question}{Question}

\author{Gunnar Þór Magnússon}
\date{\today}
\title{Direct images of K\"ahler classes are K\"ahler}

\begin{document}


\begin{abstract}
Let $f : X \to Y$ be a surjective holomorphic map between compact K\"ahler manifolds and let $k = \dim X - \dim Y$.
If $\omega$ is a K\"ahler class on $X$, we show that $f_\omega^{k+1}$ is a K\"ahler class on $Y$.
\end{abstract}


\maketitle



Consider a surjective holomorphic map $f : X \to Y$ between compact K\"ahler manifolds.
It is well-known that the pullback morphism 
\[
f^* : H^*(Y,\ZZ) \to H^*(X,\ZZ)
\]
is surjective.
One proof goes like this:

\begin{proof}
Let $\omega$ be a K\"ahler metric on $X$ and set $k = \dim X - \dim Y$.
Then $f_*\omega^k$ is a closed positive current of bidimension $\dim Y$ on $Y$, and is thus a constant.
As it is a K\"ahler current, the constant is positive.
If $\beta$ is any class on $Y$ we have
\[
f_*(\omega^k \cup f^*\beta) = f_*\omega^k \cup \beta
\]
so if $f^*\beta = 0$ we must have $\beta = 0$.
\end{proof}


This may leave the reader wondering about other direct images of K\"ahler metrics on $X$.
In general $f_*\omega^p$ is a closed positive current on $Y$ but may not be a smooth form.
The first unknown case is $f_*\omega^{k+1}$, which is at least a K\"ahler current on~$Y$.
Two special cases are interesting:

\begin{itemize}
\item
If $f : X \to Y$ is a submersion then $f_*\omega^{k+1}$ an honest K\"ahler metric by integration over the fibers.

\item
If $X$ is projective and $\omega = c_1(L)$ where $L \to X$ is very ample, the projection formula and picking very general sections of $L$ let us see that $\int_Z (f_*\omega^{k+1})^p > 0$ for every irreducible analytic set $Z \subset Y$ by induction on $p = \dim Z$, so $f_*\omega^{k+1}$ is a K\"ahler class.
\end{itemize}


In this note we prove the direct image is always a K\"ahler class:

\begin{theo}
\label{theo:main}
Let $f : X \to Y$ be a surjective holomorphic map between compact K\"ahler manifolds and let $k = \dim X - \dim Y$.
If $\omega$ is a K\"ahler class on $X$, then $f_*\omega^{k+1}$ is a K\"ahler class on $Y$.
\end{theo}


The tool we use is the Demailly--Paun~\cite{demailly2004numerical} characterization of the K\"ahler cone of a compact K\"ahler manifold:


\begin{theo}
Let $X$ be a compact K\"ahler manifold.
A class $\omega$ on $X$ is K\"ahler if and only if
\[
\int_Z \omega^p > 0
\]
for all irreducible analytic sets $Z \subset X$ of dimension $p$.
\end{theo}


We review a few basic facts about analytic sets and complex spaces, since we need to integrate our class over those.


A \emph{complex space} is a ringed space that is locally modelled on analytic sets.

If $f : X \to Y$ is a holomorphic map and $Z \subset Y$ is a complex space then $f^{-1}(Z) \subset X$ is a complex space.
We can see this either by pulling back the local equations that define $Z$ or by noting that $f^{-1}(Z) = X \times_Z Y$.

If $Z$ is analytic the sets $Z_{\text{reg}}$ and $Z_{\text{sing}}$ of smooth and singular points are open and analytic, respectively.
The regular points are dense in $Z$.

If $f : X \to Y$ is proper and $Z \subset X$ is analytic then $f(Z)$ is analytic by Remmert's theorem.

Proper surjective holomorphic maps are open in the Zariski topology:

\begin{lemm}
Let $f : X \to Y$ be a proper surjective morphism of complex spaces.
If $U = X \setminus Z$ is an open set, where $Z \subset X$ is analytic, then $f(U) \subset Y$ is open.
\end{lemm}

\begin{proof}
Since $Z$ is analytic and $f$ is proper the set $f(Z)$ is analytic.
Now
\(
Y \setminus f(U)
= f(X \setminus U)
= f(Z)
\)
is closed, so $f(U)$ is open.
\end{proof}



\begin{proof}[Proof of Theorem~\ref{theo:main}]
First note the claim is true when $f$ is a surjective submersion with compact fibers (between spaces that are not necessarily compact).
There we can pick a K\"ahler metric and integrate a power of it over the fibers of $f$ and get a K\"ahler metric on $Y$ in the right class.
We will prove the general claim by reducing to this situation on dense open subsets of our spaces.

Now let $W \subset Y$ be an irreducible analytic set of dimension $p$.

\subsubsection*{Push-pull}

First note that $Z := f^{-1}(W)$ is an analytic set in $X$.
We write $g : Z \to W$ for the restriction of $f$ to $Z$.
It is a proper surjective morphism of complex spaces.
If we write $j_Z : Z \to X$ and $j_W : W \to Y$ for the inclusions we have a commutative square
\[
\begin{tikzcd}
Z \ar[r,"j_Z"] \ar[d,"g"] &
X \ar[d,"f"]
\\
W \ar[r,"j_W"] &
Y.
\end{tikzcd}
\]
Need to justify that $[g_*(\omega^{k+1})_{|Z}] = [f_*\omega^{k+1}]_{|W}$.
This is apparently true:
\begin{itemize}
\item
\url{https://mathoverflow.net/q/386052/4054}

\item
\url{https://mathoverflow.net/q/244844/4054}

\item
\url{https://mathoverflow.net/q/22869/4054}
\end{itemize}


\subsubsection*{Restricting to dense open submanifolds}

The set $W_{\text{reg}} \subset W$ is open and dense, so the image $g^{-1}(W_{\text{reg}}) \subset Z$ is open.
Its complement is $W \setminus W_{\text{reg}} = W_{\text{sing}}$, which is analytic.
Then 
\[
g^{-1}(W_{\text{reg}}) 
= g^{-1}(W) \setminus g^{-1}(W_{\text{sing}})
= Z \setminus g^{-1}(W_{\text{sing}})
\]
is the complement of an analytic set, and is thus dense.
Then the set $U := g^{-1}(W_{\text{reg}}) \cap Z_{\text{reg}}$ is open and dense, and its complement is
\[
Z \setminus \bigl( g^{-1}(W_{\text{reg}}) \cap Z_{\text{reg}} \bigr)
= g^{-1}(W_{\text{sing}})
\cup Z_{\text{sing}},
\]
which is the union of two analytic sets and thus analytic.
It follows that $V := g(U) \subset W_{\text{reg}}$ is open and dense, and $g : U \to V$ is surjective between smooth open sets.

\subsubsection*{Restricting to a submersion}

We can now look at
\[
N = 
\bigl\{
x \in U 
\bigm| \wedge^{\dim W} dg_x = 0 
\bigr\},
\]
which is the analytic set where $g$ is not a submersion.
Once again $U \setminus N$ is open and so is $V \setminus g(N)$, and both are dense, and $g : U \setminus N \to V \setminus g(N)$ is a surjective submersion.
Then, finally,
\begin{align*}
\int_W [f_*\omega^{k+1}]_{|W}^{p+1}
&= \int_W [g_*(\omega^{k+1})_{|Z}]^{p+1}
\\
&= \int_{V \setminus g(N)} [g_*(\omega^{k+1})_{|Z}]^{p+1}
= \int_{V \setminus g(N)} (g_*\omega^{k+1})^{p+1} > 0
\end{align*}
because $g_*\omega^{k+1}$ is a K\"ahler form on $V \setminus g(N)$. 
\end{proof}


This still has a problem:
We need to know that $\omega_{|Z}$ contains a K\"ahler form.
This also appears to be surmountable.



\section{Higher images}


\begin{defi}
A class $u \in H^{p,p}(X,\RR)$ is \emph{numerically positive} if for any irreducible analytic set $Z \subset X$ of dimension $k \geq p$ and K\"ahler class $\omega$ on $X$ we have
\[
\int_Z u \cup \omega^{k-p} > 0.
\]
\end{defi}


Numerically positive $(p,p)$-classes form an open convex cone $N^p$.
We have $N^0 = \RR_+$.
A class $u$ is K\"ahler if and only if $u^p \in N^p$ for all $p = 1, \ldots, \dim X$ by Demailly--Paun.

We would like to see that if $\omega$ is a K\"ahler class and $L$ is its Lefschetz operator we have $L N^p \subset N^{p+1}$.

Then would like to see that if $f : X \to Y$ is surjective then $f_* N^{p+k}(X) \subset N^p(Y)$.





\bibliographystyle{alpha}
\bibliography{main}

\end{document}
