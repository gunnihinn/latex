\documentclass[11pt]{amsart}

\usepackage{tgpagella}
\linespread{1.1}
\usepackage[utf8]{inputenc}
\usepackage[T1]{fontenc}

\usepackage[normalem]{ulem}
\usepackage{textcomp}
\usepackage{hyperref}

\usepackage{amsmath}
\usepackage{amssymb}
\usepackage{amsthm}

\usepackage{tikz-cd}
\usepackage{color}

\newtheorem{theo}{Theorem}
\newtheorem{prop}[theo]{Proposition}
\newtheorem{lemm}[theo]{Lemma}
\newtheorem{coro}[theo]{Corollary}
\theoremstyle{definition}
\newtheorem{defi}[theo]{Definition}
\newtheorem{ques}[theo]{Question}
\newtheorem{exam}[theo]{Example}
\newtheorem{cexam}[theo]{Counterexample}

\newcommand{\kk}[1]{\mathbb{#1}}
\newcommand{\cc}[1]{\mathcal{#1}}

\def\lra{\longrightarrow}
\def\^#1{^{[#1]}}
\def\qandq{\quad\text{and}\quad}
\def\ov#1{\overline{#1}}
\def\rank{\operatorname{rank}}

\def\reg#1{#1_{\text{reg}}}
\def\sing#1{#1_{\text{sing}}}


\def\lit#1#2{\bgroup\color{#1} #2\egroup}

\def\eps{\varepsilon}

\def\pd{\operatorname{pd}}
\def\NN{\mathbf{N}}
\def\ZZ{\mathbf{Z}}
\def\QQ{\mathbf{Q}}
\def\RR{\mathbf{R}}
\def\CC{\mathbf{C}}
\def\PP{\mathbf{P}}

\DeclareMathOperator{\re}{Re}
\DeclareMathOperator{\Hess}{Hess}
\DeclareMathOperator{\pr}{pr}
\DeclareMathOperator{\Span}{span}
\DeclareMathOperator{\Gr}{Gr}
\DeclareMathOperator{\GL}{GL}
\DeclareMathOperator{\im}{Im}
\DeclareMathOperator{\Vol}{Vol}
\DeclareMathOperator{\Ker}{Ker}
\DeclareMathOperator{\Img}{Im}
\DeclareMathOperator{\End}{End}
\DeclareMathOperator{\Aut}{Aut}
\DeclareMathOperator{\Hom}{Hom}
\DeclareMathOperator{\Sym}{Sym}
\DeclareMathOperator{\id}{id}
\DeclareMathOperator{\tr}{tr}
\DeclareMathOperator{\adj}{adj}
\DeclareMathOperator{\Rm}{Rm}

\newcommand{\ext}[1]{\bigwedge{}^{\!\!#1}\,}

% Projective spaces
\newcommand{\PC}[1]{\PP(\CC^{#1})}
\newcommand{\PV}[1]{\PP(#1)}

\def\<{\langle}
\def\>{\rangle}
\def\d{\mathrm{d}}

\def\curv{\frac i{2\pi}\Theta}

\newtheorem{question}{Question}

\author{Gunnar Þór Magnússon}
\date{\today}
\title{Direct images of K\"ahler classes}

\begin{document}

\maketitle


\section{Injectivity of pullbacks}

\begin{prop}
Let $f : X \to Y$ be a smooth map between compact smooth manifolds and let $k = \dim X - \dim Y$.
If there exists $x \in H^k(X)$ such that $f_* x \not= 0$, then $f^* : H^*(Y) \to H^*(X)$ is injective.
\end{prop}

\begin{proof}
We have
\[
f_*(x \cup f^*y) = f_*x \cup y
\]
in cohomology.
For $x$ as in the statement and $y$ such that $f^*y = 0$ we get $f_* x \cup y = 0$.
Because $f_* x$ has degree zero we then get $y = 0$.
\end{proof}



\begin{prop}
Let $f : X \to Y$ be a surjective map between compact K\"ahler manifolds.
Let $k = \dim X - \dim Y$ and let $\omega$ be a K\"ahler metric on $X$.
Then the current $f_*\omega^{k}$ is a positive constant.
\end{prop}

\begin{proof}
The form $\omega^k$ is smooth and closed, so $f_*\omega^k$ is a dimension-zero current on $Y$ such that $d f_*\omega^k = 0$.
Then $\< f_*\omega^k, d\alpha \> = \< df_*\omega^k, \alpha \> = 0$ for any $(2\dim Y-1)$-form $\alpha$.
Let $dV$ be a volume form of volume 1 on $Y$ and set $c = \< f_*\omega^k ,dV\>$.
If $\eta$ is any top-degree form on $Y$ we note that
\[
\int_Y (\eta - x dV) = 0,
\]
where $x = \int_Y \eta$.
Then $\eta - x dV = d\alpha$ for some form $\alpha$, and
\[
\< f_*\omega^k, \eta \>
= \< f_*\omega^k, x dV + d\alpha \>
= cx
= \< c, \eta \>.
\]
Since $\eta$ was arbitrary we have $f_*\omega^k = c$.
Now
\[
c = \< f_*\omega^k, dV \>
= \< \omega^k, f^*dV \>
= \int_X \omega^k \wedge f^*dV 
> 0
\]
because the integrand is positive on the dense open set where $f$ is a submersion.
\end{proof}

\begin{coro}
Let $f : X \to Y$ be a surjective holomorphic map between compact complex manifolds with $X$ K\"ahler.
Then $f^*$ is injective.
\end{coro}



\section{Direct images}

Demailly--Paun~\cite[Theorem 4.5(iii)]{demailly2004numerical} proved the following for a compact K\"ahler manifold $X$:

\begin{theo}
A class $\alpha$ is nef if and only if $\int_Z \alpha \cup \omega^{p-1} \geq 0$ for every analytic set $Z \subset X$ of dimension $p$ and every K\"ahler class $\omega$.
\end{theo}

Changing the $\geq 0$ to $> 0$ does not characterize K\"ahler classes:

\begin{exam}
There exists a projective surface $X$ with strictly nef divisor $D$ (so $D \cdot C > 0$ for all curves $C$) but $D^2 = 0$.\footnote{\url{https://mathoverflow.net/q/481104/4054}}
Then $c_1(D)$ is nef but not K\"ahler.
If $\omega$ is any K\"ahler class on $X$ we must have $\int_X c_1(D) \cup \omega > 0$, because otherwise we would get $c_1(D) = 0$ by Hodge--Riemann.
\end{exam}





\begin{prop}
Let $f : X \to Y$ be surjective and set $k = \dim X - \dim Y$.
\begin{enumerate}
\item
If $\mu$ is nef on $X$ then $f_* \mu^{k+1}$ is nef on $Y$.

\item
If $\omega$ is a K\"ahler metric on $X$ then $f_*\omega^{k+1}$ is a K\"ahler current on $Y$.

\item
If $\omega$ is a K\"ahler class on $X$ then $\int_C f_*\omega^{k+1} > 0$ for any analytic set $C \subset Y$ of dimension $1$.
\end{enumerate}
\end{prop}

\begin{proof}
(1)
Let $Z \subset Y$ be analytic of dimension $p$.
Let $\eta$ be a K\"ahler class on $Y$.
Then
\[
\int_Z f_*\mu^{k+1} \cup \eta^{p-1}
= \int_{f^{-1}(Z)} \mu^{k+1} \cup f^*\eta^{p-1}.
\]
Let $\omega$ be a K\"ahler class on $X$ and $\eps > 0$.
Since $\mu$ is nef the class $\mu + \eps \omega$ is K\"ahler for all $\eps > 0$.
Since $f^*\eta$ is nef on $X$ we have
\[
0 \leq \int_{f^{-1}(Z)} (\mu + \eps\omega)^{k+1} \cup f^*\eta^{p-1}.
\]
Then
\[
\int_Z f_*\mu^{k+1} \cup \eta^{p-1}
= \lim_{\eps \to 0} \int_{f^{-1}(Z)} (\mu + \eps\omega)^{k+1} \cup f^*\eta^{p-1}
\geq 0.
\]

(2)
The form $\omega^{k+1}$ is strongly positive and $d$-closed, so $f_*\omega^{k+1}$ is positive and closed.
Let $\eta$ be a Hermitian metric on $Y$.
By compactness we have $\omega \geq t f^*\eta$ on $X$ for all $t > 0$ small enough.
Then $\omega^{k+1} - t \omega^k \wedge f^*\eta$ is strongly positive for all small $t$, and thus
\[
f_*\omega^{k+1} - t f_*\omega^k \eta \geq 0
\]
for small enough $t$.

(3)
The set $f^{-1}(C)$ is analytic of dimension $k+1$ and
\[
\int_C f_*\omega^{k+1}
= \int_{f^{-1}(C)} \omega^{k+1} > 0.
\qedhere
\]
\end{proof}

\begin{prop}
Let $f : X \to Y$ be a surjective map between projective manifolds.
Let $k = \dim X - \dim Y$ and let $\omega$ be a K\"ahler class on $X$.
If $\omega \in H^2(X,\QQ)$ then $f_*\omega^{k+1}$ is K\"ahler.
\end{prop}

\begin{proof}
Being K\"ahler is invariant under scaling by positive constants so we may assume $\omega \in H^2(X,\ZZ)$ and that $\omega = c_1(L)$, where $L \to X$ is a very ample line bundle.
Let $Z \subset Y$ be an irreducible analytic set of dimension $p$.
We will prove that the $p$-th power of $f_*\omega^{k+1}$ has positive integral over $Z$ by induction on $p$.
The case $p = 1$ is already handled, so suppose $p > 1$.

Now $f^{-1}(Z)$ is an analytic set of dimension $p+k$ and the projection formula gives
\[
\int_Z (f_*\omega^{k+1})^p
= \int_{f^{-1}(Z)} \omega^{k+1} \cup f^*(f_*\omega^{k+1})^{p-1}.
\]
Let $H$ be a hyperplane defined by a section of $L$ such that no irreducible component of $f^{-1}(Z)$ is contained in $H$.
Then $f^{-1}(Z) \cap H$ has dimension $p+k-1$ and $f(f^{-1}(Z) \cap H)$ has dimension $p-1$, and 
\begin{align*}
\int_{f^{-1}(Z)} \omega^{k+1} \cup f^*(f_*\omega^{k+1})^{p-1}
&= \int_{f^{-1}(Z) \cap H} \omega^{k} \cup f^*(f_*\omega^{k+1})^{p-1}
\\
&= f_*\omega^{k} \int_{f(f^{-1}(Z) \cap H)} (f_*\omega^{k+1})^{p-1}
> 0
\end{align*}
by the projection formula and induction.
\end{proof}


Direct images by blowups should send K\"ahler classes to K\"ahler classes.
For points we see this from the description of the cohomology ring of the blowup.

Hironaka proved (in our case) that any surjective $f : X \to Y$ can be factored as
\[
\begin{tikzcd}
X \times_Y Y' \ar[d,"g"] \ar[r,"\mu"] &
X \ar[d,"f"]
\\
Y' \ar[r,"\pi"] &
Y
\end{tikzcd}
\]
where $\pi$ is a blowup and $g$ is flat.
Presumably $\mu$ is a blowup as well.
We take $\omega$ K\"ahler on $X$.
There should be a K\"ahler class $\omega'$ on $X \times_Y X'$ such that $\mu_*\omega' = \omega$, and then $f_*\omega^{k+1} = \pi_* g_* (\omega')^{k+1}$.
If we can prove $g_* (\omega')^{k+1}$ is K\"ahler, then the pushforward by the blowup should be K\"ahler, and then so is $f_*\omega^{k+1}$.

How does it help to know that our morphism is flat?



\subsection*{Skoda}

Try Skoda's extension theorem:


\begin{theo}
Let $E \subset X$ be a closed pluripolar sem and let $\Theta$ be a closed positive current on $X \setminus E$ such that the coefficients $\Theta_{IJ}$ of $\Theta$ are measures with locally finite mass near $E$.
Then the trivial extension $\tilde \Theta$ obtained by extending the measures $\Theta_{IJ}$ by $0$ on $E$ is still closed on $X$.
\end{theo}


If $T$ is a closed positive current of dimension $(p,p)$, the \emph{trace measure} of $T$ with respect to a Hermitian metric $\omega$ is
\[
\sigma_T = \frac{1}{2^p} T \wedge \omega\^p.
\]
The \emph{mass measure} $\|T\| = \sum_{I,J} |T_{IJ}|$ of a positive current is dominated by $C \sigma_\omega(T)$, where $C > 0$ is some constant independent of $T$.


Let $\alpha$ be a smooth $(k+1,k+1)$-form on $X$.
If $f$ is a fibration on $X \setminus Z \to Y \setminus f(Z)$ then $T = f_*\alpha$ is a closed smooth positive form $\beta$ there and $f(Z)$ is analytic and in particular pluripolar.
If $\omega$ is some Hermitian metric on $Y$ then
\[
\sigma_{\beta} = \frac{1}{2^p} \< \alpha, f^* \omega\^{m-1} \>
\]
has finite mass on all of $Y$, so the coefficients of $\beta$ have finite mass on $Y \setminus f(Z)$.
Then the trivial extension $\tilde\beta$ is also closed and positive on $Y$.
It is also nef, and $[\tilde\beta]^m > 0$ by integration over the dense open set where $\tilde\beta = \beta$ is a K\"ahler form.
We have
\[
f_*\alpha = \tilde\beta + \mathbf{1}_{f(Z)} f_* \alpha.
\]
Since $f_*\alpha$ and $\tilde\beta$ are nef then so is $\tilde\beta + t \mathbf{1}_{f(Z)} f_*\alpha$ for any $0 \leq t \leq 1$.

The current $\mathbf{1}_{f(Z)} f_* \alpha$ has bidimension $(m-1,m-1)$ like $f_*\alpha$ and $\tilde\beta$.
If such a current is supported on a set of dimension $< m-1$ it is zero, so this current is nonzero only on the irreducible components of $f(Z)$ of dimension $m-1$.
If $W_1, \ldots, W_k$ are those components we have
\[
\mathbf{1}_{f(Z)} f_* \alpha
= \sum_{j=1}^k \mathbf{1}_{W_j} f_*\alpha.
\]
Those components should form the image of the set where $df$ has rank $m-1$.


Let $Z_r(f) = \{ x \in X \mid \rank df \leq r \}$.
Each of these is closed and $Z_0 \subset \cdots \subset Z_m = X$.
Hartshone, Proposition III.10.6, says that for algebraic morphisms of finite type we have $\dim f(Z_r) \leq r$.
The argument probably goes through for analytic maps.
The proof of his Proposition III.10.4 is probably interesting too.

In particular $f(Z_0)$, where $f_*\alpha = 0$, is just some points.
Then $f(Z_1)$ is a curve in $Y$.
We already know things are positive on curves.
Pick an irreducible component $W$ of $f(Z_r)$ of dimension $r$.
Look at $f^{-1}(W) \subset Z_r$; it has dimension $k+r$, and $\dim(X_y \cap f^{-1}(W)) \leq k$ for every $y$.
Also $\dim f^{-1}(W) \cap Z_{r-1} \leq k+r-1$.
So $\rank df = r$ on a dense open set $U \subset f^{-1}(W)$.
Then $f : U \to f(U) \subset W$ is a submersion (?), and $f(U) \subset W$ is open and dense?

Want: If $Z \subset Y$ is analytic of dimension $p$, there exists a dense open set $U \subset f^{-1}(Z)$ that is smooth, $f(U) \subset Z$ is dense, and $f : U \to f(U)$ is a submersion.


Want to set $U = f^{-1}(Z) \setminus Z_{p-1}$.
Possibly also take away the singular locus of $f^{-1}(Z)$.
This should take away at most a hypersurface in the analytic set, so we are left with a smooth space of dimension $p$.
Would like to see that $df$ has rank $p$ on $U$.
Maybe take $V = Z \setminus f(Z_{p-1})$ open and dense and $U = f^{-1}(V)$ open.
Set $W = f^{-1}(Z)$.
Have
\[
\begin{tikzcd}
& & 0 \ar[d] & & 
\\
& & T_{X/Y|W} \ar[d] & &
\\
0 \ar[r] &
T_W \ar[r] &
T_{X|W} \ar[d] \ar[r] &
\cc I_W \ar[r] &
0
\\
& & f^*T_{Y|W} \ar[d] & &
\\
& & 0 & & 
\end{tikzcd}
\]


If we have that we still need to check that the current plays nicely with restrictions.

We have $T_W \subset T_Y$ and $T_{f^{-1}(W)} \subset T_X$ (modulo singular weirdness).
Have a square
\[
\begin{tikzcd}
T_{f^{-1}(W)} \ar[r] \ar[d,"df"] &
T_X \ar[d,"df"] \\
T_{W} \ar[r] &
T_Y
\end{tikzcd}
\]
that gives
\[
\begin{tikzcd}
\ext r T_{f^{-1}(W)} \ar[r] \ar[d,"\wedge^r df"] &
\ext r T_X \ar[d,"\wedge^r df"] \\
\ext r T_{W} \ar[r] &
\ext r T_Y.
\end{tikzcd}
\]
How can we justify that $\ext r T_W \cap \Img \wedge^r df \not= \varnothing$ on an open set?
These are basically just two lines in $\PP(\ext r T_Y)$.

I have the same question about a submersion, though, so clearly I don't understand something here.
Well: If
\[
0 \lra S \lra V \lra Q \lra 0
\]
is a short exact sequence of vector spaces then
\[
0 \lra \ext{1} S \wedge \ext\bullet V \lra \ext\bullet V \lra \ext\bullet Q \lra 0
\]
is also exact, so $df$ being surjective implies $\wedge^r df$ is surjective.
When everything is smooth we also just restrict $f$ to $f^{-1}(W) \to W$, which is still a submersion by the above.



\subsection*{Complex analytic geometry}


\begin{defi}
An analytic set is a subset $Z \subset X$ such that for every $x \in Z$ there exists an open set $U \subset X$ containing $x$ and holomorphic functions $f_1, \ldots, f_p \in H^0(U, \cc O_X)$ such that
\[
Z \cap U = \bigcap_{j=1}^p \{ z \in U \mid f_j(z) = 0 \}.
\]
\end{defi}

\begin{defi}
A map $f : Z \to W$ between analytic sets is holomorphic if it is locally the restriction of a holomorphic map $X \to W$.
\end{defi}

Holomorphic functions $f : Z \to \CC$ on an analytic set form a sheaf $\cc O_Z$.
We have an exact sequence
\[
0 \lra \cc I_Z \lra \cc O_{X|Z} \lra \cc O_Z \lra 0,
\]
where $\cc I_Z = \{ f \in \cc O_{X|Z} \mid f_{|Z} = 0 \}$.

A \emph{complex space} is a ringed space that is locally modelled on analytic sets.


If $f : X \to Y$ is a holomorphic map and $Z \subset Y$ is analytic then $f^{-1}(Z) \subset X$ is analytic.

If $Z$ is analytic the sets $Z_{\text{reg}}$ and $Z_{\text{sing}}$ of smooth and singular points are open and analytic, respectively.
The regular points are dense.

If $f : X \to Y$ is proper and $Z \subset X$ is analytic then $f(Z)$ is analytic.


Let $f : X \to Y$ be surjective, proper and flat, and set $k = \dim X - \dim Y$.
Let $W \subset Y$ be an irreducible analytic set of dimension $p$.
Then $Z = f^{-1}(W)$ is an analytic set of dimension $p+k$.
The map $f : Z \to W$ is surjective and proper.
It won't be flat, so it won't automatically be open.
But actually I only need:

\begin{lemm}
Let $f : X \to Y$ be a proper surjective morphism of complex spaces.
If $U = X \setminus Z$ is an open set, where $Z \subset X$ is analytic, then $f(U) \subset Y$ is open.
\end{lemm}

\begin{proof}
Since $Z$ is analytic and $f$ is proper the set $f(Z)$ is analytic.
Now
\(
Y \setminus f(U)
= f(X \setminus U)
= f(Z)
\)
is closed, so $f(U)$ is open.
\end{proof}


\begin{theo}
Let $f : X \to Y$ be a surjective holomorphic map between compact K\"ahler manifolds.
Set $k = \dim X - \dim Y$.
If $\omega$ is a K\"ahler class on $X$, then $f_*\omega^{k+1}$ is a K\"ahler class on $Y$.
\end{theo}


\begin{proof}
First note the claim is true when $f$ is a surjective submersion with compact fibers (between spaces that are not necessarily compact).
Then we can pick a K\"ahler metric and integrate a power of it over the fibers of $f$ and get a K\"ahler metric on $Y$ in the right class.
We will prove the general claim by reducing to this situation on dense open subsets of our spaces.

Let $W \subset Y$ be an irreducible analytic set of dimension $p$.
We will show that $\int_W (f_*\omega^{k+1})^p > 0$, thus proving that the class is K\"ahler by Demailly--Paun~\cite{demailly2004numerical}.

\subsubsection*{Push-pull}

First note that $Z := f^{-1}(W)$ is an analytic set in $X$.
We write $g : Z \to W$ for the restriction of $f$ to $Z$.
It is a proper surjective morphism of complex spaces.
If we write $j_Z : Z \to X$ and $j_W : W \to Y$ for the inclusions we have a commutative square
\[
\begin{tikzcd}
Z \ar[r,"j_Z"] \ar[d,"g"] &
X \ar[d,"f"]
\\
W \ar[r,"j_W"] &
Y.
\end{tikzcd}
\]
Need to justify that $[g_*(\omega^{k+1})_{|Z}] = [f_*\omega^{k+1}]_{|W}$.
This is apparently true:
\begin{itemize}
\item
\url{https://mathoverflow.net/q/386052/4054}

\item
\url{https://mathoverflow.net/q/244844/4054}

\item
\url{https://mathoverflow.net/q/22869/4054}
\end{itemize}


\subsubsection*{Restricting to open dense submanifolds}

The set $W_{\text{reg}} \subset W$ is open and dense, so the image $g^{-1}(W_{\text{reg}}) \subset Z$ is open.
Its complement is $W \setminus W_{\text{reg}} = W_{\text{sing}}$, which is analytic.
Then 
\[
g^{-1}(W_{\text{reg}}) 
= g^{-1}(W) \setminus g^{-1}(W_{\text{sing}})
= Z \setminus g^{-1}(W_{\text{sing}})
\]
is the complement of an analytic set, and is thus dense.
Then the set $U := g^{-1}(W_{\text{reg}}) \cap Z_{\text{reg}}$ is open and dense, and its complement is
\[
Z \setminus \bigl( g^{-1}(W_{\text{reg}}) \cap Z_{\text{reg}} \bigr)
= g^{-1}(W_{\text{sing}})
\cup Z_{\text{sing}},
\]
which is the union of two analytic sets and thus analytic.
It follows that $V := g(U) \subset W_{\text{reg}}$ is open and dense, and $g : U \to V$ is surjective between smooth open sets.

\subsubsection*{Restricting to a submersion}

We can now look at
\[
N = 
\biggl\{
x \in U 
\Bigm| \det \Bigl( dg_x : T_{U,x} \to g^*T_{V,x} \Bigr) = 0 
\biggr\},
\]
which is the analytic set where $g$ is not a submersion.
Once again $U \setminus N$ is open and so is $V \setminus g(N)$, and both are dense, and $g : U \setminus N \to V \setminus g(N)$ is a surjective submersion.
Then, finally,
\begin{align*}
\int_W [f_*\omega^{k+1}]_{|W}^{p+1}
&= \int_W [g_*(\omega^{k+1})_{|Z}]^{p+1}
\\
&= \int_{V \setminus g(N)} [g_*(\omega^{k+1})_{|Z}]^{p+1}
= \int_{V \setminus g(N)} (g_*\omega^{k+1})^{p+1} > 0
\end{align*}
because $g_*\omega^{k+1}$ is a K\"ahler form on $V \setminus g(N)$. 
\end{proof}


This still has a problem:
We need to know that $\omega_{|Z}$ contains a K\"ahler form.
This also appears to be surmountable.



\bibliographystyle{alpha}
\bibliography{main}

\end{document}
