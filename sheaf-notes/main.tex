\documentclass[10pt,a4paper]{article}

\usepackage{lmodern}
\linespread{1.1}
\usepackage[utf8]{inputenc}
\usepackage[T1]{fontenc}

\usepackage{fancyref}
\usepackage[colorlinks=true]{hyperref}

\usepackage{amsmath}
\usepackage{amssymb}
\usepackage[amsmath,thref,thmmarks,hyperref]{ntheorem}

\usepackage{tikz-cd}

\theorembodyfont{\itshape}
\newtheorem{theo}{Theorem}[section]
\newtheorem{prop}[theo]{Proposition}
\newtheorem{lemm}[theo]{Lemma}
\newtheorem{coro}[theo]{Corollary}

\theorembodyfont{\rm}
\newtheorem{defi}[theo]{Definition}
\newtheorem{exam}[theo]{Example}
\newtheorem{rema}[theo]{Remark}
\newtheorem*{claim}{Claim}

\theoremseparator{:}
\theoremsymbol{\ensuremath{\Box}}
\newtheorem*{proof}{Proof}

\newcommand{\kk}[1]{\mathbb{#1}}
\newcommand{\cc}[1]{\mathcal{#1}}

\newcommand{\sh}[1]{#1}
\newcommand{\cat}[1]{\mathsf{#1}}

\def\qedhere{}

\def\^#1{^{[#1]}}
\def\qandq{\quad\text{and}\quad}
\def\ov#1{\overline{#1}}

\def\NN{\mathbf{N}}
\def\ZZ{\mathbf{Z}}
\def\QQ{\mathbf{Q}}
\def\RR{\mathbf{R}}
\def\CC{\mathbf{C}}

\DeclareMathOperator{\Span}{span}
\DeclareMathOperator{\Gr}{Gr}
\DeclareMathOperator{\GL}{GL}
\def\Im{\operatorname{Im}}
\DeclareMathOperator{\Vol}{Vol}
\DeclareMathOperator{\Ker}{Ker}
\DeclareMathOperator{\Coker}{Coker}
\DeclareMathOperator{\End}{End}
\DeclareMathOperator{\Aut}{Aut}
\DeclareMathOperator{\Hom}{Hom}
\DeclareMathOperator{\id}{id}
\DeclareMathOperator{\tr}{tr}

\newcommand{\ext}[1]{\bigwedge{}^{\!\!#1}\,}

\newcommand{\topop}[1]{\cat{Top}(#1)^{\mathrm{op}}}

\author{Gunnar Þór Magnússon}
\date{\today}
\title{Notes on sheaf theory}


\begin{document}

\maketitle


\section*{Introduction}

Some notes on sheaf theory I wrote for myself.
I've forgotten a lot about sheaves from back in the day, and some I never
really knew.



\section{Sheaves}

We'll work with sheaves on a topological space $X$.
I suspect it's possible to define them on more general objects, like posets
maybe, because of the functor definition of a presheaf but I'm interested in
geometry and my bias will be towards that point of view.
In particular we won't discuss sheaves in logic or similar venues.


\paragraph{Presheaves}

\begin{defi}
A \emph{presheaf} $\sh{F}$ of sets on a topological space $X$ is the following
data:
\begin{enumerate}
\item
A set $\sh{F}(U)$ for every open set $U \subset X$.

\item
For $V \subset U$ a function $\rho_{VU} : \sh{F}(U) \to \sh{F}(V)$, where $\rho_{UU} = \id$.

\item
If $W \subset V \subset U$ then the diagram
$$
\begin{tikzcd}
\sh{F}(W) \ar[r,"\rho_{WV}"] \ar[dr,"\rho_{WU}"] & \sh{F}(V) \ar[d,"\rho_{VU}"]
\\
                & \sh{F}(U)
\end{tikzcd}
$$
commutes.
\end{enumerate}
\end{defi}

Define a category $\cat{Top}(X)$ whose objects are the open sets of $X$ and the
morphisms are inclusions.

\begin{prop}
A presheaf on $X$ is exactly a functor $\topop{X} \to
\cat{Ens}$.
\end{prop}

\begin{proof}
That a presheaf defines a functor is clear.

Let $\sh{F} : \topop{X} \to \cat{Ens}$ be a functor.
Then we get objects $\sh{F}(U)$ needed for condition 1 and morphisms like needed
for conditions 2 and 3 as part of the functor data.
Condition 3 in the definition of a presheaf is met automatically.
A functor takes the identity to the identity, so we see that $\rho_{UU} 
= \sh{F}(\id_U) = \id$.
\end{proof}

We also define presheaves of Abelian groups, rings, modules, and so on by
requiring the sets $\sh{F}(U)$ to have those structures and the restriction
morphisms to be morphisms of those structures.
Or, by the above, just pick a category $\cat{C}$ and define a presheaf of $\cat{C}$
objects on $X$ to be a functor $\topop{X} \to \cat{C}$.

\begin{rema}
Note that $\varnothing$ is an initial object in $\cat{Top}(X)$.
If $U \subset X$ is open then $\varnothing \subset U$ so we get a morphism 
$\sh{F}(U) \to \sh{F}(\varnothing)$.
If $C$ has a terminal object this suggests that $\sh{F}(\varnothing)$ should be
that object, but the axioms do not require this.
Generally there should exist functors that do not take terminal objects to
terminal objects, so this may not be true for arbitrary presheaves, but is
often true for the sheaves that come up naturally.
\end{rema}


\begin{exam}
Let $\cat{C}$ be a category and $C$ some object in it.
Then $\sh{F}(U) = C$ with all morphisms the identity is a presheaf on $X$, called
a constant presheaf.
\end{exam}

\begin{exam}
On a topological space we have the presheaf of all real-valued functions, or
maybe more reasonably, the presheaf $\cc{C}$ of all continuous functions.

On a smooth manifold we have the presheaf $\cc{C}^\infty$ of smooth functions.
On a complex manifold the presheaf $\cc O$ of holomorphic functions.

Similarly we get the presheaf of sections of a vector (or fiber, or principal)
bundle over a space.
\end{exam}


\paragraph{Sheaves}

Continuous functions are determined by their local behavior, and can be glued
together to form a function defined on a larger space.
A sheaf captures this behavior.


\begin{defi}
A \emph{sheaf} $\sh{F}$ on a space $X$ is a presheaf that satisfies:
\begin{enumerate}
\item
If $(U_\alpha)$ is an open covering of $U$ and $s, t \in \sh{F}(U)$ are such that
$s_{|U_\alpha} = t_{|U_\alpha}$ for all $\alpha$, then $s = t$.

\item
If $(U_\alpha)$ is an open covering of $U$ and $s_\alpha \in \sh{F}(U_\alpha)$
are such that $\rho_{U_\beta U_\alpha}s_{\alpha} = \rho_{U_\alpha U_\beta}
s_\beta$ for all $\alpha,\beta$, then there exists $s \in \sh{F}(U)$ such that
$s_\alpha = \rho_{U_\alpha U} s$ for all $\alpha$.
\end{enumerate}
\end{defi}


\begin{exam}
Presheaves of functions are almost always sheaves.
This is the case for presheaves of continuous, smooth, and holomorphic
functions on a space where those make sense.
\end{exam}

\begin{exam}
Let $\sh{F}$ be the presheaf of bounded continuous functions on $\RR$.
This is not a sheaf:
Let $U_n = (-n, n)$ for $n > 0$ be an open covering of $\RR$ and let
$s_n = \id_{U_n}$.
These elements do not glue to a section of $\sh{F}$ on $\RR$, as the result is
not bounded.
\end{exam}


\paragraph{Morphisms}

\begin{defi}
A morphism $f : \sh{F} \to \sh{G}$ of presheaves is
a morphism $f_U : \sh{F}(U) \to \sh{G}(U)$
for every open set $U$, such that
for every open $V \subset U$ the diagram
$$
\begin{tikzcd}
\sh{F}(V) \ar[d] \ar[r,"f_V"] & \sh{G}(V) \ar[d]
\\
\sh{F}(U) \ar[r,"f_U"] & \sh{G}(U)
\end{tikzcd}
$$
commutes.
A morphism of sheaves is a morphism of the underlying presheaves.
\end{defi}

Equivalently, a morphism $f : \sh{F} \to \sh{G}$ is a natural transformation of
the corresponding functors $\sh{F} : \topop{X} \to \cat{C}$ and $\sh{G} : \topop{X}
\to \cat{D}$.
A common source of sheaf morphisms is continous maps between topological spaces.


\begin{defi}
Let $f : X \to Y$ be a continuous map between topological spaces.
Let $\sh{F}$ be a presheaf over $X$.
The \emph{direct image} of $\sh{F}$ is the presheaf over $Y$ defined as
$f_*\sh{F}(U) = \sh{F}(f^{-1}(U))$ for open sets $U \subset Y$.
\end{defi}

If $f : X \to Y$ is a continuous map, it induces a functor $f^{-1} : \topop{Y}
\to \topop{X}$ by $U \mapsto f^{-1}(U)$.
The direct image of a sheaf $\sh{F} : \topop{X} \to \cat{C}$ is the composition
$\sh{F} f^{-1} : \topop{Y} \to \cat{C}$.


\begin{prop}
If $\sh{F}$ is a sheaf over $X$, and $f : X \to Y$ is continuous, then
$f_*\sh{F}$ is a sheaf over $Y$.
\end{prop}

\begin{proof}
Let $(U_\alpha)$ be a covering of $U \subset Y$.

Let $s, t$ be sections of $f_*\sh{F}(U)$ such that $s_\alpha = t_\alpha$ for
all $\alpha$. The same is then true for the covering $(f^{-1}(U_\alpha))$ of
$f^{-1}(U)$. As $\sh{F}$ is a sheaf we then have $s = t$ in $\sh{F}(f^{-1}(U))
= f_*\sh{F}(U)$.

Now let $s_\alpha$ be sections of $f_*\sh{F}(U_\alpha) = \sh{F}(f^{-1}(U_\alpha))$
such that $s_\alpha = s_\beta$ on $U_\alpha \cap U_\beta$.
Then there exists a section $s$ of $\sh{F}(f^{-1}(U)) = f_*\sh{F}(U)$ such that
$s_{|U_\alpha} = s_\alpha$ for all $\alpha$.
\end{proof}

\begin{prop}
If $f : X \to Y$ is continuous
and $\phi : \sh{F} \to \sh{G}$ is a morphism of presheaves over $X$,
then there is an induced morphism $f_*\phi : f_*\sh{F} \to f_*\sh{G}$.
\end{prop}


We'd like to be able to move sheaves in the other direction as well; given a
sheaf on $Y$ we'd like to produce one on $X$.
The naive definition of $f^{-1}\sh{F}(U) = \sh{F}(f(U))$ doesn't work as $f(U)$
may not be open.
To work around this we need to take direct limits over open sets that
approximate $f(U)$.

\begin{defi}
Let $f : X \to Y$ be a continuous map.
Let $\sh{F}$ be a presheaf on $Y$ with values in a category with direct limits.
The \emph{inverse image} of $\sh{F}$ is the presheaf over $X$ defined as
$$
f^{-1}\sh{F}(U)
= \varinjlim_{f(U) \subset V} \sh{F}(V),
$$
where $V$ are open sets.
\end{defi}


\paragraph{Sheafification}


\begin{theo}
Let $\sh{F}$ be a presheaf on $X$.
There exists a sheaf $\widehat{\sh{F}}$ on $X$ and a presheaf morphism $\sh{F}
\to \widehat{\sh{F}}$ such that for any other sheaf $\sh{G}$ and presheaf
morphism $\sh{F} \to \sh{G}$ there is a unique sheaf morphism 
$\widehat{\sh{F}} \to \sh{G}$ such that 
$$
\begin{tikzcd}
\sh{F} \ar[r] \ar[dr] & \widehat{\sh{F}} \ar[d]
\\
             & \sh{G}
\end{tikzcd}
$$
commutes.
\end{theo}

\begin{proof}
TODO. Done in Hartshorne.
\end{proof}

The sheaf $\widehat{\sh{F}}$ is called the sheafification of $\sh{F}$.
If $\sh{F}$ is already a sheaf we of course have $\widehat{\sh{F}} = \sh{F}$.



\paragraph{Kernels and friends}

When we're working in with sheaves of objects of a good enough category (that
is, an Abelian category, which we will almost always do as we care about
geometry) we get the usual algebraic objects from morphisms:

\begin{defi}
Let $\sh{F}$ and $\sh{G}$ be a sheaves of objects in Abelian categories and let
$f : \sh F \to \sh G$ be a morphism.
\begin{itemize}
\item
$\Ker f$ is the sheaf associated to the presheaf $U \mapsto \Ker f_U$.

\item
$\Im f$ is the sheaf associated to the presheaf $U \mapsto \Im f_U$.

\item
$\Coker f$ is the sheaf associated to the presheaf $U \mapsto \Coker f_U$.
\end{itemize}
\end{defi}

This doesn't make sense right now because we haven't said what the ``sheaf
associated to a presheaf'' is.
We'll talk about that when we get to the sheafification of a presheaf.
For now, it exists and is unique enough.

This is necessary because while the kernel of a morphism of sheaves is again a
sheaf, the image and cokernel of a morphism may fail to be sheaves:

\begin{exam}
Consider the sheaf $\cc O$ of holomorphic functions and the sheaf
$\cc O^*$ of nowhere-vanishing holomorphic functions on $\CC \setminus \{0\}$.
There is a sheaf morphism $\exp : \cc O \to \cc O^*$.
If we pick an open cover $(U_\alpha)$ of $\CC \setminus \{0\}$ made up of
simply connected sets then we can find local sections $s_\alpha$ such that
$\exp s_\alpha = \id$.
However, there is no section $s$ of $\cc O$ such that $\exp s = \id$ on all of
$\CC \setminus \{0\}$ so the local identity sections of $\Im \exp$ fail to glue
to a global section.
\end{exam}





\bibliographystyle{plain}
\bibliography{main}

\end{document}
