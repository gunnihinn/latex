\documentclass[11pt]{article}

\usepackage{tgpagella}
\linespread{1.1}
\usepackage[utf8]{inputenc}
\usepackage[T1]{fontenc}

\usepackage[normalem]{ulem}
\usepackage{textcomp}
\usepackage{hyperref}

\usepackage{amsmath}
\usepackage{amssymb}
\usepackage{amsthm}

\usepackage{tikz-cd}

\newtheorem{theo}{Theorem}
\newtheorem{prop}[theo]{Proposition}
\newtheorem{lemm}[theo]{Lemma}
\newtheorem{coro}[theo]{Corollary}
\theoremstyle{definition}
\newtheorem{defi}[theo]{Definition}
\newtheorem{exam}[theo]{Example}

\newcommand{\kk}[1]{\mathbb{#1}}
\newcommand{\cc}[1]{\mathcal{#1}}

\def\^#1{^{[#1]}}
\def\qandq{\quad\text{and}\quad}
\def\ov#1{\overline{#1}}

\DeclareMathOperator{\pr}{pr}
\DeclareMathOperator{\Span}{span}
\DeclareMathOperator{\Gr}{Gr}
\DeclareMathOperator{\GL}{GL}
\DeclareMathOperator{\im}{Im}
\DeclareMathOperator{\Vol}{Vol}
\DeclareMathOperator{\Ker}{Ker}
\DeclareMathOperator{\End}{End}
\DeclareMathOperator{\Aut}{Aut}
\DeclareMathOperator{\Hom}{Hom}
\DeclareMathOperator{\Sym}{Sym}
\DeclareMathOperator{\id}{id}
\DeclareMathOperator{\tr}{tr}
\DeclareMathOperator{\adj}{adj}
\DeclareMathOperator{\Rm}{Rm}

\newcommand{\ext}[1]{\bigwedge{}^{\!\!#1}\,}

\def\<{\langle}
\def\>{\rangle}

\newtheorem{question}{Question}

\author{Gunnar Þór Magnússon}
\date{\today}
\title{Blog}

\begin{document}

\maketitle

\section{11. May 2023}

I don't think the blowup calculation is quite as simple as below.
It doesn't feel right that we can pick coordinates on the line
arbitrarily once we've picked them on the plane.
We probably have to calculate the curvature at $(0,0,[y_0:y_1])$.
We have to split into two cases depending on the standard charts.

When $y_1 \not=0$ we look at
$$
X = \{(x_1,x_2,x_3) \in \kk C^3 \mid x_1 - x_2x_3 = 0 \}
$$
and the point we want is $(0,0,z)$.

When $y_0 \not= 0$ we look at
$$
X = \{(x_1,x_2,x_3) \in \kk C^3 \mid x_1x_3 - x_2 = 0 \}
$$
and the point we want is $(0,0,z)$.

Can we center coordinates on a point in $E$?
Such a point is
$$
(0,0,Z) \in X
= \{(x,y,Z) \in \kk C^2 \times \kk P^1 \mid xw - zy = 0 \}.
$$
The group $\Aut \kk P^1$ acts on the product by
$$
f \cdot (x,y,Z) =
\Bigl(x,y, \frac{aZ + b}{cZ + d}
\Bigr)?
$$
If $Z = [z:w]$ and $w \not= 0$ this corresponds to
$$
\frac{az/w + b}{cz/w + d}
= \frac{az + bw}{cz + dw}
$$
and
$$
x(cz + dw) - y(az + bw)
= cxz + dxw - yaz - byw
\not= 0
$$
I think. Like, $(x,y) = (1,1)$ and $[z:w] = [1:1]$ and $a = 2, b = 0, c = 1, d
= 0$ gives
$$
xz - 2yz = -1.
$$
Away from $E$ we have $[z:w] = [x:y]$.
So we must have
$$
f \cdot (x,y,Z) = (ax + yb, cx + yd, f(Z)).
$$
Then
$$
\displaylines{
(ax + by)(cx + dy) - (cx+dy)(ax + by)
\hfill\cr\hfill{}
= ac x^2 + ad xy + bc xy + bd y^2
- ac x^2 - bc xy - ad xy - bd y^2
= 0.
}
$$
So $\Aut \kk P^1$ at least acts on $X \setminus E$
and sends things not in $E$ to things not in $E$.
It should act on all of $X$ by the same equation above.

We're really just using $\operatorname{GL}_2(\kk C)$ to act on $\kk C^2$ and
extending the action to $\kk P^1$, and since $\Aut \kk P^1 =
\operatorname{PGL}_2(\kk C)$ this works.
The upshot being: if we're on $E$ we can pick coordinates to center the point
we're on.

\paragraph{Fewer coordinates}
Let $V$ be a complex vector space of dimension $n > 1$.
The blowup of $V$ at the origin is
$$
X = \{ (v, [w]) \in V \times \kk P(V) \mid v \in \kk C w \}
$$
along with the map $\mu : X \to V$ induced by the projection onto the first
factor. The exceptional divisor is $E := \mu^{-1}(0)$.

The group $\GL V$ acts on $X$ by $f \cdot (v, \ell) = (f(v), f(\ell))$.
This action is in fact fixed-point free and transitive and maps $E$ to $E$.
Therefore there exist coordinates centered on any point in $E$; that is,
local coordinates $(v,w) \in V \times W$ for $V \times \kk P(V)$ such that any
point $p \in E$ is $(0,0)$.

Really we should take the coordinates on $V$ such that they are normal at $0$
and construct coordinates for the projective space from them.
Then appeal to the above to say that the coordinates we pick can center the
point we care about.
(Does that actually work? Does that preserve normalness?
Maybe? The unitary group acts transitively on the sphere, therefore on the
projective space, so we should be fine.)



\section{10. May 2023}

I got a rejection letter yesterday with a referee report that pointed out
something fun: If $g$ and $h$ are K\"ahler metrics we can consider the dual of
the sum of their duals: $(g^\dagger + h^\dagger)^\dagger$.
If the metrics have positive hsc (or positive holomorphic bisectional curvature)
this will too, basically because of Codazzi--Griffiths.
However, this metric won't generally be K\"ahler.

The upshot is that Hermitian metrics of positive hsc form a connected cone on a
manifold, like metrics of negative hsc.

\paragraph{Blowing up the plane}

Let $S$ be a compact K\"ahler surface and $p \in S$,
and let $\mu : X \to S$ be the blowup of $S$ at $p$.
We claim that if $h_S$ is a K\"ahler metric of positive hsc on $S$,
then there exists a K\"ahler metric $h_X$ of positive hsc on~$X$.

Let $E \subset X$ be the exceptional divisor.
Voisin says there exists a line bundle $L \to X$ such that $L$ is trivial outside
of $E$ and $L_{|E} = \cc O_E(1)$.
From this we get that there exists a Hermitian form $b$ on $X$ such that
$b_{|E}$ is the Fubini--Study metric.

We consider a small ball $B \subset \kk C^2$ around $p$ in $S$,
and choose its coordinates such that they are normal at the center.
Over this ball the blowup~is%
$$
X = \{ (x_1, x_2, [y_0: y_1]) \in B \times \kk P^1
\mid x_1 y_1 - x_2 y_0 = 0 \}.
$$
We pick the ball small enough that $L = p_2^*\cc O_{\kk P^1}(1)$,
and then $b = p_2^* h_{FS}$.
We consider the product metric $e^t p_1^*h_S \oplus b$ on $B \times \kk P^1$
and let $h_t$ be its restriction to $X$.
Our claim is that $h_t$ is a K\"ahler metric on $X$ with positive hsc for all
$t$ large enough.

Outside of $B$ there is no problem:
The space $X \setminus B$ is compact and we can test positivity of $h_t$ on a
compact sphere bundle, so it'll be positive for $t \gg 0$.
We can further show that if $h$ has positive hsc and $b$ is a Hermitian form
then $e^t h + b$ has positive hsc for $t \gg 0$.

On $B$ we note that $h_t$ is a K\"ahler metric for all $t$, because the direct
sum is a K\"ahler metric that we restrict to a smooth submanifold.

We also note that it is enough to prove that $h_t$ has positive hsc on $E$,
because the function $x \mapsto \inf_{T_{X,x}} H_{h_t}$ is continuous
and so $h_t$ will then have positive hsc on a small neighborhood around $E$.
By shrinking $B$ we can then conclude.

Let then $q \in X$ be a point on $E$, and pick coordinates $x_3$ on $\kk P^1$
centered on the corresponding point on the projective line (since $q$ is on $E$
its $(x_1,x_2)$ coordinates are zero).
Around this point the blowup is then described by
$$
X = \{(x_1,x_2,x_3) \in \kk C^3 \mid x_1 - x_2x_3 = 0 \}
$$
(or $x_1x_3 - x_2 = 0$ but that case is the same).
Its tangent space is
$$
T_X = \ker(dx_1 - x_3 dx_2 - x_2 dx_1)
=: \ker \alpha
\subset T_{\kk C^3|X}.
$$
Let $e_j = \partial / \partial x_j$ be the coordinate tangent fields.
By assumption they are normal at the center.
We note that $T_X$ is spanned by
$$
u = x_3 e_1 + e_2
\qandq
v = x_2 e_1 + e_3.
$$
The normal bundle to $X$ at the center is spanned by $e_1$.
The second fundamental form of $X$ at the center is then
$$
\sigma(\xi)\xi
= \alpha(D_\xi \xi) / |e_1|^2_{h_t}
= e^{-t} dx_1(D_\xi \xi)
$$
and the curvature of $h_t$ is
$$
R_{h_t}(\xi, \ov\xi, \xi, \ov\xi)
= e^{-t} (d\pi_{12}^* R_{g})(\xi, \ov\xi, \xi, \ov\xi)
+ 2 |d\pi_{3} \, \xi|^4
- e^{-2t} |dx_1(D_\xi \xi)|^2
$$
because the metric on the projective line has constant holomorphic sectional
curvature $2$.
Here $\pi_{12}: \kk C^3 \to \kk C^3$ is the projection onto the first two factors
and $\pi_3$ the projection onto the third.

Write $\xi = a u + b v = (x_3a + x_2b) e_1 + a e_2 + b e_3$ for some
holomorphic functions $a$ and $b$. As we're testing for positivity we may
assume that $|a|^2 + |b|^2 = 1$ at the origin.
We're also going to apply $dx$ so we only care about the $e_1$ factors.
We have
\begin{align*}
dx_1(D_{g,\xi}\xi)
&= (x_3a + x_2b) d_{e_1}(x_3 a + x_2b)
+ a d_{e_2}(x_3 a + x_2b)
\\
&= z^2 a d_{e_1}a + a z d_{e_2} a + ab
\\
dx_1(D_{h,\xi}\xi)
&= b D_{h,e_3}((x_3a + x_2b) e_1 + a e_2 + b e_3)
= ab
\end{align*}
at the origin, and so
$\alpha(D_{\xi} \xi) = 2 ab + az( z d_{e_1}a + d_{e_2} a)$
there. Then
$$
|\sigma(\xi)\xi|^2
= e^{-2t} |2ab + az( z d_{e_1}a + d_{e_2} a)|^2
$$
and I don't think we can control either $z$ or the derivatives of $a$.


\section{9. May 2023}

The more I think about blowups the less I understand.
Like, for example. If we blow up $Y \subset X$ we get $\mu : \widehat X \to X$
and and exceptional divisor $E = \mu^{-1}(Y)$ that is isomorphic to $\kk
P(N_{Y/X}) \to Y$.
It's also a hypersurface in $\widehat X$, so there is one tangent direction out
of $E$ in $\widehat X$.
What on earth does that correspond to in $X$, since we should already have the
directions tangent to $Y$ and every direction out of it?

What even is the tangent bundle of $\widehat X$?
Apparently there are short exact sequences
$$
0 \to \mu^*\Omega^1_X \to \Omega^1_{\widehat X} \to j_* \Omega^1_{E/Y} \to 0
$$
and
$$
0 \to T_{\widehat X} \to \mu^* T_X \to j_*T_{E/Y}(E) \to 0
$$
where $j : E \hookrightarrow \widehat X$ is the inclusion.
These are sequences of coherent sheaves, but I still can't figure out how their
rank adds up over $E$.
All the objects are smooth varieties, so $T_{\widehat X}$ and $\mu^* T_X$ are
both honest vector bundles over all of $\widehat X$, \emph{of the same rank}.


\paragraph{Example}

Let's blow up the origin in $\kk C^2$.
This is the subspace
$$
X = \{(x,y,[z:w]) \in \kk C^2 \times \kk P^1 \mid xw - yz = 0 \}
$$
along with the map $\mu : X \to \kk C^2$ induced by the projection.
Let's also pick the chart $w \not= 0$ of $\kk P^1$, in which this becomes
$$
X = \{(x,y,z) \in \kk C^3 \mid x - yz = 0 \}
$$
and $\mu(x,y,z) = (x,y)$.
The tangent space of $X$ is
$$
T_X = \ker (dx - z dy - y dz) \subset T_{\kk C^3}
$$
and is spanned by the fields $u = (z,1,0)$ and $v = (y,0,1)$ at all points of $X$.

The differential of $\mu$ is
$$
\mu_* = \begin{pmatrix}
1 & 0 & 0
\\
0 & 1 & 0
\end{pmatrix}
$$
and maps the basis of $T_X$ to $\mu_* u = (z,1)$ and $\mu_* v = (y,0)$.
This is an isomorphism when $y \not= 0$, that is, outside of the exceptional
divisor $E = \{(0,0,z) \mid z \in \kk C \}$ in $X$.
On $E$ we have $\mu_* v = 0$.

We have a short exact sequence
$$
0 \to T_X \to T_{\kk C^3} \to N_{X/\kk C^3} \to 0
$$
over $X$.
The map to the normal bundle (which is trivial) is just $\alpha = dx - zdy - ydz$.
Let $g$ and $h$ be metrics on $\kk C^2$ and $\kk P^1$ with Chern connections
$D_g$ and $D_h$.
The second fundamental form of $X$ is
$$
\sigma(\xi, \eta)
= \alpha((D_g \oplus D_h)_\xi \eta).
$$
The matrix of the product metric on $\kk C^3$ is
$$
g \oplus h
= \begin{pmatrix}
a & b & 0
\\
\ov b & c & 0
\\
0 & 0 & h
\end{pmatrix}.
$$
The inverse matrix is
$$
g \oplus h
= g^{-1} \oplus 1/h.
$$
The norm of $\alpha$ is then
$$
|\alpha|^2
= |(1,-z)|^2_{g^{-1}} + |y|^2/h.
$$

If $w = (w_1, w_2, w_3)$ is a tangent field we have
\begin{align*}
\<w, \ov u \>
= (\bar z, 1, 0) \cdot g \cdot (w_1, w_2, w_3)
&= (\bar z, 1, 0) \cdot (a w_1 + b w_2, \bar b w_1 + c w_2, h w_3)
\\
&= a w_1 \bar z + b w_2 \bar z + \bar b w_1 + c w_2
\\
&= (a \bar z + \bar b) w_1 + (b \bar z + c) w_2
\end{align*}
and
$$
\< w, \ov v \>
= (\bar y, 0, 1) \cdot (a w_1 + b w_2, \bar b w_1 + c w_2, h w_3)
= \bar y a w_1 + \bar y b w_2 + h w_3.
$$
So we need $w$ orthogonal to $(a z + b, \bar b z + c, 0)$ and $(y a, y
\bar b, h)$ in the standard metric. Take the cross product and get
$$
w
= \begin{pmatrix}
(\bar b z + c)h
\\
- (a z + b) h
\\
(a z + b)y\bar b - (\bar b z + c)ya
\end{pmatrix}
= \begin{pmatrix}
(\bar b z + c)h
\\
- (a z + b) h
\\
(|b|^2 - ac)y
\end{pmatrix}.
$$
This is not holomorphic because $a,b,c,h$ are smooth functions of $(x,y,z)$.
But it does let us calculate the norm on the normal bundle.
Actually this is just
$$
w = (g^{-1} \oplus 1/h)(1, -\bar z, -\bar y),
$$
so $w$ is the unique field such that $(g \oplus h)(w, \ov \xi) = \ov{\alpha(\xi)}$ for all $\xi$.

The projection onto $T_X^\perp \subset T_{\kk C^3}$ is then
$$
\xi \mapsto \frac{\< \xi, \ov w \>}{|w|^2} w
$$
and the norm of the projection is $|\<\xi, \ov w \>|^2/|w|^2$.

And then we can calculate the contribution of the second fundamental form of $X$
to the holomorphic sectional curvature; it is
$$
-\frac{|\< (D_g \oplus D_h)_\xi \xi, \ov w \>_{g \oplus h}|^2}{|w|^2_{g\oplus h}}
= -\frac{|\alpha((D_g \oplus D_h)_\xi \xi)|^2}{|w|^2_{g\oplus h}}.
$$
where $w$ is as above.

We consider the metrics $e^t g \oplus h$.
We have $|w|^2_{e^tg \oplus h} = |\alpha|^2_{(e^tg \oplus h)^{-1}}$
and
$$
(e^tg \oplus h)^{-1}
= e^{-t} g^{-1} \oplus 1/h
\to 0 \oplus 1/h
$$
when $t \to \infty$.
Then we should have
$$
|w|^2_{e^tg \oplus h}
\to |y|^2/h
$$
as $t \to \infty$, and then
$$
-\frac{|\< (D_g \oplus D_h)_\xi \xi, \ov w \>_{g \oplus h}|^2}{|w|^2_{g\oplus h}}
\to -h\frac{|\alpha((D_g \oplus D_h)_\xi \xi)|^2}{|y|^2}
$$
as $t \to \infty$.

We of course have
$$
|\alpha((D_g \oplus D_h)_\xi \xi)|^2
\leq |\alpha|^2 |(D_g \oplus D_h)_\xi \xi|^2
$$
because of inner products.
Then we get
$$
|\sigma(\xi, \xi)|^2
\leq |(D_g \oplus D_h)_\xi \xi|^2
$$
which will head to $\infty$ with $t$.

Let
\begin{align*}
\xi = \theta u + \psi v
&= \theta (z, 1, 0) + \psi (y, 0, 1)
\\
&= (z \theta + y \psi, \theta, 0) + (0, 0, \psi)
=: \xi' + \xi'',
\end{align*}
where $\theta$ and $\psi$ are functions on $X$.
Then
$$
(D_g + D_h)_\xi \xi
= D_{g,\xi'}\xi' + D_{h,\xi''} \xi''.
$$
We have
$$
D_{h,\xi''} \xi''
= \psi d_{e_3} \psi + \psi^2 D_{h,e_3} e_3.
$$
Just generally, if $\xi$ and $\eta$ are such that $D_\xi \xi = 0$ at a point $x$
then
$$
D_{f \xi} g \eta = f d_\xi g \, \eta + fg D_\xi \eta = f d_\xi g \, \eta
$$
at $x$.
Suppose we pick coordinates $(x_1,x_2,x_3)$ such that $(e_j = \partial
/ \partial x_j)$ is a normal frame at $0$. Then
$$
D_{h,\xi''} \xi''
= \psi d_{e_3} \psi
$$
at the center, and
\begin{align*}
D_{\xi'} \xi'
&= D_{(z\theta + y\psi)e_1 + \theta e_2}((z\theta + y\psi)e_1 + \theta e_2)
\\
&= D_{(z\theta + y\psi)e_1 + \theta e_2}(z\theta + y\psi)e_1
+ D_{(z\theta + y\psi)e_1 + \theta e_2}\theta e_2
\\
&= \bigl(
(z\theta + y\psi)d_{e_1}(z\theta + y\psi)
+ \theta d_{e_2}(z\theta + y\psi)
\bigr) e_1
\\
&\qquad
+ \bigl(
(z\theta + y\psi)d_{e_1}\theta
+ \theta d_{e_2}\theta
\bigr) e_2
\\
&=
\theta \psi \, e_1
+ \theta d_{e_2}\theta
\, e_2
\end{align*}
at the center.
Also there we have
$$
\alpha((D_{e^t g} \oplus D_h)_\xi \xi)
= dx(\theta \psi \, e_1
+ \theta d_{e_2}\theta
\, e_2)
= \theta(0) \psi(0)
$$
and $|w|^2 = |\alpha|^2 = e^{-t}$.
It looks like we end up with
\begin{align*}
R(\xi, \ov\xi, \xi, \ov\xi)
&= e^t R_g(\xi', \ov{\xi'}, \xi', \ov{\xi'})
+ R_h(\xi'', \ov{\xi''}, \xi'', \ov{\xi''})
- e^t |\theta(0) \psi(0)|^2
\\
&= e^t R_g(\xi', \ov{\xi'}, \xi', \ov{\xi'})
+ 2 |\psi(0)|^4
- e^t |\theta(0) \psi(0)|^2.
\end{align*}
If $\psi(0) = 0$ then $\theta(0) \not= 0$
so $\xi'' = 0$ and $\xi' = (0, \theta, 0)$ at $0$, so
$$
R(\xi, \ov\xi, \xi, \ov\xi)
= e^t |\theta(0)|^4 R_g(e_2, \ov{e_2}, e_2, \ov{e_2}).
$$
We are checking positivity on a compact ball in $T_{X,0}$, so if $\psi(0) = 0$
there is some minimum $m > 0$ such that $|\theta(0)| \geq m$. Then
$$
R(\xi, \ov\xi, \xi, \ov\xi)
\geq e^{-t} m^4 c > 0,
$$
where $c > 0$ is the infimum of $H_g$ over the unit ball.

If $\psi(0) \not= 0$ we get
\begin{align*}
R(\xi, \ov\xi, \xi, \ov\xi)
&= e^t |\theta(0)|^4 R_g(e_2, \ov{e_2}, e_2, \ov{e_2})
+ 2 |\psi(0)|^4
- e^t |\theta(0) \psi(0)|^2
\\
&\geq
e^{-t} |\theta(0)|^4 c
+ 2 |\psi(0)|^4
- e^t |\theta(0) \psi(0)|^2.
\end{align*}
Since the RHS goes to $\infty$ for any fixed $\theta$ and $\psi$ as $t \to
-\infty$ and we're checking positivity on a compact set, there is some $t_0$
so that this is positive for all $\theta$ and $\psi$ at $0$.

That is, at a given point $x \in X$, we can find $t_x$ so that $e^t \mu^* g + h$
has positive hsc at $x$ for $t \geq t_x$.
Since the point is arbitrary and $t_x$ varies continuously with $x$ and the
underlying manifold is compact, we can do this on the whole of the blowup.



\section{8. May 2023}

We know that if $X$ has positive hsc then $\kk P(E) \to X$ also does.
Can we find an example of $X$ that doesn't have positive hsc and where $\kk
P(E) \to X$ also doesn't?
Should be enough to take $X$ to be a curve of genus $g \geq 1$.

Somewhat related, suppose that $X$ is compact K\"ahler with positive hsc.
Let $Y \subset X$ be a subvariety that doesn't have positive hsc.
Can $\widehat X_Y$ then have positive hsc?
For example, take $X = \kk P^3$ and $Y = C \subset X$ a curve of genus $g \geq 1$.


\paragraph{Projective bundle}

If $E \to X$ is a vector bundle of rank $r$, then
$$
H^*(\kk P(E))
= H^*(X)[h] / (h^{r} + c_1 h^{r-1} + \cdots + c_{r} ),
$$
where $c_j$ are the Chern classes of $E$ and $h = c_1(\cc O_{\kk P(E)}(1))$.

We have the Euler sequence
$$
0
\to \cc O_{\kk P(E)}
\to p^*E \otimes \cc O_{\kk P(E)}(1)
\to T_{\kk P(E)/X}
\to 0
$$
and the sequence
$$
0
\to T_{\kk P(E)/X}
\to T_{\kk P(E)}
\to p^* T_X
\to 0.
$$
The total Chern class $c$ satisfies $c(E) = c(S) c(Q)$ for an exact sequence $0
\to S \to E \to Q \to 0$.
We get
$$
c(T_{\kk P(E)}) = c(T_{\kk P(E)/X}) \cdot p^* c(T_X)
$$
and
$$
c(p^*E \otimes \cc O_{\kk P(E)}(1))
= c(T_{\kk P(E)/X}).
$$
For the first Chern class this gives
$$
c_1(T_{\kk P(E)})
= c_1(T_{\kk P(E)/X}) + p^* c_1(X)
$$
and
$$
p^*c_1(E) + h
= c_1(T_{\kk P(E)/X})
$$
so
$$
c_1(T_{\kk P(E)})
= p^*c_1(E) + h + p^* c_1(X).
$$
The $(1,1)$-classes on $\kk P(E)$ are $p^* \omega + \lambda h$, where $\omega$
is a $(1,1)$-class on $X$.

Suppose now that $X$ is a curve of genus $g$. Then we have
\begin{align*}
c_1(T_{\kk P(E)}) \cdot (p^* \omega + \lambda h)
&= (p^*c_1(E) + h + p^* c_1(X))(p^* \omega + \lambda h)
\\
&= \lambda c_1(E) + \lambda c_1(X) + \omega
\\
&= \lambda c_1(E) + \lambda (1-g) + \omega.
\end{align*}
We can pick $\omega$ of volume one, $\lambda = 1$ and $E$ flat.
Then this equals $-g$ which is negative as soon as $g > 1$.
Then $\kk P(E) \to X$ doesn't have a metric of positive hsc.



\paragraph{Thoughts on lifts}

Let $\pi : X \to B$ be a family of compact manifolds.
We suppose we have a K\"ahler metric $h_B$ on $B$ and a family of K\"ahler
metrics $h_{X/B}$ on the fibers of $\pi$.
To get a Hermitian metric $h$ on $X$ we need a splitting $T_X = T_{X/B} \oplus
\pi^* T_B$.

If the family $h_{X/B}$ extends to a metric $h$ on $X$ we get this splitting in
the usual way.
Here it's easy to see when the resulting metric is K\"ahler.

Another way is to fabricate smooth lifts $\beta : \pi^*T_B \to T_X$.
Then we split a section $\xi$ of $T_X$ as
$$
\xi \mapsto (\xi - \beta \pi_* \xi) \oplus \pi_* \xi.
$$
When is the metric we get this way K\"ahler?
Can we calculate its curvature tensor?

In the case we care about we expect to have nonzero holomorphic tangent fields
on the fibers. That means the lift won't be unique.
Or even well-defined?


\section{7. May 2023}

\'Alvarez--Heier--Zheng mention that it should be true that if $\pi : X \to B$
is a holomorphic family with $X$ compact, and if the base and fibers have
metrics with positive hsc, then $X$ should have a metric with positive hsc.
I assume they mean that $X \to B$ should be locally trivial, and thus have a
single fiber~$F$?
(No: Chaturvedi and Heier prove this for an honest family.
They can't produce a K\"ahler metric on the total space though.)

Suppose we have a locally trivial family $\pi : X \to B$ with fiber $F$ and
everything compact.
Let $h_F$ and $h_B$ be K\"ahler metrics on $F$ and $B$.
Can we use Siu's harmonic lifts to split $T_X$?

In this case the Kodaira--Spencer morphism $\rho = 0$
so we'd be trying to lift a section $\xi$ of $T_B$ to a section $\hat \xi$ of
$T_X$ such that $[\partial \hat\xi_{|F}] = 0$
and so that $\hat \xi_{|F}$ is harmonic with respect to $h_F$.

\section{5. May 2023}

We can use Wu's method to prove that $h_t := e^t h + b$ has positive (negative)
hsc for $t \gg 0$ at a point if $h$ does so and $b$ is a Hermitian form:
Pick $f : D \to X$ that realizes the hsc of $h$ at a point, then the
contributions of $\nabla_\xi \xi$ to $R_{h}$ are zero.
Use Taylor expansion to show that then
$$
R_{h_t}
\geq R_{h_t} - \| \nabla \xi \|^2
= - \partial^2 \log \|\xi\|^2
= e^t R_h + O(1).
$$

I'm not sure we can use the same to prove that $b + h$ has positive hsc at a
point if $b = 0$ at that point and $h$ has positive hsc.
Might need a version of that weird lemma for that.
Also not sure that is true as written.


\begin{prop}
Let $X$ be a complex manifold.
Let $h$ be a Hermitian metric on $X$ that has positive holomorphic sectional curvature at a point $x$.
Let $b$ be a Hermitian form on $X$.
Then $h_t := e^t h + b$ is a Hermitian metric and has positive holomorphic
sectional curvature at $x$ for all large $t$.
\end{prop}

\begin{proof}
We are going to use Wu's characterization of the holomorphic sectional
curvature (hsc) of a metric.
Recall that if $L \subset E$ is a sub-line bundle in a vector bundle with a
metric $h_E$, then the curvature of the induced metric $h_L$ is
$$
R_L = R_{E|L} - \sigma^* h_{E/L},
$$
where $h_{E/L}$ is the induced metric on the quotient bundle,
$\sigma(s, \xi) = q(D_{E,\xi}s)$ is the second fundamental form,
and $q : E \to E/L$ is the quotient map.
Wu shows that equality $R_L = R_E$ can be attained at a point for some
holomorphic embedding $f : D \to T_X$.
If $\xi = f_*(\partial / \partial z)$ is a nowhere zero local section of $f(T_D)
\subset T_X$ then the orthogonal projection onto $f(T_D)^\perp$ is
$$
\eta \mapsto \eta - \frac{\< \eta, \ov\xi \>}{|\xi|^2} \xi.
$$
Equality above is thus attained exactly when
$D_{E,\xi}\xi = \lambda \xi$ at the point.

First note that $h_t$ is clearly a Hermitian metric at $x$ for all $t$ large
enough.
Then let $f : D \to X$ be such that it realizes the hsc of $h$ at $x = f(0)$
and let $\xi = f_*(\partial / \partial z)$.
We have
$$
\Bigl|
\frac{\partial}{\partial z}
\Bigr|_{f^*h_t}^2
= | \xi |^2_{h_t}
= e^t | \xi |^2_h + b(\xi, \ov \xi).
$$
The curvature of $f^*h_t$ at $0$ is then
$$
\displaylines{
- \frac{\partial^2}{\partial z \partial \bar z}
\log \Bigl|
\frac{\partial}{\partial z}
\Bigr|_{f^*h_t}^2
= e^t \frac{R_h(\xi, \ov\xi, \xi, \ov\xi)}{|\xi|_h^2}
- e^t \frac{h(D_\xi \xi, \ov{D_\xi \xi})}{|\xi|_h^2}
\hfill\cr\hfill{}
+ e^t \frac{h(D_\xi \xi, \ov \xi) h(\xi, \ov{D_\xi \xi})}{|\xi|_h^4}
- \frac{\partial}{\partial z}
\frac{\frac{\partial}{\partial \bar z} b(\xi, \ov\xi)}
{e^t |\xi|^2_h + b(\xi,\ov\xi)}
\cr{}
\hphantom{
- \frac{\partial^2}{\partial z \partial \bar z}
\log \Bigl|
\frac{\partial}{\partial z}
\Bigr|_{f^*h_t}^2
}
= e^t \frac{R_h(\xi, \ov\xi, \xi, \ov\xi)}{|\xi|_h^2}
+ O(1)
\hfill
}
$$
because $D_\xi \xi = \lambda \xi$ at $x = f(0)$.
By hypothesis $R_h(\xi, \ov\xi, \xi, \ov\xi) > 0$ so the curvature is positive
at $0$ for $t$ large enough.
By Wu the hsc of $h_t$ at $x$ is the supremum over all $f : D \to T_X$ as above,
which is then positive.
\end{proof}


\begin{exam}
Consider the disc $D$ and $h = e^{a |z|^2}$.
The curvature of $h$ at $0$ is
$$
- \frac{\partial^2}{\partial z \partial \bar z} a |z|^2
= -a
$$
so it is positive if $a < 0$.
Let $b(z) = t |z|^2$ so $b(0) = 0$ and consider $h + b$.
For small $|z|$ this is still a metric on $D$ because $h(0) = 1$.
Its curvature is
\begin{align*}
- \frac{\partial^2}{\partial z \partial \bar z}\log ( e^{a |z|^2} + t|z|^2 )
&= -\frac{\partial}{\partial z} \frac{az e^{a|z|^2} + t z}{ e^{a |z|^2} + t|z|^2 }
\\
&= -\frac{a e^{a|z|^2} + a^2 |z|^2 e^{a|z|^2} + t}{ e^{a |z|^2} + t|z|^2 }
+ \frac{|a z e^{a|z|^2} + t z|^2}{(e^{a |z|^2} + t|z|^2)^2}.
\end{align*}
At $0$ this is equal to $-a-t$, so if $t > -a$ the curvature of $h + b$ is
negative at~$0$.%
\end{exam}


Maybe the example suggests we can't use
Wu to prove that a blowup has positive hsc?
The old papers estimate $R$ on a direct sum and conclude from there, which
isn't really something we can do with Wu.


\paragraph{Blowing up the plane}

The blowup of the origin in $\kk C^2$ is the submanifold
$$
X
= \{ y z = x w \mid (x, y) \in \kk C^2, \ [z,w] \in \kk P^1 \}
\subset \kk C^2 \times \kk P^1
$$
along with the map $\mu : X \to \kk C^2$ induced by the projection.
There is a short exact sequence
$$
0 \to T_X \to T_{\kk C^2} \oplus T_{\kk P^1} \to Q \to 0,
$$
where $Q$ is a line bundle.
Let $h_1, h_2$ be the metrics on $\kk C^2$ and $\kk P^1$ and $h_t = e^t h_1 + h_2$.
The norm on $Q$ is
$$
\|x\|_Q = \inf_{q v = x} \| v \|_{h_t}.
$$
Because $X$ is a hypersurface there is a local holomorphic function $f : \kk
C^2 \times \kk P^1 \to \kk C$ such that $X = f^{-1}(0)$ ($f = yz - xw$).
The second fundamental form is
$$
\sigma(\xi, \eta)
= f_*(D_{1,\xi} \eta' + D_{2,\xi} \eta'').
$$
The curvature tensor of $h_t = e^t h_1 + h_2$ restricted to $X$ is
\begin{align*}
R_{h_t}(\xi)
&= e^t R_1(\xi) + R_2(\xi) - \|\sigma(\xi, \xi)\|_Q^2
\\
&\geq e^t R_1(\xi) + R_2(\xi) - \|D_{1,\xi}\xi + D_{2,\xi} \xi\|_{h_t}^2
\\
&= e^t R_1(\xi) + R_2(\xi) -
e^t \|D_{1,\xi}\xi\|_{h_1}^2 - \| D_{2,\xi} \xi\|_{h_2}^2
\end{align*}


\section{4. May 2023}

Locally we write $h = e^f h_{\text{std}}$ for a metric on a curve (note that
Demailly writes $e^{-f}$ which flips the curvature sign).
The curvature of this metric is
$$
R = -\frac{1}{e^f} \frac{\partial^2 f}{\partial z \partial \bar z}.
$$
The curvature is positive at a point if $-f$ is psh at the point and negative
if $f$ is psh.

If $f$ and $g$ are psh and $h_f = e^f$ and $h_g = e^g$ then $h_f + h_g = e^f +
e^g = e^{\log(e^f + e^g)}$.
One of the things Demailly proves in his book in the chapter on psh functions
is that $\log(e^f + e^g)$ is psh if $f$ and $g$ are psh.
This implies Wu's result that the sum of metrics of negative hsc has negative hsc.
It should also imply that a fibration with a negative hsc base and fibers has
negative hsc.

If $-f$ and $-g$ are psh we can say that
$$
\log(e^{-f} + e^{-g})
= \log \frac{e^f + e^g}{e^f e^g}
= \log (e^f + e^g) - f - g
$$
is psh.
This only gives a lower negative bound on the curvature of $h_f + h_g$,
which makes sense because we know the sum of positive hsc metric is not
necessarily positive hsc (see
Section~\ref{sum-of-positive-hsc-is-not-positive-hsc}).

We can split our blowup of a positive hsc manifold into three cases, depending
on the behavior of $f : D \to \widehat X$ and $\xi := f_*\partial / \partial z$:
\begin{enumerate}
\item
$\mu_* \xi \not= 0$ on $f(D)$

\item
$\mu_* \xi = 0$ on $f(D)$

\item
$\mu_* \xi \not= 0$ on $f(D \setminus \{0\})$ but $\mu_* \xi = 0$ at $f(0)$
\end{enumerate}
We've already handled case 1.
In case 2 the only thing we can look at is $b_E$, which we know to be positive there.
That leaves case 3.

Write $f^*\mu^*h_X = e^g h_{\text{std}}$ on $D$, where $g : D \to \kk R \cup
\{+\infty\}$.
Then $-g$ is smooth and psh on $D \setminus \{0\}$, and bounded from above near
$0$, so $-g$ is psh on $D$.
Similarly we can write $f^* b_E = e^{\theta} h_{\text{std}}$ for $\theta$
smooth on $D$ and we may shrink $D$ so that $-\theta$ is psh.
We don't necessarily want to prove that the sum is psh at $0$, but that we can
modify $f$ so it sends $(\partial/\partial z, 0)$ to a multiple of $(\xi, x)$ and
somehow make the pullback by that $f$ psh at $0$.


\paragraph{Prior art}
The papers by Cheung and Chaturvedi (I think?) that prove that fibrations have
negative or positive hsc use the same technique.
They both prove:

(a) If $h_1$ has positive (negative) hsc then $e^t h_1 + h_2$ has positive
(negative) hsc for $t \gg 0$.

(b) If $R$ is a curvature tensor on $V$ and $V = S \oplus Q$ and its hsc is
positive (negative) on $S$ and $Q$ and we have a standard bound on $|R|$, then
under some conditions on the min/max of hsc and $|R|$ the tensor has positive
(negative) hsc on all of $V$.

So going far away enough gives a sign on hsc, and if we can check the sign on a
splitting we may get lucky.
This is basically what we came up with too, so we're in good company.

Their proofs are all coordinates and no hope.
Might be fun to redo in a simpler way.



\section{3. May 2023}

Wu proved that if $(X,h)$ is a Hermitian manifold, we can define the
holomorphic sectional curvature of $X$ at a tangent field $\xi \in T_{X,x}$
$$
H(\xi) = \sup_{f:D\to X}\frac{i}{2\pi}\Theta_{f^*h}(0),
$$
where $f : (D,0) \to (X,x)$ is a holomorphic map such that
$f_*\frac{\partial}{\partial z} = \xi$.

The proof is pretty direct.
Because of Codazzi--Griffiths we have $\text{RHS} \leq H(\xi)$.
Because normal coordinates exist and we can pick (a multiple of) $\xi$ to be
one of the frame elements we can shove a small disc into $X$ in the right way.

Can we use this to prove that a blow-up of a manifold with positive hsc has
positive hsc?

The difficulty there (for me) is working with the global structure of the blowup.
It looks like this should let us only think about the (tangent space of the)
blowup point by point.
Handling a metric that's a sum of two other things should be simpler in
dimension one.
The strategy would be to prove that $h_t = e^t \mu^* h_X + h_{E}$ is positive
for all $t \gg 0$ at every point of $\kk P(T_{\widehat X})$, and conclude by
compactness.

If $h$ is a metric on $D$ we can write $h = e^g \frac i2 dz \wedge d\bar z$.
Now let $h$ be a metric on $X$ of dimension $n$ and let $f : D \to X$ be an embedding.
We get a metric $f^*h$ on $D$.
What is $g$?
Well,
$$
f^*h\Bigl(\frac{\partial}{\partial z},\ov{\frac{\partial}{\partial z}}\Bigr)
= h\Bigl(f_*\frac{\partial}{\partial z}, \ov{f_*\frac{\partial}{\partial z}}\Bigr)
$$
so
$$
g = \log \Bigl\|f_*\frac{\partial}{\partial z}\Bigr\|_h^2.
$$
In our case we have $h_t = e^t \mu^*h_X + b_E$ and
$$
\Bigl\|f_*\frac{\partial}{\partial z}\Bigr\|_{f^*h}^2
= e^t\|\mu_*\xi\|_{h_X}^2 + \|\xi\|^2_{b_E},
$$
where $\xi := f_*\frac{\partial}{\partial z}$.
Note that the latter term can be zero or negative.
We get
$$
\Theta
= \frac{e^t\|\mu_*\xi\|_{h_X,z\bar z}^2 + \|\xi\|^2_{b_E,z \bar z} }
{e^t\|\mu_*\xi\|_{h_X}^2 + \|\xi\|^2_{b_E}}
- \frac{e^t\|\mu_*\xi\|_{h_X,z}^2 + \|\xi\|^2_{b_E,z} }
{e^t\|\mu_*\xi\|_{h_X}^2 + \|\xi\|^2_{b_E}}
\frac{e^t\|\mu_*\xi\|_{h_X,\bar z}^2 + \|\xi\|^2_{b_E, \bar z} }
{ e^t\|\mu_*\xi\|_{h_X}^2 + \|\xi\|^2_{b_E} }
.
$$
If $\mu_*\xi \not= 0$ we can write
\begin{align*}
\frac{1}{ e^t\|\mu_*\xi\|_{h_X}^2 + \|\xi\|^2_{b_E} }
&= \frac{1}{e^t\|\mu_*\xi\|_{h_X}^2}
\frac{1}{1 + \|\xi\|^2_{b_E} / e^t\|\mu_*\xi\|_{h_X}^2}
\\
&= \frac{1}{e^t\|\mu_*\xi\|_{h_X}^2}
\sum_{n \geq 0} (-1)^n \bigl(
\|\xi\|^2_{b_E} / e^t\|\mu_*\xi\|_{h_X}^2
\bigr)^n
\end{align*}
and so
$$
\Theta
= \mu^*R_{h_X}(\xi, \ov \xi, \xi, \ov \xi) + O(e^{-t})
$$
which is positive for all $t \gg 0$ because $\mu^*R_{h_X}(\xi, \ov \xi, \xi,
\ov \xi) > 0$.
If $\mu_*\xi = 0$ we get
\begin{align*}
\Theta
&= \frac{e^t\|\mu_*\xi\|_{h_X,z\bar z}^2 + \|\xi\|^2_{b_E,z \bar z} }
{\|\xi\|^2_{b_E}}
- \frac{e^t\|\mu_*\xi\|_{h_X,z}^2 + \|\xi\|^2_{b_E,z} }{\|\xi\|^2_{b_E}}
\frac{e^t\|\mu_*\xi\|_{h_X,\bar z}^2 + \|\xi\|^2_{b_E,\bar z} }{\|\xi\|^2_{b_E}}
\\
&=\frac{e^t}{\|\xi\|^2_{b_E}}
(\|\mu_*\xi\|_{h_X,z\bar z}^2 - \|\mu_*\xi\|_{h_X,z}^2 \|\mu_*\xi\|_{h_X,\bar z}^2 )
+
\frac{\|\xi\|^2_{b_E,z}}{\|\xi\|^2_{b_E}}
- \frac{\|\xi\|^2_{b_E,z} }{\|\xi\|^2_{b_E}}
\frac{\|\xi\|^2_{b_E,\bar z} }{\|\xi\|^2_{b_E}}
\\
&=
\frac{e^t}{\|\xi\|^2_{b_E}}\mu^*R_{h_X}(\xi, \ov \xi, \xi, \ov \xi)
+ R_{b_E}(\xi, \ov \xi, \xi, \ov \xi)
> 0
\end{align*}
for all $t$ because $\mu^*R_{h_X} \geq 0$, and $\|\xi\|^2_{b_E} > 0$ and
$R_{b_E}(\xi, \ov \xi, \xi, \ov \xi) > 0$ because we are
on the exceptional divisor.

I'm not sure the identification with $\mu^*R_{h_X}(\xi, \ov \xi, \xi, \ov \xi)$
makes sense.
We can arrange for $f$ to be such that $\mu_*f_*\partial / \partial z \not= 0$
when $z \not= 0$ (I'm fairly sure; check Wu's construction).
Then $f^*\mu^*h_X$ is a singular metric on the line bundle $T_D$ with a
singularity at $0$ and if we deconstruct $f^*h_t$ we end up trying to assign a
value to the curvature current at $0$.

\section{2. May 2023}

Let $M = \{ (x,y,z) \in \kk R^3 \mid x, y, z > 0,\ xyz = 1\}$ be a hyperbola.
Its tangent space is given by $\ker(yz dx + xz dy + xy dz = dx/x + dy/y + dz/z)$.

Let $X = (x_1, x_2, x_3)$ and $Y = (y_1, y_2, y_3)$ be tangent fields at $(x,y,z)$.
The second fundamental form at $(x,y,z)$ is
$$
\alpha(X,Y)
= (d_X Y)^\perp
= \< d_X Y, n  \> n / |n|^2,
$$
where $n = (1/x,1/y,1/z)$.
The curvature tensor of $M$ is then
$$
R(X,Y,Z,W)
= \< d_X Z, n \> \<n, d_Y W \> / |n|^2
- \< d_Y Z, n\> \<n, d_X W \> / |n|^2.
$$
We have
$$
\displaylines{
d_X Y
= (d_X y_1, d_X y_2, d_X y_3)
\hfill\cr\hfill{}
= \Bigl(
x_1 \frac{\partial y_1}{\partial x}
+ x_2 \frac{\partial y_1}{\partial y}
+ x_3 \frac{\partial y_1}{\partial z},
x_1 \frac{\partial y_2}{\partial x}
+ x_2 \frac{\partial y_2}{\partial y}
+ x_3 \frac{\partial y_2}{\partial z},
x_1 \frac{\partial y_3}{\partial x}
+ x_2 \frac{\partial y_3}{\partial y}
+ x_3 \frac{\partial y_3}{\partial z}
\Bigr)
}
$$
so
$$
\displaylines{
\< d_X Y, n \>
=
\frac{x_1}{x} \frac{\partial y_1}{\partial x}
+ \frac{x_2}{x} \frac{\partial y_1}{\partial y}
+ \frac{x_3}{x} \frac{\partial y_1}{\partial z}
+ \frac{x_1}{y} \frac{\partial y_2}{\partial x}
+ \frac{x_2}{y} \frac{\partial y_2}{\partial y}
+ \frac{x_3}{y} \frac{\partial y_2}{\partial z}
\hfill\cr\hfill{}
+ \frac{x_1}{z} \frac{\partial y_3}{\partial x}
+ \frac{x_2}{z} \frac{\partial y_3}{\partial y}
+ \frac{x_3}{z} \frac{\partial y_3}{\partial z}
}
$$
Don't we only have to calculate this for two vectors?
Two such are $X = (x, 0, -z)$ and $Y = (0, y, -z)$.
Have
$$
\< d_X X, n \> = 0,
\quad
\< d_Y Y, n \> = 0,
\quad
\< d_X Y, n \> = - 1,
$$
so
$$
R(X,Y,X,Y)
= -\frac{1}{|n|^2}
= -\frac{1}{(x^2y^2 + x^2z^2 + y^2z^2)}.
$$


\section{1. May 2023}

I've wondered for a while whether I can fix my degenerate paper with a density
argument.
The connection and curvature constructions work for non-degenerate forms, and
we get a Codazzi--Griffiths theorem for those.
Non-degenerate forms are dense in the space of all forms.
Can we take a limit and conclude what we want?

Uhh... maybe?
Our goal is to prove a theorem like ``the blow-up of a manifold with positive
hsc has positive hsc''.
An approximation argument needs some way of controlling the lower bound on the hsc of the metrics that participate in the limit.
I don't think we have one.

As we all know, $\int H = s/n(n-1)$, and $\int s = c_1 \omega\^{n-1}$.
If $m \leq H \leq M$ we then get
$$
m \Vol X \leq \frac{c_1 \omega\^{n-1}}{n(n-1)} \leq M \Vol X.
$$
This gives an upper bound on $m$ (which is itself only useful if $m \geq 0$,
which is what we're trying to prove) while we need a lower bound.

We also have
$$
\int H^2 \cong |R|^2 + 4|r|^2 + s^2,
$$
which doesn't help with getting a lower bound on $H$.



\paragraph{Something else}

We have something like
$$
B(x,y)
= \tfrac14(R(x + y) + R(x + iy) + R(x - y) + R(x - iy))
- R(x) - R(y),
$$
where $R(x) = R(x, \ov x, x, \ov x)$.
Suppose $m \leq H \leq M$.
This doesn't give a useful bound on $B$.
I think there is one thought? In a paper from Hasset and some other people on negative $T_X$?



\section{29. April 2023}

Let $X = C \times C$ be the product of a curve with itself.
The curve has a class $h$ such that $\int_C h = 1$.
The K\"ahler cone of $X$ is then
$$
\cc K
= \{ (x,y) \mid x, y > 0 \}.
$$
The volume of a class $\omega = \omega(x,y) = x h_1 + y h_2$ is $xy$,
so the volume-one classes form a component of the hyperbola $xy = 1$.

At $\omega$ the primitive classes are multiples of $yh_1 - xh_2$.
The norm of that class is
$$
- (yh_1 - xh_2)^2 = 2xy.
$$
I think it follows that the volume form from the Riemannian metric at $\omega$
is just $2$ times the usual one.
In particular the set of classes of volume one has infinite volume.

This is maybe not surprising.
That set is one-dimensional so volume coincides with length,
and we already know that the Riemannian metric is complete here, so the boundary of the cone is infinitely far away.

\section{27. April 2023}

I'm not sure we can detect non-cscK metrics with the norm of hsc.
(Or even at all for that matter.)
If we can do anything at all we should already be able to do that with
\begin{align*}
|R|^2 \omega\^{n}
&= (2c_2 - c_1^2) \omega\^{n-2} + s^2 \omega\^n,
\\
|r|^2 \omega\^{n}
&= -c_1^2 \omega\^{n-2} + s^2 \omega\^n,
\end{align*}
the Cauchy--Schwarz type inequalities
\begin{align*}
	n^4 s^2 &\leq |R|^2,
	\\
	n^2 s^2 &\leq |r|^2,
\end{align*}
and the basic estimate
$$
(c_1 \omega\^{n-1})^2 \leq
\Vol X \int_X s^2 \omega\^n.
$$
Plugging this into the above we get
\begin{align*}
0 &\leq (2c_2 - c_1^2) \omega\^{n-2} + (c_1 \omega\^{n-1})^2 / \Vol X,
\\
0 &\leq -c_1^2 \omega\^{n-2} + (c_1 \omega\^{n-1})^2 / \Vol X.
\end{align*}
Both of these should hold if there exists a cscK metric in $[\omega]$.
If we can find $X$ and $\omega$ such that either of these is negative,
we conclude that there is no cscK metric in $[\omega]$.

The factor $(c_1 \omega\^{n-1})^2 / \Vol X$ is the square of a real number,
so it's never negative.
The closer $c_1$ comes to being a multiple of $\omega$ the more negative $-c_1^2\omega\^{n-2}$ is.
But if $c_1 = t \omega$ then
$$
c_1 \omega\^{n-1} = tn \omega\^n
\qandq
c_1^2 \omega\^{n-2} = t^2 \tbinom{n}{2} \omega\^n
$$
so
$$
- c_1^2 \omega\^{n-2} + (c_1\omega\^{n-1})^2 / \Vol X
= t^2 \bigl(n^2 - \tbinom{n}{2}\bigr)
= \tfrac12 t^2 n \geq 0.
$$
We can always write $c_1 = t\omega + u$, where $u$ is primitive, and reduce to
this situation (at best; generally we'll get an extra $|u|^2$ term).

If $X$ is a surface then $\int_X c_2 = e(X)$, the Euler number of $X$.
We then get
$$
\int_X (2c_2 - c_1^2) + \Bigl(\int_X c_1\omega\Bigr)^2/\Vol X
= 2 e(X) + t^2.
$$
By scaling $\omega$ we can make $t \to 0$.

Maybe look at $X = \kk P^1 \times C$, where $C$ is a curve of genus $g \gg 0$?
It'll have Euler characteristic $e(X) = e(\kk P^1) e(C) = 4(1-g)$.
Let $h = c_1(\mathcal O(1))$ and $k = c_1(L)$, where $L \to C$ is an ample generator.
K\"ahler classes on $X$ are $\omega = ah + bk$ with $a, b > 0$.
We have $c_2(X) = -hk$ and $c_1(X) = h - k$.
\begin{align*}
\omega\^2 &=\tfrac12 (ah + bk)^2 = ab \, hk,
\\
c_1 \omega &= (h-k)(ah + bk) = (b - a) hk = \frac{b-a}{ab} \omega\^2
\\
c_1^2 &= (h-k)^2 = -2 hk = \frac{-2}{ab} \omega\^2.
\end{align*}
Note that $\int_{\kk P^1} h = 2$ and $\int_{C} k = 2g-2$, so $\Vol X = 4ab(g-1)$.
Now
$$
\int_X(2c_2 - c_1^2) + (c_1 \omega)^2 / \Vol X
= -8(g-1) + 8(g-1) + \frac{(b-a)^2}{a^2} 4ab(g-1)
$$
which is always nonnegative.


\section{24. April 2023}

A couple of weeks old by now but no matter.
We know that
$$
\frac{1}{\Vol(S(T_V))} \int_{S(V)} H(\xi)^2 \, d\sigma(\xi)
= \frac{|R|^2 + 4|r|^2 + s^2}{\binom{n+1}{2}\binom{n+3}{2}}
$$
for a K\"ahler curvature tensor $R$ on $V$.
Then
\begin{align*}
0
\leq \operatorname{Var}(H)
&=
\frac{|R|^2 + 4|r|^2 + s^2}{\binom{n+1}{2}\binom{n+3}{2}}
- \frac{s^2}{\binom{n+1}{2}^2}
\\
&= \frac{1}{\binom{n+1}{2}^2\binom{n+3}{2}}
\left(
\tbinom{n+1}{2} |R|^2
+ 4\tbinom{n+1}{2} |r|^2
+ \tbinom{n+1}{2} s^2
- \tbinom{n+3}{2} s^2
\right).
\end{align*}
We have
$$
\tbinom{n+1}{2}
- \tbinom{n+3}{2}
= -2n - 3
$$
so equivalently
$$
0 \leq
\tbinom{n+1}{2} |R|^2
+ 4\tbinom{n+1}{2} |r|^2
- (2n+3) s^2.
$$
We know that
\begin{align*}
|R|^2 \omega\^{n}
&= (2c_2 - c_1^2) \omega\^{n-2} + s^2 \omega\^n,
\\
|r|^2 \omega\^{n}
&= -c_1^2 \omega\^{n-2} + s^2 \omega\^n,
\end{align*}
so we get
$$
0
\leq
\tbinom{n+1}{2} (2c_2 - 5c_1^2) \omega\^{n-2}
+ \tfrac12 (n-1)(5n+6) s^2 \, \omega\^n.
$$
If there exists a constant scalar curvature metric in $[\omega]$ its scalar
curvature is $s = c_1 \omega\^{n-1} / \omega\^n$.
Then we get the cohomological inequality
$$
0
\leq
\tbinom{n+1}{2}
\Vol X
\int_X (2c_2 - 5c_1^2) \, \omega\^{n-2}
+ \tfrac12 (n-1)(5n+6)
\left(
\int_X c_1 \omega\^{n-1}
\right)^2.
$$
Are there any Fano manifolds that don't satisfy this inequality?
Does it follow obviously from stability?

In $L^2$ we have
$$
\int_X s^2 \omega\^n
= \< s, 1 \>^2 \leq |s|^2 \Vol X^2
= (c_1 \omega\^{n-1})^2 \Vol X^2
$$
with equality iff $s$ is constant.
We get cohomological upper bounds on the $L^2$ norms of $R$ and $r$.


\section{12. April 2023}

The Riemann curvature tensor on a manifold $(M,g)$ is
$$
\Rm(u,v) z = \nabla_u \nabla_v z - \nabla_v \nabla_u z - \nabla_{[u,v]} z,
$$
where $\nabla$ is the Levi-Civita connection.
We write
$$
R(u,v,z,w) = \< \Rm(u,v)z, w \>
$$
for its contraction against the metric.
The curvature tensor has the algebraic symmetries
\begin{align*}
&R(u,v,z,w) = - R(v,u,z,w)
\\
&R(u,v,z,w) = - R(u,v,w,z)
\\
&R(u,v,z,w) = R(z,w,u,v)
\\
&R(u,v,z,w) + R(u,z,w,v) + R(u,w,v,z) = 0.
\end{align*}
The first three together say that $R$ is a symmetric bilinear form on
$\ext{2}{T_M}$.
The fourth one, the first Bianchi identity, is something else.

The scalar curvature $s$ of $R$ is its trace against the metric $\wedge^2 g$.
It can also be written as
$$
s = C \int_{S(\ext{2}T_M)} R(\alpha, \alpha) \; d\sigma(\alpha),
$$
where $\alpha \in S(\ext{2}T_M)$ and $C$ is a constant I don't remember.
Can we compare this with
$$
\int_{S(T_M)} \int_{S(T_M)} R(u,v,u,v) \; d\sigma(u) d\mu(v) ?
$$
We have a map $T_M \times T_M \to \ext{1}T_M$ given by $(u,v) \mapsto u \wedge v$.
It's neither surjective nor injective, so maybe not a big help.



\section{25. November 2022}

For some reason I can't get this out of my head, so let's go.

Let $V$ be a complex $n$-dimensional vector space with an inner product $h$.
If $X \subset V$ is a Lebesgue measurable subset and $f : X \to W$ a measurable
function, where $W$ is another complex vector space, we can define
$$
\int_X f(v) \, d\mu(v)
$$
by picking a basis and integrating coordinate by coordinate.
Similarly, since $W \cong \kk C^m$ we know what a measurable function with
values in $W$ is.

If $L : V \to W$ is a linear map then
$$
L \biggl( \int_X f(v) \, d\mu(v) \biggr)
= \int_X L f(v) \, d\mu(v)
$$
because the integral is defined step by step by looking at indicator functions,
then finite linear combinations of those, then supremums of those, and so on,
and linear maps commute with the integral of all of those.
In particular if $f : X \to \End V$, then the trace commutes with the integral.

If $v \in V$ then $v \otimes v^*$ denotes the linear map $V \to V$ given by $x
\mapsto h(x, \ov v) v$.
Consider the linear map $\pi : V^{\otimes d} \to V^{\otimes d}$ defined by
$$
\pi = \int_{S(V)} (v \otimes v^*)^{\otimes d} \, d\mu(v).
$$
We claim that if $g \in U(V)$ then $g^{\otimes d} \pi = \pi g^{\otimes d}$.
On each factor we have $g(v\otimes v^*(x)) = h(x,\ov v) g(v)$
while $v \otimes v^*(g(x)) = h(g(x), \ov v) v = h(x, \ov{g^{-1}v}) v$.
Then
\begin{align*}
\pi(g^{\otimes d}(x_1 \otimes \cdots \otimes x_d))
&= \int_{S(V)} \bigotimes_{j=1}^d h(g x_j, \ov v) v \, d\mu(v)
\\
&= \int_{S(V)} \bigotimes_{j=1}^d h(x_j, \ov{g^{-1}v}) v \, d\mu(v)
\\
&= \int_{S(V)} \bigotimes_{j=1}^d h(x_j, \ov{v}) g v \, d\mu(v)
\\
&= g^{\otimes d} \int_{S(V)} \bigotimes_{j=1}^d h(x_j, \ov{v}) v \, d\mu(v)
\\
&= g^{\otimes d} \pi(x_1 \otimes \cdots \otimes x_d),
\end{align*}
where we make the change of variable $u = gv$ in the third equality and recall
that $|\!\det g| = 1$.

This means that $\pi$ is an endomorphism of the representation $V^{\otimes d}$
of $U(V)$.
It then must be a multiple of a projection onto one of the irreducible factors
of $V^{\otimes d}$.
That factor must be $\Sym^d V$, because $\pi$ is clearly invariant under the
action of the symmetric group $S_d$.

For a fixed unit vector $v = (v_1,\ldots,v_n)$ the map $v \otimes v^*$
is $v \otimes v^*(x) = v \sum_j x_j \ov v_j$ so its trace is $\sum_j |v_j|^2 = 1$.
Then $\tr \pi = \Vol(S(V))$, so
$$
\pi = \frac{\Vol(S(V))}{\binom{n+d-1}{d}} \Pi_d,
$$
where $\Pi_d$ is the projection onto $\Sym^d V$ and $\binom{n+d-1}{d} = \dim
\Sym^d V$.
Slightly more generally, if $f : V \to V$ is a linear map, then
$\tr f (v \otimes v^*) = \tr f |v|^2 = \tr f$ for a unit vector $v$.
Also
$\langle f v, \ov v \rangle = \tr f (v \otimes v^*)$.

If $v$ has unit norm then so does $v^{\otimes n}$.
Then
$$
\frac{\Vol(S(V))}{\binom{n+d-1}{d}} \tr(f \Pi_d)
= \int_{S(V)} \langle f(v^{\otimes d}), \ov{v^{\otimes d}} \rangle \, d\mu(v)
$$
for any endomorphism $f$ of $V^{\otimes d}$.


\section{22. November 2022}

Let $V$ be an $n$-dimensional vector space.
Pick a unit vector $v \in V$ and define
$$
p : \Sym^2 V \to V,
\quad
x \odot y \mapsto \langle y, \ov v \rangle x + \langle x, \ov v \rangle y.
$$
This is a linear map.

If $x$ is orthogonal to $v$, then
$p(x \odot v) = \langle x, \ov v \rangle v + |v|^2 x = x$.
We also have $p(v \odot v) = 2 v$.
If $x = x' + a v$, where $x'$ is orthogonal to $v$, we then have
$$
p(x' \odot v + \tfrac a2 v \odot v)
= x' + a v = x
$$
so $p$ is surjective.

Let $f$ be an endomorphism of $\Sym^2 V$; we know that
$$
\tr f = C \int_{S(\Sym^2 V)} \langle f x, \ov x \rangle d\sigma(x)
$$
for a constant $C$.
The constant map factors as $\Sym^2 V \to V \to *$ via $p$
so we have
$$
\int_{S(\Sym^2 V)} \langle f x, \ov x \rangle d\sigma(x)
%= \int_{p(S(\Sym^2 V))} p_*(\langle f x, \ov x \rangle d\sigma(x))
= \int_{y \in p(S(\Sym^2 V))} \int_{x \in p^{-1}(y)}
\langle f x, \ov{x} \rangle d\sigma(x).
$$

What are the fibers of $p$?
Since it's linear, it's enough to know $\Ker p$.
Suppose $y = x$. Then
$$
p(x \odot x)
= 2 \langle x, \ov v \rangle x
$$
so $p(x \odot x) = 0$ if and only if $x$ is orthogonal to $v$.
Suppose then that $x$ and $y$ are linearly independent.
Then
$$
p(x \odot y) = \langle y, \ov v \rangle x + \langle x, \ov v \rangle y.
$$
If $x$ and $y$ are orthogonal to $v$, then this is zero.
If only $x$ is orthogonal to $v$, then $p(x \odot y) = \langle y, \ov v \rangle
x$, which is nonzero by hypothesis.
If neither $x$ nor $y$ are orthogonal to $v$ then $p(x \odot y) \not= 0$ by
linear independence.
Then
$$
\Ker p
= \{ x \odot y \mid \text{$x$ and $y$ are orthogonal to $v$} \}
= \Sym^2 v^\perp \subset \Sym^2 V.
$$
I'm not sure this is helpful because as far as I can tell the image or kernel
of $p$ don't have anything to do with squares in $\Sym^2 V$.


\paragraph{Squares in $\Sym^2 V$}

$$
x \odot y
= \frac14 \bigl( (x + y)^2 - (x - y)^2 - x^2 - y^2 \bigr)
$$
so every element is a sum of squares.
Is that helpful?
\begin{align*}
16 b(xy, \ov{xy})
&= b((x{+}y)^2 - (x{-}y)^2 - x^2 - y^2, \ov{(x{+}y)^2 - (x{-}y)^2 - x^2 - y^2})
\\
&= b((x+y)^2, \ov{(x+y)}^2)
- b((x+y)^2,(x-y)^2)
\\
&\quad
- b((x+y)^2,x^2)
- b((x+y)^2,y^2)
- \cdots
\end{align*}

\paragraph{The Veronese embedding}

The Veronese map is
$$
f : V \to \Sym^2 V,
\quad
x \mapsto x \odot x.
$$
It is holomorphic.
It isn't injective as $f(- x) = f(x)$.
However it defines an embedding
$$
f : \kk P(V) \to \kk P(\Sym^2 V),
\quad
[x] \mapsto [x^2]
$$
that has a regular inverse.
We have an inner product $h$ on $V$ and $h^2$ on $\Sym^2 V$.
We have a smooth function $g : \kk P(\Sym^2 V) \to \kk R$ defined by
$$
g([v]) = \frac{b(v, \ov{v})}{|v|^2},
$$
where $b \in B(\Sym^2 V)$ is a Hermitian form.
Let $d\mu$ be the volume form of the Fubini--Study metric.
Then
$$
\int_{\kk P(\Sym^2 V)} \frac{b(v, \ov v)}{|v|^2} d\mu
= C \tr_{h^2} b
$$
for $C = \Vol(\kk P(\Sym^2 V))$.

We have $\dim \Sym^2 V = \binom{n+1}2$ and
$$
\dim \kk P(\Sym^2 V) - \dim \kk P(V)
= \tbinom{n+1}{2} - 1 - n + 1
= \frac{n^2 - n}{2}
= \tbinom{n}{2}.
$$



\paragraph{Maybe just calculate?}


Let $(v_1, \ldots, v_n)$ be an orthonormal basis of $V$.
If $z = \sum_j z_j v_j$ we have
$$
z \odot z
= \sum_{j,k} z_j z_k \, v_j \odot v_k
$$
so
$$
b(z \odot z, \ov{z \odot z})
= \sum_{j,k,l,m} z_j z_k \ov z_l \ov z_m
\, b(v_j \odot v_k, \ov{v_l \odot v_m}).
$$
Consider the real and imaginary parts of the product of the $z$.
If their degree is not even the integral over the sphere is zero by symmetry,
so the integral is nonzero only if $j = k = l = m$ or or $j = l$ and $k = m$.


After Folland we have
\begin{align*}
\int_{S(V)} x_j^4 d\sigma &= \frac{3}{4n(n+1)} \Vol(S(V))
\\
\int_{S(V)} x_j^2 x_k^2 d\sigma &= \frac{1}{4n(n+1)} \Vol(S(V)),
\quad j \not= k,
\end{align*}
where $\dim V = n$.
As
\begin{align*}
|z_j|^4 &= x_j^4 + 2 x_j^2 y_j^2 + y_j^4
\\
|z_j|^2 |z_k|^2 &= x_j^2 x_k^2 + x_j^2 y_k^2 + x_k^2 y_j^2 + y_j^2 y_k^2,
\quad
j \not= k
\end{align*}
we get
$$
\int_{S(V)} |z_j|^4 = 8 C
\qandq
\int_{S(V)} |z_j|^2 |z_k|^2 = 4 C,
\quad j\not=k,
$$
where $C = \Vol(S(V)) / (4n(n+1))$.
Then
\begin{align*}
\int_{S(V)}
\!\!\!
b(z \odot z, \ov{z \odot z}) d\sigma
&= 8C \sum_{j} b(v_j {\odot} v_j, \ov{v_j {\odot} v_j})
+ 4C \sum_{j\not=k} b(v_j {\odot} v_k, \ov{v_j {\odot} v_k})
\\
&= 8C \sum_{j \leq k} b(v_j {\odot} v_j, \ov{v_k {\odot} v_k})
= 8C \tr_{h^2} b.
\end{align*}
This works out to the lovely equation
$$
\dim{\Sym^2 V} \int_{S(V)} \!\! f^*b(z, \ov z) \, d\sigma
= \Vol(S(V)) \tr_{h^2} b,
$$
where $f : V \to \Sym^2 V$ is the Veronese map.


\paragraph{Representation theory}

If $f \in \End \Sym^2 V$ the formula
$$
\int_{S(V)} \langle f(x \odot x), \ov{x \odot x} \rangle d\sigma
= \frac{\Vol(S(V))}{\binom{n+1}{2}} \tr f
$$
is apparently true by representation theory of unitary groups.
We can also prove it directly by more or less the same as above.

% xx^* = v \mapsto < v, x > x I guess?

From it we get that $H$ dominates $s$.

If $b$ is the Hermitian form $R$ defines on $\Sym^2 V$ then
$$
\frac{\Vol(S(V))}{\binom{n+1}{2}} |b|^2
= \int_{S(V)} |b(x \odot x, \ov{x \odot x})|^2 d\sigma
$$
so we also get that $H = 0$ implies $R = 0$.

I'm having trouble finding sources to say exactly how that formula is true.
Let's set
$$
\mu(f) := \int_{S(V)} \langle f(x \odot x), \ov{x \odot x} \rangle d\sigma
$$
for $f \in \End \Sym^2 V$.
This is a linear functional that is invariant under $U(V)$, as
$g^{-1} f g (x \odot x) = \ov g^t f (gx \odot gx)$
so once we take the inner product and integrate we find that $\mu(g^{-1}fg) =
\mu(f)$ for all $g \in U(V)$.
This means it is a class function, so it can be written as a linear combination
of characters of the representation $\Sym^2 V$ of $U(V)$.
Now what?

If $f,g \in \End \Sym^2 V$ then is $\mu(fg) = \mu(gf)$?
We of course have $fg = gf - [f, g]$

What does $xx^*$ mean?
It has to be the linear map $y \mapsto \langle y, \ov x \rangle x$.
Then
$$
\int_{S(V)} x x^* \otimes \ov{yy^*} d\sigma
= x \otimes \ov y \int_{S(V)}
\langle z, \ov x \rangle \langle y, \ov z \rangle d\sigma(z)
$$
In a basis we have $x = (x_j)$, $y = (y_j)$ and $z = (z_j)$, so
$$
\langle z, \ov x \rangle \langle y, \ov z \rangle
= \sum z_j \ov x_j \sum \ov z_k y_k
= \sum z_j \ov z_k \ov x_j y_k
$$
and $z_j \ov z_k = (a_j + ib_j)(a_k - ib_k) = a_ja_k + b_j b_k + i(- a_jb_k +
b_j a_k)$, which integrates to $0$ unless $j = k$, where we get $C$, so this
integral evaluates to $C \langle y, \ov x \rangle x \otimes \ov y$.
If we take $xx^* \otimes yy^*$ we just get zero.
I don't see how this helps.


\paragraph{A quantum of help}

Quantum theory people really like this stuff and they have some papers and
books where they ``explain'' what they're doing.
After staring at those for a little while I think I can say this:
The way to interpret
$$
\int_{S(V)} (v \otimes v^*)^{\otimes n} \, d\sigma(v)
$$
is to say that it is equivalent to give a linear map $f : V \to V$ and to give
an element $\lambda \in V^* \otimes \ov V^*$ because of the isomorphism from
the inner product $V \to \ov V^*$.
The above should then be interpreted as the linear morphism
$\pi : V^{\otimes n} \to V^{\otimes n}$
that corresponds to the linear functional
$$
\lambda(x_1 \otimes \cdots \otimes x_n
\otimes
\ov y_1 \otimes \cdots \otimes \ov y_n)
=
\int_{S(V)} \prod_{j=1}^n
\langle x_j, \ov v \rangle \langle v , \ov y_j \rangle \, d\sigma(v)
$$
under the induced inner product $V^{\otimes n} \to \ov V^{\otimes n}$.

The quantum people say that this $\pi$ now commutes with all $U(V)^{\otimes n}$,
so it has to be proportional to the projection onto an irreducible
subrepresentation of $V^{\otimes n}$.
It also fixes $\Sym^n V$, so it has to project onto that.
Then we find the exact factor by some kind of magic.

Once we've done this we still have to prove that
$$
f\biggl(
\int_{S(V)} (v \otimes v^*)^{\otimes n} \, d\sigma(v)
\biggr)
= \int_{S(V)} f(v \otimes \cdots \otimes v) \, (v^*)^{\otimes n} \, d\sigma(v)
$$
for a linear map $f : V^{\otimes n} \to V^{\otimes n}$.



\section{20. November 2022}

Let
\begin{align*}
\cc H(V)
&= \{ \text{germs of smooth Hermitian metrics on $V$ at $0$} \},
\\
\cc K(V)
&= \{ \text{germs of smooth K\"ahler metrics on $V$ at $0$} \}.
\end{align*}
We just write $\cc H$ if $V$ is clear from context and will set $V = \kk C^n$.
Note that $\cc K$ and $\cc H$ are only convex open cones in real vector spaces
and not full subspaces.

Let $B(V)$ be the real vector space of Hermitian forms on $V$.
The curvature tensor of a metric defines linear maps
and a commutative diagram
$$
\begin{tikzcd}
\cc K \ar[d] \ar[r,"\cc R"] & B(\operatorname{Sym}^2 \kk C^n)\ar[d]
\\
\cc H \ar[r, "\cc R"] & B(\kk C^n \otimes \kk C^n).
\end{tikzcd}
$$
The horizontal arrows are given evaluating curvature tensors at $0$.
The first vertical arrow is just the inclusion of K\"ahler metrics into the
space of Hermitian metrics; the second is given by the pullback by the surjection
$\kk C^n \otimes \kk C^n \to \operatorname{Sym}^2 \kk C^n$
given by the symmetrization of a tensor.
Both vertical arrows are of course injective.

Are the horizontal arrows surjective?

I think we can extend $\cc R$ to germs of smooth Hermitian (``K\"ahler'') forms
on $V$, first by defining it for nondegenerate forms and then by extending by
zero to the rest of the space. I'm not sure we need to.
In any case $\cc R$ is not linear, which defeats the point.

The map $\cc R$ is continuous, so its image is connected.
Is it open and closed?

Let $H$ be the matrix of a Hermitian metric $h$ in a neighborhood around $0$.
The Chern connection of $h$ is given by $D' = H^{-1}\partial H$ and its
curvature form is
$$
F = \tfrac i2 \bar\partial(H^{-1}\partial H)
= - \tfrac i2 H^{-1} \bar\partial H \wedge H^{-1} \partial H
- \tfrac i2 H^{-1} \partial \bar\partial H.
$$
The curvature tensor is then $R(x,\ov y, z, \ov w) = h(F(x, \ov y) z, \ov w)$.

We can pick coordinates such that $H = I_n$ at $0$.
If $h$ is K\"ahler we can further get $\partial H = 0$ at $0$.
I guess this means that $H^{-1}\partial \bar\partial H$ always comes from the
symmetric part?
Its contribution to the curvature tensor is
$$
- \ov w^t \cdot \operatorname{Hess}(H)(x, \ov y) \cdot z,
$$
where $\operatorname{Hess}(H)$ is the (block, I guess) matrix whose entries are
the Hessian matrices of the entries of $H$.

Taking $n = 2$ we have
$$
H = \begin{pmatrix}
a & c
\\
\ov c & b
\end{pmatrix}
$$
with $H(0) = I_2$.
Assuming $h$ is K\"ahler we can arrange that $\partial H(0) = 0$.
The basis is $(z \otimes z, z \otimes w, w \otimes z, w \otimes w)$.
We get
\begin{align*}
R(z, \ov z, z, \ov z) &= a_{z \ov z}
\\
R(z, \ov z, z, \ov w) &= \ov c_{z \ov z}
\\
R(z, \ov z, w, \ov z) &= c_{z \ov z}
\\
R(z, \ov z, w, \ov w) &= b_{z \ov z}
\end{align*}
so the matrix of the curvature tensor is
$$
R = \begin{pmatrix}
a_{z \ov z} & \ov c_{z \ov z} & a_{z \ov w} & \ov c_{z \ov w}
\\
c_{z \ov z} & b_{z \ov z} & c_{z \ov w} & b_{z \ov w}
\\
a_{w \ov z} & \ov c_{w \ov z} & a_{w \ov w} & \ov c_{w \ov w}
\\
c_{w \ov z} & b_{w \ov z} & c_{w \ov w} & b_{w \ov w}
\end{pmatrix}.
$$
Write the metric as
$$
\omega
= \frac i2 \bigl(
a dz \wedge d\bar z
+ c dw \wedge d \bar z
+ \ov c dz \wedge d\bar w
+ b dw \wedge d\bar w.
\bigr)
$$
The metric is K\"ahler, so $\partial \omega = \bar\partial \omega = 0$.
This means that
$$
a_w = c_{z},
\quad
a_{\ov w} = \ov c_{\ov z},
\quad
b_z = \ov c_w,
\quad
b_{\ov z} = c_{\ov w},
$$
so
$$
c_{z \ov z} = a_{w \ov z},
\quad
\ov c_{w \ov w} = b_{z \ov w},
\quad
c_{z \ov w} = a_{w \ov w},
\quad
\ov c_{w \ov z} = b_{z \ov z}
$$
Then we get
$$
R = \begin{pmatrix}
a_{z \ov z} & a_{z \ov w} & a_{z \ov w} & a_{w \ov w}
\\
a_{w \ov z} & a_{w \ov w} & a_{w \ov w} & b_{z \ov w}
\\
a_{w \ov z} & a_{w \ov w} & a_{w \ov w} & b_{z \ov w}
\\
a_{w \ov w} & b_{w \ov z} & b_{w \ov z} & b_{w \ov w}
\end{pmatrix},
$$
where a bunch of simplifications happen because $a$ is real so $a_{w \ov z} =
a_{z \ov w}$ and we can ping-pong to $b_{z \ov z} = a_{w \ov w}$.
This is the pullback of the Hermitian form
$$
R = \begin{pmatrix}
a_{z \ov z} & a_{z \ov w} & a_{w \ov w}
\\
a_{w \ov z} & a_{w \ov w} & b_{z \ov w}
\\
a_{w \ov w} & b_{w \ov z} & b_{w \ov w}
\end{pmatrix},
$$
on $\Sym^2 \kk C^2$, as expected.
Note that $a_{w \ov w}$ is real, so not every Hermitian form on
$\Sym^2 \kk C^2$ comes from a curvature tensor of a K\"ahler metric.
(Note that $a_{w \ov w} = b_{z \ov z}$ so this matrix is more symmetric in $a$
and $b$ than it appears.)


Suppose we let
$$
f(z,w)
= a_{z \ov z} z \bar z
+ a_{z \ov w} z \ov w
+ a_{w \ov z} w \ov z
+ a_{w \ov w} w \bar w.
$$
Then $f(0) = 0$ and $df(0) = 0$ and
$$
\operatorname{Hess} f(0)
= \begin{pmatrix}
a_{z \ov z} & a_{z \ov w}
\\
a_{w \ov z} & a_{w \ov w}
\end{pmatrix}.
$$
We also let
$$
g(z,w)
= a_{w \ov w} z \bar z
+ b_{z \ov w} z \ov w
+ b_{w \ov z} w \ov z
+ b_{w \ov w} w \bar w
$$
and
$$
h(z,w)
= h_1 z \bar z
+ h_2 z \ov w
+ h_3 w \ov z
+ h_4 w \bar w.
$$
We want
$$
a_{w \ov z} \ov z + a_{w \ov w} \bar w
= f_w = h_z
= h_1 \ov z + h_2 \ov w,
$$
so we set $h_1 = a_{w \ov z}$ and $h_2 = a_{w \ov w}$.
We also want
$$
a_{w \ov w} \ov z + b_{z \ov w} \ov w
= g_z = \ov h_w
= \ov h_2 \bar z + \ov h_4 \ov w
$$
so $h_4 = b_{z \ov w}$.
Let's just take $h_3 = \ov h_4$.
The germ of the K\"ahler metric
$$
h(z,w)
= \begin{pmatrix}
1 + f(z,w) & h(z,w)
\\
\ov{h(z,w)} & 1 + g(z,w)
\end{pmatrix}
$$
then has the curvature tensor $R$ above. That is,
$$
\im\bigl(
\cc R : \cc K(\kk C^2) \to B(\Sym \kk C^2)
\bigr)
= \left\{
R =
\begin{pmatrix}
a & b & c
\\
\ov b & c & e
\\
c & \ov e & f
\end{pmatrix}
\biggm|
\text{$a, c, f \in \kk R$, $b, e \in \kk C$}
\right\}.
$$
The image coincides with the kernel of the linear form
$\lambda : B(\Sym \kk C^2) \to \kk R$, $\lambda(R) = \im c$, so it is a real
hypersurface.
This is a little surprising as $\cc R$ is not linear and $\cc K$ not a linear
subspace, so a priori there's no reason the image should be a linear subspace.

Let's look at some special cases. If the tensor looks like
$$
R =
\begin{pmatrix}
a & 0 & c
\\
0 & c & 0
\\
c & 0 & f
\end{pmatrix}
$$
then
\begin{align*}
R(x, \ov x, x, \ov x)
&= (a x_1^2 + c x_2^2) \ov x_1^2
+ 4 c |x_1|^2 |x_2|^2
+ (c x_1^2 + f x_2^2) \ov x_2^2
\\
&= a |x_1|^4 + f |x_2|^4
+ 2 c \bigl(2 |x_1|^2 |x_2|^2 + \Re(x_1^2 \ov x_2^2) \bigr)
\\
r(x, \ov y)
&=
\begin{pmatrix}\ov y_1 & \ov y_2 \end{pmatrix}
\begin{pmatrix}
a + c & 0 \\ 0 & c + f
\end{pmatrix}
\begin{pmatrix} x_1 \\ x_2 \end{pmatrix},
\\
s &= a + 2c + f.
\end{align*}
For complex numbers we have $|\Re z| \leq |z|$, so we have
\begin{align*}
a |x_1|^4 + f |x_2|^4 + 2c |x_1|^2 |x_2|^2
&\leq R(x, \ov x, x, \ov x)
\\
&\leq a |x_1|^4 + f |x_2|^4 + 6c |x_1|^2 |x_2|^2.
\end{align*}
If further $x = x_1 = x_2$ then
$$
R(x, \ov x, x, \ov x)
= (a + 6c + f) |x|^4.
$$
We can find $a, c, f$ such that $a + 2c + f > 0$ but $a + 6c + f < 0$,
(like $a = f = 4$, $c = -3$), so positive scalar curvature does not imply
nonpositive holomorphic sectional curvature in all directions.

Similarly, take $a = 4$, $f = -1$ and $c = -1$. Then $a + 2c + f = 1 > 0$ but
$$
r(x, \ov y) =
\begin{pmatrix}\ov y_1 & \ov y_2 \end{pmatrix}
\begin{pmatrix}
3 & 0 \\ 0 & -2
\end{pmatrix}
\begin{pmatrix} x_1 \\ x_2 \end{pmatrix},
$$
is neither positive nor negative semidefinite, so positive scalar curvature
does not imply nonnegative Ricci curvature.

Suppose $a = f = 2$ and $c = -1$. Then
$$
r(x, \ov y) =
\begin{pmatrix}\ov y_1 & \ov y_2 \end{pmatrix}
\begin{pmatrix}
1 & 0 \\ 0 & 1
\end{pmatrix}
\begin{pmatrix} x_1 \\ x_2 \end{pmatrix}
$$
is positive-definite, but at $x = (x_1, x_1)$ we have
$$
R(x, \ov x, x, \ov x) = -4 |x_1|^2
$$
so positive Ricci curvature does not imply positive holomorphic sectional
curvature in all directions.

We have
$$
R(x, \ov x, x, \ov x)
\geq
a |x_1|^2 + 2c |x_1|^2 |x_2|^2 + f |x_2|^4
$$
and the right-hand side is the quadratic form defined by
$$
Q = \begin{pmatrix}
a & c \\ c & f
\end{pmatrix}
$$
This form is positive-definite if and only if
$$
0 < \tr Q = a + f
\qandq
0 < \det Q = af - c^2.
$$
Suppose $c = -f - \epsilon$ with $\epsilon > 0$.
Then
$$
af - c^2
= af - f^2 - 2f \epsilon - \epsilon^2
= f(a - f - 2\epsilon) - \epsilon^2.
$$
If $a > f$, this is positive for small enough $\epsilon$ (like $a = 10$, $f =
1$, $\epsilon = 1$).
But then the Ricci form is
$$
r(x, \ov y) =
\begin{pmatrix}\ov y_1 & \ov y_2 \end{pmatrix}
\begin{pmatrix}
a & 0 \\ 0 & -\epsilon
\end{pmatrix}
\begin{pmatrix} x_1 \\ x_2 \end{pmatrix}
$$
which is not positive-definite,
so positive holomorphic sectional curvature does not imply nonnegative Ricci
curvature.

\section{14. November 2022}

On $k$-forms we have $[L,\Lambda] = (k-n)\id$.
I want to look at the operators $L\Lambda$ and $\Lambda L$.

There is an orthogonal decomposition
$$
\ext{k}V^* = \bigoplus_{j \geq 0} L^{j} P_{k-2j}
$$
where $P_{k-2j}$ is the space of primitive $(k-2j)$-forms and we interpret
negative indexes as meaning the space is zero.
Let's assume, for our own sanity, that we have $(p,q)$-forms with $p+q = k$.
Each $P_{k-2j}$ is then the space of primitive $(p-j,q-j)$-forms.

If $u$ is such a form then
\begin{align*}
* L\^{j} u
&= i^{p-q} (-1)^{\binom{k-2j+1}{2}} L\^{n-k+j} u,
\\
L * L\^{j} u
&= (n-k+j+1) i^{p-q} (-1)^{\binom{k-2j+1}{2}} L\^{n-k+j+1} u,
\\
\Lambda L\^{j} u
&= (-1)^k (n-k+j+1) (-1)^k (-1)^{2 \binom{k-2j+1}{2}} L\^{j-1} u
\\
&= (n-k+j+1) L\^{j-1} u,
\\
\Lambda\^{l} L\^{j} u
&= \binom{n-k+j+l}{l} L\^{j-l} u.
\end{align*}
On those forms we thus have
\begin{align*}
L\Lambda(L\^j u)
&= j (n-k+j+1) L\^j u,
\\
\Lambda L (L\^j u)
&= (j+1) (n-k+j) L\^j u,
\\
L\^l \Lambda\^l (L\^j u)
&= \binom{j}{l} \binom{n-k+j+l}{l} L\^j u.
\end{align*}
Both operators thus preserve the orthogonal decomposition.
Their characteristic polynomials are thus the product of the characteristic
polynomials of the operator on each factor.

The matrix of $L\Lambda$ is a direct sum of multiples of the identity matrix.
The dimension of each block is the dimension of the space of primitive
forms, which is
$$
\dim P_{p-j,q-j} = h^{p-j,q-j} - h^{p-j-1,q-j-1}.
$$
The characteristic polynomial of $\lambda I$ is
$$
p_{\lambda I}(t)
= \det(t I - \lambda I)
= (t - \lambda)^n
$$
and its minimal polynomial is $\mu_{\lambda I}(t) = t - \lambda$.
The minimal polynomial of $L\Lambda$ is then
$$
\mu(t)
= \prod_{j=0}^{\min(p,q)} \!\!\! (t - j(n-k+j+1))
= t \!\!\! \prod_{j=1}^{\min(p,q)} \!\!\! (t - j(n-k+j+1)).
$$
Some of these coefficients can be evaluated. For example
\begin{align*}
\sum_{j=1}^p j(n-k+j+1)
&= \sum_{j=1}^p j^2 + (n-k+1) \sum_{j=1}^p j
\\
&= \frac{p(p+1)(2p+1)}{6} + (n-k+1) \frac{p(p+1)}{2}
\\
&= \frac{p(p+1)(3n-p-3q+4)}{6}.
\end{align*}
Vieta's formulas might help in general.


There's an increasing filtration
$$
0
\subset F_{-1}
\subset F_0 = P_{p,q}
\subset \cdots
\subset F_p = \ext{p,q} V^*
$$
where we assume $p \leq q$ and
$$
F_k = \bigoplus_{j=0}^k L^j P_{p-j,q-j}.
$$
The operator $L^j\Lambda^j$ maps things to this.

The point of looking at some of this stuff is:
The minimal polynomial of $L\^l\Lambda\^l$ is $\mu_{L\^l \Lambda\^l}(t) = t f(t)$,
where $f(0) \not= 0$.
We then have
$$
L\^l \Lambda\^l f(L\^l \Lambda\^l) u = 0
$$
for all $u$.
However $L^l$ is injective, so $f(L\^l \Lambda\^l) u \in \Ker \Lambda\^l =
F_{l-1}$.

Example: $p = q = 1, k = 2$.
The minimal polynomial of $L\Lambda$ is
$$
\mu_{L\Lambda}(t) = t(t - n).
$$
Therefore $L\Lambda u - n u$ is primitive for all $(1,1)$-classes $u$.
We write
$$
u = (u - \tfrac 1n L\Lambda u) + \tfrac 1n L \Lambda u
$$
and recognize the primitive decomposition of $u$.

Example: $p = q = 2, k = 4$.
The minimal polynomials are
$$
\mu_{L\Lambda}(t) = t(t - (n-2))(t - (n-1))
\qandq
\mu_{L\^2\Lambda\^2}(t) = t(t - \tbinom{n}{2})
$$

\section{14. November 2022}

$$
\begin{tikzcd}
\ext{n-k} V^* \ar[r,"L^k"]  &
\ext{n+k} V^* \ar[d,"L^j"]
\\
\ext{n-k-j} V^* \ar[r,"L^{k+j}"] \ar[u,"L^j"] &
\ext{n+k+j} V^*
\end{tikzcd}
$$
commutes.
The horizontal arrows are isomorphisms, the arrow going up is injective and the
arrow going down is surjective.

Write $\ext{n-k} V^* = \im L^j \oplus \Ker \Lambda^j$.
Then we can extend the above square into the commutative exact sequences
$$
\begin{tikzcd}
0 \ar[r] &
\ext{n-k-j} V^* \ar[r,"L^j_1"] \ar[d,"L^{k+j}"] &
\ext{n-k} V^* \ar[r] \ar[d,"L^{k}"] &
\Ker \Lambda^j  \ar[r]  &
0
\\
0 &
\ext{n+k+j} V^* \ar[l] &
\ext{n+k} V^* \ar[l,"L^j_2"] &
\Ker L^j_2 \ar[l] &
0.\ar[l]
\end{tikzcd}
$$
Note that $\Ker L^j$ and $\Ker \Lambda^j$ have the same dimensions because of
the exact sequences and that $L^k$ and $L^{k+j}$ are isomorphisms.

Define a map $f : \Ker L^j \to \Ker \Lambda^j$ by $x \mapsto \pi((L^k)^{-1} x)$.
A little diagram chasing shows that $f$ is injective, so $f$ is an isomorphism.
%Suppose $f(x) = 0$.
%Then we must have $(L^k)^{-1}x \in \Ker \pi = \im L^j_1$, so there is a $y$ in
%$\ext{n-k-j}V^*$ such that $L^j_1 y = (L^k)^{-1}x$.
%Now $L^{k+j} y = L^j_2 L^k L^j_1 y = L^j_2 x = 0$, but $L^{k+j}$ is an
%isomorphism so $y = 0$.
%Then $(L^k)^{-1} x = 0$, and $L^k$ is an isomorphism, so $x = 0$ and $f$ is
%injective and thus an isomorphism.

Then the inverse of $f$, that is,
$$
\Ker \Bigl( \Lambda^j : \ext{n-k} V^* \to \ext{n-k-j} V^*\Bigr)
\stackrel{L^k}{\longrightarrow}
\Ker\Bigl( L^j : \ext{n+k}V^* \to \ext{n+k+j}V^*\Bigr)
$$
is an isomorphism.
Thus $\Lambda^j x = 0$ for an $(n-k)$-form $x$ if and only if
$L^{k+j} x = 0$.
Swapping variables a little, $\Lambda^j x = 0$ for a $k$-form $x$ if and only
if $L^{n-k+j} x = 0$.


\section{13. November 2022}

Let $V$ be a (real, for once) vector space and $g$ an inner product.
If $dV$ is the volume form that $g$ defines then the Hodge star operator is
defined by
$$
\alpha \wedge * \beta = \langle \alpha, \beta \rangle dV
$$
for $p$-forms $\alpha,\beta$.

Write $g$ for all the various induced inner products on the exterior powers of
$V$ and its dual.
If $\alpha$ is a $q$-form and $v \in \bigwedge^p V$ is a $p$-blade we define
the $(q-p)$-form $\iota_v\alpha$ by
$$
\iota_v \alpha (v_1, \ldots, v_{q-p})
:= \alpha(v \wedge v_1 \cdots v_{q-p}).
$$

The Hodge star is $* \beta = -(-1)^p \iota_{g^{-1}\beta} dV$:
We have $\alpha \wedge dV = 0$, so
\begin{align*}
0 = \iota_{g^{-1}\beta} (\alpha \wedge dV)
&= (\iota_{g^{-1}\beta} \alpha) \wedge dV
+ (-1)^p \alpha \wedge \iota_{g^{-1}\beta} dV
\\
&= \langle \alpha, \beta \rangle dV
+ (-1)^p \alpha \wedge \iota_{g^{-1}\beta} dV.
\end{align*}

I want to do something with this (or something similar) to prove the
Bochner--Kodaira--Nakano equation.
Looking at the proof from agbook we need the commutation relations on Hermitian
manifolds.
We have to prove:

Let $h$ be a Hermitian metric on $U \subset \kk C^n$ and write
$\tau = [\Lambda, \partial \omega]$.
Then
$$
[\bar\partial^*, L] = i(\partial + \tau)
$$
on smooth $(p,q)$-forms on $U$.

Proving this also proves the K\"ahler identities, so maybe we should start there.
Then $\tau = 0$ so we want to prove that
$$
[\bar\partial^*, L] = i\partial.
$$

\begin{proof}
Let $u$ be a primitive $(p,q)$-form and set $k = p + q$.
Recall Weil's formula,
$$
*(L^{[j]} u)
= i^{p-q} (-1)^{k(k+1)/2} L^{[n-j-k]} u.
$$
Also recall that a $k$-form $u$ is primitive if and only if $L^{n-k+1} u = 0$.


We have
$\bar\partial^* L = - * \partial * L$, and
$$
*(L L\^{j}u)
= (j+1) i^{p-q} (-1)^{\binom{k+1}2} L\^{n-k-j-1} u.
$$
Then
$$
\partial {*}(L L\^{j}u)
= (j+1) i^{p-q} (-1)^{\binom{k+1}2} L\^{n-k-j-1} \partial u .
$$

\begin{lemm}
If $u$ is a primitive $(p,q)$-form, then $\partial u = v + L w$, where $v$ and
$w$ are primitive $(p+1,q)$ and $(p,q-1)$-forms.
\end{lemm}

\begin{proof}
The claim is equivalent to saying that $\Lambda^2 \partial u = 0$.
This happens for the $(k+1)$-form $\partial u$ if and only if $L^{n-k+1}
\partial u = 0$.
Now $u$ is a primitive $k$-form form, so $L^{n-k+1} u = 0$ and $\omega$ is
$\partial$-closed so this follows by applying $\partial$.
\end{proof}

We write $\partial u = v + L w$, where $v$ and $w$ are primitive $(p+1,q)$
and $(p,q-1)$-forms.
Then
\begin{align*}
*(L\^{n-k-j-1} v)
&= i^{p-q+1} (-1)^{\binom{k+2}2}
%L\^{n-(n-k-j-1)-k-1} v
L\^{j} v,
\\
*(L\^{n-k-j-1} Lw)
&= (n-k-j) {*}(L\^{n-k-j} w)
\\
&= (n-k-j) i^{p-q+1} (-1)^{\binom{k}2}
%L\^{n-(n-k-j)-(k-1)} w
L\^{j+1} w.
\end{align*}
All in all
\begin{align*}
\bar\partial^*L u
&=
(j+1) i (-1)^{\binom{k+1}2 + k}
\Bigl(
(-1)^{\binom{k+2}{2}} L\^{j} v
+ (-1)^{\binom{k}{2}} (n-k-j) L\^{j+1} w
\Bigr)
\\
&=
(j+1) i
\bigl(
L\^{j} v - (n-k-j) L\^{j+1} w
\bigr).
\end{align*}
% k(k+1)/2 + (k+1)(k+2)/2 = (k+1)(2k + 2)/2 = (k+1)^2 = k+1 (mod 2)
% k(k+1)/2 + k(k-1)/2 = k(2k)/2 = k^2 = k (mod 2)

On the other hand we have $L\bar\partial^* = -L*\partial*$ and
$$
\partial * u
= \partial i^{p-q} (-1)^{\binom{k+1}2} L\^{n-k-j} u
= i^{p-q} (-1)^{\binom{k+1}2} L\^{n-k-j} \partial u.
$$
As before we have $\partial u = v + Lw$ and
\begin{align*}
*(L\^{n-k-j} v)
&= i^{p-q+1} (-1)^{\binom{k+2}2}
%L\^{n-(n-k-j)-k-1} v
L\^{j-1} v,
\\
*(L\^{n-k-j} Lw)
&= (n-k-j+1) {*}(L\^{n-k-j+1} w)
\\
&= (n-k-j+1) i^{p-q+1} (-1)^{\binom{k}2}
%L\^{n-(n-k-j+1)-(k-1)} w
L\^{j} w.
\end{align*}
Then
\begin{align*}
\bar\partial^* u
&= i (-1)^{\binom{k+1}2+k}
\Bigl(
(-1)^{\binom{k+2}2} L\^{j-1} v
+ (n-k-j+1) (-1)^{\binom{k}2} L\^{j} w
\Bigr)
\\
&=
i \bigl(
L\^{j-1} v - (n-k-j+1) L\^{j} w
\bigr)
\end{align*}
and so
$$
L \bar\partial^* u
= i \bigl(
j L\^{j} v - (n-k-j+1)(j+1) L\^{j+1} w
\bigr).
$$
Put together, this gives
$$
\displaylines{
[\bar\partial^*, L] L\^j u
=
(j+1) i (-1)^{k+1}
\bigl( L\^{j} v - (n-k-j) L\^{j+1} w \bigr)
\hfill\cr\hfill{}
-
i
\bigl( j L\^{j} v - (n-k-j+1)(j+1) L\^{j+1} w \bigr)
\cr{}
\phantom{[\bar\partial^*, L] L\^j u}
= i (L\^j v + (j+1) L\^{j+1} w)
\hfill\cr{}
\phantom{[\bar\partial^*, L] L\^j u}
= i L\^j (v + L w)
= i L\^j \partial u
= i \partial(L\^j u).
\hfill
}
$$
Therefore $[\bar\partial^*, L] = i\partial$ by linearity.
\end{proof}


I thought I had come up with that and felt very clever, but the same idea is in
Huybrechts.

If we want to try to do the same on for a Hermitian metric, we get (when
calculating $\bar\partial^*L (L\^j u)$)
$$
\displaylines{
\partial {*}(L L\^{j}u)
= (j+1) i^{p-q} (-1)^{\binom{k+1}2}
\hfill\cr\hfill{}
\bigl(
(n-k-j-1) L\^{n-k-j-2} \partial \omega \wedge u
+ L\^{n-k-j-1} \partial u
\bigr).
}
$$
We have to apply $*$ to this and I don't know how to do that to $\partial
\omega \wedge u$.
Actually I don't know how to do it to $\partial u$ by itself either, because we
have that $L^{n-k+1} u = 0$ but only get
$$
0 =
L^{n-k+1} \partial u
+ (n-k+1) L^{n-k} \partial \omega \wedge u
= L^{n-k}( L\partial u + (n-k+1) \partial \omega \wedge u)
$$
so that sum has to play the role that $\partial u$ played before.
Before $u$ was a $k$-form and $\partial u$ a $(k+1)$-form, but now we have the
$(k+3)$-forms $L \partial u$ and $\partial \omega \wedge u$.
If they were primitive they'd live in the kernel of $L^{n-k-3}$, but we only
know their sum is in the kernel of $L^{n-k}$, so we should expect a four-term
expansion instead of a two-term one like before.

On top of that we only have a four-term expansion for
$$
L\partial u + (n-k+1) \partial \omega \wedge u
$$
but actually have
$$
\displaylines{
L\^{n-k-j-1} \partial u
+ (n-k-j-1) L\^{n-k-j-2} \partial \omega \wedge u
\hfill\cr\hfill{}
= L\^{n-k-j-2}\Bigl(
\frac{1}{n-k-j-1} L \partial u
+ (n-k-j-1) \partial \omega \wedge u
\Bigr)
}
$$
so after balancing things out we'd have a multiple of $\partial \omega \wedge
u$ to deal with.


\section{11. November 2022}

Recall that if $h$ is a Hermitian metric on $E$, then the curvature form of
$e^\psi h$ is
$$
\frac i2 \Theta_{e^\psi h}
= - \frac i2 \partial \bar\partial \psi \otimes \id_E
+ \frac i2 \Theta_{h}.
$$
Taking $X \subset \kk C^n$ and $h$ flat on $T_X$ we get a Hermitian metric with
curvature tensor
$$
R(\alpha, \ov\beta, \gamma, \ov\delta)
= -H(\psi)(\alpha, \ov\beta) h(\gamma, \ov\delta),
$$
where $H(\psi)$ is the Hessian of $\psi$.
We'll write $h$ for the conformal metric here and $\omega$ for its K\"ahler
form (which is not closed if $n > 1$).
The Ricci forms of the metric are then
$$
-n \frac i2 \partial \bar\partial \psi,
\quad
-\frac i2 \partial \bar\partial \psi,
\quad
- \Delta_{h} \psi \cdot \omega,
$$
which are all different.
They give two different scalar curvatures:
$$
-n \Delta \psi
\qandq
- \Delta \psi.
$$


\section{9. November 2022}

Is there any nice way of viewing curvature forms on a Hermitian or K\"ahler
manifold that makes the various Ricci forms pop out?

Let $(X,h)$ be a Hermitian manifold. The curvature form of $h$ is $F \xi = D^2
\xi$, where $D$ is the Chern connection of $h$.
This is a $(1,1)$-form that takes values in $\End T_X$, that is, a section of
$T_X^* \otimes \ov T_X^* \otimes T_X^* \otimes T_X$.
If $h$ is K\"ahler then we can swap the first and third spaces.
Using the metric we get an isomorphism
$$
\begin{tikzcd}
T_X^* \otimes \ov T_X^* \otimes T_X^* \otimes T_X
\ar[r] &
T_X^* \otimes T_X \otimes T_X^* \otimes T_X.
\end{tikzcd}
$$
On the right-hand side we can pair off $T_X^*$ and $T_X$ in $2 \times 2 = 4$
different ways and take a trace.

Denoting the traces by $\tau_{jk}$ for $j \in \{1, 3\}$ and $k \in \{2, 4\}$,
being K\"ahler tells us that $\tau_{12} = \tau_{32}$ and $\tau_{14} = \tau_{34}$.
Does being K\"ahler imply we can swap the second and fourth places as well?
It should.
If so we get $\tau_{12} = \tau_{32} = \tau_{34} = \tau_{14}$.

The $\End T_X$ part of the curvature tensor is Hermitian, which should imply
some kind of symmetry. (So is the $T_X^* \otimes \ov T_X^*$ part.)
For the latter, there is a map given by the composite
$$
T_X^* \otimes \ov T_X^* \stackrel{\text{conj}}{\longrightarrow}
\ov T_X^* \otimes T_X^* \stackrel{\text{swap}}{\longrightarrow}
T_X^* \otimes \ov T_X^*
$$
under which the curvature form is invariant
which should imply that $\tau_{12}$ is real.

\section{8. November 2022}

There was a fun paper on the arXiv today: https://arxiv.org/abs/2211.03469

In it they review a construction of Donaldson.
Let $(X,\omega)$ be a K\"ahler manifold.
Then there exists a complex manifold $p : Z \to X$ called the extension of $X$
such that $p^*[\omega] = 0$.
It satisfies the universal property that if $f : Y \to X$ is any manifold with
this property, then $f$ factors through $Y \to Z \to X$.

There are a couple of different constructions of $Z$, one being by looking at
the extension
$$
0 \longrightarrow
\Omega_X^1 \longrightarrow
E \stackrel{q}{\longrightarrow}
\mathcal O_X \longrightarrow 0
$$
that $[\omega]$ defines and letting $Z = p^{-1}(1)$.
This is where the ``extension'' name comes from.

It is conjectured that $T_X$ is nef if and only if $Z_X$ is Stein.
The paper on the arXiv proves that $T_X$ nef implies $Z_X$ Stein, but has to
assume that a conjecture of Campana and Peternell is true to do that.
That $Z_X$ Stein implies $T_X$ nef is apparently wide open (known for curves
and most projective surfaces).

It's fun to think how far we can get towards doing anything to $\cc O(1) \to
\kk P(T_X)$ by using the universal property.
We have the projective bundle $\pi : \kk P(T_X) \to X$ and know that
$\pi^*[\omega] \not= 0$.
Then we get an extension $f : Z_{\kk P(T_X)} \to \kk P(T_X)$ and a commutative diagram
$$
\begin{tikzcd}
Z_{\kk P(T_X)}
\ar[r,"g"] \ar[d,"f"]
&
Z_X
\ar[d,"p"]
\\
\kk P(T_X)
\ar[r, "\pi"]
&
X.
\end{tikzcd}
$$
We know that $\pi$ is a fibration and can show that $f$ and $p$ are as well.
As $dp \circ dg$ is then surjective, this only tells us that $dp$ is
surjective, which we knew.
Oh well.

Suppose now that $Z_X$ is Stein.
The fibers of $g$ are probably an affine manifold, thus Stein.
There's a paper on the arXiv\footnote{https://arxiv.org/abs/0705.1715}
that says that
if in addition $g$ is a fibration and there's a neighborhood $U$ around every
point in $Z_X$ such that $g^{-1}(U)$ is Stein, then $Z_{\kk P(T_X)}$ is Stein.
But we don't know if $g$ is a fibration.
This would also imply that $T_{\kk P(T_X)}$ is nef, which may be suspect.


\section{7. November 2022}
\label{sum-of-positive-hsc-is-not-positive-hsc}

If $h$ is a Hermitian metric and $H$ is its matrix in a local frame, then the
connection and curvature are
$$
D's = H^{-1}\partial H \wedge s
$$
and
$$
F s = - H^{-1} \bar\partial H \wedge H^{-1} \partial H \wedge s
- H^{-1} \partial \bar\partial H \wedge s.
$$
We have $H^{-1} = \adj H / \det H$, so
$$
\det(H) D's = \adj(H) \partial H \wedge s
$$
and
$$
\det(H)^2 Fs
= - \adj(H) \bar\partial H \wedge \adj(H) \partial H s
- \det(H) \adj(H) \partial\bar\partial H s.
$$
Then
\begin{align*}
\det(H)^2 h(Fs, \ov t)
&= - t^\dagger \! H \adj(H) \bar\partial H \adj(H) \partial H s
- t^\dagger \! H \det(H) \adj(H) \partial\bar\partial H s
\\
&= - \det(H) t^\dagger \bar\partial H \wedge \adj(H) \partial H s
- \det(H)^2 t^\dagger \partial\bar\partial H s
\end{align*}
I don't think this helps. $\det H$ isn't a tensor.


\medskip

Let's do something different and find an example of positive hsc metrics whose
sum isn't like that, in complex dimension 1.
Let $U$ be a neighborhood around $0 \in \kk C$.
If $h$ is a Hermitian metric on $U$ then $h(\xi, \ov\eta) = e^\psi \xi \ov\eta$.
We have
$$
\partial h(\xi, \ov\eta)
= \partial \psi h(\xi, \ov\eta) + h(\partial \xi, \ov\eta)
= h(\partial \psi \otimes \xi + \partial \xi, \ov\eta)
$$
so $D\xi = \partial \psi \otimes \xi + d \xi$.
Then
\begin{align*}
D^2 \xi
&= \partial \psi \wedge (\partial \psi \otimes \xi + d \xi)
+ d (\partial \psi \otimes \xi + d \xi)
\\
&= \partial \psi \wedge \bar\partial \xi
- \partial \bar\partial \psi \otimes \xi
- \partial \psi \wedge \bar\partial\xi
= - \partial \bar\partial \psi \otimes \xi
\end{align*}
so the curvature form is $\frac{i}{2} \Theta \xi = - \frac i2 \partial
\bar\partial \psi \otimes \xi$
and the curvature tensor is
$$
R\Bigl(\frac{\partial}{\partial z},
\frac{\partial}{\partial \bar z},
\frac{\partial}{\partial z},
\frac{\partial}{\partial \bar z}
\Bigr)
= - \frac{\partial^2 \psi}{\partial z \partial \bar z} e^\psi.
$$

Now let $g = e^\psi$ and $h = e^\phi$ be metrics.
We have
$$
\sigma(\xi)
= D_{g} \xi - D_h \xi
= (\partial \psi - \partial \phi) \otimes \xi.
$$
The quotient metric $q$ is
$$
q
= g (g+h)^{-1} h + h (g+h)^{-1} g
= \frac{2 e^\psi e^\phi}{e^\psi + e^\phi}.
$$
The curvature tensor of the sum is then
$$
R_{g+h}
= R_g + R_h - \sigma^* q
= - \frac{\partial^2 \psi}{\partial z \partial \bar z} e^\psi
- \frac{\partial^2 \psi}{\partial z \partial \bar z} e^\psi
- \Bigl|
\frac{\partial \psi}{\partial z} - \frac{\partial \phi}{\partial z}
\Bigr|^2
\frac{2 e^\psi e^\phi}{e^\psi + e^\phi}.
$$
If we pick $\psi$ and $\phi$ such that $\psi(0) = \phi(0) = 0$ this
simplifies to
$$
R_{g+h}
= - \frac{\partial^2 \psi}{\partial z \partial \bar z}
- \frac{\partial^2 \psi}{\partial z \partial \bar z}
- \Bigl|
\frac{\partial \psi}{\partial z} - \frac{\partial \phi}{\partial z}
\Bigr|^2
$$
there.

Let $f(z) = a_0 (z + \bar z) + \sum_{n \geq 1} a_n z^n \bar z^n$.
Then
$$
\frac{\partial f}{\partial z} = a_0 + o(|z|)
\qandq
\frac{\partial^2 f}{\partial z \partial \bar z}
= a_1 + o(|z|^2),
$$
so if $a_1 < 0$ the metric $e^f$ has positive hsc at $0$.

Let's take $\psi(z) = \exp(a_0(z + \bar z) - |z|^2)$ and $\phi(z) = \exp(-
|z|^2)$, so
$$
R_{g+h} = 2 - |a_0|^2
$$
at $0$, which can be arbitrarily negative.



\section{6. November 2022}

Maybe we can recover some of what we want by density arguments.
Chern connections and curvatures exist for non-degenerate forms, which are
dense in the space of all forms.
I think this is generally true:

Let $E \to X$ be a smooth vector bundle with fiber $V$.
Let $U \subset V$ be a dense open set.
The set of sections $s$ of $E$ whose values are in $U$ is dense in the space of
all sections of $E$.

To prove that we should reduce to a product $E = X \times V$ locally, find a
suitable compact set inside $X$ (one with open interior), and should be able to
conclude by a norm argument and then glue with a partition of unity.


\medskip

Let $\theta(g,h) = g (g+h)^{-1} h + h (g+h)^{-1} g$ for Hermitian forms $g,h$
with $g+h$ invertible.
This is a smooth function from a subset of the space of Hermitian forms to
itself.
We have $\lim_{g \to 0} \theta(g,h) = 0$.
I claim that $\lim_{g \to \infty} \theta(g,h) = 2 h$ if $g$ is invertible.
This is maybe too much. We only use that
$$
\lambda g (\lambda g + h)^{-1} h + h (\lambda g + h)^{-1} \lambda g
= g (g + h/\lambda)^{-1} h + h (g + h / \lambda)^{-1} g
$$
and $g + h/\lambda \to g$ so
$$
\theta(\lambda g,h)
\to g g^{-1} h + h g^{-1} g = 2 h.
$$
The general limit involves a term $h g^{-1}(\id + g^{-1}h)^{-1}g$ and I don't
know how to argue that this converges to $h$.

When $g$ is not invertible we can approximate it by things that are, but the
limit should involve the kernel of $g$.
We should split $V = \Ker g \oplus V / \Ker g$ and apply the above on $V / \Ker g$.
Then $\theta(\lambda g, h) \to 2 h_{| \Ker g^\perp}$.



\medskip

I want to prove that if $g$ has positive hsc and $h$ is any form then
$e^\lambda g + h$ eventually has positive hsc.
This is actually the same as proving that $g + h$ has positive hsc for $h$
small enough.
But that just follows from the continuity of the curvature tensor.



\medskip

Let $Y \subset X$ be smooth and let $\mu : \hat X \to X$ be the blowup of $X$
along $Y$.
If $h_X$ has positive hsc then $\mu^* h_X$ has positive hsc on $\hat X \setminus E$.
Over $E$ we have
$$
0 \longrightarrow
T_E \longrightarrow
T_{\hat X|E} \longrightarrow
N_{E/\hat X} \longrightarrow
0
$$
and
the divisor $E$ identifies with the total space of $p:\kk P(N_{Y/X}) \to Y$.
We then have a short exact sequence
$$
0 \longrightarrow
T_{\kk P(N_{Y/X}/Y)} \longrightarrow
T_{\kk P(N_{Y/X})} \longrightarrow
p^* T_{Y} \longrightarrow
0.
$$
Let $b$ be an extension of the Fubini--Study metric on the fibers of $\kk
P(N_{Y/X})$ to a small relatively compact neighborhood of $E$.
We consider $h_\lambda = \mu^* h_X + e^{-\lambda} b$ on $\hat X$.

For $\lambda$ small enough this is a metric on $\hat X$ and has positive hsc on
$\hat X \setminus E$.
On $T_{\kk P(N_{Y/X}/Y)} = \Ker \mu_*$ the form $\mu^* h_X$ is zero, so
$h_\lambda$ identifies with the relative Fubini--Study metric there, which has
positive hsc.






\section{3. November 2022}

Richard pointed out the example $h(z) = |z|^2$ on the complex plane, which is
degenerate at $0$ and must have the Chern connection $\frac 1z dz$ outside of
$0$, which can't be extended to $0$.
This torpedos the results I thought I had proved and I don't think they can be
salvaged.
I'm now confused about what people proved in the Riemannian case.

It might be worth looking into whether we can prove some of the corollaries
anyway, with nasty local calculations.
Does a blow-up of positively curved things have positive curvature?

Let $V$ be a vector space. The blow-up of $V$ at $0$ is
$$
\hat V = \{ (z, l) \in V \times \kk P(V) \mid z \in l \}
$$
equipped with the holomorphic map $\mu : \hat V \to V$ induced by the
projection $V \times \kk P(V) \to V$.
We can also identify $\hat V$ with the total space of the line bundle $p : \cc
O(-1) \to \kk P(V)$.

If $V$ (or, more usefully, a neighborhood of $0$ in $V$) has a metric with
positive hsc, then $V \times \kk P(V)$ also has such a metric because products
of manifolds are easy enough to handle.
So we're asking whether the submanifold $\hat V \subset V \times \kk P(V)$ also
has such a metric, which is far from automatic.

Let $E = \{0\} \times \kk P(V) = \mu^{-1}(0)$ be the exceptional divisor, and
let $U = \mu^{-1}(B_\epsilon(0))$ be a tubular neighborhood around it.

Let $U_j = \{ [w] \in \kk P(V) \mid w_j \not= 0 \}$ be one of the standard
neighborhoods in $\kk P(V)$. The Fubini--Study metric on $U_j$ is
$$
h_{FS}(\alpha, \ov\beta; w)
= \frac{\langle \alpha, \ov\beta \rangle}{1+|w|^2}
- \frac{\langle \alpha, \ov w \rangle}{1+|w|^2}
\frac{\langle w, \ov \beta \rangle}{1+|w|^2}
= -\frac i2 \partial \bar\partial \log(1+|w|^2)(\alpha, \ov\beta).
$$
We have
$$
\frac i2 \partial \bar\partial |z|^2(\alpha, \ov\beta)
= \langle \alpha, \ov\beta \rangle,
$$
so $\log \bigl( e^{|z|^2} / (1 + |w|^2) \bigr)$ is a potential for a K\"ahler
metric on $V \times \kk P(V)$.
We care because if $\theta$ is a bump function supported on $U_\epsilon$, then
$\log \bigl( \theta(z) e^{|z|^2} / (1 + |w|^2) \bigr)$
is a potential for a closed $(1,1)$-form that is a K\"ahler metric around $E$.

We're proving that if $h$ is a K\"ahler metric on $E$, then there exists a
tubular neighborhood $U$ around $E$ and a closed Hermitian form $\tilde h$ on
$\hat V$ such that $\tilde h$ is a K\"ahler metric on $U$ and its restriction
to $E$ is equal to $h$.
This is not automatic; we're showing that the restriction morphism
$$
H^{1,1}(\hat V,\kk R) \to H^{1,1}(E, \kk R)
$$
is surjective, which it isn't always (take a high-degree hypersurface in three-dimensional projective space).

It would be much better to prove that there exists a K\"ahler metric on $\hat
V$ that restricts to $h$ on $E$.



\section{2. November 2022}

If $g$ and $h$ are Hermitian metrics on $X$ then
\[
R_{g + h} = R_g + R_h - \sigma^*q,
\]
so
$
H_{g+h}(\xi) \leq H_g(\xi)|\xi|_g^4 + H_h(\xi) |\xi|_h^4
$
on the $(g+h)$-unit sphere $S_{g+h}$.
If we integrate this over the $(g+h)$-unit sphere we get a multiple of the
scalar curvature of $g+h$ on the left-hand side.
The multiple is by a constant $C = C(\dim X)$.

We have
$$
\int_{S_g(r)} H_g(\xi) d\mu_r
= \int_{S_g} H_g(r \eta) d\mu(r \eta)
= r^{2n+3} \int_{S_g} H_g(\eta) d\mu
= C r^{2n+3}.
$$
Then
\begin{align*}
\int_{S_{g+h}} H_{g}(\xi) |\xi|^4_g d\mu
&= \int_m^M \int_{S_g(r)} H_g(\eta) d\mu_r(\eta) dr
\\
&= \int_m^M r^{2n+3} \int_{S_g} H_g(\xi) d\mu(\xi) dr
\\
&= C s_g \int_m^M r^{2n+3} dr
= C s_g \frac{1}{2(n+2)}(M^{2n+4} - m^{2n+4}),
\end{align*}
where
$$
0 < m := \inf_{S_{g+h}} |\xi|_g \leq M := \sup_{S_{g+h}} |\xi|_g < 1.
$$
Note that
$$
\inf_{S_{g+h}} |\xi|_h = \sqrt{1-m^2}
\qandq
\sup_{S_{g+h}} |\xi|_h = \sqrt{1-M^2}.
$$
Integrating $H_h$ over the unit sphere and adding the two then gives
$$
2(n+2) s_{g+h}
\leq
\bigl(b^{n+2} - a^{n+2}\bigr) s_g
+ \bigl((1-a)^{n+2} - (1-b)^{n+2}\bigr) s_h,
$$
where $a = m^2$ and $b = M^2$.

We have
$$
\frac{a^n - b^n}{a - b}
= \sum_{k=0}^{n-1} a^{n-1-k} b^k
\geq (n-1) b^{n-1},
$$
so if both curvatures are negative we get
\begin{align*}
\frac{2(n+2)}{n+1} s_{g+h}
&\leq
(a-b)(b^{n-1} s_g + (1-a)^{n-1} s_h)
\\
&\leq
(a-b) \min\{1-a, b\}^{n-1}(s_g + s_h).
\end{align*}

Set
$$
f(a, b) = b^n - a^n + (1-a)^n - (1-b)^n.
$$
We have $f_a = -n a^{n-1} - n(1-a)^{n-1}$ and $f_b = n b^{n-1} + n(1-b)^{n-1}$
so the gradiant of $f$ is
$$
\nabla f = n \begin{pmatrix}
- a^{n-1} - (1-a)^{n-1}
\\
b^{n-1} + (1-b)^{n-1}
\end{pmatrix}
$$
and Hessian of $f$ is
$$
H(f) =
n(n-1)
\begin{pmatrix}
(1-a)^{n-2} - a^{n-2} & 0
\\
0 & b^{n-2} - (1-b)^{n-2}
\end{pmatrix}.
$$
The gradient is zero when $a^{n-1} + (1-a)^{n-1} = 0$ and same for $b$. As
$0 < a, b < 1$, this doesn't happen inside $[0,1] \times [0,1]$.
We also have the constraint $a \leq b$, and $f(a,a) = 0$.
Other extrema have to happen on the boundary on that triangle. We have
$f(0,b) = b^n - (1-b)^n$, which should take its maximum at $f(0,\frac12) =
1/2^{n-1}$, and $f(a,1) = (1-a)^n - a^n$, which same. So
$$
0 \leq f(a,b) \leq \frac 1{2^{n-1}}
$$
and both bounds are optimal.

I had already calculated this at some point and gotten the same identities, but
with the contribution of the second fundamental form written as $\langle
\sigma(\xi), \ov{\sigma(\eta)} \rangle$ and $|\sigma|^2$.


\section{31. October 2022}

Suppose that $g$ and $h$ have positive holomorphic sectional curvatures.
Then
\[
R_{g + h} = R_g + R_h - \sigma^*q,
\]
where $\sigma = D_{g} - D_{h}$.
This is the difference between two Chern connections, so $\sigma$ is a
$(1,0)$-form with values in $\operatorname{End} E$.
Let $\xi$ be a given tangent field of $X$.
It either is or isn't in the kernel of $\sigma(\xi)$, no matter what frame we pick.

Can we calculate $\partial \bar\partial \log |\xi|^2$? Anything there?

What's the Chern connection of $q$?


Here we go again:
\begin{align*}
\partial \bar\partial \log |\xi|^2
= \partial \frac{h(\xi, \ov{D\xi})}{|\xi|^2}
&= \frac{h(\xi, \ov{F\xi})}{|\xi|^2}
+ \frac{h(D\xi, \ov{\xi})}{|\xi|^2} \wedge \frac{h(\xi, \ov{D\xi})}{|\xi|^2}
\\
&= - \frac{h(F\xi, \ov{\xi})}{|\xi|^2}
+ \frac{h(D\xi, \ov{\xi})}{|\xi|^2} \wedge \frac{h(\xi, \ov{D\xi})}{|\xi|^2},
\end{align*}
so (one day I really have to figure out how these signs work out)
\[
\frac i2 \partial_{\xi} \bar\partial_{\ov\xi} \log|\xi|^2
= - H(\xi) |\xi|^2 + \frac{1}{|\xi|^4} \bigl| h(D_\xi \xi, \ov\xi) \bigr|^2.
\]
So at the center of a normal coordinate system where one of the tangent fields
at the origin is $\xi$ we get
\[
\frac i2 \partial_{\xi} \bar\partial_{\ov\xi} \log|\xi|^2
= - H(\xi) |\xi|^2.
\]
Likewise we should be able to arrange things so that
\[
\Delta \log |\xi|^2 = - \frac{\operatorname{Ric}(\xi, \ov\xi)}{|\xi|^2}
\]
at the center of a normal coordinate system, which is probably just Bochner.


Anyway.
Look at $b := (g+h)^{-1}g^*h$.
We have
$$
b(\xi,\ov\eta) = h((g+h)^{-1}g(\xi), \ov{(g+h)^{-1}g(\eta)}).
$$
Then
$$
\displaylines{
\partial b(\xi,\ov\eta)
= \partial h((g+h)^{-1}g(\xi), \ov{(g+h)^{-1}g(\eta)})
\hfill\cr\hfill{}
= h(D_h' (g+h)^{-1}g(\xi), \ov{(g+h)^{-1}g(\eta)})
+ h((g+h)^{-1}g(\xi), \ov{\bar\partial (g+h)^{-1}g(\eta)}).
}
$$
Now
$$
\displaylines{
D_h'(g+h)^{-1}g(\xi)
= -(g+h)^{-1}(D_h' g + D'_h h) (g+h)^{-1} g(\xi)
\hfill\cr\hfill{}
+ (g+h)^{-1} (D'_h g) (\xi)
+ (g+h)^{-1} g (D'_h \xi).
}
$$
We have $D_h h = 0$, and
$$
\displaylines{
-(g+h)^{-1}(D'_h g) (g+h)^{-1} g
+ (g+h)^{-1} (D'_h g)
\hfill\cr{}
\qquad
= -(g+h)^{-1}(D'_h g) (g+h)^{-1} g
+ (g+h)^{-1} (D'_h g) (g+h)^{-1} (g + h)
\hfill
\cr{}
\qquad
= (g+h)^{-1} (D'_h g) (g+h)^{-1} h .
\hfill
}
$$
So
$$
\partial b(\xi, \ov\eta)
= b((g^{-1} (D_h g) (g+h)^{-1} h) \xi, \ov\eta)
+ b(D_h \xi, \ov\eta)
$$
(I think, I haven't accounted for the $\bar\partial$).
Now do the same with $(g+h)^{-1}h^*g$ and add up.

This might be slightly crazy.
Griffiths tells you how to calculate $D_{g+h}$ and $D_q$ from $D_g \oplus D_h$
and the second fundamental form. Why not do that?
In fact
$$
D_{g + h, \xi} s
= j^\dagger (D_{g, \xi} s \oplus D_{h, \xi} s)
= (g+h)^{-1}(g(D_{g, \xi} s) + h(D_{h, \xi} s)).
$$
Then $D_q s = \pi(D_{g} \oplus D_h (\pi^\dagger s))$, which is something.


Anyway.
As usual $0 \to S \to E \to Q \to 0$.
We have $D_S s = j^\dagger D_E(js)$.
Then
\begin{align*}
D^2_S s = j^\dagger D_E(j j^\dagger D_E (js))
&= j^\dagger D_E(D_E(js) - q^\dagger q D_E(js))
\\
&= j^\dagger D^2_E(js) - j^\dagger D_E( q^\dagger \sigma(s))
\\
&= j^\dagger D^2_E(js) - j^\dagger (D q^\dagger) \wedge \sigma(s)
\\
&= j^\dagger D^2_E(js) + \sigma^\dagger \wedge \sigma(s)
\end{align*}
because $j^\dagger q^\dagger = 0$ and $D q^\dagger = -j \sigma^\dagger$.


The form $\sigma$ is made up of pieces of the form $f \otimes \alpha$, where
$\alpha$ is a $(1,0)$-form, and $\sigma^\dagger$ of ones of the form $g^\dagger
\otimes \ov\beta$, where $\beta$ is a $(1,0)$-form. Then
$$
\tfrac i2 \sigma^\dagger \wedge \sigma
= \tfrac i2 g^\dagger \otimes \ov\beta \wedge f \otimes \alpha
= - g^\dagger f \otimes \tfrac i2 \alpha \wedge \ov\beta.
$$
Evaluating this at tangent fields $\xi, \eta$ and sections $s,t$ we get
\begin{align*}
\langle \tfrac i2 \sigma^\dagger \wedge \sigma(\xi, \ov\eta) s, \ov t \rangle
&= -\langle g^\dagger f s \cdot \alpha(\xi) \ov{\beta(\eta)}  , \ov t \rangle
\\
&= - \langle \alpha(\xi) fs, \ov{\beta(\eta) g t} \rangle
= - \langle \sigma(\xi) s, \ov{\sigma(\eta) t} \rangle.
\end{align*}
Thus
$$
R_S = j^*R_E - \sigma^* h_Q.
$$

Is this correct? Is $\alpha \wedge \ov\beta(\xi, \ov\eta) = \alpha(\xi)
\ov{\beta(\eta)}$? Or is it the imaginary part of that?



\section{30. October 2022}

Wu \cite{wu1973remark} proved that if $g$ and $h$ are Hermitian metrics on a
complex manifold and their holomorphic sectional curvatures satisfy $H_g \leq
-K_g < 0$ and $H_h \leq -K_h < $ then the holomorphic sectional curvature of
their sum satisfies
\[
H_{g + h} \leq \frac{-K_g K_h}{K_g + K_h} < 0.
\]
He does this by proving that the holomorphic sectional curvature in the
direction of a tangent field $\xi$ at $p$ can be characterized as the maximum
of curvatures at the center of analytic disks $\gamma : D \to X$ such that
$\gamma(0) = p$ and $\gamma'(0) = \xi$ and then invoking the same theorem in
dimension 1, which had been proven by Grauert and Reckziegel.

We know that
\[
R_{g + h} = R_g + R_h - \sigma^* q
\]
and that $q$ is positive-definite.
Then
\[
H_{g+h}(\xi)
\leq \frac{H_g(\xi) |\xi|^4_g + H_h(\xi) |\xi|^4_h}{|\xi|^4_{g + h}}
= \frac{H_g(\xi) |\xi|^4_g + H_h(\xi) |\xi|^4_h}{(|\xi|^2_{g} + |\xi|^2_h)^2}.
\]
Thus $H_{g + h} \leq 0$ if $H_g \leq 0$ and $H_h \leq 0$.

As Wu notes,
for any positive real numbers $a, b, x, y$ we have
\[
\frac{xy}{x + y} \leq \frac{a^2 x + b^2 y}{(a + b)^2}
\]
by elementary algebra.
The inequality is reversed if $x$ and $y$ are negative.
If the holomorphic sectional curvatures of $g$ and $h$ are negative we thus get
\[
H_{g + h} \leq \frac{H_g H_h}{H_g + H_h} < 0.
\]

For positively curved metrics we get
\[
H_{g + h}(\xi)
\geq \frac{H_g H_h}{H_g + H_h}
- \frac{|D_{g,\xi}\xi - D_{h,\xi}\xi|^2_q}
{(|\xi|_g^2 + |\xi|_h^2)^2}.
\]
Can we set $g = \lambda g$ and take a limit as $\lambda \to \infty$? We have
$q_\lambda \to g$ as $\lambda \to \infty$ so the negative term tends to zero and the positive term tends to $H_g > 0$.

\section{26. October 2022}

Let $\pi : X \to B$ be a projective bundle over a compact base.
Suppose we only have $X$.
Can we figure out that $B$ and $\pi$ exist at all?
Or what they are?

To put it another way:
Let $X$ be a compact K\"ahler manifold with $-K_X$ nef.
Suppose that $X$ admits a K\"ahler metric with positive holomorphic sectional
curvature.
Is $X$ the total space of a Grassmannian bundle?

What can you even try to fiber it over?
The Albanese variety can be trivial.




\section{11. October 2022}

The statement is that rational points are \emph{Zariski} dense.
A lot of things are Zariski dense.
On a curve, any infinite set is Zariski dense.
On a variety, any set that is open in the classical topology is Zariski dense;
otherwise it would be contained in the zero set of some holomorphic function,
which would then be zero on an open set.


\section{7. October 2022}


I really feel like this is true:


\begin{prop}
Let $Y \subset \kk P^{n}$ be a smooth complex hypersurface defined by a homogeneous polynomial  $f$ with real coefficients.
Let $\sigma$ be the second fundamental form of $Y$ in~$\kk P^{n}$.
If real points are dense in $Y$, then $\sigma = 0$.
\end{prop}

\begin{proof}
We work in homogeneous coordinates, on the affine space that covers $\kk P^{n}$.
The tangent bundle of $Y$ is
\[
  T_{Y} = \Ker df
  = \biggl\{
  \xi \in T_{\kk P^{n}}
  \Bigm| \sum_{j=0}^{n} \xi_{j} \frac{\partial f}{\partial z_{j}} = 0
  \biggr\}.
\]
As $\bar f = f$ at a real point, we have $T_{Y} = (df^{\sharp})^{\perp}$ there,
where
$$
df^{\sharp} =
\smash{\sum_{j=0}^{n} \frac{\partial f}{\partial z_{j}} \frac{\partial}{\partial z_{j}}}
$$
and the inner product is the standard one on $\kk C^{n+1}$.
In particular, $N_{Y/\kk P^{n}}$ is spanned by $df^{\sharp}$ at such a point.

The Fubini--Study metric on $\kk P^{n}$ is
\[
  \langle \xi, \ov\nu \rangle_{FS}
  = \frac{\langle \xi, \ov\nu \rangle}{|\eta|^{2}}
  - \frac{\langle \xi, \ov\eta \rangle}{|\eta|^{2}}
  \frac{\langle \eta, \ov\nu \rangle}{|\eta|^{2}},
\]
where $\eta = \sum_{j=0}^{n} z_{j} \frac{\partial}{\partial z_{j}}$ is the Euler field.
If we're at a real point and take $\xi \in T_{Y}$, we then have
\[
  \langle \xi, \ov{df^{\sharp}} \rangle_{FS}
  = \frac{\langle \xi, \ov{df^{\sharp}} \rangle}{|\eta|^{2}}
  - \frac{\langle \xi, \ov\eta \rangle}{|\eta|^{2}}
  \frac{\langle \eta, \ov{df^{\sharp}} \rangle}{|\eta|^{2}}
  = 0
\]
because $\xi \in (df^{\sharp})^{\perp}$ and the Euler field satisfies $\eta \cdot f = (\deg f) f = 0$ on the zero set of a homogeneous complex polynomial.

The Chern connection of the Fubini--Study metric on $\kk P^{n}$ is
\[
  D_{\nu} \xi = d_{\nu}\xi
  - \frac{\langle \xi, \ov\eta\rangle}{|\eta|^{2}} \nu
  - \frac{\langle \nu, \ov\eta\rangle}{|\eta|^{2}} \xi.
\]
The second fundamental form of $Y$ is then
\[
  \sigma(\xi, \nu)
  = q(D_{\nu}\xi),
\]
where $q : T_{\kk P^{n}|Y} \to T_{\kk P^{n}|Y}$ is the orthogonal projection onto $T_{Y}^{\perp}$.
In fact, we have
\[
  \sigma(\xi, \nu)
  = \frac{\langle D_{\nu} \xi, \ov{df^{\sharp}} \rangle_{FS}}{|df^{\sharp}|_{FS}^{2}} df^{\sharp}
\]
at real points of $Y$.

Now suppose real points are dense in $Y$.
Then $\langle \xi, \ov{df^{\sharp}} \rangle_{FS} = 0$ in a dense set in a neighborhood around our point of interest, and thus on the whole neighborhood.
We then get
\[
  0 = \partial_{\nu} \langle \xi, \ov{df^{\sharp}} \rangle_{FS}
  = \langle D_{\nu}\xi,  \ov{df^{\sharp}} \rangle_{FS}
  + \langle \xi,  \ov{\bar\partial_{\ov\nu}df^{\sharp}} \rangle_{FS}
  = \langle D_{\nu}\xi,  \ov{df^{\sharp}} \rangle_{FS},
\]
where the last equality holds because the field $df^{\sharp}$ is holomorphic by inspection.
But then $\sigma(\xi, \nu) = 0$ at the point of interest.
\end{proof}


\begin{coro}
If $Y \subset \kk P^{n}$ is a hypersurface with dense real points, then $Y \cong \kk P^{n-1}$.
\end{coro}

\begin{proof}
By the above, Codazzi--Griffiths implies that $R_{Y} = R_{FS|Y}$, so $T_{Y}$ is Griffiths-positive, and $Y$ is therefore isomorphic to $\kk P^{n-1}$.
\end{proof}

This conclusion is nuts because real points are not dense in $\kk P^{n-1}$; they form a real subvariety of real dimension $n-1$.

\begin{exam}
Consider the circle $C(k) = \{(x,y) \in k^{2} \mid x^{2} + y^{2} = 1\}$ over a field $k$, where we'll care about $k = \kk Q, \kk R, \kk C$.
We know that the rational points of $C(\kk Q)$ are dense in $C(\kk R)$.
Take now a point like $(i,\sqrt 2)$ in $C(\kk C)$.
If $(x,y) \in  C(\kk C)$ is a real point, then
\[
  |(x,y) - (i,\sqrt 2)|
  \geq ||(x,y)| - |(i,\sqrt 2)||
  = \sqrt 3 - 1 > 0
\]
so rational or real points are not dense in $C(\kk C)$.
\end{exam}


The contribution of the second fundamental form to the curvature tensor of $Y$ is
\[
  \langle \sigma(\alpha, \beta), \ov{\sigma(\gamma, \delta)} \rangle_{FS}
  = \frac{\langle d_{\alpha}\beta, \ov{df^{\sharp}} \rangle_{FS}
  \langle df^{\sharp}, \ov{d_{\gamma}\delta} \rangle_{FS}}{|df^{\sharp}|^{2}_{FS}}
  = \frac{\langle d_{\alpha}\beta, \ov{df^{\sharp}} \rangle
  \langle df^{\sharp}, \ov{d_{\gamma}\delta} \rangle }{|df^{\sharp}|^{2}|\eta|^{2}}
\]
at a real point. Its contribution to the holomorphic sectional curvature is then
(TODO: Correct for missing $|df^{\sharp}|^{2}$ factors)
\begin{align*}
  \frac{\langle \sigma(\alpha, \alpha), \ov{\sigma(\alpha, \alpha)} \rangle_{FS}}{|\alpha|^{4}_{FS}}
  &= \frac{\langle d_{\alpha}\alpha, \ov{df^{\sharp}} \rangle}{|\eta|^{2}}
  \frac{\langle df^{\sharp}, \ov{d_{\alpha}\alpha} \rangle}{|\eta|^{2}}
  \frac{|\eta|^{4}}{|\alpha|^{4}}
    \\
  &= \frac{|\langle d_{\alpha}\alpha, \ov{df^{\sharp}} \rangle|^{2}}{|\alpha|^{4}}
  \leq \frac{|d_{\alpha}\alpha|^{2} |df^{\sharp}|^{2}}{|\alpha|^{4}}.
\end{align*}
If $|\alpha|^{2}_{FS} = 1$ then $|\alpha|^{2} = |\eta|^{2}$.
If $\alpha = \sum_{j} \alpha_{j} \partial / \partial z_{j}$ then
\[
  d_{\alpha} \alpha
  = \sum_{j,k} \alpha_{k} \frac{\partial{\alpha_{j}}}{\partial z_{k}} \frac{\partial}{\partial z_{j}}
\]
so
\[
  |d_{\alpha} \alpha|^{2}
  \leq \sum_{j,k}
  \biggl|
  \alpha_{k} \frac{\partial{\alpha_{j}}}{\partial z_{k}}
  \biggr|^{2}
\]
I'm not sure this is going anywhere. Let's try something else.

If $\alpha \in T_{Y}$ then $\alpha \cdot f = 0$. We calculate that
\[
  \beta \cdot (\alpha \cdot f)
  = (\beta \cdot \alpha) \cdot f
  + \alpha^{t} H(f) \beta,
\]
where $H(f)$ is the complex Hessian of $f$ and
$\beta \cdot \alpha = d_{\beta}\alpha$ is the directional derivative of $\alpha$
in the direction of $\beta$, which is not an invariant thing.
For our $f$ and at a real point, we have $(\beta \cdot \alpha) \cdot f = \langle \beta \cdot \alpha, \ov{df^{\sharp}} \rangle$.
Then
\[
  \langle \sigma(\alpha, \beta), \ov{\sigma(\gamma, \delta)} \rangle_{FS}
  = \frac{\beta^{t}H(f)\alpha \; \ov{\delta^{t}H(f)\gamma}}
  {|df^{\sharp}|^{2}|\eta|^{2}}.
\]
If we take an orthonormal basis $(e_{1}, \ldots, e_{n})$, where $e_{n}$ is colinear with $df^{\sharp}$, then the contribution of this to the scalar curvature is
\[
  \sum_{j,k=1}^{n-1} \frac{e_{k}^{t}H(f)e_{j} \; \ov{e_{k}^{t}H(f)e_{j}}}
  {|df^{\sharp}|^{2}|\eta|^{2}}
 = \frac{1}{|df^{\sharp}|^{2}|\eta|^{2}}
 \sum_{j,k=1}^{n-1} |H(f)_{jk}|^{2}
 \leq \frac{|H(f)|^{2}}{|df^{\sharp}|^{2}|\eta|^{2}},
\]
where $|H(f)|$ is the Frobenius norm.


\begin{exam}
Let $f(z,w) = z^{d} + w^{d}$.
Then $df = (dz^{d-1}, dw^{d-1})$ and
\[
H(f) =
d(d-1)
\begin{pmatrix}
z^{d-2} & 0
\\
0 & w^{d-2}
\end{pmatrix}.
\]
At a real point we have $\langle \eta, \ov{df} \rangle = 0$.
We have $\eta^{t}H\eta = 0$, and all other factors involve $df$, so here we get $0$.
\end{exam}



\begin{exam}
Let $f(x,y,z) = x^{d} + y^{d} + z^{d}$.
Then $df = d(x^{d-1}, y^{d-1}, z^{d-1})$, and
\[
H(f) =
d(d-1)
\begin{pmatrix}
x^{d-2} & 0 & 0
\\
0 & y^{d-2} & 0
\\
0 & 0 & z^{d-2}
\end{pmatrix}.
\]
Again $\eta$ is orthogonal to $df$ at a real point, and $\eta^{t}H\eta = 0$.
A third orthogonal vector is
\begin{align*}
  \eta \times df
  &= d(
  y^{d}z^{d-1} - y^{d-1}z^{d} ,
  z^{d}x^{d-1} - x^{d}z^{d-1} ,
  x^{d}y^{d-1} - y^{d}x^{d-1}
  )
  \\
  &= d(
  y^{d-1}z^{d-1}(y - z),
  x^{d-1}z^{d-1}(z - x),
  x^{d-1}y^{d-1}(x - y)
  ).
\end{align*}
We have
\[
  H (\eta \times df)
  = d^{2}(d-1)(xyz)^{d-2}(
  yz(y-z), xz(z-x), xy(x-y)
  )
\]
so
\[
  \eta^{t} H (\eta \times df)
  = d^{2}(d-1)(xyz)^{d-1}
  (y-z + z - x + x - y)
  = 0
\]
and
\[
  (\eta \times df)^{t} H (\eta \times df)
  = d^{3}(d-1) (xyz)^{3d-2}
  (
  (y-z)^{2} + (z-x)^{2} + (x-y)^{2}
  ),
\]
which can be nonzero (and is for real points).
To find the Frobenius contribution we have to normalize the vector, so we get
\[
  \frac{d(d-1) (xyz)^{3d-2}
  (
  (y-z)^{2} + (z-x)^{2} + (x-y)^{2}
  )}{
  y^{2d-2}z^{2d-2}(y-z)^{2}
  + x^{2d-2}z^{2d-2}(x-z)^{2}
  + x^{2d-2}y^{2d-2}(x-y)^{2}
}
\]
We'd then have to divide this by $|df^{\sharp}|^{2}|\eta|^{2}$.
\end{exam}


In general we have $H(f) \eta = (d-1) \, df^{\sharp}$.
Does it make sense to look at the orbit of $\eta$?
We can write $\kk C^{n+1} = \kk C \eta \oplus V \oplus \kk C df^{\sharp}$, then this implies we can restrict $H(f)$ to $V$ and only look at what happens there.
For three-dimensional spaces we can try to do as above.
Then
\[
  \eta \times df^{\sharp}
  = (
  y f_{z} - z f_{y},
  z f_{x} - x f_{z},
  x f_{z} - z f_{x}
  )
\]
is homogeneous of degree $d$.
We have
\[
  H(f)(\eta \times df^{\sharp})
  = (
  f_{xx}(yf_{z} - zf_{y})
  + f_{xy}(zf_{x} - xf_{z})
  + f_{xz}(xf_{z} - zf_{x}),
  )
\]
\[
  \frac{
    |(\eta \times df^{\sharp})
    H(f)
    (\eta \times df^{\sharp})|^{2}
  }{|df^{\sharp}|^{2}|z|^{2}}
\]


\section{13. September 2022}

I know why the result I stated on the scalar curvature of a submanifold was so
surpising: It's not true.

For one, suppose it were. Then take a K3 surface and a curve in it. By that
statement, the curve must admit a metric of negative scalar curvature. But there
are K3 surfaces with rational curves (like the Fermat surface, or Kummer
surfaces).

The problem is that we only sum over those elements of the frame that are in
$T_Y$, which isn't enough to recover all of the Ricci tensor or scalar curvature
of the ambient space. For example, for a hypersurface we have
\[
r_Y(\alpha, \ov\beta)
= r_X(\alpha, \ov\beta)
- R_X(\alpha, \ov\beta, e_n, \ov{e_n})
- \langle \sigma(\alpha), \ov{\sigma(\beta)} \rangle
\]
and
\begin{align*}
s_Y
&= (s_X - r_X(e_n, \ov{e_n}))
- (r_X(e_n, \ov{e_n}) - H_X(e_n))
- |\sigma|^2
\\
&= s_X - 2 r_X(e_n, \ov{e_n}) + H_X(e_n) - |\sigma|^2,
\end{align*}
where $H$ is the holomorphic sectional curvature.

\section{8. September 2022}

We may have overcomplicated things yesterday.

Let $Y \subset X$ be a complex submanifold of dimension $m$ in a complex
manifold of dimension $n$.
Let $h_X$ be a K\"ahler metric on $X$ and $h_Y$ the induced metric on $Y$.
We have
$$
0 \to T_Y \to T_{X|Y} \to N_{Y/X} \to 0
$$
as usual and
$$
R_Y = R_X - \sigma^* h_N.
$$
Here $h_N$ is the metric $h_X$ induces on $N_{Y/X}$, $R_X$ is the curvature
tensor of $h_X$ and similar for $R_Y$, and $\sigma \in \Hom(\operatorname{Sym}^2
T_Y, N_{Y/X})$ is the second fundamental form of $Y$ in $X$. It descends to the
symmetric product instead of living in $\Hom(T_Y^* \otimes T_Y^*, N_{Y/X})$
because of the K\"ahler condition.

Fix a point $y \in Y$.
Let $(e_1,\ldots,e_m,e_{m+1},\ldots,e_n)$ be a local orthonormal frame of $T_X =
T_Y \oplus N_{Y/X}$.
To find the Ricci tensor of $h_Y$ we consider $R_Y(\alpha, \ov\beta, \gamma,
\ov\delta)$ and take the trace over $\gamma$ and $\delta$. For the second
fundamental form this gives
$$
\sum_{j=1}^m h_N(\sigma(\alpha)(e_j), \ov{\sigma(\beta)(e_j)})
= h_{\Hom(T_Y, N_{Y/X})}(\sigma(\alpha), \ov{\sigma(\beta)}),
$$
so
$$
r_Y(\alpha, \ov\beta)
= r_X(\alpha, \ov\beta)
- h_{\Hom(T_Y, N_{Y/X})}(\sigma(\alpha), \ov{\sigma(\beta)}).
$$
For the scalar curvature, take the trace again and get
$$
s_Y = s_X - |\sigma|^2_{\Hom(\operatorname{Sym}^2 T_Y, N_{Y/X})}.
$$
Surprisingly this says that if $s_Y = s_X$ at a point, then $R_Y = R_X$ there.


For any holomorphic tangent field $\xi$ of $T_Y$ around $y$ we have
$h_X(\xi, \ov e_k) = 0$ for $k > m$. Then
$$
0 = \partial h_X(\xi, \ov e_k)
= h_X(D_X^{1,0} \xi, \ov e_k) + h_X(\xi, \ov{\bar\partial e_k})
= h_X(D_X \xi, \ov e_k) + h_X(\xi, \ov{\bar\partial e_k}).
$$
It follows that
\begin{align*}
0
= \!\!\! \sum_{k=m+1}^n \!\!\! \partial h_X(\xi, \ov e_k) e_k
&= \!\!\! \sum_{k=m+1}^n \!\!\! h_X(D_X \xi, \ov e_k)e_k + h_X(\xi, \ov{\bar\partial e_k}) e_k
\\
&= \pi(D_X \xi) + \!\!\! \sum_{k=m+1}^n \!\!\! h_X(\xi, \ov{\bar\partial e_k}) e_k,
\end{align*}
where $\pi : T_{X|Y} \to N_{Y/X}$ is the orthogonal projection.
Therefore the second fundamental form vanishes when either (a) all the $e_k$ are
holomorphic, or (b) when $\bar\partial e_k$ lands in $\Omega_Y^{0,1} \otimes
N_{Y/X}$ for every $k$ (the former is stronger than the latter, but nice to keep
in mind).


This morning I thought I could prove this:

Let $Y \subset \kk P^n$ be a hypersurface defined by a homogeneous polynomial
$f$ with real coefficients.
If the real points of $Y$ are dense in $Y$, then the second fundamental form of
$Y$ in $\kk P^n$ is zero.

Making $f$ have real coefficients means we can swap $(f_1,\ldots,f_n)$ out for
its conjugate when taking inner products (at real points!), so $df^\sharp$ will
actually be orthogonal to $T_Y$ (as $T_Y = \ker df$).

Doing this at a real point means that orthogonality in the Euclidean metric
implies orthogonality in the Fubini--Study metric, for similar reasons involving
inner products against the Euler field.
Therefore at real points, a holomorphic tangent field is orthogonal to $T_Y$.

If real points are dense, this is true on a dense subset of $Y$, and therefore
everywhere by continuity.
Thus our caveat above applies, and $\sigma = 0$.
As rational points are real, this also applies if rational points are dense.

This would imply that $R_Y = R_X$, so $T_Y$ would be Griffiths-positive, and $Y$
would thus in fact be $\kk P^{n-1}$.
From a cursory read of the Wikipedia page of rational points, that seems too
good to be true. There should be non-projective hypersurfaces of a projective
space with dense rational points.



\section{7. September 2022}

Let $X$ be a complex manifold, let $L \to X$ be a line bundle and let $s$ be a
section of $L$ that defines a complex submanifold $Y \subset X$. There is a
short exact sequence
$$
0 \to T_Y \to T_{X|Y} \to L_{|Y} \to 0,
$$
where the holomorphic map $ds : T_{X|Y} \to L$ is defined by picking a connection $D$
on $L$ and letting $\xi \mapsto D_\xi s$. This is independent of the connection
chosen (on $Y$!) and holomorphic. This identifies $N_{Y/X}$ with $L_{|Y}$.

Let $h_X$ be a K\"ahler metric on $X$, and denote by $h_Y$ and $h_L$ the induced
metrics on $Y$ and $L$. Note that $h_L$ is \emph{not} defined outside of $Y$.
We write $\sigma$ for the second fundamental form of $Y$
in $X$. The curvature tensor of $h_Y$ is then
\[
R_Y = j^* R_X - \sigma^* h_L,
\]
where $j : Y \to X$ is the inclusion map. We'd like to describe $h_L$ and
$\sigma$ better and look at, for example, the scalar curvature of $h_Y$.

Let $D_L$ be the Chern connection of $h_L$. The second fundamental form is
\[
    \sigma(\xi)(\eta)
    = ds(D_{X,j_*\xi} j_* \eta)
    = D_{L,D_{X,j_*\xi} j_* \eta} s.
\]

Can we describe the adjoints any better? Let
$$
0 \to Y \to X \to L \to 0
$$
be a short exact sequence of vector spaces where $\dim L = 1$, and let $h$ be
metrics like before. Let $q : X \to L$ be the quotient map, so $Y = \ker q$.
As $\Hom(X,L) \cong X^* \otimes L$ and $L$ is one-dimensional we can find $x \in
X$ and $t \in L$ such that $\bar x \otimes t \mapsto q$ under $h_X \otimes
\id_L$.

\section{21. August 2022}

The positivity result we need is this:

Let $R_1$ and $R_2$ be curvature tensors on $V$.
Suppose that $V = W_1 \oplus W_2$, where $R_1$ is positive on $W_1$,
$R_2$ is positive on $W_2$,
and suppose $R_2 \geq 0$ on $V$.
Then there exists $\lambda$ such that $R_\lambda := R_1 + e^\lambda R_2$ is
positive on $V$.

Proof:\quad
If we pick a reference inner product on $V$ it is enough to show positivity on
the unit sphere $S(V)$ in $V$.
Let $m_1 = \inf_{v \in S(V)} R_1(v, \bar v, v, \bar v)$.
As $S(V)$ is compact, this minimum is attained at a point $v$.
If $v \in V_1$, then $R$ is positive as $R_1$ is positive there.
Otherwise the projection of $v$ to $W_2$ is nonzero, so $m_2 := R_2(v, \bar v, v, \bar v) > 0$.
If we pick $\lambda$ so that $m_1 + e^\lambda m_2 > 0$
then $R$ will then be positive on all of $S(V)$.



\section{19. August 2022}

I think we can prove something like this:

\medskip
\noindent
\textbf{Theorem}\quad
{\it
Let $\pi : X \to B$ be a family of compact complex manifolds over a compact base
$B$.
Suppose there exists a Hermitian form $h_{X/B}$ on $X$ whose restriction to any
fiber $X_b$ is a Hermitian metric.
Let $h_B$ be a Hermitian metric on $B$.

\begin{enumerate}
\item
If the curvature tenors of the metrics are positive, then $h_{X/B} + e^\lambda \pi^* h_B$ is positive for large $\lambda$.

\item
If the curvature tenors of the metrics are negative, then $h_{X/B} + e^\lambda \pi^* h_B$ is negative for large $\lambda$.
\end{enumerate}
}
\medskip

Write $h_\lambda = h_{X/B} + e^\lambda \pi^* h_B$.
The curvature tensor of this is
$$
R_\lambda
= R_{X/B}
+ e^\lambda \pi^* R_B
- (D_{X/B} - \pi^* D_{B})^* h_{q,\lambda},
$$
where $h_{q,\lambda}$ is the ``quotient'' metric.
Note that $\Ker \pi_* \subset \Ker h_q$.
Since $h_{X/B}$ is positive-definite on $T_{X/B}$ it defines a smooth splitting
$T_X = T_{X/B} \oplus \pi^* T_B$.
If $j$ and $q$ are the inclusion and projection from $T_X$ to $\pi^*T_B$ we have
$\lim_{\lambda \to \infty} h_{q,\lambda} = p^* j^* h_{X/B}$.

Suppose that both curvature tensors are positive.
First we deal with $T_{X/B}$, where $\pi^*R_B = 0$.
Let $(x,b)$ be a point in $X$, so that $x \in X_b$.
There exists a local normal frame of $T_{X_b}$ for $h_{X/B,b}$ centered at $x$.
In this frame, we have $D_{X/B} - \pi^* D_B = 0$ at $x$.
The only part of our curvature tensor that survives at $x$ is thus $R_\lambda =
R_{X/B}$,
which is positive, so $R_\lambda$ is positive on $T_{X/B}$.

Now pick an auxiliary Hermitian metric on $X$ and define a compact sphere bundle
$S(T_X) \subset T_X$ to test positivity on.
Let
\begin{align*}
m_\lambda &=
\inf_{S(T_X)} R_{X/B}
- (D_{X/B} - \pi^* D_{B})^* h_{q,\lambda},
\\
m_\infty &=
\inf_{S(T_X)} R_{X/B}
- (D_{X/B} - \pi^* D_{B})^* p^* j^* h_{X/B}.
\end{align*}
As $S(T_X)$ is compact, the infimum $m_\infty$ is attained at a point $v$.
If $v \in \Ker \pi^*$, then we're done as we know that $R_{X/B}$ is positive
there.
Otherwise $\pi^*(v) \not= 0$, so $m_2 := \pi^*R_B(v, \bar v, v, \bar v) > 0$.
We then pick $\lambda_0$ big enough so that $m_\infty + e^{\lambda_0} m_2 > 0$.
Now note that
$m_\lambda \to m_\infty$ as $\lambda \to \infty$ because $h_{\lambda,q} \to
p^*j^*h_{X/B}$.
For $\lambda > \lambda_0$ we then have
$$
m_\lambda + e^\lambda m_2
\geq m_\lambda + e^{\lambda_0} m_2
\to m_\infty  + e^{\lambda_0} m_2
> 0.
$$
For all large enough $\lambda$, it follows that $R_\lambda$ will be positive.

The proof for the negative curvature case is the same.



\section*{11. August 2022}

Can we deform a hypersurface in the direction of its normal bundle?

Let $(X, g)$ be a Riemannian manifold and let $Y \subset X$ be a submanifold.
Let $\xi$ be a tangent field that trivializes $N_{Y/X}$.
Consider the exponential map
$$
\exp_p : T_{X,p} \to X
$$
(let's suppose $X$ is complete so this is defined everywhere).
We define $Y(\xi)$ to be the set of points
$$
Y(\xi) := \{ \exp_x(\xi(x)) \mid x \in Y \}.
$$

Claim/hope 1: $Y(\xi)$ is again a submanifold of dimension $\dim Y$ for $\xi$ small.

Claim/hope 2: $D\Vol(Y(\xi))|_{\xi = 0} = \int_Y s |\xi| \, dV$, where $s$ is the
scalar curvature.

(1) should hold by the tubular neighborhood theorem.
For $\xi$ small (inside the tube) $Y(\xi)$ should just be the image of $\xi$
under the tube map in~$X$.

Maybe the best way to do this is to pull everything back to the normal bundle
and work there?



\section*{6. August 2022}

A problem I saw on lobste.rs (of all places) is:
Suppose $f : [0,1] \to \kk R$ is integrable and that
$\int_0^1 f(x) dx = 1$ and $\int_0^1 xf(x) dx = 1$.
Show that
$$
\int_0^1 f(x)^2 dx \geq 4.
$$
The solution given considers the integral of $(f(x) - (ax + b))^2$ and minimizes
it as a function of $a$ and $b$.
I feel like this is a little heavy.

There's an easy first bound:
The hypotheses are $\langle f, 1 \rangle = 1$ and $\langle f, x \rangle = 1$.
Apply Cauchy--Schwarz to the second one and get
\[
1
= \langle f, x \rangle^2
\leq |f|^2 \int_0^1 x^2 dx
= |f|^2 / 3
\]
so $3 \leq |f|^2 = \int_0^1 f(x)^2 dx$.

The solution given is to define
$$
g(t,s) =
|f - (tx + s)|^2.
$$
We have
\begin{align*}
0 \leq g(t,s)
&= \langle f - (tx + s), f - (tx + s) \rangle
\\
&= |f|^2 - 2 \langle f, tx + s \rangle + |tx + s|^2
\\
&= |f|^2 - 2(t + s) + t^2/3 + ts + s^2.
\end{align*}





\section*{2. August 2022}

When I worked at the Icelandic Science Web I saw a question I've thought about
on and off since then:

\begin{quote}
How much farther does a car that drives the Icelandic ring road on the right
drive than a car that drives on the left?
\end{quote}

The ring road is, topologically, a circle.
Somewhat more interestingly, it is also a closed path in what we'll pretend is
a plane.
The starting point of our setup is thus a closed smooth path $\gamma : [0,L] \to
\kk R^2$, where $\gamma(0) = \gamma(L)$.
We'll assume $\gamma$ is parametrized by arc length, so $|\gamma'(t)| = 1$ for
all $t$ and $L$ is the length of the path.
Under these conditions we can find a vector field $n$ over $\gamma$ such that $|n(t)| = 1$
and $n$ is orthogonal to $\gamma'$ and $(\gamma', n)$ is a positively oriented
frame.
The curvature of the path $\gamma$ is then the positive real function $\kappa$ that
satisfies
$\kappa(t) = |\gamma''(t)| = |k(t) n(t)|$,
where $k = \pm \kappa$ is a real function.
We recall the second Frenet--Serret formula, which gives $n'(t) = -\kappa(t)
\gamma'(t)$.

With this in mind, we take as our $\gamma$ the middle of the road in question.
The left- and right-sides of the roads are then given by $\beta_s(t) := \gamma(t)
+ s n(t)$, for some small $s$.
The length of the curve $\beta_s$ is
$$
L(\beta_s)
= \int_0^L |\beta_s'(t)| dt
= \int_0^L |\gamma'(t) - s \kappa(t) \gamma'(t)| dt
= \int_0^L |1 - s \kappa(t)| dt.
$$
For $|s| < 1/ \sup \kappa(t)$, this is
$$
L(\beta_s)
= \int_0^L 1 - s \kappa(t) dt
= L - 2\pi s
$$
by the Gauss--Bonnet theorem.

Now, the distance from the right-side of the road to the center is perhaps something like 1.5m.
Is it reasonable to assert that $\kappa < 1/1.5m = 2/3m$ everywhere?
Recall that $\kappa(t)$ can also be defined as one over the radius of an oscillating circle to $\gamma(t)$ at the point $t$.
No car can be expected to make a turn around a circle of radius $1.5m$ at speeds
of at least 20km/h (try it! safely!), so we argue that yes.
The hypotheses of our derivation therefore hold, so the answer to our question is
$$
|L(\beta_{1.5}) - L(\beta_{-1.5})|
= |(L - 2\pi * 1.5) - (L - 2\pi * (-1.5))|
= 6\pi
$$
meters.



\section*{14. July 2022}

I don't think this semipositive form of mine actually works. If it did, why
isn't the Hopf surface K\"ahler? That is an elliptic curve bundle over the
projective line, and I'm fairly sure the flat metric on the curve is invariant
under the automorphism group.


\section*{10. July 2022}

Let $E \to X$ be a holomorphic vector bundle and consider a Grassmannian bundle
$\Gr(E,k) \to X$. I claim that there is a closed real $(1,1)$-form $\omega$ on
the total space whose restriction to each fiber is the K\"ahler--Einstein metric
on that fiber.

I want to construct this form by saying that its value at a point is the
K\"ahler--Einstein metric on the fiber associated to the point. If we then take
a local trivialization of $E$ we get a local trivialization of the Grassmannian
bundle, and the form is just the pullback of the metric on the Grassmannian
factor to the product. This is closed, real, restricts to what I said, and is
\emph{semipositive}.

This kind of sounds too good to be true. Does this really work?

If it does, then the metric we get on the total space of the Grassmannian bundle
is locally the sum of pullbacks from the different factors, so the ``quotient''
metric degenerates. This makes the curvature formula of the sum just the sum of
the curvatures.


\begin{prop}
Let $\pi : X \to B$ be a locally trivial holomorphic family with fiber $F$.
Assume there is a K\"ahler metric $\omega$ on $F$ that is invariant under $\Aut F$.
Then there is a closed semipositive $(1,1)$-form on $X$ that restricts to $\omega$ on each fiber and whose kernel is $\pi^*T_B$.
\end{prop}

\begin{proof}
We'll work locally and may assume the family is a product $X = F \times B$.
Set $\omega' = \pr_{F}^*\omega$.
This form satisfies our conclusions if we can prove it is globally defined.
If we trivialize the family over a different open set and change between the two, we act by an automorphism of $F$.
Our $\omega$ is invariant under that automorphism, so the construction glues.
\end{proof}


Unrelated, but let $F$ and $B$ be manifolds with nontrivial automorphism groups
and let $\Aut B \to \Aut F$ be an injection. Then you should be able to
construct a locally trivial family $X \to B$ with fiber $F$ in the usual way.


\section*{1. July 2022}

Can we solve the equation
\[
  Lu = v
\]
for $u$, where $L$ is the Hodge operator?

We have $m \id = L\Lambda - \Lambda L$ for some $m$. Apply it to $v$ and get
\[
  m v = L \Lambda v - \Lambda L v.
\]

\section*{24. June 2022}


We can calculate the projection onto primitive $(p,q)$-classes using $L\Lambda$ and the Cayley--Hamilton theorem.

Let's consider $(p,q)$-classes with $k := p + q \leq n$ (because for $k > n$ there are no primitive classes) and the operator $L\Lambda$. It is an endomorphism of the space of $(p,q)$-classes and has a characteristic polynomial $p(x) = x^n + c_{n-1}x^{n-1} + \cdots + c_1 x + c_0$. Since primitive classes exist we have $c_0 = \det (L\Lambda) = 0$. By the Cayley--Hamilton theorem we then have
$$
0 = p(L\Lambda)
= (L\Lambda)^n + c_{n-1}(L\Lambda)^{n-1} + \cdots + c_1 L\Lambda
= (L\Lambda)^m f,
$$
where $f := (L\Lambda)^{n-m} + c_{n-1} (L\Lambda)^{n-m-1} + \cdots + c_m \operatorname{id}$ and $c_m \not= 0$.

We then have $(L\Lambda)^m f(u) = 0$ for every class $u$. As $L$ is injective, we conclude that $(L\Lambda)^{m-1} f(u)$ is primitive. But $Lv$ is never primitive for a nonzero class $v$, so by induction on $m$ we must in fact have that $f(u)$ is primitive.

Now suppose $u$ is primitive.
Then $f(u) = c_m u$, so by rescaling $f$ we obtain a projection operator onto
the space of primitive classes. Then the Hodge theorems tell us that
$$
u - \frac{1}{c_m} f(u) = Lv
$$
for some $(p-1,q-1)$-class $v$, and we recurse.

I think we can calculate the coefficients $c_{j}$ as well by using the basis that's composed of primitive forms and their multiples by $L$. The operator $L\Lambda$ is diagonal in that basis. Let $u$ be a primitive $(p-i,q-i)$-class and consider $L^{i}u$.
We have
$$
[L^{[i]},\Lambda]u = (k-n+i-1)L^{[i-1]}u
$$
for all $k$-classes $u$. If $u$ is primitive, we get
$$
(k-n+i-1)L^{[i-1]}u
= [L^{[i]},\Lambda]u = - \Lambda L^{[i]}u
$$
so
$$
L\Lambda L^{[i]}u
= -(k-n+1-1) L L^{[i-1]}u
= - i(k-n+i-1)L^{[i]}u.
$$
These are then the eigenvalues of $L\Lambda$ on $k$-classes.

For classes of even degree $2k$ we get eigenvalues $\lambda_{i} = -i(2k-n+i-1)$ for $i = 0, \ldots, k$.
The multiplicity of $\lambda_{i}$ is equal to the dimension of primitive classes of degree $2k-2i$, which is $b_{2k-2i} - b_{2k-2i-2}$. The nonzero coefficient $c_{m}$ we were looking at above is then
\begin{align*}
\prod_{i=1}^{k} \lambda_{i}^{b_{2k-2i}-b_{2k-2i-2}}
&= \prod_{i=1}^{k} (-i(2k-n+i-1))^{b_{2k-2i}-b_{2k-2i-2}}
\\
&= (-1)^{1-b_{2k-2}}\prod_{i=1}^{k}
(i(2k-n+i-1))^{b_{2k-2i}-b_{2k-2i-2}}
\end{align*}

Let's look at examples. Pick $k = 2$. Then we have $\lambda_{0} = 0$ and $\lambda_{1} = 2-n$, of multiplicities $m_{0} = b_{2} - 1$ and $m_{1} = 1$. The characteristic polynomial is
$$
p(x) = x^{b_{2}-1}(x+n-2)
$$
so
$$
p(L\Lambda) = (L\Lambda)^{b_{2}-1}(L\Lambda + (n-2)\id)
$$
and
$$
f(L\Lambda) = L\Lambda + (n-2)\id
$$
so the projection operator is $\id + \frac1{n-2} L\Lambda$. There's a sign error
here.


\section*{20. June 2022}

More on the quotient metric.

\begin{prop}
If $x \in \Ker h_{1}$, then $x \in \Ker h_{Q}$.
\end{prop}

\begin{proof}
We have
\[
h_{Q} = h_{1} h^{-1} h_{2} + h_{2} h^{-1} h_{1}
\]
and $x = h^{-1}h(x) = h^{-1}h_{1}(x) + h^{-1}h_{2}(x)$. If $x \in \Ker h_{1}$ this means that $x = h^{-1}h_{2}(x)$. Plugging this into the above we get
\[
h_{Q}(x)
= h_{1} h^{-1} h_{2}(x) + h_{2} h^{-1} h_{1}(x)
= h_{1}(x) = 0.
\qedhere
\]
\end{proof}

One of our Hermitian forms is often the pullback of a Hermitian metric on a base, so this maybe simplifies some calculations. At least it should let us say that $h_{Q}$ then becomes a Hermitian form on $\pi^{*}T_{B}$ (but not a pullback of a form on $T_{B}$).

I'd like to be able to prove the estimate below in a coordinate-invariant way. Maybe we can use that
\begin{align*}
h
= hh^{-1}h
&= h_{1}h^{-1}h_{1}
+ h_{1}h^{-1}h_{2}
+ h_{2}h^{-1}h_{1}
+ h_{2}h^{-1}h_{2}
\\
&= h_{1}h^{-1}h_{1}
+ h_{2}h^{-1}h_{2}
+ h_{Q}.
\end{align*}
We'd need to be able to say that $h_{1}h^{-1}h_{1} + h_{2}h^{-1}h_{2}$ is semipositive-definite, though.


Suppose that our original setup is a short exact sequence $0 \to S \to V \to Q \to 0$ and that $h = h_{V/S} + q^{*} h_{Q}$. By the above the Hermitian form on the quotient is zero on $\Ker q = S$ so it defines a Hermitian form on $Q$. The contribution of the second fundamental form of the sum to the curvature is
\[
\langle D_{V/S,x} z - D_{Q,x}z, \ov{D_{V/S,y}w - D_{Q,y}w} \rangle.
\]
Once we push this to $Q$ I think this becomes
\[
\langle \sigma_{V/S,x} z - q^{*}D_{Q,x}z, \ov{\sigma_{V/S,y}w - q^{*}D_{Q,y}w} \rangle,
\]
where $\sigma_{V/S}$ is the second fundamental form of $h_{V/S}$.


\section*{19. June 2022}

There's a basic estimate for the ``quotient'' metric I've been looking at.

\begin{prop}
  We have
  \[
    \|x\|_{h_{Q}} \leq \| x \|_{h}
  \]
  with equality if and only if $x = 0$.
\end{prop}

\begin{proof}
Let $0 \to V \to V \oplus V \to V \to 0$ be the usual short exact sequence and
$h_{1}$, $h_{2}$ Hermitian forms on $V$ such that $h := h_{1} + h_{2}$ is
positive-definite and denote by $h_{Q}$ the induced Hermitian form on the quotient. Diagonalize $h$ and $h_{1}$ so $h_{2} = h - h_{1}$ is diagonal as well and get that $h_{Q}$ is diagonal with entries
\[
\frac{2 a_{j} b_{j}}{a_{j} + b_{j}},
\]
where we know that $a_{j} + b_{j} > 0$. By basic algebra we have
\[
\frac{2 a_{j} b_{j}}{a_{j} + b_{j}}
< a_{j} + b_{j}
\]
and the equality cannot be attained. Therefore
\[
  \|x\|_{h_{Q}} < \|x\|_{h}
\]
for all $x \not= 0$.
\end{proof}

\section*{18. June 2022}

Let $E \to (X,\omega_{X})$ be a holomorphic vector bundle of rank $r$ over a
manifold of dimension $n$. We want to construct a metric on the associated
Grassmannian bundle $\Gr(k, E) \to X$ and want to start by looking at the local picture.

Let $U\subset X$ be an open neighborhood that trivializes $E$, so $E_{|U} \cong E_{0} \times U$, where $E_{0}$ is the fiber of $E$. The Grassmannian bundle also trivializes over $U$ as $\Gr(k,E)_{|U} \cong \Gr(k, E_{0}) \times U$. We then have a local holomorphic splitting $T_{\Gr(k,E)} = T_{\Gr(k,E_{0})} \oplus T_{U}$.

We know that there exists a unique K\"ahler--Einstein metric $\omega_{\Gr}$ of volume 1 on $\Gr(k,E_{0})$. We define our metric on $\Gr(k,E)_{|U}$ by
\[
  \omega_{\lambda} = \lambda \omega_{\Gr} +  \omega_{X}.
\]
If $\omega_{X}$ is K\"ahler, then this metric is also K\"ahler. (I think this is more or less how Kodaira  calculated the curvature of projective bundles in his paper on his embedding theorem.) We apply our degenerate Codazzi--Griffiths formula and conclude that its curvature tensor is
\[
  \displaylines{
  R_{\lambda}(x,\ov y,z,\ov w)
  = \lambda R_{\Gr}(x,\ov y,z,\ov w)
  +  R_{X}(x,\ov y,z,\ov w)
  \hfill\cr\hfill{}
  - \langle
  D_{\Gr,x} z - D_{X,x} z,
  \ov{D_{\Gr,y}w - D_{X,y}w}
  \rangle_{\lambda},
}
\]
where the inner product is the one on the ``quotient'' in $0 \to V \to V \oplus V \to V \to 0$ where the first map is the diagonal one. We know that it converges to a Hermitian form $h_{\infty}$ on $V$ as $\lambda \to \infty$.

Consider then the case when $X$ admits a K\"ahler metric of positive holomorphic sectional curvature. The formula above for $R_{\lambda}$ shows that it will eventually have positive holomorphic sectional curvature, as $R_{\Gr}$ does so and the negative contribution is bounded.






\section*{10. June 2022}

Someone on math.stackexchange asked if the Kahler cone of a product of manifolds is the product of their Kahler cones (for K3 surfaces). I'm not sure if this happens when $H^{1}(X) = H^{2,0}(X) = 0$ (and for $Y$ as well). In that case we have
\[
  H^{1,1}(X \times Y) = H^{1,1}(X) \oplus H^{1,1}(Y)
\]
and both manifolds are projective. I thought that given $\omega$ Kahler on $X \times Y$ we could look at some linear combination of the pullbacks of $\pi_{1,*} \omega^{\dim X+1}$ and $\pi_{2,*}\omega^{\dim Y+1}$. To show these define the same Kahler class as $\omega$ we need to pick a subvariety $Z \subset X \times Y$ and push it forward to $X$ and $Y$ and integrate. Maybe it works out by abstract machinery?

To write some details, let $p : X \times Y \to X$ and $q : X \times Y \to Y$ be the projections, let $\omega$ be a Kahler class on $X \times Y$, and let $n = \dim X$ and $m = \dim Y$.

My first claim is that $p_{*} \omega^{m+1}$ is a Kahler class on $X$. Let $Z \subset X$ be a subspace of dimension $k$. Then
\[
  \int_{Z} (p_{*} \omega^{m+1})^{k}
  = \int_{Z \times Y} \omega^{m+k+1}
  > 0
\]
because $\omega$ is Kahler. This needs some functorial justification. This should be true fairly generally; if $f : X \to Y$ is a submersion (or weaker?) and $\omega$ is Kahler on $X$ then $f_{*}\omega^{\dim f^{-1}(y)+1}$ should be Kahler on $Y$.

My second claim is that if $Z \subset X \times Y$ is a subspace we can push an integral over it to $X$ and $Y$ and then down from there. This again needs some functorial justification.

I wonder if I'm being stupid about this. If $\omega$ is Kahler on $X \times Y$ then there exist $\alpha$ on $X$ and $\beta$ on $Y$ such that $\omega = p^{*}\alpha + q^{*}\beta$. As $\omega$ is Kahler, then so is its restriction to any subvariety, so $\omega_{|X} = \alpha$ is Kahler, and similarly for $\beta$. Donezo.

\section*{7. June 2022}

Let's try to calculate ball volumes.

We write $B^{n}(r)$ for the ball of radius $r$ in $n$-dimensional space, or $B^{n}$ or just $B$ for the unit ball. We choose coordinates and consider the function
\[
  f : B^{n} \to B^{1},\quad
  x \mapsto x_{n}.
\]
It is surjective and its differential is $df(\xi) = \xi_{n}$, which is also surjective. Its fiber over $t$ is $f^{-1}(t) = B^{n-1}(\sqrt{1-t^{2}})$.

If we write $\alpha = dx_{1} \wedge \cdots \wedge dx_{n-1}$, then $dV = \alpha \wedge f^{*} dx_{n}$, where $dV$ is the volume form on $R^{n}$. We get
\begin{align*}
  \Vol(B^{n})
  = \int_{B^{1}} f_{*}(\alpha \wedge f^{*}dx_{n})
  &= \int_{-1}^{1} \Vol(B^{n-1}(\sqrt{1-t^{2}})) dt
    \\
  &= \Vol(B^{n-1})\int_{-1}^{1}\sqrt{1-t^{2}}^{n-1} dt.
\end{align*}
Wolfram Alpha says that
$$
\int_{-1}^{1}\sqrt{1-t^{2}}^{n-1} dt
= \frac{\sqrt\pi \Gamma((n+1)/2)}{\Gamma(n/2 + 1)}
$$
which sounds fine, but might be difficult to verify.

Some googling of ``integrate powers of sine'' turned up that this integral is actually what people transform those into, so Wolfram Alpha might be full of it.
Let's split by cases; set $I_{n} = \int_{-1}^{1}\sqrt{1-x^{2}}^{n}$.

We have
$$
I_{2n} = \int_{-1}^{1}(1-x^{2})^{n} dx.
$$
Integrating by parts a couple of times (with $u = (1-x^{2})^{n-k}$ and $v' = x^{2k}$, then similar) gives
\begin{align*}
I_{2n}
&= 2n \int_{-1}^{1} x^{2} (1-x^{2})^{n-1} dx
\\
&= \frac{2^{2} n(n-1)}{1 \cdot 3} \int_{-1}^{1} x^4 (1-x^{2})^{n-2} dx
\\
&= \cdots =
\frac{2^{n} n!}{1 \cdot 3 \cdots (2n-1)} \int_{-1}^{1} x^{2n} dx
\\
&= \frac{2^{n+1} n!}{1 \cdot 3 \cdots (2n-1)(2n+1)}
\\
&= \frac{2^{n+1} n!}{(2n+1)!/(2^{n}n!)}
= \frac{2^{2n+1} (n!)^{2}}{(2n+1)!}.
\end{align*}
For odd numbers, we have
\[
I_{2n+1} = \int_{-1}^{1} \sqrt{1-x^{2}} (1-x^{2})^{n} dx.
\]
I'm not sure what to do about that. Maybe turn it into an integral of $\sin \theta$ and use an angle doubling formula?

\section*{31. January 2022}

Let $E \to X$ be a holomorphic vector bundle. I saw a claim that Finsler metrics
on $E$ correspond to Hermitian metrics on $\cc O_{\kk P(E)}(-1) \to \kk P(E)$ and I'd like
to verify that.

Suppose $h$ is a Finsler metric on $E \to X$. This is a norm on the fibers of
$E$ that varies smoothly when the fibers move. Let $s,t$ be local sections of
the bundle of lines in $E$. We can write $s = fe$ and $t = ge$ for some
nonvanishing section $e$, and set
$$
\langle s, \ov t \rangle
:= f \ov g \, h(e)^2.
$$
If $e'$ is a different such section, then $e = \alpha e'$ and $s = (f/ \alpha)
e'$ and $t = (g/\alpha)e'$ and we get
$$
(f/\alpha) \ov{(g/\alpha)} \, h(e')^2
= (f/\alpha) \ov{(g/\alpha)} \, |\alpha|^2 h(e)^2
= f \ov{g} \, h(e)^2
$$
so the definition makes sense. This gives a Hermitian metric on $\cc O(-1)$.

Suppose we have a Hermitian metric $h$ on $\cc O(-1)$.

\section*{26. January 2022}

Someone in IRC asked about the function $f: \kk R^n \to \kk R$ defined by
$$
f(u)
= \int_{S^{n-1}} e^{i \langle u, v \rangle}\, d\sigma(v),
$$
where $d\sigma$ is the Lebesgue measure. Can we find out what it is explicitly?

It's smooth, and not constant as we see by scaling $u$. It's invariant under the
action of $SO(n)$ on $\kk R^n$. We can rotate $u$ so it's in the line
$e_1 = (1,0,\ldots,0)$. The inner product is then
$$
\langle t e_1, v \rangle
= t v_1.
$$
Writing $e^{i\langle u, v \rangle} = \cos\langle u, v \rangle + i \sin\langle u,
v \rangle$ and noting that $\sin$ is odd we see its integral over the sphere is
zero. We write
$$
\cos\langle u, v \rangle
= \sum_{k = 0}^\infty (-1)^k t^{2k} \frac{v_1^{2k}}{(2k)!}
$$
and integrate over the sphere. We have
$$
\int_{S^{n-1}} v_1^{2k} d\sigma(v)
= \frac{2\Gamma(k + \frac12)\Gamma(\frac12)^{n-1}}{\Gamma(k+\frac n2)}
= 2\Gamma(\tfrac12)^{n-1} \frac{\Gamma(k + \frac12)}{\Gamma(k+\frac n2)}.
$$
We have
$
\Gamma(x + 1) = x \Gamma(x)
$
so
$$
\Gamma(k+x)
= x \Gamma((k-1) x)
= \cdots
= x^k \, \Gamma(x)
$$
which gives
$$
\frac{\Gamma(k + \frac12)}{\Gamma(k+\frac n2)}
= \frac{(\frac12)^k\Gamma(\frac12)}{(\frac n2)^k\Gamma(\frac n2)}
= \frac{1}{n^k} \frac{\Gamma(\frac12)}{\Gamma(\frac n2)}.
$$
So the coefficients of the series are $(-1)^k /n^k (2k)!$, which is something.


\section*{1. June 2021}

\subsection*{Entire maps of finite degree are polynomials}

Let $f : \kk C \to \kk C$ be a holomorphic map of degree $d$. If $f$ has an essential singularity at infinity, big Picard says it takes every value infinitely often, so that can't happen. If $f$ has a removable singularity at infinity, it is constant by Liouville. Therefore $f$ has a pole at infinity, which implies that it is a polynomial, necessarily of degree $d$.




\section*{17. May 2021}
\subsection*{Integrating over fibers}

From math.SE\footnote{https://math.stackexchange.com/q/4140847/3225}:

Let $f : \kk R \to \kk R$ be a continuous function, and let $v \in \kk R^n$. Then
\[
\int_{S^{n-1}} f(\langle x, v \rangle) \, d\sigma(x)
= \omega_{n-2}\int_{-1}^1 f(|v| t) (1-t^2)^{(n-3)/2} dt,
\]
where $\omega_n$ is the volume of the $n$-sphere.


I want to show this by factorizing the constant map $S^{n-1} \to \{*\}$ as $S^{n-1} \to [-|v|, |v|] \to \{*\}$, apply functoriality of the pushforward, and use integration over fibers.

First assume $v = 0$. Then the statement we want to prove is
\[
\Vol(S^{n-1}) = \Vol(S^{n-2}) \int_{-1}^1 (1-t^2)^{(n-3)/2} dt.
\]
We have
\[
\int_{-1}^1 (1-t^2)^{(n-3)/2} dt
= \frac{\sqrt \pi \Gamma((n-1)/2)}{\Gamma(n/2)}
\]
and
\[
\Vol(S^{n-1}) = \frac{2 \pi^{n/2}}{\Gamma(n/2)}
\]
so
\begin{align*}
\Vol(S^{n-2})
\int_{-1}^1 (1-t^2)^{(n-3)/2} dt
&= \frac{2 \pi^{(n-1)/2}}{\Gamma((n-1)/2)}
\frac{\sqrt \pi \Gamma((n-1)/2)}{\Gamma(n/2)}
\\
&= \frac{2\pi^{n/2}}{\Gamma(n/2)}
= \Vol(S^{n-1}).
\end{align*}
This checks out and is useful for checking that the exponent is correct. We don't have to go through this: the case $v = 0$ is equivalent to calculating $f = 1$ for a non-zero $v$.

Now assume $v \not= 0$. Consider the map $p : \kk R^n \to \kk R$ given by $p(x) = \langle x, v \rangle / |v|$.
We have
\[
|p(x)|^2
= \frac{\langle x, v \rangle^2}{|v|^2}
\leq \frac{|x|^2 |v|^2}{|v|^2} = 1
\]
with equality when $x$ is a multiple of $v$, so $p(x)$ ranges from $-1$ to $1$.
If $H = \{ x \in \kk R^n \mid \langle x, v \rangle = 0\}$
we get an orthogonal splitting
\[
\kk R^n \to H \oplus \kk R \frac{1}{|v|}v,
\quad
x \mapsto \Bigl(x - p(x) \frac{1}{|v|} v\Bigr) \oplus p(x) \frac{1}{|v|}v.
\]
Let $x \in S^{n-1}$, so
\begin{align*}
\Bigl|x - p(x)\frac{1}{|v|}v\Bigr|^2
&= |x|^2 - 2 \frac{p(x)}{|v|} \langle x, v \rangle + \frac{p(x)^2}{|v|^2} |v|^2
\\
&= 1 - 2 \frac{\langle x, v \rangle^2}{|v|^2} + \frac{\langle x, v \rangle^2}{|v|^2}
= 1 - p(x)^2.
\end{align*}
The fibers of $p$ are $p^{-1}(t) = \{ x \in S^{n-1} \mid \langle x, v \rangle/|v| = t \}$, and on them we have
\[
\Bigl|x-p(x)\frac{1}{|v|}v\Bigr|^2
= 1 - t^2,
\]
so $x-p(x)\frac{1}{|v|}v \in S^{n-2}(\sqrt{1-t^2})$.

We now want to calculate
\[
\int_{S^{n-1}} f(\langle x, v \rangle) \, d\sigma(x)
= \int_{-1}^{1} p_*\bigl(f(\langle x, v \rangle) \, d\sigma(x)\bigr) dt.
\]
Pushforwards respect the wedge product, and we have $f(\langle x, v \rangle) = f(t|v|)$ on the fibers of $p$, so we get
\[
\int_{-1}^{1} p_*\bigl(f(\langle x, v \rangle) \, d\sigma_{n-1}(x)\bigr) dt
= \int_{-1}^{1} f(t|v|) p_*\bigl(\sigma_{n-1}(x)\bigr) dt.
\]
Let $\frac{\partial}{\partial t}$ be the unit tangent field on $[-1,1]$. The differential of $p$ is
\[
p_*(x, \alpha) = \frac{\langle \alpha, v \rangle}{|v|},
\]
where $\alpha \in T_{S^{n-1},x}$, that is, $\alpha$ is such that $\langle \alpha, x \rangle = 0$. Let $\alpha(x) = v/|v| - \langle v, x \rangle/|v| x$. Then
\[
\langle \alpha, x \rangle
= \frac{1}{|v|}(\langle v, x \rangle - \langle v, x \rangle |x|^2) = 0
\]
so $\alpha(x)$ is tangent to the sphere at $x$,
and
\[
p_*(x, \alpha)
= \frac{\langle v - \langle v, x \rangle x, v \rangle}{|v|^2}
= \frac{|v|^2 - \langle v, x \rangle^2}{|v|^2}
= 1 - p(x)^2
\]
so outside of $x = \pm v/|v|$ we have
\[
p_* \Bigl( \frac{1}{1-p(x)^2} \alpha \Bigr) = \frac{\partial}{\partial t}.
\]
We also have
\[
|\alpha(x)|^2
= 1 - 2 \langle v, x \rangle^2 / |v|^2 + \langle v, x \rangle^2 / |v|^2
= 1 - p(x)^2.
\]

The pushforward of the volume form on the sphere is the $1$-form on $[-1,1]$ defined by
\[
(p_* \sigma_{n-1})\Bigl( \frac{\partial}{\partial t}; t \Bigr)
= \int_{p^{-1}(t)} \sigma_{n-1}(x_1, \ldots, x_{n-2}, \alpha),
\]
where $x_1,\ldots,x_{n-2}$ span the tangent space of $p^{-1}(t)$.



The formula for the volume element on $j : S^{n-1} \subset \kk R^n$ is
\[
d\sigma(x) = j^*(\iota_x dV),
\]
where $x \in S^{n-1}$ and $dV$ is the volume element on $\kk R^n$. Then
\[
\iota_{\alpha(x)} d\sigma(x)
= j^*(\iota_{v/|v|} \iota_x dV)
\]
because of anticommutativity of volume elements.
We can pick $x_1,\ldots,x_{n-2}$ to be orthogonal to $x$; then
\[
\sigma_{n-1}(x_1,\ldots,x_{n-2},\alpha)
= \Vol(x_1,\ldots,x_{n-2},v/|v|,x),
\]
where the volume is of the parallelepiped in $\kk R^n$ spanned by the vectors. All except $v/|v|$ are orthogonal to each other. We get
\[
\sigma_{n-1}(x_1,\ldots,x_{n-2},\alpha)
= 1 - \sum_{j=1}^{n-2} \langle x_j, v \rangle / |v| - \langle x, v \rangle / |v|.
\]
(This might be missing a square root and some squares. Is the norm of $\alpha$ also 1?)
Integrating the odd-degree polynomial factors over the sphere annihilates them, so we get
\begin{align*}
p_* \sigma_{n-1}\Bigl(\frac{\partial}{\partial t}; t \Bigr)
&= \int_{p^{-1}(t)} \bigl(1 - \langle x, v \rangle / |v| \bigr) \, \sigma_{n-2}
\\
  &= (1-t^2)^{(n-2)/2} \Vol(S^{n-2}) \bigl(1 - \langle x, v \rangle / |v| \bigr).
\end{align*}
Then we should have (after remembering that we needed to scale by one over $1-p(x)^2 = 1-t^2$)
\[
\int_{S^{n-1}} f(\langle x, v \rangle) \sigma(x)
= \int_{-1}^1 f(t|v|) \frac{(1-t^2)^{(n-2)/2} (1 - t)}{|v| (1-t^2)} \Vol(S^{n-2}) \, dt.
\]


SIDEBAR: Let $v_1,\ldots,v_{n-1}$ be orthonormal vectors in $\kk R^n$ and let $x \in \kk R^n$. Consider
\[
x' = \sum_{j=1}^{n-1} \langle x, v_j \rangle v_j.
\]
Then $x - x'$ is orthogonal to all of the $v_j$ because $\langle x - x', v_j \rangle = \langle x, v_j \rangle - \langle x, v_j \rangle = 0$. So
\[
x = \sum_{j=1}^{n-2} \langle x, v_j \rangle v_j + (x - x')
\]
is the orthogonal projection of $x$ onto the hyperplane spanned by
the elements $v_1,\ldots,v_{n-1}$.
Let $v_n$ be orthogonal to all of the $v_j$ and of norm
$1$, so $x - x' = \langle x - x', v_n \rangle v_n$. Then
\[
\det(v_1,\ldots,v_{n-1}, x)
= \det(v_1,\ldots,v_{n-1}, \langle x, v_n \rangle v_n)
= \langle x, v_n \rangle,
\]
and
\[
\langle x, v_n \rangle^2
= |x|^2 - \sum_{j=1}^{n-1} \langle x, v_j \rangle^2.
\]




\section*{5. May 2021}
\subsection*{Return to kernels}

Let $V$ be a vector space and $| \, \cdot \, |$ a norm on $V$. Set
\[
K(x,y) = \frac{|x+y|-|x-y|}{|x+y|+|x-y|}.
\]
We'd like to prove the kernel matrix associated to $n$ different points in $V$ is positive-semidefinite.

Someone on math.SE did this for the original question by using a theorem of Schur on positive-semidefinite Hadamard products. It says that if $A$ and $B$ are positive-semidefinite, then the Hadamard product $A \odot B$ is as well. They applied this to the kernel matrices defined by $A(x,y) = |x+y|-|x-y|$ and $B(x,y) = 1/(|x+y|+|x-y|)$; we assume the points are picked in the open set where $|x+y| > |x-y|$.

Let $x_1,\ldots,x_n$ be pairwise distinct points in $V$. Define
\(
A = (A(x_j,x_k))
\).
We know that
\[
|x+y|-|x-y| \leq |x + y + x - y| = 2 |x|.
\]
Using $|x-y|=|y-x|$ we get the upper bound $2|y|$, so
\[
|x+y|-|x-y| \leq 2 \min(|x|,|y|) = |x| + |y| - ||x| - |y||.
\]

The set of positive-definite Hermitian matrices is contractible: If $h_1$ and $h_2$ are positive-definite inner products and $x \in V$, then
\[
p(t) = (1-t) h_1(x, \ov x) + t h_2(x, \ov x)
\]
is a linear polynomial that satisfies $p(0) > 0$ and $p(1) > 0$, so $p(t) > 0$ for all $t$ in between. The same works for any interpolating functions that result in a monotonic $p$.



I'd like to see a result where we can prove positive-definiteness by sandwiching a
kernel between functions that are positive-definite.

\begin{prop}
Let $K_1, K_2, K_3 : V \times V \to \kk R$ be continuous symmetric functions. Suppose $K_1$ and $K_3$ are positive-definite and that $K_1 \leq K_2 \leq K_3$. Then $K_2$ is positive-definite.
\end{prop}


\section*{28. April 2021}
\subsection*{A possibly semipositive kernel function}

On math.SE\footnote{https://math.stackexchange.com/q/4116252/3225} someone came up with this. Let
\[
C = \{ x \in \kk R^n \mid x_j \geq 0 \text{ for all $j$}\}
\]
and define a function $K : C \times C \setminus \{(0,0)\} \to \kk R$ by
\[
K(x,y) = \frac{\sum_{j} \min(x_j, y_j)}{\sum_j \max(x_j, y_j)}.
\]
This function is continuous, smooth outside of the different diagonals, is symmetric, and satisfies $0 \leq K(x,y) \leq 1$ with $K(x,y) = 1$ if and only if $x = y$. It is not homogeneous, but satisfies $\lambda K(x,y) = K(\lambda x, \lambda y)$ for $\lambda \geq 0$.

The question on math.SE asks whether the function $K$ defines a positive-semidefinite kernel. That is, let $x_1, \ldots, x_n \in \kk R^n$ and define a matrix $B = (K(x_j,x_k))$. Is $B$ semi-positive definite? I think if we restrict to the interior of $C$ we can ask whether $B$ is positive-definite.

The ``shape'' of the matrix $B$ is
\[
B =
\begin{pmatrix}
1  & \cdots & b_{1n}
\\
\vdots & \ddots & \vdots
\\
b_{n1} & \cdots & 1
\end{pmatrix}
\]
where $b_{jk} = b_{kj}$ and $0 \leq b_{jk} \leq 1$. If $n = 2$ these matrices satisfy $\tr B = 2 > 0$ and $\det B = 1 - b_{12}^2 \geq 0$, so they are positive-definite on the interior of $C$. In general, these matrices are not positive-semidefinite. Having numpy cruch some examples shows that the ratio of positive-definite matrices to general ones goes from $1$ when $n = 2$ to about $0.8$ when $n = 3$, and is less than one-millionth already when $n = 10$. As $n$ grows, we then expect almost none of these matrices to be positive-definite.

However, having numpy randomly sample the space of matrices generated by the kernel function $K$ has \emph{only} turned up matrices that are positive-definite in all dimensions I've tried.

Recall that
\[
\min(x,y) = \frac{x + y - |x - y|}{2}
\qandq
\max(x,y) = \frac{x + y + |x - y|}{2}.
\]
Set $\nu = (1,\ldots,1) \in \kk R^n$. Then
\begin{align*}
\frac{\sum_j \min(x_j,y_j)}{\sum_j \max(x_j,y_j)}
= \frac{\langle x + y, \nu \rangle - |x-y|_1}{\langle x + y, \nu \rangle + |x-y|_1}
= \frac{|x+y|_1 - |x-y|_1}{|x+y|_1 + |x-y|_1},
\end{align*}
where $|\,\cdot\,|_1$ is the $1$-norm, because $\langle x, \nu \rangle = |x|_1$ for vectors $x$ with all $x_j \geq 0$.

We have
\[
K(x,y)
= 1 - \frac{2 t}{1 + t} = \frac{1-t}{1+t},
\]
where $t = |x-y|_1/|x+y|_1$, which looks very familiar. It's close to being the angle-doubling formula for tan, but the denominator should have $t^2$. Our $t > 0$, so we can use its square root $s = \sqrt t$ and write
\[
K(x,y) = \frac{1-s^2}{1+s^2}
= \frac{1-\tan^2 \theta}{1 + \tan^2 \theta}
= \frac{\cos^2\theta - \sin^2\theta}{\cos^2\theta + \sin^2 \theta}
= \cos 2\theta,
\]
where $\tan \theta = s = \sqrt t = \sqrt{|x-y|_1 / |x+y|_1}$.
That is,
\[
\theta(x,y) = \tan^{-1} \sqrt{\frac{|x-y|_1}{|x+y|_1}}.
\]
I'm not sure this is helpful. As $0 \leq K(x,y) \leq 1$ we can just write $K(x,y) = \cos \theta(x,y)$ for $0 \leq \theta \leq \pi/2$, or $0 \leq 2\theta \leq \pi$.

Let $v_1, \ldots, v_n \in \kk R^n$ and define a symmetric matrix $B = (K(v_j, v_k))$. Let $x = (x_1, \ldots, x_n)$, then the associated quadratic form is
\begin{align*}
q(x)
= \sum_{jk} K(v_j, v_k) x_j x_k
&= \sum_{jk} \bigl(
1 - 2 \sin^2 \theta_{jk}
\bigr) x_j x_k.
\end{align*}
We'd like to write this as a square, something like
\begin{align*}
q(x) = \sum_{jk} \bigl(\cos 2\theta_{jk} \bigr) x_j x_k
&= \Bigl(\sum_{j} \cos \rho_jx_j \Bigr)^2
\\
&= \sum_{jk} \cos \rho_j\cos \rho_k \, x_j x_k.
\end{align*}
I think the condition to decompose the angle like this will only be fulfilled when $K$ satisfies a version of the parallelogram rule.


\section*{23. April 2021}

\subsection*{A natural bilinear form}

Someone on math.SE found a natural quadratic form on special kinds of vector spaces. Let $V$ be a vector space over $k$ and consider $V \oplus V^*$. On this space, we have
\[
q(x, \alpha) = \alpha(x).
\]
This is a quadratic form, as $q(\lambda(x,\alpha)) = q(\lambda x, \lambda\alpha) = \lambda^2 q(x, \alpha)$. The associated bilinear form is
\[
b((x, \alpha), (y, \beta))
= \frac12\bigl(q(x + y, \alpha + \beta) - q(x, \alpha) - q(y,\beta)\bigr)
= \frac12 \bigl(\alpha(y) + \beta(x)\bigr).
\]
We can also find this form by noting the linear morphism
\[
V \oplus V^* \to (V \oplus V^*)^* = V^* \oplus V,
\quad
(x, \alpha) \mapsto (\alpha, x)
\]
yields the same bilinear form (up to a constant). This bilinear form is
nondegenerate. Over $\kk R$ or $\kk C$ it is of mixed signature (if
$q(x,\alpha)$ is positive then $q(-x,\alpha)$ is negative).

If $(v_1, \ldots, v_n)$ is a basis of $V$, the matrix of this bilinear form is
\[
\begin{pmatrix}
0 & I_n \\
I_n & 0
\end{pmatrix}.
\]
If we look at the augmented matrix
\[
\begin{pmatrix}
0 & I_n & I_n & 0 \\
I_n & 0 & 0 & I_n
\end{pmatrix}
\]
and perform elementary operations, we see it is similar to
\[
\begin{pmatrix}
I_n & 0 \\
0 & -I_n
\end{pmatrix},
\]
so over $\kk R$ the signature of the form is $(\dim V, \dim V)$. (Over $\kk C$ there's only one nondegenerate bilinear form.)

There's nothing stopping us from doing this on a holomorphic bundle $E \oplus E^* \to X$. We get a holomorphic nondegenerate bilinear form $b$. We could look for connections that are compatible with it, in the sense that
\[
d b(s, t) = b(Ds, t) + b(s, Dt)
\]
for sections $s, t$ of $E \oplus E^*$. If $s$ and $t$ are holomorphic, so is the function $b(s, t)$, so I think $D^{0,1} = \bar\partial$. I think we can split $\partial b(s, t)$ in two halves and define
\[
\frac 12 d b(s, t) = b(Ds, t) = b(s, Dt)?
\]
But then whatever we get won't satisfy the Leibniz rule.


\section*{19. April 2021}
\subsection*{Nowhere-vanishing holomorphic one-form}


I saw a paper on the arXiv\footnote{https://arxiv.org/abs/2104.07074} about varieties with a nowhere-vanishing holomorphic one-form. Some examples of those are a projective bundle over a torus, and large symmetric powers of curves that fiber over the curve's Jacobian. (Or compact Kahler manifolds that surject onto their Albanese variety, like ones with nef anticanonical bundle\footnote{https://arxiv.org/abs/1305.4397}, I suppose.) They have in common that some of the higher Chern classes of the manifold vanish.

I was thinking whether we can construct a connection on the tangent bundle of the manifold whose curvature form vitnesses some of the vanishing? Let $\theta \in H^0(X, \Omega^1)$ be a nowhere-zero form. We get a short exact sequence
\[
\begin{tikzcd}
0 \ar[r] &
\Ker \theta \ar[r] &
T_X \ar[r] &
L \ar[r] &
0,
\end{tikzcd}
\]
where $L = T_X / \Ker \theta$ is a line bundle, and in fact an induced isomorphism $\theta : L \to \cc O_X$.

From this we get a smooth section $\nu \in \cc C^\infty(X, T_X)$ such that $\theta(\nu) = 1$ on $X$. We can only guarantee a smooth section, not holomorphic, because we don't know when the obstruction to a lift of $\theta^{-1}(1)$ in $H^1(X, \Ker \theta)$ is zero.


We'd like to define a connection $D$ on $T_X$ that satisfies
\[
\theta(D\xi) = d \theta(\xi).
\]
If $\xi$ is a section of $T_X$, we write
\[
\xi = (\xi - \theta(\xi)\nu) + \theta(\xi) \nu
\]
for the decomposition of $\xi$ into a part in $\Ker \theta$ and an orthogonal complement,
and define
\[
D\xi = d(\theta(\xi)) \otimes \nu.
\]
Then $\theta(D\xi) = d \theta(\xi)$. If $f$ is a smooth function, we have
\[
D(f\xi)
= d(\theta(f\xi)) \otimes \nu
= df \otimes \theta(\xi) \nu + f d\theta(\xi) \otimes \nu
= df \otimes \theta(\xi) \nu + f D\xi,
\]
which is not equal to $df \otimes \xi + f D\xi$,
so $D$ is not a connection.

This is ``lucky'', because if it were the curvature form of $D$ would be
\begin{align*}
D^2 \xi
= D(D\xi)
&= D(d \theta(\xi) \otimes \nu)
\\
&= d^2 \theta(\xi) \otimes \nu - d\theta(\xi) \wedge D\nu
\\
&= - d\theta(\xi) \wedge d\theta(\nu) \otimes \nu
= 0
\end{align*}
as $\theta(\nu) = 1$, so $D$ would be a flat connection on $T_X$. This contradicts the basic examples that exist, because some of their Chern classes are nonzero (and they're not complex tori).

I suppose the moral of the story is that we can't define a connection by just working on a piece of a decomposition of the bundle we care about. Or that I shouldn't try to do math in my head in the middle of the night while my daughter won't sleep.


\paragraph{}

But wait. Suppose we pick a connection $D$ on $\Ker \theta$. We then define an operator $\nabla$ on $T_X$ by
\[
\nabla \xi
= D(\xi - \theta(\xi) \nu) + d\theta(\xi) \otimes \nu.
\]
If $f$ is a smooth function, we have
\begin{align*}
\nabla(f \xi)
&= D(f\xi - f(\theta(\xi) \nu) + df \otimes \theta(\xi) \nu + f d\theta(\xi) \nu
\\
&= df \otimes (\xi - \theta(\xi) \nu) + df \otimes \theta(\xi) \nu + f D(\xi - \theta(\xi) \nu) + f d\theta(\xi) \otimes \nu
\\
&=
df \otimes \xi + f \nabla \xi,
\end{align*}
so $\nabla$ is a connection. We have
\[
\nabla^2 \xi = D^2(\xi - \theta(\xi) \nu),
\]
as the other part is zero as we saw above. Note that
\[
\nabla^2 \nu = D^2(\nu - \nu) = 0,
\]
so $\dim \Ker \nabla^2 \geq 1$ everywhere on $X$. Does it follow that $\wedge^n \nabla^2 = 0$? If so, then the Chern form $c_n(\nabla) = \tr \wedge^n \nabla^2 = 0$, and by Chern--Weil we get $c_n(T_X) = 0$.

After taking a walk, this follows directly from the short exact sequence
\[
\begin{tikzcd}
0 \ar[r] &
\Ker \theta \ar[r] &
T_X \ar[r] &
\cc O_X \ar[r] &
0
\end{tikzcd}
\]
and the additivity of the total Chern class $c(T_X) = c(\Ker \theta) \cup c(\cc O_X)$, which gives
\[
c_n(T_X) = c_{n-1}(\Ker \theta) \cup c_1(\cc O_X) = 0.
\]


\section*{10. March 2021}

\subsection*{Invariant Cartan structure equation}

In complex differential geometry, the Cartan structure equation
$$
i\Theta
= -iH^{-1} \partial \bar\partial H
+ i H^{-1} \partial H \wedge H^{-1} \bar\partial H
$$
for the curvature form of a Chern connection on a holomorphic vector bundle $(E, h) \to X$ lets us calculate a lot of things. The problem (to me) is that it uses the moving frame formulation of the curvature form, while I want to use the connection formulation for everything. I'd like to have some invariant way of using this equation.

I've played around with taking various derivatives, like $\partial\bar\partial$ of $\langle s, t \rangle$ or $\log |\langle s, t\rangle|^2$. This gives things that sometimes look interesting but are either not what we're looking for or problematic in other ways. I wonder if I'm not overthinking this? By just taking the derivatives of the metric we get
$$
\langle \Theta s, t \rangle
= \langle D s, D t \rangle
- \partial\bar\partial \langle s, t \rangle
$$
which might be enough for some of the applications where the Cartan equations are used.

For example, if $\cc S \to \Gr(k, V)$ is the universal subbundle of the Grassmannian, and $f,g \in \Hom(S,V/S)$ are elements of the tangent space at $S$, then
\begin{align*}
\partial_f\bar\partial_g h_S(s,t)
= \partial_f \bar\partial_g h_V(js,jt)
&= \partial_f h_V(js, (\partial_g j)(t) + jD_St)
\\
&= h((\partial_f j)(s) + jD_S s,(\partial_g j)(t) + jD_St).
\end{align*}
We have $(\partial_fj)(s) = f(s)$ (somehow), so
$$
\partial_f\bar\partial_g h_S(s,t)
= h_S(D_{S,f} s, D_{S,g} t) + h_Q(f(s),f(g)).
$$
Rearranging according to our version of Cartan's equation gives
$$
\langle F_{fg}s,t \rangle
= h_S(D_{S,f} s, D_{S,g} t) - \partial_f\bar\partial_g h_S(s,t)
= -h_Q(f(s),g(t)).
$$
The subbundle is then seminegative; we have
$$
\langle F_{ff}s,s \rangle
= -h_Q(f(s),f(s)).
$$
If $\dim S > 1$ then there always exists a nonzero morphism $f \in \Hom(S, V/S)$ such that $\Ker f \not= 0$. For such a morphism and $s \in \Ker f \setminus \{0\}$ we have $\langle F_{ff}s, s \rangle = 0$ while $f \not=0$ and $s \not= 0$, so $\cc S$ is only seminegative in this case.


\section*{2. March 2021}
\subsection*{Estimating the norm of the exterior product}

Let \(V\) be a real finite-dimensional vector space, and let \(\bigwedge^\bullet V\) be the exterior algebra of \(V\).
An inner product on \(V\) induces inner products on each component \(\bigwedge^k V\) of the exterior algebra.
Recall that we have the exterior product
\[
\ext{p} V \times \ext{q} V \to \ext{p+q} V,
\quad
(u, v) \mapsto u \wedge v.
\]
Since all the spaces here are finite-dimensional, there exist constants \(C = C(p,q,n)\) such that
\[
|u \wedge v| \leq C |u| \, |v|
\]
for all \(u \in \ext p V\) and \(v \in \ext q V\). Can we estimate these constants?

The answer apparently has applications to physics.
I don't understand the connection to physics or the physical motivation, but in a \href{https://aip.scitation.org/doi/10.1063/1.1703969}{1961 paper} the physicist \href{https://en.wikipedia.org/wiki/Yang_Chen-Ning}{Yang Chen-Ning} conjectured that for even-dimensional spaces the optimal bound is achieved on powers of the standard symplectic form.
As
\[
\frac{\omega^p}{p!} \wedge \frac{\omega^q}{q!} = \binom{p+q}p \frac{\omega^{p+q}}{(p+q)!},
\]
where \(\omega = \sum_{i=1}^n dx_{2i-1} \wedge dx_{2i}\) and \(|\omega^p / p!|^2 = \binom np\), the conjecture is that
\[
C(2p,2q,2n) = \tbinom{p+q}p\sqrt{\frac{\binom{n}{p+q}}{\binom np \binom nq}}.
\]
Yang claims to prove this for the case of \(p = 1\) in his 1961 paper, though I don't understand his proof.
A \href{https://arxiv.org/abs/1409.3931}{preprint from 2014} claims to prove more cases of the conjecture; again the details of the proof defeat me.

There are two ways of getting rather brutal estimates for these constants.
Let \((v_1, \ldots, v_n)\) be an orthonormal basis of \(V\).
Then \((v_I)_{|I| = p}\) and \((v_J)_{|J|=q}\) are orthonormal bases of \(\ext p V\) and \(\ext q V\).
If \(u\) and \(v\) are elements of these bases, then \(u \wedge v\) is either 0 or has norm 1.

For our first attempt, let \(x = \sum_{|I|=p} x_I v_I\) and \(y = \sum_{|J|=q} y_J v_J\) be elements of \(\bigwedge^p V\) and \(\bigwedge^q V\).
Then
\[
|x \wedge y|
=
\biggl|
\sum_{|I|=p, |J|=q} x_I y_J v_I \wedge v_J
\biggr|
\leq
\sum_{|I|=p, |J|=q} |x_I y_J|
= \sum_{|I|=p} |x_I| \cdot \sum_{|J|=q} |y_J|.
\]
By taking the inner product of \(x\) and the vector whose elements are the signs of each \(x_I\) and applying Cauchy--Schwarz, we get
\[
\sum_{|I|=p} |x_I| \leq \sqrt{\tbinom np} |x|.
\]
This gives
\[
|x \wedge y| \leq \sqrt{\tbinom np \tbinom nq} |x| |y|.
\]
Our first estimate is thus \(C(p,q,n) \leq \sqrt{\binom np \binom nq}\).
This is likely to be a bad estimate, because when applying the triangle inequality above we assigned equal weight to all expressions \(|v_I \wedge v_J|\), even though many of them were zero.
We also applied the Cauchy--Schwarz inequality on top of the triangle one.
The conditions for those inequalities to be exact are orthogonal -- the triangle one is exact when all the vectors lie on a straight line, while Cauchy--Schwarz is exact when all the vectors are orthogonal -- which should yield suboptimal bounds.

For our second attempt, we consider the bilinear map
\[
b : \ext p V \otimes \ext q V \to \ext {p+q} V,
\quad
u \otimes v \mapsto u \wedge v.
\]
Its operator norms satisfy \(\|b\|^2 \leq |b|^2\), where
\[
\|b\|^2 = \sup_{x,y} \frac{|b(x,y)|^2}{|x|^2|y|^2}
\]
and where \(|b|\) is the Hilbert--Schmidt norm. We can calculate that norm by counting the number of non-zero entries in the matrix for \(b\).

Let then \(I = (i_1, \ldots, i_p)\) be a multiindex. We can choose \(\binom{n-p}{q}\) indices \(J = (j_1, \ldots, j_q)\) such that the corresponding basis elements \(v_I\) and \(v_J\) satisfy \(b(v_I \otimes v_J) = v_I \wedge v_J \not= 0\). As there are \(\binom np\) ways of picking \(I\), we conclude that the Hilbert--Schmidt norm of \(b\) is
\[
|b|^2 = \tbinom np \tbinom {n-p}q = \tbinom n{n-p} \tbinom {n-p}q.
\]
We then get a slightly better estimate
\[
C(p,q,n)
\leq \sqrt{\tbinom n{n-p} \tbinom {n-p}q}
\leq \sqrt{\tbinom np \tbinom nq}
\]
with equality in the second place if and only if \(p = 0\) or \(p = n\).

A fun fact is that these estimates are arbitrarily bad.
When \(p + q = n\), the Hodge star operator gives an isometry \(\bigwedge^q V \cong \bigwedge^p V\) and the Cauchy--Schwarz inequality gives
\[
|u \wedge v| \leq |u| |v|,
\]
so \(C(p,n-p,n) = 1\), which agrees with Yang's conjectured value when \(p\) and \(n\) are even.
Our best estimate in this case is
\[
C(2p,2n-2p,2n) \leq \sqrt{\tbinom {2n-2p}{2p}},
\]
which goes to infinity as \(n\) grows.


\bibliographystyle{alpha}
\bibliography{main}


\end{document}
