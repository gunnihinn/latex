\documentclass[10pt,a4paper]{article}

\usepackage{tgpagella}
\linespread{1.1}
\usepackage[utf8]{inputenc}
\usepackage[T1]{fontenc}

\usepackage{fancyref}
\usepackage[colorlinks=true]{hyperref}

\usepackage{amsmath}
\usepackage{amssymb}
\usepackage[amsmath,thref,thmmarks,hyperref]{ntheorem}

\usepackage{tikz-cd}

\theorembodyfont{\itshape}
\newtheorem{theo}{Theorem}[section]
\newtheorem{prop}[theo]{Proposition}
\newtheorem{lemm}[theo]{Lemma}
\newtheorem{coro}[theo]{Corollary}

\theorembodyfont{\rm}
\newtheorem{defi}[theo]{Definition}
\newtheorem{exam}[theo]{Example}
\newtheorem*{claim}{Claim}

\theoremseparator{:}
\theoremsymbol{\ensuremath{\Box}}
\newtheorem*{proof}{Proof}

\newcommand{\kk}[1]{\mathbb{#1}}
\newcommand{\cc}[1]{\mathcal{#1}}

\def\qedhere{}

\def\^#1{^{[#1]}}
\def\qandq{\quad\text{and}\quad}
\def\ov#1{\overline{#1}}

\DeclareMathOperator{\Ric}{Ric}
\DeclareMathOperator{\pr}{pr}
\DeclareMathOperator{\Span}{span}
\DeclareMathOperator{\Gr}{Gr}
\DeclareMathOperator{\GL}{GL}
\DeclareMathOperator{\im}{Im}
\DeclareMathOperator{\Vol}{Vol}
\DeclareMathOperator{\Ker}{Ker}
\DeclareMathOperator{\End}{End}
\DeclareMathOperator{\Aut}{Aut}
\DeclareMathOperator{\Hom}{Hom}
\DeclareMathOperator{\id}{id}
\DeclareMathOperator{\tr}{tr}

\newcommand{\ext}[1]{\bigwedge{}^{\!\!#1}\,}

\newtheorem{question}{Question}

\author{Gunnar Þór Magnússon}
\date{\today}
\title{Grassmannian bundle curvature calculation project}


\begin{document}

\maketitle



\section{Plan}

We want to calculate the curvature tensor of a Grassmannian bundle and prove that if the base has positive holomorphic sectional curvature, then so does the total space of the bundle.

We are going to do this by:
\begin{itemize}
  \item Define a Kahler metric on the total space. It is the pullback of one from the base plus one coming from the second fundamental form's pullback of the Hermitian metric on a relevant homomorphism bundle.

  \item Prove a Codazzi--Griffiths formula for general Hermitian forms. Use it to calculate the curvature tensor of a sum of two Hermitian forms whose sum is a Hermitian metric.

  \item To do the above we need some preliminaries on general Hermitian linear algebra (that is, the linear algebra of potentially degenerate Hermitian forms). I don't know a reference for this, so we'll write one.

  \item We can then calculate the curvature tensor of a Grassmannian bundle. There are difficulties: the second fundamental form is not holomorphic, but it is holomorphic in the tangent directions we need to say something about the positivity of its curvature tensor; to ensure positivity of the holomorphic sectional curvature we have to scale the pullback metric and argue that the result is eventually positive, which should work out because we can diagonalize the quotient metric and show that its limit under this process exists.
\end{itemize}

Once we do this, we get that flag manifolds have positive holomorphic sectional curvature, because by forgetting a step in the flag we get a Grassmannian bundle over a flag manifold and continue by induction.



\section{Grassmannian bundles}


\subsection*{Metric basics}

We first review some basic facts about the K\"ahler--Einstein metric
on the Grassmannian.
Let $V$ be a complex vector space of dimension $n$ and let
$0 < k < n$.
The Grassmannian is the set $\Gr := \Gr(k, V)$ of $k$-dimensional subspaces of $V$.

There is a tautological vector bundle $\cc S \to \Gr(k,V)$, whose fiber at $S$ is
just $S$.
We have a short exact sequence
\[
    0 \to \cc S \to \underline V \to \cc Q \to 0
\]
of holomorphic vector bundles, where $\underline V$ is the trivial vector bundle
with fiber $V$ and $\cc Q$ is the quotient bundle.

There is a K\"ahler--Einstein metric on the Grassmannian.
It can be constructed in at least two ways:

First we can consider the Pl\"{u}cker embedding $\Gr(k,V) \to \kk P(\bigwedge\!\!{}^k
V)$ and pull back the Fubini--Study metric.
As the automorphism group of the Grassmannian is a subgroup of the automorphism
group of $\kk P(\bigwedge\!\!{}^kV)$, this shows the metric is invariant under
all of $\Aut \Gr$.

Second we can note like Griffiths that if we fix a Hermitian inner product on
$V$ and consider the induced flat metric on $\underline V$, the second
fundamental form associated to the short exact sequence above defines a
holomorphic isomorphism $\sigma : T_{\Gr} \to \Hom(\cc S, \cc Q)$.
The $\Hom$ bundle has the Hilbert--Schmidt metric induced by the inner product
we chose.
It is easy to compute the Chern connection of this metric and check that its
torsion tensor is zero, so it is a K\"ahler metric.
The Codazzi--Griffiths formulas then show the metric is Griffiths-semipositive.
In fact its curvature tensor is
\[
R(\alpha,\ov\beta,\gamma,\ov\delta)
= h(\sigma(\beta)^\dagger \sigma(\alpha),
    \ov{\sigma(\gamma)^\dagger \sigma(\delta)})_{\End \cc S}
+ h(\sigma(\beta)^\dagger \sigma(\gamma),
    \ov{\sigma(\alpha)^\dagger \sigma(\delta)})_{\End \cc S}.
\]

We can calculate the Ricci forms of each metric and check they are both
K\"ahler--Einstein, so they are multiples of each other because $h^{1,1}(\Gr(k,
V)) = 1$.
We conclude that there is a unique Griffits-semipositive K\"ahler--Einstein
metric on $\Gr(k, V)$ that is invariant under $\Aut \Gr(k, V)$ and has volume
one. When we speak of a K\"ahler metric on $\Gr(k, V)$ from now on we mean
that one.



\subsection*{Grassmannian bundles}


\begin{prop}
Let $\pi : X \to B$ be a Grassmannian bundle over a complex manifold.
There exists a closed $(1,1)$-form $\tau$ on $X$ whose restriction to any fiber
is the K\"ahler--Einstein metric on that fiber.
\end{prop}

\begin{proof}
If $X = \Gr(k, E)$, where $E \to B$ is a vector bundle, we can do this by doing
a relative Pl\"ucker embedding and pulling back the curvature form of the
tautological bundle.

Otherwise we probably have to copy Kodaira.

sorry
\end{proof}



Let $\pi : X \to B$ be a family of Grassmannian manifolds over a compact
K\"ahler manifold. Let $\omega_B$ be a K\"ahler metric on $B$ and let $\tau$ be
the above relative K\"ahler metric on $X$. We set $\omega_\lambda = \tau +
e^\lambda \omega_B$ for $\lambda \in \kk R$ and assume we've picked $\lambda$
big enough so that $\omega_\lambda$ is positive-definite.
We write $h_\lambda$ for the Hermitian metric associated to $\omega_\lambda$,
and $q_\lambda$ for the ``quotient'' form. The curvature tensor of $h_\lambda$
is
$$
\displaylines{
R_\lambda(\alpha, \ov\beta, \gamma, \ov\delta)
= R_\tau(\alpha, \ov\beta, \gamma, \ov\delta)
+ e^\lambda \pi^* R_B(\alpha, \ov\beta, \gamma, \ov\delta)
\hfill\cr\hfill{}
- q_\lambda(D_{\tau,\alpha}\gamma - \pi^*D_{B,\alpha}\gamma,
\ov{D_{\tau,\beta}\delta - \pi^*D_{B,\beta}\delta}).
}
$$
We know that $T_{X/B} = \Ker \pi_* \subset \Ker q_\lambda$, and that $q_\lambda
\to q$, where $q$ is a Hermitian form, when $\lambda \to \infty$. We also know
that $q_\lambda \geq 0$ for all $\lambda$ big enough.

Pick a point in $X$ and split $T_X = T_{X/B} \oplus \pi^*T_B$ orthogonally at
that point. For $\alpha \in T_{X/B}$ we have
\begin{align*}
H_\lambda(\alpha)
&= R_\tau(\alpha, \ov\alpha, \alpha, \ov\alpha)
- q_\lambda(D_{\tau,\alpha}\alpha, \ov{D_{\tau,\alpha}\alpha}).
\\
&\geq
R_\tau(\alpha, \ov\alpha, \alpha, \ov\alpha)
- h_\lambda(D_{\tau,\alpha}\alpha, \ov{D_{\tau,\alpha}\alpha}).
\end{align*}








\section{Maybe salvageable}


\begin{coro}
\label{grassmannian-bundle-positive}
Let $\pi : X \to B$ be a Grassmannian bundle over a compact K\"ahler manifold
$B$.
If $B$ admits a K\"ahler metric with either
\begin{itemize}
    \item positive holomorphic sectional curvature, or
    \item positive Ricci curvature, or
    \item positive scalar curvature
\end{itemize}
then $X$ admits a K\"ahler metric with the same property.
\end{coro}

\begin{proof}
The claim about positive holomorphic sectional curvature is clear from our
local expression for the curvature tensor of the metric.

For the Ricci curvature, pick a point of $X$ and split the family locally around
it. We can then construct an orthonormal basis
$(\zeta_1,\ldots,\zeta_n,\zeta_{n+1},\ldots,\zeta_{n+m})$ of $T_{X}$ at that point such that
$(\zeta_1,\ldots,\zeta_n)$ is an orthonormal basis of $T_{\Gr}$ and
$(\zeta_{n+1},\ldots,\zeta_{n+m})$ is an orthonormal basis of $T_B$. Then the Ricci
curvature of the metric at that point is
\begin{align*}
\sum_{j=1}^{n+m} R(\alpha, \ov\beta, \zeta_j, \ov \zeta_j)
&= \sum_{j=1}^{n} \pr_{\Gr}^*R_{\omega_{\Gr}}(\alpha, \ov{\beta}, \zeta_j, \ov \zeta_j)
+ \sum_{j=n+1}^{m} \pi^* R_{\omega_{B}}(\alpha, \ov{\beta}, \zeta_j, \ov \zeta_j)
\\
&= \pr_{\Gr}^*\Ric_{\omega_{\Gr}}(\alpha, \ov\beta)
+ \pi^*\Ric_{\omega_B}(\alpha, \ov\beta),
\end{align*}
which is positive as the Ricci curvatures of the metrics on $\Gr$ and $B$ are
positive.

The claim about the scalar curvature follows from similar calculations as the
Ricci curvature.
\end{proof}


\begin{coro}
Let $\pi : X \to B$ be a family of flag manifolds over a compact K\"ahler
manifold.
If $B$ admits a K\"ahler metric with either
\begin{itemize}
    \item positive holomorphic sectional curvature, or
    \item positive Ricci curvature, or
    \item positive scalar curvature
\end{itemize}
then $X$ admits a K\"ahler metric with the same property.
\end{coro}

\begin{proof}
If we forget one flag in a flag manifold $P$ we get a new flag manifold $P'$ and
a holomorphic submersion $P \to P'$ whose fiber is a Grassmannian. Do this in
the whole family, and we get a new family $\pi' : X' \to B$ whose fibers are
flag manifolds along with a submersion $X \to X'$ whose fibers are
Grassmannians. This way we recurse via Corollary~\ref{grassmannian-bundle-positive}
to the base case of a family of Grassmannians over a compact K\"ahler manifold,
where we know the assertion holds.
\end{proof}


\bibliographystyle{plain}
\bibliography{main}


\end{document}
