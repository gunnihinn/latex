\section{Schemes}

Let $A$ be a commutative ring with unit.


\subsection*{Zariski topology}

\begin{defi}
$\Spec A$ is the set of prime ideals of $A$.
\end{defi}

If $\ff a$ is any ideal of $A$, we define $V(\ff a)$ to be the set of prime ideals of $A$ that contain $\ff a$.
Krull's theorem shows that $V(\ff a)$ is not empty for any given ideal.


\begin{prop}
\begin{enumerate}
\item
$V(\ff a \ff b) = V(\ff a) \cup V(\ff b)$.

\item
If $(\ff a_i)$ is any set of ideals then $V(\sum \ff a_i) = \bigcap V(\ff a_i)$.

\item
If $\ff a$ and $\ff b$ are ideals then $V(\ff a) \subset V(\ff b)$ if and only if $\sqrt a \supset \sqrt b$.
\end{enumerate}
\end{prop}

\begin{proof}
(1) Let $\ff a \ff b \subset \ff p$ and suppose $\ff b \not\subset \ff p$.
Then there is some $b \in \ff b$ such that $b \not\in \ff p$.
For any $a \in \ff a$ we have $ab \in \ff a \ff b \subset \ff p$, and since $\ff p$ is prime we get $a \in \ff p$.
Conversely, suppose $\ff a \subset \ff p$.
If $\sum a_i b_i$ is an element of $\ff a \ff b$ then it is in $\ff p$ as well by primality.

(2) Since $\ff a_i \subset \sum \ff a_j$ for any $i$ we clearly have $V(\sum \ff a_i) \subset \bigcap V(\ff a_i)$.
Suppose that $\ff a_i \subset \ff p$ for every $i$.
Then $\sum \ff a_i \subset \ff p$ because it's an ideal.

(3) Recall that 
\[
\sqrt \ff a = \{ r \in A \mid r^n \in \ff a \text{ for some $n$}\}
= \bigcap_{\ff a \subset \ff p} \ff p.
\]
Then the statement is clear.
\end{proof}


\begin{defi}
The \emph{Zariski topology} on $\Spec A$ is defined by taking the sets generated by the $V(\ff a)$ to be closed.
\end{defi}

We write $D(\ff a) = \Spec A \setminus V(\ff a)$ for the open complement of a closed basis set.
These sets form a basis for the Zariski topology.

If $a \in A$ is any element then $Aa \subset A$ is a principal ideal.
We write $V(a)$ and $D(a)$ for the closed and open sets defined by $Aa$ in slight abuse of notation.

If $\ff a$ is any ideal and $a \in \ff a$ then $Aa \subset \ff a$.
Therefore $\sum_{a \in \ff a} Aa \subset \ff a$, and the reverse inclusion is clear so we have equality.
Thus $V(\ff a) = V(\sum_{a \in \ff a} Aa) = \bigcap_{a \in \ff a} V(a)$, so both the $V(a)$ and $D(a)$ also generate the Zariski topology.


\begin{prop}
$\Spec : \cat{Ring} \to \cat{Top}$ is a contravariant functor from the category of commutative rings to the category of topological spaces.
\end{prop}

\begin{proof}
Let $f : A \to B$ be a ring morphism.
If $\ff p \subset B$ is prime then so is $f^{-1}(\ff p) \subset A$:
If $a b \in f^{-1}(\ff p)$ then $f(a) f(b) \in \ff p$ so either $f(a)$ or $f(b)$ is in $\ff p$, and then either $a$ or $b$ is in $f^{-1}(\ff p)$.
Therefore $\Spec f: \Spec B \to \Spec A$, $\ff p \mapsto f^{-1}(\ff p)$, is well-defined.
We want to see that it is continuous.

Let then $V(\ff a)$ be a closed basis set in $\Spec A$.
Then
\begin{align*}
(\Spec f)^{-1}(V(\ff a))
&= \{ \ff p \in \Spec B \mid \ff a \subset f^{-1}(\ff p) \}
\\
&= \{ \ff p \in \Spec B \mid f(\ff a) \subset \ff p \}
= V(f(\ff a))
\end{align*}
is closed, so $\Spec f$ is continuous.
\end{proof}


\begin{prop}
If $X$ is some set of elements of $A$ then
\[
\Spec A = \bigcup_{f \in X} D(f)
\]
if and only if $1 \in (X)$, the ideal generated by the elements of $X$.
\end{prop}

\begin{proof}
The space $\Spec A$ is the union of all the $D(f)$ if and only if every point $\ff p$ does not contain some $f \in X$.
This means no prime ideal contains $(X)$.
But every proper ideal is contained in a maximal ideal, which is prime, so this happens if and only if $1 \in (X)$.
\end{proof}

Note that $1 \in (X)$ if and only if there are some $g_j \in A$ and $f_j \in X$ such that
\[
1 = \sum_{j=1}^k g_j f_j.
\]
According to Mumford this plays the role of a partition of unity in differential geometry.






\begin{exam}
If $I \subset A$ is an ideal then $A/I$ is also a commutative ring and we have a surjective morphism $p : A \to A/I$.
We get a corresponding continuous map $f : \Spec A/I \to \Spec A$.
Suppose that $f(\ff p) = f(\ff q)$ in $\Spec A$.
This means that $p^{-1}(\ff p) = p^{-1}(\ff q)$ in $A$.
Let $f \in \ff p$.
Find some $g \in p^{-1}(\ff p)$ such that $p(g) = f$.
Then $g \in p^{-1}(\ff q)$ so $f = p(f) \in p(p^{-1}(\ff q)) = \ff q$ because $p$ is surjective.
Therefore $\ff p \subset \ff q$, and by symmetry $\ff p = \ff q$.
That is, the map $f : \Spec A/I \to \Spec A$ is injective, and embeds $\Spec A/I$ in $\Spec A$.
We claim that $f(\Spec A/I) = V(I)$.
If $\ff p \in \Spec A/I$ is prime then $0 \in \ff p$ and so $p^{-1}(\ff p)$ is a prime ideal that contains $I$, so it is in $V(I)$.
Conversely, if $\ff p \in V(I)$ we claim that $p(\ff p)$ is prime.
Suppose $ab \in p(\ff p)$.
There exist $f,g \in A$ such that $p(f) = a$ and $p(g) = b$, so $fg \in \ff p$.
Then we may suppose that $f \in \ff p$, and then $a \in p(\ff p)$.
\end{exam}




\subsection*{Localization of a ring}

Recall that a set $S \subset A$ is \emph{multiplicative} if it is closed under multiplication and contains $1$.
If $\ff p \subset A$ is a prime ideal then $S = A \setminus \ff p$ is a multiplicative set.
If $a \in A$ is any element then $\< a \> := \{a^n \mid n \geq 0\}$ is also a multiplicative set.
Given any such set we define the localization $S^{-1}A$ as $(A \times S) / \sim$, where
\[
(a_1, s_1) \sim (a_2, s_2)
\iff
t(a_1 s_2 - a_2 s_1) = 0
\]
for some $t \in S$.
The class of $(a,s)$ is denoted by any of $a/s$ or $s^{-1}a$.

\begin{prop}
$S^{-1}A$ is a commutative ring and there is a homomorphism $j : A \to S^{-1}R$ given by $j(a) = a/1$.
Every element of $j(S)$ is a unit.
It satisfies this universal property:
If $f : A \to T$ is a homomorphism that maps every element of $S$ to a unit in $T$, there exists a unique homomorphism $g : S^{-1}R \to T$ such that
\[
\begin{tikzcd}
A \ar[r,"j"] \ar[rd,"f"] &
S^{-1}A \ar[d,"g"]
\\
& T
\end{tikzcd}
\]
commutes.
\end{prop}

\begin{proof}
We define
\[
a_1/s_1 + a_2/s_2
= (a_1 s_2 + a_2 s_1) / (s_1 s_2)
\qandq
(a_1/s_1) (a_2/s_2) = (a_1a_2)/(s_1s_2).
\]
This makes $S^{-1}A$ into a commutative ring with unit $1/1$.
The map $j$ is a ring homomorphism.
If $s \in S$ then $s/1 \cdot 1/s = 1$, so $j(s) = s/1$ is a unit.

Now let $f : A \to T$ be a ring morphism such that $f(S)$ consists of units in $T$.
If $t(a_1 s_2 - a_2 s_1) = 0$ then
\[
0 = f(t(a_1 s_2 - a_2 s_1))
= f(t)(f(a_1) f(s_2) - f(a_2) f(s_1)).
\]
Thus $f \times f$ is constant on the fibers of $A \times S \to S^{-1}A$, so it defines a map $g : S^{-1}A \to T$ by $g(a/s) = f(a) / f(s)$, and $g(j(a)) = g(a/1) = f(a)$.

If $g' : S^{-1}A \to T$ is any other such map then $(g - g')j = 0$.
Since $a/s = j(a) \cdot 1/s$ for any element of $S^{-1}A$ it follows that $g = g'$.
\end{proof}

If $\ff p \subset A$ is prime we write $A_{\ff p}$ for $(A \setminus \ff p)^{-1}A$.
If $a \in A$ is any element we write $A_a$ for $\< a \>^{-1}A$.




\subsection*{The structure sheaf}


For an open set $U \subset \Spec A$ we define \(\cc O(U)\) to be the set of functions $s : U \to \coprod_{\ff p \in U} A_{\ff p}$ such that $s(\ff p) \in A_{\ff p}$ and $s$ is locally a quotient of elements of $A$:
For every $\ff p \in U$ there is a neighborhood $\ff p \in V \subset U$ and $a,f \in A$ such that for each $\ff q \in V$, $f \not\in \ff q$ and $s(\ff q) = a/f$ in $A_{\ff q}$.

The sums and products of elements of $\cc O(U)$ is defined stalk-by-stalk, and $0$ and $1$ give neutral elements.
Then $\cc O(U)$ is a ring.
If $V \subset U$ then we get restriction maps $\cc O(U) \to \cc O(V)$, so we have a presheaf.
Hartshorne says it's clear this is a sheaf and I'd like to believe him.
From now on when we talk about $\Spec A$ we mean the ringed space $(\Spec A, \cc O)$.


\begin{prop}
Let $A$ be a ring.
\begin{enumerate}
\item
For any $\ff p \in \Spec A$, we have $\cc O_{\ff p} = A_{\ff p}$.

\item
For any $f \in A$ the ring $\cc O(D(f))$ is isomorphic to $A_{f}$.

\item
$\cc O(\Spec A) \cong A$.
\end{enumerate}
\end{prop}


If $U \subset \Spec A$ is open there exists an open covering $(D(f_i))_{i \in I}$ of $U$ by basis sets.
We have $D(f_i) \cap D(f_j) = D(f_i f_j)$ for $i \not= j$.
The sequence
\[
\begin{tikzcd}
0 \ar[r] &
\cc O(U) \ar[r] &
\displaystyle
\prod_{i \in I} A_{f_i} \ar[r] &
\displaystyle
\prod_{i,j \in I} A_{f_i f_j}
\end{tikzcd}
\]
is exact.
So $\cc O(U)$ is a subring of the product.



\begin{exam}
If $A$ is any ring then $(0)$ is a prime ideal so $(0) \in \Spec A$.
Let $C \subset \Spec A$ be a closed set.
Then there are $\ff a_i \in A$ such that $C = \bigcap_{i \in I} V(\ff a_i)$.
Now $0 \in \ff a_i$ for every $i$, so $(0) \in C$.
Therefore the closure of $(0)$ is all of $\Spec A$.
\end{exam}


\begin{exam}
If $k$ is a field then the only proper ideal of $k$ is $\{0\}$.
Therefore $\Spec k$ is a single point and its structure sheaf is $k$.
\end{exam}

\begin{exam}
Let $k$ be a field and let $A = k[x]$.
This is a principal ideal domain so for any ideal $I \subset A$ there exists $f \in k[x]$ such that $I = (f)$.
We may suppose $f$ is monic.
Then $I$ is prime if and only if $f$ is irreducible (or zero).
Therefore as a set $\Spec k[x]$ is isomorphic to the set of irreducible polynomials over $k$ along with zero.
If $f$ is a nonzero irreducible polynomial then $(f) \subset V(f)$.
If $(f) \subset \ff p$ we write $\ff p = (g)$ for some irreducible $g$.
Then $f \mid g$ so $f = g$ up to a unit because both are irreducible.
Thus $(f) = V(f)$, so $(f)$ is a closed point.
\end{exam}
