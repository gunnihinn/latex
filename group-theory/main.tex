\documentclass[11pt]{article}

\usepackage{tgpagella}
\linespread{1.1}
\usepackage[utf8]{inputenc}
\usepackage[T1]{fontenc}

\usepackage[normalem]{ulem}
\usepackage{textcomp}
\usepackage{hyperref}
\usepackage{tikz-cd}

\usepackage{amsmath}
\usepackage{amssymb}
\usepackage{amsthm}

\newtheorem{theo}{Theorem}
\newtheorem{prop}[theo]{Proposition}
\newtheorem{lemm}[theo]{Lemma}
\newtheorem{coro}[theo]{Corollary}
\theoremstyle{definition}
\newtheorem{defi}[theo]{Definition}
\newtheorem{exam}[theo]{Example}

\newcommand{\kk}[1]{\mathbb{#1}}
\newcommand{\cc}[1]{\mathcal{#1}}

\def\eps{\varepsilon}
\def\empty{\varnothing}

\def\oint#1#2{(#1, #2)}
\def\cint#1#2{[#1, #2]}
\def\coint#1#2{[#1, #2)}
\def\ocint#1#2{(#1, #2]}

\def\ov#1{\overline{#1}}

\def\grp{\mathsf{Grp}}
\def\set{\mathsf{Set}}
\def\CC{\mathbf{C}}
\def\EE{\mathcal{E}}
\def\FF{\mathcal{F}}
\def\NN{\mathbf{N}}
\def\RR{\mathbf{R}}
\def\QQ{\mathbf{Q}}
\def\ZZ{\mathbf{Z}}

\DeclareMathOperator{\aut}{Aut}
\DeclareMathOperator{\ke}{Ker}
\DeclareMathOperator{\im}{Im}
\DeclareMathOperator{\sgn}{sgn}
\DeclareMathOperator{\GL}{GL}

\author{Gunnar Þór Magnússon}
\date{\today}
\title{The things I remember about group theory}

\begin{document}

\maketitle


\section{Definition and general properties}

The orbital bombardment definition of a group is the one from category theory:

\begin{defi}
A \emph{group} is a category with one object whose arrows are all isomorphisms.
\end{defi}

Unpacking this leads to the usual definition given to undergraduates (at least
when the category in question is backed by sets):

\begin{defi}
A \emph{group} is a nonempty set $G$ with a binary operation $\cdot$ that satisfy the following axioms:
\begin{enumerate}
\item
There exists an element $e$ such that $e \cdot g = g \cdot e$ for all $g \in G$.

\item
For every element $g$ there exists an element $g^{-1}$ such that $g \cdot g^{-1} = g^{-1} \cdot g = e$.

\item
The binary operation is associative: For all elements $g_1, g_2, g_3$ we have $g_1 \cdot (g_2 \cdot g_3) = (g_1 \cdot g_2) \cdot g_3$.
\end{enumerate}
\end{defi}

The element $e$ is called the \emph{neutral} element of the group.
An element $g^{-1}$ is called the \emph{inverse} of $g$.
Before looking at some examples let's establish the uniqueness of some elements
here.


\begin{prop}
The neutral element of a group is unique.
\end{prop}

\begin{proof}
Let $e_1$ and $e_2$ be two neutral elements of a group.
Then
$e_1 = e_1 \cdot e_2 = e_2$
by the axioms for neutral elements.
\end{proof}


\begin{prop}
The inverse of an element is unique.
\end{prop}

\begin{proof}
Let $h_1$ and $h_2$ be two inverses of $g$.
Then
$h_1 g = e = h_2 g$
and multiplying this on the right by, say, $h_1$ gives $h_1 = h_2$.
\end{proof}

\begin{prop}
The inverse of $g \cdot h$ is $h^{-1} \cdot g^{-1}$.
\end{prop}

\begin{proof}
We have $(h^{-1} \cdot g^{-1}) \cdot (g \cdot h) = e$ and
$(g \cdot h) \cdot (h^{-1} \cdot g^{-1}) = e$ by associativity,
so the result holds by uniqueness of the inverse.
\end{proof}



\paragraph{Commutative groups}

Many familiar objects are groups, like the numbers $\ZZ$, $\QQ$, $\RR$ and
$\CC$ under addition.
So are the quotients $\ZZ / n \ZZ$ for $n > 1$ or $\QQ / \ZZ$ or $\RR / \QQ$.
These are all \emph{commutative}, meaning that $g \cdot h = h \cdot g$ for all
elements $g,h$ in the group.
We usually write $+$ for the binary operation in a commutative group.

\paragraph{Noncommutative groups}

Some more examples are matrix groups, like $\GL_n(\RR)$, where
the operation is matrix multiplication.
These are noncommutative.
So are the finite permutation groups $S_n$, which are the groups of
permutations of a finite set of size $n$.
In general the group $\operatorname{Aut} X$ of bijective functions $X \to X$ of
any set $X$ is a noncommutative group under function composition.

We often skip writing out a symbol for the binary operation of a noncommutative
group and just write $gh$ instead of $g \cdot h$.
This is called multiplicative notation, in contrast to the additive notation
used for commutative groups.


\begin{defi}
A subset $H \subset G$ is a \emph{subgroup} of $G$ if it is a group with the
binary operation of $G$.
\end{defi}

In any group $G$ the neutral element $\{ e \}$ is a subgroup.
So is the whole group $G$.
There may not be any others, even if the group is larger, like in $\ZZ / p \ZZ$
where $p$ is prime.
There may also be lots others, like in a permutation group where we can look at
the permutations that fix some set of elements.


\begin{defi}
A \emph{morphism} of groups is a function $f : G \to H$ such that
$f(g \cdot h) = f(g) \cdot f(h)$ for all elements $g,h \in G$.
\end{defi}


Some useful facts about morphisms are easy to establish:


\begin{prop}
If $f : G \to H$ is a morphism then $f(e) = e$.
\end{prop}

\begin{proof}
We have $e^2 = e$ so
$f(e) = f(e^2) = f(e)^2$.
Multiplying both sides by $f(e)^{-1}$ gives $f(e) = e$.
\end{proof}

We apologize for the notational abuse above.
We should have written $e_G$ and $e_H$ to distinguish between the neutral elements of each group.
The reader will just have to get used to this.
No one writes these out.


\begin{prop}
If $f : G \to H$ is a morphism and $g \in G$ then $f(g^{-1}) = f(g)^{-1}$.
\end{prop}

\begin{proof}
We have 
\begin{align*}
e &= f(e) = f(g g^{-1}) = f(g) f(g^{-1}),
\\
e &= f(e) = f(g^{-1} g) = f(g^{-1}) f(g)
\end{align*}
and conclude by the uniqueness of the inverse of an element.
\end{proof}


\begin{prop}
Let $f : G \to H$ be a morphism.
\begin{itemize}
\item
If $N \subset G$ is a subgroup, then $f(N) \subset H$ is a subgroup.

\item
If $N \subset H$ is a subgroup, then $f^{-1}(N) \subset G$ is a subgroup.
\end{itemize}
\end{prop}

\begin{proof}
The first property follows directly from the defining property of a morphism.

For the second, let $x,y \in N$ and let $z,w$ be such that $f(z) = x$ and $f(w)
= y$.
Then $f(zw) = f(z) f(w) = xy \in N$, so $zw \in f^{-1}(N)$.
We also have $f(e) = e$ so $e \in f^{-1}(N)$.
If $f(z) = x$ then $f(z^{-1}) = f(z)^{-1} = x^{-1} \in N$ so $z^{-1} \in
f^{-1}(N)$.
\end{proof}



We could also have defined morphisms as functors between categories of one object.
A morphism is injective, surjective, an isomorphism if it is those things as a
function.
Given a morphism $f : G \to H$ we define its kernel as
$$
\ke f = \{ g \in G \mid f(g) = e \}
$$
and its image as
$$
\im f = \{ f(g) \mid g \in G \}.
$$
These are subgroups of $G$ and $H$, respectively.
A morphism is injective if and only if $\ke f = 1$ and surjective if and only
if $\im f = H$.
Here we have written $1$ for the group that contains one element, which is
unique up to isomorphism.


\begin{theo}[Caley]
Any group is isomorphic to a subgroup of a permutation group.
\end{theo}

\begin{proof}
Let $G$ be a group and consider the permutation group $\aut G$.
An element $g$ of $G$ acts on $G$ by $g \cdot h = gh$ and thus defines an
element of $\aut G$; this in turn defines a function $j : G \to \aut G$.
If $g_1, g_2$ are in $G$ and $h$ is also in $G$ then we have
$$
j(g_1 g_2)(h)
= (g_1 g_2) h
= g_1 (g_2 h)
= j(g_1)(j(g_2)(h))
= j(g_1) j(g_2) (h)
$$
so $j$ is a group morphism. If $j(g_1) = j(g_2)$ then $g_1 = g_1 e = j(g_1)(e)
= j(g_2)(e) = g_2 e = g_2$, so $j$ is injective.
\end{proof}


\begin{defi}
If $H$ is a subgroup of a group $G$ we call the sets
$$
gH = \{ gh \mid h \in H \},
$$
where $g$ varies over $G$, the \emph{left cosets} of $H$.
The \emph{right cosets} of $H$ are the sets $Hg = \{ hg \mid h \in H \}$.
The \emph{index} of $H$ in $G$, written $|G : H|$, is the number of left cosets
of $H$ in $G$.
\end{defi}

The left and right coset of an element are isomorphic as sets; the map
$$
gH \longrightarrow Hg,
\quad
x \mapsto g^{-1} x g
$$
is an isomorphism.
They may not be equal.
TODO: Find example.


\begin{theo}[Lagrange]
Let $H \subset G$ be a group and a subgroup.
The left cosets of $H$ define a partition of $G$ into disjoint subsets that are
all isomorphic.
In particular,
$$
|G| = |G : H| \, |H|.
$$
\end{theo}

\begin{proof}
Consider the left cosets of $H$ in $G$; that is, the sets $g H$ where $g$
varies over the elements of $G$.
Clearly $G = \bigcup_{g \in G} gH$ so they cover all of $G$.

We'll now show that the cosets are disjoint.
Suppose $g_1 H \cap g_2 H \not= \varnothing$.
Then there exist $h_1$ and $h_2$ in $H$ such that $g_1 h_1 = g_2 h_2$.
Then $g_1 = g_2 (h_2 h_1^{-1})$, so $g_1 H \subset g_2 H$.
Similarly we get $g_2 H \subset g_1 H$, so $g_1 H = g_2 H$.

Now note that if $gH$ is a coset, then $f : H \to gH$ given by $f(h) = gh$ is
an isomorphism of sets:
If $f(h_1) = f(h_2)$ then we multiply by $g^{-1}$ and find $h_1 = h_2$, so $f$ is injective.
If $gh \in gH$ then $gh = f(h)$, so $f$ is surjective.
\end{proof}


Lagrange's theorem helps restrict the possible subgroups of a group.
Consider for example the group $\ZZ / p \ZZ$, where $p$ is prime.
Its order is $p$, so from Lagrange's theorem any of its subgroups must have
order 1 or $p$, so its only subgroups are the trivial group 1 and itself.

Likewise the group $S_n$ of permutations of a set of size $n$ has order $n!$,
so we might expect it to have lots of subgroups.
Looking at the subgroups that permute $n$ elements but fix $k$ of them we find
subgroups of order $(n-k)!$ for $k = 0, \ldots, n$.
It's less clear if $S_n$ has a subgroup of order $k$ for $k = 2, \ldots, n-1$.
Lagrange's theorem only says they might exist, not that they do.



\paragraph{Quotients}

We have kernels and images and would like to have cokernels.
For that we need to be able to take quotients.
That doesn't always work in groups.

Let $H \subset G$ be a group and a subgroup.
A way to start to define a quotient of $G$ by $H$ is to note that $H$ defines
an equivalence relation on $G$ by $x \sim y$ if and only if $x^{-1}y$ is in
$H$.
This is equivalent to saying that $x H = y H$, that is, that $x$ and $y$ define
the same left coset of $H$.
This is indeed an equivalence relation:
\begin{itemize}
\item
$x \sim x$ because $x^{-1} x = e \in H$.

\item
If $x \sim y$ then $x^{-1} y \in H$, which is a subgroup so the inverse
$(x^{-1} y)^{-1} = y^{-1} x \in H$, which means that $y \sim x$.

\item
If $x \sim y$ and $y \sim z$ then $x^{-1}y \in H$ and $y^{-1} z \in H$, so
because $H$ is a group we get $x^{-1} z = x^{-1} y y^{-1} z \in H$.
\end{itemize}
We can then form the set-theoretic quotient $G/H$ by this relation and get the
projection map $\pi : G \to G/H$ that sends an element to its equivalence class.



\begin{theo}
Let $H \subset G$ be a group and a subgroup.
The following are equivalent:
\begin{itemize}
\item
The set $G / H$ can be given a group structure that makes $\pi : G \to G/H$
into a morphism.

\item
The subgroup $H$ satisfies $g^{-1} H g = H$ for all $g$ in $G$.
\end{itemize}
\end{theo}

\begin{proof}
If $\pi$ is a morphism
and $h$ is in $H$ then we must have
$$
\pi(g) 
= \pi(eg)
= \pi(e) \pi(g)
= \pi(h) \pi(g)
= \pi(hg)
$$
for any $g$ in $G$.
Multiplying by $\pi(g)^{-1} = \pi(g^{-1})$ we get
$$
\pi(e) = \pi(g^{-1} h g),
$$
which must be true for all $h \in H$ and all $g \in G$.
That is, the group $H$ must satisfy $g^{-1} H g = H$ for any $g$ in $G$.


Suppose then that $H$ satisfies this condition.
Let $x$ and $y$ be elements of $G/H$.
Pick some $g_x$ and $g_y$ in $G$ such that $\pi(g_x) = x$ and $\pi(g_y) = y$.
Then we want
$$
xy = \pi(g_x) \pi(g_y) = \pi(g_x g_y)
$$
to hold.
This has to work independently of the elements $g_x$ and $g_y$ we picked.
Let then $h_x$ and $h_y$ be some elements of $H$.
Then we have
$$
\pi(g_x h_x g_y h_y)
= \pi(g_x h_x g_y)
= \pi(g_x g_y (g_y^{-1} h_x g_y))
= \pi(g_x g_y)
$$
because $g_y^{-1} h_x g_y \in H$ by hypothesis.
\end{proof}


\begin{defi}
A subgroup $H \subset G$ that satisfies $g^{-1} H g = H$ for all $g \in G$
is called a \emph{normal subgroup}, denoted $H \triangleleft G$.
\end{defi}


John Baez has a neat way of looking at what this condition means.\footnote{https://math.ucr.edu/home/baez/normal.html}
Briefly, the operation $x \mapsto g^{-1} x g$ is a kind of coordinate or
point-of-view change.
A subgroup is then normal if it looks the same from everywhere,
or if no particular choice was made when defining it.
This immediately leads to an example of a nonnormal subgroup by defining one by
making an arbitrary choice:


\begin{exam}
Consider $S_3$, the group of permutations on $X = \{1,2,3\}$.
Consider also $H \subset S_3$ defined as the group of permutations of $X$ that
fix the element $3$, arbitrarily.
This is a subgroup of $S_3$, isomorphic to $S_2$.
Let $g$ be the element of $S_3$ that swaps 3 and 2.
Let $h$ be the element of $H$ that swaps 2 and 1.
Then $g^{-1} h g$ is the element of $S_3$ that swaps 1 and 3, which is not in $H$.
Therefore $H$ is not a normal subgroup.
% 123
% 132 g
% 312 hg
% 321 g^{-1}hg

In contrast, consider the element $h$ that maps 1 to 2, 2 to 3 and 3 to 1 and
the subgroup $N = \{e, h, h^2\}$ generated by it.
This subgroup is invariant under conjugation by the other elements of $S_3$,
which swap some pair of elements, so it is normal.
\end{exam}


TODO: Examine the failure for non-normal subgroups.
Doing that for $S_3$ involves looking at $6 \cdot 6 = 36$ products.
We can do it, just not tonight.


\begin{theo}[Universal property of quotients]
Let $H \triangleleft G$ be a normal subgroup and $f : G \to N$ a morphism.
If $H \subset \ke f$ then there exists a unique morphism $g : G/H \to N$ such
that the diagram
$$
\begin{tikzcd}
G \ar[d,"\pi"] \ar[dr,"f"] & 
\\
G/H \ar[r,"g"] & N
\end{tikzcd}
$$
commutes.
\end{theo}

\begin{proof}
We propose to define $g : G/H \to N$ by $g(x) = f(y)$, where $y$ is an element
of $G$ such that $\pi(y) = x$.
If $h \in H$ then $\pi(yh) = x$ as well and $f(yh) = f(y) f(h) = f(y)$ by
hypothesis, so this defines a map.

If $x_1, x_2$ are in $G/H$ and $y_1, y_2$ map to them under $\pi$, then
$$
g(x_1 x_2)
= f(y_1 y_2)
= f(y_1) f(y_2)
= g(x_2) g(x_2)
$$
so $g$ is a morphism.

Suppose now that $g' : G/H \to N$ is another morphism that fits into this diagram.
Let $x \in G/H$ and let $y$ map to it.
Then
$
g'(x) = f(y) = g(x)
$
so $g' = g$.
\end{proof}





\paragraph{Kernels and normal subgroups}

There is a close relationship between kernels of morphisms and normal subgroups.
So close that they are actually the same:

\begin{prop}
Let $f : G \to H$ be a morphism.
Then $\ke f \triangleleft G$.
\end{prop}

\begin{proof}
Let $h \in \ke f$ and $g \in G$. Then
\begin{equation*}
f(g^{-1} h g)
= f(g)^{-1} f(h) f(g)
= f(g)^{-1} f(g)
= e.
\qedhere
\end{equation*}
\end{proof}

Conversely, if $H \triangleleft G$ then $H = \ke \pi$, where $\pi : G \to G/H$
is the quotient morphism.


\begin{theo}[First isomorphism theorem]
Let $f : G \to H$ be a morphism.
Then $\im f \cong G / \ke f$.
\end{theo}

\begin{proof}
We have a quotient morphism $\pi : G \to G / \ke f$ and a surjective morphism
$f : G \to \im f$ that is constant on the fibers of $\pi$.
The universal property of quotients gives a morphism $g$
$$
\begin{tikzcd}
G \ar[d,"\pi"] \ar[dr,"f"] & 
\\
G/\ke f \ar[r,"g"] & \im f
\end{tikzcd}
$$
such that the above diagram commutes.
As $f$ is surjective then so is $g$.
Suppose that $g(x) = e$ and let $y \in G$ be such that $\pi(y) = x$.
Then $f(y) = e$, so $y \in \ke f$, and $x = e$, so $g$ is injective.
\end{proof}




\paragraph{More categories}

Groups with their morphisms form a category $\grp$.
It contains an initial and terminal object; the group $1$.
Products exist in this category:

\begin{prop}
Let $I$ be an index set and $(G_i)_{i \in I}$ a family of groups indexed by $I$.
There exists a group $\prod_{i \in I}G_i$ that is the category theoretic
product of the $(G_i)$.
\end{prop}

\begin{proof}
We define $\prod_{i \in I} G_i$ as a set in the usual way and define its binary
operation component-wise.
The identity element is $\prod_{i \in I} e_{G_i}$, and the inverse of an
element $\prod g_i$ is $\prod g_i^{-1}$.
There are clearly morphisms $\pi_j : \prod G_i \to G_j$ for every $j \in I$.

Now let $H$ be a group and $f_i : H \to G_i$ a family of morphisms indexed by $I$.
We define a morphism $f : H \to \prod G_i$ by $f(h) = \prod f_i(h)$.
This makes the necessary diagram commute.
If $f' : H \to \prod G_i$ is another such morphism we have $\pi_j f' = \pi_j f$
for all $j$, so $f' = f$ because both are based on sets.
\end{proof}

There is a forgetful functor $\grp \to \set$ that sends a group to its
underlying set.
It has an adjoint $\set \to \grp$ that sends a set to the \emph{free group}
generated by the set:

\begin{defi}
Let $X$ be a set. The \emph{free group} generated by $X$ is the group
$\prod_{x \in X} \ZZ$.
\end{defi}



\section{Permutations}


Let's focus on the permutation groups of finite sets.
The group $S_n$ is defined as the group of permutations of a set of $n$ elements.
It doesn't matter which set:

\begin{prop}
Let $X$ and $Y$ be sets of $n$ elements and $f : X \to Y$ a bijection.
Then $\aut X \cong \aut Y$.
\end{prop}

\begin{proof}
The pullback $f^{-1} : \aut Y \to \aut X$ is a group isomorphism.
\end{proof}

We very often consider the set $\{1, \ldots, n\}$ and its permutations, as it
is both right there and useful in applications.


\begin{prop}
$$
|S_n| = n!
$$
\end{prop}


\begin{proof}
Clearly $|S_1| = 1$.
For $n > 1$ we consider the subgroup $H \subset S_n$ of permutations that fix
the element $n$, which is isomorphic to $S_{n-1}$.
By considering the elements that swap $n$ and $k$ for $k = 1, \ldots, n$ we see
that $H$ has at least $n$ left cosets.
Now let $g$ be such that $g H \not= H$, which entails $n \not= g(n)$.
Combining $g$ with the permutation that swaps $g(n)$ and $n$ we then obtain a permutation that fixes $n$, or an element of $H$.
Thus any left coset of $H$ is conjugate to one of the above and we have exactly
$n$ cosets.
Therefore
$|S_n| = n \, |S_{n-1}|$ and we conclude by induction.
\end{proof}



\begin{theo}
If $n > 1$ there exists a nontrivial morphism $\sgn : S_n \to \ZZ^\times$.
\end{theo}

\begin{proof}
Let $\{1, \ldots, n\}$ be a set of $n$ elements that $S_n$ acts on
and consider the $\ZZ$-module $\ZZ^{n}$.
Each element of $S_n$ defines an automorphism of this module by permuting the
basis elements.
We thus get an embedding $S_n \to \GL_n(\ZZ)$, and the determinant gives a
group morphism $\GL_n(\ZZ) \to \ZZ^\times$.

Consider now the element that swaps $1$ and $2$.
Its image under this embedding is the automorphism given by the matrix
$$
\begin{pmatrix}
0 & 1
\\
1 & 0
\end{pmatrix}
\oplus I_{n-2},
$$
whose determinant is $-1$, so the morphism is nontrivial.
\end{proof}


\begin{defi}
The \emph{even} permutations on $n$ are $A_n := \ke(\sgn : S_n \to \ZZ^\times)$.
The \emph{odd} permutations on $n$ are the nontrivial coset of $A_n$.
\end{defi}



The even permutations are a normal subgroup of $S_n$.
They're often the only normal subgroup.
That makes Caley's theorem, that every group embeds in a permutation group,
perhaps less interesting since it often can't embed as a normal subgroup just
for degree reasons.



\section{Classification of commutative groups}

\section{Sylow theorems}



\end{document}
