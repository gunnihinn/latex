%\documentclass[10pt,a4paper]{smfart}
\documentclass[10pt,a4paper]{amsart}

\usepackage{lmodern}
\linespread{1.1}
\usepackage[utf8]{inputenc}
\usepackage[T1]{fontenc}

\usepackage{fancyref}
\usepackage[colorlinks=true]{hyperref}

\usepackage{amsmath}
\usepackage{amssymb}
\usepackage{amsthm}
\usepackage{tikz-cd}

\newtheorem{theo}{Theorem}
\newtheorem{prop}[theo]{Proposition}
\newtheorem{lemm}[theo]{Lemma}
\newtheorem{coro}[theo]{Corollary}
\theoremstyle{definition}
\newtheorem{exam}[theo]{Example}

\def\ov#1{\overline{#1}}
\def\qandq{\quad\text{and}\quad}
\newcommand{\kk}[1]{\mathbb{#1}}
\newcommand{\cc}[1]{\mathcal{#1}}

\DeclareMathOperator{\Sym}{Sym}
\DeclareMathOperator{\im}{Im}
\DeclareMathOperator{\id}{Id}
\DeclareMathOperator{\tr}{tr}
\DeclareMathOperator{\Vol}{Vol}
\DeclareMathOperator{\End}{End}

\author{Gunnar \TH\'or Magn\'usson}
\address{Hafnarfjörður, Iceland}
\email{gunnar@magnusson.io}
\date{\today}
\title{Examples of K\"ahler curvature tensors}

\begin{document}

\begin{abstract}
We work out examples that show which folklore implications between
positivity of the various curvature tensors of K\"ahler metrics hold.
\end{abstract}

\maketitle

\section{Introduction}

It is well-known that the curvature properties of K\"ahler metrics ``flow
downward''.
To give meaning to this statement,
let us say that a tensor $\alpha$ \emph{dominates} a tensor $\beta$
if any of the following implications hold:
\begin{align*}
\alpha > 0 &\implies \beta > 0,
&
\alpha &\geq 0 \implies \beta \geq 0,
\\
\alpha < 0 &\implies \beta < 0,
&
\alpha &\leq 0 \implies \beta \leq 0.
\end{align*}
Strictly speaking we need to specify what it means for a tensor to be
``positive'' or ``negative'' for this to make sense.
This is fairly clear for the tensors we'll discuss here, except for the
K\"ahler curvature tensor itself which we mean to be Griffiths positive (or
negative).

This is an equivalence relation on tensors (where it's usually more useful to
restrict to only one of the inequalities holding).
If $R$ is the curvature tensor of a K\"ahler metric, its various derived
curvature tensors are the holomorphic sectional curvature $H$, the Ricci
tensor $r$ and the scalar curvature~$s$.
Then
$$
\begin{tikzcd}
& R \ar[dl] \ar[dr] &
\\
H \ar[dr] & & r \ar[dl]
\\
& s &
\end{tikzcd}
$$
where an arrow $\alpha \to \beta$ means that $\alpha$ dominates $\beta$.

The witches of the field know that neither $H$ nor $r$ dominates the other.
The best evidence for this seems to be indirect.
We for example know that Hirzebruch surfaces (surfaces that are projective
bundles over $\kk P^1$) carry metrics with positive holomorphic sectional
curvature, but carry no metric of positive Ricci curvature as their anticanonical
bundle is not ample.
There are some suggestive results in the negatively curved case, though,
where a compact K\"ahler manifold that carries a metric of negative holomorphic
sectional curvature also carries a possibly different metric of negative Ricci
curvature.
Could it be the same metric?
We don't know.
Wouldn't it be nice if some easy calculations said it was?
Sure would.

It is also folklore that all of the reverse relations in the diagram fail.
Some do not hold by general considerations; for example Hirzebruch surfaces
again carry a metric of positive holomorphic sectional curvature but not one of
positive Griffiths curvature, for otherwise they would be $\kk P^2$ by the
Yau's Frankel theorem and, lo, they are not.

We feel like the time has come to roll up our sleeves and work out some
examples that show explicitly what doesn't hold here.
That is what we do in this note.



\section{Algebraic curvature tensors}



\subsection*{K\"ahler curvature tensors come from symmetric tensors}

Let $V$ be a complex vector space of dimension $n$, which we think of as the
tangent space of a complex manifold at a given point.
The curvature tensor $R$ of a Hermitian metric on the manifold identifies with
a Hermitian form $q$ on $V \otimes V$, defined by
$$
R(x, \ov y, z, \ov w)
= q(x \otimes z, \ov{y \otimes w}).
$$
If the metric is K\"ahler we get an additional symmetry
$R(x, \ov y, z, \ov w) = R(z, \ov y, x, \ov w)$
(and the ones induced by conjugating).

We write $B(V)$ for the real vector space of Hermitian forms on $V$.
The curvature tensor of a Hermitian metric is then just a member of $B(V
\otimes V)$. We call such an element an \emph{algebraic Hermitian curvature
tensor}, and one that satisfies the additional symmetry of a K\"ahler curvature
tensor an \emph{algebraic K\"ahler curvature tensor}.

The decomposition $V \otimes V = \bigwedge^2 V \oplus \Sym^2 V$ is standard.
It implies that a Hermitian form on $V \otimes V$ decomposes into components
$$
q = \begin{pmatrix}
q_{\wedge^2 V} & q_{(\Sym^2V, \wedge^2 V)}
\\
q_{(\wedge^2 V, \Sym^2V)} & q_{\Sym^2 V},
\end{pmatrix}
$$
where $\ov{q}_{(\wedge^2 V, \Sym^2V)}^t = q_{(\Sym^2V, \wedge^2 V)}$.
Denote the symmetrization map by
$$
\pi : V \otimes V \to \Sym^2 V,
\quad
x \otimes y \mapsto \tfrac 12 (x \otimes y + y \otimes x).
$$
It is a surjective linear morphism that realizes the space of symmetric tensors
as a subspace of $V \otimes V$.
The usual definition of that space is as a quotient of $V \otimes V$ by the
ideal generated by $x \otimes y - y \otimes x$.
If we need to distinguish between the two we'll write $S^2V$ for the quotient
space.
As the field $\kk C$ has characteristic zero these spaces are isomorphic,
so the difference between them isn't very important to us (for now).


\begin{prop}
A tensor $R \in B(V \otimes V)$ is an algebraic K\"ahler curvature tensor
if and only if there exists an element $\hat R \in B(\Sym^2 V)$ such that $R =
\pi^* \hat R$.
\end{prop}

\begin{proof}
Suppose $R \in B(V \otimes V)$ is an algebraic K\"ahler curvature tensor.
We define
$$
\hat R(x \odot z, \ov{y \odot w})
= R(x, \ov y, z, \ov w),
$$
which is well-defined because $R$ is K\"ahler, and we have $R = \pi^* \hat R$.

Conversely, let $\hat R \in B(\Sym^2 V)$, and define $R = \pi^* \hat R$.
Then
$$
R(x, \ov y, z, \ov w)
= \hat R(x \odot z, \ov{y \odot w})
= \hat R(z \odot x, \ov{w \odot y})
= R(z, \ov y, x, \ov w)
$$
is an algebraic K\"ahler curvature tensor.
\end{proof}

As an aside, this proposition explains why we only ever talk about Griffits
positivity of K\"ahler metrics.
A Hermitian metric is Nakano positive if its curvature tensor is
positive-definite as a Hermitian form on $V \otimes V$.
However, a Hermitian form that is the pullback by a morphism with nontrivial
kernel is never positive-definite.
Therefore a K\"ahler metric is never Nakano positive when $\pi$ is not
injective, which happens when
$n^2 = \dim V \otimes V > \dim \Sym^2 V = \binom {n+1}2$,
that is, when $n > 1$.

There should then be a notion of positivity for K\"ahler curvature tensors that
interpolates between Griffiths positivity and Nakano positivity and is perhaps
more geometrically motivated than Griffiths positivity, where we would say that
such a tensor is positive if its Hermitian form on $\Sym^2 V$ is
positive-definite.
In light of Siu and Yau's Frankel theorem any compact K\"ahler manifold that
admits a metric so positive is just projective space, so we don't need to think
about it anymore.



\subsection*{Holomorphic sectional curvature determines the whole tensor}

It is well-known that the holomorphic sectional curvature of a K\"ahler metric
determines the whole curvature tensor and that its average over the unit sphere
is a multiple of the scalar curvature.
The proofs of this in the litarature are somewhat mysterious;
the first proceeds by algebraic manipulations that are not explained or
motivated and only seem justified post-hoc by virtue of giving the right answer;
the second is by a brute force integration that seems to imply the result is
due to the seeming coincidence that integrating $|z_j^4|$ and $|z_j|^2 |z_k|^2$
over the sphere gives the same result up to a factor of 2.
We intend to show that both results flow from the same representation-theoretic
source.


Our starting point there involves vector-valued integration, so let's recall
some basic facts.
Let $V$ and $W$ be finite-dimensional complex vector spaces equipped with their
Lebesgue measures.
If $f : V \to W$ is a continuous function and $X \subset V$ a measurable subset
we define $\int_X f(v) d\mu(v) \in W$ to be the vector we get after picking
bases and integrating coordinate by coordinate.
If $L : W \to Z$ is a linear map then $L \int_X f(v) d\mu(v) = \int_X Lf(v)
d\mu(v)$, because that holds at every step of the definition of the integral of
a measurable function.
This implies the integral is well-defined, as it behaves correctly with respect
to a change of bases.
It also implies that if $f : V \to V$ is continuous then the trace commutes
with the integral.

Let's suppose $V$ is equipped with a Hermitian inner product $h$
and let's write $d\mu$ for the induced volume form on the unit sphere $S(V)$ in
$V$. For $v \in V$ we define $v \otimes v^*$ to be the linear map
$h^*(\ov v) v = x \mapsto h(x, \ov v) v$.
Note that if $f \in \End V$ then $f \circ v \otimes v^* = h^*(\ov v) f(v)$
and thus $\tr(f \circ v \otimes v^*) = h(fv, \ov v)$.

\begin{prop}
\label{prop:int-id}
Denote by $\Pi_d : V^{\otimes d} \to V^{\otimes d}$ the projection onto $\Sym^d V$.
Then
$$
\frac{1}{\Vol S(V)}
\int_{S(V)} (v \otimes v^*)^{d} d\mu(v)
= \frac{1}{\binom{n+d-1}{d}} \Pi_d.
$$
\end{prop}

\begin{proof}
Let's denote the map defined by the integral by $L$. If $g \in \End V$ is
unitary with respect to $h$ we have $h(g x, \ov v) = h(x, \ov{g^{-1} v})$ and
$|\!\det g| = 1$ so the change of basis formula implies that $L g^d = g^d L$.
The map $L$ is then an interleaving operator of the representation $V^{\otimes d}$
of the unitary group.

The integral is thus a sum of multiples of the projections onto the irreducible
factors of $V^{\otimes d}$.
Note however that the integral takes values in $\Sym^d V$ because the integrand
is invariant under $S_d$, so $L = \lambda \Pi_d$ for some scalar $\lambda$.
For a unit vector $v$ we have $\tr v \otimes v^* = |v|^2 = 1$.
Taking the trace we then find that $1 = \lambda \binom{n+d-1}{d}$,
which implies the result.
\end{proof}


\begin{coro}
\begin{align*}
\frac{1}{\Vol S(V)} \int_{S(V)} H(v) \, d\mu(v)
&= \frac{1}{\binom{n+1}{2}} s.
\\
\frac{1}{\Vol S(V)}
\int_{S(V)} H(v)^2 \, d\mu(v)
&=
\frac{6}{\binom{n+3}4} (|R|^2 + 2|r|^2 + s^2).
\end{align*}
\end{coro}


\begin{proof}
If $f \in \End \Sym^2 V$ then taking the trace of the equation in
Proposition~\ref{prop:int-id} for $d = 2$ gives
$$
\frac{1}{\Vol S(V)}
\int_{S(V)} h^{\odot 2}(f(v \odot v), \ov{v \odot v}) \, d\mu(v)
= \frac{1}{\binom{n+1}{2}} \tr f.
$$
If $q \in B(\Sym^2)$ is the Hermitian form defined by the curvature tensor $R$,
then applying this to $f = (h^{\odot 2})^{-1} b$ gives the first equation.

For the second equation,
we consider $f \otimes f \in \End V^{\otimes 4}$ and get that
$$
\frac{1}{\Vol S(V)}
\int_{S(V)} h(f(v \otimes v), \ov{v \otimes v})^2 \, d\mu(v)
= \frac{1}{\binom{n+3}{4}}
\tr (f \circ \Pi_4).
$$
We know that $\Pi_d = \sum_{\sigma \in S_d} W_\sigma$, where $S_d$ is the
symmetric group on $d$ letters and $W_\sigma(v_1 \otimes \cdots \otimes v_d) =
v_{\sigma(1)} \otimes \cdots \otimes v_{\sigma(d)}$.
We note that $W_{\sigma\rho} = W_\sigma W_\rho$.
The group $S_4$ is generated by
$\sigma = (1 \  2)$ and
$\rho = (1 \  2 \  3 \  4)$, explicitly as
\begin{align*}
(1\;2\;3\;4) &= \id
		&
(2\;1\;3\;4) &= \sigma
		&
(2\;3\;4\;1) &= \rho
		&
(1\;3\;4\;2) &= \rho\sigma
\\
(3\;2\;4\;1) &= \sigma\rho
		&
(3\;4\;1\;2) &= \rho \rho
		&
(3\;1\;4\;2) &= \sigma \rho \sigma
		&
(3\;4\;2\;1) &= \rho \rho\sigma
\\
(2\;4\;1\;3) &= \rho \sigma \rho
		&
(4\;3\;1\;2) &= \sigma\rho \rho
		&
(4\;1\;2\;3) &= \rho \rho \rho
		&
(1\;4\;2\;3) &=  \rho\sigma \rho \sigma
\\
(4\;3\;2\;1) &= \sigma \rho \rho \sigma
		&
(4\;2\;1\;3) &= \rho \rho \rho\sigma
		&
(4\;1\;3\;2) &= \rho \rho \sigma \rho
		&
(3\;1\;2\;4) &= \rho \sigma \rho \rho
\\
(4\;2\;3\;1) &= \rho \rho \sigma \rho\sigma
		&
(3\;2\;1\;4) &= \rho \sigma \rho \rho \sigma
		&
(1\;4\;3\;2) &= \sigma \rho \rho \sigma \rho
		&
(1\;3\;2\;4) &= \rho \rho \rho \sigma \rho
\\
(1\;2\;4\;3) &= \rho \rho \sigma \rho \rho
		&
(2\;4\;3\;1) &= \sigma \rho \rho \sigma \rho \sigma
		&
(2\;3\;1\;4) &= \rho \rho \rho \sigma \rho \sigma
		&
(2\;1\;4\;3) &= \rho \rho \sigma \rho \rho \sigma
\end{align*}
as the reader can have fun verifying (or write a program to generate these, as
I did). As $f$ is symmetric we have $f \otimes f \circ W_{\sigma} = f \otimes f$
so we only need to figure out what the traces of $f \otimes f \circ W_{\rho^k}$
are for $k = 1, 2, 3$.

To do this we pick a basis $(e_j)$ of $V$ and denote by $(e_{jklm} = e_j
\otimes e_k \otimes e_l \otimes e_m)$ the induced basis of $V^{\otimes 4}$.
Denote by $f_{jk,lm}$ the coefficients of the matrix of $f$ in the
corresponding basis $(e_j \otimes e_k)$ of $V^{\otimes 2}$, so the matrix of
$f \otimes f$ has the coefficients $f_{jk,lm} f_{pq,rs}$.
We note that $f^2$ has the matrix $f_{jk,lm} f_{lm, rs}$ so $\tr f^2 = f_{jk,lm} f_{lm,jk}$.
Using Einstein notation we then have
\begin{align*}
\tr(f \otimes f \circ W_{\rho})
&= f_{kp,jk} f_{qj,pq}
= f_{jk,kp} f_{pq,qj}
= |\! \tr_{23} f|^2,
\\
\tr(f \otimes f \circ W_{\rho^2})
&= f_{pq,jk} f_{jk,pq} = \tr(f^2),
\\
\tr(f \otimes f \circ W_{\rho^3})
&= f_{qj,jk} f_{kp,pq}
= |\! \tr_{13} f|^2,
\end{align*}
where $\tr_{jk} : \End V^{\otimes 4} \to \End V^{\otimes 2}$ is the partial
trace in $j$ and $k$ and the second equality in the first line holds because
$f^t = f$.
As $f$ is symmetric these are all equal to the Ricci form of~$R$.%

Looking at the generator table above, we see that after discarding $\sigma$
(which $f \otimes f$ is invariant under, there are 6 entries each with each
power of $\rho$, and gathering terms we arrive at the announced equation.
\end{proof}


I haven't seen the second equation before, but it explains why the holomorphic
sectional curvature determines the entire curvature tensor; if $H = 0$ we must
have $|R|^2 = 0$ as well.
It also shows that knowing a bound on $H$ gives a bound on $R$, which is
usually observed as a corollary of a polarization identity.
The polarization identity shows that if $|H|^2 \leq C$ then $|R|^2 \leq
24 C$ because it involves applying the triangle inequality to a sum of 24 factors.
We get here that $|R|^2 \leq 6\binom{n+3}{4} C$, which is worse!


\subsection*{Dominance}

The following statement is classical but quickly proved.

\begin{prop}
If $R$ is an algebraic K\"ahler curvature tensor, then
$$
\begin{tikzcd}
& R \ar[dl] \ar[dr] &
\\
H \ar[dr] & & r \ar[dl]
\\
& s &
\end{tikzcd}
$$
\end{prop}

\begin{proof}
We prove the statements under the assumption that the tensors are positive.
The proofs for semipositivity and (semi)negativity are the same.

If $R$ is Griffiths positive, then by definition $R(x, \ov x, y, \ov y) > 0$
for all nonzero $x, y$. Then $R(x, \ov x, x, \ov x) > 0$ so $H(x) > 0$.
The Ricci tensor is the trace
$$
r(x, \ov y) = \sum_{j = 1}^n R(x, \ov y, e_j, \ov e_j)
$$
in an orthonormal basis $(e_j)$, so it is positive-definite.
The scalar curvature is the trace
$$
s = \sum_{j=1}^n r(e_j, \ov e_j)
$$
in an orthonormal basis, so $s > 0$ if $r$ is positive-definite.
As we saw before we have
\[
\int_{S(V)} H(v) \, d\mu(v)
= \frac{\Vol(S(V))}{\binom{n+1}{2}} s
\]
so $s > 0$ if $H > 0$.
\end{proof}



\section{Examples of non-dominance in dimension two}

Let's define
\begin{align*}
\cc H(V)
&= \{ \text{germs of smooth Hermitian metrics on $V$ at $0$} \},
\\
\cc K(V)
&= \{ \text{germs of smooth K\"ahler metrics on $V$ at $0$} \}.
\end{align*}
We just write $\cc H$ if $V$ is clear from context and will usually mean $V =
\kk C^n$. Note that $\cc K$ and $\cc H$ are only convex open cones in real
vector spaces and not full subspaces.
Evaluating the curvature tensor of a metric at $0$ defines nonlinear maps and a
commutative diagram
$$
\begin{tikzcd}
\cc K \ar[d] \ar[r,"\cc R"] & B(\operatorname{Sym}^2 \kk C^n)\ar[d,"\pi^*"]
\\
\cc H \ar[r, "\cc R"] & B(\kk C^n \otimes \kk C^n).
\end{tikzcd}
$$
Both vertical arrows are of course injective.
The horizonal arrows are neither injective nor surjective in general.


Let $H$ be the matrix of a Hermitian metric $h$ in a neighborhood around $0$ in
$V$. The Chern connection of $h$ is given by $D^{1,0} = H^{-1}\partial H$ and its
curvature form is
$$
F = \tfrac i2 \bar\partial(H^{-1}\partial H)
= - \tfrac i2 H^{-1} \bar\partial H \wedge H^{-1} \partial H
- \tfrac i2 H^{-1} \partial \bar\partial H.
$$
The curvature tensor is then $R(x,\ov y, z, \ov w) = h(F(x, \ov y) z, \ov w)$.

We can pick coordinates such that $H = I_n$ at $0$.
If $h$ is K\"ahler we can further get $\partial H = 0$ at $0$.
We will arrange this in our examples.
We now restrict to $V = \kk C^2$.

\begin{prop}
The image of $\cc K(\kk C^2)$ in $B(\Sym^2 \kk C^2)$ is
$$
\left\{
\begin{pmatrix}
a & b & c
\\
\ov b & c & e
\\
c & \ov e & f
\end{pmatrix}
\biggm|
\text{$a, c, f \in \kk R$, $b, e \in \kk C$}
\right\}.
$$
\end{prop}


\begin{proof}
If $h$ is a Hermitian metric on a neighborhood of $0$ in $\kk C^2$
we can write it in matrix form as
$$
H = \begin{pmatrix}
a & c
\\
\ov c & b
\end{pmatrix}.
$$
After a change of coordinates, we can assume that $H(0) = I_2$.
Assuming $h$ is K\"ahler we can also arrange that $\partial H(0) = 0$.
Then the matrix of the curvature tensor in the basis
\def\tg#1#2{\frac{\partial}{\partial #1} \otimes \frac{\partial}{\partial #2}}
$(\tg zz, \tg zw, \tg wz, \tg ww)$
is
$$
R = \begin{pmatrix}
a_{z \ov z} & \ov c_{z \ov z} & a_{z \ov w} & \ov c_{z \ov w}
\\
c_{z \ov z} & b_{z \ov z} & c_{z \ov w} & b_{z \ov w}
\\
a_{w \ov z} & \ov c_{w \ov z} & a_{w \ov w} & \ov c_{w \ov w}
\\
c_{w \ov z} & b_{w \ov z} & c_{w \ov w} & b_{w \ov w}
\end{pmatrix}.
$$
We've written, for example, $a_{z \ov w}$ instead of $\frac{\partial^2 a}{\partial z \partial \ov w}$.
The metric $h$ is K\"ahler, so we have
$$
a_w = c_{z},
\quad
a_{\ov w} = \ov c_{\ov z},
\quad
b_z = \ov c_w,
\quad
b_{\ov z} = c_{\ov w},
$$
and thus
$$
c_{z \ov z} = a_{w \ov z},
\quad
\ov c_{w \ov w} = b_{z \ov w},
\quad
c_{z \ov w} = a_{w \ov w},
\quad
\ov c_{w \ov z} = b_{z \ov z}
$$
Then we get
$$
R = \begin{pmatrix}
a_{z \ov z} & a_{z \ov w} & a_{z \ov w} & a_{w \ov w}
\\
a_{w \ov z} & a_{w \ov w} & a_{w \ov w} & b_{z \ov w}
\\
a_{w \ov z} & a_{w \ov w} & a_{w \ov w} & b_{z \ov w}
\\
a_{w \ov w} & b_{w \ov z} & b_{w \ov z} & b_{w \ov w}
\end{pmatrix},
$$
where some simplifications happen because $a$ is real so $a_{w \ov z} =
a_{z \ov w}$ and we can ping-pong to $b_{z \ov z} = a_{w \ov w}$.
This is the pullback of the Hermitian form
$$
R = \begin{pmatrix}
a_{z \ov z} & a_{z \ov w} & a_{w \ov w}
\\
a_{w \ov z} & a_{w \ov w} & b_{z \ov w}
\\
a_{w \ov w} & b_{w \ov z} & b_{w \ov w}
\end{pmatrix},
$$
on $\Sym^2 \kk C^2$, as expected.
Note that $a_{w \ov w}$ is real, so not every Hermitian form on
$\Sym^2 \kk C^2$ comes from a curvature tensor of a K\"ahler metric.
(Note that $a_{w \ov w} = b_{z \ov z}$ so this matrix is more symmetric in $a$
and $b$ than it appears.)


We will conclude the proof by taking the above Hermitian matrix as given and
constructing a germ of a K\"ahler metric whose curvature tensor agrees with it
at $0$.
Suppose we let
$$
f(z,w)
= a_{z \ov z} z \bar z
+ a_{z \ov w} z \ov w
+ a_{w \ov z} w \ov z
+ a_{w \ov w} w \bar w.
$$
Then $f(0) = 0$ and $df(0) = 0$ and
$$
\operatorname{Hess} f(0)
= \begin{pmatrix}
a_{z \ov z} & a_{z \ov w}
\\
a_{w \ov z} & a_{w \ov w}
\end{pmatrix}.
$$
We also let
$$
g(z,w)
= a_{w \ov w} z \bar z
+ b_{z \ov w} z \ov w
+ b_{w \ov z} w \ov z
+ b_{w \ov w} w \bar w
$$
and
$$
h(z,w)
= h_1 z \bar z
+ h_2 z \ov w
+ h_3 w \ov z
+ h_4 w \bar w.
$$
We want
$$
a_{w \ov z} \ov z + a_{w \ov w} \bar w
= f_w = h_z
= h_1 \ov z + h_2 \ov w,
$$
so we set $h_1 = a_{w \ov z}$ and $h_2 = a_{w \ov w}$.
We also want
$$
a_{w \ov w} \ov z + b_{z \ov w} \ov w
= g_z = \ov h_w
= \ov h_2 \bar z + \ov h_4 \ov w
$$
so $h_4 = b_{z \ov w}$.
There's no constraint on $h_3$ so we just take $h_3 = \ov h_4$.
The germ of the K\"ahler metric
$$
h(z,w)
= \begin{pmatrix}
1 + f(z,w) & h(z,w)
\\
\ov{h(z,w)} & 1 + g(z,w)
\end{pmatrix}
$$
then has the curvature tensor $R$ above.
\end{proof}

Let's look at some special cases. If the tensor looks like
$$
R =
\begin{pmatrix}
a & 0 & c
\\
0 & c & 0
\\
c & 0 & f
\end{pmatrix}
$$
then
\begin{align*}
R(x, \ov x, x, \ov x)
&= (a x_1^2 + c x_2^2) \ov x_1^2
+ 4 c |x_1|^2 |x_2|^2
+ (c x_1^2 + f x_2^2) \ov x_2^2
\\
&= a |x_1|^4 + f |x_2|^4
+ 2 c \bigl(2 |x_1|^2 |x_2|^2 + \Re(x_1^2 \ov x_2^2) \bigr)
\\
r(x, \ov y)
&=
\begin{pmatrix}\ov y_1 & \ov y_2 \end{pmatrix}
\begin{pmatrix}
a + c & 0 \\ 0 & c + f
\end{pmatrix}
\begin{pmatrix} x_1 \\ x_2 \end{pmatrix},
\\
s &= a + 2c + f,
\end{align*}
and the sign of $R(x, \ov x, x, \ov x)$ is the same as the sign of the
holomorphic sectional curvature $H(x)$.

For complex numbers we have $|\Re z| \leq |z|$, so we have
\begin{align*}
a |x_1|^4 + f |x_2|^4 + 2c |x_1|^2 |x_2|^2
&\leq R(x, \ov x, x, \ov x)
\\
&\leq a |x_1|^4 + f |x_2|^4 + 6c |x_1|^2 |x_2|^2.
\end{align*}
If further $x = x_1 = x_2$ then
$$
R(x, \ov x, x, \ov x)
= (a + 6c + f) |x|^4.
$$

\begin{exam}[$s \not\to H$]
We can find $a, c, f$ such that $a + 2c + f > 0$ but $a + 6c + f < 0$,
(like $a = f = 4$, $c = -3$), so the scalar curvature does not dominate
the holomorphic sectional curvature.
\end{exam}

\begin{exam}[$s \not\to r$]
Take $a = 4$, $f = -1$ and $c = -1$. Then $a + 2c + f = 1 > 0$ but
$$
r(x, \ov y) =
\begin{pmatrix}\ov y_1 & \ov y_2 \end{pmatrix}
\begin{pmatrix}
3 & 0 \\ 0 & -2
\end{pmatrix}
\begin{pmatrix} x_1 \\ x_2 \end{pmatrix},
$$
is neither positive nor negative semidefinite, so the scalar curvature
does not dominate the Ricci curvature.
\end{exam}

\begin{exam}[$r \not\to H$]
Let $a = f = 2$ and $c = -1$. Then
$$
r(x, \ov y) =
\begin{pmatrix}\ov y_1 & \ov y_2 \end{pmatrix}
\begin{pmatrix}
1 & 0 \\ 0 & 1
\end{pmatrix}
\begin{pmatrix} x_1 \\ x_2 \end{pmatrix}
$$
is positive-definite, but at $x = (x_1, x_1)$ we have
$$
R(x, \ov x, x, \ov x) = -4 |x_1|^2
$$
so the Ricci curvature does not dominate the holomorphic sectional curvature.
Note that the Ricci form is a multiple of the metric, so the metric is
K\"ahler--Einstein at the origin.
Any statement like ``this metric is K\"ahler--Einstein so its holomorphic
sectional curvature has a sign'' thus needs more information about the metric
or the manifold to have a chance of being true.
\end{exam}

\begin{exam}[$H \not\to r$]
We have
$$
R(x, \ov x, x, \ov x)
\geq
a |x_1|^2 + 2c |x_1|^2 |x_2|^2 + f |x_2|^4
$$
and the right-hand side is the quadratic form defined by
$$
Q = \begin{pmatrix}
a & c \\ c & f
\end{pmatrix}
$$
This form is positive-definite if and only if
$$
0 < \tr Q = a + f
\qandq
0 < \det Q = af - c^2.
$$
Suppose $c = -f - \epsilon$ with $\epsilon > 0$.
Then
$$
af - c^2
= af - f^2 - 2f \epsilon - \epsilon^2
= f(a - f - 2\epsilon) - \epsilon^2.
$$
If $a > f$, this is positive for small enough $\epsilon$ (like $a = 10$, $f =
1$, $\epsilon = 1$).
But then the Ricci form is
$$
r(x, \ov y) =
\begin{pmatrix}\ov y_1 & \ov y_2 \end{pmatrix}
\begin{pmatrix}
a & 0 \\ 0 & -\epsilon
\end{pmatrix}
\begin{pmatrix} x_1 \\ x_2 \end{pmatrix}
$$
which is not positive-definite,
so the holomorphic sectional curvature does not dominate the Ricci
curvature.
\end{exam}

\begin{exam}[$r \not\to R$]
We have
$$
\det R = acf - c^3 = c(af - c^2) = c \det r.
$$
It is then enough to pick a metric with $r$ positive-definite but $c < 0$
(like $a = f = 2$, $c = -1$)
to get a curvature tensor whose Ricci tensor is positive but is not itself
Griffiths positive, because it will satisfy $R(z, \ov w, z, \ov w) < 0$.
\end{exam}

\begin{exam}[$H \not\to R$]
Let $a = f = 2$ and $c = -1$.
Then
$$
R(x, \ov x, x, \ov x)
\geq |x|^4 + (|x_1|^2 - |x_2|^2)^2 > 0
$$
so the holomorphic sectional curvature is positive, but
for $x = (x,0)$ and $y = (0,y)$ we have
$$
q(x \odot y)
= R(x, \ov x, y, \ov y)
= - (|x|^2 + |y|^2 + x \ov y + y \ov x)
\leq 0
$$
and the inequality can be strict, so the holomorphic sectional curvature does
not dominate the Griffiths curvature.
\end{exam}


\bibliographystyle{plain}
\bibliography{main}


\end{document}
