\documentclass[10pt,a4paper]{amsart}

\usepackage{lmodern}
\linespread{1.1}
\usepackage[utf8]{inputenc}
\usepackage[T1]{fontenc}

\usepackage{fancyref}
\usepackage[colorlinks=true]{hyperref}
\usepackage{xcolor}

\usepackage{amsmath}
\usepackage{amssymb}
\usepackage{amsthm}
\usepackage{tikz-cd}

\newtheorem{theo}{Theorem}
\newtheorem{prop}[theo]{Proposition}
\newtheorem{lemm}[theo]{Lemma}
\newtheorem{coro}[theo]{Corollary}
\theoremstyle{definition}
\newtheorem{exam}[theo]{Example}

\def\ov#1{\overline{#1}}
\def\qandq{\quad\text{and}\quad}
\newcommand{\kk}[1]{\mathbf{#1}}
\newcommand{\cc}[1]{\mathcal{#1}}
\def\sq{\sqrt{2}}

\DeclareMathOperator{\re}{Re}
\DeclareMathOperator{\Sym}{Sym}
\DeclareMathOperator{\im}{Im}
\DeclareMathOperator{\id}{Id}
\DeclareMathOperator{\tr}{tr}
\DeclareMathOperator{\Vol}{Vol}
\DeclareMathOperator{\End}{End}

\allowdisplaybreaks

\author{Gunnar \TH\'or Magn\'usson}
\address{Hafnarfjörður, Iceland}
\email{gunnar@magnusson.io}
\date{\today}
\title{Examples of K\"ahler curvature tensors}

\begin{document}

\begin{abstract}
Let $R$ be the curvature tensor of a K\"ahler metric, and let $H$, $r$ and $s$
be its holomorphic sectional, Ricci and scalar curvatures.
A K\"ahler metric has a curvature tensor, which yields the holomorphic sectional
curvature, the Ricci tensor and the scalar curvature of the metric.
It is well-known that the positivity of some of these tensors implies the
positivity of others.
We work out local examples that illustrate the folklore that the reverse
implications between the positivity of these various tensors of all fail.
\end{abstract}

\maketitle

\section*{Introduction}

It is well-known that the curvature properties of K\"ahler metrics ``flow
downward''.
To give meaning to this statement,
let us say that a tensor $\alpha$ \emph{dominates} a tensor $\beta$
if any of the following implications hold:
\begin{align*}
\alpha > 0 &\implies \beta > 0,
&
\alpha &\geq 0 \implies \beta \geq 0,
\\
\alpha < 0 &\implies \beta < 0,
&
\alpha &\leq 0 \implies \beta \leq 0.
\end{align*}
Strictly speaking we need to specify what it means for a tensor to be
``positive'' or ``negative'' for this to make sense.
This is fairly clear for the tensors we'll discuss here, except for the
K\"ahler curvature tensor itself which we mean to be Griffiths positive (or
negative).

If $R$ is the curvature tensor of a K\"ahler metric, its various derived
curvature tensors are the holomorphic sectional curvature $H$, the Ricci
tensor $r$ and the scalar curvature~$s$.
Then
$$
\begin{tikzcd}
& R \ar[dl] \ar[dr] &
\\
H \ar[dr] & & r \ar[dl]
\\
& s &
\end{tikzcd}
$$
where an arrow $\alpha \to \beta$ means that $\alpha$ dominates $\beta$
(see for example \cite{zheng2000complex}).

The witches of the field know that neither $H$ nor $r$ dominates the other.
The best evidence for this seems to be indirect.
We for example know that Hirzebruch surfaces (surfaces that are projective
bundles over $\kk P^1$) carry metrics with positive holomorphic sectional
curvature, but carry no metric of positive Ricci curvature as their anticanonical
bundle is not ample;
see \cite{hitchin1975curvature,alvarez2018projectivized,yang2019hirzebruch}.
There are some suggestive results in the negatively curved case, though,
where a compact K\"ahler manifold that carries a metric of negative holomorphic
sectional curvature also carries a possibly different metric of negative Ricci
curvature; see \cite{wu2016negative,tosatti2017extension,diverio2019quasi}.
Could it be the same metric?
We don't know, but it would be very nice if some easy local calculations said
it was. In this note we show it is not that easy.

It is also folklore that all of the reverse relations in the diagram fail (see
again \cite{zheng2000complex}). Some do not hold by general considerations; for
example Hirzebruch surfaces
again carry a metric of positive holomorphic sectional curvature but not one of
positive Griffiths curvature, for otherwise they would be $\kk P^2$ by
Siu--Yau's solution of the Frankel conjecture~\cite{siu1980compact} and, lo,
they are not.

We feel like it would be nice to have some examples that show explicitly what
doesn't hold here.
In this note we work out local examples that show that the reverse of any known
positivity implication fails.
As other people have noted it would be nice to have compact examples of these
failures but it is fairly difficult to get one's hands on those.
Suggestions are welcome.



\section{Algebraic curvature tensors}



Let $V$ be a complex vector space of dimension $n$, which we think of as the
tangent space of a complex manifold at a given point.
The curvature tensor $R$ of a Hermitian metric on the manifold identifies with
a Hermitian form $q$ on $V \otimes V$, defined by
$$
R(x, \ov y, z, \ov w)
= q(x \otimes z, \ov{y \otimes w}).
$$
If the metric is K\"ahler we get an additional symmetry
$R(x, \ov y, z, \ov w) = R(z, \ov y, x, \ov w)$
(and the ones induced by conjugating).
A nice alternate reference for what we discuss here is
\cite{algebraic-kahler-curvature}.

We write $B(V)$ for the real vector space of Hermitian forms on $V$.
The curvature tensor of a Hermitian metric is then just a member of $B(V
\otimes V)$.
We'll call such an element an algebraic curvature tensor, and one that
satisfies the symmetries of a K\"ahler curvature tensor an algebraic
K\"ahler curvature tensor.


\begin{prop}
\label{prop:dominance}
If $R$ is an algebraic K\"ahler curvature tensor, then
$$
\begin{tikzcd}
& R \ar[dl] \ar[dr] &
\\
H \ar[dr] & & r \ar[dl]
\\
& s &
\end{tikzcd}
$$
\end{prop}

This is both well-known and mostly easily proved by taking traces in an
orthonormal basis.
The trickiest impliciation is to show that $H \to s$.
Berger~\cite{berger1965varietes} gave the original proof by showing that
$$
\frac{1}{\Vol S^{2n-1}} \int_{S^{2n-1}} \!\!\! H(v) \, d\sigma(v)
= \frac{1}{\binom{n+1}{2}} s
$$
by writing $v = \sum_{j=1}^n v_j e_j$ in an orthonormal basis $(e_j)$,
expanding the curvature tensor $R(v, \bar v, v, \bar v)$, and integrating
polynomials in $v_j$ over the sphere. (The reader who wants to try their
hand at this can see Folland~\cite{folland} for the last part.)


\section{Examples of non-dominance in dimension two}

Let's define
\begin{align*}
\cc H(V)
&= \{ \text{germs of smooth Hermitian metrics on $V$ at $0$} \},
\\
\cc K(V)
&= \{ \text{germs of smooth K\"ahler metrics on $V$ at $0$} \}.
\end{align*}
We just write $\cc H$ if $V$ is clear from context and will usually mean $V =
\kk C^n$. Note that $\cc K$ and $\cc H$ are only convex open cones in real
vector spaces and not full subspaces.
Evaluating the curvature tensor of a metric at $0$ defines nonlinear maps and a
commutative diagram
$$
\begin{tikzcd}
\cc K \ar[d] \ar[r,"\cc R"] & B(\operatorname{Sym}^2 \kk C^n)\ar[d,"\pi^*"]
\\
\cc H \ar[r, "\cc R"] & B(\kk C^n \otimes \kk C^n).
\end{tikzcd}
$$
Both vertical arrows are of course injective.
The horizonal arrows are neither injective nor surjective in general.


Let $H$ be the matrix of a Hermitian metric $h$ in a neighborhood around $0$ in
$V$. The Chern connection of $h$ is given by $D^{1,0} = H^{-1}\partial H$ and its
curvature form is
$$
F = \tfrac i2 \bar\partial(H^{-1}\partial H)
= - \tfrac i2 H^{-1} \bar\partial H \wedge H^{-1} \partial H
- \tfrac i2 H^{-1} \partial \bar\partial H.
$$
The curvature tensor is then $R(x,\ov y, z, \ov w) = h(F(x, \ov y) z, \ov w)$.

We can pick coordinates such that $H = I_n$ at $0$.
If $h$ is K\"ahler we can further get $\partial H = 0$ at $0$.
We will arrange this in our examples.
We now restrict to $V = \kk C^2$.

\begin{prop}
The image of $\cc K(\kk C^2)$ in $B(\kk C^2 \otimes \kk C^2)$ is
$$
\left\{
\begin{pmatrix}
x & a & a & c
\\
\bar a & z & z & b
\\
\bar a & z & z & b
\\
\bar c & \bar b & \bar b & y
\end{pmatrix}
\biggm|
\text{$x, y, z \in \kk R$, $a, b, c \in \kk C$}
\right\}.
$$
\end{prop}


\begin{proof}
If $h$ is a Hermitian metric on a neighborhood of $0$ in $\kk C^2$
we can write it in matrix form as
$$
H = \begin{pmatrix}
a & c
\\
\ov c & b
\end{pmatrix}.
$$
After a change of coordinates, we can assume that $H(0) = I_2$.
Assuming $h$ is K\"ahler we can also arrange that $\partial H(0) = 0$.
Then the matrix of the curvature tensor in the basis
\def\tg#1#2{\frac{\partial}{\partial #1} \otimes \frac{\partial}{\partial #2}}
$(\tg zz, \tg zw, \tg wz, \tg ww)$
is
$$
R = \begin{pmatrix}
a_{z \ov z} & \ov c_{z \ov z} & a_{z \ov w} & \ov c_{z \ov w}
\\
c_{z \ov z} & b_{z \ov z} & c_{z \ov w} & b_{z \ov w}
\\
a_{w \ov z} & \ov c_{w \ov z} & a_{w \ov w} & \ov c_{w \ov w}
\\
c_{w \ov z} & b_{w \ov z} & c_{w \ov w} & b_{w \ov w}
\end{pmatrix}.
$$
We've written, for example, $a_{z \ov w}$ instead of $\frac{\partial^2 a}{\partial z \partial \ov w}$.
The metric $h$ is K\"ahler, so we have
$$
a_w = c_{z},
\quad
a_{\ov w} = \ov c_{\ov z},
\quad
b_z = \ov c_w,
\quad
b_{\ov z} = c_{\ov w},
$$
and thus
$$
c_{z \ov z} = a_{w \ov z},
\quad
\ov c_{w \ov w} = b_{z \ov w},
\quad
c_{z \ov w} = a_{w \ov w},
\quad
\ov c_{w \ov z} = b_{z \ov z}
$$
Then we get
$$
R = \begin{pmatrix}
a_{z \ov z} & a_{z \ov w} & a_{z \ov w} & a_{w \ov w}
\\
a_{w \ov z} & a_{w \ov w} & a_{w \ov w} & b_{z \ov w}
\\
a_{w \ov z} & a_{w \ov w} & a_{w \ov w} & b_{z \ov w}
\\
a_{w \ov w} & b_{w \ov z} & b_{w \ov z} & b_{w \ov w}
\end{pmatrix},
$$
where some simplifications happen because $a$ is real so $a_{w \ov z} =
a_{z \ov w}$ and we can ping-pong to $b_{z \ov z} = a_{w \ov w}$.
This is exactly of the form we announced.

We will conclude the proof by taking the above Hermitian matrix as given and
constructing a germ of a K\"ahler metric whose curvature tensor agrees with it
at $0$.
Suppose we let
$$
f(z,w)
= a_{z \ov z} z \bar z
+ a_{z \ov w} z \ov w
+ a_{w \ov z} w \ov z
+ a_{w \ov w} w \bar w.
$$
Then $f(0) = 0$ and $df(0) = 0$ and
$$
\operatorname{Hess} f(0)
= \begin{pmatrix}
a_{z \ov z} & a_{z \ov w}
\\
a_{w \ov z} & a_{w \ov w}
\end{pmatrix}.
$$
We also let
$$
g(z,w)
= a_{w \ov w} z \bar z
+ b_{z \ov w} z \ov w
+ b_{w \ov z} w \ov z
+ b_{w \ov w} w \bar w
$$
and
$$
h(z,w)
= h_1 z \bar z
+ h_2 z \ov w
+ h_3 w \ov z
+ h_4 w \bar w.
$$
We want
$$
a_{w \ov z} \ov z + a_{w \ov w} \bar w
= f_w = h_z
= h_1 \ov z + h_2 \ov w,
$$
so we set $h_1 = a_{w \ov z}$ and $h_2 = a_{w \ov w}$.
We also want
$$
a_{w \ov w} \ov z + b_{z \ov w} \ov w
= g_z = \ov h_w
= \ov h_3 \bar z + \ov h_4 \ov w
$$
so $h_3 = \ov{h_2}$ and $h_4 = b_{z \ov w}$.
The germ of the K\"ahler metric
$$
h(z,w)
= \begin{pmatrix}
1 + f(z,w) & h(z,w)
\\
\ov{h(z,w)} & 1 + g(z,w)
\end{pmatrix}
$$
then has the curvature tensor $R$ above.
\end{proof}

Let's look at some special cases. If the tensor looks like
$$
R =
\begin{pmatrix}
x & 0 & 0 & z
\\
0 & z & z & 0
\\
0 & z & z & 0
\\
z & 0 & 0 & y
\end{pmatrix}
$$
and we set $\xi = (u, v)$ then
\begin{align*}
R(\xi, \ov \xi, \xi, \ov \xi)
&= |u|^4 x + 2 (2 |u|^2 |v|^2 + \re(\bar u^2 v^2))z + |v|^4 y
\\
r(\xi, \ov \eta)
&=
\ov\eta^t
\begin{pmatrix}
x + z & 0 \\ 0 & z + x
\end{pmatrix}
\xi
\\
s &= x + 2z + y,
\end{align*}
and the sign of $R(\xi, \ov \xi, \xi, \ov \xi)$ is the same as the sign of the
holomorphic sectional curvature $H(\xi)$.

For complex numbers we have $|\re z| \leq |z|$, so we have
\begin{align*}
x |u|^4 + y |v|^4 + 2z |u|^2 |v|^2
&\leq R(\xi, \ov \xi, \xi, \ov \xi)
\\
&\leq x |u|^4 + y |v|^4 + 6z |u|^2 |v|^2.
\end{align*}
If further $u = v$ then
$$
R(\xi, \ov \xi, \xi, \ov \xi)
= (x + 6z + y) |u|^4.
$$

\begin{exam}[$s \not\to H$]
We can find $x, y, z$ such that $x + 2z + y > 0$ but $x + 6z + y < 0$,
(like $x = y = 4$, $z = -3$), so the scalar curvature does not dominate
the holomorphic sectional curvature.
\end{exam}

\begin{exam}[$s \not\to r$]
Take $x = 4$, $y = -1$ and $z = -1$. Then $x + 2z + y = 1 > 0$ but
$$
r(\xi, \ov \eta) =
\ov\eta^t
\begin{pmatrix}
3 & 0 \\ 0 & -2
\end{pmatrix}
\xi,
$$
is neither positive nor negative semidefinite, so the scalar curvature
does not dominate the Ricci curvature.
\end{exam}

\begin{exam}[$r \not\to H$]
Let $x = y = 2$ and $z = -1$. Then
$$
r(\xi, \ov \eta) =
\ov\eta^t
\begin{pmatrix}
1 & 0 \\ 0 & 1
\end{pmatrix}
\xi
$$
is positive-definite, but at $\xi = (u, u)$ we have
$$
R(\xi, \ov \xi, \xi, \ov \xi) = -4 |u|^4
$$
so the Ricci curvature does not dominate the holomorphic sectional curvature.
Note that the Ricci form is a multiple of the metric, so the metric is
K\"ahler--Einstein at the origin.
Any statement like ``this metric is K\"ahler--Einstein so its holomorphic
sectional curvature has a sign'' thus needs more information about the metric
or the manifold to have a chance of being true.
\end{exam}

\begin{exam}[$H \not\to r$]
We have
$$
R(\xi, \ov \xi, \xi, \ov \xi)
\geq
x |u|^4 + 2z |u|^2 |v|^2 + c |v|^4
$$
and the right-hand side is the quadratic form defined by
$$
Q = \begin{pmatrix}
x & z \\ z & y
\end{pmatrix}.
$$
This form is positive-definite if and only if
$$
0 < \tr Q = x + z
\qandq
0 < \det Q = xy - z^2.
$$
Suppose $z = -y - \epsilon$ with $\epsilon > 0$.
Then
$$
xy - z^2
= xy - y^2 - 2y \epsilon - \epsilon^2
= y(x - y - 2\epsilon) - \epsilon^2.
$$
If $x > y$, this is positive for small enough $\epsilon$ (like $x = 10$, $y =
1$, $\epsilon = 1$).
But then the Ricci form is
$$
r(\xi, \ov \eta) =
\ov\eta^t
\begin{pmatrix}
x & 0 \\ 0 & -\epsilon
\end{pmatrix}
\xi
$$
which is not positive-definite,
so the holomorphic sectional curvature does not dominate the Ricci
curvature.
\end{exam}

\begin{exam}[$r \not\to R$]
We have $\det R = 0$, because the curvature tensor of a K\"ahler metric
always has a nontrivial kernel (on manifolds of dimension $> 1$).
It is then enough to pick a metric with $r$ positive
to get a curvature tensor whose Ricci tensor is positive but is not itself
Griffiths positive.
Otherwise this direction is also covered by the $r \not\to H$ one.
\end{exam}

\begin{exam}[$H \not\to R$]
Let $x = y = 2$ and $z = -1$.
Then
$$
R(\xi, \ov \xi, \xi, \ov \xi)
\geq 2 |u|^4 + 2 |v|^4 - 2 |u|^2|v|^2
= |u|^4 + |v|^4 + (|u|^2 - |v|^2)^2 > 0
$$
so the holomorphic sectional curvature is positive, but
for $\xi = (u,0)$ and $\eta = (0,v)$ we have
$$
= R(\xi, \ov \xi, \eta, \ov \eta)
= - (|u|^2 + |v|^2 + u \ov v + v \ov u)
\leq 0
$$
and the inequality can be strict, so the holomorphic sectional curvature does
not dominate the Griffiths curvature.
\end{exam}


\subsection*{Known compact examples}


\begin{exam}[$H > 0 \not\to r > 0$ or $R > 0$]
As noted earlier, a Hirzebruch surface carries a K\"ahler metric with positive
holomorphic sectional curvature. There are such surfaces that do not have ample
canonical bundle, so they do not carry any K\"ahler metric with positive Ricci
curvature.
Those surfaces are also not the projective space, so they do not carry a
Griffiths-positive metric.
\end{exam}

We do not know of an example that shows that $H < 0 \not\to r < 0$ \emph{for
the given metric}.
By Wu--Yau--Tosatti--Yang it is probably difficult to find one.
Broder~\cite{broder} has a variety of examples of compact manifolds where $r <
0 \not\to H<0$.

\bibliographystyle{plainurl}
\bibliography{main}


\end{document}
