\documentclass[11pt]{article}

\usepackage[utf8]{inputenc}
\usepackage[T1]{fontenc}

\usepackage[normalem]{ulem}
\usepackage{textcomp}
\usepackage{hyperref}

\usepackage{amsmath}
\usepackage{amssymb}
\usepackage{amsthm}

\newcommand{\kk}[1]{\mathbb{#1}}
\newcommand{\cc}[1]{\mathcal{#1}}

\DeclareMathOperator{\im}{Im}
\DeclareMathOperator{\Ker}{Ker}
\DeclareMathOperator{\End}{End}
\DeclareMathOperator{\id}{id}

\newtheorem{question}{Question}

\author{Gunnar Þór Magnússon}
\date{\today}
\title{The savage garden of curvature tensors}

\hypersetup{
 pdfauthor={Gunnar Þór Magnússon},
 pdftitle={The savage garden of curvature tensors},
 pdfkeywords={},
 pdfsubject={},
 pdfcreator={Emacs 27.1 (Org mode 9.3)},
 pdflang={English}}


\begin{document}

\maketitle
\tableofcontents


\section{Topics}
\label{sec:org093d592}

Discuss positivity concepts: Nakano, Griffits positive. Ricci positive. Bisectional curvature; holomorphic bisectional curvature.

Can prove that holomorphic bisectional curvature > 0 -> scalar curvature > 0.


\section{Flat metrics}
\label{sec:org504b250}

Let $(E,h) \to X$ be a holomorphic Hermitian vector bundle. The curvature form $\Theta$ of $h$ is a smooth section of $\bigwedge^{1,1}T_X \otimes \End E$. The simplest such section is of course the zero section. We say that the metric $h$ is \emph{flat} if its curvature form is zero.


\subsection{Euclidean metric}

The prime example of such a metric is the standard Euclidean metric on any open set $U \subset \kk C^n$. The metric $h$ defined by the usual inner product,
$$
h(\xi, \eta) = \sum_{j=1}^n \xi_j \overline{\eta_j}.
$$
The associated K\"ahler form is
$$
\omega = \sum_{j=1}^n \frac{i}{2} dz_j \wedge d\bar z_j.
$$
This is clearly $d$-closed, so the Euclidean metric is K\"ahler.\footnote{It has a global potential $\|z\|^2$. This is seldom useful.} We have
$$
h(D\xi, \eta) + h(\xi, D\eta)
= d h(\xi, \eta)
= \sum_{j=1}^n d\xi_j \otimes \overline{\eta_j} + \xi_j \otimes \overline{d\eta_j}
$$
so the Chern connection of $h$ is equal to the exterior derivative.

% TODO: Is any of the following relevant to examples of curvature tensors?
% Do we want to only do examples or also have a review of some things that are known about curvature?

\subsection{Compact manifolds}

There are also compact manifolds with flat metrics. Let $\Lambda \subset \kk C^n$ be a lattice, that is, an abelian group of rank $2n$. Then $X := \kk C^n / \Lambda$ is a compact complex manifold, called a \emph{complex torus}. As $\Lambda$ acts by translations on $\kk C^n$, it preserves the Euclidean metric, which descends to $X$.

One can ask whether the converse holds: If $X$ is a compact complex manifold with a flat metric, then is $X$ a complex torus?

The answer is subtle. First, the proposed answer cannot be true, as some complex tori admit subgroups that act freely, and whose quotients are thus again compact complex manifolds with flat metrics that are not tori. In the K\"ahler case, that is the extent of the problems that can arise. We can see this quickly if we're prepared to nuke a mosquito:

Suppose $X$ is a compact K\"ahler manifold that admits a flat metric. All of its Chern classes are then zero, in particular the first one. By \cite{beauville1983}, the universal covering of $X$ splits into a product of $\kk C^n$ and compact manifolds with zero first Chern class. The vanishing of all Chern classes of $X$ implies the product only involves $\kk C^n$.

For Hermitian manifolds, the answer is slightly more complicated: A flat Hermitian manifold is covered by a simply connected complex Lie group~\cite{boothby1958}.


\subsection{Gauss--Manin}

% TODO: I'm not sure this should be here at all.

Let $\pi : X \to S$ be a family of compact K\"ahler manifolds over a connected smooth base $S$. All of the manifolds $X_s := \pi^{-1}(s)$ in the family are diffeomorphic, so they all have isomorphic cohomology groups $H^k(X_s, \kk Z)$. These can be assembled into a holomorphic vector bundle $E \to S$ (by taking direct images of constant sheaves) whose fibers are exactly
$$
E_s = H^k(X_s, \kk C).
$$
The construction also yields a connection $\nabla$ on this vector bundle, called the \emph{Gauss--Manin connection}. It is flat, and if $\alpha$ is a local section of $E$ and $Y \subset X_s$ a $2k$-dimensional submanifold, then
$$
d \! \int_{X_s}\!\!\!\! \alpha = \int_{X_s} \!\!\! \nabla \alpha.
$$
The Gauss--Manin connection is generally not the Chern connection of a metric on $E$.

Something mildly interesting happens in the middle cohomology. Suppose the manifolds in the family are of dimension $n$. Then there is an intersection form on $n$-forms, which yields a sesquilinear form
$$
b(\alpha, \beta) = \int_{X_s} i^{n^2} \alpha \wedge \overline{\beta}.
$$
This form is non-degenerate, but has mixed signature. One can still define connections for such forms, and the usual proof of the existance and uniqueness of the Chern connection carries through for them. One can check that the Gauss--Manin connection is the Chern connection of the intersection form in this sense. This does not appear to have any applications.



\section{Conformal metrics}
\label{sec:org65fcbad}

Let \((E,h) \to X\) be a holomorphic Hermitian vector bundle, and let \(D\) be its Chern connection.

If \(f\) is a smooth real-valued function on \(X\), then \(h' := e^f h\) is again a Hermitian metric, \emph{conformal} to \(h\). The formulas for the Chern connection and curvature of conformal metrics are less complicated than the equivalent formulas for Riemannian metrics. We have
$$
h'(D_{h'} s, t) + h'(s, D_{h'} t)
= d h'(s, t)
= df \otimes h'(s, t) + h'(D_h s, t) + h'(s, D_h t).
$$
Writing \(df = \partial f + \bar\partial f\) we see that the Chern connection of \(h'\) is
$$
D_{h'} s = \partial f \otimes s + D_h s.
$$
Using the expressions for the covariant exterior derivative, we also have
\begin{align*}
D_{h'}^2 s
&= D_{h'}(\partial f \otimes s + D_h s)
\\
&= d(\partial f) \otimes s - \partial f \wedge D_{h'} s + D_{h'}(D_h s)
\\
&= -\partial\bar\partial f \otimes s - \partial f \wedge (\partial f \otimes s + D_h s) + \partial f \wedge D_h s + D_h^2 s
\\
&= -\partial\bar\partial f \otimes s + D_h^2 s.
\end{align*}

The Chern curvature of \(h'\) is thus
$$
D^2_{h'} = -\partial\bar\partial f \otimes \id_E + D^2_h.
$$

It's fun to work out what this gives for metrics conformal to a flat metric. We'll do that later when we study a particular metric on the \hyperref[sec:org8f5818e]{Hopf manifold}.

\subsection{Conformal to K\"ahler is not K\"ahler}
\label{sec:org7b1cfdf}

It's worth mentioning that if \(h\) is a K\"ahler metric on a manifold of dimension \(n > 1\) and \(f\) is non-constant, then \(h'\) is not a K\"ahler metric. The reason is that if \(\omega\) is the symplectic form associated to \(h\), then the symplectic form of \(h'\) is \(e^f \omega\) and
$$
d(e^f \omega) = \omega \wedge (e^f df)
$$
and the linear morphism from one- to three-forms defined by wedging with the symplectic form \(\omega\) is injective.\footnote{The hard Lefschetz theorem generalizes this to cohomology.}
The proof reduces to linear algebra by a calculation in local coordinates, either Darbeaux ones or in holomorphic ones that are orthonormal at a point.


\section{Riemann surfaces}
\label{sec:org776713b}

In Riemannian geometry, the full curvature tensor doesn't appear until in dimension $4$. Before that, the low dimensions of the tangent space restrict what tensors can appear. The equivalent situation in complex geometry only occurs in complex dimension one, one Riemann surfaces. Complex dimension two is real dimension four and already contains all the complexities of the higher dimensions. Conversely, complex dimension one is only real dimension two and too simple to give a good idea of what happens in higher dimensions.

\subsection{Everything is K\"ahler}
Let $U \subset \kk C$ be an open set. We can imagine it is a small neighborhood around a point in a Riemann surface we care about. Let $h$ be a Hermitian metric on $U$. The K\"ahler form of the metric can be written as
$$
\omega = e^{f(z)} \frac{i}{2} dz \wedge d\bar z
$$
on $U$. Then $d\omega = 0$ for dimensional reasons, so any Hermitian metric on $U$ is K\"ahler.

\subsection{Curvature}
By the above, we also have $h = e^f h_{\mathrm{std}}$, so the Chern connection and curvature of the metric can be deduced from our discussion on conformal metrics. Notably, the curvature form is
$$
\Theta
= -\frac i2\partial\bar\partial f
= -\frac{1}{e^f}\frac{\partial^2f}{\partial z \partial \bar z} \; e^f \frac{i}{2} dz \wedge d\bar z
= -\Delta_\omega f \; \omega.
$$
There should be a tensor product with the identity map on $T_X$ in the first equality, but the identity map on a one-dimensional space is just multiplication by $1$ so we can skip it.

One can note that an endomorphism on a one-dimensional space can be identified with its trace. From that point of view, one can say that the curvature form in complex dimension one can be identified with the form one gets by taking the trace of the curvature endomorphisms. In some sense, the full curvature form collapses to the Ricci form. In some other sense, the above computations imply the Ricci form only contains the scalar curvature. There's not enough space for complex curvature tensors.


\subsection{Poincar\'e disk}

Let $D = \{z \in \kk C \mid |z| < 1\}$ be the unit disk in the complex plane. We can pull a Hermitian metric out of our hat by setting
$$
\omega = \frac 1{(1-|z|^2)^2} \frac i2 dz \wedge d\bar z.
$$
As
$$
\frac 1{(1-|z|^2)^2} = e^{-2\log(1-|z|^2)}
$$
the curvature form of $\omega$ is
\begin{align*}
2\frac i2 \partial\bar\partial \log(1-|z|^2)
&= 2\frac i2 \partial \frac{-z d\bar z}{1-|z|^2}
\\
&= 2\frac i2 \frac{- dz \wedge d\bar z}{1-|z|^2}
- 2\frac i2 \frac{-\bar z dz}{1-|z|^2} \wedge \frac{-zd\bar z}{1-|z|^2}
\\
&= 2\biggl(\frac{-1}{1-|z|^2} - \frac{|z|^2}{(1-|z|^2)^2} \biggr) \frac i2 dz \wedge d\bar z
\\
&= -2 \omega.
\end{align*}
The Poincar\'e metric is the first negatively curved K\"ahler metric we see. Most Riemann surfaces are negatively curved. Compact ones can be organized according to their genus, which is one-half of the dimension of their first homology group. Any compact Riemann surface of genus zero is isomorphic to the projective line and so is positively curved; a Riemann surface of genus one is a torus and thus flat; and any other Riemann surface is covered by the unit disk and negatively curved.



\section{Projective space}
\label{sec:orgcfabeed}

The complex projective space is the space of lines in a given complex vector space. That is, if $V$ is a complex vector space of dimension $n$ > 1, then
$$
\kk P(V) := \{ v \in V \mid v \not= 0 \} / \kk C^*,
$$
where $\kk C^*$ acts by multiplication. By convention we write $\kk P^n := \kk P(\kk C^{n+1})$.\footnote{Some people write $\kk C \kk P^n$ to distinguish between projective spaces over different fields. We'll stick to $\kk C$ so there's little to distinguish from.}

There is a projection map $\pi : V \setminus \{0\} \to \kk P(V)$ and we equip $\kk P(V)$ with the quotient topology. If we fix a Hermitian inner product $h$ on $V$, then the quotient map factors through the unit sphere:
$$
V \setminus \{0\} \to S(V, h) \to \kk P(V).
$$
As the unit sphere is compact, it follows that the projective space $\kk P(V)$ is compact.


\subsection{Manifold structure}

We're going to construct local holomorphic charts on $\kk P(V)$. Let $\lambda \in V^* \setminus \{0\}$ and define
$$
U_\lambda := \{ [v] \in \kk P(V) \mid \lambda(v) \not= 0 \}.
$$
This is an open set in $\kk P(V)$ equipped with the quotient topology, as its preimage under the projection is the complement of the hyperplane $\{v \in V \mid \lambda(v) = 0\}$, which is open.

We define a map $f_\lambda: U_\lambda \to V$ by setting
$$
f([v]) = v/\lambda(v).
$$
This is well defined by the linearity of $\lambda$ and by definition of $U_\lambda$. This map takes values in the affine hyperplane
$$
H_\lambda := \{ v \in V \mid \lambda(v) = 1 \}.
$$
It is in fact a bijection onto this set: If $v \in H_\lambda$ then $[v]$ is an element of $U_\lambda$ that maps to $v$, so $f_\lambda$ is surjective. If $[v], [w] \in U_\lambda$ are such that $f_\lambda(v) = f_\lambda(w)$, then $v/\lambda(v) = w/\lambda(w)$ for any representatives $v, w$ of those classes. Then $v = (\lambda(v)/\lambda(w)) w$, so $[v] = [w]$ and $f_\lambda$ is injective.

The map $f_\lambda$ is continuous: Let $U \subset H_\lambda$ be open. Then $\pi^{-1}(f_\lambda^{-1}(U)) = \kk C^* \cdot U$, which is open. Its inverse is also continuous: The map $f_\lambda^{-1}$ is just the restriction of $\pi$ to $H_\lambda$. Thus $f_\lambda$ is a homeomorphism.

The collection $(f_\lambda : U_\lambda \to H_\lambda)_{\lambda \in V^* \setminus \{0\}}$ covers $\kk P(V)$ and provides local homeomorphisms to spaces biholomorphic to $\kk C^{\dim V - 1}$. If $\lambda$ and $\lambda'$ are two nonzero elements of the dual space, then the transition map between charts is
$$
f_{\lambda'} \circ f_{\lambda}^{-1} : \{v \in H_\lambda \mid \lambda'(v) \not= 0 \} \to H_{\lambda'},
\quad
v \mapsto v/\lambda'(v).
$$
This map is a composition of field operations and a linear map, so it is holomorphic. The collection above thus forms a holomorphic atlas.


\subsection{Tautological bundle}


If we consider the trivial vector bundle $V \to \kk P(V)$, then the definition of projective space gives a line bundle $\cc O(-1) \subset V$ whose fiber over a point $[v]$ is
the line $\kk C \cdot v$. This \emph{tautological line bundle} is holomorphic:

On a local chart $H_\lambda$, the line bundle is given by $\kk C \cdot v \subset V$. It is trivialized by the Euler section $\xi(v) = v$, which is certainly holomorphic on the local chart. Changing coordinates to ones defined by $\lambda'$ multiplies the section by $1/\lambda'(\xi)$, which is holomorphic.



\subsection{Curvature of tautological bundle}

Fix a Hermitian inner product $h$ on $V$. This defines a flat Hermitian metric on the trivial vector bundle $V \to \kk P(V)$. It follows that the induced metric on $\cc O(-1)$ is non-positive.\footnote{Hence the $-1$.}

On $H_\lambda$, we have the Euler section $\xi$ of $\cc O(-1)$ given by $\xi(v) = v$. The curvature form of the line bundle is then
$$
-\frac i2 \partial \bar\partial \log h(\xi, \xi)
= -\frac i2\partial \frac{h(\xi, \partial \xi)}{h(\xi, \xi)}
= -\frac i2\frac{h(\partial \xi, \partial \xi)}{h(\xi, \xi)}
+ \frac i2\frac{h(\partial \xi, \xi)}{h(\xi, \xi)} \wedge \frac{h(\xi, \partial \xi)}{h(\xi, \xi)}.
$$
The Euler field satisfies $\partial_u \xi = u$ for holomorphic vector fields $u$. The sesquilinear form defined by the curvature form is then
$$
\phi(\alpha, \beta)
= -\frac{h(\alpha, \beta)}{h(\xi, \xi)}
+ \frac{h(\alpha, \xi)}{h(\xi, \xi)} \cdot \frac{h(\xi, \beta)}{h(\xi, \xi)}.
$$
We claim that this is negative-definite. We have
\begin{align*}
\phi(\alpha, \alpha)
&= -\frac{h(\alpha, \alpha)}{h(\xi, \xi)}
+ \frac{|h(\alpha, \xi)|^2}{h(\xi, \xi)^2}
\\
&\leq -\frac{h(\alpha, \alpha)}{h(\xi, \xi)}
+ \frac{h(\alpha, \alpha) h(\xi, \xi)}{h(\xi, \xi)^2}
= 0
\end{align*}
with equality if and only if $\alpha$ is a multiple of the Euler field $\xi$. But note that $H_\lambda$ is defined so that $\lambda(\xi) = 1$, while its tangent space identifies with the set of vectors $\alpha$ such that $\lambda(\alpha) = 0$. Therefore $\alpha$ is not a multiple of $\xi$.


\subsection{Fubini--Study metric}

The dual of the tautological line bundle is denoted by $\cc O(1) := \cc O(-1)^*$. By the above, it is a positive line bundle on the projective space $\kk P(V)$. Its curvature form $\omega$ is called the \emph{Fubini--Study metric} on the projective space. It is a K\"ahler metric. In local coordinates it is given by the Hermitian form $-\phi$ above.

We want to compute the Chern connection and curvature of this metric. We \emph{could} try to do it directly from the expression for $\phi$, but we're going to try to cheat. We will look at the Hermitian metric $\psi := -h(\xi,\xi) \phi$, given by
$$
\psi(\alpha, \beta) = h(\alpha, \beta) h(\xi, \xi) - h(\alpha, \xi) h(\xi, \beta).
$$
We have
\begin{align*}
\partial\psi(\alpha, \beta)
&= h(\partial\alpha, \beta) h(\xi, \xi)
+ h(\alpha, \beta) h(\partial\xi, \xi)
- h(\partial\alpha, \xi) h(\xi, \beta)
- h(\alpha, \xi) h(\partial\xi, \beta)
\\
&= \psi(\partial\alpha, \beta)
+ h\biggl(\frac{h(\partial\xi, \xi)}{h(\xi,\xi)} \alpha, \beta\biggr) h(\xi,\xi)
- h(\alpha, \xi) h(\partial\xi, \beta)
\\
&= \psi(\partial\alpha, \beta)
+ \psi\biggl(\frac{h(\partial\xi, \xi)}{h(\xi,\xi)} \alpha, \beta\biggr)
+ \frac{h(\partial\xi, \xi)h(\alpha,\xi) h(\xi,\beta)}{h(\xi,\xi)}
- \frac{h(\alpha, \xi) h(\partial\xi, \beta)h(\xi,\xi)}{h(\xi,\xi)}
\\
&= \psi(\partial\alpha, \beta)
+ \psi\biggl(\frac{h(\partial\xi, \xi)}{h(\xi,\xi)} \alpha, \beta\biggr)
+ \frac{h(\alpha,\xi)}{h(\xi,\xi)} \bigl(h(\partial\xi,\xi)h(\xi,\beta) - h(\partial\xi,\beta)h(\xi,\xi)\bigr)
\\
&=\psi(\partial\alpha, \beta)
+ \psi\biggl(\frac{h(\partial\xi, \xi)}{h(\xi,\xi)} \alpha, \beta\biggr)
- \psi\biggl(\frac{h(\alpha,\xi)}{h(\xi,\xi)} \partial\xi,\beta\biggr)
\\
&=
\psi\biggl(
\partial\alpha
+ \frac{h(\partial\xi, \xi)}{h(\xi,\xi)} \alpha
- \frac{h(\alpha,\xi)}{h(\xi,\xi)} \partial\xi
, \beta
\biggr).
\end{align*}
The Chern connection of $\psi$ is then
$$
D_\psi \alpha = d\alpha
+ \frac{h(\partial\xi, \xi)}{h(\xi,\xi)} \alpha
- \frac{h(\alpha,\xi)}{h(\xi,\xi)} \partial\xi.
$$
Its curvature form is then
$$
\bar\partial D_\psi \alpha
= -\frac{h(\partial\xi,\partial\xi)}{h(\xi,\xi)} \alpha
+ \frac{h(\partial\xi,\xi)}{h(\xi,\xi)}\wedge \frac{h(\xi,\partial\xi)}{h(\xi,\xi)} \alpha
- \frac{h(\alpha,\partial\xi)}{h(\xi,\xi)} \wedge \partial\xi
+ \frac{h(\alpha,\xi)}{h(\xi,\xi)}\frac{h(\xi,\partial\xi)}{h(\xi,\xi)} \wedge \partial\xi,
$$
where we have been very careful with the signs when calculating $\bar\partial h(\partial\xi, \xi)$.
We're of course not interested in the curvature form of $\psi$, but we have
$$
\psi = e^{\log h(\xi,\xi)^2} \phi.
$$
Then we get
$$
\bar\partial \log h(\xi,\xi)^2
= 2\bar\partial \log h(\xi,\xi)
= 2 \frac{h(\xi,\partial\xi)}{h(\xi,\xi)}
$$
and
$$
\partial\bar\partial \log h(\xi,\xi)^2
= 2 \frac{h(\partial\xi,\partial\xi)}{h(\xi,\xi)}
- 2 \frac{h(\partial\xi,\xi)}{h(\xi,\xi)} \wedge \frac{h(\xi,\partial\xi)}{h(\xi,\xi)}.
$$
We know that
$$
\Theta_\psi = - \partial\bar\partial \log h(\xi,\xi)^2 \otimes \id + \Theta_\phi
$$
so
\begin{align*}
\Theta_\phi \alpha
&= \frac{h(\partial\xi,\partial\xi)}{h(\xi,\xi)} \alpha
- \frac{h(\partial\xi,\xi)}{h(\xi,\xi)}\wedge \frac{h(\xi,\partial\xi)}{h(\xi,\xi)} \alpha
+ \partial\xi \wedge \frac{h(\alpha,\partial\xi)}{h(\xi,\xi)}
- \partial\xi \wedge \frac{h(\alpha,\xi)}{h(\xi,\xi)}\frac{h(\xi,\partial\xi)}{h(\xi,\xi)}
\\
&= \phi(\partial\xi, \partial\xi) \wedge \alpha
+ \partial\xi \wedge \phi(\alpha, \partial\xi).
\end{align*}


\subsection{Fubini--Study curvature tensors}

From the above discussion, we see that
$$
R(\alpha,\beta,\gamma,\delta)
= \phi(\alpha,\beta)\phi(\gamma,\delta) + \phi(\gamma,\beta)\phi(\alpha,\delta).
$$
The holomorphic bisectional curvature is then
$$
B(\alpha,\beta) = \frac{R(\alpha,\alpha,\beta,\beta)}{\phi(\alpha,\alpha)\phi(\beta,\beta)}
= \frac{\phi(\alpha,\alpha)\phi(\beta,\beta)+|\phi(\alpha,\beta)|^2}{\phi(\alpha,\alpha)\phi(\beta,\beta)}.
$$
We have
$$
1 \leq B(\alpha,\beta) \leq 2
$$
by Cauchy--Schwarz. The holomorphic sectional curvature is
$$
H(\alpha) = B(\alpha,\alpha) = 2.
$$
To find the Ricci curvature we pick a holomorphic frame $(\zeta_1, \ldots, \zeta_n)$ that's orthonormal at a point $z$. There we have
$$
r(\alpha,\beta)
= \sum_{j=1}^n R(\alpha,\beta,\zeta_j,\zeta_j)
= \sum_{j=1}^n \phi(\alpha,\beta) + \phi(\alpha,\zeta_j)\phi(\zeta_j,\beta)
= (n+1) \phi(\alpha,\beta).
$$
The Fubini--Study metric is thus a K\"ahler--Einstein metric. Contracting the Ricci tensor we find that its scalar curvature is
$$
s = n(n+1).
$$


\section{Projective space redux}

Let $V$ be a complex vector space of dimension $n > 1$ as before, and let $h$ be a Hermitian inner product on $V$. We take as given that the projective space $\kk P(V)$ exists and is a holomorphic manifold. We denote by $\pi : V \setminus \{0\} \to \kk P(V)$ the holomorphic projection map.

As before we have the trivial vector bundle $V \to \kk P(V)$ and the tautological line bundle $\cc O(-1) \to \kk P(V)$. We also have such bundles over the complex manifold $V \setminus \{0\}$, and its tangent bundle is in fact $T_{V \setminus \{0\}} = V$. We also see that the pullback of $\pi^*\cc O(-1)$ is the ``same'' line bundle $\cc O(-1)$ over $V \setminus \{0\}$. There is also a short exact sequence
$$
0 \longrightarrow
\cc O(-1) \longrightarrow
T_{V\setminus\{0\}} \longrightarrow
\pi^* T_{\kk P(V)} \longrightarrow
0
$$
of vector bundles over $V \setminus \{0\}$.\footnote{This is not the Euler sequence, which exists over $\kk P(V)$ and includes extra twists.}

On $V\setminus\{0\}$, we have the flat Hermitian metric $h$. We equip $\cc O(-1)$ with the Hermitian metric induced by the inclusion into the tangent bundle. The Euler section $\xi(v) = v$ over $V \setminus \{0\}$ is a nowhere-zero holomorphic section that trivializes the bundle $\cc O(-1)$. The curvature form of the metric on $\cc O(-1)$ is thus
$$
\omega
= -\frac i2\partial\bar\partial \log h(\xi,\xi)
= -\frac i2 \frac{h(\partial\xi,\partial\xi)}{h(\xi,\xi)}
+ \frac i2 \frac{h(\partial\xi,\xi)}{h(\xi,\xi)}\wedge\frac{h(\xi,\partial\xi)}{h(\xi,\xi)}.
$$
Evaluated on tangent vectors, we have
$$
\omega(\alpha,\beta)
= - \frac{h(\alpha,\beta)}{h(\xi,\xi)}
+ \frac{h(\alpha,\xi)}{h(\xi,\xi)} \cdot \frac{h(\xi,\beta)}{h(\xi,\xi)}.
$$
Cauchy--Schwarz shows that $\omega(\alpha,\alpha) \leq 0$ with equality if and only if $\alpha$ is a multiple of $\xi$. Note that $\pi^*T_{\kk P(V)}$ contains exactly the fields that are not multiples of $\xi$. The form $\omega$ is clearly invariant under the action of $\kk C^*$, which means that it descends to define a closed $(1,1)$-form on $T_{\kk P(V)}$. That form is negative-definite there.

Like before we now consider $\cc O(1)$ and the induced K\"ahler metric, which we denote $\omega$ in a slight abuse of notation. Our game will be to compute the curvature form of $\pi^*\omega$ on $\pi^*T_{\kk P(V)}$. For variety, we'll do this directly instead of appealing to conformal metrics like before.

Denote the Chern connection of $\omega$ on $\kk P(V)$ by $D$. Upstairs, we have
\begin{align*}
\pi^*\omega(\pi^*D \alpha, \beta)
&= \partial \pi^*\omega(\alpha,\beta)
\\
&= \frac{h(\partial\alpha, \beta)}{h(\xi,\xi)}
- \frac{h(\partial\xi, \xi)}{h(\xi,\xi)}
 \frac{h(\alpha,\beta)}{h(\xi,\xi)}
\\
&\qquad{}
- \frac{h(\partial\alpha,\xi)}{h(\xi,\xi)}
\frac{h(\xi,\beta)}{h(\xi,\xi)}
+ \frac{h(\partial\xi,\xi)}{h(\xi,\xi)}
\frac{h(\alpha,\xi)}{h(\xi,\xi)}
\frac{h(\xi,\beta)}{h(\xi,\xi)}
\\
&\qquad{}
- \frac{h(\alpha,\xi)}{h(\xi,\xi)} \frac{h(\partial\xi,\beta)}{h(\xi,\xi)}
+\frac{h(\alpha, \xi)}{h(\xi,\xi)}
\frac{h(\partial\xi,\xi)}{h(\xi,\xi)}
\frac{h(\xi,\beta)}{h(\xi,\xi)}
\\
&= \pi^*\omega(\partial\alpha,\beta)
- \pi^*\omega\biggl(\frac{h(\alpha,\xi)}{h(\xi,\xi)} \partial\xi, \beta \biggr)
- \pi^*\omega\biggl(\frac{h(\partial\xi,\xi)}{h(\xi,\xi)} \alpha, \beta \biggr)
\end{align*}
so the pullback of the Chern connection is
$$
\pi^*D \alpha
= d\alpha
- \frac{h(\alpha,\xi)}{h(\xi,\xi)} \partial\xi
- \frac{h(\partial\xi,\xi)}{h(\xi,\xi)} \alpha.
$$
For the curvature form, we then have
\begin{align*}
\pi^*\frac i2\bar\partial D\alpha
&= -\frac i2 \partial\bar\partial \alpha
- \frac i2 \frac{h(\alpha,\partial\xi)}{h(\alpha,\xi)} \wedge \partial \xi
+ \frac i2 \frac{h(\alpha,\xi)}{h(\xi,\xi)} \frac{h(\xi,\partial\xi)}{h(\xi,\xi)} \wedge \partial\xi
\\
&\qquad{}
+ \frac i2\frac{h(\partial\xi,\partial\xi)}{h(\xi,\xi)} \alpha
- \frac i2\frac{h(\partial\xi,\xi)}{h(\xi,\xi)} \wedge \frac{h(\xi,\partial\xi)}{h(\xi,\xi)} \alpha
\\
&= - \pi^*\omega(\alpha, \partial\xi) \wedge \partial \xi
+ \pi^*\omega(\partial\xi, \partial\xi) \alpha.
\end{align*}
As before we have to be careful about signs when computing $\bar\partial \frac{h(\partial\xi,\xi)}{h(\xi,\xi)}$. Note also the sign flip when we (silently) commute $\partial\xi$ below. We conclude that the pullback of the curvature tensor is
$$
\pi^* R(\alpha,\beta,\gamma,\delta)
= \pi^*\omega(\alpha, \beta) \pi^*\omega(\gamma, \delta)
+ \pi^*\omega(\alpha, \delta) \pi^*\omega(\gamma, \beta).
$$
We can extend the pullback of the curvature tensor to all of $T_{V\setminus\{0\}}$; it simply vanishes when any of the entries is a multiple of $\xi$ as $\pi^*\omega$ vanishes in that case.


\section{TODO: Higher-dimensional hyperbolic metrics}
\label{sec:org21fa1aa}

\section{TODO: Grassmannian}
\label{sec:org34425b6}

\section{TODO: Flag manifold}
\label{sec:orga0ef6a0}

These are K\"ahler. There is a ``natural'' non-K\"ahler metric on them.

\section{Hopf manifold}
\label{sec:org8f5818e}

Let \(\lambda \in \kk C\) be a complex number such that \(0 < |\lambda| < 1\). The \emph{Hopf manifold} is the quotient
$$
X := (\kk C^n \setminus \{0\}) / \Gamma,
$$
where \(\Gamma \cong \kk Z\) is the group generated by \(\lambda\) that acts by
$$
\lambda \cdot (z_1, \ldots, z_n) = (\lambda z_1, \ldots, \lambda z_n).
$$
The Hopf manifold is compact and is diffeomorphic to \(S^{2n-1} \times S^1\). In particular,
$$
H^2(X, \kk C) \cong H^2(S^{2n-1}, \kk C) \oplus H^1(S^{2n-1}, \kk C) \otimes H^1(S^1, \kk C) = 0,
$$
so it is not K\"ahler.

Let \(\pi : \kk C^n \setminus \{0\} \to X\) be the projection. If \(h\) is a Hermitian metric on \(X\), then its pullback \(\pi^*h\) is a Hermitian metric on \(\kk C^n \setminus \{0\}\) that is invariant under the action of \(\Gamma\). If we write its K\"ahler form as \(\sum_{j,k} a_{jk}(z) \tfrac{i}{2} dz_j \wedge d\bar z_k\), then the smooth functions \(a_{jk}\) must satisfy
$$
a_{jk}(\lambda z) = \frac{1}{|\lambda|^2} a_{jk}(z).
$$
We can pick one such metric to inspect; we'll choose \(h = \frac{1}{\|z\|^2} h_{\mathrm{std}}\), that is, a metric that is conformal to the standard metric on \(\kk C^n \setminus \{0\}\). The Ph.D. thesis \cite{istrati:tel-02156198} has a nice discussion of the history of these metrics on the Hopf manifold.

As the metric is conformal to a K\"ahler metric, and the conformal factor is non-constant, the metric is not K\"ahler. (We already knew this because \emph{no} metric on the Hopf manifold is K\"ahler, but it's nice to check.)


\subsection{Curvature tensor}
\label{sec:org96d544d}

We've \hyperref[sec:org65fcbad]{already computed} the curvature of a conformal metric, so we know the curvature form of the metric $h$ is
$$
D^2 s = \partial\bar\partial \log \|z\|^2 \otimes s.
$$
Let's compute this and express the curvature tensor of the metric. To do that we'll use the Euler field
$$
\xi = \sum_{j=1}^n z_j \frac{\partial}{\partial z_j}.
$$
It is a holomorphic tensor field whose norm is $\|\xi\|^2 = \|z\|^2$ and satisfies $\partial_\alpha \xi = \alpha$ for holomorphic tensor fields $\alpha$. We have
$$
\bar\partial \log \|z\|^2
= \frac{\langle \xi, \partial\xi \rangle}{\langle \xi,\xi \rangle}
$$
and
$$
\partial\bar\partial \log \|z\|^2
= \frac{\langle \partial \xi, \partial \xi \rangle}{\langle \xi, \xi \rangle}
- \frac{\langle \partial\xi, \xi\rangle}{\langle \xi, \xi \rangle} \wedge \frac{\langle \xi, \partial \xi \rangle}{\langle \xi, \xi \rangle}
= h(\partial\xi, \partial\xi) - h(\partial\xi, \xi) \wedge h(\xi, \partial\xi).
$$
The curvature tensor of the metric \(h\) on the Hopf manifold is then
$$
R(\alpha,\beta,\gamma,\delta)
= h(\alpha, \beta) h(\gamma, \delta)
- h(\alpha, \xi) h(\xi, \beta) h(\gamma, \delta).
$$
We note that it has the expected conjugate symmetries, that is, that \(R(\beta, \alpha, \delta, \gamma) = \overline{R(\alpha, \beta, \gamma, \delta)}\), but \(R(\gamma, \delta, \alpha, \beta) \not= R(\alpha, \beta, \gamma, \delta)\) like it would if this were the curvature tensor of a K\"ahler metric.


\subsection{Holomorphic sectional curvature}
\label{sec:org6471503}

The holomorphic bisectional curvature of the metric is
$$
B(\alpha,\beta)
= \frac{R(\alpha,\beta,\alpha,\beta)}{\|\alpha\|^2\|\beta\|^2}
= \frac{|h(\alpha,\beta)|^2 - h(\alpha,\xi)h(\xi,\beta)h(\alpha,\beta)}{\|\alpha\|^2\|\beta\|^2}.
$$
The holomorphic sectional curvature of the Hopf manifold is
$$
H(\alpha)
= B(\alpha,\alpha)
= \frac{\|\alpha\|^4 - |h(\alpha,\xi)|^2 \|\alpha\|^2}{\|\alpha\|^4}
= \frac{\|\alpha\|^2 - |h(\alpha,\xi)|^2}{\|\alpha\|^2}.
$$
We have the bounds
$$
0 \leq H(\alpha) \leq 1
$$
that are obtained when $\alpha$ is a multiple of $\xi$ and when it is orthogonal to $\xi$, respectively.

\subsection{Ricci tensors}
\label{sec:org3942125}

The curvature tensor can be contracted in three ways to obtain a \((1,1)\)-form. On a K\"ahler manifold, all three ways give the same result; on a non-K\"ahler manifold they may not.

The easiest of these to compute for us is the one given by taking the traces of the endomorphisms in the curvature form. As those endomorphisms are the identity here, we find that
$$
r_1(\alpha, \beta)
= n \bigl( h(\alpha, \beta) - h(\alpha, \xi) h(\xi, \beta) \bigr).
$$
This is the same as we obtain by contracting the curvature tensor along \(\delta\) and \(\gamma\). As before, Cauchy--Schwarz gives us the estimates
$$
0
\leq \frac{r_1(\alpha, \alpha)}{\|\alpha\|^2}
\leq n
$$
which are sharp under the same conditions as before. This form is the curvature form of the anti-canonical bundle on $X$ when equipped with the metric induced by $h$, and surprisingly it is semipositive. It cannot be positive, of course, as that would imply that $X$ were projective, but it comes as close as it can.

Our second contraction is along \(\alpha\) and \(\beta\). The only relevant part of the curvature tensor we don't know how to contract is \(h(\alpha, \xi)h(\xi, \beta)\). Let \((\zeta_1, \ldots, \zeta_n)\) be a local holomorphic frame that's orthonormal at a point \(z\) we care about. We have
$$
\sum_{j=1}^n h(\zeta_j, \xi) h(\xi, \zeta_j) = h(\xi, \xi) = 1
$$
as \(\xi = \sum_{j=1}^n h(\xi, \zeta_j) \zeta_j\) and \(h(\xi,\xi) = 1\).
Then
$$
r_2(\gamma, \delta)
= n h(\gamma, \delta) - h(\gamma, \delta)
= (n-1) h(\gamma, \delta).
$$
This form is not only different from \(r_1\) but it is positive-definite.

The third contraction is along \(\beta\) and \(\gamma\). We let \((\zeta_1, \ldots, \zeta_n)\) be a local holomorphic frame that's orthonormal at a point \(z\) as before. We have
$$
\sum_{j=1}^n h(\alpha, \zeta_j) h(\zeta_j, \delta)
= h(\alpha, \delta).
$$
Also
$$
\sum_{j=1}^n h(\alpha, \xi) h(\xi, \zeta_j) h(\zeta_j, \delta)
= h(\alpha, \xi) \sum_{j=1}^n  h(\xi, \zeta_j) h(\zeta_j, \delta)
= h(\alpha, \xi) h(\xi, \delta).
$$
Together, we get
$$
r_3(\alpha, \delta)
= h(\alpha, \delta) - h(\alpha, \xi) h(\xi, \delta)
= \frac{1}{n} r_1(\alpha, \delta).
$$

\subsection{Scalar curvature}
\label{sec:orgf9212d2}

We can contract any of the Ricci-forms we got to obtain the scalar curvature of the Hopf manifold. Picking the first two, we get
$$
s = n(n-1),
$$
while picking the third gives \(1/n\) times that, so Hopf manifolds have positive constant scalar curvature.


\section{TODO: Iwasawa manifold}
\label{sec:orgd67c2ff}

\section{TODO: Direct image curvatures}
\label{sec:org5619e67}

Weil--Peterson, maybe.

Berndtsson has papers from about ten years ago we could look at. Cao et al took that further recently, but that's probably too advanced for what we want to do.


\section{TODO: Algebraic curvature tensors}
\label{sec:orga0722e9}

Dimension of their subspace.
Kobayashi-Nomitzu products.
Holomorphic bisectional curvature determines curvature tensor.


\section{TODO: Intuitive explanation for curvature forms}
\label{sec:org2b43ecb}

Wikipedia has a handwavy explanation of curvature as what happens when we parallel transport a section along a parallelogram. Can we make this precise?



\section{TODO: Hermitian metrics}
\label{sec:orgad8dbb1}

Discuss extra symmetries we get when \(E = T_X\).
Talk about torsion tensor and how it's related to the K\"ahler form
Define sectional curvatures, Ricci tensors, scalar curvature.

Discuss K\"ahler metrics. Extra curvature symmetries. All Ricci forms are equal.


\section{TODO: Hom-metrics}

Somewhere you have a draft that talks about the space of functions $f : X \to Y$ and how we can put a metric on that when given metrics on $X$ or $Y$. I think we computed the curvature tensors we get from that.


\section{TODO: Moduli space of complex tori}

That space has a fairly explicit metric whose curvature tensor we can compute. Might be related to the Grassmannian?


\section{TODO: Hilbert scheme or cycle space}

Check Barlet and Magnusson's book. Maybe difficult.

\section{TODO: Curves in projective space}

Done in \cite{fischer1983differential}, apparently. Looks inscrutable.

\section{TODO: Hypersurfaces in projective space}

Done in \cite{vitter1974curvature}, which I can verify once scihub's search comes back.



\bibliographystyle{plain}
\bibliography{sgct}

\end{document}
