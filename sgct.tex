\documentclass[11pt]{article}

\usepackage{tgpagella}
\linespread{1.1}
\usepackage[utf8]{inputenc}
\usepackage[T1]{fontenc}

\usepackage[normalem]{ulem}
\usepackage{textcomp}
\usepackage{hyperref}

\usepackage{amsmath}
\usepackage{amssymb}
\usepackage{amsthm}

\newtheorem{theo}{Theorem}
\newtheorem{prop}[theo]{Proposition}
\newtheorem{lemm}[theo]{Lemma}
\newtheorem{coro}[theo]{Corollary}
\theoremstyle{definition}
\newtheorem{defi}[theo]{Definition}
\newtheorem{exam}[theo]{Example}

\newcommand{\kk}[1]{\mathbb{#1}}
\newcommand{\cc}[1]{\mathcal{#1}}

\def\^#1{^{[#1]}}
\def\qandq{\quad\text{and}\quad}

\DeclareMathOperator{\im}{Im}
\DeclareMathOperator{\Vol}{Vol}
\DeclareMathOperator{\Ker}{Ker}
\DeclareMathOperator{\End}{End}
\DeclareMathOperator{\Hom}{Hom}
\DeclareMathOperator{\id}{id}
\DeclareMathOperator{\tr}{tr}

\newcommand{\ext}[1]{\bigwedge{}^{\!\!#1}\,}

\newtheorem{question}{Question}

\author{Gunnar Þór Magnússon}
\date{\today}
\title{The savage garden of curvature tensors}

\hypersetup{
 pdfauthor={Gunnar Þór Magnússon},
 pdftitle={The savage garden of curvature tensors},
 pdfkeywords={},
 pdfsubject={},
 pdfcreator={Emacs 27.1 (Org mode 9.3)},
 pdflang={English}}


\begin{document}

\maketitle
\tableofcontents


\section{Theory}
\label{sec:org093d592}

Let $E \to X$ be a holomorphic vector bundle. We denote by $\cc A^k(E)$ the sheaf $\bigwedge^k T_X^* \otimes E$ of $k$-forms with values in $E$. Similarly we denote by $\cc A^{p,q}(E)$ the sheaf $\bigwedge^{p,q} T_X^* \otimes E$ of $(p,q)$-forms with values in $E$.


\subsection{Connections}

\begin{defi}
A \emph{connection} $D$ on a vector bundle $E$ is a $\kk C$-linear morphism $D : \cc A^0(E) \to \cc A^1(E)$ that satisfies
$$
D(f \otimes s) = df \otimes s + f Ds
$$
for all smooth functions $f$ and sections $s$.
\end{defi}

We can extend the connection to forms of higher degree $D : \cc A^k(E) \to \cc A^{k+1}(E)$ by requiring that
$$
D(\omega \otimes s) = d \omega \otimes s + (-1)^{k} \omega \wedge D s
$$
for any $k$-form $\omega$ and section $s$ of $E$.


\begin{prop}
  Let $\alpha$ be a $p$-form and $\beta$ be a $q$-form with values in $E$. Then
$$
D(\alpha \wedge \beta)
= (d\alpha) \wedge \beta
+ (-1)^p \alpha \wedge D(\beta).
$$
\end{prop}

\begin{proof}
Assume that $\beta = \gamma \otimes s$, where $\gamma$ is a $q$-form and $s$ a section of $E$. Then
\begin{align*}
D(\alpha \wedge \beta)
= D(\alpha \wedge \gamma \otimes s)
&= d(\alpha \wedge \gamma) \otimes s
+ (-1)^{p+q} \alpha \wedge \gamma \wedge Ds
\\
&= d\alpha \wedge \gamma \otimes s
+ (-1)^p \alpha \wedge d\gamma \otimes s
+ (-1)^{p+q} \alpha \wedge \gamma \wedge Ds
\\
&=
d\alpha \wedge \beta
+ (-1)^p \alpha \wedge D\beta.
\end{align*}
The result follows from linearity.
\end{proof}


A holomorphic vector bundle almost comes with a free connection. The $\bar\partial$ operator can be defined on the sections of any holomorphic vector bundle. It is $\kk C$-linear, but only satisfies
$$
\bar\partial(f \otimes s) = \bar\partial f \otimes s + f \bar\partial s.
$$
We'll see later how adding some information to the bundle can yield connections related to the $\bar\partial$ operator.



\subsection{Covariant derivative}

\begin{defi}
If $\xi$ is a tangent field on $X$ and $s$ a section of $E$, we define the \emph{covariant derivative} of $s$ in the direction of $\xi$ by the contraction
$$
D_\xi s := \iota_\xi D s.
$$
\end{defi}

This operation is $\kk C$-linear on sections of $E$, and $C^\infty$-linear in $\xi$.

\begin{prop}
The covariant derivative of a $k$-form $\alpha$ with values in $E$ is
$$
\displaylines{
(D \alpha)(\xi_0, \ldots, \xi_k)
= \smash{\sum_{j=0}^k} (-1)^j D_{\xi_j} \alpha(\xi_0, \ldots, \hat{\xi}_j, \ldots, \xi_k)
\hfill\cr\hfill{}
+ \sum_{i < j} (-1)^{i+j} \alpha([\xi_i,\xi_j], \xi_0, \ldots, \hat{\xi}_i, \ldots, \hat{\xi}_j, \ldots, \xi_k).
}
$$
\end{prop}

\begin{proof}
Recall that the exterior derivative of a $k$-form can be defined by the formula
$$
\displaylines{
(d\omega)(\xi_0, \ldots, \xi_{k})
  = \smash{\sum_{j=0}^k} (-1)^j d_{\xi_j} \omega(\xi_0, \ldots, \hat{\xi}_j, \ldots, \xi_k)
  \hfill\cr\hfill{}
+ \sum_{i < j} (-1)^{i+j} \omega([\xi_i,\xi_j], \xi_0, \ldots, \hat{\xi}_i, \ldots, \hat{\xi}_j, \ldots, \xi_k),
}
$$
where $[\xi_i,\xi_j]$ is the Lie bracket and $\hat{\xi}_j$ means the element $\xi_j$ is omitted. For a decomposable $k$-form $\omega \otimes s$ we have
$$
D(\omega \otimes s) = d\omega \otimes s + (-1)^k \omega \wedge Ds.
$$
Evaluating this on tangent fields we get
$$
\displaylines{
D(\omega \otimes s)(\xi_0, \ldots, \xi_k)
= \smash{\sum_{j}} (-1)^j d_{\xi_j} \omega(\xi_0, \ldots, \hat \xi_j, \ldots, \xi_k) \cdot s
\hfill\cr\hfill{}
+ \smash{\sum_{i < j}} (-1)^{i+j} \omega([\xi_i,\xi_j], \xi_0, \ldots, \hat\xi_i, \ldots, \hat\xi_j, \ldots, \xi_k) \cdot s
\cr\hfill{}
+ (-1)^k(\omega \wedge Ds)(\xi_0, \ldots, \xi_k)
\cr{}
\phantom{D(\omega \otimes s)(\xi_0, \ldots, \xi_k)}
= \smash{\sum_{j}}\bigl( (-1)^j d_{\xi_j} \omega(\xi_0, \ldots, \hat \xi_j, \ldots, \xi_k) \cdot s
\hfill\cr\hfill{}
+ (-1)^k (-1)^j \omega(\xi_0, \ldots, \hat\xi_j, \ldots, \xi_k) D_{\xi_j} s\bigr)
\cr\hfill{}
+ \sum_{i < j} (-1)^{i+j} \omega([\xi_i,\xi_j], \xi_0, \ldots, \hat\xi_i, \ldots, \hat\xi_j, \ldots, \xi_k) \cdot s
\cr{}
\phantom{D(\omega \otimes s)(\xi_0, \ldots, \xi_k)}
= \smash{\sum_j} (-1)^j D_{\xi_j} \alpha(\xi_0, \ldots, \hat\xi_j, \ldots, \xi_k)
\hfill\cr\hfill{}
\sum_{i < j} (-1)^{i+j} \alpha([\xi_i,\xi_j], \xi_0, \ldots, \hat\xi_i, \ldots, \hat\xi_j, \ldots, \xi_k)
}
$$
and we conclude by extending by linearity.
\end{proof}


\subsection{Curvature form}

\begin{defi}
The \emph{curvature form} $F^D$ of a connection $D$ is defined by
$$
F^D \wedge s = D^2 s
$$
for sections $s$ of $E$. It is a $2$-form that takes values in $\End E$.
\end{defi}


\begin{prop}
Let $\alpha$ be a $k$-form with values in $E$. Then
$$
D^2 \alpha = F^D \wedge \alpha.
$$
\end{prop}

\begin{proof}
This is true for sections by definition. Let $\alpha = \omega \otimes s$ be a decomposable form. We have
$$
D\alpha = d\omega \otimes s + (-1)^k \omega \wedge Ds
$$
and
\begin{align*}
D^2 \alpha
&= d^2 \omega \otimes s - (-1)^k d\omega \wedge Ds
+ (-1)^k d\omega \wedge Ds + \omega \wedge D^2 s
\\
&= \omega \wedge F^D \wedge s
= F^D \wedge \alpha.
\end{align*}
The result follows by linearity.
\end{proof}

The above proof shows in particular that if $f$ is a smooth function (that is, a $0$-form) then $F^D(f s) = f F^D s$ for any section $s$. That is, the curvature is $C^\infty$-linear. This justifies our claim that the curvature form takes values in $\End E$.


\begin{prop}
\label{curvature-commutative}
Let $\xi, \eta$ be tangent fields on $X$ and $s$ a section of $E$. Then
$$
F^D(\xi, \eta) \wedge s
= D_\xi D_\eta s - D_\eta D_\xi s - D_{[\xi,\eta]} s.
$$
\end{prop}

\begin{proof}
Apply our formula for the exterior covariant derivative to the $1$-form $Ds$ which takes values in $E$. This gives
\[
F^D(\xi,\eta) s
= D(Ds)(\xi,\eta)
= D_\xi D_\eta s - D_\eta D_\xi s - D_{[\xi,\eta]} s.
\qedhere
\]
\end{proof}


\begin{prop}
The curvature form is $\cc C^\infty$-linear in any of its variables.
\end{prop}

\begin{proof}
Let $f$ be a smooth function. We have
$$
D(fs) = df \otimes s + f Ds
$$
so
\begin{align*}
D^2(fs)
&= D(df \otimes s + f Ds)
\\
&= d^2f \otimes s - df \wedge Ds + df \otimes Ds + f D^2 s
= f D^2 s.
\end{align*}
For any tangent field, we have $D_{f\xi}s = f D_\xi s$. Then
$$
D_\eta D_{f \xi} s
= D_\eta f D_\xi s
= d_\eta f \, D_\xi s + f D_\eta D_\xi s.
$$
We also have $[f\xi, \eta] = f[\xi,\eta] - d_\eta f \cdot \xi$, so
$$
D_{[f\xi,\eta]}s
= D_{f[\xi,\eta]}s - D_{d_\eta f \cdot \xi} s
= f D_{[\xi,\eta]} s - d_\eta f D_\xi s.
$$
Putting this together with Proposition~\ref{curvature-commutative} we get
\begin{align*}
F(f \xi, \eta) s
&= f D_\xi D_\eta s - (d_\eta f D_\xi s + f D_\eta D_\xi s) - (f D_{[\xi,\eta]} s - d_\eta f D_\xi s)
\\
&= f D_\xi D_\eta s - f D_\eta D_\xi s - f D_{[\xi,\eta]} s
= f F(\xi, \eta) s.
\end{align*}
We then conclude that $F(\xi,f \eta)s = f F(\xi,\eta) s$ by anticommutativity of $F$.
\end{proof}


\subsection{Metrics}

\begin{defi}
A \emph{Hermitian metric} $h$ on a holomorphic vector bundle $E \to X$ is a smooth section of $\bigwedge^{1,1}E^* := E^* \otimes \overline E^*$ that satisfies $h(t, s) = \overline{h(s, t)}$ for all sections $s, t$ of $E$.
\end{defi}

Any holomorphic vector bundle admits a Hermitian metric. This is clear over any trivializing neighborhood, and we can patch those together with a partition of unity.


\begin{defi}
Let $(E, h) \to X$ be a Hermitian holomorphic vector bundle. A connection $D$ on $E$ is \emph{compatible} with $h$ if
$$
d h(s, t) = h(Ds, t) + h(s, Dt)
$$
for all sections $s, t$ of $E$.
\end{defi}

A Hermitian metric comes with a unique connection:


\begin{prop}
Let $(E, h) \to X$ be a Hermitian holomorphic vector bundle. There exists a unique connection $D$ on $E$ that is compatible with the metric $h$ and whose $(0,1)$-part is $\bar\partial$. This is the \emph{Chern} connection of $(E,h)$.
\end{prop}

\begin{proof}
Let's view the metric $h$ as an isomorphism $E \to \overline E^*$. The map $t \mapsto \partial_\xi h(s, t)$ defines a $(0,1)$-form on $E$, so there exists a unique smooth section $A(\xi,s)$ of $E$ such that
$$
\partial_\xi h(s, t) = h(A(\xi, s), t)
$$
for all $t$.\footnote{What we'd \emph{really} like to do is to say that $\partial h$ is a $(1,0)$-form with values in $\bigwedge^{1,1}E$, and thus $h^{-1}\partial h$ is a $(1,0)$-form with values in $\End E$. Adding $\bar\partial$ gives the Chern connection. This doesn't work because the exterior derivative isn't well-defined on $\bigwedge^{1,1}E$, so $\partial h$ is meaningless. However, this \emph{is} how the construction of the Chern connection goes in a local frame. One then proves that the local construction glues along the coordinate change maps and is well-defined globally. We're able to get away with this here because $h(s,t)$ is a perfectly good smooth function that we can take exterior derivatives of.}
As $\partial$ is linear in $\xi$, then so is $A$. We have $\partial_\xi h(f s, t) = \partial_\xi f h(s,t) + f \partial_\xi h(s, t)$, so $A(\xi, fs) = \partial f A(\xi, s) + f A(\xi, s)$. The morphism $A$ depends smoothly on $\xi$ and $s$ because the metric is smooth.

It follows that $(\xi, s) \mapsto A(\xi, s)$ is a connection on holomorphic sections of $E$. Such an object can be extended uniquely to a connection on smooth sections of $E$: If $s$ is holomorphic and $f$ is smooth, we set $A(\xi, f s) = df \otimes s + f A(\xi, s)$. If $s$ is a smooth section, we pick local holomorphic sections $(s_1, \ldots, s_r)$ and find smooth functions $f_j$ such that $s = \sum_j f_j s_j$, and set $A(\xi,s) = \sum_j A(\xi, f_j s_j)$.

By construction, the resulting connection is compatible with the metric, and its $(0,1)$-part is $\bar\partial$.


LOCAL:

Let $(s_1, \ldots, s_r)$ be a holomorphic frame of $E$ over a trivializing neighborhood $U$. We write $H = (h(s_j, s_k))_{jk}$ for the matrix of $h$ in this frame, and define
$$
D s := H^{-1}\partial (H s) + \bar\partial s
$$
for any section $s$ of $E$ over $U$. This object $D$ is $\kk C$-linear, and if $f$ is a smooth function then
\begin{align*}
D(fs)
= H^{-1}\partial (H fs) + \bar\partial (fs)
&= H^{-1}(\partial f \otimes Hs + f\partial( H s)) + \bar\partial f \otimes s + f \bar\partial s
\\
&= df \otimes s + f Ds
\end{align*}
so $D$ is a connection on $E$ over $U$. Its $(0,1)$-part is clearly $\bar\partial$.

If $s = f \otimes s_j$ and $t = g \otimes s_k$ are smooth decomposable sections, then $h(s,t) = f\overline{g} h(s_j,s_k)$. Then
$$
d h(s,t)
= df \otimes \bar g h(s_j,s_k)
+ f d\bar g \otimes h(s_j,s_k)
+ f\bar g\partial h(s_j,s_k)
+ f\bar g\bar\partial h(s_j,s_k).
$$
Now, $Ds_j$ is defined exactly so that $h(Ds_j, s_k) = \partial h(s_j,s_k)$, and conjugating we find that $\bar\partial h(s_j,s_k) = h(s_j, D s_k)$. Putting these together we get
\begin{align*}
d h(s,t)
&= h(df \otimes s_j, g s_k)
+ h(f s_j, d g \otimes s_k)
+ h(f D s_j,g s_k)
+ h(f s_j,g D s_k)
\\
&= h(Ds, t) + h(s, Dt)
\end{align*}
so $D$ is compatible with $h$.

Finally, let $(t_1, \ldots, t_r)$ be a holomorphic frame over a trivializing neighborhood $V$ and assume that $U \cap V \not= \varnothing$. There is a holomorphic frame change matrix $G$ from the frame $(s_1, \ldots, s_r)$ to $(t_1,\ldots,t_r)$. The matrix of $h$ becomes $H = \overline{G}^{t}H'G$ under this frame change. Then
$$
H^{-1} \partial H
= G^{-1} (H')^{-1} (\overline G^t)^{-1} \overline G^t \partial(H'G)
= G^{-1} ((H')^{-1} \partial H') G + G^{-1} \partial G
$$
so $D$ glues together to define a connection on all of $E$.
\end{proof}


\begin{prop}
The curvature form of the Chern connection is a purely imaginary $(1,1)$-form.
\end{prop}

\begin{proof}
Let $s, t$ be sections of $E$. We have
\begin{align*}
0
= d^2 h(s, t)
&= d h(Ds, t) + d h(s, Dt)
\\
&= h(F s, t) - h(Ds, Dt) + h(Ds, Dt) + h(s, F t)
\\
&=
h(Fs,t) + h(\overline{F^{t}}s, t).
\end{align*}
It follows that $\overline{F^t} = -F$, so $F$ is imaginary.

We have $D = D^{1,0} + \bar\partial$, so $F = (D^{1,0})^2 + (\bar\partial D^{1,0} + D^{1,0}\bar\partial)$ because $\bar\partial^2 = 0$. That is, $F$ has no $(0,2)$-part. But then it also has no $(2,0)$-part by the above.
\end{proof}

For this reason we often consider the curvature form of a Chern connection to be $\Theta := \frac i2 F$. That is a real $(1,1)$-form with values in $\End E$.


Discuss positivity concepts: Nakano, Griffits positive. Ricci positive. Bisectional curvature; holomorphic bisectional curvature.

Can prove that holomorphic sectional curvature > 0 -> scalar curvature > 0.



\subsection{Hermitian metrics}

We now focus on $E = T_X$, the tangent bundle of a given manifold $X$.


\begin{prop}
Let $\operatorname{Herm}(T_X)$ denote the sheaf of smooth Hermitian forms on $T_X$. Then
$$
\operatorname{Herm}(T_X) \to \ext{1,1} T_X^*,
\quad
h \mapsto \im h
$$
is an isomorphism.
\end{prop}

\begin{proof}
Let $h$ be a Hermitian form. For tangent fields $\alpha$ and $\beta$ we have
$$
\overline{\im h(\alpha,\beta)}
= -\im \overline{h(\alpha, \beta)}
= -\im h(\beta, \alpha)
$$
so the morphism is well-defined. It is clearly $\kk R$-linear (even $C^\infty$-linear for real-valued functions). If $h$ is a Hermitian form such that $\im h = 0$, then $h$ is real for all tangent fields. In particular,
$$
h(i\alpha, \alpha) = i h(\alpha,\alpha)
$$
is real, so $h(\alpha,\alpha) = 0$ for all $\alpha$. It follows that $h = 0$, so the morphism is injective. By inspection in a local frame, both bundles have the same rank, so $h$ is surjective.
\end{proof}

TODO: For this to make sense we really have to look at $T_X^{\kk C} = T_X^{1,0} \oplus T_X^{0,1}$.

\begin{defi}
If $h$ is a Hermitian form, the $(1,1)$-form $\omega = -\im h$ is the \emph{K\"ahler form} of $h$.
\end{defi}


We will write $\nabla$ for the Chern connection of a given Hermitian metric $h$. By the discussion in the previous sections, we can apply this connection to tensor fields on $X$. In particular, we can consider the identity morphism $\id : T_X \to T_X$, which we view as a $(1,0)$-form with values in $T_X$. Taking its covariant derivative, we obtain
$$
(\nabla \id)(\alpha, \beta)
= \nabla_\alpha \beta - \nabla_\beta \alpha - [\alpha, \beta].
$$
This is better known as the torsion tensor of the connection:

\begin{defi}
The \emph{torsion} of $\nabla$ is
$$
\tau(\alpha,\beta)
= \nabla_\alpha \beta - \nabla_\beta \alpha - [\alpha, \beta].
$$
\end{defi}

\begin{prop}
The torsion tensor is antisymmetric and $\cc C^\infty$-linear in both of its variables.
\end{prop}

\begin{proof}
  That the tensor is antisymmetric is clear from its definition. If $f$ is a smooth function, we have
\begin{align*}
\tau(f\alpha,\beta)
&= \nabla_{f\alpha}\beta - \nabla_\beta(f\alpha) - [f\alpha,\beta]
\\
&= f\nabla_\alpha \beta - (d_\beta f \alpha + f \nabla_\beta \alpha) - (f[\alpha,\beta] - d_\beta f \alpha)
= f \tau(\alpha,\beta).
\qedhere
\end{align*}
\end{proof}

There's a very nice relationship between the torsion tensor and the K\"ahler form of the metric.

\begin{prop}
For holomorphic tangent fields, we have
$$
h(\tau(\alpha,\beta), \gamma)
= \partial\omega(\alpha,\beta,\overline\gamma).
$$
\end{prop}

\begin{proof}
By definition of the Chern connection, we have
$$
\partial \omega(\id, \id)
= \omega(\nabla \id, \id).
$$
Evaluating these $(2,1)$-forms on $\alpha, \beta$ and $\overline\gamma$, we get
\[
\partial\omega(\alpha,\beta,\overline{\gamma})
= h(\tau(\alpha,\beta), \gamma).
\qedhere
\]
\end{proof}

\begin{coro}
\label{kahler-zero-torsion}
The torsion of $\nabla$ is zero if and only if $d\omega = 0$.
\end{coro}


We'll have more to say about this condition later. For now we'll say a little more about the curvature of a Hermitian metric.


\begin{defi}
The \emph{curvature tensor} of a Hermitian metric $h$ is the $(2,2)$-tensor defined by
$$
R(\alpha,\beta,\gamma,\delta)
= h(\tfrac i2 F_{\alpha\beta}\gamma, \delta).
$$
\end{defi}

As $h$ and $F$ are $\cc C^\infty$-linear in their variables, so is $R$. We have
$$
R(\alpha,\beta,\gamma,\delta)
= h(\tfrac i2 F_{\alpha\beta}\gamma, \delta)
= h(\gamma, \tfrac i2 F_{\beta\alpha}\delta)
= \overline{h(\tfrac i2 F_{\beta\alpha}\delta, \gamma)}
= \overline{R(\beta,\alpha,\delta,\gamma)}.
$$
In general, this is the only symmetry the curvature tensor of a Hermitian metric satisfies. The obstruction to other symmetries we might expect is governed by the torsion tensor. This is explained by the Bianchi identities, but we'll see clearer how the absense of torsion results in extra symmetries later.


\begin{prop}[Bianchi identities]
The curvature form of a Hermitian metric satisfies
$$
F \wedge \id_{T_X} = \nabla \tau
\quad\text{and}\quad
\nabla F = 0.
$$
\end{prop}

These are the first and second Bianchi identities.

\begin{proof}
  For the first equality, take the covariant exterior derivative of the $1$-form $\id_{T_X}$ twice and recall the definition of $\tau$. For the second, take the covariant exterior derivative of the first to get
$$
\nabla F \wedge \id_{T_X} + F \wedge \tau = F \wedge \tau
$$
and cancel $F \wedge \tau$.
\end{proof}


The above does not say that $F_{\alpha\beta}\gamma = \nabla_\alpha \tau(\beta,\gamma)$. It is an equality of $3$-forms with values in $T_X$ and involves the permutations of all of the tensor fields.

The first Bianchi identity can only be stated for metrics on the tangent bundle, as it involves the torsion. The second Bianchi identity is true for the curvature form $F$ of an arbitrary connection $D$ on a vector bundle $E$, where it says that $D_{\End E} F = 0$.


\subsection{Derived curvature tensors}


A curvature tensor is a complicated thing. We humans have proven bad at understanding objects as multilinear as these. For many applications it is enough to look at simpler tensors that are constructed from the full curvature tensor.

The first two of these are the holomorphic bisectional and sectional curvatures.

\begin{defi}
  The \emph{holomorphic bisectional curvature} of a Hermitian metric $h$ is
$$
B(x,y) = \frac{R(x,x,y,y)}{h(x,x)h(y,y)}.
$$
The \emph{holomorphic sectional curvature} is
$$
H(x) = B(x,x) = \frac{R(x,x,x,x)}{h(x,x)^2}.
$$
\end{defi}

Both of these are defined to be invariant under scalings, so $B(fx,y) = B(x,y)$ for all smooth $f$. The holomorphic bisectional curvature is real-valued by the symmetry of a Hermitian curvature tensor, and so is the holomorphic sectional curvature.


\paragraph{}
It is also possible to use the metric to contract the curvature tensor and obtain a $(1,1)$-tensor.

\begin{defi}
  Let $(v_1, \ldots, v_n)$ be a local holomorphic frame that's orthonormal at a given point. There we define the \emph{first}, \emph{second}, and \emph{third Ricci tensors} by
\begin{align*}
r_1(x,y) &= \sum_{j=1}^n R(x,y,v_j,v_j),
\\
r_2(x,y) &= \sum_{j=1}^n R(v_j, v_j, x,y),
\\
r_3(x,y) &= \sum_{j=1}^n R(x,v_j,v_j,y).
\end{align*}
\end{defi}

There is a fourth possible contraction ($\sum_{j=1}^n R(v_j,y,x,v_j)$) but it is conjugate to $r_3$.

These definitions are independent of the holomorphic frame used. It's easiest to see this by rewriting them using the metric isomorphism $h : T_X \to \overline T_X^*$ and its inverse and taking traces. In general these definitions are really different (see Section~\ref{sec:org8f5818e}), but we'll see later that they all agree for K\"ahler metrics.

The first of these is a tensor associated to the curvature form of a line bundle, as its associated $(1,1)$-form is clearly equal to $\tr \frac i2 F$. That form is also closed, as it can be defined by $\frac i2 \partial\bar\partial \log \det h$. It's the curvature form of the anticanonical bundle $-K_X$ with the metric induced by $h$.




\subsection{K\"ahler metrics}

\begin{defi}
A \emph{K\"ahler metric} $h$ is a Hermitian metric with zero torsion.
\end{defi}


A metric is K\"ahler if and only if its K\"ahler form is closed by Corollary~\ref{kahler-zero-torsion}, which again means that it is a symplectic form. As such, many operations involving the K\"ahler metric descend to the level of cohomology on a K\"ahler manifold and restrict its structure.

In practice it is usually much easier to verify that the K\"ahler form of a metric is closed than to compute its torsion tensor. For example, a large class of K\"ahler metrics arise as the curvature forms of holomorphic line bundles (we'll see the details later), where it is obvious that the K\"ahler form is closed.

The historical definition of a K\"ahler metric is as a Hermitian metric whose K\"ahler form is closed~\cite{kahler}, where the condition was used to locally approximate the Euclidean metric around any point. As far as I know symplectic geometry -- that is, the study of manifolds that admit a non-degenerate closed $2$-form -- didn't take off as its own subject until Arnol'd~\cite{arnold} pointed out its connection to classical mechanics.

I'm unsure of when the vanishing of the torsion tensor was pointed out explicitly, but we've chosen it as our definition as it makes the connection between K\"ahler and Riemannian geometry very clear. It is my favorite condition, as it suggests that we should investigate what the isomorphism between harmonic forms and the cohomology groups of a compact smooth manifold looks like in the K\"ahler case, which leads quickly to the invention of Hodge theory. There are already plenty of books about Hodge theory, so we won't get into it here.

The first consequence of the vanishing of the torsion we'll note is that the curvature tensor of a K\"ahler metric has extra symmetries.

\begin{prop}
Let $R$ be the curvature tensor of a K\"ahler metric. Then
$$
R(\alpha,\beta,\gamma,\delta) = R(\gamma,\beta,\alpha,\delta)
\quad\text{and}\quad
R(\alpha,\beta,\gamma,\delta) = R(\alpha,\delta,\gamma,\beta).
$$
\end{prop}

\begin{proof}
The first Bianchi identity for a metric with vanishing torsion is $F \wedge \id = 0$. When we evaluate these $(2,1)$-form on tangent fields, we get
$$
F_{\alpha\beta}\gamma - F_{\gamma\beta}\alpha = 0
$$
% (1 2 3) = - (1 3 2) = (3 1 2) = -(3 2 1)
from which we conclude the first equality. The second follows from the first and the conjugate symmetry of the curvature form.
\end{proof}



\subsection{Algebraic K\"ahler curvature tensors}

Let $V$ be a complex vector space of dimension $n$. Let $\cc R$ be the subset of the space of $V \otimes \overline V \otimes V \otimes \overline V \to \kk C$ defined by elements $R$ that satisfy
$$
R(x, y, z, w) = \overline{R(y, x, w, z)}
\quad\text{and}\quad
R(x, y, z, w) = R(z, y, x, w)
$$
for all $x,y,z,w \in V$. These are the symmetries that the curvature tensor of a K\"ahler metric satisfies (the curvature tensor of a Hermitian metric only satisfies the conjugate symmetry). We discuss some properties of this set; our reference is \cite{algebraic-kahler-curvature}.

The set $\cc R$ is closed under addition and real multiplication, so it is a real subspace.


\begin{prop}
The set $\cc R$ is isomorphic to the space of Hermitian forms on $S^2 V$.
\end{prop}

\begin{proof}
Let $R$ be an element of our subspace. We define a bilinear form $b$ on $S^2 V$ by
$$
b(x \odot z, y \odot w)
= R(x, y, z, w).
$$
We have $R(x,y,z,w) = R(z,y,x,w)$ by hypothesis, and
$$
R(x,w,z,y)
= \overline{R(w,x,y,z)}
= \overline{R(y,x,w,z)}
= R(x,y,z,w)
$$
by using both properties of elements of $\cc R$. Our form $b$ is thus well-defined on the symmetric product of $V$, and Hermitian by the first property of elements of $\cc R$. The morphism this defines is $\kk R$-linear, and clearly injective.

Given a Hermitian form $b$ on $S^2V$, we also define
$$
R(x,y,z,w) = b(x \odot z, y \odot w).
$$
Then $R$ satisfies the properties required by the elements of $\cc R$. It is linear in its first and third variables, and conjugate linear in the others. It is thus a well-defined element of $\cc R$. The two maps defined here are inverses of each other, so the two subspaces are isomorphic.
\end{proof}


\begin{coro}
$$
\dim \cc R
= \binom{n+1}{2}^2.
$$
\end{coro}


\begin{proof}
The dimension of the space of Hermitian forms on a vector space of dimension $n$ has dimension $n^2$, and $S^k V$ has dimension $\binom{n + k - 1}{k}$.
\end{proof}


It's natural to wonder about the positivity properties of a tensor $R$ under this isomorphism. It is clear that if the image of a tensor is positive-definite, then the original tensor is Griffiths positive. The converse is false in general; Griffiths positivity does not imply that the image of the tensor is positive-definite, only that it is positive on vectors of the form $x \odot y$. Very surprisingly (to me, at least) the first counterexample of this can only show up in dimension~4:



\begin{lemm}
Let $V$ be a complex vector space. Then the map
$$
V \times V \to S^2 V,
\quad
(x, y) \mapsto x \odot y
$$
of topological spaces is continuous, and surjective if $\dim V \leq 3$.
\end{lemm}

\begin{proof}
  Let $(v_1, \ldots, v_n)$ be a basis of $V$. Then $(v_j \odot v_j ; v_j \odot v_k)_{1 \leq j \leq n; 1 \leq j < k \leq n}$ is a basis of $S^2V$. Here the meaning of the notation is that in the basis the elements $(v_j \odot v_j)$ come first, and then all elements $(v_j \odot v_k)$ with $j < k$. Let $(z_1, \ldots, z_n; w_1 \ldots, w_n)$ be coordinates for $V \times V$. The map in question is then equal to
$$
f(z_1, \ldots, z_n; w_1, \ldots, w_n)
=( z_j w_j; z_j w_k + z_k w_j )_{1 \leq j \leq n; 1 \leq j < k \leq n}.
$$
Its entries are polynomials, so it is continuous.

Let $a = (a_j; a_{jk})$ be an element of $S^2V$. We would like to solve the system of equations
\begin{align*}
  z_j w_j &= a_j &1 \leq j \leq n,\\
  z_j w_k + z_k w_j &= a_{jk} &1 \leq j < k \leq n.
\end{align*}
To this end, multiply each equation of the second form by $z_jz_k$. They become
$$
z_j^2 z_k w_k + z_jz_k^2 w_j = a_{jk} z_jz_k.
$$
Use now $z_jw_j = a_j$ and $z_kw_k = a_k$ to rewrite this as
$$
a_k z_j^2 + a_j z_k^2 = a_{jk} z_jz_k.
$$
This gives $\binom n2$ quadratic equations to solve simultaneously in $\kk C^n$. By B\'ezout that's going to be possible for all values of $a$ only when $\binom n2 \leq n$, which happens when $n \leq 3$.
\end{proof}

This suggests a notion of positivity for K\"ahler curvature tensors that should be stronger than Griffiths positivity in higher dimensions. It is related to notions of $k$-positivity for curvature forms on $E \otimes T_X$, as a tensor that is $k$-positive on $S^2(T_X)$ is $k$-semipositive on $T_X \otimes T_X$. It's not clear whether this distinction is meaningful or results in any interesting theorems about K\"ahler curvature tensors.



\paragraph{}





We often view a curvature form as a Hermitian form on $E \otimes T_X$. This corresponds here to the space of curvature tensors of Hermitian metrics. This space then has dimension $(n^2)^2 = n^4$. We do have that $\binom{n+1}{2}^2 \leq n^4$ with equality if and only if $n = 1$, which is a nice sanity check.

When $E = T_X$ we can also consider the space of Riemannian curvature tensors on the smooth manifold underlying $X$. The corresponding space of algebraic Riemannian curvature tensors has real dimension
$$
\frac{(2n)^2((2n)^2 - 1)}{12}
= \frac{n^2(4n^2 - 1)}{3}.
$$
We have
$$
\binom{n+1}{2}^2 \leq n^4 \leq \frac{n^2(4n^2 - 1)}{3}
$$
for all $n \geq 1$ with equality in either place if and only if $n = 1$, so in a given complex dimension there are more Hermitian curvature tensors than K\"ahler ones, and more Riemannian tensors than Hermitian ones. (Recall that a K\"ahler metric is Riemannian, but a Riemannian metric doesn't have to be compatible with any complex structure.) The space of K\"ahler tensors can be embedded in the space of Riemannian tensors (clearly), but Hermitian tensors do not embed there (as they're generally not the curvature tensors of Riemannian metrics).


\paragraph{}

A slightly surprising property of K\"ahler curvature tensors is that the holomorphic sectional curvature determines the whole tensor. The holomorphic bisectional curvature clearly does so, but it's not obvious that restricting to the diagonal doesn't lose information.

\begin{prop}
If $R_1$ and $R_2$ are elements of $\cc R$ such that
$$
R_1(x,x,x,x) = R_2(x,x,x,x)
$$
for all $x \in V$, then $R_1 = R_2$.
\end{prop}

\begin{proof}
We apply our favorite isomorphism and obtain two Hermitian forms $h_1$ and $h_2$ on $S^2V$. Our hypothesis is that $h_1(x \odot x, x \odot x) = h_2(x \odot x, x \odot x)$ for all $x \in V$.

We first show that this implies that $h_1(x \odot y, x \odot y) = h_2(x \odot y, x \odot y)$ for all $x, y \in V$. The result follows from this by a standard polarization argument for Hermitian forms. To do so, we take $t \in \kk C$ and an arbitrary Hermitian form $h$ and calculate
\begin{align*}
&h((x + ty) \odot (x + ty), (x + ty) \odot (x + ty))
\\
&= h(x \odot x, x \odot x) + 2 h(x \odot x, tx \odot y) + h(x \odot x, t^2 y \odot y)
\\
&\qquad{}
+ 2 h(t x \odot y, x \odot x) + 4h(t x \odot x, tx \odot y) + 2h(t x \odot x, t^2 y \odot y)
\\
&\qquad{}
+ h(t^22 y \odot y, x \odot x) + 2h(y \odot y, tx \odot y) + h(t^2 y \odot y, t^2 y \odot y).
\end{align*}
Adding the expressions for $t$ and $-t$ we get
$$
\displaylines{
h((x + ty) \odot (x + ty), (x + ty) \odot (x + ty))
+ h((x - ty) \odot (x - ty), (x - ty) \odot (x - ty))
\hfill\cr\hfill{}
= 2 h(x \odot x, x \odot x)
+ \overline t^2 h(x \odot x, y \odot y)
+ 2 |t|^2 h(x \odot y, x \odot y)
\cr\hfill{}
+ t^2 h(y \odot y, x \odot x)
+ 2 |t|^4 h( y \odot y, y \odot y).
}
$$
Adding up what we get for $t = 1$ and $t = i$ then gives
\begin{align*}
&h((x + y) \odot (x + y), (x + y) \odot (x + y))
\\
&\qquad{}
+ h((x - y) \odot (x - y), (x - y) \odot (x - y))
\\
&\qquad{}
+ h((x + iy) \odot (x + iy), (x + iy) \odot (x + iy))
\\
&\qquad{}
+ h((x - iy) \odot (x - iy), (x - iy) \odot (x - iy))
\\
&=
4 h(x \odot x, x \odot x)
+ h(x \odot x, y \odot y)
+ 2 h(x \odot y, x \odot y)
+ h(y \odot y, x \odot x)
\\
&\qquad{}
- h(x \odot x, y \odot y)
+ 2 h(x \odot y, x \odot y)
- h(y \odot y, x \odot x)
+ 4 h(y \odot y, y \odot y)
\\
&= 4 \bigl(
h(x \odot x, x \odot x)
+ h(x \odot y, x \odot y)
+ h(y \odot y, y \odot y)
\bigr).
\end{align*}
The moral of this is that we can express $h(x \odot y, x \odot y)$ as a sum of values of the form $h(z \odot z, z \odot z)$. Thus if $h_1$ and $h_2$ agree on $x \odot x$ for all $x$, they agree for all $x \odot y$. Using the standard polarization identity for Hermitian forms we now conclude that $h_1(x \odot y, z \odot w) = h_2(x \odot y, z \odot w)$ for all $x,y,z,w \in V$.
\end{proof}

There's an alternate proof of this fact in \cite[Lemma~7.19]{zheng2000complex} for those who find this argument boring. I can't say I love it, but it does show that the reason this works is because being able to square vector elements lets us run the polarization argument twice to bootstrap ourselves from $x \odot x$ to $x \odot y$ to all decomposable tensors.

I only know of one place in the wild where knowledge of the holomorphic sectional curvature is used to compute the rest of the curvature tensor: Siu~\cite{siu1986curvature} does this in an impressive paper where he computes the curvature of the Weil--Petersson metric on moduli spaces of manifolds with negative first Chern class. Reading that paper, one gets the sense that there are people who do differential geometry, and the rest of us.

Knowing a polarization identity like this lets us prove things like that a bound on the holomorphic sectional curvature extends to a bound on the whole curvature tensor. One can prove that if $|H(x)| \leq C$ for all $x$, then $|B(x,y)| \leq 6 C$ for all $x,y$. From there we can polarize again to obtain $|R(x,y,z,w)| \leq 24C |x||y||z||w|$ for all $x,y,z,w$. I believe this is mostly useful to then quote general structure results from Riemannian geometry, but don't remember particular applications. These bounds are obtained by applying the triangle inequality to cases where equality is clearly not obtained and are likely not optimal. I don't know what the optimal bounds on $B$ and $R$ are given a bound on $H$.\footnote{If I had to guess I'd think the optimal bound would be attained on constant holomorphic curvature, where we have $|H| = C \implies |B| \leq 2C$.} It doesn't seem to be a question whose answer would have applications.

\paragraph{}
A Hermitian metric generally has three different Ricci forms. For a K\"ahler metric, these are all equal.

\begin{prop}
Let $R \in \cc R$. All the Ricci forms of $R$ are equal.
\end{prop}

\begin{proof}
  Pick a Hermitian inner product $h$ on $V$ and an orthonormal basis $(v_1,\ldots,v_n)$. We want to compare
\begin{align*}
r_1(x,y) &= \sum_{j=1}^n R(x,y,v_j,v_j),
\\
r_2(x,y) &= \sum_{j=1}^n R(v_j, v_j, x,y),
\\
r_3(x,y) &= \sum_{j=1}^n R(x,v_j,v_j,y).
\end{align*}
By the symmetries of K\"ahler curvature tensors we have $R(x,y,v_j,v_j) = R(x,v_j,v_j,y)$, so $r_1 = r_3$. Again by the same, we have $R(x,v_j,v_j,y) = R(v_j,v_j,x,y)$, so $r_3 = r_2$.
\end{proof}

For a K\"ahler metric we thus speak of \emph{the} Ricci form and tensor of the metric. It also follows that the Ricci form of a K\"ahler metric is closed.


\paragraph{}
We can again contract the Ricci tensor of a K\"ahler metric and get the \emph{scalar curvature} of the metric. It is equal to the scalar curvature of the underlying Riemannian metric.

\begin{defi}
The \emph{scalar curvature} of a K\"ahler metric with curvature tensor $R$ is
$$
s = \sum_{k=1}^n r(v_k,v_k) = \sum_{j,k=1}^n R(v_j,v_j,v_k,v_k).
$$
\end{defi}


\begin{prop}
Let $\omega$ be the K\"ahler form of a K\"ahler metric, and let $r$ be its Ricci-form. Then
$$
s \, \omega\^{n} = r \wedge \omega\^{n-1}.
$$
\end{prop}

\begin{proof}
The right-hand side is equal to $\Lambda(r) \omega\^n$, which is just the contraction of the Ricci-form by the metric, which is the scalar curvature.
\end{proof}

The Ricci-form of a K\"ahler metric represents the cohomology class $c_1(X)$, as it is the curvature form of a Hermitian metric on $-K_X$. The above proposition then implies that on a compact K\"ahler manifold, the integral of the scalar curvature is a cohomological invariant.



\paragraph{}

Let's discuss some positivity properties of the various curvature tensors we have. The tangent bundle is just another vector bundle, so we can speak of Nakano-positivity and Griffiths-positivity of its curvature. Beyond that, the derived curvature tensors can also be positive. We can summarize the relationships between the positivity of the various tensors as below.


\begin{prop}
\label{derived-tensor-positivity}
  Let $R$ be a K\"ahler curvature tensor.
  \begin{enumerate}
    \item If $R$ is Griffiths-positive, then $B$ is positive.
    \item If $B$ is positive, then $H$ and $r$ are positive.
    \item If $H$ is positive, then $s$ is positive.
    \item If $r$ is positive, then $s$ is positive.
  \end{enumerate}
\end{prop}

The proposition remains true if we replace ``positive'' with ``semipositive'', ``negative'' or ``seminegative'' everywhere.

\begin{proof}
  \begin{enumerate}
  \item True by definition of $B$.
  \item The positivity of $H$ is obvious. We note that
    $$
    r(x,x) = \sum_{j=1}^n R(x,x,v_j,v_j) = |x|^2 \sum_{j=1}^n B(x,v_j)
    $$
    from which the positivity of $r$ follows.
  \item Follows from Lemma~\ref{holomorphic-sectional-to-scalar}.
  \item The scalar curvature is the trace of the Ricci form, so this is obvious.
  \end{enumerate}
\end{proof}


\begin{lemm}[Berger~\cite{berger1965varietes}\footnote{Apparently. I haven't found a copy of this source to double-check. I believe one is available in the library in CIRM in Marseille.  Diverio may know where to find a copy, and Yang or Zheng possibly do as well; at least they've all cited this source.}]
\label{holomorphic-sectional-to-scalar}
Let $h$ be a Hermitian inner product on $V$, and $H$ and $s$ the holomorphic sectional and scalar curvatures of a K\"aher curvature tensor. Then
$$
\int_{S(V,h)} H(x) \, d\sigma(x) = \frac{\Vol(S^{2n-1})}{n(n+1)} s.
$$
\end{lemm}

\begin{proof}
  We pick an orthonormal basis $(v_1,\ldots,v_n)$ of $V$ and write $z = \sum_{j=1}^n z_j v_j$. On the unit sphere we then have
$$
H(z)
= R(z,z,z,z)
= \sum_{j,k,l,m=1}^n z_j \bar z_k z_l \bar z_m R_{jklm}.
$$
Consider the real and imaginary parts of the polynomial here. By inspecting degrees, we see that the result is odd in one of its variables and so its integral over the unit sphere vanishes unless $j = k$ and $l = m$ or $j = m$ and $l = k$. We thus have to evaluate
$$
\int_{S^{2n-1}} |z_j|^4 \, d\sigma
\quad\text{and}\quad
\int_{S^{2n-1}} |z_j|^2 |z_k|^2 \, d\sigma
\quad(j \not= k).
$$
After Folland~\cite{folland}, we know that
\begin{align*}
\int_{S^{2n-1}} d\sigma
&= \Vol(S^{2n-1}) = \frac{2\Gamma(\tfrac12)^{2n}}{\Gamma(n)},
\\
\int_{S^{2n-1}} x_j^4 \, d\sigma
&= \frac{2\Gamma(\tfrac 52)\Gamma(\tfrac 12)^{2n-1}}{\Gamma(n+2)}
\\
&= \frac{\Gamma(\tfrac52)}{\Gamma(\tfrac12)(n+1)n} \Vol(S^{2n-1})
= \frac{3}{4n(n+1)} \Vol(S^{2n-1}),
\\
\int_{S^{2n-1}} x_j^2 x_k^2 \, d\sigma
&= \frac{2\Gamma(\tfrac32)^2\Gamma(\tfrac 12)^{2n-2}}{\Gamma(n+2)}
\\
&= \frac{\Gamma(\tfrac32)^2}{\Gamma(\tfrac12)^2 (n+1)n} \Vol(S^{2n-1})
= \frac{1}{4n(n+1)} \Vol(S^{2n-1})
\end{align*}
for real variables $x_j, x_k$ when $j \not= k$. As
\begin{align*}
|z_j|^4 &= x_j^4 + 2 x_j^2 y_j^2 + y_j^4,
\\
|z_j|^2 |z_k|^2 &= x_j^2 x_k^2 + x_j^2 y_k^2 + y_j^2 x_k^2 + y_j^2 y_k^2
\end{align*}
we get that
\begin{align*}
\int_{S^{2n-1}} |z_j|^4 \, d\sigma
&= \frac{2}{n(n+1)} \Vol(S^{2n-1}),
\\
\int_{S^{2n-1}} |z_j|^2 |z_k|^2 \, d\sigma
&= \frac{1}{n(n+1)} \Vol(S^{2n-1})
\quad(j \not= k).
\end{align*}
This finally gives
\begin{align*}
\int_{S^{2n-1}} H(z) \,d\sigma
&= \sum_{j=1}^n \int_{S^{2n-1}} |z_j|^4 \, d\sigma
+ \sum_{j \not= k}^n \int_{S^{2n-1}} |z_j|^2|z_k|^2 \, d\sigma
\\
&= \frac{2}{n(n+1)} \Vol(S^{2n-1}) \sum_{j=1}^n R(v_j,v_j,v_j,v_j)
\\
&\qquad{}+ \frac{1}{n(n+1)} \Vol(S^{2n-1}) \sum_{j \not= k}^n R(v_j,v_j,v_k,v_k)
\\
&\qquad{}+
\frac{1}{n(n+1)} \Vol(S^{2n-1}) \sum_{j \not= k}^n R(v_j,v_k,v_k,v_j)
\\
&= \frac{1}{n(n+1)} \Vol(S^{2n-1}) \sum_{j,k=1}^n R(v_j,v_j,v_k,v_k)
\\
&\qquad{}+
\frac{1}{n(n+1)} \Vol(S^{2n-1}) \sum_{j,k=1}^n R(v_j,v_k,v_k,v_j)
\\
&= \frac{2}{n(n+1)} \Vol(S^{2n-1}) \sum_{j,k=1}^n R(v_j,v_j,v_k,v_k)
\\
&= \frac{1}{n(n+1)} \Vol(S^{2n-1}) \cdot s
\end{align*}
where we have used the symmetries of the curvature tensor.
\end{proof}


\begin{proof}[Second attempt]
Fix a Hermitian inner product $h$ on $V$ and let it induce an inner
product $g$ on $S^2V$. For any Hermitian form $b$ on $S^2V$, we have
$$
\tr_{g} b
= C \int_{S(S^2V, g)} g(g^{-1}b x, x)
= C \int_{S(S^2V, g)} b(x, x).
$$
Taking $b = g$ gives $\dim S^2V = C \Vol(S(S^2V, g))$, which determines $C$. Pick an orthonormal basis $(v_1,\ldots,v_n)$ of $V$, and let $(v_j v_k)$ be the induced orthonormal basis of $S^2V$. If $x = \sum x_{jk} v_{jk}$, we have
$$
b(x, x)
= \sum_{jk,lm} x_{jk} \overline x_{lm} b(v_{jk}, v_{lm}).
$$
The integral of these over the unit sphere is zero unless $j = l$ and $k = m$. We have
$$
\displaylines{
b(v_{jk}, v_{jk})
= b(v_j v_k, v_j v_k)
= b(\alpha_{jk}, \alpha_{jk})
+ b(\beta_{jk}, \beta_{jk})
+ b(\gamma_{jk}, \gamma_{jk})
\hfill\cr\hfill{}
+ b(\delta_{jk},\delta_{jk})
- b(v_{jj},v_{jj})
- b(v_{kk},v_{kk}),
}
$$
where $\alpha_{jk} = (v_j + v_k)^2/2$,
$\beta_{jk} = (v_j - v_k)^2/2$,
$\gamma_{jk} = (v_j + iv_k)^2/2$
and $\delta_{jk} = (v_j - iv_k)^2/2$.

These greek vectors are unit vectors. How do we reduce the integral to the unit sphere in $V$?
\end{proof}

The nicest thing one can say about this proof is that it works. I really want there to be a proof of this lemma based around the polarization identity we used to show the holomorphic sectional curvature determines the whole curvature tensor, but can't figure out how to reduce the integral over the unit sphere in $S^2V$ (from which we get equality with the trace) to one over the unit sphere in $V$. There's of course the Veronese embedding $j : V \to S^2V$ given by $v \mapsto v \odot v$, but it's not clear why pulling anything back by it helps.


The reverse implications in Proposition~\ref{derived-tensor-positivity} all fail. The reader can amuse themselves by constructing algebraic curvature tensors that exhibit those failures.\footnote{If the reader does not find this amusing, they can see how we construct some such tensors in the next subsection.}

The relationship between the holomorphic sectional and Ricci curvatures is not obvious. The positivity or negativity of one does not imply the positivity or negativity of the other. For example, there are known natural metrics with positive holomorphic sectional curvature that do not have positive Ricci curvature~\cite{hitchin1975curvature,alvarez2016positive,yang2019hirzebruch}. In the next section we construct some algebraic curvature tensors that show this concretely.

\begin{exam}
The holomorphic sectional curvature determines the whole curvature tensor, so having trivial holomorphic sectional curvature implies trivial Ricci curvature. The converse is false:

Let $X$ be a compact K\"ahler manifold with $c_1(X) = 0$, and let $h$ be a Ricci-flat K\"ahler metric on $X$. Recall also that a compact K\"ahler manifold is flat if and only if it is a complex torus. If $X$ is not a complex torus (like a Calabi--Yau manifold, K3 surface or a hyperk\"ahler manifold~\cite{beauville1983}), then $R$ is nontrivial, so the holomorphic sectional curvature of $h$ is also nontrivial. By the above lemma, its average over the unit sphere is zero at any point, so it cannot have a definitive sign. In particular, zero Ricci curvature does not imply trivial holomorphic sectional curvature.
\end{exam}

There are some results where the existence of a metric with either tensor positive is used to prove the existence of a different metric with the other tensor positive. For example, Kobayashi conjectured that a manifold that admits a metric with negative holomorphic sectional curvature should have ample canonical bundle; this was proved by Wu and Yau~\cite{wu2016negative} for projective manifolds. See also the survey by Diverio~\cite{diverio2020kobayashi}.



\paragraph{}
We didn't talk about Nakano positivity when discussing positivity of K\"ahler curvature tensors. The reason is that none of those tensors are Nakano positive or negative.

\begin{prop}
Let $R$ be an algebraic K\"ahler curvature tensor, and let $Q$ be the Hermitian form it defines on $V \otimes V$. Then $\dim \Ker Q \geq \binom n2$.
\end{prop}

\begin{proof}
The form $Q$ is defined by
$$
Q(x \otimes z, y \otimes w) = R(x,y,z,w)
$$
for $x,y,z,w \in V$ and extended by linearity. When viewed as a linear morphism $V \otimes V \to \overline V^* \otimes \overline V^*$, the symmetries of the curvature tensor entail that
$$
Q(x \otimes z) = Q(z \otimes x)
$$
for all $x, z \in V$.

Let $(v_1,\ldots,v_n)$ be a basis of $V$, so $(v_j \otimes v_k)_{1 \leq j,k \leq n}$ is a basis of $V \otimes V$. Then $v_j \otimes v_l - v_l \otimes v_j \in \Ker Q$ for all $j, l$, and there are $\binom n2$ ways of picking nontrivial such elements.
\end{proof}

\begin{coro}
\label{no-nakano-positive}
A K\"ahler metric is never $2$-positive. In particular, it is never Nakano positive or Nakano negative.
\end{coro}

I haven't seen this proposition or corollary pointed out in the literature, but they must be known to the experts.


The Nakano vanishing theorem is often used to prove that there is no Nakano positive metric on the projective space~\cite[Example~8.4]{demailly-complex}: One computes its cohomology groups and notes that some that the Nakano vanishing theorem would annihilate are in fact nontrivial. This is not a case of nuking a mosquito, as we can only say that no \emph{K\"ahler} metric on the projective space is Nakano positive, and can say nothing about \emph{Hermitian} metrics.



\subsection{Constructing algebraic curvature tensors}


An easy way to construct elements of $\cc R$ is to pick a basis for $V$ and just write down some tensors. This is in some way unsatisfactory because we'd like to see curvature tensors that arise in the wild.\footnote{We're not above doing this, we just feel bad about it.} To do so we'd like to have coordinate-invariant ways of constructing elements of $\cc R$. We point out two of those here.

Suppose we have a Hermitian form $b$ on $V$. It induces a Hermitian form on $S^2 V$, defined by
$$
(b \odot b)(x \odot y, z \odot w)
= \tfrac 12 \bigl( b(x,z)b(y,w) + b(x,w)b(y,z) \bigr).
$$
% x \odot z = 1/2 (x \otimes z + z \otimes x)
% y \odot w = 1/2 (y \otimes w + w \otimes y)
% < x \odot z, y \odot w >
% = < x \otimes z + z \otimes x, y \otimes w + w \otimes y >
% = 1/4 (< x \otimes z, y \otimes w > + < x \otimes z, w \otimes y>
% + < z \otimes x, y \otimes w > + < z \otimes x, w \otimes y>)
% = 1/4 (<x,y> <z,w> + <x,w> <z,y> + <z,y> <x,w> + <z,w> <x,y>)
% = 1/2 (<x,y> <z,w> + <x,w> <z,y>)
This is the complex-geometric version of the Kulkarni--Nomizu product. It has the property that if $(v_1,\ldots,v_n)$ is an orthonormal basis of $V$, then $(v_j \odot v_k)_{1 \leq j \leq k \leq n}$ is an orthonormal basis of $S^2V$.

If $a$ and $b$ are Hermitian forms on $V$ we can polarize this identity and get a Hermitian form on $S^2V$ by setting
$$
\displaylines{
(a \odot b)(x \odot y, z \odot w)
= \tfrac 14 \bigl(a(x,y) b(z,w)
+ a(x,w) b(z,y)
\hfill\cr\hfill{}
+ a(z,y) b(x,w)
+ a(z,w) b(x,y)\bigr).
}
$$

\begin{prop}
  If $a$ and $b$ are Hermitian forms on $V$, then $a \odot b$ is a Hermitian form on $S^2V$, and thus defines a K\"ahler curvature tensor. Its derived curvature tensors are
\begin{align*}
H(x) &= \frac{a(x,x)b(x,x)}{h(x,x)^2},
\\
r(x,y) &= \tfrac 14 \bigl(
a(x,y) \langle b, h \rangle
+ (ah^{-1}b)(x,y) + (bh^{-1}a)(x,y)
+ b(x,y) \langle a, h \rangle
\bigr),
\\
s &= \tfrac 12 \bigl(
\langle a, h \rangle \langle b, h \rangle
+ \langle a, b \rangle
\bigr),
\end{align*}
where $h$ is a Hermitian inner product on $V$.

In particular, for an inner product $h$ the tensor $h \odot h$ is Griffiths-positive, has constant holomorphic scalar curvature equal to $1$, its Ricci tensor is $r = \frac{n+1}{2} h$, and its scalar curvature is $s = \frac{n(n+1)}2$.
\end{prop}

\begin{proof}
  Only the claim about the Ricci tensor is not clear. The way to interpret the expression $ah^{-1}b$ is to view $a, b, h$ as linear morphisms $V \to \overline V^*$ and use that $h$ is an isomorphism.

Let's pick an orthonormal basis $(v_1, \ldots, v_n)$ for $h$. The Ricci tensor of $a \odot b$ is then the trace of the form with respect to the induced inner product on $S^2V$. We'll compute the trace $\sum (a\odot b) (x,y,v_j,v_j)$.

Recall that $\sum_j a(v_j,v_j) = \langle a, h \rangle$, where the inner product is the one that $h$ induces on $\Hom(V, \overline V^*)$. This takes care of the first and fourth terms in the trace of the curvature tensor.

Now write $x = \sum_{l} x_l v_l$ and $y = \sum_{k} y_k v_k$. Then
$$
\sum_{j=1}^n a(x,v_j) b(v_j, y)
= \sum_{j,k,l=1}^n x_l \bar y_k a_{lj} b_{jk}
= x^t A I_n B \bar y,
$$
where $A$ and $B$ are the matrices of $a$ and $b$ in the chosen basis. This shows that $\sum a(x,v_j)b(v_j,y) = (ah^{-1}b)(x,y)$, and the remaining term is similar.

For an inner product $h$, we have
$$
(h \odot h)(x \odot y, x \odot y)
= \tfrac12 \bigl(
h(x,x)h(y,y) + |h(x,y)|^2
\bigr),
$$
which is positive for all $x$ and $y$, so $h \odot h$ is Griffiths positive. The other claims are clear.
\end{proof}


\begin{exam}
\label{not-griffiths-positive}
Let $V$ be of dimension $2$, let $h$ be a Hermitian inner product, and let $a$ be given by
$$
A = \begin{pmatrix} 1 & 0 \\ 0 & -1 \end{pmatrix}
$$
in an orthonormal basis. Consider the curvature tensor $R = a \odot a$. Its holomorphic sectional curvature is
$$
H(x) = \frac{a(x,x)^2}{h(x,x)^2} \geq 0,
$$
but its holomorphic bisectional curvature is
$$
B(x,y) = \frac{a(x,x)a(y,y)}{h(x,x)h(y,y)}
$$
which does not have a definite sign (take $x$ such that $a(x,x) < 0$ and $y$ such that $a(y,y) > 0$). Thus the holomorphic sectional curvature does not dominate the holomorphic bisectional curvature.

We have $\langle a, h \rangle = 0$ by construction, and the Ricci tensor of $R$ is
$$
r(x,y) = \tfrac12 h(x, y),
$$
which is positive definite, while the holomorphic sectional curvature is only semipositive. This also shows that Ricci positivity does not imply positivity of the holomorphic bisectional curvature.
\end{exam}


Another way of constructing these kinds of elements is to pull back Hermitian forms to the symmetric product. In that case we get weaker positivity properties than above, simply because any linear morphism we pull back by will have a nontrivial kernel:

\begin{prop}
Let $f \in \Hom(S^2V, V)$ and let $h$ be a Hermitian form on $V$. Then
$$
(f^*h)(x \odot y, z \odot w)
= h(f(x \odot y), f(z \odot w))
$$
is a Hermitian form on $S^2V$, and thus a K\"ahler curvature tensor.

If $h$ is a Hermitian inner product, then $f^*h$ has semipositive holomorphic sectional curvature and is Griffiths-semipositive.
\end{prop}


We claimed that these constructions would exhibit tensors that arise in the wild. For the Kulkarni--Nomizu product this is clear, as the Fubini--Study metric (see Section~\ref{sec:orgcfabeed}) has constant holomorphic sectional curvature and is thus of that form. The same is true for the curvature tensor of the Bergman metric. It's less clear that any ``natural'' K\"ahler metric has a curvature tensor of the pullback form but that does happen; see the metric on the complexification of the K\"ahler cone in Section~\ref{complexified-kahler-cone}.



\subsection{Positivity counterexamples in dimension two}

Let's examine the first nontrivial case, when $\dim V = 2$. We pick an inner product on $V$ and let $(v_1,v_2)$ be an orthonormal basis. Then $(v_1 \odot v_1, v_1 \odot v_2, v_2 \odot v_2)$ is an orthonormal basis of $S^2V$ under the inner product induced by the one we picked. A curvature tensor $R$ identifies with a Hermitian form on $S^2V$, that is, with a Hermitian matrix
$$
h = \begin{pmatrix}
  h_{11} & h_{12} & h_{13} \\
  h_{21} & h_{22} & h_{23} \\
  h_{31} & h_{32} & h_{33}
\end{pmatrix}.
$$
The various derived positivity notions we've seen can now be translated as follows:

\smallskip\noindent$\bullet$\quad
  $R$ has positive holomorphic sectional curvature if $h$ is positive on decomposable tensors $x \odot x$. Write $x = x_1v_1 + x_2v_2$. Then $x \odot x = x_1^2 v_1 \odot v_1 + 2x_1x_2 v_1 \odot v_2 + x_2^2 v_2 \odot v_2$, and
\begin{align*}
H(x)
&= \frac{R(x,x,x,x)}{|x|^4}
\\
&= \begin{pmatrix} \bar x_1^2 & 2\bar x_1\bar x_2 & \bar x_2^2 \end{pmatrix}
\begin{pmatrix}
  h_{11} & h_{12} & h_{13} \\
  h_{21} & h_{22} & h_{23} \\
  h_{31} & h_{32} & h_{33}
\end{pmatrix}
\begin{pmatrix} x_1^2 \\ 2x_1x_2 \\ x_2^2 \end{pmatrix}
\biggm/
(|x_1|^2 + |x_2|^2)^2.
\end{align*}
Calculating the matrix product, we get
\begin{align*}
\begin{pmatrix}
  h_{11} & h_{12} & h_{13} \\
  h_{21} & h_{22} & h_{23} \\
  h_{31} & h_{32} & h_{33}
\end{pmatrix}
\begin{pmatrix} x_1^2 \\ 2x_1x_2 \\ x_2^2 \end{pmatrix}
= \begin{pmatrix}
h_{11} x_1^2 + 2h_{12} x_1x_2 + h_{13} x_2^2
\\
h_{21} x_1^2 + 2h_{22} x_1x_2 + h_{23} x_2^2
\\
h_{31} x_1^2 + 2h_{32} x_1x_2 + h_{33} x_2^2
\end{pmatrix}
\end{align*}
so
\begin{align*}
R(x,x,x,x)
&=
(h_{11} x_1^2 + 2h_{12} x_1x_2 + h_{13} x_2^2)\bar x_1^2
\\
&\qquad{}+
(h_{21} x_1^2 + 2h_{22} x_1x_2 + h_{23} x_2^2)2 \bar x_1 \bar x_2
\\
&\qquad{}+
(h_{31} x_1^2 + 2h_{32} x_1x_2 + h_{33} x_2^2) \bar x_2^2.
\end{align*}

\smallskip\noindent$\bullet$\quad
The Ricci tensor of $R$ is
\begin{align*}
  r(x,y)
  &= R(x,y,v_1,v_1) + R(x,y,v_2,v_2)
  \\
  &= h(x \odot v_1, y \odot v_1) + h(x \odot v_2, y \odot v_2).
\end{align*}
Write $x = x_1v_1 + x_2v_2$ and $y = y_1 v_1 + y_2v_2$. Then
\begin{align*}
h\bigl((x_1 v_1 + x_2v_2) \odot v_1, (y_1v_1 + y_2v_2) \odot v_1 \bigr)
&= x_1\bar y_1 h_{11} + x_1\bar y_2 h_{12} + x_2\bar y_1 h_{21} + x_2\bar y_2 h_{22}
\\
h\bigl((x_1 v_1 + x_2v_2) \odot v_2, (y_1v_1 + y_2v_2) \odot v_2 \bigr)
&= x_1\bar y_1 h_{22} + x_1\bar y_2 h_{23} + x_2\bar y_1 h_{32} + x_2\bar y_2 h_{33}
\end{align*}
so
$$
r(x,y)
= \begin{pmatrix}\bar y_1 & \bar y_2\end{pmatrix}
  \begin{pmatrix}
  h_{11} + h_{22}  & h_{21} + h_{32}
    \\
  h_{12} + h_{23} & h_{22} + h_{33}
  \end{pmatrix}
  \begin{pmatrix}x_1 \\ x_2 \end{pmatrix}.
$$


\smallskip\noindent$\bullet$\quad
The scalar curvature of $R$ is the trace of the Ricci tensor, so
$$
s = h_{11} + 2 h_{22} + h_{33}.
$$

\smallskip
Consider a diagonal matrix $h$. The holomorphic sectional curvature of the corresponding tensor $R$ is
$$
H(x,y)
= \frac{h_{11} |x|^4 + 4 h_{22} |x|^2|y|^2 + h_{33}|y|^4}{(|x|^2+|y|^2)^2},
$$
its Ricci tensor is
$$
r(x,y) = (h_{11} + h_{22})|x|^2 + (h_{22} + h_{33})|y|^2
$$
and its scalar curvature is
$$
s = h_{11} + 2h_{22} + h_{33}.
$$
We can now setup a variety of situations to show that positivity properties only ``flow downwards''.

\begin{exam}
Take $h_{11} > 0$, $h_{33} < 0$ and set $2h_{22} + h_{33} = 0$. Then we get a tensor such that the scalar curvature is positive, but the Ricci tensor has no definite sign.
\end{exam}

\begin{exam}
Note that
$$
H(x,x) = \tfrac 14 \bigl(h_{11} + 4h_{22} + h_{33} \bigr).
$$
We can then clearly pick coefficients such that $s > 0$ but $H(x,x) < 0$. Thus positive scalar curvature does not imply positive holomorphic sectional curvature.
\end{exam}

\begin{exam}
If we pick coefficients such that $h_{11} + h_{22} > 0$ and $h_{22} + h_{33} > 0$ but $h_{11} + 4h_{22} + h_{33} < 0$ we then have a tensor with $r > 0$ everywhere but $H < 0$ in some directions.
\end{exam}

\begin{exam}
The term that controls the sign of the holomorphic sectional curvature can be written as (picking $h_{11} = 1$ to simplify)
$$
(|x|^2 - \sqrt{h_{33}}|y|^2)^2 + (4h_{22} + 2\sqrt{h_{33}})|x|^2|y|^2.
$$
The corresponding Ricci tensor is
$$
r(x,y) = (1 + h_{22})|x|^2 + (h_{22} + h_{33}) |y|^2.
$$
Set $h_{22} = -\sqrt{h_{33}}/2$. Then $H \geq 0$, but
$$
r(x,y) = (1-\sqrt{h_{33}}/2)|x|^2 + (h_{33} - \sqrt{h_{33}}/2)|y|^2
$$
and taking $h_{33} > 4$ yields a Ricci tensor that is negative in some directions. Taking $h_{22} = -\sqrt{h_{33}}/2 + \varepsilon$ then gives $H > 0$ everywhere but $r < 0$ somewhere.
\end{exam}

We already saw that having $r > 0$ does not imply the holomorphic bisectional curvature is positive, and that $H \geq 0$ does not imply that either (Example~\ref{not-griffiths-positive}).


\begin{exam}
Let $V$ be of dimension two with a Hermitian form $h$ and orthonormal basis $(e_1,e_2)$. We consider the curvature tensor defined by
$$
R(x,y,z,w) = \tfrac12( h(x,y)h(z,w) + h(x,w)h(z,y)).
$$
It can be viewed as a Hermitian form both on $V \otimes V$ and $S^2V$. These spaces have orthonormal bases $(e_1 \otimes e_1, e_1 \otimes e_2, e_2 \otimes e_1, e_2 \otimes e_2)$ and $(e_1 \odot e_1, e_1 \odot e_2, e_2 \odot e_2)$, and the matrices of the Hermitian forms defined by $R$ in these bases are
$$
A = \begin{pmatrix}
  1 & 0 & 0 & 0 \\
  0 & \tfrac12 & \tfrac12 & 0 \\
  0 & \tfrac12 & \tfrac12 & 0 \\
  0 & 0 & 0 & 1
\end{pmatrix}
\qandq
B = \begin{pmatrix}
  1 & 0 & 0 \\
  0 & 1 & 0 \\
  0 & 0 & 1
\end{pmatrix}.
$$
The form on $S^2V$ is positive-definite, but the one on $V \otimes V$ has a one-dimensional kernel and is positive on its complement. The tensor $R$ is thus Griffiths positive and Nakano semipositive. This is the most positivity one can hope for in lieu of Corollary~\ref{no-nakano-positive}.

The tensor $R$ is the curvature tensor of the Fubini--Study metric on the projective space, as we will see later.
\end{exam}



\begin{exam}
  Sticking with the same setup as in the last example, we consider the Hermitian form $b$ on $V \otimes V$ defined by everyone's favorite matrix
$$
\begin{pmatrix}
  1 & 0 & 0 & 0 \\
  0 & 1 & 0 & 0 \\
  0 & 0 & 1 & 0 \\
  0 & 0 & 0 & 1
\end{pmatrix}.
$$
The associated curvature tensor $R$ is Nakano positive, but is of course not the curvature tensor of a K\"ahler metric. The reader can check that, for example, $R(v_1,v_2,v_2,v_1) \not= R(v_1,v_2,v_1,v_2)$.
\end{exam}


\section{Flat metrics}
\label{sec:org504b250}

Let $(E,h) \to X$ be a holomorphic Hermitian vector bundle. The curvature form $\Theta$ of $h$ is a smooth section of $\bigwedge^{1,1}T_X \otimes \End E$. The simplest such section is of course the zero section. We say that the metric $h$ is \emph{flat} if its curvature form is zero.


TODO: Show that the universal covering of a flat manifold is a complex Lie group.

TODO: Show that a compact complex Lie group is commutative, i.e., a torus. Any compact flat manifold is thus K\"ahler.


\subsection{Euclidean metric}

The prime example of such a metric is the standard Euclidean metric on any open set $U \subset \kk C^n$. The metric $h$ defined by the usual inner product,
$$
h(\xi, \eta) = \sum_{j=1}^n \xi_j \overline{\eta_j}.
$$
The associated K\"ahler form is
$$
\omega = \sum_{j=1}^n \frac{i}{2} dz_j \wedge d\bar z_j.
$$
This is clearly $d$-closed, so the Euclidean metric is K\"ahler.\footnote{It has a global potential $\|z\|^2$. This is seldom useful.} We have
$$
h(D\xi, \eta) + h(\xi, D\eta)
= d h(\xi, \eta)
= \sum_{j=1}^n d\xi_j \otimes \overline{\eta_j} + \xi_j \otimes \overline{d\eta_j}
$$
so the Chern connection of $h$ is equal to the exterior derivative.

% TODO: Is any of the following relevant to examples of curvature tensors?
% Do we want to only do examples or also have a review of some things that are known about curvature?

\subsection{Compact manifolds}

There are also compact manifolds with flat metrics. Let $\Lambda \subset \kk C^n$ be a lattice, that is, an abelian group of rank $2n$. Then $X := \kk C^n / \Lambda$ is a compact complex manifold, called a \emph{complex torus}. As $\Lambda$ acts by translations on $\kk C^n$, it preserves the Euclidean metric, which descends to $X$.

One can ask whether the converse holds: If $X$ is a compact complex manifold with a flat metric, then is $X$ a complex torus?

The answer is subtle. First, the proposed answer cannot be true, as some complex tori admit subgroups that act freely, and whose quotients are thus again compact complex manifolds with flat metrics that are not tori. In the K\"ahler case, that is the extent of the problems that can arise. We can see this quickly if we're prepared to nuke a mosquito:

Suppose $X$ is a compact K\"ahler manifold that admits a flat metric. All of its Chern classes are then zero, in particular the first one. By \cite{beauville1983}, the universal covering of $X$ splits into a product of $\kk C^n$ and compact manifolds with zero first Chern class. The vanishing of all Chern classes of $X$ implies the product only involves $\kk C^n$.

For Hermitian manifolds, the answer is slightly more complicated: A flat Hermitian manifold is covered by a simply connected complex Lie group~\cite{boothby1958}.


\subsection{Gauss--Manin}

% TODO: I'm not sure this should be here at all.

Let $\pi : X \to S$ be a family of compact K\"ahler manifolds over a connected smooth base $S$. All of the manifolds $X_s := \pi^{-1}(s)$ in the family are diffeomorphic, so they all have isomorphic cohomology groups $H^k(X_s, \kk Z)$. These can be assembled into a holomorphic vector bundle $E \to S$ (by taking direct images of constant sheaves) whose fibers are exactly
$$
E_s = H^k(X_s, \kk C).
$$
The construction also yields a connection $\nabla$ on this vector bundle, called the \emph{Gauss--Manin connection}. It is flat, and if $\alpha$ is a local section of $E$ and $Y \subset X_s$ a $2k$-dimensional submanifold, then
$$
d \int_{Y} \alpha = \int_{Y} \nabla \alpha.
$$
The Gauss--Manin connection is generally not the Chern connection of a metric on $E$.

Something mildly interesting happens in the middle cohomology of $4n$-dimensional manifolds. Then there is an intersection form on $2n$-forms, which yields a sesquilinear form
$$
b(\alpha, \beta) = \int_{X_s} \alpha \wedge \overline{\beta}.
$$
This form is non-degenerate, but has mixed signature. One can still define connections for such forms, and the usual proof of the existance and uniqueness of the Chern connection carries through for them. One can check that the Gauss--Manin connection is the Chern connection of the intersection form in this sense.\footnote{This is where we use that the forms are of even degree to show compatibility with the form $b$; otherwise there's an extra $(-1)^n$ involved.}
This does not appear to have any applications.



\section{Conformal metrics}
\label{sec:org65fcbad}

Let \((E,h) \to X\) be a holomorphic Hermitian vector bundle, and let \(D\) be its Chern connection.

If \(f\) is a smooth real-valued function on \(X\), then \(h' := e^f h\) is again a Hermitian metric, \emph{conformal} to \(h\). The formulas for the Chern connection and curvature of conformal metrics are less complicated than the equivalent formulas for Riemannian metrics. We have
$$
h'(D_{h'} s, t) + h'(s, D_{h'} t)
= d h'(s, t)
= df \otimes h'(s, t) + h'(D_h s, t) + h'(s, D_h t).
$$
Writing \(df = \partial f + \bar\partial f\) we see that the Chern connection of \(h'\) is
$$
D_{h'} s = \partial f \otimes s + D_h s.
$$
Using the expressions for the covariant exterior derivative, we also have
\begin{align*}
D_{h'}^2 s
&= D_{h'}(\partial f \otimes s + D_h s)
\\
&= d(\partial f) \otimes s - \partial f \wedge D_{h'} s + D_{h'}(D_h s)
\\
&= -\partial\bar\partial f \otimes s - \partial f \wedge (\partial f \otimes s + D_h s) + \partial f \wedge D_h s + D_h^2 s
\\
&= -\partial\bar\partial f \otimes s + D_h^2 s.
\end{align*}

The Chern curvature of \(h'\) is thus
$$
D^2_{h'} = -\partial\bar\partial f \otimes \id_E + D^2_h.
$$

It's fun to work out what this gives for metrics conformal to a flat metric. We'll do that later when we study a particular metric on the \hyperref[sec:org8f5818e]{Hopf manifold}.

\subsection{Conformal to K\"ahler is not K\"ahler}
\label{sec:org7b1cfdf}

It's worth mentioning that if \(h\) is a K\"ahler metric on a manifold of dimension \(n > 1\) and \(f\) is non-constant, then \(h'\) is not a K\"ahler metric. The reason is that if \(\omega\) is the symplectic form associated to \(h\), then the symplectic form of \(h'\) is \(e^f \omega\) and
$$
d(e^f \omega) = \omega \wedge (e^f df)
$$
and the linear morphism from one- to three-forms defined by wedging with the symplectic form \(\omega\) is injective.\footnote{The hard Lefschetz theorem generalizes this to cohomology.}
The proof reduces to linear algebra by a calculation in local coordinates, either Darbeaux ones or in holomorphic ones that are orthonormal at a point.


\section{Riemann surfaces}
\label{sec:org776713b}

In Riemannian geometry, the full curvature tensor doesn't appear until in dimension $4$. Before that, the low dimensions of the tangent space restrict what tensors can appear. The equivalent situation in complex geometry only occurs in complex dimension one, on Riemann surfaces. Complex dimension two is real dimension four and already contains all the complexities of the higher dimensions. Conversely, complex dimension one is only real dimension two and too simple to give a good idea of what happens in higher dimensions.

\subsection{Everything is K\"ahler}
Let $U \subset \kk C$ be an open set. We can imagine it is a small neighborhood around a point in a Riemann surface we care about. Let $h$ be a Hermitian metric on $U$. The K\"ahler form of the metric can be written as
$$
\omega = e^{f(z)} \frac{i}{2} dz \wedge d\bar z
$$
on $U$. Then $d\omega = 0$ for dimensional reasons, so any Hermitian metric on $U$ is K\"ahler.

\subsection{Curvature}
By the above, we also have $h = e^f h_{\mathrm{std}}$, so the Chern connection and curvature of the metric can be deduced from our discussion on conformal metrics. Notably, the curvature form is
$$
\Theta
= -\frac i2\partial\bar\partial f
= -\frac{1}{e^f}\frac{\partial^2f}{\partial z \partial \bar z} \; e^f \frac{i}{2} dz \wedge d\bar z
= -\Delta_\omega f \; \omega.
$$
There should be a tensor product with the identity map on $T_X$ in the first equality, but the identity map on a one-dimensional space is just multiplication by $1$ so we can skip it.

One can note that an endomorphism on a one-dimensional space can be identified with its trace. From that point of view, one can say that the curvature form in complex dimension one can be identified with the form one gets by taking the trace of the curvature endomorphisms. In some sense, the full curvature form collapses to the Ricci form. In some other sense, the above computations imply the Ricci form only contains the scalar curvature. There's not enough space for complex curvature tensors.


\subsection{Poincar\'e disk}

Let $D = \{z \in \kk C \mid |z| < 1\}$ be the unit disk in the complex plane. We can pull a Hermitian metric out of our hat by setting
$$
\omega = \frac 1{(1-|z|^2)^2} \frac i2 dz \wedge d\bar z.
$$
As
$$
\frac 1{(1-|z|^2)^2} = e^{-2\log(1-|z|^2)}
$$
the curvature form of $\omega$ is
\begin{align*}
2\frac i2 \partial\bar\partial \log(1-|z|^2)
&= 2\frac i2 \partial \frac{-z d\bar z}{1-|z|^2}
\\
&= 2\frac i2 \frac{- dz \wedge d\bar z}{1-|z|^2}
- 2\frac i2 \frac{-\bar z dz}{1-|z|^2} \wedge \frac{-zd\bar z}{1-|z|^2}
\\
&= 2\biggl(\frac{-1}{1-|z|^2} - \frac{|z|^2}{(1-|z|^2)^2} \biggr) \frac i2 dz \wedge d\bar z
\\
&= -2 \omega.
\end{align*}
The Poincar\'e metric is the first negatively curved K\"ahler metric we see. Most Riemann surfaces are negatively curved. Compact ones can be organized according to their genus, which is one-half of the dimension of their first homology group. Any compact Riemann surface of genus zero is isomorphic to the projective line and so is positively curved; a Riemann surface of genus one is a torus and thus flat; and any other Riemann surface is covered by the unit disk and negatively curved.



\section{Projective space}
\label{sec:orgcfabeed}

The complex projective space is the space of lines in a given complex vector space. That is, if $V$ is a complex vector space, then we define the projective space as the set
$$
\kk P(V) := \{ v \in V \mid v \not= 0 \} / \kk C^*,
$$
where $\kk C^*$ acts by multiplication. If $V$ is the zero space, then $\kk P(V) = \varnothing$ by this definition; if $V$ is a line, then $\kk P(V)$ is a point. People usually take care not to be in either of those cases.

There is a projection map $\pi : V \setminus \{0\} \to \kk P(V)$ and we equip $\kk P(V)$ with the quotient topology. If we fix a Hermitian inner product $h$ on $V$, then the quotient map factors through the unit sphere:
$$
V \setminus \{0\} \to S(V, h) \to \kk P(V).
$$
As the unit sphere is compact, it follows that the projective space $\kk P(V)$ is compact.


\subsection{Manifold structure}

We're going to construct local holomorphic charts on $\kk P(V)$. Let $\lambda \in V^* \setminus \{0\}$ and define
$$
U_\lambda := \{ [v] \in \kk P(V) \mid \lambda(v) \not= 0 \}.
$$
This is an open set in $\kk P(V)$ equipped with the quotient topology, as its preimage under the projection is the complement of the hyperplane $\{v \in V \mid \lambda(v) = 0\}$, which is open.

We define a map $f_\lambda: U_\lambda \to V$ by setting
$$
f([v]) = v/\lambda(v).
$$
This is well defined by the linearity of $\lambda$ and by definition of $U_\lambda$. This map takes values in the affine hyperplane
$$
H_\lambda := \{ v \in V \mid \lambda(v) = 1 \}.
$$
It is in fact a bijection onto this set: If $v \in H_\lambda$ then $[v]$ is an element of $U_\lambda$ that maps to $v$, so $f_\lambda$ is surjective. If $[v], [w] \in U_\lambda$ are such that $f_\lambda(v) = f_\lambda(w)$, then $v/\lambda(v) = w/\lambda(w)$ for any representatives $v, w$ of those classes. Then $v = (\lambda(v)/\lambda(w)) w$, so $[v] = [w]$ and $f_\lambda$ is injective.

The map $f_\lambda$ is continuous: Let $U \subset H_\lambda$ be open. Then $\pi^{-1}(f_\lambda^{-1}(U)) = \kk C^* \cdot U$, which is open. Its inverse is also continuous: The map $f_\lambda^{-1}$ is just the restriction of $\pi$ to $H_\lambda$. Thus $f_\lambda$ is a homeomorphism.

The collection $(f_\lambda : U_\lambda \to H_\lambda)_{\lambda \in V^* \setminus \{0\}}$ covers $\kk P(V)$ and provides local homeomorphisms to spaces biholomorphic to $\kk C^{\dim V - 1}$. If $\lambda$ and $\lambda'$ are two nonzero elements of the dual space, then the transition map between charts is
$$
f_{\lambda'} \circ f_{\lambda}^{-1} : \{v \in H_\lambda \mid \lambda'(v) \not= 0 \} \to H_{\lambda'},
\quad
v \mapsto v/\lambda'(v).
$$
This map is a composition of field operations and a linear map, so it is holomorphic. The collection above thus forms a holomorphic atlas.


\subsection{Tautological bundle}


If we consider the trivial vector bundle $V \to \kk P(V)$, then the definition of projective space gives a line bundle $\cc O(-1) \subset V$ whose fiber over a point $[v]$ is
the line $\kk C \cdot v$. This \emph{tautological line bundle} is holomorphic:

On a local chart $H_\lambda$, the line bundle is given by $\kk C \cdot v \subset V$. It is trivialized by the Euler section $\xi(v) = v$, which is certainly holomorphic on the local chart. Changing coordinates to ones defined by $\lambda'$ multiplies the section by $1/\lambda'(\xi)$, which is holomorphic.



\subsection{Curvature of tautological bundle}

Fix a Hermitian inner product $h$ on $V$. This defines a flat Hermitian metric on the trivial vector bundle $V \to \kk P(V)$. It follows that the induced metric on $\cc O(-1)$ is non-positive.

On $H_\lambda$, we have the Euler section $\xi$ of $\cc O(-1)$ given by $\xi(v) = v$. The curvature form of the line bundle is then
$$
-\frac i2 \partial \bar\partial \log h(\xi, \xi)
= -\frac i2\partial \frac{h(\xi, \partial \xi)}{h(\xi, \xi)}
= -\frac i2\frac{h(\partial \xi, \partial \xi)}{h(\xi, \xi)}
+ \frac i2\frac{h(\partial \xi, \xi)}{h(\xi, \xi)} \wedge \frac{h(\xi, \partial \xi)}{h(\xi, \xi)}.
$$
The Euler field satisfies $\partial_\alpha \xi = \alpha$ for holomorphic vector fields $\alpha$. The sesquilinear form defined by the curvature form is then
$$
\phi(\alpha, \beta)
= -\frac{h(\alpha, \beta)}{h(\xi, \xi)}
+ \frac{h(\alpha, \xi)}{h(\xi, \xi)} \cdot \frac{h(\xi, \beta)}{h(\xi, \xi)}.
$$
We claim that this is negative-definite. Cauchy--Schwarz gives
$$
\phi(\alpha, \alpha)
= -\frac{h(\alpha, \alpha)}{h(\xi, \xi)}
+ \frac{|h(\alpha, \xi)|^2}{h(\xi, \xi)^2}
\leq -\frac{h(\alpha, \alpha)}{h(\xi, \xi)}
+ \frac{h(\alpha, \alpha) h(\xi, \xi)}{h(\xi, \xi)^2}
= 0
$$
with equality if and only if $\alpha$ is a multiple of the Euler field $\xi$. But note that $H_\lambda$ is defined so that $\lambda(\xi) = 1$, while its tangent space identifies with the set of vectors $\alpha$ such that $\lambda(\alpha) = 0$. Therefore $\alpha$ is not a multiple of $\xi$.


\subsection{Fubini--Study metric}

The dual of the tautological line bundle is denoted by $\cc O(1) := \cc O(-1)^*$. By the above, it is a positive line bundle on the projective space $\kk P(V)$. Its curvature form $\omega$ is called the \emph{Fubini--Study metric} on the projective space. It is a K\"ahler metric. In local coordinates it is given by the Hermitian form $\psi := -\phi$.

We want to compute the Chern connection of this metric. By definition, it satisfies $\partial \psi(\alpha,\beta) = \psi(D\alpha,\beta)$ for holomorphic tangent fields $\alpha$ and $\beta$, and it is enough to know the connection on such fields. We have
\begin{align*}
\partial \psi(\alpha,\beta)
&= \frac{h(\partial\alpha, \beta)}{h(\xi,\xi)}
- \frac{h(\partial\xi, \xi)}{h(\xi,\xi)}
 \frac{h(\alpha,\beta)}{h(\xi,\xi)}
\\
&\qquad{}
- \frac{h(\partial\alpha,\xi)}{h(\xi,\xi)}
\frac{h(\xi,\beta)}{h(\xi,\xi)}
+ \frac{h(\partial\xi,\xi)}{h(\xi,\xi)}
\frac{h(\alpha,\xi)}{h(\xi,\xi)}
\frac{h(\xi,\beta)}{h(\xi,\xi)}
\\
&\qquad{}
- \frac{h(\alpha,\xi)}{h(\xi,\xi)} \frac{h(\partial\xi,\beta)}{h(\xi,\xi)}
+\frac{h(\alpha, \xi)}{h(\xi,\xi)}
\frac{h(\partial\xi,\xi)}{h(\xi,\xi)}
\frac{h(\xi,\beta)}{h(\xi,\xi)}
\\
&= \psi(\partial\alpha,\beta)
- \psi\biggl(\frac{h(\alpha,\xi)}{h(\xi,\xi)} \partial\xi, \beta \biggr)
- \psi\biggl(\frac{h(\partial\xi,\xi)}{h(\xi,\xi)} \alpha, \beta \biggr)
\end{align*}
so the Chern connection is
$$
 D \alpha
= d\alpha
- \frac{h(\alpha,\xi)}{h(\xi,\xi)} \partial\xi
- \frac{h(\partial\xi,\xi)}{h(\xi,\xi)} \alpha.
$$
For the curvature form, we then have
\begin{align*}
\frac i2\bar\partial D\alpha
&= -\frac i2 \partial\bar\partial \alpha
- \frac i2 \frac{h(\alpha,\partial\xi)}{h(\alpha,\xi)} \wedge \partial \xi
+ \frac i2 \frac{h(\alpha,\xi)}{h(\xi,\xi)} \frac{h(\xi,\partial\xi)}{h(\xi,\xi)} \wedge \partial\xi
\\
&\qquad{}
+ \frac i2\frac{h(\partial\xi,\partial\xi)}{h(\xi,\xi)} \alpha
- \frac i2\frac{h(\partial\xi,\xi)}{h(\xi,\xi)} \wedge \frac{h(\xi,\partial\xi)}{h(\xi,\xi)} \alpha
\\
&= - \psi(\alpha, \partial\xi) \wedge \partial \xi
+ \psi(\partial\xi, \partial\xi) \alpha.
\end{align*}
We have to be careful about signs here, as $\partial \xi \wedge \bar\partial \xi = - \bar\partial \xi \wedge \partial \xi$. Note also the sign flip when we commute $\partial\xi$ and $\bar\partial\xi$ below, where we conclude that the curvature tensor is
$$
R(\alpha,\beta,\gamma,\delta)
= h(\tfrac i2\Theta_{\alpha \beta} \gamma, \delta)
= \psi(\alpha, \beta) \psi(\gamma, \delta)
+ \psi(\alpha, \delta) \psi(\gamma, \beta).
$$
The extra symmetries
$$
R(\alpha,\beta,\gamma,\delta) = R(\gamma,\beta,\alpha,\delta)
\quad\text{and}\quad
R(\alpha,\beta,\gamma,\delta) = R(\alpha,\delta,\gamma,\beta)
$$
that a K\"ahler curvature tensor has compared to a Hermitian curvature tensor are very clear here.


The holomorphic bisectional curvature is
$$
B(\alpha,\beta) = \frac{R(\alpha,\alpha,\beta,\beta)}{\phi(\alpha,\alpha)\phi(\beta,\beta)}
= \frac{\phi(\alpha,\alpha)\phi(\beta,\beta)+|\phi(\alpha,\beta)|^2}{\phi(\alpha,\alpha)\phi(\beta,\beta)}.
$$
We have
$$
1 \leq B(\alpha,\beta) \leq 2
$$
by Cauchy--Schwarz. The holomorphic sectional curvature is
$$
H(\alpha) = B(\alpha,\alpha) = 2.
$$
To find the Ricci curvature we pick a holomorphic frame $(\zeta_1, \ldots, \zeta_n)$ that's orthonormal at a point $z$. There we have
$$
r(\alpha,\beta)
= \sum_{j=1}^n R(\alpha,\beta,\zeta_j,\zeta_j)
= \sum_{j=1}^n \phi(\alpha,\beta) + \phi(\alpha,\zeta_j)\phi(\zeta_j,\beta)
= (n+1) \phi(\alpha,\beta).
$$
The Fubini--Study metric is thus a K\"ahler--Einstein metric. Contracting the Ricci tensor we find that its scalar curvature is
$$
s = n(n+1).
$$




\section{Bergman metric}
\label{sec:org21fa1aa}

We're going to look at a special case of the \emph{Bergman metric}. Let $B = \{ z \in \kk C^n \mid |z|<1 \}$ be the unit ball. We define a $(1,1)$-form on $B$ by $\omega = \frac i2 \partial \bar \partial \log f$, where
$$
f(z) = \frac{1}{1-|z|^2}.
$$
As the definition suggests, this is the curvature form of a line bundle on $B$. In general, the Bergman metrics are the curvature forms of Hermitian metrics on the canonical bundle of a manifold, but we are going to skip that part of the theory here.

We have
$$
\bar\partial \log f
= - \bar\partial \log (1 -|z|^2)
= \frac{\langle z, dz\rangle}{1 - |z|^2}
$$
so
$$
\omega
= \frac i2 \partial\bar\partial \log f
= \frac{\frac i2 \langle dz, dz\rangle}{1-|z|^2}
+ \frac i2 \frac{\langle dz, z \rangle}{1-|z|^2}\wedge \frac{\langle z, dz \rangle}{1-|z|^2}.
$$
The Hermitian form associated to $\omega$ is thus
$$
h(\alpha,\beta)
=\frac{\langle \alpha, \beta\rangle}{1-|z|^2}
+ \frac{\langle \alpha, z \rangle}{1-|z|^2}
\cdot \frac{\langle z, \beta \rangle}{1-|z|^2}.
$$
As before, we recall that $|\langle \alpha, z \rangle| \leq |\alpha| |z|$, so
$$
h(\alpha,\alpha)
\geq \frac{|\alpha|^2}{1-|z|^2}
+ \frac{|\alpha|^2|z|^2}{(1-|z|^2)^2}
= \frac{|\alpha|^2}{(1-|z|^2)^2}
> 0
$$
if $\alpha \not= 0$, so $h$ is a K\"ahler metric.

To compute the Chern connection of $h$, we note that for holomorphic tangent fields we have
\begin{align*}
\partial_\gamma h(\alpha,\beta)
&= \frac{\langle d_\gamma\alpha, \beta\rangle}{1-|z|^2}
+ \frac{\langle \gamma, z \rangle}{1-|z|^2}
\cdot \frac{\langle \alpha, \beta\rangle}{1-|z|^2}
\\
&\qquad{}
+ \frac{\langle d_\gamma \alpha, z \rangle}{1-|z|^2}
\cdot \frac{\langle z, \beta \rangle}{1-|z|^2}
+ \frac{\langle \gamma, z \rangle}{1-|z|^2}
\cdot \frac{\langle \alpha, z\rangle}{1-|z|^2}
\cdot \frac{\langle z, \beta \rangle}{1-|z|^2}
\\
&\qquad{}
+ \frac{\langle \alpha, z\rangle}{1-|z|^2}
\cdot \frac{\langle \gamma, \beta \rangle}{1-|z|^2}
+
\frac{\langle \alpha, z\rangle}{1-|z|^2}
\cdot \frac{\langle \gamma, z \rangle}{1-|z|^2}
\cdot \frac{\langle z, \beta \rangle}{1-|z|^2}
\\
&=
h(d_\gamma \alpha, \beta)
+ h\biggl( \frac{\langle \gamma, z \rangle}{1-|z|^2} \alpha , \beta\biggr)
+ h\biggl( \frac{\langle \alpha, z \rangle}{1-|z|^2} \gamma , \beta\biggr)
\end{align*}
so the Chern connection of $h$ is
$$
D \alpha
= d \alpha
+ \frac{\langle dz, z \rangle}{1-|z|^2} \alpha
+ \frac{\langle \alpha, z \rangle}{1-|z|^2} dz.
$$
To find the curvature form, we then calculate
$$
\displaylines{
\bar\partial D \alpha
=
-\frac{\langle dz, dz \rangle}{1-|z|^2} \alpha
- \frac{\langle dz, z \rangle}{1-|z|^2}
\wedge \frac{\langle z, dz \rangle}{1-|z|^2} \alpha
\hfill\cr\hfill{}
+ \frac{\langle \alpha, dz \rangle}{1-|z|^2} \wedge dz
+ \frac{\langle \alpha, z \rangle}{1-|z|^2}
\cdot \frac{\langle \alpha, dz \rangle}{1-|z|^2} \wedge dz.
}
$$
After permuting some fields, taking the inner product with a fourth tangent field, and minding the order of wedge products, this results in the curvature tensor
$$
R(\alpha,\beta,\gamma,\delta)
= -h(\alpha,\beta) h(\gamma,\delta)
- h(\alpha,\delta) h(\gamma,\beta).
$$
By inspection, we recognize this as the curvature tensor of a metric with constant holomorphic sectional curvature $-2$. It follows that this is a K\"ahler--Einstein metric with $r(\alpha,\beta) = - (n+1) h(\alpha,\beta)$, and that its scalar curvature is $-n(n+1)$.


\section{TODO: Grassmannian}
\label{sec:org34425b6}

\section{TODO: Flag manifold}
\label{sec:orga0ef6a0}

These are K\"ahler. There is a ``natural'' non-K\"ahler metric on them.

\section{Hopf manifold}
\label{sec:org8f5818e}

Let \(\lambda \in \kk C\) be a complex number such that \(0 < |\lambda| < 1\). The \emph{Hopf manifold} is the quotient
$$
X := (\kk C^n \setminus \{0\}) / \Gamma,
$$
where \(\Gamma \cong \kk Z\) is the group generated by \(\lambda\) that acts by
$$
\lambda \cdot (z_1, \ldots, z_n) = (\lambda z_1, \ldots, \lambda z_n).
$$
The Hopf manifold is compact and is diffeomorphic to \(S^{2n-1} \times S^1\). In particular,
$$
H^2(X, \kk C) \cong H^2(S^{2n-1}, \kk C) \oplus H^1(S^{2n-1}, \kk C) \otimes H^1(S^1, \kk C) = 0,
$$
so it is not K\"ahler.

Let \(\pi : \kk C^n \setminus \{0\} \to X\) be the projection. If \(h\) is a Hermitian metric on \(X\), then its pullback \(\pi^*h\) is a Hermitian metric on \(\kk C^n \setminus \{0\}\) that is invariant under the action of \(\Gamma\). If we write its K\"ahler form as \(\sum_{j,k} a_{jk}(z) \tfrac{i}{2} dz_j \wedge d\bar z_k\), then the smooth functions \(a_{jk}\) must satisfy
$$
a_{jk}(\lambda z) = \frac{1}{|\lambda|^2} a_{jk}(z).
$$
We can pick one such metric to inspect; we'll choose \(h = \frac{1}{\|z\|^2} h_{\mathrm{std}}\), that is, a metric that is conformal to the standard metric on \(\kk C^n \setminus \{0\}\). The Ph.D. thesis \cite{istrati:tel-02156198} has a nice discussion of the history of these metrics on the Hopf manifold.

As the metric is conformal to a K\"ahler metric, and the conformal factor is non-constant, the metric is not K\"ahler. (We already knew this because \emph{no} metric on the Hopf manifold is K\"ahler, but it's nice to check.)


\subsection{Curvature tensor}
\label{sec:org96d544d}

We've \hyperref[sec:org65fcbad]{already computed} the curvature of a conformal metric, so we know the curvature form of the metric $h$ is
$$
D^2 s = \partial\bar\partial \log \|z\|^2 \otimes s.
$$
Let's compute this and express the curvature tensor of the metric. To do that we'll use the Euler field
$$
\xi = \sum_{j=1}^n z_j \frac{\partial}{\partial z_j}.
$$
It is a holomorphic tensor field whose norm is $\|\xi\|^2 = \|z\|^2$ and satisfies $\partial_\alpha \xi = \alpha$ for holomorphic tensor fields $\alpha$. We have
$$
\bar\partial \log \|z\|^2
= \frac{\langle \xi, \partial\xi \rangle}{\langle \xi,\xi \rangle}
$$
and
$$
\partial\bar\partial \log \|z\|^2
= \frac{\langle \partial \xi, \partial \xi \rangle}{\langle \xi, \xi \rangle}
- \frac{\langle \partial\xi, \xi\rangle}{\langle \xi, \xi \rangle} \wedge \frac{\langle \xi, \partial \xi \rangle}{\langle \xi, \xi \rangle}
= h(\partial\xi, \partial\xi) - h(\partial\xi, \xi) \wedge h(\xi, \partial\xi).
$$
The curvature tensor of the metric \(h\) on the Hopf manifold is then
$$
R(\alpha,\beta,\gamma,\delta)
= h(\alpha, \beta) h(\gamma, \delta)
- h(\alpha, \xi) h(\xi, \beta) h(\gamma, \delta).
$$
We note that it has the expected conjugate symmetries, that is, that \(R(\beta, \alpha, \delta, \gamma) = \overline{R(\alpha, \beta, \gamma, \delta)}\), but \(R(\gamma, \delta, \alpha, \beta) \not= R(\alpha, \beta, \gamma, \delta)\) like it would if this were the curvature tensor of a K\"ahler metric.


\subsection{Holomorphic sectional curvature}
\label{sec:org6471503}

The holomorphic bisectional curvature of the metric is
$$
B(\alpha,\beta)
= \frac{R(\alpha,\beta,\alpha,\beta)}{\|\alpha\|^2\|\beta\|^2}
= \frac{|h(\alpha,\beta)|^2 - h(\alpha,\xi)h(\xi,\beta)h(\alpha,\beta)}{\|\alpha\|^2\|\beta\|^2}.
$$
The holomorphic sectional curvature of the Hopf manifold is
$$
H(\alpha)
= B(\alpha,\alpha)
= \frac{\|\alpha\|^4 - |h(\alpha,\xi)|^2 \|\alpha\|^2}{\|\alpha\|^4}
= \frac{\|\alpha\|^2 - |h(\alpha,\xi)|^2}{\|\alpha\|^2}.
$$
We have the bounds
$$
0 \leq H(\alpha) \leq 1
$$
that are obtained when $\alpha$ is a multiple of $\xi$ and when it is orthogonal to $\xi$, respectively.

\subsection{Ricci tensors}
\label{sec:org3942125}

The curvature tensor can be contracted in three ways to obtain a \((1,1)\)-form. On a K\"ahler manifold, all three ways give the same result; on a non-K\"ahler manifold they may not.

The easiest of these to compute for us is the one given by taking the traces of the endomorphisms in the curvature form. As those endomorphisms are the identity here, we find that
$$
r_1(\alpha, \beta)
= n \bigl( h(\alpha, \beta) - h(\alpha, \xi) h(\xi, \beta) \bigr).
$$
This is the same as we obtain by contracting the curvature tensor along \(\delta\) and \(\gamma\). As before, Cauchy--Schwarz gives us the estimates
$$
0
\leq \frac{r_1(\alpha, \alpha)}{\|\alpha\|^2}
\leq n
$$
which are sharp under the same conditions as before. This form is the curvature form of the anti-canonical bundle on $X$ when equipped with the metric induced by $h$, and surprisingly it is semipositive. It cannot be positive, of course, as that would imply that $X$ were projective, but it comes as close as it can.

Our second contraction is along \(\alpha\) and \(\beta\). The only relevant part of the curvature tensor we don't know how to contract is \(h(\alpha, \xi)h(\xi, \beta)\). Let \((\zeta_1, \ldots, \zeta_n)\) be a local holomorphic frame that's orthonormal at a point \(z\) we care about. We have
$$
\sum_{j=1}^n h(\zeta_j, \xi) h(\xi, \zeta_j) = h(\xi, \xi) = 1
$$
as \(\xi = \sum_{j=1}^n h(\xi, \zeta_j) \zeta_j\) and \(h(\xi,\xi) = 1\).
Then
$$
r_2(\gamma, \delta)
= n h(\gamma, \delta) - h(\gamma, \delta)
= (n-1) h(\gamma, \delta).
$$
This form is not only different from \(r_1\) but it is positive-definite.

The third contraction is along \(\beta\) and \(\gamma\). We let \((\zeta_1, \ldots, \zeta_n)\) be a local holomorphic frame that's orthonormal at a point \(z\) as before. We have
$$
\sum_{j=1}^n h(\alpha, \zeta_j) h(\zeta_j, \delta)
= h(\alpha, \delta).
$$
Also
$$
\sum_{j=1}^n h(\alpha, \xi) h(\xi, \zeta_j) h(\zeta_j, \delta)
= h(\alpha, \xi) \sum_{j=1}^n  h(\xi, \zeta_j) h(\zeta_j, \delta)
= h(\alpha, \xi) h(\xi, \delta).
$$
Together, we get
$$
r_3(\alpha, \delta)
= h(\alpha, \delta) - h(\alpha, \xi) h(\xi, \delta)
= \frac{1}{n} r_1(\alpha, \delta).
$$

\subsection{Scalar curvature}
\label{sec:orgf9212d2}

We can contract any of the Ricci-forms we got to obtain the scalar curvature of the Hopf manifold. Picking the first two, we get
$$
s = n(n-1),
$$
while picking the third gives \(1/n\) times that, so Hopf manifolds have positive constant scalar curvature.


\section{Iwasawa manifold}
\label{sec:orgd67c2ff}

Let $G$ be the complex Lie group formed by upper-triangular complex matrices
$$
\begin{pmatrix}
  1 & x & z
  \\
  0 & 1 & y
  \\
  0 & 0 & 1
\end{pmatrix}
$$
and let $\Gamma \subset G$ be the discrete subgroup of matricies of this form whose entries are Gaussian integers. Then $X = G / \Gamma$ is a compact complex manifold of dimension $3$, called the \emph{Iwasawa manifold}.

Let $M := M(x,y,z)$ be a matrix as above. Then $M^{-1}dM$ is a matrix of $1$-forms that is invariant under the action of $G$, so it descends to define $1$-forms on $X$. Writing this out, we have
$$
M^{-1}dM
= \begin{pmatrix}
  1 & -x & -z
  \\
  0 & 1 & -y
  \\
  0 & 0 & 1
\end{pmatrix}
\begin{pmatrix}
  0 & dx & dz
  \\
  0 & 0 & dy
  \\
  0 & 0 & 0
\end{pmatrix}
=
\begin{pmatrix}
  0 & dx & dz - x dy
  \\
  0 & 0 & dy
  \\
  0 & 0 & 0
\end{pmatrix}.
$$
That is, the holomorphic $1$-forms $dx$, $dy$ and $dz - x dy$ on $G$ descend to $X$. Note that $d(dz - xdy) = -dx \wedge dy \not= 0$, so $X$ is not a K\"ahler manifold, as any holomorphic $1$-form on a compact K\"ahler manifold is closed.

These three $1$-forms are linearly independent at any point of $X$, so they trivialize the cotangent bundle $\Omega_X$. That is, they define a holomorphic bundle isomorphism $\Omega_X \to \kk C^3$. Pulling back the flat metric on the trivial bundle then gives a flat metric on $\Omega_X$, and taking duals gives a flat Hermitian metric on $X$. Let's work this out concretely.

Pulling back the flat metric amounts to defining our Hermitian metric on $\Omega_X$ so that $(dx, dy, dz - xdy)$ is an orthonormal frame. The matrix of the map from $(dx, dy, dz)$ coordinates to $(dx, dy, dz - xdy)$ is
$$
A = \begin{pmatrix}
1 & 0 & 0 \\
0 & 1 & 0 \\
0 & -x & 1
\end{pmatrix}
$$
so its inverse gives the coordinate change going the other way. The pullback of the standard metric in $(dx, dy, dz-x dy)$ coordiates to $(dx, dy, dz)$ ones is thus given by
$$
H = \overline{(A^{-1})^t} I_3 A^{-1}
=
\begin{pmatrix}
1 & 0 & 0 \\
0 & 1 & \bar x \\
0 & 0 & 1
\end{pmatrix}
\begin{pmatrix}
1 & 0 & 0 \\
0 & 1 & 0 \\
0 & x & 1
\end{pmatrix}
=
\begin{pmatrix}
1 & 0 & 0 \\
0 & 1+|x|^2 & \bar x \\
0 & x & 1
\end{pmatrix}.
$$
The Hermitian metric on $T_X$ is then given by
$$
H' := \overline{H}^{-1}
= \begin{pmatrix}
1 & 0 & 0 \\
0 & 1 & - x \\
0 & -\bar x & 1+|x|^2
\end{pmatrix}.
$$
We have
$$
(H')^{-1} = \begin{pmatrix}
  1 & 0 & 0 \\
  0 & 1 + |x|^2 & x \\
  0 & \bar x & 1
\end{pmatrix}
$$
so the Chern connection of this metric is
$$
(H')^{-1} \partial H'
= \begin{pmatrix}
  1 & 0 & 0 \\
  0 & 1 + |x|^2 & x \\
  0 & \bar x & 1
\end{pmatrix}
\begin{pmatrix}
0 & 0 & 0 \\
0 & 0 & -dx \\
0 & 0 & \bar x dx
\end{pmatrix}
= \begin{pmatrix}
0 & 0 & 0 \\
0 & 0 & -dx \\
0 & 0 & 0
\end{pmatrix}.
$$
This satisfies $\bar\partial((H')^{-1} \partial H') = 0$ as expected, which is a nice sanity check. A neat property of this metric is that its Chern connection is not trivial, but its curvature is still zero.

A fairly common question among those who learn complex geometry is to what extent properties of the curvature tensor determine the metric?\footnote{See Berger~\cite[Section~4.5]{berger} for a discussion of this question in the Riemannian case.} The Iwasawa manifold gives an example where a Hermitian metric has a flat curvature tensor, which of course coincides exactly with the curvature tensor of a K\"ahler metric, but where the manifold itself admits no K\"ahler metric. So the curvature tensor can not detect whether a metric is K\"ahler. Bo and Zheng~\cite{yang2016curvature} discuss some of these questions.



\section{TODO: Projectivized bundles}



\subsection{Curvature of tautological bundle}

Let $(E, h) \to X$ be a holomorphic Hermitian vector bundle of rank $r + 1$ over a complex manifold $X$. The group $\kk C^*$ acts fiberwise on $E$, and we can quotient by its action to form a projective bundle $\kk P(E) \to X$. In general there are projective bundles that do not arise as the projectivization of vector bundles, but we are not going to look at those here.

As before there is a tautological line bundle $\cc O(-1) \to \kk P(E)$, and the metric $h$ defines a Hermitian metric on it. We want to compute the curvature form of this metric, and later see if we can turn that form into an honest K\"ahler metric.

It is enough to perform these computations locally. In this case I think we are even forced to. If we try to play the same trick as in section~\ref{projective-space-redux} we end up having to compute the value of the pullback of a connection on the Euler section that trivializes $\cc O(-1)$ over the total space of $E$ minus the zero section. But the pullback of the connection is defined on pullbacks of sections and extended by linearity, and the Euler section is not the pullback of a section downstairs, so we have to look at the local picture.

In any case, we pick a point of interest $(x_0, [v_0])$ of $\kk P(E)$. As $E \to X$ is a vector bundle, there is a neighborhood $U \subset X$ of $x$ such that $E|_U \cong U \times V$, where $V$ is the fiber of $E$. It follows that the projective bundle also splits holomorphically over that neighborhood; $\kk P(E)|_U \cong U \times \kk P(V)$. We pick $\lambda \in V^*$ that does not vanish on any representative of $[v]$, and consider the neighborhood $U \times H_\lambda$ of $(x, [v])$. There we have the holomorphic Euler section $\xi(x, v) = v$ that trivializes the tautological bundle. We want to compute the (negative of the) curvature form
$$
\psi = -\frac i2 \partial \bar\partial \log h(\xi, \xi).
$$
The difference with the case of projective space is that there we had a flat Hermitian metric $h$ (one that did not depend on the base variable $x$), while here the metric has curvature. We also have to take derivatives in the base and fiber directions, instead of only in the fiber directions. (This is thankfully easy as the bundle splits holomorphically locally.)

So let $(\alpha, s)$ and $(\beta, t)$ be local holomorphic sections of $T_{U \times H_\lambda}$. We have
$$
\bar\partial_{(\beta, t)} \log h(\xi,\xi)
= \frac{h(\xi, \nabla_\beta \xi + t)}{h(\xi,\xi)},
$$
where $\nabla$ is the Chern connection of $h$. Then
\begin{align*}
\psi((\alpha,s), (\beta,t))
&= -\frac{h(\nabla_\alpha \xi + s, \nabla_\beta \xi + t)}{h(\xi,\xi)}
- \frac{h(\xi, \bar\partial_\alpha \nabla_\beta \xi)}{h(\xi,\xi)}
+ \frac{h(\nabla_\alpha \xi + s, \xi)}{h(\xi,\xi)}
\frac{h(\xi, \nabla_\beta \xi + t)}{h(\xi,\xi)}
\\
&= -\omega(\nabla_\alpha \xi + s, \nabla_\beta \xi + t)
- \frac{h(\xi, F^\nabla_{\alpha\beta}\xi)}{h(\xi,\xi)},
\end{align*}
where $\omega$ is the Fubini--Study metric on $\kk P(V)$ and we recall that on holomorphic sections we have $\bar\partial \nabla \xi = F^\nabla \xi$ and $F^\nabla$ is anti-Hermitian.

This curvature form does usually not have a definite sign. The Fubini--Study part can equal zero without the tangent fields in question being zero, and in general pretty much anything can appear in the curvature term of $h$. One exception is when $(E,h)$ is Griffiths negative. In that case, the curvature form $\psi$ is positive-definite, which implies that $E^*$ is ample.\footnote{We should be looking at $\kk P(E^*)$ like algebraic geometers. If we did, Griffiths positive would imply $E$ ample (that is, $\cc O(1) \to \kk P(E^*)$ would be ample).}




SKETCH:

$$
\bar\partial \log h(\xi,\xi)
= \frac{h(\xi, \nabla \xi)}{h(\xi,\xi)}
+ \frac{h(\xi, \id_H)}{h(\xi,\xi)}
$$
and
$$
\displaylines{
\partial\bar\partial \log h(\xi,\xi)
% Term 1, base derivatives
= \frac{h(\nabla \xi, \nabla \xi)}{h(\xi,\xi)}
+ \frac{h(\xi, \bar\partial_U \nabla \xi)}{h(\xi,\xi)}
- \frac{h(\nabla \xi, \xi)}{h(\xi,\xi)}
\wedge \frac{h(\xi, \nabla \xi)}{h(\xi,\xi)}
\hfill\cr\hfill{}
% Term 1, fiber derivatives
+ \frac{h(\id_H, \nabla \xi)}{h(\xi,\xi)}
- \frac{h(\id_H, \xi)}{h(\xi,\xi)}
\wedge \frac{h(\xi, \nabla \xi)}{h(\xi,\xi)}
\cr\hfill{}
% Term 2, base derivatives
+ \frac{h(\nabla \xi, \id_H)}{h(\xi,\xi)}
- \frac{h(\nabla \xi, \xi)}{h(\xi, \xi)}
\wedge \frac{h(\xi, \id_H)}{h(\xi,\xi)}
\cr\hfill{}
% Term 2, fiber derivatives
+ \frac{h(\id_H, \id_H)}{h(\xi,\xi)}
- \frac{h(\id_H, \xi)}{h(\xi,\xi)}
\wedge \frac{h(\xi, \id_H)}{h(\xi,\xi)}
}
$$
That is as a $(1,1)$-form. The associated sesquilinear form evaluated on $s \oplus \alpha$ and $t \oplus \beta$ is
$$
\psi(s \oplus \alpha, t \oplus \beta)
= -h_{\mathrm{FS}}(\nabla_\alpha \xi + s, \nabla_\beta \xi + t)
+ \frac{h(\frac i2\Theta_{\alpha\beta}\xi, \xi)}{h(\xi,\xi)}.
$$
I'd like to be able to say that the Fubini--Study metric doesn't depend on the Hermitian inner product used to define it, so even though $h$ is used in the definition of $h_{\mathrm{FS}}$, that inner product is equal to the pullback from the projective space factor. That is, it doesn't depend on the base variables. (So the only dependency on those comes from the vector fields we put in here.)


In any case, the curvature form of the tautological bundle has a definite sign when restricted to the fiber directions. This points the way to showing that the projective bundle is a K\"ahler manifold, at least when the base is a compact K\"ahler manifold: We pull back a K\"ahler metric from the base and multiply it with a big enough constant to offset the potential negativity of the curvature form in non-fiber directions.


\subsection{Splitting the tangent bundle}

We have a short exact sequence of vector bundles
$$
0
\longrightarrow T_{\kk P(E) / X}
\stackrel{j}{\longrightarrow} T_{\kk P(E)}
\stackrel{\pi_*}{\longrightarrow} \pi^* T_X
\longrightarrow 0.
$$
We also have a real $(1,1)$-form $\psi$ on $\kk P(E)$ that is positive-definite on the subbundle $T_{\kk P(E)/X}$. If $\psi$ were positive-definite on the whole tangent bundle, we could use it to split the exact sequence smoothly. In fact, knowing only that it is positive-definite on the subbundle is enough to do this.

One way to define a splitting of the exact sequence is to construct a linear morphism $j^* : T_{\kk P(E)} \to T_{\kk P(E)/X}$ such that $j^* j = \id_{T_{\kk P(E)/X}}$; the splitting is then given by
$$
T_{\kk P(E)} \to T_{\kk P(E)/X} \oplus \pi^*T_X,
\quad
\Xi \mapsto j^*(\Xi) \oplus \pi_*(\Xi).
$$
We can view the $(1,1)$-form $\psi$ as a morphism $\psi : T_{\kk P(E)} \to \overline{T}_{\kk P(E)}^*$. By hypothesis, it is an isomorphism when restricted to $T_{\kk P(E)/X}$. We then set
$$
j^* := \psi^{-1} \circ \bar{j}^* \circ \psi.
$$
It helps to look at what this gives on given fields. Let $A$ and $B$ be tangent fields on $\kk P(E)$. Then
$$
\psi(A) = (B \mapsto \psi(A, B)),
$$
and
$$
\bar j^* \circ \psi(A) = (b \mapsto \psi(A, b)),
$$
where $b \in T_{\kk P(E)/X}$. This defines an element of $\overline T_{\kk P(E)/X}^*$, where $\psi$ is an isomorphism, so $\psi^{-1} \circ \bar j^* \circ \psi(A)$ is the unique element $a$ of $T_{\kk P(E)/X}$ such that
$$
\psi(a, b) = \psi(A, b)
$$
for all $b$ in $T_{\kk P(E)/X}$. Tracing back through this shows that $j^* \circ j = \id_{T_{\kk P(E)/X}}$.

To work out what this gives for $s \oplus \alpha$ for our curvature form, we have
$$
t \mapsto \psi(s \oplus \alpha, t) = h_{\mathrm{FS}}(D^h_\alpha \xi + s, t)
$$
so $j^*(s \oplus \alpha) = D^h_\alpha \xi + s$. In our local coordinates, the splitting is the smooth isomorphism
$$
\Phi: T_H \oplus T_U \to T_H \oplus T_U,
\quad
s \oplus \alpha
\mapsto
\begin{pmatrix}
  \id & D^h_\bullet \xi
  \\
  0 & \id
\end{pmatrix}
\begin{pmatrix}
  s \\ \alpha
\end{pmatrix}.
$$
The curvature form of the tautological bundle is the pullback under $\Phi$ of the direct sum of the Fubini--Study metric on $T_H$ and the sesquilinear form
$$
\theta(\alpha, \beta) = -\frac{h(\frac i2\Theta_{\alpha\beta}\xi, \xi)}{h(\xi,\xi)}
$$
on $\pi^*T_X$. For what it's worth, that form extends to all of $T_{\kk P(E)}$ by $\theta(A, B) = \theta(\pi_*A, \pi_*B)$.

Note that we can add the pullback of any form we want from the base without changing the smooth splitting of the tangent bundle.

\subsection{Attempts at the Chern connection}

If $D$ is the Chern connection of the Fubini--Study metric, this would give us
$$
\displaylines{
\partial_{\gamma \oplus z} \psi(\alpha \oplus s, \beta \oplus t)
% First term, base directions
% Wait, the FS connection should only differentiate in fiber directions
= h_{\mathrm{FS}}(\partial_\gamma(\nabla_\alpha \xi + s), \nabla_\beta \xi + t)
+ h_{\mathrm{FS}}(\nabla_\alpha \xi + s, \bar\partial_\gamma(\nabla_\beta \xi + t))
+ h_{\mathrm{FS}}(\nabla_\alpha \xi + s, \tfrac{i}{2} \Theta_{\gamma\beta} \xi)
% First term, fiber directions
\hfill\cr\hfill{}
+ h_{\mathrm{FS}}(\nabla_\alpha z + \partial s, \nabla_\beta \xi + t)
\cr\hfill{}
% Second term, base directions
+ \frac{h(\nabla_\gamma\frac i2\Theta_{\alpha\beta}\xi, \xi)}{h(\xi,\xi)}
- \frac{h(\nabla_\gamma\xi,\xi)}{h(\xi,\xi)}
\frac{h(\frac i2\Theta_{\alpha\beta}\xi, \xi)}{h(\xi,\xi)}
\cr\hfill{}
% Second term, fiber directions
% We can flip the curvature endomorphism here twice and get that his is equal to
% h_{FS}(z, \tfrac i2 \Theta_{\alpha\beta} \xi)
+ \frac{h(\frac i2\Theta_{\alpha\beta}z, \xi)}{h(\xi,\xi)}
- \frac{h(z, \xi)}{h(\xi,\xi)}
\frac{h(\frac i2\Theta_{\alpha\beta}\xi, \xi)}{h(\xi,\xi)}
}
$$


\subsection{Some kind of connection}

We have a real closed smooth $(1,1)$-form on $\kk P(V)$. It may or may not be degenerate, so there is not a unique connection compatible with the form whose $(0,1)$-part is $\bar\partial$. We can nonetheless calculate something.

Let $(\gamma, u)$ be another tangent field. We have
$$
\displaylines{
\partial_{(\gamma,u)} \psi((\alpha,s),(\beta,t))
= \omega(D_u (\nabla_\alpha \xi + s), \nabla_\beta\xi + t)
+ \omega(\nabla_\alpha u + \partial_u s, \nabla_\beta\xi + t)
\hfill\cr\hfill{}
+ \frac{h(\nabla_\gamma \frac i2\Theta_{\alpha\beta} \xi, \xi)}{h(\xi,\xi)}
- \frac{h(\frac i2\Theta_{\alpha\beta} \xi, \xi)}{h(\xi,\xi)}
\frac{h(\nabla_\gamma\xi,\xi)}{h(\xi,\xi)}
\cr\hfill{}
+ \frac{h(\frac i2\Theta_{\alpha\beta} u, \xi)}{h(\xi,\xi)}
- \frac{h(\frac i2\Theta_{\alpha\beta} \xi, \xi)}{h(\xi,\xi)}
\frac{h(u,\xi)}{h(\xi,\xi)}
}
$$
This needs work. There will be a term that involves $d \Theta$. This would need to be rewritten using $h^{-1}$ (or equivalent), which we have stubbornly refused to do until now. Shows the need to use one of the more local pictures for calculations. Maybe we can cheat to some extent and say that $h(\partial\Theta_{\alpha\beta}\xi,\xi)$ is a $1$-form so there exists a unique vector $\theta$ such that $\psi(\theta,\xi)$ equals it?




\section{TODO: Direct image curvatures}
\label{sec:org5619e67}

Weil--Peterson, maybe.

Berndtsson has papers from about ten years ago we could look at. Cao et al took that further recently, but that's probably too advanced for what we want to do.


\section{TODO: Intuitive explanation for curvature forms}
\label{sec:org2b43ecb}

Wikipedia has a handwavy explanation of curvature as what happens when we parallel transport a section along a parallelogram. Can we make this precise?



\section{TODO: Hom-metrics}

Somewhere you have a draft that talks about the space of functions $f : X \to Y$ and how we can put a metric on that when given metrics on $X$ or $Y$. I think we computed the curvature tensors we get from that.


\section{TODO: Moduli space of complex tori}

That space has a fairly explicit metric whose curvature tensor we can compute. Might be related to the Grassmannian?


\section{TODO: Hilbert scheme or cycle space}

Check Barlet and Magnusson's book. Maybe difficult.

\section{TODO: Curves in projective space}

Done in \cite{fischer1983differential}, apparently. Looks inscrutable.

\section{TODO: Hypersurfaces in projective space}

Done in \cite{vitter1974curvature}, which I can verify once scihub's search comes back.


\section{TODO: Complexified K\"ahler cone}
\label{complexified-kahler-cone}



\bibliographystyle{plain}
\bibliography{sgct}

\end{document}
