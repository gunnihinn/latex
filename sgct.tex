\documentclass[10pt,a4paper]{article}

\usepackage{lmodern}
\linespread{1.1}
\usepackage[utf8]{inputenc}
\usepackage[T1]{fontenc}

\usepackage{fancyref}
\usepackage[colorlinks=true]{hyperref}

\usepackage{amsmath}
\usepackage{amssymb}
\usepackage[amsmath,thref,thmmarks,hyperref]{ntheorem}

\usepackage{tikz-cd}

\theorembodyfont{\itshape}
\newtheorem{theo}{Theorem}[section]
\newtheorem{prop}[theo]{Proposition}
\newtheorem{lemm}[theo]{Lemma}
\newtheorem{coro}[theo]{Corollary}

\theorembodyfont{\rm}
\newtheorem{defi}[theo]{Definition}
\newtheorem{exam}[theo]{Example}
\newtheorem*{claim}{Claim}

\theoremseparator{:}
\theoremsymbol{\ensuremath{\Box}}
\newtheorem*{proof}{Proof}

\newcommand{\kk}[1]{\mathbb{#1}}
\newcommand{\cc}[1]{\mathcal{#1}}

\def\qedhere{}

\def\^#1{^{[#1]}}
\def\qandq{\quad\text{and}\quad}
\def\ov#1{\overline{#1}}

\DeclareMathOperator{\Span}{span}
\DeclareMathOperator{\Gr}{Gr}
\DeclareMathOperator{\GL}{GL}
\DeclareMathOperator{\im}{Im}
\DeclareMathOperator{\Vol}{Vol}
\DeclareMathOperator{\Ker}{Ker}
\DeclareMathOperator{\End}{End}
\DeclareMathOperator{\Aut}{Aut}
\DeclareMathOperator{\Hom}{Hom}
\DeclareMathOperator{\id}{id}
\DeclareMathOperator{\tr}{tr}

\newcommand{\ext}[1]{\bigwedge{}^{\!\!#1}\,}

\newtheorem{question}{Question}

\author{Gunnar Þór Magnússon}
\date{\today}
\title{The savage garden of curvature tensors}

\hypersetup{
 pdfauthor={Gunnar Þór Magnússon},
 pdftitle={The savage garden of curvature tensors},
 pdfkeywords={},
 pdfsubject={},
 pdfcreator={Emacs 27.1 (Org mode 9.3)},
 pdflang={English}}


\begin{document}

\maketitle



\section*{Introduction}
\label{sec:introduction}

These are notes on complex differential geometry, often focused on the curvatures of K\"ahler metrics. There are three main ways of doing differential geometry; with lots of indices in local coordinates, via Cartan's moving frames, and by using connections and intrinsic formulations. Modern Riemannian geometers have settled on the third as its preferred way of doing things. For whatever reason, complex differential geometers prefer the first and second routes, to the point that the intrinsic versions of fundamental results are not even stated in the literature. The goal of these notes is to state and prove the basics of complex differential geometry and to work out its fundamental examples using the language of connections as much as possible. Our motivation for doing this is manifold.

First, I have personally spent enough time on this Earth staring at Einstein sums in local coordinates or weird differential forms. I'm tired of it, do not understand what they mean, and am jealous of my Riemannian geometer friends who can read and write results in very nice notation. A prime example of this is the expression for the curvature tensor of a holomorphic subbundle. Originally it was stated in the moving frame formulation and proved using local coordinates; later expositions have at most proved it also using frames. Our statement of the result is that if $0 \to S \to E \to Q \to 0$ is a short exact sequence, then
\[
R_S(\xi,\overline\eta,s,\overline t)
= R_E(\xi,\overline\eta,s,\overline t)
- h_Q(b(\xi,s),\overline{b(\eta,t)}),
\]
where $R$ are curvature tensors, $h$ is a metric and $b$ is the second fundamental form. Our proof is short and uses only the invariant formulation. This is a nice thing and we can have more of it!
As the professionals in the field are busy getting things done, they will not be the ones to rewrite any of the rest of the basics in different notation for my pleasure. It is thus up to us to do it.

Second, I often wished for more invariant forms for curvature expressions when I was trying to work out curvatures of metrics on the base of deformations of complex manifolds. The way these metrics are often constructed on the base is by taking metrics on the total space or fibers of whatever is being deformed and pushing them forward via direct images. The curvature tensors of these metrics are often computed via local expressions on the fibers (at least the ones Siu, Schumacher and their collaborators did for the Weil--Petersson metric in the 1980s are). I have trouble following those calculations and would like to be able to work on whole fibers at a time.

Third, I've spent some time during COVID-19 lockdowns playing with the Lean theorem prover in an effort to keep my sanity. At the time of writing the state of affairs there seems very promising for the future of mathematics. After watching videos of people formalizing things like that the \hyperlink{https://www.youtube.com/watch?v=deppJ2q_5a0}{sphere is a manifold}, it seems clear that the most convenient way to formalize differential geometry is via the invariant connection formulation. To be able to do that, we have to actually have the results we want to formalize stated and proved in that notation. The above video was actually the inspiration for our treatment of the projective space in these notes.

Finally, I think that complex differential geometers use local coordinates and moving frames out of simple habit. If we show them that there is another way of doing things, that I hope we will show here is no harder in general and often in fact easier, some of them can be convinced to adopt it. In my retirement then, perhaps, I can read the arXiv without my eyes glazing over.


\tableofcontents


\section{Preliminaries}
\label{sec:org093d592}

Let $E \to X$ be a holomorphic vector bundle. We denote by $\cc A^k(E)$ the sheaf $\bigwedge^k T_X^* \otimes E$ of $k$-forms with values in $E$. Similarly we denote by $\cc A^{p,q}(E)$ the sheaf $\bigwedge^{p,q} T_X^* \otimes E$ of $(p,q)$-forms with values in $E$.


\subsection{Connections}

\begin{defi}
A \emph{connection} $D$ on a vector bundle $E$ is a $\kk C$-linear morphism $D : \cc A^0(E) \to \cc A^1(E)$ that satisfies
$$
D(f \otimes s) = df \otimes s + f Ds
$$
for all smooth functions $f$ and sections $s$.
\end{defi}

We can extend the connection to forms of higher degree $D : \cc A^k(E) \to \cc A^{k+1}(E)$ by requiring that
$$
D(\omega \otimes s) = d \omega \otimes s + (-1)^{k} \omega \wedge D s
$$
for any $k$-form $\omega$ and section $s$ of $E$.


\begin{prop}
  Let $\alpha$ be a $p$-form and $\beta$ be a $q$-form with values in $E$. Then
$$
D(\alpha \wedge \beta)
= (d\alpha) \wedge \beta
+ (-1)^p \alpha \wedge D(\beta).
$$
\end{prop}

\begin{proof}
Assume that $\beta = \gamma \otimes s$, where $\gamma$ is a $q$-form and $s$ a section of $E$. Then
\begin{align*}
D(\alpha \wedge \beta)
= D(\alpha \wedge \gamma \otimes s)
&= d(\alpha \wedge \gamma) \otimes s
+ (-1)^{p+q} \alpha \wedge \gamma \wedge Ds
\\
&= d\alpha \wedge \gamma \otimes s
+ (-1)^p \alpha \wedge d\gamma \otimes s
+ (-1)^{p+q} \alpha \wedge \gamma \wedge Ds
\\
&=
d\alpha \wedge \beta
+ (-1)^p \alpha \wedge D\beta.
\end{align*}
The result follows from linearity.
\end{proof}


A holomorphic vector bundle almost comes with a free connection. The $\bar\partial$ operator can be defined on the sections of any holomorphic vector bundle. It is $\kk C$-linear, but only satisfies
$$
\bar\partial(f \otimes s) = \bar\partial f \otimes s + f \bar\partial s.
$$
We'll see later how adding some information to the bundle can yield connections related to the $\bar\partial$ operator.



\subsection{Covariant derivative}

\begin{defi}
If $\xi$ is a tangent field on $X$ and $s$ a section of $E$, we define the \emph{covariant derivative} of $s$ in the direction of $\xi$ by the contraction
$$
D_\xi s := \iota_\xi D s.
$$
\end{defi}

This operation is $\kk C$-linear on sections of $E$, and $C^\infty$-linear in $\xi$.

\begin{prop}
The covariant derivative of a $k$-form $\alpha$ with values in $E$ is
$$
\displaylines{
(D \alpha)(\xi_0, \ldots, \xi_k)
= \smash{\sum_{j=0}^k} (-1)^j D_{\xi_j} \alpha(\xi_0, \ldots, \hat{\xi}_j, \ldots, \xi_k)
\hfill\cr\hfill{}
+ \sum_{i < j} (-1)^{i+j} \alpha([\xi_i,\xi_j], \xi_0, \ldots, \hat{\xi}_i, \ldots, \hat{\xi}_j, \ldots, \xi_k).
}
$$
\end{prop}

\begin{proof}
Recall that the exterior derivative of a $k$-form can be defined by the formula
$$
\displaylines{
(d\omega)(\xi_0, \ldots, \xi_{k})
  = \smash{\sum_{j=0}^k} (-1)^j d_{\xi_j} \omega(\xi_0, \ldots, \hat{\xi}_j, \ldots, \xi_k)
  \hfill\cr\hfill{}
+ \sum_{i < j} (-1)^{i+j} \omega([\xi_i,\xi_j], \xi_0, \ldots, \hat{\xi}_i, \ldots, \hat{\xi}_j, \ldots, \xi_k),
}
$$
where $[\xi_i,\xi_j]$ is the Lie bracket and $\hat{\xi}_j$ means the element $\xi_j$ is omitted. For a decomposable $k$-form $\omega \otimes s$ we have
$$
D(\omega \otimes s) = d\omega \otimes s + (-1)^k \omega \wedge Ds.
$$
Evaluating this on tangent fields we get
$$
\displaylines{
D(\omega \otimes s)(\xi_0, \ldots, \xi_k)
= \smash{\sum_{j}} (-1)^j d_{\xi_j} \omega(\xi_0, \ldots, \hat \xi_j, \ldots, \xi_k) \cdot s
\hfill\cr\hfill{}
+ \smash{\sum_{i < j}} (-1)^{i+j} \omega([\xi_i,\xi_j], \xi_0, \ldots, \hat\xi_i, \ldots, \hat\xi_j, \ldots, \xi_k) \cdot s
\cr\hfill{}
+ (-1)^k(\omega \wedge Ds)(\xi_0, \ldots, \xi_k)
\cr{}
\phantom{D(\omega \otimes s)(\xi_0, \ldots, \xi_k)}
= \smash{\sum_{j}}\bigl( (-1)^j d_{\xi_j} \omega(\xi_0, \ldots, \hat \xi_j, \ldots, \xi_k) \cdot s
\hfill\cr\hfill{}
+ (-1)^k (-1)^j \omega(\xi_0, \ldots, \hat\xi_j, \ldots, \xi_k) D_{\xi_j} s\bigr)
\cr\hfill{}
+ \sum_{i < j} (-1)^{i+j} \omega([\xi_i,\xi_j], \xi_0, \ldots, \hat\xi_i, \ldots, \hat\xi_j, \ldots, \xi_k) \cdot s
\cr{}
\phantom{D(\omega \otimes s)(\xi_0, \ldots, \xi_k)}
= \smash{\sum_j} (-1)^j D_{\xi_j} \alpha(\xi_0, \ldots, \hat\xi_j, \ldots, \xi_k)
\hfill\cr\hfill{}
\sum_{i < j} (-1)^{i+j} \alpha([\xi_i,\xi_j], \xi_0, \ldots, \hat\xi_i, \ldots, \hat\xi_j, \ldots, \xi_k)
}
$$
and we conclude by extending by linearity.
\end{proof}


\subsection{Curvature form}

\begin{defi}
The \emph{curvature form} $F^D$ of a connection $D$ is defined by
$$
F^D \wedge s = D^2 s
$$
for sections $s$ of $E$. It is a $2$-form that takes values in $\End E$.
\end{defi}


\begin{prop}
Let $\alpha$ be a $k$-form with values in $E$. Then
$$
D^2 \alpha = F^D \wedge \alpha.
$$
\end{prop}

\begin{proof}
This is true for sections by definition. Let $\alpha = \omega \otimes s$ be a decomposable form. We have
$$
D\alpha = d\omega \otimes s + (-1)^k \omega \wedge Ds
$$
and
\begin{align*}
D^2 \alpha
&= d^2 \omega \otimes s - (-1)^k d\omega \wedge Ds
+ (-1)^k d\omega \wedge Ds + \omega \wedge D^2 s
\\
&= \omega \wedge F^D \wedge s
= F^D \wedge \alpha.
\end{align*}
The result follows by linearity.
\end{proof}

The above proof shows in particular that if $f$ is a smooth function (that is, a $0$-form) then $F^D(f s) = f F^D s$ for any section $s$. That is, the curvature is $C^\infty$-linear. This justifies our claim that the curvature form takes values in $\End E$.


\begin{prop}
\label{prop:curvature-commutative}
Let $\xi, \eta$ be tangent fields on $X$ and $s$ a section of $E$. Then
$$
F^D(\xi, \eta) \wedge s
= D_\xi D_\eta s - D_\eta D_\xi s - D_{[\xi,\eta]} s.
$$
\end{prop}

\begin{proof}
Apply our formula for the exterior covariant derivative to the $1$-form $Ds$ which takes values in $E$. This gives
\[
F^D(\xi,\eta) s
= D(Ds)(\xi,\eta)
= D_\xi D_\eta s - D_\eta D_\xi s - D_{[\xi,\eta]} s.
\]
\end{proof}


\begin{prop}
The curvature form is $\cc C^\infty$-linear in any of its variables.
\end{prop}

\begin{proof}
Let $f$ be a smooth function. We have
$$
D(fs) = df \otimes s + f Ds
$$
so
\begin{align*}
D^2(fs)
&= D(df \otimes s + f Ds)
\\
&= d^2f \otimes s - df \wedge Ds + df \otimes Ds + f D^2 s
= f D^2 s.
\end{align*}
For any tangent field, we have $D_{f\xi}s = f D_\xi s$. Then
$$
D_\eta D_{f \xi} s
= D_\eta f D_\xi s
= d_\eta f \, D_\xi s + f D_\eta D_\xi s.
$$
We also have $[f\xi, \eta] = f[\xi,\eta] - d_\eta f \cdot \xi$, so
$$
D_{[f\xi,\eta]}s
= D_{f[\xi,\eta]}s - D_{d_\eta f \cdot \xi} s
= f D_{[\xi,\eta]} s - d_\eta f D_\xi s.
$$
Putting this together with \thref{prop:curvature-commutative} we get
\begin{align*}
F(f \xi, \eta) s
&= f D_\xi D_\eta s - (d_\eta f D_\xi s + f D_\eta D_\xi s) - (f D_{[\xi,\eta]} s - d_\eta f D_\xi s)
\\
&= f D_\xi D_\eta s - f D_\eta D_\xi s - f D_{[\xi,\eta]} s
= f F(\xi, \eta) s.
\end{align*}
We then conclude that $F(\xi,f \eta)s = f F(\xi,\eta) s$ by anticommutativity of $F$.
\end{proof}



\subsection{Induced connections and curvatures}
\label{sec:induced-connections}

Any ``natural'' construction in the category of vector spaces defines the same construction in the category of vector bundles.\footnote{``Natural'' here means any construction that can be proven to be independent of the choice of a basis.} Thus we have duals, tensor products and powers, symmetric and exterior powers of bundles, and so on. If we have connections on the component bundles of our construction, we get an induced connection on the new bundle as well. Whenever there is a product available on the underlying vector spaces, the induced connection is defined in such a way as to preserve the product. For example, the connection on the space of linear morphisms is defined so that a version of the product rule $(f(x))' = f'(x) + f(x')$ holds.


\begin{prop}
  Let $E$ and $F$ be vector bundles over $X$, and let $D_E$ and $D_F$ be connections on $E$ and $F$. Let $s$ be a section of $\cc A^p(E)$ and $t$ a section of $\cc A^q(F)$. Then:
  \begin{enumerate}
\item There is a connection $D_{E \oplus F}$ on $E \oplus F$ defined by
\[
D_{E \oplus F}(s \oplus t) = D_E s \oplus D_F t.
\]
Its curvature is
\[
\Theta_{E \oplus F} = \Theta_E \oplus \Theta_F.
\]
\item There is a connection $D_{E \otimes F}$ on $E \otimes F$ defined by
\[
D_{E \otimes F}(s \otimes t)
= D_E s \otimes t + (-1)^{\deg s} s \otimes D_F t.
\]
Its curvature is
\[
\Theta_{E \otimes F} = \Theta_E \otimes \id_F \oplus \id_E \otimes \Theta_F.
\]
\item There is a connection $D_{E^*}$ on $E^*$ defined by
\[
d(\alpha(s))= (D_{E^*}\alpha)(s) + (-1)^{\deg s} \alpha(D_E s),
\]
where $\alpha$ is a section of $E^*$.
Its curvature is
\[
\Theta_{E^*} = -\Theta_E^{\dagger}.
\]
\item There is a connection $D_{\Hom(E,F)}$ on $\Hom(E,F)$ defined by
\[
D_F(f(s)) = (D_{\Hom(E,F)}f)(s) + (-1)^{\deg s} f(D_E s),
\]
where $f$ is a section of $\Hom(E,F)$. Under the isomorphism $\Hom(E,F) = E^* \otimes F$, it corresponds to the connection induced by $D_E$ and $D_F$ on the tensor product. Its curvature is
\[
\Theta_{\Hom(E,F)} = -\Theta_E^{\dagger} \otimes \id_F + \id_E \otimes \Theta_F.
\]
\item There is a connection $D_{S^kE}$ on $S^kE$ defined by
\[
D_{S^kE} (s_1 \odot \cdots \odot s_k)
= \sum_{j=1}^k (-1)^{\deg s_1 + \cdots + \deg s_{j-1}} s_1 \odot \cdots \odot D_E s_j \odot \cdots \odot s_k.
\]
Its curvature is
\[
\Theta_{S^kE} (s_1 \odot \cdots \odot s_k)
= \sum_{j=1}^k s_1 \odot \cdots \odot \Theta_E s_j \odot \cdots \odot s_k.
\]
\item There is a connection $D_{\bigwedge^kE}$ on $\bigwedge^k E$ defined by
\[
D_{\bigwedge^kE} (s_1 \wedge \cdots \wedge s_k)
= \sum_{j=1}^k (-1)^{\deg s_1 + \cdots + \deg s_{j-1}} s_1 \wedge \cdots \wedge D_E s_j \wedge \cdots \wedge s_k.
\]
Its curvature is
\[
\Theta_{\bigwedge^kE} (s_1 \wedge \cdots \wedge s_k)
= \sum_{j=1}^k s_1 \wedge \cdots \wedge \Theta_E s_j \wedge \cdots \wedge s_k.
\]
\item There is a connection $D_{\det E}$ on $\det E$ whose curvature is
\[
\Theta_{\det E}
= \tr(\Theta_E).
\]
\end{enumerate}
\end{prop}


\begin{proof}
  TODO.
\end{proof}


A word of warning: The curvature form $\Theta_{\bigwedge^kE}$ is not the same as the exterior power $\bigwedge^k \Theta_E$. The former is a $(1,1)$-form, while the latter is a $(k,k)$-form, and they live on different bundles. The latter comes up in Chern--Weil theory when constructing differential forms that represent the Chern classes of bundles.


\begin{prop}[Bianchi identity]
\label{prop:bianchi-general}
\[
D_{\Hom(E,E)} \Theta_E = 0.
\]
\end{prop}

\begin{proof}
Let $s$ be a section of $\cc A^p(E)$. Then
\[
D_{\Hom(E,E)}(\Theta_E)(s)
= D_E(\Theta_E s) - \Theta_E(D_E s)
= D_E^3 s - D_E^3 s = 0.
\]
\end{proof}


\subsection{Metrics}

\begin{defi}
A \emph{Hermitian metric} $h$ on a holomorphic vector bundle $E \to X$ is a smooth section of $\bigwedge^{1,1}E^* := E^* \otimes \overline E^*$ that satisfies $h(t,\ov s) = \overline{h(s,\ov t)}$ for all sections $s, t$ of $E$.
\end{defi}

Any holomorphic vector bundle admits a Hermitian metric. This is clear over any trivializing neighborhood, and we can patch those together with a partition of unity.


\begin{defi}
Let $(E, h) \to X$ be a Hermitian holomorphic vector bundle. A connection $D$ on $E$ is \emph{compatible} with $h$ if
$$
d h(s, \ov t)
= h(Ds, \ov t) + h(s,\ov{Dt})
$$
for all sections $s, t$ of $E$.
\end{defi}

If we view the metric as a linear morphism $h : E \to \ov E^*$, then $h(s, \ov t) = h(s)(\ov t)$. Using the connections that a connection $D$ induces on derived bundles, we get
\begin{align*}
d h(s, \ov t)
&= (D_{\Hom(E, \ov E^*)}h)(s)(\ov t)
+ h(D_E s)(\ov t)
+ h(s)(D_{\ov E}\ov{t})
\\
&= (D_{\Hom(E, \ov E^*)}h)(s)(\ov t)
+ h(D_E s, \ov t)
+ h(s, \ov{D_E t}).
\end{align*}
A compatible connection is thus one for which the metric is parallel.
A Hermitian metric comes with a unique compatible connection:


\begin{prop}
Let $(E, h) \to X$ be a Hermitian holomorphic vector bundle. There exists a unique connection $D$ on $E$ that is compatible with the metric $h$ and whose $(0,1)$-part is $\bar\partial$. This is the \emph{Chern connection} of $(E,h)$.
\end{prop}

\begin{proof}
Let's view the metric $h$ as an isomorphism $E \to \overline E^*$. The map $\ov t \mapsto \partial_\xi h(s, \ov t)$ defines a $(0,1)$-form on $E$, so there exists a unique smooth section $A(\xi,s) = h^{-1}(\partial_\xi h(s, \ov t))$ of $E$ such that
\[
\partial_\xi h(s, \ov t) = h(A(\xi, s), \ov t)
\]
for all $t$.\footnote{What we'd \emph{really} like to do is to say that $\partial h$ is a $(1,0)$-form with values in $\bigwedge^{1,1}E$, and thus $h^{-1}\partial h$ is a $(1,0)$-form with values in $\End E$. Adding $\bar\partial$ gives the Chern connection. This doesn't work because the exterior derivative isn't well-defined on $\bigwedge^{1,1}E$, so $\partial h$ is meaningless. However, this \emph{is} how the construction of the Chern connection goes in a local frame. One then proves that the local construction glues along the coordinate change maps and is well-defined globally. We're able to get away with this here because $h(s,t)$ is a perfectly good smooth function that we can take exterior derivatives of.}
As $\partial$ is linear in $\xi$, then so is $A$. We have $\partial_\xi h(f s, \ov t) = \partial_\xi f h(s,\ov t) + f \partial_\xi h(s, \ov t)$, so $A(\xi, fs) = \partial f A(\xi, s) + f A(\xi, s)$. The morphism $A$ depends smoothly on $\xi$ and $s$ because the metric is smooth.

It follows that $(\xi, s) \mapsto A(\xi, s)$ is a connection on holomorphic sections of $E$. Such an object can be extended uniquely to a connection on smooth sections of $E$: If $s$ is holomorphic and $f$ is smooth, we set $A(\xi, f s) = df \otimes s + f A(\xi, s)$. If $s$ is a smooth section, we pick local holomorphic sections $(s_1, \ldots, s_r)$ and find smooth functions $f_j$ such that $s = \sum_j f_j s_j$, and set $A(\xi,s) = \sum_j A(\xi, f_j s_j)$.

By construction, the resulting connection is compatible with the metric, and its $(0,1)$-part is $\bar\partial$.
\end{proof}


\begin{prop}
The curvature form of the Chern connection is a purely imaginary $(1,1)$-form.
\end{prop}

\begin{proof}
Let $s, t$ be sections of $E$. We have
\begin{align*}
0
= d^2 h(s, \ov t)
&= d h(Ds, \ov t) + d h(s, \ov{Dt})
\\
&= h(F s, \ov t) - h(Ds, \ov{Dt}) + h(Ds, \ov{Dt}) + h(s, \ov{F t})
\\
&=
h(Fs,\ov t) + h(F^{\dagger}s, \ov{t}).
\end{align*}
It follows that $F^\dagger = -F$, so $F$ is imaginary.

We have $D = D^{1,0} + \bar\partial$, so $F = (D^{1,0})^2 + (\bar\partial D^{1,0} + D^{1,0}\bar\partial)$ because $\bar\partial^2 = 0$. That is, $F$ has no $(0,2)$-part. But then it also has no $(2,0)$-part by the above.
\end{proof}

For this reason we often consider the curvature form of a Chern connection to be $\Theta := \frac i2 F$. That is a real $(1,1)$-form with values in $\End E$.



\subsection{Hermitian metrics}

We now focus on $E = T_X$, the tangent bundle of a given manifold $X$.


\begin{prop}
Let $\operatorname{Herm}(T_X)$ denote the sheaf of smooth Hermitian forms on $T_X$. Then
$$
\operatorname{Herm}(T_X) \to \ext{1,1} T_X^*,
\quad
h \mapsto \im h
$$
is an isomorphism onto the space of real $(1,1)$-forms.
\end{prop}

\begin{proof}
Let $h$ be a Hermitian form. For tangent fields $\alpha$ and $\beta$ we have
$$
\overline{\im h(\alpha,\ov\beta)}
= \overline{\frac{1}{2i}(h(\alpha,\ov\beta) - h(\beta,\ov\alpha))}
= -\frac{1}{2i}(h(\beta,\ov\alpha) - h(\alpha,\ov\beta))
= \im h(\alpha, \ov\beta)
$$
so $\im h$ is a real $(1,1)$-form. It is clearly $\kk R$-linear (even $C^\infty$-linear for real-valued functions). If $h$ is a Hermitian form such that $\im h = 0$, then $h$ is real for all tangent fields. In particular,
$$
h(i\alpha, \ov\alpha) = i h(\alpha,\ov\alpha)
$$
is real, so $h(\alpha,\ov\alpha) = 0$ for all $\alpha$. It follows that $h = 0$, so the morphism is injective.
\end{proof}


\begin{defi}
If $h$ is a Hermitian form, the $(1,1)$-form $\omega = -\im h$ is the \emph{K\"ahler form} of $h$.
\end{defi}

The sign deserves an explanation. Suppose we're on a complex vector space $V$ of dimension $n$, and that we have a basis $(v_1,\ldots,v_n)$. In this basis, the Hermitian form $h$ can be written as
\[
h(z, \ov w)
= \sum_{jk} h(v_j, \ov v_k) z_j \ov w_k
= \Bigl(
\sum_{jk} h(v_j, \ov v_k) \, dv_j \otimes d\bar v_k
\Bigr)(z, \ov w).
\]
where $z = \sum_j z_j v_j$ and $w = \sum_k w_k v_k$.
The imaginary part of this is
$$
\displaylines{
\frac{1}{2i}
\sum_{jk}
\bigl(
 h(v_j, \ov v_k) \, d v_j \otimes d \bar v_k
- \ov{h(v_j, \ov v_k)} \, d \bar v_j \otimes d v_k
\bigr)
\hfill\cr\hfill{}
\begin{aligned}
&= \frac{1}{2i}
\sum_{jk}
\bigl(
 h(v_j, \ov v_k) \, d v_j \otimes d \bar v_k
- h(v_k, \ov v_j) \, d \bar v_j \otimes d v_k
\bigr)
\\
&= -
\sum_{jk}
h(v_j, \ov v_k)
\frac i2 \bigl(
d v_j \otimes d \bar v_k - d \bar v_k \otimes d v_j
\bigr)
= -2
\sum_{jk}
 h(v_j, \ov v_k)
\, \frac i2 d v_j \wedge d \bar v_k.
\end{aligned}
}
$$
The extra factor of $2$ appears because $d v_j \wedge d \bar v_k = \frac 12 (d v_j \otimes d\bar v_k - d\bar v_k \otimes v_j)$, but the elements $\frac i2 dv_j \wedge d\bar v_k$ define an orthonormal basis for the space of $(1,1)$-forms, so we prefer writing things in terms of them. In any case, the imaginary part of $h$ is clearly a \emph{negative} $(1,1)$-form if $h$ is positive-definite. This is unfortunate, so we take the K\"ahler form to be the negative imaginary part to get a positive $(1,1)$-form when $h$ is positive-definite.



We will write $\nabla$ for the Chern connection of a given Hermitian metric $h$. By the discussion in the previous sections, we can apply this connection to tensor fields on $X$. In particular, we can consider the identity morphism $\id : T_X \to T_X$, which we view as a $(1,0)$-form with values in $T_X$. Taking its covariant derivative, we obtain
$$
(\nabla \id)(\alpha, \beta)
= \nabla_\alpha \beta - \nabla_\beta \alpha - [\alpha, \beta].
$$
This is better known as the torsion tensor of the connection:

\begin{defi}
The \emph{torsion} of $\nabla$ is
$$
\tau(\alpha,\beta)
= \nabla_\alpha \beta - \nabla_\beta \alpha - [\alpha, \beta].
$$
\end{defi}

\begin{prop}
The torsion tensor is antisymmetric and $\cc C^\infty$-linear in both of its variables.
\end{prop}

\begin{proof}
That the tensor is antisymmetric is clear from its definition. If $f$ is a smooth function, we have
\begin{align*}
\tau(f\alpha,\beta)
&= \nabla_{f\alpha}\beta - \nabla_\beta(f\alpha) - [f\alpha,\beta]
\\
&= f\nabla_\alpha \beta - (d_\beta f \alpha + f \nabla_\beta \alpha) - (f[\alpha,\beta] - d_\beta f \alpha)
= f \tau(\alpha,\beta).
\qedhere
\end{align*}
\end{proof}

There's a very nice relationship between the torsion tensor and the K\"ahler form of the metric.

\begin{prop}
For holomorphic tangent fields $\alpha, \beta$ and $\gamma$, we have
\[
\partial\omega(\alpha,\gamma,\ov\beta)
= \omega(\tau(\alpha,\gamma), \ov\beta).
\]
\end{prop}

\begin{proof}
From the definition of $\omega$ as the imaginary part of $h$ we get
\[
\partial\omega(\alpha,\ov\beta)
= -\partial\im h(\alpha, \ov\beta)
= -\im h(\nabla\alpha,\ov\beta))
= \omega(\nabla\alpha,\ov\beta).
\]
Then
\begin{align*}
\partial\omega(\alpha,\gamma,\ov\beta)
&= \partial_\alpha\omega(\gamma,\ov\beta)
- \partial_\gamma\omega(\alpha,\ov\beta)
- \omega([\alpha,\gamma], \ov\beta)
\\
&= \omega(\nabla_\alpha\gamma, \ov\beta)
- \omega(\nabla_\gamma\alpha, \ov\beta)
- \omega([\alpha,\gamma], \ov\beta)
= \omega(\tau(\alpha,\gamma), \ov\beta).
\end{align*}
\end{proof}

\begin{coro}
\label{coro:kahler-zero-torsion}
The torsion of $\nabla$ is zero if and only if $d\omega = 0$.
\end{coro}


We'll have more to say about this condition later. For now we'll say a little more about the curvature of a Hermitian metric.


\begin{defi}
The \emph{curvature tensor} of a Hermitian metric $h$ is the $(2,2)$-tensor defined by
$$
R(\alpha,\beta,\gamma,\delta)
= h(\tfrac i2 F_{\alpha\beta}\gamma, \delta).
$$
\end{defi}

As $h$ and $F$ are $\cc C^\infty$-linear in their variables, so is $R$. We have
$$
R(\alpha,\beta,\gamma,\delta)
= h(\tfrac i2 F_{\alpha\beta}\gamma, \delta)
= h(\gamma, \tfrac i2 F_{\beta\alpha}\delta)
= \overline{h(\tfrac i2 F_{\beta\alpha}\delta, \gamma)}
= \overline{R(\beta,\alpha,\delta,\gamma)}.
$$
In general, this is the only symmetry the curvature tensor of a Hermitian metric satisfies. The obstruction to other symmetries we might expect is governed by the torsion tensor. This is explained by the Bianchi identities, but we'll see clearer how the absense of torsion results in extra symmetries later.


\begin{prop}[Bianchi identities]
The curvature form of a Hermitian metric satisfies
$$
F \wedge \id_{T_X} = \nabla \tau
\quad\text{and}\quad
\nabla F = 0.
$$
\end{prop}

These are the first and second Bianchi identities.

\begin{proof}
  For the first equality, take the covariant exterior derivative of the $1$-form $\id_{T_X}$ twice and recall the definition of $\tau$. For the second, take the covariant exterior derivative of the first to get
$$
\nabla F \wedge \id_{T_X} + F \wedge \tau = F \wedge \tau
$$
and cancel $F \wedge \tau$.
\end{proof}


The above does not say that $F_{\alpha\beta}\gamma = \nabla_\alpha \tau(\beta,\gamma)$. It is an equality of $3$-forms with values in $T_X$ and involves the permutations of all of the tensor fields.

The first Bianchi identity can only be stated for metrics on the tangent bundle, as it involves the torsion. The second Bianchi identity is true for the curvature form $F$ of an arbitrary connection $D$ on a vector bundle $E$, where it says that $D_{\End E} F = 0$; see \thref{prop:bianchi-general}.


\subsection{Derived curvature tensors}


A curvature tensor is a complicated thing. We humans have proven bad at understanding objects as multilinear as these. For many applications it is enough to look at simpler tensors that are constructed from the full curvature tensor.

The first two of these are the holomorphic bisectional and sectional curvatures.

\begin{defi}
  The \emph{holomorphic bisectional curvature} of a Hermitian metric $h$ is
$$
B(x,y) = \frac{R(x,x,y,y)}{h(x,x)h(y,y)}.
$$
The \emph{holomorphic sectional curvature} is
$$
H(x) = B(x,x) = \frac{R(x,x,x,x)}{h(x,x)^2}.
$$
\end{defi}

Both of these are defined to be invariant under scalings, so $B(fx,y) = B(x,y)$ for all smooth $f$. The holomorphic bisectional curvature is real-valued by the symmetry of a Hermitian curvature tensor, and so is the holomorphic sectional curvature.


\paragraph{}
It is also possible to use the metric to contract the curvature tensor and obtain a $(1,1)$-tensor.

\begin{defi}
  Let $(v_1, \ldots, v_n)$ be a local holomorphic frame that's orthonormal at a given point. There we define the \emph{first}, \emph{second}, and \emph{third Ricci tensors} by
\begin{align*}
r_1(x,y) &= \sum_{j=1}^n R(x,y,v_j,v_j),
\\
r_2(x,y) &= \sum_{j=1}^n R(v_j, v_j, x,y),
\\
r_3(x,y) &= \sum_{j=1}^n R(x,v_j,v_j,y).
\end{align*}
\end{defi}

There is a fourth possible contraction ($\sum_{j=1}^n R(v_j,y,x,v_j)$) but it is conjugate to $r_3$.

These definitions are independent of the holomorphic frame used. It's easiest to see this by rewriting them using the metric isomorphism $h : T_X \to \overline T_X^*$ and its inverse and taking traces. In general these definitions are really different (see \fref{sec:org8f5818e}), but we'll see later that they all agree for K\"ahler metrics.

The first of these is a tensor associated to the curvature form of a line bundle, as its associated $(1,1)$-form is clearly equal to $\tr \frac i2 F$. That form is also closed, as it can be defined by $\frac i2 \partial\bar\partial \log \det h$. It's the curvature form of the anticanonical bundle $-K_X$ with the metric induced by $h$.




\subsection{K\"ahler metrics}

\begin{defi}
A \emph{K\"ahler metric} $h$ is a Hermitian metric with zero torsion.
\end{defi}


\begin{prop}
Let $h$ be a Hermitian metric. The following are equivalent:
\begin{enumerate}
\item $h$ is K\"ahler.
\item $d\omega = 0$.
\item In local coordinates $(z_1,\ldots,z_n)$, we have
\[
\frac{\partial h_{jk}}{\partial z_l}
= \frac{\partial h_{lk}}{\partial z_j}
\]
for all $j,k,l$, where $h_{jk} = h(z_j,z_k)$.
\item The metric approximates the standard flat metric to the second order. That is, at any point, there exist local holomorphic coordinates centered at that point such that
\[
\omega = \omega_{\text{std}} + O(|z|^2).
\]
\end{enumerate}
\end{prop}

\begin{proof}
We know that $h$ is K\"ahler if and only if its K\"ahler form is closed by \thref{coro:kahler-zero-torsion}, so the first two points are equivalent.

If $H = (h_{jk})$ is the matrix of $h$ in local coordinates, then $\omega = \sum_{jk} \frac i2 h_{jk} dz_j \wedge d\bar z_k$. Then
\[
\partial\omega
= \sum_{jkl} \frac i2 \frac{\partial h_{jk}}{\partial z_l} dz_l \wedge dz_j \wedge d\bar z_k
= \sum_{j<k,l} \frac i2
\Bigl(
\frac{\partial h_{lk}}{\partial z_j}
- \frac{\partial h_{jk}}{\partial z_l}
\Bigr)
dz_j \wedge dz_l \wedge d\bar z_k,
\]
so the second and third conditions are equivalent.

If the metric can be approximated to the second order by the flat metric at a given point, then the K\"ahler form is closed at that point. The point was arbitrary, so the form is closed. Thus the fourth condition implies the second.

Suppose then that the metric satisfies the first condition. In local coordinates $(z_1,\ldots,z_n)$ centered at a given point, we write its matrix as $H = H_0 + \sum h_{jk}(z_j + \bar z_k) + O(|z|^2)$, where $\overline h_{jk} = h_{kj}$. Because of Gram--Schmidt, we can find a linear change of coordinates in which the metric becomes $H = I_n + \sum h_{jk}(z_j + \bar z_k) + O(|z|^2)$.
TODO: FINISH
\end{proof}

The K\"ahler form of a K\"ahler metric is a closed nondegenerate two-form, that is, a symplectic form. As such, many operations involving the K\"ahler metric descend to the level of cohomology on a K\"ahler manifold and restrict its structure.

In practice it is usually much easier to verify that the K\"ahler form of a metric is closed than to compute its torsion tensor. For example, a large class of K\"ahler metrics arise as the curvature forms of holomorphic line bundles (we'll see the details later), where it is obvious that the K\"ahler form is closed.

The historical definition of a K\"ahler metric is as a Hermitian metric whose K\"ahler form is closed~\cite{kahler}, where the condition was used to locally approximate the Euclidean metric around any point. As far as I know symplectic geometry -- that is, the study of manifolds that admit a non-degenerate closed $2$-form -- didn't take off as its own subject until Arnol'd~\cite{arnold} pointed out its connection to classical mechanics.

I'm unsure of when the vanishing of the torsion tensor was pointed out explicitly, but we've chosen it as our definition as it makes the connection between K\"ahler and Riemannian geometry very clear. It is my favorite condition, as it suggests that we should investigate what the isomorphism between harmonic forms and the cohomology groups of a compact smooth manifold looks like in the K\"ahler case, which leads quickly to the invention of Hodge theory. There are already plenty of books about Hodge theory, so we won't get into it here.


There are two constructions of manifolds from existing ones that preserve K\"ahler metrics:

\begin{prop}
  \begin{itemize}
  \item If $X$ is a K\"ahler manifold and $Y \subset X$ is a submanifold, then $Y$ is K\"ahler.
  \item If $X$ and $Y$ are K\"ahler manifolds, then $X \times Y$ is a K\"ahler manifold.
  \end{itemize}
\end{prop}

\begin{proof}
  Both points are easily proved using the closedness of the K\"ahler form. If $\omega$ is a differential form on $X$, then $d(f^*\omega) = f^*d\omega$ for all holomorphic maps $f$, so if $j : Y \hookrightarrow X$ is the inclusion then $j^*\omega$ is a K\"ahler form on~$Y$.%

Similarly, if $\omega_X$ and $\omega_Y$ are K\"ahler forms on $X$ and $Y$, then $\omega := \pi_X^*\omega_X + \pi_Y^*\omega_Y$ is a K\"ahler form on $X \times Y$.
\end{proof}

It is difficult to point out other many other general ways of constructing K\"ahler manifolds from existing ones. We can prove that the blow-up of a point on a compact K\"ahler manifold is K\"ahler, and that a projective bundle over a compact K\"ahler manifold is K\"ahler.

A fiber space over a compact base with compact fibers does not have to be K\"ahler even if the base and fibers are; in fact many examples of non-K\"ahler manifolds arise this way, like the Hopf and Iwasawa manifolds.

Small deformations of compact K\"ahler manifolds are K\"ahler~\cite{kodaira1960deformations}; large ones do not have to be~\cite{hironaka1962example}. The (smooth part of) the base of versal deformations of compact K\"ahler manifolds tends to be K\"ahler, because direct image sheaves are often positive~\cite{berndtsson2009curvature} and we can use K\"ahler metrics on the fibers to get a Hermitian metric on the relative canonical bundle. This is how the Weil--Petersson metrics for moduli of curves and K\"ahler--Einstein manifolds are constructed.

Interesting examples can often be constructed explicitly from projective manifolds by ad-hoc methods; Serre~\cite{serre1958topologie} showed that every finite group is the fundamental group of a projective variety, and Schreieder~\cite{schreieder2015construction} constructed compact manifolds with every possible Hodge diamond.


Another consequence of the vanishing of the torsion is that the curvature tensor of a K\"ahler metric has extra symmetries.

\begin{prop}
Let $R$ be the curvature tensor of a K\"ahler metric. Then
$$
R(\alpha,\beta,\gamma,\delta) = R(\gamma,\beta,\alpha,\delta)
\quad\text{and}\quad
R(\alpha,\beta,\gamma,\delta) = R(\alpha,\delta,\gamma,\beta).
$$
\end{prop}

\begin{proof}
The first Bianchi identity for a metric with vanishing torsion is $F \wedge \id = 0$. When we evaluate these $(2,1)$-form on tangent fields, we get
$$
F_{\alpha\beta}\gamma - F_{\gamma\beta}\alpha = 0
$$
% (1 2 3) = - (1 3 2) = (3 1 2) = -(3 2 1)
from which we conclude the first equality. The second follows from the first and the conjugate symmetry of the curvature form.
\end{proof}




\section{Algebraic K\"ahler curvature tensors}
\label{sec:algebraic-curvature-tensors}

Let $V$ be a complex vector space of dimension $n$. Let $\cc R$ be the subset of the space of $V \otimes \overline V \otimes V \otimes \overline V \to \kk C$ defined by elements $R$ that satisfy
$$
R(x, y, z, w) = \overline{R(y, x, w, z)}
\quad\text{and}\quad
R(x, y, z, w) = R(z, y, x, w)
$$
for all $x,y,z,w \in V$. These are the symmetries that the curvature tensor of a K\"ahler metric satisfies (the curvature tensor of a Hermitian metric only satisfies the conjugate symmetry). We discuss some properties of this set; our reference is \cite{algebraic-kahler-curvature}.

The set $\cc R$ is closed under addition and real multiplication, so it is a real subspace.


\begin{prop}
The set $\cc R$ is isomorphic to the space of Hermitian forms on $S^2 V$.
\end{prop}

\begin{proof}
Let $R$ be an element of our subspace. We define a bilinear form $b$ on $S^2 V$ by
$$
b(x \odot z, y \odot w)
= R(x, y, z, w).
$$
We have $R(x,y,z,w) = R(z,y,x,w)$ by hypothesis, and
$$
R(x,w,z,y)
= \overline{R(w,x,y,z)}
= \overline{R(y,x,w,z)}
= R(x,y,z,w)
$$
by using both properties of elements of $\cc R$. Our form $b$ is thus well-defined on the symmetric product of $V$, and Hermitian by the first property of elements of $\cc R$. The morphism this defines is $\kk R$-linear, and clearly injective.

Given a Hermitian form $b$ on $S^2V$, we also define
$$
R(x,y,z,w) = b(x \odot z, y \odot w).
$$
Then $R$ satisfies the properties required by the elements of $\cc R$. It is linear in its first and third variables, and conjugate linear in the others. It is thus a well-defined element of $\cc R$. The two maps defined here are inverses of each other, so the two subspaces are isomorphic.
\end{proof}


\begin{coro}
$$
\dim \cc R
= \binom{n+1}{2}^2.
$$
\end{coro}


\begin{proof}
The dimension of the space of Hermitian forms on a vector space of dimension $n$ has dimension $n^2$, and $S^k V$ has dimension $\binom{n + k - 1}{k}$.
\end{proof}


It's natural to wonder about the positivity properties of a tensor $R$ under this isomorphism. It is clear that if the image of a tensor is positive-definite, then the original tensor is Griffiths positive. The converse is false in general; Griffiths positivity does not imply that the image of the tensor is positive-definite, only that it is positive on vectors of the form $x \odot y$. Very surprisingly (to me, at least) the first counterexample of this can only show up in dimension~4:



\begin{lemm}
Let $V$ be a complex vector space. Then the map
$$
V \times V \to S^2 V,
\quad
(x, y) \mapsto x \odot y
$$
of topological spaces is continuous, and surjective if $\dim V \leq 3$.
\end{lemm}

\begin{proof}
  Let $(v_1, \ldots, v_n)$ be a basis of $V$. Then $(v_j \odot v_j ; v_j \odot v_k)_{1 \leq j \leq n; 1 \leq j < k \leq n}$ is a basis of $S^2V$. Here the meaning of the notation is that in the basis the elements $(v_j \odot v_j)$ come first, and then all elements $(v_j \odot v_k)$ with $j < k$. Let $(z_1, \ldots, z_n; w_1 \ldots, w_n)$ be coordinates for $V \times V$. The map in question is then equal to
$$
f(z_1, \ldots, z_n; w_1, \ldots, w_n)
=( z_j w_j; z_j w_k + z_k w_j )_{1 \leq j \leq n; 1 \leq j < k \leq n}.
$$
Its entries are polynomials, so it is continuous.

Let $a = (a_j; a_{jk})$ be an element of $S^2V$. We would like to solve the system of equations
\begin{align*}
  z_j w_j &= a_j &1 \leq j \leq n,\\
  z_j w_k + z_k w_j &= a_{jk} &1 \leq j < k \leq n.
\end{align*}
To this end, multiply each equation of the second form by $z_jz_k$. They become
$$
z_j^2 z_k w_k + z_jz_k^2 w_j = a_{jk} z_jz_k.
$$
Use now $z_jw_j = a_j$ and $z_kw_k = a_k$ to rewrite this as
$$
a_k z_j^2 + a_j z_k^2 = a_{jk} z_jz_k.
$$
This gives $\binom n2$ quadratic equations to solve simultaneously in $\kk C^n$. By B\'ezout that's going to be possible for all values of $a$ only when $\binom n2 \leq n$, which happens when $n \leq 3$.
\end{proof}

This suggests a notion of positivity for K\"ahler curvature tensors that should be stronger than Griffiths positivity in higher dimensions. It is related to notions of $k$-positivity for curvature forms on $E \otimes T_X$, as a tensor that is $k$-positive on $S^2(T_X)$ is $k$-semipositive on $T_X \otimes T_X$. It's not clear whether this distinction is meaningful or results in any interesting theorems about K\"ahler curvature tensors.



\paragraph{}





We often view a curvature form as a Hermitian form on $E \otimes T_X$. This corresponds here to the space of curvature tensors of Hermitian metrics. This space then has dimension $(n^2)^2 = n^4$. We do have that $\binom{n+1}{2}^2 \leq n^4$ with equality if and only if $n = 1$, which is a nice sanity check.

When $E = T_X$ we can also consider the space of Riemannian curvature tensors on the smooth manifold underlying $X$. The corresponding space of algebraic Riemannian curvature tensors has real dimension
$$
\frac{(2n)^2((2n)^2 - 1)}{12}
= \frac{n^2(4n^2 - 1)}{3}.
$$
We have
$$
\binom{n+1}{2}^2 \leq n^4 \leq \frac{n^2(4n^2 - 1)}{3}
$$
for all $n \geq 1$ with equality in either place if and only if $n = 1$, so in a given complex dimension there are more Hermitian curvature tensors than K\"ahler ones, and more Riemannian tensors than Hermitian ones. (Recall that a K\"ahler metric is Riemannian, but a Riemannian metric doesn't have to be compatible with any complex structure.) The space of K\"ahler tensors can be embedded in the space of Riemannian tensors (clearly), but Hermitian tensors do not embed there (as they're generally not the curvature tensors of Riemannian metrics).


\paragraph{}

A slightly surprising property of K\"ahler curvature tensors is that the holomorphic sectional curvature determines the whole tensor. The holomorphic bisectional curvature clearly does so, but it's not obvious that restricting to the diagonal doesn't lose information.

\begin{prop}
If $R_1$ and $R_2$ are elements of $\cc R$ such that
$$
R_1(x,x,x,x) = R_2(x,x,x,x)
$$
for all $x \in V$, then $R_1 = R_2$.
\end{prop}

\begin{proof}
We apply our favorite isomorphism and obtain two Hermitian forms $h_1$ and $h_2$ on $S^2V$. Our hypothesis is that $h_1(x \odot x, x \odot x) = h_2(x \odot x, x \odot x)$ for all $x \in V$.

We first show that this implies that $h_1(x \odot y, x \odot y) = h_2(x \odot y, x \odot y)$ for all $x, y \in V$. The result follows from this by a standard polarization argument for Hermitian forms. To do so, we take $t \in \kk C$ and an arbitrary Hermitian form $h$ and calculate
\begin{align*}
&h((x + ty) \odot (x + ty), (x + ty) \odot (x + ty))
\\
&= h(x \odot x, x \odot x) + 2 h(x \odot x, tx \odot y) + h(x \odot x, t^2 y \odot y)
\\
&\qquad{}
+ 2 h(t x \odot y, x \odot x) + 4h(t x \odot x, tx \odot y) + 2h(t x \odot x, t^2 y \odot y)
\\
&\qquad{}
+ h(t^22 y \odot y, x \odot x) + 2h(y \odot y, tx \odot y) + h(t^2 y \odot y, t^2 y \odot y).
\end{align*}
Adding the expressions for $t$ and $-t$ we get
$$
\displaylines{
h((x + ty) \odot (x + ty), (x + ty) \odot (x + ty))
+ h((x - ty) \odot (x - ty), (x - ty) \odot (x - ty))
\hfill\cr\hfill{}
= 2 h(x \odot x, x \odot x)
+ \overline t^2 h(x \odot x, y \odot y)
+ 2 |t|^2 h(x \odot y, x \odot y)
\cr\hfill{}
+ t^2 h(y \odot y, x \odot x)
+ 2 |t|^4 h( y \odot y, y \odot y).
}
$$
Adding up what we get for $t = 1$ and $t = i$ then gives
\begin{align*}
&h((x + y) \odot (x + y), (x + y) \odot (x + y))
\\
&\qquad{}
+ h((x - y) \odot (x - y), (x - y) \odot (x - y))
\\
&\qquad{}
+ h((x + iy) \odot (x + iy), (x + iy) \odot (x + iy))
\\
&\qquad{}
+ h((x - iy) \odot (x - iy), (x - iy) \odot (x - iy))
\\
&=
4 h(x \odot x, x \odot x)
+ h(x \odot x, y \odot y)
+ 2 h(x \odot y, x \odot y)
+ h(y \odot y, x \odot x)
\\
&\qquad{}
- h(x \odot x, y \odot y)
+ 2 h(x \odot y, x \odot y)
- h(y \odot y, x \odot x)
+ 4 h(y \odot y, y \odot y)
\\
&= 4 \bigl(
h(x \odot x, x \odot x)
+ h(x \odot y, x \odot y)
+ h(y \odot y, y \odot y)
\bigr).
\end{align*}
The moral of this is that we can express $h(x \odot y, x \odot y)$ as a sum of values of the form $h(z \odot z, z \odot z)$. Thus if $h_1$ and $h_2$ agree on $x \odot x$ for all $x$, they agree for all $x \odot y$. Using the standard polarization identity for Hermitian forms we now conclude that $h_1(x \odot y, z \odot w) = h_2(x \odot y, z \odot w)$ for all $x,y,z,w \in V$.
\end{proof}

There's an alternate proof of this fact in \cite[Lemma~7.19]{zheng2000complex} for those who find this argument boring. I can't say I love it, but it does show that the reason this works is because being able to square vector elements lets us run the polarization argument twice to bootstrap ourselves from $x \odot x$ to $x \odot y$ to all decomposable tensors.

I only know of one place in the wild where knowledge of the holomorphic sectional curvature is used to compute the rest of the curvature tensor: Siu~\cite{siu1986curvature} does this in an impressive paper where he computes the curvature of the Weil--Petersson metric on moduli spaces of manifolds with negative first Chern class. Reading that paper, one gets the sense that there are people who do differential geometry, and the rest of us.

Knowing a polarization identity like this lets us prove things like that a bound on the holomorphic sectional curvature extends to a bound on the whole curvature tensor. One can prove that if $|H(x)| \leq C$ for all $x$, then $|B(x,y)| \leq 6 C$ for all $x,y$. From there we can polarize again to obtain $|R(x,y,z,w)| \leq 24C |x||y||z||w|$ for all $x,y,z,w$. I believe this is mostly useful to then quote general structure results from Riemannian geometry, but don't remember particular applications. These bounds are obtained by applying the triangle inequality to cases where equality is clearly not obtained and are likely not optimal. I don't know what the optimal bounds on $B$ and $R$ are given a bound on $H$.\footnote{If I had to guess I'd think the optimal bound would be attained on constant holomorphic curvature, where we have $|H| = C \implies |B| \leq 2C$.} It doesn't seem to be a question whose answer would have applications.

\paragraph{}
A Hermitian metric generally has three different Ricci forms. For a K\"ahler metric, these are all equal.

\begin{prop}
Let $R \in \cc R$. All the Ricci forms of $R$ are equal.
\end{prop}

\begin{proof}
  Pick a Hermitian inner product $h$ on $V$ and an orthonormal basis $(v_1,\ldots,v_n)$. We want to compare
\begin{align*}
r_1(x,y) &= \sum_{j=1}^n R(x,y,v_j,v_j),
\\
r_2(x,y) &= \sum_{j=1}^n R(v_j, v_j, x,y),
\\
r_3(x,y) &= \sum_{j=1}^n R(x,v_j,v_j,y).
\end{align*}
By the symmetries of K\"ahler curvature tensors we have $R(x,y,v_j,v_j) = R(x,v_j,v_j,y)$, so $r_1 = r_3$. Again by the same, we have $R(x,v_j,v_j,y) = R(v_j,v_j,x,y)$, so $r_3 = r_2$.
\end{proof}

For a K\"ahler metric we thus speak of \emph{the} Ricci form and tensor of the metric. It also follows that the Ricci form of a K\"ahler metric is closed.


\paragraph{}
We can again contract the Ricci tensor of a K\"ahler metric and get the \emph{scalar curvature} of the metric. It is equal to the scalar curvature of the underlying Riemannian metric.

\begin{defi}
The \emph{scalar curvature} of a K\"ahler metric with curvature tensor $R$ is
$$
s = \sum_{k=1}^n r(v_k,v_k) = \sum_{j,k=1}^n R(v_j,v_j,v_k,v_k).
$$
\end{defi}


\begin{prop}
Let $\omega$ be the K\"ahler form of a K\"ahler metric, and let $r$ be its Ricci-form. Then
$$
s \, \omega\^{n} = r \wedge \omega\^{n-1}.
$$
\end{prop}

\begin{proof}
The right-hand side is equal to $\Lambda(r) \omega\^n$, which is just the contraction of the Ricci-form by the metric, which is the scalar curvature.
\end{proof}

The Ricci-form of a K\"ahler metric represents the cohomology class $c_1(X)$, as it is the curvature form of a Hermitian metric on $-K_X$. The above proposition then implies that on a compact K\"ahler manifold, the integral of the scalar curvature is a cohomological invariant.



\paragraph{}

Let's discuss some positivity properties of the various curvature tensors we have. The tangent bundle is just another vector bundle, so we can speak of Nakano-positivity and Griffiths-positivity of its curvature. Beyond that, the derived curvature tensors can also be positive. We can summarize the relationships between the positivity of the various tensors as below.


\begin{prop}
\label{prop:derived-tensor-positivity}
  Let $R$ be a K\"ahler curvature tensor.
  \begin{enumerate}
    \item If $R$ is Griffiths-positive, then $B$ is positive.
    \item If $B$ is positive, then $H$ and $r$ are positive.
    \item If $H$ is positive, then $s$ is positive.
    \item If $r$ is positive, then $s$ is positive.
  \end{enumerate}
\end{prop}

The proposition remains true if we replace ``positive'' with ``semipositive'', ``negative'' or ``seminegative'' everywhere.

\begin{proof}
  \begin{enumerate}
  \item True by definition of $B$.
  \item The positivity of $H$ is obvious. We note that
    $$
    r(x,x) = \sum_{j=1}^n R(x,x,v_j,v_j) = |x|^2 \sum_{j=1}^n B(x,v_j)
    $$
    from which the positivity of $r$ follows.
  \item Follows from \thref{lemm:holomorphic-sectional-to-scalar}.
  \item The scalar curvature is the trace of the Ricci form, so this is obvious.
  \end{enumerate}
\end{proof}


\begin{lemm}[Berger~\cite{berger1965varietes}\footnote{Apparently. I haven't found a copy of this source to double-check. I believe one is available in the library in CIRM in Marseille.  Diverio may know where to find a copy, and Yang or Zheng possibly do as well; at least they've all cited this source.}]
\label{lemm:holomorphic-sectional-to-scalar}
Let $h$ be a Hermitian inner product on $V$, and $H$ and $s$ the holomorphic sectional and scalar curvatures of a K\"aher curvature tensor. Then
$$
\int_{S(V,h)} H(x) \, d\sigma(x) = \frac{\Vol(S^{2n-1})}{n(n+1)} s.
$$
\end{lemm}

\begin{proof}
  We pick an orthonormal basis $(v_1,\ldots,v_n)$ of $V$ and write $z = \sum_{j=1}^n z_j v_j$. On the unit sphere we then have
$$
H(z)
= R(z,z,z,z)
= \sum_{j,k,l,m=1}^n z_j \bar z_k z_l \bar z_m R_{jklm}.
$$
Consider the real and imaginary parts of the polynomial here. By inspecting degrees, we see that the result is odd in one of its variables and so its integral over the unit sphere vanishes unless $j = k$ and $l = m$ or $j = m$ and $l = k$. We thus have to evaluate
$$
\int_{S^{2n-1}} |z_j|^4 \, d\sigma
\quad\text{and}\quad
\int_{S^{2n-1}} |z_j|^2 |z_k|^2 \, d\sigma
\quad(j \not= k).
$$
After Folland~\cite{folland}, we know that
\begin{align*}
\int_{S^{2n-1}} d\sigma
&= \Vol(S^{2n-1}) = \frac{2\Gamma(\tfrac12)^{2n}}{\Gamma(n)},
\\
\int_{S^{2n-1}} x_j^4 \, d\sigma
&= \frac{2\Gamma(\tfrac 52)\Gamma(\tfrac 12)^{2n-1}}{\Gamma(n+2)}
\\
&= \frac{\Gamma(\tfrac52)}{\Gamma(\tfrac12)(n+1)n} \Vol(S^{2n-1})
= \frac{3}{4n(n+1)} \Vol(S^{2n-1}),
\\
\int_{S^{2n-1}} x_j^2 x_k^2 \, d\sigma
&= \frac{2\Gamma(\tfrac32)^2\Gamma(\tfrac 12)^{2n-2}}{\Gamma(n+2)}
\\
&= \frac{\Gamma(\tfrac32)^2}{\Gamma(\tfrac12)^2 (n+1)n} \Vol(S^{2n-1})
= \frac{1}{4n(n+1)} \Vol(S^{2n-1})
\end{align*}
for real variables $x_j, x_k$ when $j \not= k$. As
\begin{align*}
|z_j|^4 &= x_j^4 + 2 x_j^2 y_j^2 + y_j^4,
\\
|z_j|^2 |z_k|^2 &= x_j^2 x_k^2 + x_j^2 y_k^2 + y_j^2 x_k^2 + y_j^2 y_k^2
\end{align*}
we get that
\begin{align*}
\int_{S^{2n-1}} |z_j|^4 \, d\sigma
&= \frac{2}{n(n+1)} \Vol(S^{2n-1}),
\\
\int_{S^{2n-1}} |z_j|^2 |z_k|^2 \, d\sigma
&= \frac{1}{n(n+1)} \Vol(S^{2n-1})
\quad(j \not= k).
\end{align*}
This finally gives
\begin{align*}
\int_{S^{2n-1}} H(z) \,d\sigma
&= \sum_{j=1}^n \int_{S^{2n-1}} |z_j|^4 \, d\sigma
+ \sum_{j \not= k}^n \int_{S^{2n-1}} |z_j|^2|z_k|^2 \, d\sigma
\\
&= \frac{2}{n(n+1)} \Vol(S^{2n-1}) \sum_{j=1}^n R(v_j,v_j,v_j,v_j)
\\
&\qquad{}+ \frac{1}{n(n+1)} \Vol(S^{2n-1}) \sum_{j \not= k}^n R(v_j,v_j,v_k,v_k)
\\
&\qquad{}+
\frac{1}{n(n+1)} \Vol(S^{2n-1}) \sum_{j \not= k}^n R(v_j,v_k,v_k,v_j)
\\
&= \frac{1}{n(n+1)} \Vol(S^{2n-1}) \sum_{j,k=1}^n R(v_j,v_j,v_k,v_k)
\\
&\qquad{}+
\frac{1}{n(n+1)} \Vol(S^{2n-1}) \sum_{j,k=1}^n R(v_j,v_k,v_k,v_j)
\\
&= \frac{2}{n(n+1)} \Vol(S^{2n-1}) \sum_{j,k=1}^n R(v_j,v_j,v_k,v_k)
\\
&= \frac{1}{n(n+1)} \Vol(S^{2n-1}) \cdot s
\end{align*}
where we have used the symmetries of the curvature tensor.
\end{proof}


\begin{proof}[Second attempt]
Fix a Hermitian inner product $h$ on $V$ and let it induce an inner
product $g$ on $S^2V$. For any Hermitian form $b$ on $S^2V$, we have
$$
\tr_{g} b
= C \int_{S(S^2V, g)} g(g^{-1}b x, x)
= C \int_{S(S^2V, g)} b(x, x).
$$
Taking $b = g$ gives $\dim S^2V = C \Vol(S(S^2V, g))$, which determines $C$. Pick an orthonormal basis $(v_1,\ldots,v_n)$ of $V$, and let $(v_j v_k)$ be the induced orthonormal basis of $S^2V$. If $x = \sum x_{jk} v_{jk}$, we have
$$
b(x, x)
= \sum_{jk,lm} x_{jk} \overline x_{lm} b(v_{jk}, v_{lm}).
$$
The integral of these over the unit sphere is zero unless $j = l$ and $k = m$. We have
$$
\displaylines{
b(v_{jk}, v_{jk})
= b(v_j v_k, v_j v_k)
= b(\alpha_{jk}, \alpha_{jk})
+ b(\beta_{jk}, \beta_{jk})
+ b(\gamma_{jk}, \gamma_{jk})
\hfill\cr\hfill{}
+ b(\delta_{jk},\delta_{jk})
- b(v_{jj},v_{jj})
- b(v_{kk},v_{kk}),
}
$$
where $\alpha_{jk} = (v_j + v_k)^2/2$,
$\beta_{jk} = (v_j - v_k)^2/2$,
$\gamma_{jk} = (v_j + iv_k)^2/2$
and $\delta_{jk} = (v_j - iv_k)^2/2$.

These greek vectors are unit vectors. How do we reduce the integral to the unit sphere in $V$?
\end{proof}

The nicest thing one can say about this proof is that it works. I really want there to be a proof of this lemma based around the polarization identity we used to show the holomorphic sectional curvature determines the whole curvature tensor, but can't figure out how to reduce the integral over the unit sphere in $S^2V$ (from which we get equality with the trace) to one over the unit sphere in $V$. There's of course the Veronese embedding $j : V \to S^2V$ given by $v \mapsto v \odot v$, but it's not clear why pulling anything back by it helps.


The reverse implications in \thref{prop:derived-tensor-positivity} all fail. The reader can amuse themselves by constructing algebraic curvature tensors that exhibit those failures.\footnote{If the reader does not find this amusing, they can see how we construct some such tensors in the next subsection.}

The relationship between the holomorphic sectional and Ricci curvatures is not obvious. The positivity or negativity of one does not imply the positivity or negativity of the other. For example, there are known natural metrics with positive holomorphic sectional curvature that do not have positive Ricci curvature~\cite{hitchin1975curvature,alvarez2016positive,yang2019hirzebruch}. In the next section we construct some algebraic curvature tensors that show this concretely.

\begin{exam}
The holomorphic sectional curvature determines the whole curvature tensor, so having trivial holomorphic sectional curvature implies trivial Ricci curvature. The converse is false:

Let $X$ be a compact K\"ahler manifold with $c_1(X) = 0$, and let $h$ be a Ricci-flat K\"ahler metric on $X$. Recall also that a compact K\"ahler manifold is flat if and only if it is a complex torus. If $X$ is not a complex torus (like a Calabi--Yau manifold, K3 surface or a hyperk\"ahler manifold~\cite{beauville1983}), then $R$ is nontrivial, so the holomorphic sectional curvature of $h$ is also nontrivial. By the above lemma, its average over the unit sphere is zero at any point, so it cannot have a definitive sign. In particular, zero Ricci curvature does not imply trivial holomorphic sectional curvature.
\end{exam}

There are some results where the existence of a metric with either tensor positive is used to prove the existence of a different metric with the other tensor positive. For example, Kobayashi conjectured that a manifold that admits a metric with negative holomorphic sectional curvature should have ample canonical bundle; this was proved by Wu and Yau~\cite{wu2016negative} for projective manifolds. See also the survey by Diverio~\cite{diverio2020kobayashi}.



\paragraph{}
We didn't talk about Nakano positivity when discussing positivity of K\"ahler curvature tensors. The reason is that none of those tensors are Nakano positive or negative.

\begin{prop}
Let $R$ be an algebraic K\"ahler curvature tensor, and let $Q$ be the Hermitian form it defines on $V \otimes V$. Then $\dim \Ker Q \geq \binom n2$.
\end{prop}

\begin{proof}
The form $Q$ is defined by
$$
Q(x \otimes z, y \otimes w) = R(x,y,z,w)
$$
for $x,y,z,w \in V$ and extended by linearity. When viewed as a linear morphism $V \otimes V \to \overline V^* \otimes \overline V^*$, the symmetries of the curvature tensor entail that
$$
Q(x \otimes z) = Q(z \otimes x)
$$
for all $x, z \in V$.

Let $(v_1,\ldots,v_n)$ be a basis of $V$, so $(v_j \otimes v_k)_{1 \leq j,k \leq n}$ is a basis of $V \otimes V$. Then $v_j \otimes v_l - v_l \otimes v_j \in \Ker Q$ for all $j, l$, and there are $\binom n2$ ways of picking nontrivial such elements.
\end{proof}

\begin{coro}
\label{coro:no-nakano-positive}
A K\"ahler metric is never $2$-positive. In particular, it is never Nakano positive or Nakano negative.
\end{coro}

I haven't seen this proposition or corollary pointed out in the literature, but they must be known to the experts.


The Nakano vanishing theorem is often used to prove that there is no Nakano positive metric on the projective space~\cite[Example~8.4]{demailly-complex}: One computes its cohomology groups and notes that some that the Nakano vanishing theorem would annihilate are in fact nontrivial. This is not a case of nuking a mosquito, as we can only say that no \emph{K\"ahler} metric on the projective space is Nakano positive, and can say nothing about \emph{Hermitian} metrics.



\subsection{Constructing algebraic curvature tensors}


An easy way to construct elements of $\cc R$ is to pick a basis for $V$ and just write down some tensors. This is in some way unsatisfactory because we'd like to see curvature tensors that arise in the wild.\footnote{We're not above doing this, we just feel bad about it.} To do so we'd like to have coordinate-invariant ways of constructing elements of $\cc R$. We point out two of those here.

Suppose we have a Hermitian form $b$ on $V$. It induces a Hermitian form on $S^2 V$, defined by
$$
(b \odot b)(x \odot y, z \odot w)
= \tfrac 12 \bigl( b(x,z)b(y,w) + b(x,w)b(y,z) \bigr).
$$
% x \odot z = 1/2 (x \otimes z + z \otimes x)
% y \odot w = 1/2 (y \otimes w + w \otimes y)
% < x \odot z, y \odot w >
% = < x \otimes z + z \otimes x, y \otimes w + w \otimes y >
% = 1/4 (< x \otimes z, y \otimes w > + < x \otimes z, w \otimes y>
% + < z \otimes x, y \otimes w > + < z \otimes x, w \otimes y>)
% = 1/4 (<x,y> <z,w> + <x,w> <z,y> + <z,y> <x,w> + <z,w> <x,y>)
% = 1/2 (<x,y> <z,w> + <x,w> <z,y>)
This is the complex-geometric version of the Kulkarni--Nomizu product. It has the property that if $(v_1,\ldots,v_n)$ is an orthonormal basis of $V$, then $(v_j \odot v_k)_{1 \leq j \leq k \leq n}$ is an orthonormal basis of $S^2V$.

If $a$ and $b$ are Hermitian forms on $V$ we can polarize this identity and get a Hermitian form on $S^2V$ by setting
$$
\displaylines{
(a \odot b)(x \odot y, z \odot w)
= \tfrac 14 \bigl(a(x,y) b(z,w)
+ a(x,w) b(z,y)
\hfill\cr\hfill{}
+ a(z,y) b(x,w)
+ a(z,w) b(x,y)\bigr).
}
$$

\begin{prop}
  If $a$ and $b$ are Hermitian forms on $V$, then $a \odot b$ is a Hermitian form on $S^2V$, and thus defines a K\"ahler curvature tensor. Its derived curvature tensors are
\begin{align*}
H(x) &= \frac{a(x,x)b(x,x)}{h(x,x)^2},
\\
r(x,y) &= \tfrac 14 \bigl(
a(x,y) \langle b, h \rangle
+ \langle a(x), b(y) \rangle
+ \langle b(x), a(y) \rangle
+ b(x,y) \langle a, h \rangle
\bigr),
\\
s &= \tfrac 12 \bigl(
\langle a, h \rangle \langle b, h \rangle
+ \langle a, b \rangle
\bigr),
\end{align*}
where $h$ is a Hermitian inner product on $V$ and we write $a(x)$ for the value of the Hermitian form viewed as a morphism $a : V \to \overline V^*$.

In particular, for an inner product $h$ the tensor $h \odot h$ is Griffiths-positive, has constant holomorphic scalar curvature equal to $1$, its Ricci tensor is $r = \frac{n+1}{2} h$, and its scalar curvature is $s = \frac{n(n+1)}2$.
\end{prop}

\begin{proof}
  Only the claim about the Ricci tensor is not clear. The way to interpret the expression $ah^{-1}b$ is to view $a, b, h$ as linear morphisms $V \to \overline V^*$ and use that $h$ is an isomorphism.

Let's pick an orthonormal basis $(v_1, \ldots, v_n)$ for $h$. The Ricci tensor of $a \odot b$ is then the trace of the form with respect to the induced inner product on $S^2V$. We'll compute the trace $\sum (a\odot b) (x,y,v_j,v_j)$.

Recall that $\sum_j a(v_j,v_j) = \langle a, h \rangle$, where the inner product is the one that $h$ induces on $\Hom(V, \overline V^*)$. This takes care of the first and fourth terms in the trace of the curvature tensor.

Now write $x = \sum_{l} x_l v_l$ and $y = \sum_{k} y_k v_k$. Then
\[
a(x) = \sum_{jk} a_{jk}x_k v_j^*
\qandq
b(y) = \sum_{jk} b_{jk} y_k v_j^*
\]
so
\begin{align*}
\langle a(x), b(y) \rangle
= \sum_{j,k,m} a_{jk} \bar b_{jm} x_k \bar y_m
&= \sum_{j,k,m} a_{jk} b_{mj} x_k \bar y_m
\\
&= \sum_{j}a(x)(\bar v_j) b(v_j)(\bar y)
= \sum_j a(x,v_j) b(v_j, y)
\end{align*}
This shows that $\sum a(x,v_j)b(v_j,y) = \langle a(x), b(y) \rangle$, and the remaining term is similar.

For an inner product $h$, we have
$$
(h \odot h)(x \odot y, x \odot y)
= \tfrac12 \bigl(
h(x,x)h(y,y) + |h(x,y)|^2
\bigr),
$$
which is positive for all $x$ and $y$, so $h \odot h$ is Griffiths positive. The other claims are clear.
\end{proof}



\begin{exam}
\label{exam:not-griffiths-positive}
Let $V$ be of dimension $2$, let $h$ be a Hermitian inner product, and let $a$ be given by
$$
A = \begin{pmatrix} 1 & 0 \\ 0 & -1 \end{pmatrix}
$$
in an orthonormal basis. Consider the curvature tensor $R = a \odot a$. Its holomorphic sectional curvature is
$$
H(x) = \frac{a(x,x)^2}{h(x,x)^2} \geq 0,
$$
but its holomorphic bisectional curvature is
$$
B(x,y) = \frac{a(x,x)a(y,y)}{h(x,x)h(y,y)}
$$
which does not have a definite sign (take $x$ such that $a(x,x) < 0$ and $y$ such that $a(y,y) > 0$). Thus the holomorphic sectional curvature does not dominate the holomorphic bisectional curvature.

We have $\langle a, h \rangle = 0$ by construction, and the Ricci tensor of $R$ is
$$
r(x,y) = \tfrac12 h(x, y),
$$
which is positive definite, while the holomorphic sectional curvature is only semipositive. This also shows that Ricci positivity does not imply positivity of the holomorphic bisectional curvature.
\end{exam}


Another way of constructing these kinds of elements is to pull back Hermitian forms to the symmetric product. In that case we get weaker positivity properties than above, simply because any linear morphism we pull back by will have a nontrivial kernel:

\begin{prop}
\label{prop:algebraic-curvature-second-fundamental}
Let $f \in \Hom(S^2V, W)$ and let $h_W$ be a Hermitian form on $W$. Then
$$
(f^*h_W)(x \odot y, z \odot w)
= h_W(f(x \odot y), f(z \odot w))
$$
is a Hermitian form on $S^2V$, and thus a K\"ahler curvature tensor.

If $h_W$ is a Hermitian inner product, then $f^*h_W$ has semipositive holomorphic sectional curvature and is Griffiths-semipositive. Its scalar curvature is $|f|^2$, where the norm is the Fr\"obenius norm defined by $h_W$ and the inner product on $V$.
\end{prop}

\begin{proof}
All of the claims are clear except maybe the one about the scalar curvature. Let $(v_1, \ldots, v_n)$ be an orthonormal basis of $V$. Then the scalar curvature is
\[
\sum_{j \leq k} |f(v_j \odot v_k)|^2_{h_W},
\]
which is the sum of the squares of norms of the column vectors of $f$ in the basis. This is the square of the Fr\"obenius norm of $f$.
\end{proof}



These constructions give tensors that are of fundamental interest. The Kulkarni--Nomizu product yields tensors of constant holomorphic sectional curvature, like that of the Fubini--Study or Bergman metrics. The pullback construction yields the term that the second fundamental form of a submanifold contributes to its curvature tensor.



The scalar curvature of a tensor $R$ is just the inner product $\langle R, h \odot h \rangle$, where $h$ is an inner product on $V$. The space of curvature tensors with given scalar curvature is thus a hyperplane in $\cc R$. On the other extreme, since the holomorphic sectional curvature determines the whole tensor, the space of tensors with constant holomorphic sectional curvature is a line (one tensor for every value of the constant). There is a middle ground -- the space of K\"ahler--Einstein tensors whose Ricci tensor is a multiple of the inner product -- that we can't describe very accurately. It would be nice to find an invariant construction that yields tensors in this space. We can bodge some of those from the Kulkarni--Nomizu product in special cases\footnote{When $\dim V$ is even we can take $a$ to be the form with alternating $1$ and $-1$ on its diagonal and zeros elsewhere, and $b = -a$. Then $a$ and $b$ are orthogonal to $h$ but $\langle a(x), b(x) \rangle = - |x|^2$, so $r(x,x) = -\frac12 |x|^2$. Other ``square roots'' of the inner product that are orthogonal to it give the same.}, but don't have an estimate how much of the space those cases give.



\subsection{Positivity counterexamples in dimension two}

Let's examine the first nontrivial case, when $\dim V = 2$. We pick an inner product on $V$ and let $(v_1,v_2)$ be an orthonormal basis. Then $(v_1 \odot v_1, v_1 \odot v_2, v_2 \odot v_2)$ is an orthonormal basis of $S^2V$ under the inner product induced by the one we picked. A curvature tensor $R$ identifies with a Hermitian form on $S^2V$, that is, with a Hermitian matrix
$$
h = \begin{pmatrix}
  h_{11} & h_{12} & h_{13} \\
  h_{21} & h_{22} & h_{23} \\
  h_{31} & h_{32} & h_{33}
\end{pmatrix}.
$$
The various derived positivity notions we've seen can now be translated as follows:

\smallskip\noindent$\bullet$\quad
  $R$ has positive holomorphic sectional curvature if $h$ is positive on decomposable tensors $x \odot x$. Write $x = x_1v_1 + x_2v_2$. Then $x \odot x = x_1^2 v_1 \odot v_1 + 2x_1x_2 v_1 \odot v_2 + x_2^2 v_2 \odot v_2$, and
\begin{align*}
H(x)
&= \frac{R(x,x,x,x)}{|x|^4}
\\
&= \begin{pmatrix} \bar x_1^2 & 2\bar x_1\bar x_2 & \bar x_2^2 \end{pmatrix}
\begin{pmatrix}
  h_{11} & h_{12} & h_{13} \\
  h_{21} & h_{22} & h_{23} \\
  h_{31} & h_{32} & h_{33}
\end{pmatrix}
\begin{pmatrix} x_1^2 \\ 2x_1x_2 \\ x_2^2 \end{pmatrix}
\biggm/
(|x_1|^2 + |x_2|^2)^2.
\end{align*}
Calculating the matrix product, we get
\begin{align*}
\begin{pmatrix}
  h_{11} & h_{12} & h_{13} \\
  h_{21} & h_{22} & h_{23} \\
  h_{31} & h_{32} & h_{33}
\end{pmatrix}
\begin{pmatrix} x_1^2 \\ 2x_1x_2 \\ x_2^2 \end{pmatrix}
= \begin{pmatrix}
h_{11} x_1^2 + 2h_{12} x_1x_2 + h_{13} x_2^2
\\
h_{21} x_1^2 + 2h_{22} x_1x_2 + h_{23} x_2^2
\\
h_{31} x_1^2 + 2h_{32} x_1x_2 + h_{33} x_2^2
\end{pmatrix}
\end{align*}
so
\begin{align*}
R(x,x,x,x)
&=
(h_{11} x_1^2 + 2h_{12} x_1x_2 + h_{13} x_2^2)\bar x_1^2
\\
&\qquad{}+
(h_{21} x_1^2 + 2h_{22} x_1x_2 + h_{23} x_2^2)2 \bar x_1 \bar x_2
\\
&\qquad{}+
(h_{31} x_1^2 + 2h_{32} x_1x_2 + h_{33} x_2^2) \bar x_2^2.
\end{align*}

\smallskip\noindent$\bullet$\quad
The Ricci tensor of $R$ is
\begin{align*}
  r(x,y)
  &= R(x,y,v_1,v_1) + R(x,y,v_2,v_2)
  \\
  &= h(x \odot v_1, y \odot v_1) + h(x \odot v_2, y \odot v_2).
\end{align*}
Write $x = x_1v_1 + x_2v_2$ and $y = y_1 v_1 + y_2v_2$. Then
\begin{align*}
h\bigl((x_1 v_1 + x_2v_2) \odot v_1, (y_1v_1 + y_2v_2) \odot v_1 \bigr)
&= x_1\bar y_1 h_{11} + x_1\bar y_2 h_{12} + x_2\bar y_1 h_{21} + x_2\bar y_2 h_{22}
\\
h\bigl((x_1 v_1 + x_2v_2) \odot v_2, (y_1v_1 + y_2v_2) \odot v_2 \bigr)
&= x_1\bar y_1 h_{22} + x_1\bar y_2 h_{23} + x_2\bar y_1 h_{32} + x_2\bar y_2 h_{33}
\end{align*}
so
$$
r(x,y)
= \begin{pmatrix}\bar y_1 & \bar y_2\end{pmatrix}
  \begin{pmatrix}
  h_{11} + h_{22}  & h_{21} + h_{32}
    \\
  h_{12} + h_{23} & h_{22} + h_{33}
  \end{pmatrix}
  \begin{pmatrix}x_1 \\ x_2 \end{pmatrix}.
$$


\smallskip\noindent$\bullet$\quad
The scalar curvature of $R$ is the trace of the Ricci tensor, so
$$
s = h_{11} + 2 h_{22} + h_{33}.
$$

\smallskip
Consider a diagonal matrix $h$. The holomorphic sectional curvature of the corresponding tensor $R$ is
$$
H(x,y)
= \frac{h_{11} |x|^4 + 4 h_{22} |x|^2|y|^2 + h_{33}|y|^4}{(|x|^2+|y|^2)^2},
$$
its Ricci tensor is
$$
r(x,y) = (h_{11} + h_{22})|x|^2 + (h_{22} + h_{33})|y|^2
$$
and its scalar curvature is
$$
s = h_{11} + 2h_{22} + h_{33}.
$$
We can now setup a variety of situations to show that positivity properties only ``flow downwards''.

\begin{exam}
Take $h_{11} > 0$, $h_{33} < 0$ and set $2h_{22} + h_{33} = 0$. Then we get a tensor such that the scalar curvature is positive, but the Ricci tensor has no definite sign.
\end{exam}

\begin{exam}
Note that
$$
H(x,x) = \tfrac 14 \bigl(h_{11} + 4h_{22} + h_{33} \bigr).
$$
We can then clearly pick coefficients such that $s > 0$ but $H(x,x) < 0$. Thus positive scalar curvature does not imply positive holomorphic sectional curvature.
\end{exam}

\begin{exam}
If we pick coefficients such that $h_{11} + h_{22} > 0$ and $h_{22} + h_{33} > 0$ but $h_{11} + 4h_{22} + h_{33} < 0$ we then have a tensor with $r > 0$ everywhere but $H < 0$ in some directions.
\end{exam}

\begin{exam}
The term that controls the sign of the holomorphic sectional curvature can be written as (picking $h_{11} = 1$ to simplify)
$$
(|x|^2 - \sqrt{h_{33}}|y|^2)^2 + (4h_{22} + 2\sqrt{h_{33}})|x|^2|y|^2.
$$
The corresponding Ricci tensor is
$$
r(x,y) = (1 + h_{22})|x|^2 + (h_{22} + h_{33}) |y|^2.
$$
Set $h_{22} = -\sqrt{h_{33}}/2$. Then $H \geq 0$, but
$$
r(x,y) = (1-\sqrt{h_{33}}/2)|x|^2 + (h_{33} - \sqrt{h_{33}}/2)|y|^2
$$
and taking $h_{33} > 4$ yields a Ricci tensor that is negative in some directions. Taking $h_{22} = -\sqrt{h_{33}}/2 + \varepsilon$ then gives $H > 0$ everywhere but $r < 0$ somewhere.
\end{exam}

We already saw that having $r > 0$ does not imply the holomorphic bisectional curvature is positive, and that $H \geq 0$ does not imply that either (\thref{exam:not-griffiths-positive}).


\begin{exam}
Let $V$ be of dimension two with a Hermitian form $h$ and orthonormal basis $(e_1,e_2)$. We consider the curvature tensor defined by
$$
R(x,y,z,w) = \tfrac12( h(x,y)h(z,w) + h(x,w)h(z,y)).
$$
It can be viewed as a Hermitian form both on $V \otimes V$ and $S^2V$. These spaces have orthonormal bases $(e_1 \otimes e_1, e_1 \otimes e_2, e_2 \otimes e_1, e_2 \otimes e_2)$ and $(e_1 \odot e_1, e_1 \odot e_2, e_2 \odot e_2)$, and the matrices of the Hermitian forms defined by $R$ in these bases are
$$
A = \begin{pmatrix}
  1 & 0 & 0 & 0 \\
  0 & \tfrac12 & \tfrac12 & 0 \\
  0 & \tfrac12 & \tfrac12 & 0 \\
  0 & 0 & 0 & 1
\end{pmatrix}
\qandq
B = \begin{pmatrix}
  1 & 0 & 0 \\
  0 & 1 & 0 \\
  0 & 0 & 1
\end{pmatrix}.
$$
The form on $S^2V$ is positive-definite, but the one on $V \otimes V$ has a one-dimensional kernel and is positive on its complement. The tensor $R$ is thus Griffiths positive and Nakano semipositive. This is the most positivity one can hope for in lieu of \thref{coro:no-nakano-positive}.

The tensor $R$ is the curvature tensor of the Fubini--Study metric on the projective space, as we will see later.
\end{exam}



\begin{exam}
  Sticking with the same setup as in the last example, we consider the Hermitian form $b$ on $V \otimes V$ defined by everyone's favorite matrix
$$
\begin{pmatrix}
  1 & 0 & 0 & 0 \\
  0 & 1 & 0 & 0 \\
  0 & 0 & 1 & 0 \\
  0 & 0 & 0 & 1
\end{pmatrix}.
$$
The associated curvature tensor $R$ is Nakano positive, but is of course not the curvature tensor of a K\"ahler metric. The reader can check that, for example, $R(v_1,v_2,v_2,v_1) \not= R(v_1,v_2,v_1,v_2)$.
\end{exam}


\section{Flat metrics}
\label{sec:org504b250}

Let $(E,h) \to X$ be a holomorphic Hermitian vector bundle. The curvature form $\Theta$ of $h$ is a smooth section of $\bigwedge^{1,1}T_X \otimes \End E$. The simplest such section is of course the zero section. We say that the metric $h$ is \emph{flat} if its curvature form is zero.


TODO: Show that the universal covering of a flat manifold is a complex Lie group.

TODO: Show that a compact complex Lie group is commutative, i.e., a torus. Any compact flat manifold is thus K\"ahler.


\subsection{Euclidean metric}

The prime example of such a metric is the standard Euclidean metric on any open set $U \subset \kk C^n$. The metric $h$ defined by the usual inner product,
$$
h(\xi, \eta) = \sum_{j=1}^n \xi_j \overline{\eta_j}.
$$
The associated K\"ahler form is
$$
\omega = \sum_{j=1}^n \frac{i}{2} dz_j \wedge d\bar z_j.
$$
This is clearly $d$-closed, so the Euclidean metric is K\"ahler.\footnote{It has a global potential $\|z\|^2$. This is seldom useful.} We have
$$
h(D\xi, \eta) + h(\xi, D\eta)
= d h(\xi, \eta)
= \sum_{j=1}^n d\xi_j \otimes \overline{\eta_j} + \xi_j \otimes \overline{d\eta_j}
$$
so the Chern connection of $h$ is equal to the exterior derivative.

% TODO: Is any of the following relevant to examples of curvature tensors?
% Do we want to only do examples or also have a review of some things that are known about curvature?

\subsection{Compact manifolds}

There are also compact manifolds with flat metrics. Let $\Lambda \subset \kk C^n$ be a lattice, that is, an abelian group of rank $2n$. Then $X := \kk C^n / \Lambda$ is a compact complex manifold, called a \emph{complex torus}. As $\Lambda$ acts by translations on $\kk C^n$, it preserves the Euclidean metric, which descends to $X$.

One can ask whether the converse holds: If $X$ is a compact complex manifold with a flat metric, then is $X$ a complex torus?

The answer is subtle. First, the proposed answer cannot be true, as some complex tori admit subgroups that act freely, and whose quotients are thus again compact complex manifolds with flat metrics that are not tori. In the K\"ahler case, that is the extent of the problems that can arise. We can see this quickly if we're prepared to nuke a mosquito:

Suppose $X$ is a compact K\"ahler manifold that admits a flat metric. All of its Chern classes are then zero, in particular the first one. By \cite{beauville1983}, the universal covering of $X$ splits into a product of $\kk C^n$ and compact manifolds with zero first Chern class. The vanishing of all Chern classes of $X$ implies the product only involves $\kk C^n$.

For Hermitian manifolds, the answer is slightly more complicated: A flat Hermitian manifold is covered by a simply connected complex Lie group~\cite{boothby1958}.


\subsection{Gauss--Manin}

% TODO: I'm not sure this should be here at all.

Let $\pi : X \to S$ be a family of compact K\"ahler manifolds over a connected smooth base $S$. All of the manifolds $X_s := \pi^{-1}(s)$ in the family are diffeomorphic, so they all have isomorphic cohomology groups $H^k(X_s, \kk Z)$. These can be assembled into a holomorphic vector bundle $E \to S$ (by taking direct images of constant sheaves) whose fibers are exactly
$$
E_s = H^k(X_s, \kk C).
$$
The construction also yields a connection $\nabla$ on this vector bundle, called the \emph{Gauss--Manin connection}. It is flat, and if $\alpha$ is a local section of $E$ and $Y \subset X_s$ a $2k$-dimensional submanifold, then
$$
d \int_{Y} \alpha = \int_{Y} \nabla \alpha.
$$
The Gauss--Manin connection is generally not the Chern connection of a metric on $E$.

Something mildly interesting happens in the middle cohomology of $4n$-dimensional manifolds. Then there is an intersection form on $2n$-forms, which yields a sesquilinear form
$$
b(\alpha, \beta) = \int_{X_s} \alpha \wedge \overline{\beta}.
$$
This form is non-degenerate, but has mixed signature. One can still define connections for such forms, and the usual proof of the existance and uniqueness of the Chern connection carries through for them. One can check that the Gauss--Manin connection is the Chern connection of the intersection form in this sense.\footnote{This is where we use that the forms are of even degree to show compatibility with the form $b$; otherwise there's an extra $(-1)^n$ involved.}
This does not appear to have any applications.



\section{Conformal metrics}
\label{sec:org65fcbad}

Let \((E,h) \to X\) be a holomorphic Hermitian vector bundle, and let \(D\) be its Chern connection.

If \(f\) is a smooth real-valued function on \(X\), then \(h' := e^f h\) is again a Hermitian metric, \emph{conformal} to \(h\). The formulas for the Chern connection and curvature of conformal metrics are less complicated than the equivalent formulas for Riemannian metrics. We have
$$
h'(D_{h'} s, t) + h'(s, D_{h'} t)
= d h'(s, t)
= df \otimes h'(s, t) + h'(D_h s, t) + h'(s, D_h t).
$$
Writing \(df = \partial f + \bar\partial f\) we see that the Chern connection of \(h'\) is
$$
D_{h'} s = \partial f \otimes s + D_h s.
$$
Using the expressions for the covariant exterior derivative, we also have
\begin{align*}
D_{h'}^2 s
&= D_{h'}(\partial f \otimes s + D_h s)
\\
&= d(\partial f) \otimes s - \partial f \wedge D_{h'} s + D_{h'}(D_h s)
\\
&= -\partial\bar\partial f \otimes s - \partial f \wedge (\partial f \otimes s + D_h s) + \partial f \wedge D_h s + D_h^2 s
\\
&= -\partial\bar\partial f \otimes s + D_h^2 s.
\end{align*}

The Chern curvature of \(h'\) is thus
$$
D^2_{h'} = -\partial\bar\partial f \otimes \id_E + D^2_h.
$$

It's fun to work out what this gives for metrics conformal to a flat metric. We'll do that later when we study a particular metric on the \hyperref[sec:org8f5818e]{Hopf manifold}.

\subsection{Conformal to K\"ahler is not K\"ahler}
\label{sec:org7b1cfdf}

It's worth mentioning that if \(h\) is a K\"ahler metric on a manifold of dimension \(n > 1\) and \(f\) is non-constant, then \(h'\) is not a K\"ahler metric. The reason is that if \(\omega\) is the symplectic form associated to \(h\), then the symplectic form of \(h'\) is \(e^f \omega\) and
$$
d(e^f \omega) = \omega \wedge (e^f df)
$$
and the linear morphism from one- to three-forms defined by wedging with the symplectic form \(\omega\) is injective.\footnote{The hard Lefschetz theorem generalizes this to cohomology.}
The proof reduces to linear algebra by a calculation in local coordinates, either Darbeaux ones or in holomorphic ones that are orthonormal at a point.


\section{Sub- and quotient bundles}
\label{sec:sub-quotient}

Let $X$ be a complex manifold and let
\[
0 \longrightarrow
S \stackrel{j}{\longrightarrow}
E \stackrel{q}{\longrightarrow}
Q \longrightarrow
0
\]
be a short exact sequence of holomorphic vector bundles. Suppose that $h_E$ is a Hermitian metric on $E$. It induces a smooth splitting $E \cong S \oplus Q$ of $E$, and incudes Hermitian metrics $h_S$ and $h_Q$ on the bundles $S$ and $Q$. We want to express the curvatures of $S$ and $Q$ in terms of the curvature of $E$. This was first done by Griffiths in~\cite{griffiths1965hermitian}, which the reader really should look up, it's great. We mostly follow Demailly~\cite[Chapter~5.14]{demailly-complex}, which I have spent years reading.


\begin{defi}
The \emph{second fundamental form} of $S$ in $E$ is
\[
b(s) := q(D_E(js) - jD_S(s)).
\]
\end{defi}

\begin{prop}
\label{prop:second-fundamental-form}
The second fundamental form is an element of $\cc A^{1,0}(\Hom(S,Q))$.
We have
\[
b(s)
= q(D_{\Hom(S,E)}(j)(s))
= q(D_E(js))
\]
for sections $s$ of $S$.
\end{prop}

\begin{proof}
As $qj = 0$ it is clear that $b(s) = q(D_E(js))$. We also have
\[
D_E(js) = D_{\Hom(S,E)}(j)(s) + j(D_Ss),
\]
so $b(s) = q(D_{\Hom(S,E)}(j)(s))$.

As $j$ is holomorphic, we have $D_{\Hom(S,E)}j = D'_{\Hom(S,E)}j$ so $b$ has no $(0,1)$-part. It is also clearly $\cc C^\infty$-linear in its tensor field variable. If $f$ is a smooth function, we have
\[
D_E(j(fs))
= df \otimes js + f D_E(js)
\]
so $b(fs) = q(D_E(j(fs))) = q(f D_E(js)) = fb(s)$.
\end{proof}


We have defined the second fundamental form almost like it is done in Riemannian geometry. Recall that $h_S(s,t) = h_E(js,jt)$ by definition. For holomorphic sections of $S$ we then have
\[
0
= h_E(D_E(js), jt) - h_S(D_Ss,t)
= h_E(D_E(js) - jD_Ss, jt),
\]
and $D_E(js) - jD_Ss$ is a tensor. By the above calculation it is a $(1,0)$-form that takes values in the orthogonal bundle $S^\perp$. Here a Riemannian geometer would emit a content sigh and move on with their life. They can do this because they work in the smooth category, and to them working with the smooth bundle $S^\perp$ is just fine. We prefer having holomorphic bundles, even if the sections we look at end up only being smooth, and thus compose with the quotient map and get values in $Q$.

This also leads us to having two more expressions for the second fundamental form. The different expressions have different interpretations and uses. The one we used for the definition is a metric one, useful when doing differential geometry. The one involving $D_{\Hom(S,E)}$ comes in handy when viewing the subbundle as deforming inside $E$, which is a Grassmannian viewpoint. Finally the one involving only $D_E$ is quite useful for calculations involving submanifolds of a given space.



\paragraph{}

Our goal is to calculate the curvature forms of the sub- and quotient bundles. The first stop on the way is a wildly useful list of formulas from Demailly~{{\cite[Theorem~14.3]{demailly-complex}}}.


\begin{prop}
\label{prop:seq-formulas}
\begin{alignat*}{2}
D'_{\Hom(S,E)}j &= q^\dagger \circ b,
&
\qquad
\bar\partial j &= 0,
\\
D'_{\Hom(E, Q)} q &= - b \circ j^\dagger,
&
\bar\partial q &= 0,
\\
D'_{\Hom(E,S)} j^\dagger &= 0,
&
\bar\partial j^\dagger &= b^\dagger \circ q,
\\
D'_{\Hom(Q,E)} q^\dagger &= 0,
&
\bar\partial q^\dagger &= - j \circ b^\dagger,
\\
D'_{\Hom(S,Q)} b &= 0,
&
\bar\partial b^\dagger &= 0.
\end{alignat*}
\end{prop}

\begin{proof}
Let $s$ be a section of $S$. Then
\[
j(D_S s) + q^\dagger b(s)
= D_E(js)
= D_{\Hom(S,E)}j (s) + j(D_S s),
\]
where the first equality is by definition of $D_S$ and $b$, so
\[
D_{\Hom(S,E)}j = q^\dagger \circ b.
\]
The morphism $j$ is holomorphic, so $\bar\partial j = 0$. This proves the first line. The third line follows from the first by taking adjoints.


For the second line, the morphism $q$ is also holomorphic, so $\bar\partial q = 0$. Taking adjoints, we get that $D'_{\Hom(Q,E)} q^\dagger = 0$.
We have $\id_E = j \circ j^\dagger + q^\dagger \circ q$, so
\begin{align*}
0
&= D'_{\End E} \id_E
\\
&= D'_{\Hom(S,E)} j \circ j^\dagger + j \circ D'_{\Hom(E,S)}j^\dagger
+ D'_{\Hom(Q,E)}q^\dagger \circ q + q^\dagger D'_{\Hom(E,Q)} q
\\
&= q^\dagger \circ b \circ j^\dagger + q^\dagger \circ D'_{\Hom(E,Q)}q.
\end{align*}
Applying $q$, we conclude that $D'_{\Hom(E,Q)}q = - b \circ j^\dagger$. This proves the second line. Taking adjoints we get the last piece of the fourth line.

Finally we note that
\begin{align*}
D'_{\Hom(S,Q)} b
&= D'_{\Hom(S,Q)} (q D'_{\Hom(S,E)}j)
\\
&= - b \circ j^\dagger \circ q^\dagger \circ b
+ q^\dagger (D'_{\Hom(S,Q)})^2 j
= 0
\end{align*}
as we're dealing with curvature tensors of Hermitian forms which are of type $(1,1)$. The statement about $\bar\partial b^\dagger$ follows by taking adjoints.
\end{proof}



\begin{coro}
The splitting $E \to S \oplus Q$ is holomorphic if and only if $b = 0$.
\end{coro}

\begin{coro}
The $(0,1)$-form $b^\dagger$ defines a cohomology class in $H^{0,1}(X, \Hom(Q,S))$.
\end{coro}

One can actually prove that the class of $b^\dagger$ represents the extension class of $E$ among all extensions of $S$ by $Q$; see Demailly~\cite[Proposition~14.9]{demailly-complex}. That is, if $0 \to S \to F \to Q \to 0$ is another short exact sequence, then $[b_{S,F}^\dagger] = [b_{S,E}^\dagger]$ if and only if there is a holomorphic isomorphism $E \cong F$. In particular, if $[b_S^\dagger] = 0$ if and only if $E = S \oplus Q$. This is a neat result, but I don't recall applications of it. If the reader knows of any they're welcome to educate me.




\begin{prop}
Under the smooth splitting $E \to S \oplus Q$ defined by $s \mapsto j^\dagger s \oplus qs$ the Chern connection and curvature form of $E$ is
\begin{align*}
D_E s &=
\begin{pmatrix}
j^\dagger & q
\end{pmatrix}^\dagger
\begin{pmatrix}
D_S & - b^\dagger
\\
b & D_Q
\end{pmatrix}
\begin{pmatrix}
j^\dagger \\ q
\end{pmatrix}
(s),
\\
D_E^2 s &=
\begin{pmatrix}
j^\dagger & q
\end{pmatrix}^\dagger
\begin{pmatrix}
D^2_S - b^\dagger \wedge b & -D'_{\Hom(Q,S)} b^\dagger
\\
\bar\partial b & D^2_Q - b \wedge b^\dagger
\end{pmatrix}
\begin{pmatrix}
  j^\dagger \\ q
\end{pmatrix}(s).
\end{align*}
\end{prop}


\begin{proof}
Let $s$ be a section of $E$. We write
\[
s = j (j^\dagger s) + q^\dagger( qs).
\]
Then
\[
D_E s
= q^\dagger \circ b (j^\dagger s)
+ j \circ D_S( j^\dagger s)
- j \circ b^\dagger (qs)
+ q^\dagger \circ D_Q (qs).
\]
This proves the first statement.
For each of the terms here, we have
\begin{align*}
D_E(q^\dagger \circ b (j^\dagger s))
&= -j \circ b^\dagger \circ b (j^\dagger s)
+ q^\dagger \circ \bar\partial b (j^\dagger s)
- q^\dagger \circ b \circ D_S(j^\dagger s),
\\
D_E(j \circ D_S( j^\dagger s))
&= q^\dagger \circ b \circ D_S(j^\dagger s)
+ j \circ D_S^2 (j^\dagger s),
\\
D_E(-j \circ b^\dagger (qs))
&= - q^\dagger \circ b \circ b^\dagger (qs)
- j \circ D'_{\Hom(Q,S)}b^\dagger (qs)
+ j \circ b^\dagger \circ D_Q(qs),
\\
D_E(q^\dagger \circ D_Q (qs))
&= -j \circ b^\dagger \circ D_Q(qs)
+ q^\dagger \circ D_Q^2 (qs).
\end{align*}
Grouping these together by pre- and postfix morphisms, we get
\begin{align*}
D_E^2 s
&= j \bigl( D_S^2 - b^\dagger \circ b \bigr) j^\dagger s
- j \bigl( D'_{\Hom(Q,S)} b^\dagger \bigr) qs
\\
&\qquad
+ q^\dagger \bigl( \bar\partial b \bigr) j^\dagger s
+ q^\dagger \bigl( D^2_Q - b \circ b^\dagger \bigr) qs
\\
&=
\begin{pmatrix}
j^\dagger & q
\end{pmatrix}^\dagger
\begin{pmatrix}
D^2_S - b^\dagger \wedge b & -D'_{\Hom(Q,S)} b^\dagger
\\
\bar\partial b & D^2_Q - b \wedge b^\dagger
\end{pmatrix}
\begin{pmatrix}
  j^\dagger \\ q
\end{pmatrix}(s).
\end{align*}
\end{proof}



The next corollary is immediate, but will be useful later when discussing the curvature of the Grassmannian.

\begin{coro}
If $E$ is flat, then $\bar\partial b = 0$.
\end{coro}


The real money is however in the following corollary, which is the holomorphic analogue of the Codazzi equation in Riemannian geometry.


\begin{coro}[Codazzi--Griffiths equations]
\label{prop:codazzi-equation}
The curvature tensors of $S$ and $Q$ satisfy
\begin{align*}
R_S(\alpha, \ov\beta, s, \ov t)
&= R_E(\alpha, \ov\beta, js, \ov{jt})
- \tfrac i2 h_Q(b(\alpha, s), \ov{b(\beta, t)}),
\\
R_Q(\alpha, \ov\beta, s, \ov t)
&= R_E(\alpha, \ov\beta, q^\dagger s, \ov{q^\dagger t})
+ \tfrac i2 h_S(b^\dagger(\ov\beta, s), \ov{b^\dagger(\ov\alpha, t)}).
\end{align*}
In particular, $R_S \leq R_E \leq R_Q$, with equality on either side if and only if the splitting $E = S \oplus Q$ is holomorphic.
\end{coro}

\begin{proof}
The task here is to convince the reader that our signs are correct and that the arguments go in the right place. Recall that we have a $(1,0)$-form $b$ with values in $\Hom(S,Q)$. Then $b \circ b^\dagger$ and $b^\dagger \circ b$ are both $(1,1)$-forms, with values in $\End Q$ and $\End S$, respectively. Our first claims are that the $(1,1)$-forms
\[
\tfrac i2 h_Q(b \circ b^\dagger(s), \ov s)
\qandq
\tfrac i2 h_S(b^\dagger \circ b(s), \ov s)
\]
are real and seminegative and semipositive for all sections $s$ (of $S$ and $Q$ as appropriate), respectively. For the realness, we have
\begin{align*}
\overline{\rho(\alpha, \ov\beta)}
:= \overline{\tfrac i2 h_Q(b \circ b^\dagger(\alpha, \ov\beta, s), \ov s)}
&= -\tfrac i2 h_Q(s, \ov{b \circ b^\dagger(\alpha,\ov\beta, s)})
\\
&= -\tfrac i2 h_Q(b \circ b^\dagger(\ov\alpha, \beta, s), \ov{s})
\\
&= \tfrac i2 h_Q(b \circ b^\dagger(\beta, \ov\alpha, s), \ov{s})
= \rho(\beta, \ov\alpha),
\end{align*}
so the first form is real. The second is also real by a very similar calculation.

TODO: We'd like to show that
\[
\tfrac i2 b \wedge b^\dagger (\alpha, \ov\beta, s)
= b(\alpha, b^\dagger(\ov\beta, s)),
\]
where the left-hand side is the value of the $(1,1)$-form $\frac i2 b \wedge b^\dagger$. Maybe this needs to be interpreted as ``the $(1,1)$-form is the imaginary part of a Hermitian form with values in $\End Q$, which is the right-hand side''.

Now note that
\[
\tfrac i2 h_S(b \circ b^\dagger(\alpha, \ov\beta, s), \ov t)
= \tfrac i2 h_S(b(\alpha, b^\dagger(\ov \beta, s)), \ov t)
= \tfrac i2 h_S(b^\dagger(\ov \beta, s), \ov{b^\dagger(\ov \alpha, t)}),
\]
so the second form is semipositive. The first form looks very similar, but a careful inspection will reveal that if we were to write it out in local coordinates its terms would be of the form $\frac i2 d\bar z_k \wedge dz_j$. This explains why we have
\begin{align*}
\tfrac i2 h_Q(b^\dagger \circ b(\alpha, \ov\beta, s), \ov t)
&= -\tfrac i2 h_Q(b^\dagger \circ b(\ov\beta, \alpha, s), \ov t)
\\
&= -\tfrac i2 h_Q(b^\dagger(\ov\beta, b(\alpha, s)), \ov t)
= -\tfrac i2 h_Q(b(\alpha, s), \ov{b(\ov\beta, t)}),
\end{align*}
so the form is seminegative.

The announced formulas both follow from these calculations. The same calculations show that we have equality $R_S = R_E$ or $R_E = R_Q$ if and only if $|b|_{\Hom(S,Q)} = 0$ or $|b^\dagger|_{\Hom(Q,S)} = 0$, which happens if and only if the splitting $E = S \oplus Q$ is holomorphic.
\end{proof}



The Codazzi--Griffiths equations show that the curvature of a holomorphic subbundle is less or equal to the curvature of the ambient bundle. In particular, the curvature of a submanifold is dominated by the curvature of the ambient manifold. The same is of course not true in Riemannian geometry; see for example the sphere in Euclidean space.

There are many open questions regarding the relationship between hyperbolicity, ample canonical bundles, and negative holomorphic sectional curvature for compact complex manifolds. For example, a manifold with negative holomorphic sectional curvature is hyperbolic by a version of the Schwarz lemma. By the Codazzi equation, any submanifold of such a manifold is thus also hyperbolic. The converse is conjectured to hold in some form; that if all the subvarieties of a manifold are hyperbolic, then so is the manifold itself.




\paragraph{}

We can amuse ourselves by calculating curvature tensors of bundles derived from the ones we have. To pick one at random that will absolutely not show up when we calculate the curvature of the Grassmannian:

\begin{prop}
\label{prop:hom-bundle-curvature}
The curvature tensor of $\Hom(S,Q)$ is
$$
\displaylines{
R_{\Hom(S,Q)}(\alpha,\ov\beta,f,\ov g)
= R_{\End E}(\alpha, \ov\beta, q^\dagger f j^\dagger, \ov{q^\dagger g j^\dagger})
\hfill\cr\hfill{}
+ h_{\End S}(b(\ov\beta)^\dagger \circ b(\alpha), \ov{f^\dagger \circ g})
+ h_{\End S}(b(\ov\beta)^\dagger \circ f, \ov{b^\dagger(\ov\alpha) \circ g}),
}
$$
where $b$ is the second fundamental form of $S$,
$\alpha,\beta \in T_X$ and $f,g \in \cc \Hom(S,Q)$.
\end{prop}


\begin{proof}
The morphisms $j : S \to E$ and $q : E \to Q$ induce a surjective holomorphic bundle morphism $\End E \to \Hom(S,Q)$ via $f \mapsto q \circ f \circ j$. We're basically going to calculate the induced curvature tensor by hand instead of appealing to the Codazzi equations.

Let then $f,g \in \Hom(S,Q)$. These induce smooth sections of $\End E$ via the adjoints as $q^\dagger f j^\dagger$, and similar for $g$, and we have the equality
\[
\langle f, \ov g \rangle_{\Hom(S,Q)}
= \langle q^\dagger f j^\dagger, \ov{q^\dagger g j^\dagger} \rangle_{\End E}
\]
by definition of the inner product on the quotient bundle.

A formal calculation shows that
\[
\Theta_{\End E} (q^\dagger f j^\dagger)
= \Theta_{\Hom(Q,E)} q^\dagger f j^\dagger
+ q^\dagger \Theta_{\Hom(S,Q)} f j^\dagger
+ q^\dagger f \Theta_{\Hom(E,S)} j^\dagger.
\]
Referring to \thref{prop:seq-formulas} we see that
\begin{align*}
\Theta_{\Hom(Q,E)} q^\dagger
&= D'_{\Hom(Q,E)} \bar\partial q^\dagger
= -D'_{\Hom(Q,E)} (j b^\dagger)
= -q^\dagger b \wedge b^\dagger - j D'_{\Hom(Q,S)}b^\dagger,
\\
\Theta_{\Hom(E,S)} j^\dagger
&= D'_{\Hom(E,S)} \bar\partial j^\dagger
= D'_{\Hom(E,S)}(b^\dagger q)
= D'_{\Hom(Q,S)}b^\dagger q - b^\dagger \wedge b j^\dagger.
\end{align*}
Then we have
\begin{align*}
\langle \Theta_{\Hom(Q,E)} q^\dagger f j^\dagger, \ov{q^\dagger g j^\dagger} \rangle_{\End E}
&= -\langle q^\dagger b \wedge b^\dagger f j^\dagger, \ov{q^\dagger g j^\dagger} \rangle_{\End E}
\\
&\qquad - \langle j D'_{\Hom(Q,S)}b^\dagger f j^\dagger, \ov{q^\dagger g j^\dagger} \rangle_{\End E}
\\
&= -\langle b^\dagger f, \ov{b^\dagger g} \rangle_{\Hom(S,Q)},
\end{align*}
where the second term is zero because we can flip $j$ from the left-hand side to its adjoint on the right and get $j^\dagger q^\dagger = 0$. We also get
\begin{align*}
\langle q^\dagger f \Theta_{\Hom(E,S)} j^\dagger, \ov{q^\dagger g j^\dagger} \rangle_{\End E}
&= \langle q^\dagger f D'_{\Hom(Q,S)}b^\dagger q , \ov{q^\dagger g j^\dagger} \rangle_{\End E}
\\
&\qquad
- \langle q^\dagger f b^\dagger \wedge b j^\dagger, \ov{q^\dagger g j^\dagger} \rangle_{\End E}
\\
&= - \langle b^\dagger \wedge b, \ov{f^\dagger g} \rangle_{\Hom(S,Q)},
\end{align*}
where the first term is zero because it is of the form
\[
\langle f q, \ov{g j^\dagger} \rangle_{\End E}
= \tr(j g^\dagger f q)
= \tr(g^\dagger f q j) = 0
\]
and the last equality holds because $q j = 0$.
Putting this all together, we get the announced result.
\end{proof}



\subsection{Failing to construct K\"ahler metrics}


Let $X$ be a complex manifold. We're going to consider the situation where we have a short exact sequence
\[
\begin{tikzcd}
0 \arrow[r] & T_X \arrow[r,"j"] & E \arrow[r,"q"] & Q \arrow[r] & 0
\end{tikzcd}
\]
and a Hermitian metric on $E$. We want to know when the metric induced on $X$ is K\"ahler? This can be seen as a generalization of what happens when we consider a submanifold of a K\"ahler manifold. We can also consider the situation where $T_X$ is the quotient $Q$; it is mostly dual to this one.

The induced metric can be K\"ahler: a submanifold of a K\"ahler manifold is K\"ahler and the induced metric arises in this way. This can also not happen: take any non-K\"ahler manifold and consider it as a submanifold in itself times a curve. No matter what metric we put on the product, the induced metric won't be K\"ahler.

The following is not an answer to the question:



\begin{prop}
Let $X$ be a complex manifold. Suppose there is a short exact sequence
\[
\begin{tikzcd}
0 \arrow[r] & T_X \arrow[r,"j"] & E \arrow[r,"q"] & Q \arrow[r] & 0
\end{tikzcd}
\]
of holomorphic vector bundles over $X$, and that $E$ is equipped with a Hermitian metric. The induced metric on $X$ is K\"ahler.
\end{prop}

\begin{proof}
Recall that the torsion tensor is the covariant derivative of the identity morphism, that is, $\tau = D_{\End T_X} \id_{T_X}$.
We have $j^\dagger \circ j = \id_{T_X}$. Then
\[
\tau
= D_{\End T_X} \id_{T_X}
= D_{\Hom(E,T_X)}j^\dagger \circ j + j^\dagger \circ D_{\Hom(T_X,E)}j.
\]
By \thref{prop:seq-formulas} we have
\[
D_{\Hom(E,T_X)}j^\dagger
= b^\dagger \circ q
\qandq
D_{\Hom(T_X,E)}j
= q^\dagger \circ b,
\]
where $b : \cc A^{1,0}(\Hom(T_X,Q))$ is the second fundamental form and $q^\dagger$ is the adjoint of $q$.
As $q \circ j = 0$ and $j^\dagger \circ q^\dagger = 0$, we get
\[
\tau
= b^\dagger \circ q \circ j
+ j^\dagger \circ q^\dagger \circ b
= 0.
\]
\end{proof}

This very nice result is unfortunately not true, as it implies every complex manifold is K\"ahler. The problem with this ``proof'' is in the statement that the torsion tensor is the covariant derivative of the identity morphism. If $D$ is a connection on a vector bundle $E$, we have
\[
D_E s
= D_E(\id_{E} s)
= (D_{\End E}\id_E)(s) + \id_E(D_Es)
= (D_{\End E}\id_E)(s) + D_Es
\]
for any section $s$ of $E$, so $D_{\End E}\id_E = 0$. The correct statement is that the torsion tensor is the covariant \emph{exterior} derivative of the identity morphism considered as a one-form with values in $E = T_X$, and the two notions are not the same. If $f \in \End E$ then $D_{\End E}f \in \cc A^1(\End E)$. For $T_X$, this is $\cc A^1(\End T_X) \cong \cc C^\infty(X,T_X^* \otimes T_X^* \otimes T_X)$. Meanwhile the torsion tensor $\tau = d^D \id_{T_X}$ is antisymmetric, and thus an element of $\cc C^\infty(X,\bigwedge^2 T_X^* \otimes T_X)$. What the above argument establishes at some length is that $D_{\End T_X} \id_{T_X} = 0$, which is true for any connection as we have just seen.


\paragraph{}

Consider now a projective manifold $X$, and pick an ample line bundle $L$ on it. Then there exists some integer $N$ such that $\Omega_X^1 \otimes L^{\otimes k}$ is generated by global sections for $k \geq N$. That is, there is a surjective morphism
\[
\begin{tikzcd}
H^0(X, \Omega_X^1 \otimes L^{\otimes k}) \otimes L^{\otimes k} \arrow[r] & \Omega_X \otimes L^{\otimes k} \arrow[r] & 0
\end{tikzcd}
\]
of vector bundles over $X$, where the global sections form a trivial bundle. Dualizing and tensoring by the invertible sheaf $L^*$ we get a short exact sequence
\[
\begin{tikzcd}
0 \arrow[r] & T_X \arrow[r] & H^0(X, \Omega_X^1 \otimes L^{\otimes k}) \otimes (L^*)^{\otimes k} \arrow[r] & Q \arrow[r] & 0.
\end{tikzcd}
\]
If we put a flat metric on the trivial bundle and a positive metric on $L$, we get a Hermitian metric on $X$. Playing the same game with $T_X$ instead of $\Omega_X^1$ realizes the tangent bundle as the quotient of a vector bundle on $X$.

This construction is due to Demailly, who showed it to me when I asked him if he knew of a ``natural'' example of a non-K\"ahler metric. If he explained to me why the induced metric isn't K\"ahler in general, I never understood it and have forgotten why. If we assume he knew what he was talking about (generally a good idea), then an exercise for the reader is to prove:

\begin{prop}
The Hermitian metric constructed on $X$ in this way is ``generally'' not K\"ahler.
\end{prop}

My attempts to show this have all ended in tautologies that go ``the metric is K\"ahler when it's K\"ahler''. While true, it is unhelpful.


\subsection{Sums of metrics}

The curvature tensor of a sum of metrics is not the sum of the curvature tensors of each metric. This is very sad. We can quantify exactly how sad using the Codazzi--Griffiths equations. This section amounts to the solution to an exercise given in Zheng's book~\cite{zheng2000complex}.

Let $E \to X$ be a holomorphic vector bundle, and let $h_1$ and $h_2$ be Hermitian metrics on it. We write $D_{h_1}$ and $R_{h_1}$ for the Chern connection and curvature tensor of $h_1$, and similar for $h_2$. Consider the short exact sequence
\[
\begin{tikzcd}
0 \arrow[r] &
E \arrow[r,"j"] &
E \oplus E \arrow[r,"q"] &
E \arrow[r] &
 0
\end{tikzcd}
\]
where $j(v) = v \oplus v$ and $q(v \oplus w) = v - w$. We equip the subbundle with the metric $h_1 + h_2$, the direct sum with the metric $h_1 \oplus h_2$, and the quotient with the induced metric.

\begin{prop}
The second fundamental form of $E$ is
\[
b(v) = (D_{h_1} - D_{h_2})v.
\]
\end{prop}

\begin{proof}
We calculate
\begin{align*}
b(v)
= q(D_{h_1 \oplus h_2} j(v))
&= q(D_{h_1 \oplus h_2} (v \oplus v))
\\
&= q(D_{h_1} v \oplus v + v \oplus D_{h_2} v)
= (D_{h_1} - D_{h_2})v.
\end{align*}
\end{proof}


\begin{coro}
The curvature tensor of $h_1 + h_2$ is
$$
\displaylines{
R_{h_1 + h_2}(\alpha,\ov\beta,s,\ov t)
= R_{h_1}(\alpha,\ov\beta,s,\ov t)
+ R_{h_2}(\alpha,\ov\beta,s,\ov t)
\hfill\cr\hfill{}
- \langle (D_{h_1,\alpha} - D_{h_2,\alpha}) s,
\ov{(D_{h_1,\beta} - D_{h_2,\beta}) t} \rangle_Q,
}
$$
where the inner product labeled with $Q$ is the one on the ``quotient'' space $E$.
\end{coro}


One would think it would be nice to be able to say what the inner product on the quotient is. We can write it down and find out that it isn't so much.

\begin{prop}
The inner product on the quotient space $E$ is
\begin{align*}
h(x, \ov y)
&=
h_1\bigl(((h_1+h_2)^{-1}h_2(x),
\ov{(h_1+h_2)^{-1}h_2(y)}\bigr)
\\
&\qquad
+ h_2\bigl(((h_1+h_2)^{-1}h_1(x),
\ov{(h_1+h_2)^{-1}h_1(y)}\bigr).
\end{align*}
\end{prop}

\begin{proof}
First we note that the adjoint of $j : S \to E$ is defined by
\[
(h_1 + h_2)(j^\dagger x \oplus y, \ov v)
= (h_1\oplus h_2)( x \oplus y, \ov{v \oplus v})
= h_1(x, \ov v) + h_2(y, \ov v)
\]
so
\[
j^\dagger(x \oplus y)
= (h_1+h_2)^{-1}(h_1(x) + h_2(y)).
\]
Then $x \oplus y = jj^\dagger (x\oplus y) + q^\dagger q(x \oplus y)$, so we find that
\[
q^\dagger q(x \oplus y)
= \bigl(
x - (h_1+h_2)^{-1}(h_1(x) + h_2(y))
\bigr) \oplus
\bigl(
y - (h_1+h_2)^{-1}(h_1(x) + h_2(y))
\bigr).
\]
We write $x = (h_1+h_2)^{-1}(h_1 + h_2)(x)$ and get
\[
x - (h_1+h_2)^{-1}(h_1(x) + h_2(y))
= (h_1+h_2)^{-1}h_2(x - y).
\]
Similarly we get
\[
y - (h_1+h_2)^{-1}(h_1(x) + h_2(y))
= -(h_1+h_2)^{-1}h_1(x - y).
\]
Then
\[
q^\dagger q(x \oplus y)
= \bigl(
(h_1+h_2)^{-1}h_2(x - y)
\bigr)
\oplus
\bigl(
-(h_1+h_2)^{-1}h_1(x - y)
\bigr).
\]
In particular, the adjoint of $q$ is
\[
q^\dagger(v)
=\bigl(
(h_1+h_2)^{-1}h_2(v)
\bigr)
\oplus
\bigl(
-(h_1+h_2)^{-1}h_1(v)
\bigr).
\]
The inner product on the quotient is the pullback of $h_1 \oplus h_2$ by $q^\dagger$, which completes the proof.
\end{proof}

Working in coordinates clarifies a little what this metric is.  If we
simultaneously diagonalize the Hermitian metrics $h_1$ and $h_2$, say as
matrices with $(a_j)$ and $(b_j)$ on their diagonals, the matrix of the quotient
metric is also diagonal and the entries on its diagonal are
\[
\frac{a_jb_j}{a_j+b_j}.
\]


If $h$ is a Hermitian metric on $V$, we denote by $S(V,h)$ the unit sphere in $V$ and by $S(V,h,r)$ the sphere of radius $r$. Let $h_1$ and $h_2$ be Hermitian metrics on $V$. Then we have a map
\[
\pi : S(V,h_1+h_2) \to K,
\quad
v \mapsto |v|_{h_1},
\]
where $K = [\min |v|_{h_1},\max |v|_{h_1}] =: [m,M]$ is a compact interval. We assume $m < M$ (if $h_1 = h_2$ then $m = M$). We have $0 < \min |v|_{h_1}$ and $\max |v|_{h_1} < 1$ (because $v$ has to have nonzero $h_2$-norm to be on the sphere).
The fiber over a point $x$ is $S(V,h_1,x)$. There is a map $S(V,h_1) \to S(V,h_1,t)$ given by $v \mapsto xv$ that's bijective for $x \not= 0$. If $f : V \to \kk R$ is a continuous homogeneous function of degree $k$, then
\begin{align*}
\int_{S(V,h_1+h_2)} f(v) \, d\sigma(v)
&= \int_m^M \int_{S(V,h_1,x)} f(v) \, d\sigma(v) dx
\\
&= \int_m^M \int_{S(V,h_1)} f(xv) \, d\sigma(xv) dx
\\
&= \int_m^M \int_{S(V,h_1)} x^{2n-1+k} f(v) \, d\sigma(v) dx
\\
&= \frac{M^{2n+k}-m^{2n+k}}{2n+k} \int_{S(V,h_1)} f(v) \, d\sigma(v) dx
\end{align*}
Let $R_1$ be the curvature tensor of $h_1$. We get
\begin{align*}
\int_{S(V,h_1+h_2)} R_1(\alpha, \ov\beta, t, \ov t) \,d\sigma(t)
= \frac{M^{2n+2}-m^{2n+2}}{2(n+1)} \frac{n}{\Vol(S^{2n-1})} r_1(\alpha, \ov\beta),
\end{align*}
where $r_1$ is the trace of $R_1$. Then
\[
r(\alpha, \ov\beta)
= \frac{\max |v|_{h_1}^{2n+2} - \min |v|_{h_1}^{2n+2}}{2(n+1)} r_1(\alpha, \ov\beta)
+ \frac{\max |v|_{h_2}^{2n+2} - \min |v|_{h_2}^{2n+2}}{2(n+1)} r_2(\alpha, \ov\beta)
- h_{\End E}(b(\alpha), \ov{b(\beta)}),
\]
where $\End E = \Hom(E,E)$ is equipped with the metric induced by $h_1+h_2$ on the first $E$ and $h_Q$ on the second $E$.

Suppose that $v'$ is such that $|v'|_{h_1} = \max |v|_{h_1}$. Then $|v|_{h_1} \leq |v'|_{h_1}$ for all $v$. We then have
\[
|v|_{h_2}^2 = 1 - |v|_{h_1}^2 \geq 1 - |v'|_{h_1}^2 = |v'|_{h_2}^2
\]
for all $v$, so $|v'|_{h_2} = \min |v|_{h_2}$. A similar swap happens for vectors that acheive the minimum for $h_1$, so
\[
\max |v|_{h_1}^2 = 1 - \min |v|_{h_2}^2
\qandq
\min |v|_{h_1}^2 = 1 - \max |v|_{h_2}^2.
\]
Setting $m$ and $M$ to be the min and max for $h_1$, we get
\[
r(\alpha, \ov\beta)
= \frac{{M^2}^{2n}-{m^2}^{2n}}{2(n+1)} r_1(\alpha, \ov\beta)
+ \frac{(1-m^2)^{2n}-(1-M^2)^{2n}}{2(n+1)} r_2(\alpha, \ov\beta)
- h_{\End E}(b(\alpha), \ov{b(\beta)}).
\]
How different can $m$ and $M$ get? They are equal when $h_1$ and $h_2$ are multiples of each other.

We have
\[
(x-1)^{2k} - (y-1)^{2k}
= \sum_{j=0}^{2k}\binom{2k}{j} x^j (-1)^{2k-j} - y^j (-1)^{2k-j}
= \sum_{j=0}^{2k}\binom{2k}{j} (-1)^{j}(x^j - y^j).
\]
Recall that
\[
\sum_{l=0}^{j-1} x^l = \frac{x^j-1}{x-1}
\]
so
\[
x^j-y^j
= y^j((x/y)^j - 1)
= y^j ((x/y)-1) \sum_{l=0}^{j-1} (x/y)^l
= (x - y) \sum_{l=0}^{j-1} x^l y^{j-1-l}.
\]
Then
\[
\sum_{j=0}^{2k}\binom{2k}{j} (-1)^{j}(x^j - y^j)
= \sum_{j=0}^{2k}\binom{2k}{j} (-1)^{j} (x - y) \sum_{l=0}^{j-1} x^l y^{j-1-l}
\]



\subsection{Submanifolds}

A common situation where this is applicable is when we have a submanifold $Y \subset X$ of a complex manifold. Then we have a short exact sequence
\[
\begin{tikzcd}
0 \arrow[r] & T_Y \arrow[r] & T_{X|Y} \arrow[r] & N_{X/Y} \arrow[r] & 0
\end{tikzcd}
\]
of holomorphic vector bundles overmetric $h_X$, it induces Hermitian metrics $h_Y$ and $h_{N}$ on $Y$ and $N_{X/Y}$. The Codazzi equation then determines the curvature tensor of $Y$.


If the metric on $X$ is K\"ahler, then so is the metric on $Y$. The curvature tensor of a K\"ahler metric has extra symmetries, and the ones for $Y$ will be reflected somehow in the term that involves the second fundamental form. If we recall our Riemannian geometry, we can guess at how this will happen:

\begin{prop}
If $h_X$ is K\"ahler, then the second fundamental form of $Y$ in $X$ is symmetric.
\end{prop}

\begin{proof}
Let $j : Y \hookrightarrow X$ be the inclusion. As $Y$ is a submanifold, we have $[j\xi,j\eta] = j[\xi,\eta]$ for all tangent fields $\xi,\eta$ of $Y$. A straightforward calculation then gives
\[
b(\xi,\eta) - b(\eta,\xi)
= (D_{X,\xi}\eta - D_{Y,\xi}\eta)
- (D_{X,\eta}\xi - D_{Y,\eta}\xi)
= \tau_X(\xi,\eta) - \tau_Y(\xi,\eta),
\]
where $\tau$ are the torsion tensors of the metrics. When the metrics are K\"ahler, these tensors are zero.
\end{proof}

The curvature tensor of a submanifold of a K\"ahler manifold is thus of the form
\[
R_Y(\alpha,\beta,\gamma,\delta)
= R_X(\alpha,\beta,\gamma,\delta)
- h_{X/Y}(b(\alpha,\gamma), b(\beta,\delta)),
\]
where $b \in \Hom(S^2 T_Y, N_{X/Y})$. When $R_X = 0$ we saw some algebraic curvature tensors of this type back in \thref{prop:algebraic-curvature-second-fundamental}.



\begin{prop}
Let $Y \subset X$ be a hypersurface. Suppose $Y$ is locally defined by $Y = f^{-1}(0)$, where $f : X \to \kk C$ is a holomorphic function. The second fundamental form of $Y$ in $X$ is
\[
b(\alpha,\beta) = f_*(D_\alpha \beta),
\]
where $D$ is the Chern connection on $X$.
\end{prop}

\begin{proof}
The quotient morphism in the relevant short exact sequence is exactly $f_*$. The result then follows immediately from \thref{prop:second-fundamental-form}.
\end{proof}


Suppose that $L \to X$ is a holomorphic line bundle, that $\sigma \in H^0(X,L)$ is a holomorphic section such that $d\sigma_x \not= 0$ for all $x \in Y := \sigma^{-1}(0)$. Then $Y \subset X$ is a smooth submanifold of $X$.

\begin{prop}
There is a natural isomorphim $N_{X/Y} = L_{|Y}$.
\end{prop}

\begin{proof}
Let $D$ be a linear connection on $L$, and define
\[
\phi : T_{X|Y} \to L_{|Y},
\quad
\phi(\xi) = D_\xi \sigma.
\]
Then $\phi$ is surjective and its kernel is $T_Y$, so $L_{|Y} = T_{X|Y} / T_Y = N_{X/Y}$. If $D'$ is a different linear connection, then $D\sigma - D'\sigma = f \wedge \sigma$, where $f$ is a $1$-form with values in $\End(L)$. Then $D\sigma = D'\sigma$ on $Y$, so $\phi$ does not depend on the choice of connection.
\end{proof}

Suppose $h_X$ is a Hermitian metric on $X$. We have two short exact sequences
\[
\begin{tikzcd}
0 \ar[r] & T_Y \ar[r]\ar[d] & T_{X|Y} \ar[d] \ar[r,"q"] & N_{Y/X} \ar[r] \ar[d] & 0
\\
0 \ar[r] & T_Y \ar[r] & T_{X|Y} \ar[r,"\xi \mapsto \nabla_\xi\sigma"] & L_{|Y} \ar[r] & 0
\end{tikzcd}
\]
where $\nabla$ is a linear connection on $L$. We also have a map $(\xi,\eta) \mapsto D_{X,\xi}\eta - D_{Y,\xi}\eta$ whose values are orthogonal to $T_Y$.

The differential of the section is a map $\sigma_* : T_X \to \sigma^* T_L$. Once restricted to $Y$, its kernel is $T_Y$.



\section{Riemann surfaces}
\label{sec:org776713b}

In Riemannian geometry, the full curvature tensor doesn't appear until in dimension $4$. Before that, the low dimensions of the tangent space restrict what tensors can appear. The equivalent situation in complex geometry only occurs in complex dimension one, on Riemann surfaces. Complex dimension two is real dimension four and already contains all the complexities of the higher dimensions. Conversely, complex dimension one is only real dimension two and too simple to give a good idea of what happens in higher dimensions.

\subsection{Everything is K\"ahler}
Let $U \subset \kk C$ be an open set. We can imagine it is a small neighborhood around a point in a Riemann surface we care about. Let $h$ be a Hermitian metric on $U$. The K\"ahler form of the metric can be written as
$$
\omega = e^{f(z)} \frac{i}{2} dz \wedge d\bar z
$$
on $U$. Then $d\omega = 0$ for dimensional reasons, so any Hermitian metric on $U$ is K\"ahler.

\subsection{Curvature}
By the above, we also have $h = e^f h_{\mathrm{std}}$, so the Chern connection and curvature of the metric can be deduced from our discussion on conformal metrics. Notably, the curvature form is
$$
\Theta
= -\frac i2\partial\bar\partial f
= -\frac{1}{e^f}\frac{\partial^2f}{\partial z \partial \bar z} \; e^f \frac{i}{2} dz \wedge d\bar z
= -\Delta_\omega f \; \omega.
$$
There should be a tensor product with the identity map on $T_X$ in the first equality, but the identity map on a one-dimensional space is just multiplication by $1$ so we can skip it.

One can note that an endomorphism on a one-dimensional space can be identified with its trace. From that point of view, one can say that the curvature form in complex dimension one can be identified with the form one gets by taking the trace of the curvature endomorphisms. In some sense, the full curvature form collapses to the Ricci form. In some other sense, the above computations imply the Ricci form only contains the scalar curvature. There's not enough space for complex curvature tensors.


\subsection{Poincar\'e disk}

Let $D = \{z \in \kk C \mid |z| < 1\}$ be the unit disk in the complex plane. We can pull a Hermitian metric out of our hat by setting
$$
\omega = \frac 1{(1-|z|^2)^2} \frac i2 dz \wedge d\bar z.
$$
As
$$
\frac 1{(1-|z|^2)^2} = e^{-2\log(1-|z|^2)}
$$
the curvature form of $\omega$ is
\begin{align*}
2\frac i2 \partial\bar\partial \log(1-|z|^2)
&= 2\frac i2 \partial \frac{-z d\bar z}{1-|z|^2}
\\
&= 2\frac i2 \frac{- dz \wedge d\bar z}{1-|z|^2}
- 2\frac i2 \frac{-\bar z dz}{1-|z|^2} \wedge \frac{-zd\bar z}{1-|z|^2}
\\
&= 2\biggl(\frac{-1}{1-|z|^2} - \frac{|z|^2}{(1-|z|^2)^2} \biggr) \frac i2 dz \wedge d\bar z
\\
&= -2 \omega.
\end{align*}
The Poincar\'e metric is the first negatively curved K\"ahler metric we see. Most Riemann surfaces are negatively curved. Compact ones can be organized according to their genus, which is one-half of the dimension of their first homology group. Any compact Riemann surface of genus zero is isomorphic to the projective line and so is positively curved; a Riemann surface of genus one is a torus and thus flat; and any other Riemann surface is covered by the unit disk and negatively curved.



\section{Projective space}
\label{sec:orgcfabeed}

The complex projective space is the space of lines in a given complex vector space. That is, if $V$ is a complex vector space, then we define the projective space as the set
$$
\kk P(V) := \{ v \in V \mid v \not= 0 \} / \kk C^*,
$$
where $\kk C^*$ acts by multiplication. If $V$ is the zero space, then $\kk P(V) = \varnothing$ by this definition; if $V$ is a line, then $\kk P(V)$ is a point. People usually take care not to be in either of those cases.

There is a projection map $\pi : V \setminus \{0\} \to \kk P(V)$ and we equip $\kk P(V)$ with the quotient topology. If we fix a Hermitian inner product $h$ on $V$, then the quotient map factors through the unit sphere:
$$
V \setminus \{0\} \to S(V, h) \to \kk P(V).
$$
As the unit sphere is compact, it follows that the projective space $\kk P(V)$ is compact.


\subsection{Manifold structure}

We're going to construct local holomorphic charts on $\kk P(V)$. Let $\lambda \in V^* \setminus \{0\}$ and define
$$
U_\lambda := \{ [v] \in \kk P(V) \mid \lambda(v) \not= 0 \}.
$$
This is an open set in $\kk P(V)$ equipped with the quotient topology, as its preimage under the projection is the complement of the hyperplane $\{v \in V \mid \lambda(v) = 0\}$, which is open.

We define a map $f_\lambda: U_\lambda \to V$ by setting
$$
f([v]) = v/\lambda(v).
$$
This is well defined by the linearity of $\lambda$ and by definition of $U_\lambda$. This map takes values in the affine hyperplane
$$
H_\lambda := \{ v \in V \mid \lambda(v) = 1 \}.
$$
It is in fact a bijection onto this set: If $v \in H_\lambda$ then $[v]$ is an element of $U_\lambda$ that maps to $v$, so $f_\lambda$ is surjective. If $[v], [w] \in U_\lambda$ are such that $f_\lambda(v) = f_\lambda(w)$, then $v/\lambda(v) = w/\lambda(w)$ for any representatives $v, w$ of those classes. Then $v = (\lambda(v)/\lambda(w)) w$, so $[v] = [w]$ and $f_\lambda$ is injective.

The map $f_\lambda$ is continuous: Let $U \subset H_\lambda$ be open. Then $\pi^{-1}(f_\lambda^{-1}(U)) = \kk C^* \cdot U$, which is open. Its inverse is also continuous: The map $f_\lambda^{-1}$ is just the restriction of $\pi$ to $H_\lambda$. Thus $f_\lambda$ is a homeomorphism.

The collection $(f_\lambda : U_\lambda \to H_\lambda)_{\lambda \in V^* \setminus \{0\}}$ covers $\kk P(V)$ and provides local homeomorphisms to spaces biholomorphic to $\kk C^{\dim V - 1}$. If $\lambda$ and $\lambda'$ are two nonzero elements of the dual space, then the transition map between charts is
$$
f_{\lambda'} \circ f_{\lambda}^{-1} : \{v \in H_\lambda \mid \lambda'(v) \not= 0 \} \to H_{\lambda'},
\quad
v \mapsto v/\lambda'(v).
$$
This map is a composition of field operations and a linear map, so it is holomorphic. The collection above thus forms a holomorphic atlas.


\subsection{Tautological bundle}


If we consider the trivial vector bundle $V \to \kk P(V)$, then the definition of projective space gives a line bundle $\cc O(-1) \subset V$ whose fiber over a point $[v]$ is
the line $\kk C \cdot v$. This \emph{tautological line bundle} is holomorphic:

On a local chart $H_\lambda$, the line bundle is given by $\kk C \cdot v \subset V$. It is trivialized by the Euler section $\xi(v) = v$, which is certainly holomorphic on the local chart. Changing coordinates to ones defined by $\lambda'$ multiplies the section by $1/\lambda'(\xi)$, which is holomorphic.



\subsection{Curvature of tautological bundle}

Fix a Hermitian inner product $h$ on $V$. This defines a flat Hermitian metric on the trivial vector bundle $V \to \kk P(V)$. It follows that the induced metric on $\cc O(-1)$ is non-positive.

On $H_\lambda$, we have the Euler section $\xi$ of $\cc O(-1)$ given by $\xi(v) = v$. The curvature form of the line bundle is then
$$
-\frac i2 \partial \bar\partial \log h(\xi, \xi)
= -\frac i2\partial \frac{h(\xi, \partial \xi)}{h(\xi, \xi)}
= -\frac i2\frac{h(\partial \xi, \partial \xi)}{h(\xi, \xi)}
+ \frac i2\frac{h(\partial \xi, \xi)}{h(\xi, \xi)} \wedge \frac{h(\xi, \partial \xi)}{h(\xi, \xi)}.
$$
The Euler field satisfies $\partial_\alpha \xi = \alpha$ for holomorphic vector fields $\alpha$. The sesquilinear form defined by the curvature form is then
$$
\phi(\alpha, \beta)
= -\frac{h(\alpha, \beta)}{h(\xi, \xi)}
+ \frac{h(\alpha, \xi)}{h(\xi, \xi)} \cdot \frac{h(\xi, \beta)}{h(\xi, \xi)}.
$$
We claim that this is negative-definite. Cauchy--Schwarz gives
$$
\phi(\alpha, \alpha)
= -\frac{h(\alpha, \alpha)}{h(\xi, \xi)}
+ \frac{|h(\alpha, \xi)|^2}{h(\xi, \xi)^2}
\leq -\frac{h(\alpha, \alpha)}{h(\xi, \xi)}
+ \frac{h(\alpha, \alpha) h(\xi, \xi)}{h(\xi, \xi)^2}
= 0
$$
with equality if and only if $\alpha$ is a multiple of the Euler field $\xi$. But note that $H_\lambda$ is defined so that $\lambda(\xi) = 1$, while its tangent space identifies with the set of vectors $\alpha$ such that $\lambda(\alpha) = 0$. Therefore $\alpha$ is not a multiple of $\xi$.


\subsection{Fubini--Study metric}

The dual of the tautological line bundle is denoted by $\cc O(1) := \cc O(-1)^*$. By the above, it is a positive line bundle on the projective space $\kk P(V)$. Its curvature form $\omega$ is called the \emph{Fubini--Study metric} on the projective space. It is a K\"ahler metric. In local coordinates it is given by the Hermitian form $\psi := -\phi$.

We want to compute the Chern connection of this metric. By definition, it satisfies $\partial \psi(\alpha,\beta) = \psi(D\alpha,\beta)$ for holomorphic tangent fields $\alpha$ and $\beta$, and it is enough to know the connection on such fields. We have
\begin{align*}
\partial \psi(\alpha,\beta)
&= \frac{h(\partial\alpha, \beta)}{h(\xi,\xi)}
- \frac{h(\partial\xi, \xi)}{h(\xi,\xi)}
 \frac{h(\alpha,\beta)}{h(\xi,\xi)}
\\
&\qquad{}
- \frac{h(\partial\alpha,\xi)}{h(\xi,\xi)}
\frac{h(\xi,\beta)}{h(\xi,\xi)}
+ \frac{h(\partial\xi,\xi)}{h(\xi,\xi)}
\frac{h(\alpha,\xi)}{h(\xi,\xi)}
\frac{h(\xi,\beta)}{h(\xi,\xi)}
\\
&\qquad{}
- \frac{h(\alpha,\xi)}{h(\xi,\xi)} \frac{h(\partial\xi,\beta)}{h(\xi,\xi)}
+\frac{h(\alpha, \xi)}{h(\xi,\xi)}
\frac{h(\partial\xi,\xi)}{h(\xi,\xi)}
\frac{h(\xi,\beta)}{h(\xi,\xi)}
\\
&= \psi(\partial\alpha,\beta)
- \psi\biggl(\frac{h(\alpha,\xi)}{h(\xi,\xi)} \partial\xi, \beta \biggr)
- \psi\biggl(\frac{h(\partial\xi,\xi)}{h(\xi,\xi)} \alpha, \beta \biggr)
\end{align*}
so the Chern connection is
$$
 D \alpha
= d\alpha
- \frac{h(\alpha,\xi)}{h(\xi,\xi)} \partial\xi
- \frac{h(\partial\xi,\xi)}{h(\xi,\xi)} \alpha.
$$
For the curvature form, we then have
\begin{align*}
\frac i2\bar\partial D\alpha
&= -\frac i2 \partial\bar\partial \alpha
- \frac i2 \frac{h(\alpha,\partial\xi)}{h(\alpha,\xi)} \wedge \partial \xi
+ \frac i2 \frac{h(\alpha,\xi)}{h(\xi,\xi)} \frac{h(\xi,\partial\xi)}{h(\xi,\xi)} \wedge \partial\xi
\\
&\qquad{}
+ \frac i2\frac{h(\partial\xi,\partial\xi)}{h(\xi,\xi)} \alpha
- \frac i2\frac{h(\partial\xi,\xi)}{h(\xi,\xi)} \wedge \frac{h(\xi,\partial\xi)}{h(\xi,\xi)} \alpha
\\
&= - \psi(\alpha, \partial\xi) \wedge \partial \xi
+ \psi(\partial\xi, \partial\xi) \alpha.
\end{align*}
We have to be careful about signs here, as $\partial \xi \wedge \bar\partial \xi = - \bar\partial \xi \wedge \partial \xi$. Note also the sign flip when we commute $\partial\xi$ and $\bar\partial\xi$ below, where we conclude that the curvature tensor is
$$
R(\alpha,\beta,\gamma,\delta)
= h(\tfrac i2\Theta_{\alpha \beta} \gamma, \delta)
= \psi(\alpha, \beta) \psi(\gamma, \delta)
+ \psi(\alpha, \delta) \psi(\gamma, \beta).
$$
The extra symmetries
$$
R(\alpha,\beta,\gamma,\delta) = R(\gamma,\beta,\alpha,\delta)
\quad\text{and}\quad
R(\alpha,\beta,\gamma,\delta) = R(\alpha,\delta,\gamma,\beta)
$$
that a K\"ahler curvature tensor has compared to a Hermitian curvature tensor are very clear here.


The holomorphic bisectional curvature is
$$
B(\alpha,\beta) = \frac{R(\alpha,\alpha,\beta,\beta)}{\phi(\alpha,\alpha)\phi(\beta,\beta)}
= \frac{\phi(\alpha,\alpha)\phi(\beta,\beta)+|\phi(\alpha,\beta)|^2}{\phi(\alpha,\alpha)\phi(\beta,\beta)}.
$$
We have
$$
1 \leq B(\alpha,\beta) \leq 2
$$
by Cauchy--Schwarz. The holomorphic sectional curvature is
$$
H(\alpha) = B(\alpha,\alpha) = 2.
$$
To find the Ricci curvature we pick a holomorphic frame $(\zeta_1, \ldots, \zeta_n)$ that's orthonormal at a point $z$. There we have
$$
r(\alpha,\beta)
= \sum_{j=1}^n R(\alpha,\beta,\zeta_j,\zeta_j)
= \sum_{j=1}^n \phi(\alpha,\beta) + \phi(\alpha,\zeta_j)\phi(\zeta_j,\beta)
= (n+1) \phi(\alpha,\beta).
$$
The Fubini--Study metric is thus a K\"ahler--Einstein metric. Contracting the Ricci tensor we find that its scalar curvature is
$$
s = n(n+1).
$$




\section{Bergman metric}
\label{sec:org21fa1aa}

We're going to look at a special case of the \emph{Bergman metric}. Let $B = \{ z \in \kk C^n \mid |z|<1 \}$ be the unit ball. We define a $(1,1)$-form on $B$ by $\omega = \frac i2 \partial \bar \partial \log f$, where
$$
f(z) = \frac{1}{1-|z|^2}.
$$
As the definition suggests, this is the curvature form of a line bundle on $B$. In general, the Bergman metrics are the curvature forms of Hermitian metrics on the canonical bundle of a manifold, but we are going to skip that part of the theory here.

We have
$$
\bar\partial \log f
= - \bar\partial \log (1 -|z|^2)
= \frac{\langle z, dz\rangle}{1 - |z|^2}
$$
so
$$
\omega
= \frac i2 \partial\bar\partial \log f
= \frac{\frac i2 \langle dz, dz\rangle}{1-|z|^2}
+ \frac i2 \frac{\langle dz, z \rangle}{1-|z|^2}\wedge \frac{\langle z, dz \rangle}{1-|z|^2}.
$$
The Hermitian form associated to $\omega$ is thus
$$
h(\alpha,\beta)
=\frac{\langle \alpha, \beta\rangle}{1-|z|^2}
+ \frac{\langle \alpha, z \rangle}{1-|z|^2}
\cdot \frac{\langle z, \beta \rangle}{1-|z|^2}.
$$
As before, we recall that $|\langle \alpha, z \rangle| \leq |\alpha| |z|$, so
$$
h(\alpha,\alpha)
\geq \frac{|\alpha|^2}{1-|z|^2}
+ \frac{|\alpha|^2|z|^2}{(1-|z|^2)^2}
= \frac{|\alpha|^2}{(1-|z|^2)^2}
> 0
$$
if $\alpha \not= 0$, so $h$ is a K\"ahler metric.

To compute the Chern connection of $h$, we note that for holomorphic tangent fields we have
\begin{align*}
\partial_\gamma h(\alpha,\beta)
&= \frac{\langle d_\gamma\alpha, \beta\rangle}{1-|z|^2}
+ \frac{\langle \gamma, z \rangle}{1-|z|^2}
\cdot \frac{\langle \alpha, \beta\rangle}{1-|z|^2}
\\
&\qquad{}
+ \frac{\langle d_\gamma \alpha, z \rangle}{1-|z|^2}
\cdot \frac{\langle z, \beta \rangle}{1-|z|^2}
+ \frac{\langle \gamma, z \rangle}{1-|z|^2}
\cdot \frac{\langle \alpha, z\rangle}{1-|z|^2}
\cdot \frac{\langle z, \beta \rangle}{1-|z|^2}
\\
&\qquad{}
+ \frac{\langle \alpha, z\rangle}{1-|z|^2}
\cdot \frac{\langle \gamma, \beta \rangle}{1-|z|^2}
+
\frac{\langle \alpha, z\rangle}{1-|z|^2}
\cdot \frac{\langle \gamma, z \rangle}{1-|z|^2}
\cdot \frac{\langle z, \beta \rangle}{1-|z|^2}
\\
&=
h(d_\gamma \alpha, \beta)
+ h\biggl( \frac{\langle \gamma, z \rangle}{1-|z|^2} \alpha , \beta\biggr)
+ h\biggl( \frac{\langle \alpha, z \rangle}{1-|z|^2} \gamma , \beta\biggr)
\end{align*}
so the Chern connection of $h$ is
$$
D \alpha
= d \alpha
+ \frac{\langle dz, z \rangle}{1-|z|^2} \alpha
+ \frac{\langle \alpha, z \rangle}{1-|z|^2} dz.
$$
To find the curvature form, we then calculate
$$
\displaylines{
\bar\partial D \alpha
=
-\frac{\langle dz, dz \rangle}{1-|z|^2} \alpha
- \frac{\langle dz, z \rangle}{1-|z|^2}
\wedge \frac{\langle z, dz \rangle}{1-|z|^2} \alpha
\hfill\cr\hfill{}
+ \frac{\langle \alpha, dz \rangle}{1-|z|^2} \wedge dz
+ \frac{\langle \alpha, z \rangle}{1-|z|^2}
\cdot \frac{\langle \alpha, dz \rangle}{1-|z|^2} \wedge dz.
}
$$
After permuting some fields, taking the inner product with a fourth tangent field, and minding the order of wedge products, this results in the curvature tensor
$$
R(\alpha,\beta,\gamma,\delta)
= -h(\alpha,\beta) h(\gamma,\delta)
- h(\alpha,\delta) h(\gamma,\beta).
$$
By inspection, we recognize this as the curvature tensor of a metric with constant holomorphic sectional curvature $-2$. It follows that this is a K\"ahler--Einstein metric with $r(\alpha,\beta) = - (n+1) h(\alpha,\beta)$, and that its scalar curvature is $-n(n+1)$.


\section{Grassmannian}
\label{sec:org34425b6}

Let $V$ be a complex vector space of dimension $n$.
The Grassmannian $\Gr(k, V)$ is the set of $k$-dimensional subspaces of $V$. When $k = 1$ this is just the projective space we already saw.

Our discussion of the non-metric aspects of the Grassmannian mostly follows Voisin~\cite{voisin2002theorie}, while the metric aspects take some inspiration from Zheng~\cite{zheng2000complex}. See also Demailly~\cite[Chapter~5.16]{demailly-complex} for a whirlwind tour.

If $S \subset V$ is a subspace, then there is a short exact sequence
$$
0 \to S \to V \to V/S \to 0.
$$
Taking duals, we get
$$
0 \to (V/S)^* \to V^* \to S^* \to 0.
$$
Because our spaces are finite-dimensional we have $(S^*)^* = S$. This means that the correspondances $S \mapsto (V/S)^* \mapsto (V^* / (V/S)^*)^* = (S^*)^* = S$ define a natural bijection
$$
\Gr(k, V) = \Gr(n-k, V^*).
$$
We're used to seeing this in the projective space, where a hyperplane corresponds to the kernel of a line of linear functionals.

We can also pick an isomorphism $V \to V^*$, from which we get a non-canonical bijection $\Gr(k, V) \cong \Gr(n-k, V)$.

\subsection{Manifold structure}

Our first objective is to show the Grassmannian is a complex manifold.

\begin{prop}
$\Gr(k, V)$ is a complex manifold of dimension $k(n-k)$.
\end{prop}


\begin{proof}
Let $S \in \Gr(k, V)$, and pick $W \subset V$ such that $S \oplus W = V$. We denote the projections onto each factor by $\pi_S : V \to S$ and $\pi_W : V \to W$. We define the set of subspaces close to $S$ by
$$
U_S =
\{
S' \in \Gr(k, V)
\mid
S' \cap W = 0
\}.
$$

\begin{claim}
There is a bijection $U_s \cong \Hom(S, W)$ between $U_S$ and the set of linear morphisms $S \to W$.
\end{claim}

\begin{proof}
Let $S' \in U_S$. Since $\Ker \pi_S = W$ and $S' \cap W = 0$, the restriction of $\pi_S$ to $S'$ is an injective linear morphism $\pi_S : S' \to S$. As $S$ and $S'$ both have dimension $k$, $\pi_S$ is an isomorphism. We then define
$$
A: U_S \to \Hom(S, W),
\quad
S' \mapsto \pi_W \circ (\pi_S|_{S'})^{-1}.
$$
Going the other way, suppose $f \in \Hom(S, W)$. The graph of $f$, that is
$$
\Gamma(f) = \{ (s, f(s)) \in S \oplus W \mid s \in S \},
$$
identifies via $S \oplus W = V$ with a linear subspace $\Gamma(f) \subset V$. As the projection onto the first factor is an isomorphism with $S$, this subspace has dimension $k$. If $(s, f(s)) \in W$, then $s \in W \cap S$, so $s = 0$. Thus $\Gamma(f) \cap W = 0$, so $\Gamma(f) \in U_S$. This defines a map
$$
B: \Hom(S, W) \to U_S,
\quad
f \mapsto \Gamma(f).
$$
Let's now take $f \in \Hom(S, W)$. Then its image under the second map is the subspace $\Gamma(f) = \{(s, f(s)) \mid s \in S\}$. The restriction of $\pi_S$ to $\Gamma(f)$ is $\pi_S((s, f(s))) = s$. Then $\pi_W \circ (\pi_S|_{\Gamma(f)})^{-1}(s) = \pi_W((s, f(s))) = f(s)$, so $AB = \id_{\Hom(S, W)}$.

Finally, take $S' \in U_S$. It maps to $\pi_W \circ (\pi_S|_{S'})^{-1}$. The graph of this morphism is $\{(s, \pi_W(s')) \mid s \in S\}$ where $s' \in S'$ is the unique element such that $\pi_S(s') = s$. This is equal to the set $\{(\pi_s(s'), \pi_W(s')) \mid s' \in S'\}$. That is just the image of $S'$ under the isomorphism $V \to S \oplus W$, so applying its inverse $S \oplus W \to V$ we get $BA = \id_{U_S}$.
\end{proof}

Each set $\Hom(S, W)$ is a complex vector space, and we transport its topology to the set $U_S$. This induces a topology on the Grassmannian. By construction, each point has a neighborhood homomorphic to a complex vector space. To show that we have a complex structure, we need to show the following:


\begin{claim}
  The transition morphisms between these neighborhoods are holomorphic.
\end{claim}

\begin{proof}
Let $S, S' \in \Gr(k,V)$ be subspaces such that $U_S \cap U_{S'} \not= \varnothing$. We let $W,W' \subset V$ be the complementary subspaces chosen such that $S \oplus W = V$ and $S' \oplus W' = V$. Let $A = \{ f \in \Hom(S, W) \mid \Gamma(f) \cap W' = 0 \}$. This is an open set; by putting a Hermitian metric on $V$ we can test the distance of a subspace from $W$ on the unit sphere, where compactness lets us deform $f$ a little in any direction.

The coordinate change map $A \to \Hom(S',W')$ first maps $f$ to $\Gamma(f)$ and then to $\pi_{W'} \circ (\pi_{S'|\Gamma(f)})^{-1}$. The map $(\pi_{S'|\Gamma(f)})^{-1}$ takes $s' \in S'$ and maps it to the unique $(s, f(s))$ such that $\pi_{S'}(s, f(s)) = s'$ (this makes sense because $f \in A$). This is just $s' \mapsto (\pi_{S'|S}^{-1}(s'), f \circ \pi_{S'|S}^{-1}(s'))$, so the coordinate change map is
$$
f \mapsto \pi_{W'} \circ f \circ (\pi_{S'}|_{S})^{-1}.
$$
This is the composition of some complex linear morphisms, and thus holomorphic in $f$.
\end{proof}

The Grassmannian is thus a complex manifold of dimension
\[
\dim \Hom(S, W) = k (n-k).
\qedhere
\]
\end{proof}

When we defined this complex structure we had to pick complementary subspaces $W$ for each point $S \in \Gr(k,V)$. As we have isomorphisms $\Hom(S, W) = \Hom(S, V/S)$ for every such $W$, changing the complementary subspace results in isomorphic local affine models.

In fact we can just define the local models to be \emph{all} collections $U_S(W)$ for \emph{any} complementary subspace $W$ to $S$. Changing from one such $U_S(W)$ to another $U_S(W')$ (same $S$) amounts to a coordinate change map, which we have just shown are holomorphic, so we end up with the same manifold structure.

We can't use $\Hom(S, V/S)$ directly as a local model for the manifold structure. This makes me sad because I don't like making arbitrary choices.\footnote{As you may have noticed from the rest of these notes.} However, this would lead us to look at sets of $S' \subset S \oplus V/S$ such that $S' \to S$ is an isomorphism. But then $S'$ is not a subspace of $V$, so it's not an element of the Grassmannian. We need to fix an isomorphism $V \cong S \oplus V/S$ to be able to proceed, and then we'd have to show that the result doesn't depend on the choice of that isomorphism.



\paragraph{Grassmannians are homogeneous and compact}

The general linear group $\GL(V)$ acts transitively on $V$ and sends $k$-dimensional subspaces to $k$-dimen\-sional subspaces. It thus acts on the Grassmannian as well.

\begin{prop}
$\GL(V)$ acts transitively on $\Gr(k,V)$.
\end{prop}

\begin{proof}
It's easiest to prove this by picking bases (sigh). Let $S,S' \in \Gr(k, V)$. Pick a basis $(v_1,\ldots,v_n)$ of $V$ such that $S = \Span(v_1,\ldots,v_k)$, and a basis $(v_1', \ldots, v_n')$ such that $S' = \Span(v_1', \ldots, v_k')$. Then we can define $f \in \GL(V)$ by $f(v_j) = v_j'$ for all $j$, and $f(S) = S'$.
\end{proof}

The orbit-stabilizer theorem then gives:

\begin{coro}
For any $S \in \Gr(k, V)$, we have $\Gr(k, V) \cong \GL(V) / H_S$, where $H_S = \{ f \in \GL(V) \mid f(S) = S \}$ is the stabilizer of $S$.
\end{coro}

We're working over $\kk C$, where we also have actions by other groups. In particular, if we fix an inner product on $V$, then Gram--Schmidt implies that any subspace admits an orthonormal basis that extends to an orthonormal basis of $V$. That is:

\begin{prop}
If $h$ is a Hermitian inner product on $V$, then $U(V,h)$ acts transitively on $\Gr(k, V)$.
\end{prop}

As $U(V,h)$ is a compact Lie group, this gives:

\begin{coro}
\label{coro:grassmannian-is-compact}
$\Gr(k, V)$ is compact.
\end{coro}

It's possible to prove the compactness of the Grassmannian without resorting to Lie groups. As it's a lot of fun to do so, we'll outline how to do~it.%

\begin{proof}[Sketch of alternate proof of \thref{coro:grassmannian-is-compact}]
Consider the incidence variety
$$
X = \{
(l, S) \in \kk P(V) \times \Gr(k, V)
\mid
l \subset S
\}.
$$
It is a smooth submanifold of $\kk P(V) \times \Gr(k, V)$.

The projections $\pi_1 : X \to \kk P(V)$ and $\pi_2 : X \to \Gr(k, V)$ are holomorphic and surjective. The fiber $\pi_1^{-1}(l) = \{ S \in \Gr(k,V) \mid l \subset S \}$ identifies with the space of $(k-1)$-subspaces in $V / l$, so $\pi_1 : X \to \kk P(V)$ is a $\Gr(k-1,V/l)$-bundle. The fiber $\pi_2^{-1}(S)$ identifies with $\kk P(S)$, so $\pi_2 : X \to \Gr(k, V)$ is a projective bundle.

We now reason by induction on $\dim V$ and $k$, the cases of $\dim V$ arbitrary and $k = 1$ having already been proven. Suppose that $\Gr(j, W)$ is compact for all $W$ with $\dim W < \dim V$ and $j < k$. The family $\pi_1 : X \to \kk P(V)$ is then a bundle with compact fibers over a compact base, so the total space $X$ is compact. The map $\pi_2 : X \to \Gr(k,V)$ is then a surjective continuous map, so $\Gr(k,V)$ is compact.
\end{proof}



\subsection{Universal bundles}

Consider the trivial vector bundle $\cc V \to \Gr(k,V)$ of rank $\dim V$. Similarly to the case of projective space, each point of $\Gr(k,V)$ gives a subspace $S \subset V$. This defines the \emph{universal subbundle} $\cc S \subset \cc V$ whose fiber over $S$ is the subspace~$S$. We also get a quotient bundle $\cc Q = \cc V/\cc S$ and a short exact sequence
$$
0 \to \cc S \to \cc V \to \cc Q \to 0
$$
of vector bundles over $\Gr(k,V)$. The bundle $\cc S$ has rank $k$ and the bundle $\cc Q$ has rank $n-k$.


\begin{prop}
The bundles $\cc S$ and $\cc Q$ are holomorphic.
\end{prop}

\begin{proof}
  It's enough to prove that $\cc S$ is holomorphic, because then $\cc Q = \cc V/\cc S$ is a quotient of holomorphic bundles.

  Let $S \in \Gr(k, V)$ and let $W$ be an orthogonal complement. Over an affine chart $U_S$, the bundle $\cc S \to U_S$ is isomorphic to a trivial bundle as $\cc S \to U_S \times S$ via $(\id_{U_S}, \pi_S)$. If $S'$ is a different element with affine chart $U_{S'}$, the coordinate change function on the intersection $U_S \cap U_{S'}$ is $\pi_{S'}$. The trivializations are then related by $(U_S \cap U_{S'}) \times S \to (U_{S'} \cap U_S) \times S'$ by $(\pi_{S'}, \pi_{S'})$, which is holomorphic.
\end{proof}


\begin{prop}
The tangent bundle of $\Gr(k,V)$ is isomorphic to $\cc S^* \otimes \cc Q$.
\end{prop}


\begin{proof}
The Grassmannian is covered by neighborhoods around each $S$ that are isomorphic to $\Hom(S, W)$, where $W$ is a complement of $S$. All of these are isomorphic to $\Hom(S, V/S)$ via the quotient map $q : V \to V/S$. Thus
\[
\cc T_{\Gr(k,V),S}
\cong \Hom(S,W)
= \Hom(S,V/S)
= (\cc S^* \otimes \cc Q)_S.
\]
The first isomorphism depends on the choice of $W$. Any difference from picking a different one gets eliminated by the passage to $\Hom(S,V/S)$, so the total isomorphism doesn't depend on $W$.
\end{proof}


\begin{coro}
The canonical bundle is $K_{\Gr(k, V)} = -\det \cc Q^{n}$.
\end{coro}

\begin{proof}
The short exact sequence
$$
0 \to \cc S \to \cc V \to \cc Q \to 0
$$
gives $\det \cc S \otimes \det \cc Q = \cc O$. Then
\begin{align*}
K_{\Gr(k,V)}
= \det \cc T_{\Gr(k,V)}^*
&= \det (\cc S^* \otimes \cc Q)^*
\\
&= - (\det \cc S^*)^{n-k} \otimes (\det \cc Q)^k
= - (\det \cc Q)^{n}.
\qedhere
\end{align*}
\end{proof}

As a sanity check, on projective space we have $\det Q = \cc O(1)$ and this gives $K_{\kk P(V)} = \cc O(-\dim V)$. The usual formula is $K_{\kk P^n} = \cc O(-n-1)$, which matches with our convention here.


If $S \subset V$ is a subspace of dimension $k$, then $\bigwedge^k S \subset \bigwedge^k V$ is also a subspace. As $\bigwedge^k S$ is a line, this defines a map $\Gr(k,V) \to \kk P(\bigwedge^k V)$ called the \emph{Pl\"ucker embedding}. We're not going to use it, but will just mention that it is an honest holomorphic embedding, and that the pullback of $\cc O(1)$ via it is $\det \cc Q$.



\subsection{Metric and curvature}


We want to calculate the Chern connection and curvature of the metric on the Grassmannian. We begin with some technical points.

\begin{prop}
\label{prop:metric-parallel}
Let $(E,h) \to X$ be a Hermitian holomorphic vector bundle, and let $D$ be a connection on $E$. Then
\[
d h(s,t)
= h(D_Es, t) + h(s, D_Et)
+ (D_{\Hom(E,\overline E^*)}h)(s,t),
\]
so $D$ is compatible with $h$ if and only if $D_{\Hom(E,\overline E^*)}h = 0$.
\end{prop}

\begin{proof}
We have $h(s,t) = h(s)(\ov t)$ when we view $h$ as a morphism $h : E \to \ov E^*$, so
\begin{align*}
d h(s,t)
&= D_{\ov E^*}(h(s))(\ov t)
+ h(s)(\ov{D_Et})
\\
&= (D_{\Hom(E,\overline E^*)}h)(s)(\ov t)
+ h(D_E s)(\ov t)
+ h(s)(\ov{D_Et}).
\end{align*}
\end{proof}


\begin{prop}
\label{prop:trace-integral}
Let $V$ be a complex vector space of dimension $n$, $h$ an inner product on $V$, and $f \in \End V$. Then
\[
\int_{S(V,h)} h(f(x),x) d\sigma(x)
= \frac{2\pi^n}{n!} \tr f,
\]
where $S(V,h)$ is the unit sphere in $V$ and $d\sigma$ is the Lebesgue measure.
\end{prop}


\begin{proof}
Let $(v_1,\ldots,v_n)$ be an orthonormal basis of $V$. If $(f_{jk})$ is the matrix of $f$ in this basis, and $z = \sum_k z_k v_k$ a vector, then $f(z) = \sum_{jk} f_{jk} z_k v_j$ and
\[
h(f(z), z)
= \sum_{jkl} f_{jk} z_k \ov z_l h(v_j, v_l)
= \sum_{jk} f_{jk} z_k \ov z_j.
\]
Terms that have $j \not= k$ are odd functions on the sphere, and their integral vanishes. Folland~\cite{folland} says that
\[
\int_{S^{2n-1}} x_j^2 d\sigma(x)
= \frac{2\Gamma(3/2)\Gamma(1/2)^{2n-1}}{\Gamma(n+1)}
= \frac{\pi^n}{n!}.
\]
Applying this to the real and imaginary parts of $|z_j|^2$ we get
\[
\int_{S(V,h)} h(f(x),x) d\sigma(x)
= \frac{2\pi^n}{n!} \tr f.
\]
\end{proof}

The value of the constant here is $\Vol(S(V,h)) / n$, but that won't be important.


\begin{prop}
The second fundamental form of $S$ is holomorphic and an isomorphism
\[
b : T_{\Gr} \to \Hom(S,Q)
\]
of holomorphic vector bundles.
\end{prop}

\begin{proof}
The trivial bundle $(V,h) \to \Gr(k, V)$ is flat and we know this implies that $\bar\partial b = 0$, so $b$ is holomorphic.
\end{proof}


To construct a metric on $\Gr(k,V)$, we're going to fix an inner product $h_V$ on $V$. The second fundamental form gives an isomorphism
\[
b : T_{\Gr} \to \Hom(\cc S, \cc Q)
\]
and the fiber of the right-hand side at $S \in \Gr(k,V)$ is $\Hom(S, Q)$, where $Q = V/S$. The inner product on $V$ induces inner products on $S$ and $Q$, and in fact Hermitian metrics on $\cc S$ and $\cc Q$. We then have the Fr\"obenius (or Hilbert--Schmidt) inner product on $\Hom(S,Q)$, defined by
\[
h(f, g) = \tr(h_S^{-1} \ov g^* h_Q f),
\]
where $f, g \in \Hom(S,Q)$. As $h_S$ and $h_Q$ vary smoothly with $S$, this defines a smooth Hermitian metric on $\Hom(\cc S,\cc Q)$ that we pull back via $b$ to a metric on $\Gr(k,V)$.


\begin{prop}
The Chern connection of this metric is $D\alpha = D_{\Hom(S,Q)}(b(\alpha))$, where the connection is constructed from the Chern connections induced by the flat metric $h_V$ on $V$.
\end{prop}

\begin{proof}
If $f,g \in \Hom(S,Q)$ and $x, y \in S$, then $\ov g \circ h_Q \circ f(x)(\ov y) = h_Q(f(x), g(y))$. Then there exists a unique $\theta \in \End S$ such that
\[
h_S(\theta(x), y) = h_Q(f(x), g(y))
\]
for all $x,y \in S$
and we have $h(f, g) = \tr \theta$. Furthermore,
\[
d h(f, g)
= d \tr \theta
= \tr D_{\End S} \theta
= \frac{n!}{2\pi^n} \int_{S(S,h)} h_S(D_{\End S} \theta(x), x) d\sigma(x)
\]
by \thref{prop:trace-integral}.
Note that
\[
h(f, g)
= \frac{n!}{2\pi^n} \int_{S(S,h)} h_S(\theta(x),x) d\sigma(x)
= \frac{n!}{2\pi^n} \int_{S(S,h)} h_Q(f(x),g(x)) d\sigma(x)
\]
for any $f, g$.
We now take the exterior derivative of both sides of the equality  $h_S(\theta(x), y) = h_Q(f(x), g(y))$.
The left-hand side gives
\[
d h_S(\theta(x), y)
= h_S(D_{\End S}\theta (x), y)
+ h_S(\theta(D_Sx), y)
+ h_S(\theta(x), D_S y)
\]
while the right-hand side gives
\begin{align*}
d h_Q(f(x), g(y))
&= h_Q(D_{\Hom(S,Q)}f (x), g(y))
+ h_Q(f(D_S x), g(y))
\\
&\qquad
{}+ h_Q(f(x), D_{\Hom(S,Q)} g(y))
+ h_Q(f(x), g(D_Sy)
\\
&= h_Q(D_{\Hom(S,Q)}f (x), g(y))
+ h_S(\theta(D_Sx), y)
\\
&\qquad
+ h_Q(f(x), D_{\Hom(S,Q)} g(y))
+ h_S(\theta(x), D_S y).
\end{align*}
Comparing the two yields
\[
h_S(D_{\End S} \theta(x), y)
= h_Q(D_{\Hom(S,Q)}f (x), g(y))
+ h_Q(f(x), D_{\Hom(S,Q)} g(y)).
\]
for all $x, y \in S$.
Then we get
\[
d h(f, g)
= h(D_{\Hom(S,Q)}f, g)
+ h(f, D_{\Hom(S,Q)} g)
\]
for all $f,g$.
\end{proof}


\begin{proof}[Alternate]
Recall that $D_{\Hom(E,\ov E^*)}h = 0$ for the Chern connection of a metric on a bundle $E$ by \thref{prop:metric-parallel}.
We calculate
\begin{align*}
d h(f,g)
&= d \tr(h_S^{-1} \ov g^* h_Q f)
\\
&= \tr((D_{\Hom(\ov S^*, S)} h_S^{-1})\ov g^* h_Q f)
+ \tr(h_S^{-1} D_{\Hom(S,\ov S^*)}(\ov g^* h_Q f))
\\
&= \tr(h_S^{-1} (D_{\Hom(\ov Q^*, \ov S^*)}\ov g^*)  h_Q f)
+ \tr(h_S^{-1} \ov g^* D_{\Hom(S,\ov Q^*)}( h_Q f))
\\
&= \tr(h_S^{-1} (D_{\Hom(\ov Q^*, \ov S^*)}\ov g^*)  h_Q f)
+ \tr(h_S^{-1} \ov g^* (D_{\Hom(Q,\ov Q^*)} h_Q) f)
\\
&\qquad
+ \tr(h_S^{-1} \ov g^*  h_Q (D_{\Hom(S,Q)} f)
\\
&= \tr(h_S^{-1} \ov{(D_{\Hom(S, Q)} g)^*}  h_Q f)
+ \tr(h_S^{-1} \ov g^*  h_Q (D_{\Hom(S,Q)} f)
\\
&= h(D_{\Hom(S,Q)} f, g) + h(f, D_{\Hom(S, Q)} g).
\end{align*}
\end{proof}

\begin{proof}[Alternate 2: Electric bugaloo]
As the metric on $V$ is flat, we have $D_{\Hom(S,Q)}b = 0$. Then
\[
D_{\Hom(S,Q)}(b(\alpha))
= (D_{\Hom(S,Q)}b)(\alpha) + b(D\alpha)
= b(D\alpha).
\]
In particular, because $\bar\partial b = 0$, we see that $D^{0,1} = \bar\partial$.
Now,
\begin{align*}
d\langle \alpha, \ov\beta \rangle
= d \langle b(\alpha), \ov{b(\beta)} \rangle
&= \langle D(b(\alpha)), \ov{b(\beta)} \rangle
+ \langle b(\alpha), \ov{D(b(\beta))} \rangle
\\
&= \langle b(D\alpha)), \ov{b(\beta)} \rangle
+ \langle b(\alpha), \ov{b(D\beta)} \rangle
\\
&= \langle D\alpha, \ov \beta \rangle
+ \langle \alpha, \ov{D\beta} \rangle
\end{align*}
so $D$ is compatible with the metric.
\end{proof}


\begin{prop}
The metric $h$ on $\Gr(k,V)$ is K\"ahler.
\end{prop}

\begin{proof}
If $\alpha$ and $\beta$ are tangent fields, then
\(
b(D_\alpha \beta)
= D_{\Hom(S,Q),\alpha} b(\beta)
\),
so
\begin{align*}
b(\tau(\alpha,\beta))
&= b(D_\alpha\beta) - b(D_\beta\alpha) - b([\alpha,\beta])
\\
&= D_{\Hom(S,Q),\alpha} b(\beta)
- D_{\Hom(S,Q),\beta} b(\alpha)
- b([\alpha,\beta])
\\
&= (D_{\Hom(S,Q)}b)(\alpha,\beta).
\end{align*}
As $V$ is flat we have $D_{\Hom(S,Q)} b = 0$, so $b(\tau) = 0$. As $b$ is an isomorphism, we conclude that $\tau = 0$.
\end{proof}

\begin{coro}
The metric $h$ is invariant under the action of $U(V,h)$ on $\Gr(k,V)$.
\end{coro}

\begin{proof}
The Fr\"obenius metric on $\End V$, from which our metric is constructed, is invariant under that action.
\end{proof}


\begin{prop}
The curvature tensor of the Grassmannian $\Gr(k,V)$ at a point $S$ is
\[
R(\alpha,\ov\beta,\gamma,\ov\delta)
= h(b(\beta)^\dagger b(\alpha), \ov{b(\gamma)^\dagger b(\delta)})_{\End S}
+ h(b(\beta)^\dagger b(\gamma), \ov{b(\alpha)^\dagger b(\delta)})_{\End S},
\]
where $\alpha,\beta,\gamma,\delta \in \cc T_{\Gr,S}$ and $b$ is the second fundamental form of the universal subbundle.
\end{prop}


\begin{proof}
The second fundamental form of $S$ gives an holomorphic bundle isomorphism $b : T_{\Gr} \to \Hom(S, Q)$. Note that the bundle $\End V$ is flat. Then the result follows by \thref{prop:hom-bundle-curvature}.
\end{proof}



\begin{prop}
\begin{itemize}
\item
The holomorphic bisectional curvature satisfies
\[
0 \leq B(\alpha,\beta)
\leq 2.
\]
If $1 < k < n$ then the holomorphic bisectional curvature can be zero.
\item
The holomorphic sectional curvature satisfies
\[
\frac{2}{k^2} \leq H(\alpha) \leq 2.
\]
\item
The Ricci tensor is
\[
r(\alpha,\ov\beta) = n \, h(\alpha,\ov\beta)
\]
so the metric is K\"ahler--Einstein and the Grassmannian is a Fano manifold.
\item
The scalar curvature is constant and equal to $k (n-k) n$.
\end{itemize}
\end{prop}


\begin{proof}
Recall that the holomorphic bisectional curvature is
\[
B(\alpha,\beta)
= \frac{R(\alpha,\ov\alpha,\beta,\ov\beta)}{|\alpha|^2|\beta|^2}.
\]
We already know that $R$ is Griffiths-semipositive, as there is a holomorphic surjection $\End V \to \Hom(S,Q)$ that induces the metric on the latter, and $\End V$ is flat. Thus $B \geq 0$.

In what follows we won't distinguish between a tangent field and its image under the second fundamental form.
We have
\[
R(\alpha,\ov\alpha,\beta,\ov\beta)
= \langle \alpha^\dagger \alpha, \ov{\beta^\dagger \beta} \rangle
+ |\alpha^\dagger \beta|^2
\]
If $1 < k < n-1$, then both $S$ and $Q$ have dimension greater than 1.
We can then find two orthogonal lines in $S$ and two orthogonal lines
in $Q$. Let $\alpha : S \to Q$ send one line in $S$ to one line in $Q$
and other vectors to 0, and $\beta : S \to Q$ be similar but to
operate on the other pair of lines.
Then $\alpha^\dagger \beta = 0$, and $\alpha^\dagger \alpha$ is a morphism that scales one line in $S$ by some nonzero number, and likewise for $\beta^\dagger \beta$. As the lines are orthogonal, we have $\langle \alpha^\dagger \alpha, \beta^\dagger \beta \rangle = 0$. Then $B(\alpha,\beta) = 0$.

If $k = 1$ or $k = n-1$ then we're looking at the Fubini--Study metric on projective space, which we know to have positive holomorphic bisectional curvature.

For the upper bound,
Cauchy--Schwarz gives
\[
|\langle \alpha^\dagger \alpha, \ov{\beta^\dagger \beta} \rangle|^2
\leq
|\alpha^\dagger \alpha|^2 |\beta^\dagger \beta|^2
\]
with equality if and only if $\alpha^\dagger\alpha$ and $\beta^\dagger\beta$ are linearly dependent.
The Fr\"obenius norm is submultiplicative, so we further get that
\[
|\langle \alpha^\dagger \alpha, \ov{\beta^\dagger \beta} \rangle|
\leq |\alpha|^2 |\beta|^2
\qandq
|\alpha^\dagger \beta|^2
\leq |\alpha|^2 |\beta|^2.
\]
Then
\[
B(\alpha,\beta) \leq 2
\]
and equality can be acheived.

The holomorphic sectional curvature is
\[
H(\alpha) = 2 \frac{|\alpha^\dagger \alpha|^2}{|\alpha|^4} \leq 2.
\]
We have
\[
|\alpha|^2
= \langle \alpha^\dagger \alpha, \id_S \rangle
\leq k \, |\alpha^\dagger \alpha|
\]
by Cauchy--Schwarz, so
\[
\frac{2}{k^2} \leq H(\alpha) \leq 2
\]
for all $\alpha$, with equality on the left-hand side if and only if
$\alpha^\dagger \alpha$ is a multiple of the identity. This happens
when the nonzero morphism $\alpha : S \to Q$ is injective; thus it \emph{always} happens when $k = 1$, and \emph{never} happens when $k > n-k$.\footnote{The situations for $k$ and $n-k$ are the same, as $\Gr(k,V) = \Gr(n-k,V^*)$, so for $k > n-k$ the optimal bound lower bound is $2/(n-k)^2$.}


For the Ricci tensor, we let $(v_1,\ldots,v_n)$ be an orthonormal basis of $V$ such that $(v_1,\ldots,v_k)$ is a basis of $S$ and identify $Q$ with $S^\perp = \Span(v_{k+1},\ldots,v_n)$.
Then $v_{lm} = v_m \otimes v_l^*$ for $1 \leq l \leq k$ and $k+1 \leq m \leq n$ form an orthonormal basis of $\Hom(S,Q)$.
We want to calculate
\[
r(\alpha,\ov\beta)
= \sum_{l,m} R(\alpha,\ov\beta,v_{lm}, \ov v_{lm})
= \sum_{l,m}
\langle \beta^\dagger \alpha, \ov{ v_{lm}^\dagger v_{lm}} \rangle
+ \langle \beta^\dagger v_{lm}, \ov{ \alpha^\dagger v_{lm} }\rangle.
\]
For the first sum, note that $v_{lm}^\dagger v_{lm} = (v_m^* \otimes v_l)(v_m \otimes v_l^*) = v_l^* \otimes v_l$. Then
\[
\sum_{l,m} \langle \beta^\dagger \alpha, \ov{ v_{lm}^\dagger v_{lm}} \rangle
= \sum_{l,m} \langle \beta^\dagger \alpha, \ov{ v_{l}^* v_{l}} \rangle
= \sum_{m} \langle \beta^\dagger \alpha, \ov{\id_S}\rangle
= (n-k) \langle \alpha, \ov \beta \rangle.
\]
If $f : S \to Q$ is a linear morphism, we identify it with $\sum_{r,s}
f_{sr} v_r^* \otimes v_s$. Then $f^\dagger = \sum_{r,s} \ov f_{rs}
v_r \otimes v_s^*$, and
\[
f^\dagger v_{lm}
= \sum_{r,s} \ov f_{rs} (v_r \otimes v_s^*)(v_m \otimes v_l^*)
= \sum_{r} \ov f_{rm} v_r \otimes v_l^*.
\]
Then
\begin{align*}
\sum_{l,m} \langle \beta^\dagger v_{lm}, \ov{ \alpha^\dagger v_{lm} }\rangle
&= \sum_{l,m} \sum_{r,s} \alpha_{sm} \ov \beta_{rm}
\langle v_r \otimes v_l^*, \ov{v_s \otimes v_l^*} \rangle
\\
&= \sum_{l} \sum_{m,r} \alpha_{rm} \ov \beta_{rm}
= k \langle \alpha, \ov\beta \rangle.
\end{align*}
Together, we get $r(\alpha,\ov\beta) = n \, h(\alpha, \ov\beta)$.\footnote{This \emph{does} match with what we found for the Fubini--Study metric. There we got the Einstein constant $n+1$ on the projective space of dimension $n$, but that is the dimension of the vector space $V$ we started from and not of the projective space.}

Finally the claim about the scalar curvature follows most easily by taking the trace of the Ricci tensor, which is then equal to $n$ times the dimension of the tangent space.
\end{proof}


Projective space is characterized by having Griffiths-positive
curvature. We have seen the Grassmannian has Griffiths-semipositive
curvature, and could ask whether that characterizes the Grassmannians?
This is much too optimistic, as the product of a projective space and
a torus has semipositive curvature. There are also in fact more
irreducible manifolds with semipositive curvature.
Mok~\cite{mok1988uniformization} found all of them by studying a version of Hamilton's Ricci flow for K\"ahler manifolds. The manifolds he obtains include the
Grassmannians, but also other symmetric spaces.

\section{Hopf manifold}
\label{sec:org8f5818e}

Let \(\lambda \in \kk C\) be a complex number such that \(0 < |\lambda| < 1\). The \emph{Hopf manifold} is the quotient
$$
X := (\kk C^n \setminus \{0\}) / \Gamma,
$$
where \(\Gamma \cong \kk Z\) is the group generated by \(\lambda\) that acts by
$$
\lambda \cdot (z_1, \ldots, z_n) = (\lambda z_1, \ldots, \lambda z_n).
$$
The Hopf manifold is compact and is diffeomorphic to \(S^{2n-1} \times S^1\). In particular,
$$
H^2(X, \kk C) \cong H^2(S^{2n-1}, \kk C) \oplus H^1(S^{2n-1}, \kk C) \otimes H^1(S^1, \kk C) = 0,
$$
so it is not K\"ahler.

Let \(\pi : \kk C^n \setminus \{0\} \to X\) be the projection. If \(h\) is a Hermitian metric on \(X\), then its pullback \(\pi^*h\) is a Hermitian metric on \(\kk C^n \setminus \{0\}\) that is invariant under the action of \(\Gamma\). If we write its K\"ahler form as \(\sum_{j,k} a_{jk}(z) \tfrac{i}{2} dz_j \wedge d\bar z_k\), then the smooth functions \(a_{jk}\) must satisfy
$$
a_{jk}(\lambda z) = \frac{1}{|\lambda|^2} a_{jk}(z).
$$
We can pick one such metric to inspect; we'll choose \(h = \frac{1}{\|z\|^2} h_{\mathrm{std}}\), that is, a metric that is conformal to the standard metric on \(\kk C^n \setminus \{0\}\). The Ph.D. thesis \cite{istrati:tel-02156198} has a nice discussion of the history of these metrics on the Hopf manifold.

As the metric is conformal to a K\"ahler metric, and the conformal factor is non-constant, the metric is not K\"ahler. (We already knew this because \emph{no} metric on the Hopf manifold is K\"ahler, but it's nice to check.)


\subsection{Curvature tensor}
\label{sec:org96d544d}

We've \hyperref[sec:org65fcbad]{already computed} the curvature of a conformal metric, so we know the curvature form of the metric $h$ is
$$
D^2 s = \partial\bar\partial \log \|z\|^2 \otimes s.
$$
Let's compute this and express the curvature tensor of the metric. To do that we'll use the Euler field
$$
\xi = \sum_{j=1}^n z_j \frac{\partial}{\partial z_j}.
$$
It is a holomorphic tensor field whose norm is $\|\xi\|^2 = \|z\|^2$ and satisfies $\partial_\alpha \xi = \alpha$ for holomorphic tensor fields $\alpha$. We have
$$
\bar\partial \log \|z\|^2
= \frac{\langle \xi, \partial\xi \rangle}{\langle \xi,\xi \rangle}
$$
and
$$
\partial\bar\partial \log \|z\|^2
= \frac{\langle \partial \xi, \partial \xi \rangle}{\langle \xi, \xi \rangle}
- \frac{\langle \partial\xi, \xi\rangle}{\langle \xi, \xi \rangle} \wedge \frac{\langle \xi, \partial \xi \rangle}{\langle \xi, \xi \rangle}
= h(\partial\xi, \partial\xi) - h(\partial\xi, \xi) \wedge h(\xi, \partial\xi).
$$
The curvature tensor of the metric \(h\) on the Hopf manifold is then
$$
R(\alpha,\beta,\gamma,\delta)
= h(\alpha, \beta) h(\gamma, \delta)
- h(\alpha, \xi) h(\xi, \beta) h(\gamma, \delta).
$$
We note that it has the expected conjugate symmetries, that is, that \(R(\beta, \alpha, \delta, \gamma) = \overline{R(\alpha, \beta, \gamma, \delta)}\), but \(R(\gamma, \delta, \alpha, \beta) \not= R(\alpha, \beta, \gamma, \delta)\) like it would if this were the curvature tensor of a K\"ahler metric.


\subsection{Holomorphic sectional curvature}
\label{sec:org6471503}

The holomorphic bisectional curvature of the metric is
$$
B(\alpha,\beta)
= \frac{R(\alpha,\beta,\alpha,\beta)}{\|\alpha\|^2\|\beta\|^2}
= \frac{|h(\alpha,\beta)|^2 - h(\alpha,\xi)h(\xi,\beta)h(\alpha,\beta)}{\|\alpha\|^2\|\beta\|^2}.
$$
The holomorphic sectional curvature of the Hopf manifold is
$$
H(\alpha)
= B(\alpha,\alpha)
= \frac{\|\alpha\|^4 - |h(\alpha,\xi)|^2 \|\alpha\|^2}{\|\alpha\|^4}
= \frac{\|\alpha\|^2 - |h(\alpha,\xi)|^2}{\|\alpha\|^2}.
$$
We have the bounds
$$
0 \leq H(\alpha) \leq 1
$$
that are obtained when $\alpha$ is a multiple of $\xi$ and when it is orthogonal to $\xi$, respectively.

\subsection{Ricci tensors}
\label{sec:org3942125}

The curvature tensor can be contracted in three ways to obtain a \((1,1)\)-form. On a K\"ahler manifold, all three ways give the same result; on a non-K\"ahler manifold they may not.

The easiest of these to compute for us is the one given by taking the traces of the endomorphisms in the curvature form. As those endomorphisms are the identity here, we find that
$$
r_1(\alpha, \beta)
= n \bigl( h(\alpha, \beta) - h(\alpha, \xi) h(\xi, \beta) \bigr).
$$
This is the same as we obtain by contracting the curvature tensor along \(\delta\) and \(\gamma\). As before, Cauchy--Schwarz gives us the estimates
$$
0
\leq \frac{r_1(\alpha, \alpha)}{\|\alpha\|^2}
\leq n
$$
which are sharp under the same conditions as before. This form is the curvature form of the anti-canonical bundle on $X$ when equipped with the metric induced by $h$, and surprisingly it is semipositive. It cannot be positive, of course, as that would imply that $X$ were projective, but it comes as close as it can.

Our second contraction is along \(\alpha\) and \(\beta\). The only relevant part of the curvature tensor we don't know how to contract is \(h(\alpha, \xi)h(\xi, \beta)\). Let \((\zeta_1, \ldots, \zeta_n)\) be a local holomorphic frame that's orthonormal at a point \(z\) we care about. We have
$$
\sum_{j=1}^n h(\zeta_j, \xi) h(\xi, \zeta_j) = h(\xi, \xi) = 1
$$
as \(\xi = \sum_{j=1}^n h(\xi, \zeta_j) \zeta_j\) and \(h(\xi,\xi) = 1\).
Then
$$
r_2(\gamma, \delta)
= n h(\gamma, \delta) - h(\gamma, \delta)
= (n-1) h(\gamma, \delta).
$$
This form is not only different from \(r_1\) but it is positive-definite.

The third contraction is along \(\beta\) and \(\gamma\). We let \((\zeta_1, \ldots, \zeta_n)\) be a local holomorphic frame that's orthonormal at a point \(z\) as before. We have
$$
\sum_{j=1}^n h(\alpha, \zeta_j) h(\zeta_j, \delta)
= h(\alpha, \delta).
$$
Also
$$
\sum_{j=1}^n h(\alpha, \xi) h(\xi, \zeta_j) h(\zeta_j, \delta)
= h(\alpha, \xi) \sum_{j=1}^n  h(\xi, \zeta_j) h(\zeta_j, \delta)
= h(\alpha, \xi) h(\xi, \delta).
$$
Together, we get
$$
r_3(\alpha, \delta)
= h(\alpha, \delta) - h(\alpha, \xi) h(\xi, \delta)
= \frac{1}{n} r_1(\alpha, \delta).
$$

\subsection{Scalar curvature}
\label{sec:orgf9212d2}

We can contract any of the Ricci-forms we got to obtain the scalar curvature of the Hopf manifold. Picking the first two, we get
$$
s = n(n-1),
$$
while picking the third gives \(1/n\) times that, so Hopf manifolds have positive constant scalar curvature.


\section{Iwasawa manifold}
\label{sec:orgd67c2ff}

Let $G$ be the complex Lie group formed by upper-triangular complex matrices
$$
\begin{pmatrix}
  1 & x & z
  \\
  0 & 1 & y
  \\
  0 & 0 & 1
\end{pmatrix}
$$
and let $\Gamma \subset G$ be the discrete subgroup of matricies of this form whose entries are Gaussian integers. Then $X = G / \Gamma$ is a compact complex manifold of dimension $3$, called the \emph{Iwasawa manifold}.

Let $M := M(x,y,z)$ be a matrix as above. Then $M^{-1}dM$ is a matrix of $1$-forms that is invariant under the action of $G$, so it descends to define $1$-forms on $X$. Writing this out, we have
$$
M^{-1}dM
= \begin{pmatrix}
  1 & -x & -z
  \\
  0 & 1 & -y
  \\
  0 & 0 & 1
\end{pmatrix}
\begin{pmatrix}
  0 & dx & dz
  \\
  0 & 0 & dy
  \\
  0 & 0 & 0
\end{pmatrix}
=
\begin{pmatrix}
  0 & dx & dz - x dy
  \\
  0 & 0 & dy
  \\
  0 & 0 & 0
\end{pmatrix}.
$$
That is, the holomorphic $1$-forms $dx$, $dy$ and $dz - x dy$ on $G$ descend to $X$. Note that $d(dz - xdy) = -dx \wedge dy \not= 0$, so $X$ is not a K\"ahler manifold, as any holomorphic $1$-form on a compact K\"ahler manifold is closed.

These three $1$-forms are linearly independent at any point of $X$, so they trivialize the cotangent bundle $\Omega_X$. That is, they define a holomorphic bundle isomorphism $\Omega_X \to \kk C^3$. Pulling back the flat metric on the trivial bundle then gives a flat metric on $\Omega_X$, and taking duals gives a flat Hermitian metric on $X$. Let's work this out concretely.

Pulling back the flat metric amounts to defining our Hermitian metric on $\Omega_X$ so that $(dx, dy, dz - xdy)$ is an orthonormal frame. The matrix of the map from $(dx, dy, dz)$ coordinates to $(dx, dy, dz - xdy)$ is
$$
A = \begin{pmatrix}
1 & 0 & 0 \\
0 & 1 & 0 \\
0 & -x & 1
\end{pmatrix}
$$
so its inverse gives the coordinate change going the other way. The pullback of the standard metric in $(dx, dy, dz-x dy)$ coordiates to $(dx, dy, dz)$ ones is thus given by
$$
H = \overline{(A^{-1})^t} I_3 A^{-1}
=
\begin{pmatrix}
1 & 0 & 0 \\
0 & 1 & \bar x \\
0 & 0 & 1
\end{pmatrix}
\begin{pmatrix}
1 & 0 & 0 \\
0 & 1 & 0 \\
0 & x & 1
\end{pmatrix}
=
\begin{pmatrix}
1 & 0 & 0 \\
0 & 1+|x|^2 & \bar x \\
0 & x & 1
\end{pmatrix}.
$$
The Hermitian metric on $T_X$ is then given by
$$
H' := \overline{H}^{-1}
= \begin{pmatrix}
1 & 0 & 0 \\
0 & 1 & - x \\
0 & -\bar x & 1+|x|^2
\end{pmatrix}.
$$
We have
$$
(H')^{-1} = \begin{pmatrix}
  1 & 0 & 0 \\
  0 & 1 + |x|^2 & x \\
  0 & \bar x & 1
\end{pmatrix}
$$
so the Chern connection of this metric is
$$
(H')^{-1} \partial H'
= \begin{pmatrix}
  1 & 0 & 0 \\
  0 & 1 + |x|^2 & x \\
  0 & \bar x & 1
\end{pmatrix}
\begin{pmatrix}
0 & 0 & 0 \\
0 & 0 & -dx \\
0 & 0 & \bar x dx
\end{pmatrix}
= \begin{pmatrix}
0 & 0 & 0 \\
0 & 0 & -dx \\
0 & 0 & 0
\end{pmatrix}.
$$
This satisfies $\bar\partial((H')^{-1} \partial H') = 0$ as expected, which is a nice sanity check. A neat property of this metric is that its Chern connection is not trivial, but its curvature is still zero.

A fairly common question among those who learn complex geometry is to what extent properties of the curvature tensor determine the metric?\footnote{See Berger~\cite[Section~4.5]{berger} for a discussion of this question in the Riemannian case.} The Iwasawa manifold gives an example where a Hermitian metric has a flat curvature tensor, which of course coincides exactly with the curvature tensor of a K\"ahler metric, but where the manifold itself admits no K\"ahler metric. So the curvature tensor can not detect whether a metric is K\"ahler. Bo and Zheng~\cite{yang2016curvature} discuss some of these questions.



\section{Complexified K\"ahler cone}
\label{sec:complexified-kahler-cone}


Let $X$ be a compact K\"ahler manifold of dimension $\dim_{\kk C} X = n$.
A real $(1,1)$-class $\omega$ on $X$ is called a K\"ahler class if it contains a $(1,1)$-form of a K\"ahler metric on $X$.
The K\"ahler cone $\cc K$ of $X$ is the set of all K\"ahler classes on $X$. It is an open convex cone in the finite-dimensional real vector space $H^{1,1}(X,\kk R)$.

We can obtain a complexified version of the K\"ahler cone by considering the product
\[
H^{1,1}(X, \kk R) \oplus i \cdot \cc K \subset H^{1,1}(X,\kk C).
\]
That is, this is the set of $(1,1)$-classes whose imaginary part is a K\"ahler class. We call this the complexified K\"ahler cone of $X$, and denote it by $\cc C$. It is again an open convex cone in $H^{1,1}(X,\kk C)$. We're going to see that there is a natural K\"ahler metric on this cone and compute its curvature.

This does not seem to be a particularily important metric as such things go. Its Riemannian version on the K\"ahler cone has been studied without spectacular results~\cite{magnusson2020cohomological,trenner2011asymptotic,wilson2004sectional} but it has a special place in my heart, and is useful to us for other reasons. I like this metric because we got to know each other very well during my Ph.D. studies; I spent a long time working out how to compute its curvature tensor by using only operations in the cohomology ring (versus using harmonic forms or coordinizing the ring). This work was largely for nothing, as the expression I obtained does not tell us very much. However, this metric is useful to us as an example of a metric that does not enjoy the symmetries of the metrics on projective space, Grassmannians, flag manifolds or the unit ball, but whose curvature tensor we can nevertheless compute explicitly using our methods. It bears witness to my claim that those methods are a useful way to do complex differential geometry and do not just apply to the most symmetric examples.


\paragraph{}


Recall that if $\omega$ is a K\"ahler class on $X$, then it comes with a Lefschetz operator $L$ on cohomology, which has an adjoint $\Lambda$. We write $\omega\^k := \omega^k/k!$, $L\^k := L^k / k!$ and $\Lambda\^k := \Lambda^k / k!$. With this convention we have, for example, $\omega^j \cup \omega\^k = \binom{j+k}{j} \omega\^{j+k}$.

We'll lean heavily on Huybrecht's book \cite[Chapter~1.2]{huybrechts2005complex} in what follows for results on the Lefschets operator, its adjoint, and how they interact with primitive classes. We will use without mention that the complexified K\"ahler cone is an open set in a complex vector space, and as such admits an exterior derivative, and identify tangent fields on the cone with elements of that vector space that vary smoothly.


\begin{lemm}
Let $\alpha_1, \ldots, \alpha_k$ be $(1,1)$-classes on $X$. Then
\begin{align*}
\partial_{\alpha} \Lambda\^k(\alpha_1 \cup \cdots \cup \alpha_k)
&=
- \frac{1}{2i} \Lambda(\alpha) \Lambda\^k(\alpha_1 \cup \cdots \cup \alpha_k)
\\
&\qquad
+ \sum_{j=1}^k \Lambda\^k(\alpha_1 \cup \cdots \cup \partial_{\alpha}\alpha_j \cup \cdots \cup \alpha_k)
\\
&\qquad
+ \frac{1}{2i}\Lambda^{k+1}(\alpha \cup \alpha_1 \cup \cdots \cup \alpha_k),
\\
\bar\partial_{\ov\alpha} \Lambda\^k(\alpha_1 \cup \cdots \cup \alpha_k)
&=
\frac{1}{2i} \Lambda(\ov\alpha) \Lambda\^k(\alpha_1 \cup \cdots \cup \alpha_k)
\\
&\qquad
+ \sum_{j=1}^k \Lambda\^k(\alpha_1 \cup \cdots \cup \bar\partial_{\ov\alpha}\alpha_j \cup \cdots \cup \alpha_k)
\\
&\qquad
- \frac{1}{2i}\Lambda^{k+1}(\ov\alpha \cup \alpha_1 \cup \cdots \cup \alpha_k).
\end{align*}
\end{lemm}

\begin{proof}
Recall that
\[
\Lambda\^k(\alpha_1 \cup \cdots \cup \alpha_k) \omega\^n
= \alpha_1 \cup \cdots \cup \alpha_k \cup \omega\^{n-k}.
\]
As $\omega$ is the imaginary part of the Euler field, we have
\[
\partial_{\alpha} \omega\^k = \frac {1}{2i} \alpha \cup \omega\^{k-1}.
\]
Taking $\partial$ of both sides of the above equation gives
$$
\displaylines{
\partial_{\alpha} \Lambda\^k(\alpha_1 \cup \cdots \cup \alpha_k)
\omega\^n
+ \frac{1}{2i} \Lambda(\alpha) \Lambda\^k(\alpha_1 \cup \cdots \cup \alpha_k) \omega\^n
\hfill\cr\hfill{}
= \sum_{j=1}^k \alpha_1 \cup \cdots \cup \partial_{\alpha}\alpha_j \cup \cdots \cup \alpha_k \cup \omega\^{n-k}
+ \frac{1}{2i} \alpha_1 \cup \cdots \cup \alpha_k \cup \alpha \cup \omega\^{n-k-1},
}
$$
which is equivalent to the first equation after applying the adjoint Lefschetz operator, cancelling by $\omega\^n$ and rearranging terms. The second equality is proved in the same way.
\end{proof}

\begin{prop}
The Hermitian metric on $H^{1,1}(X,\kk C)$ defined by
\[
h(\alpha, \ov\beta)_{\xi}
= \Lambda \alpha \cdot \ov{\Lambda \beta} - \Lambda\^2 (\alpha \cup \ov\beta),
\]
where $\Lambda$ corresponds to the adjoint Lefschetz operator of $\im \xi$, is K\"ahler.
\end{prop}

\begin{proof}
The adjoint Lefschetz operator varies smoothly with the K\"ahler class, so the above expression defines a smooth Hermitian form. If $\alpha$ is a $(1,1)$-class, its primitive decomposition with respect to a K\"ahler class $\omega$ is
\[
\alpha
= \tfrac 1n \Lambda(\alpha) \omega
+ (\alpha - \tfrac 1n \Lambda(\alpha) \omega).
\]
Note also that if $\alpha$ is a primitive $(1,1)$-class, then $*\alpha = - \alpha \cup \omega\^{n-2}$.
The primitive decomposition is orthogonal with respect to the inner product a K\"ahler class defines. Applying this to $\alpha$ and $\beta$ we get
\begin{align*}
\langle \alpha, \ov \beta \rangle \omega\^n
&=
\langle \tfrac 1n \Lambda(\alpha) \omega,
\tfrac 1n \Lambda(\ov\beta) \omega \rangle
\omega\^n
+ \langle
\alpha - \tfrac 1n \Lambda(\alpha) \omega,
\ov{\beta - \tfrac 1n \Lambda(\beta) \omega}
\rangle \omega\^n
\\
&=
\frac{n}{n^2}
\Lambda(\alpha) \Lambda(\ov\beta)
 \omega\^n
- (\alpha - \tfrac 1n \Lambda(\alpha) \omega)
\cup (\ov\beta - \tfrac 1n \Lambda(\ov\beta)\omega) \omega\^{n-2}
\\
&=
\frac{1}{n}
\Lambda(\alpha) \Lambda(\ov\beta)
\omega\^n
- \alpha \cup \ov \beta \cup \omega\^{n-2}
\\
&\qquad{}
+ 2\frac{n-1}{n} \Lambda(\alpha) \Lambda(\ov\beta) \omega\^n
- \frac{2}{n^2} \binom{n}{2} \Lambda(\alpha) \Lambda(\ov\beta) \omega\^n
\\
&=
\Lambda(\alpha) \Lambda(\ov\beta) \omega\^n
- \Lambda\^2(\alpha \cup \ov\beta) \omega\^n
\end{align*}
because
\[
\frac 1n
+ 2 \frac{n-1}{n}
- \frac{2}{n^2} \binom{n}2
= 1.
\]
The Hermitian inner product defined by $\omega$ is thus equal to the expression we gave.

Consider now the smooth function
\[
\log \Vol(X, \im \alpha)
= \log \int_X \omega\^n.
\]
on $\cc C$. Then
\begin{align*}
\bar\partial \log\Vol
&= -\frac{1}{2i} \int_X d\bar\xi \cup \omega\^{n-1}
\Bigm/ \int_X \omega\^n
\\
&= -\frac{1}{2i} \int_X \Lambda(d\bar\xi) \omega\^n
\Bigm/ \int_X \omega\^n
= -\frac{1}{2i} \Lambda(d\bar\xi),
\end{align*}
where $\xi$ is the Euler field on $\cc C$, and
\[
\partial\bar\partial \log\Vol
= -\frac 14 \Lambda(d\xi) \wedge \Lambda(d\bar\xi)
+ \frac 14 \Lambda(d\xi \wedge d\bar\xi).
\]
The $(1,1)$-form defined by $\frac i2 \partial\bar\partial \log \Vol$ is thus a constant multiple of our Hermitian metric, so it is K\"ahler.
\end{proof}



\begin{exam}
If $h^{1,1} = 1$, then every $(1,1)$-class is a multiple of a single K\"ahler class $\omega_0$. Then $\alpha = a \omega_0$ and $\beta = b \omega_0$, and
\[
\langle \alpha, \ov\beta \rangle_{\tau}
= a \ov b |\omega_0|_{\tau}^2
= \frac{n}{|\im \tau|^2}\, a \ov b.
\]
This is the Poincar\'e metric on the disk.
\end{exam}



\begin{prop}
The Chern connection of the metric is
\[
D_{\gamma} \alpha
= d_{\gamma}\alpha
- \frac{1}{2i} \Lambda(\gamma)\alpha
- \frac{1}{2i} \Lambda(\alpha)\gamma
+ \frac{1}{2i} \Lambda(\alpha\cup\gamma).
\]
\end{prop}

\begin{proof}
Let $\alpha$, $\beta$ and $\gamma$ be holomorphic tangent fields. Then
$$
\displaylines{
\partial_\gamma h(\alpha, \ov\beta)
= -\frac{1}{2i} \Lambda(\gamma) \Lambda(\alpha) \Lambda(\ov\beta)
+ \Lambda(\partial_\gamma\alpha) \Lambda(\ov\beta)
+ \frac{1}{2i}\Lambda\^2(\alpha\gamma)\Lambda(\ov\beta)
\hfill\cr{}
-\frac{1}{2i} \Lambda(\gamma) \Lambda(\alpha) \Lambda(\ov\beta)
+\Lambda(\alpha)\Lambda\^2(\gamma\ov\beta)
\cr\hfill{}
+ \frac{1}{2i} \Lambda(\gamma) \Lambda\^2(\alpha \ov\beta)
- \Lambda\^2(\partial_\gamma\alpha \ov\beta)
- \frac{1}{2i} \Lambda\^3(\alpha  \gamma  \ov\beta)
\cr{}
\phantom{\partial_\gamma h(\alpha, \ov\beta)}
= h(\partial_\gamma\alpha,\ov\beta)
- \frac{1}{2i} h(\Lambda(\gamma)\alpha, \ov\beta)
- \frac{1}{2i} h(\Lambda(\alpha)\gamma, \ov\beta)
\hfill\cr\hfill{}
+ \frac{1}{2i}\Lambda\^2(\alpha\gamma)\Lambda(\ov\beta)
- \frac{1}{2i} \Lambda\^3(\alpha \gamma \ov\beta).
}
$$
To make sense of the last two terms here,
we write $\beta = \frac 1n \Lambda(\beta) \omega + (\beta - \frac 1n \Lambda(\beta) \omega)$ for the primitive decomposition of $\beta$. Then
\begin{align*}
* (\beta \cup \omega)
&= *\Bigl(\frac 2n \Lambda(\beta) \omega\^2\Bigr)
+ *\Bigl(\Bigl(\beta - \frac 1n \Lambda(\beta) \omega\Bigr)\cup \omega\Bigr)
\\
&=
\frac 2n \Lambda(\beta) \omega\^{n-2}
- \Bigl(\beta - \frac 1n \Lambda(\beta) \omega\Bigr) \cup \omega\^{n-3}
\\
&= \Lambda(\beta) \omega\^{n-2}
- \beta \cup \omega\^{n-3}.
\end{align*}
It follows that
\begin{align*}
\langle \Lambda(\alpha \cup \gamma), \ov\beta \rangle \omega\^n
&= \langle \alpha \cup \gamma, \ov{\beta \cup \omega} \rangle \omega\^n
\\
&= \alpha \cup \gamma \cup
( \Lambda(\ov\beta) \omega\^{n-2}
- \ov\beta \cup \omega\^{n-3})
\\
&=
\bigl(\Lambda\^2(\alpha\cup\gamma)\Lambda(\ov\beta)
- \Lambda\^3(\alpha\cup\gamma\cup\ov\beta) \bigr) \omega\^n.
\end{align*}
\end{proof}


\begin{prop}
The curvature tensor of the metric is
\[
R(\alpha,\ov\beta,\gamma,\ov\delta)
=
\tfrac14 \langle \alpha, \ov\beta \rangle \langle \gamma, \ov\delta \rangle
+ \tfrac14 \langle \gamma, \ov\beta \rangle \langle \alpha, \ov\delta \rangle
- \tfrac14 \langle \alpha \cup \gamma, \ov{\beta \cup \delta} \rangle.
\]
\end{prop}

\begin{proof}
Let $\alpha,\beta,\gamma,\delta$ be holomorphic vector fields.
We want to calculate
\begin{align*}
R(\alpha,\ov\beta,\gamma,\ov\delta)
= h(\Theta_{\alpha\ov\beta}\gamma, \ov\delta)
&= h(\bar\partial_{\ov\beta} D_\alpha \gamma, \ov\delta)
\\
&= h\Bigl(\bar\partial_{\ov\beta} \Bigl(
d_{\alpha}\gamma
- \frac{1}{2i} \Lambda(\alpha)\gamma
- \frac{1}{2i} \Lambda(\gamma)\alpha
+ \frac{1}{2i} \Lambda(\alpha\cup\gamma)
\Bigr)
, \ov\delta\Bigr).
\end{align*}
To begin with, we have
\[
\bar\partial_{\ov\beta} d_\alpha \gamma = 0
\]
because all the fields are holomorphic, and
\[
-\frac{1}{2i} \bar\partial_{\ov\beta} \Lambda(\alpha) \gamma
= \frac14 \Lambda(\alpha)\Lambda(\ov\beta) \gamma
- \frac14 \Lambda\^2(\alpha \cup \ov\beta) \gamma
= \frac14 \langle \alpha, \ov\beta \rangle \gamma
\]
again for the same reason. The problematic term is
\[
\bar\partial_{\ov\beta}(\Lambda(\alpha\cup\gamma))
= (\bar\partial_{\ov\beta}\Lambda) (\alpha\cup\gamma).
\]
Recall that
\[
\langle \Lambda(\alpha \cup \gamma), \ov\delta \rangle
= \Lambda\^2(\alpha\cup\gamma) \Lambda(\ov\delta)
- \Lambda\^3(\alpha\cup\gamma\cup\ov\delta).
\]
Taking the $\bar\partial$ of this in the direction of $\ov\beta$ we get
$$
\displaylines{
\langle (\bar\partial_{\ov\beta} \Lambda)(\alpha \gamma), \ov\delta \rangle
+ \langle \Lambda(\alpha \gamma), \ov{D_\beta\delta} \rangle
\hfill\cr{}
\qquad{}
= \frac{1}{2i} \Lambda(\ov\beta) \Lambda\^2(\alpha\gamma) \Lambda(\ov\delta)
- \frac{1}{2i} \Lambda\^3(\alpha\gamma\ov\beta) \Lambda(\ov\delta)
\hfill\cr\qquad\qquad{}
+ \frac{1}{2i} \Lambda(\ov\beta) \Lambda\^2(\alpha\gamma) \Lambda(\ov\delta)
+ \Lambda\^2(\alpha\gamma) \Lambda(\ov{\partial_\beta\delta})
- \frac{1}{2i} \Lambda\^2(\alpha\gamma) \Lambda\^2(\ov{\beta\delta})
\hfill\cr\qquad\qquad{}
- \frac{1}{2i} \Lambda(\ov\beta) \Lambda\^3(\alpha\gamma\ov\delta)
- \Lambda\^3(\alpha\gamma\ov{\partial_\beta\delta})
+ \frac{1}{2i} \Lambda\^4(\alpha\gamma\ov\beta\ov\delta)
\hfill\cr{}
\qquad{}
= \frac{1}{2i} \langle \Lambda(\alpha\gamma), \ov{\Lambda(\delta) \beta} \rangle
+ \frac{1}{2i} \langle \Lambda(\alpha\gamma), \ov{\Lambda(\beta) \delta} \rangle
\hfill\cr\qquad\qquad{}
+ \langle \Lambda(\alpha\gamma), \ov{\partial_\beta\delta} \rangle
- \frac{1}{2i} \Lambda\^2(\alpha\gamma) \Lambda\^2(\ov{\beta\delta})
+ \frac{1}{2i} \Lambda\^4(\alpha\gamma\ov\beta\ov\delta)
\hfill
}
$$
On the left-hand side, we have the term
$$
\displaylines{
\langle \Lambda(\alpha \gamma), \ov{D_\beta\delta} \rangle
= \langle \Lambda(\alpha \gamma), \ov{
\partial_\beta\delta
} \rangle
+ \frac{1}{2i} \langle \Lambda(\alpha \gamma), \ov{\Lambda(\beta)\delta} \rangle
\hfill\cr\hfill{}
+ \frac{1}{2i} \langle \Lambda(\alpha \gamma), \ov{\Lambda(\delta)\beta} \rangle
- \frac{1}{2i} \langle \Lambda(\alpha \gamma), \ov{\Lambda(\beta\delta)} \rangle.
}
$$
Comparing the two, we are left with
\[
\langle (\bar\partial_{\ov\beta} \Lambda)(\alpha \gamma), \ov\delta \rangle
= \frac{1}{2i} \langle \Lambda(\alpha \gamma), \ov{\Lambda(\beta\delta)} \rangle
- \frac{1}{2i} \Lambda\^2(\alpha\gamma) \Lambda\^2(\ov{\beta\delta})
+ \frac{1}{2i} \Lambda\^4(\alpha\gamma\ov\beta\ov\delta).
\]
Putting everything together, we have found that
$$
\displaylines{
R(\alpha,\ov\beta,\gamma,\ov\delta)
= \frac14 \langle \alpha, \ov\beta \rangle \langle \gamma, \ov\delta \rangle
+ \frac14 \langle \gamma, \ov\beta \rangle \langle \alpha, \ov\delta \rangle
\hfill\cr\hfill{}
- \frac14 \langle \Lambda(\alpha \gamma), \ov{\Lambda(\beta\delta)} \rangle
+ \frac{1}{4} \Lambda\^2(\alpha\gamma) \Lambda\^2(\ov{\beta\delta})
- \frac{1}{4} \Lambda\^4(\alpha\gamma\ov\beta\ov\delta)
\cr
\phantom{R(\alpha,\ov\beta,\gamma,\ov\delta)}
=
\frac14 \langle \alpha, \ov\beta \rangle \langle \gamma, \ov\delta \rangle
+ \frac14 \langle \gamma, \ov\beta \rangle \langle \alpha, \ov\delta \rangle
- \frac14 \langle \alpha\gamma, \ov{\beta\delta} \rangle
\hfill
}
$$
where the last equality holds because
\[
\Lambda\^4(\alpha\gamma\ov\beta\ov\delta)
= \langle \alpha\gamma, \ov{\beta\delta} \rangle
- \langle \Lambda(\alpha\gamma), \ov{\Lambda(\beta\delta)} \rangle
+ \langle \Lambda\^2(\alpha\gamma), \ov{\Lambda\^2(\beta\delta)} \rangle
\]
by Magn\'usson~\cite{magnusson2016inner}.
\end{proof}


The curvature tensor of the metric looks like what we'd get if the complexified K\"ahler cone were embedded in a space with constant holomorphic sectional curvature $1/2$ and the second fundamental form of the embedding were $b(\alpha,\gamma) = \alpha \cup \gamma$. If such a space existed, we'd have a short exact sequence
\[
\begin{tikzcd}
0 \ar[r] &
H^{1,1}(X,\kk C) \ar[r] &
? \ar[r] &
H^{2,2}(X,\kk C) \ar[r] &
0
\end{tikzcd}
\]
of vector bundles over $\cc C$. (To be precise, the quotient would only need to be a subspace of the cohomology group, so it could maybe be a Grassmannian. It would need to include the image of $H^{1,1}$ under the cup product.) Whatever space we're looking at should depend on the K\"ahler class. The obvious candidate is the space of primitive classes, but that doesn't vary holomorphically inside the cohomology group. The problem might be that we shouldn't be looking for holomorphic objects at all, but symplectic ones. If we believe in mirror symmetry~\cite{hori2003mirror}, then deformations of the complex structure of a manifold (which are holomorphic) should correspond to deformations of the K\"ahler structures (which are not) on a different manifold. If this is the case, then us looking at the K\"ahler or Riemannian structure on the K\"ahler cone is a tangent that leads nowhere. We should instead perhaps focus on the K\"ahler metric here giving us a symplectic form and study that.

In any case, this is the first explicit curvature tensor we meet that we can't say much about. It's a mix of positive and negative terms, and does not have a definite sign in general. It seems impossible to calculate its Ricci tensor, as to do so we'd need to know the structure of the cohomology ring on an arbitrary compact K\"ahler manifold.







\section{TODO: Projectivized bundles}



\subsection{Curvature of tautological bundle}

Let $(E, h) \to X$ be a holomorphic Hermitian vector bundle of rank $r + 1$ over a complex manifold $X$. The group $\kk C^*$ acts fiberwise on $E$, and we can quotient by its action to form a projective bundle $\kk P(E) \to X$. In general there are projective bundles that do not arise as the projectivization of vector bundles, but we are not going to look at those here.

As before there is a tautological line bundle $\cc O(-1) \to \kk P(E)$, and the metric $h$ defines a Hermitian metric on it. We want to compute the curvature form of this metric, and later see if we can turn that form into an honest K\"ahler metric.

It is enough to perform these computations locally. In this case I think we are even forced to. If we try to play the same trick as in \thref{sec:projective-space-redux} we end up having to compute the value of the pullback of a connection on the Euler section that trivializes $\cc O(-1)$ over the total space of $E$ minus the zero section. But the pullback of the connection is defined on pullbacks of sections and extended by linearity, and the Euler section is not the pullback of a section downstairs, so we have to look at the local picture.

In any case, we pick a point of interest $(x_0, [v_0])$ of $\kk P(E)$. As $E \to X$ is a vector bundle, there is a neighborhood $U \subset X$ of $x$ such that $E|_U \cong U \times V$, where $V$ is the fiber of $E$. It follows that the projective bundle also splits holomorphically over that neighborhood; $\kk P(E)|_U \cong U \times \kk P(V)$. We pick $\lambda \in V^*$ that does not vanish on any representative of $[v]$, and consider the neighborhood $U \times H_\lambda$ of $(x, [v])$. There we have the holomorphic Euler section $\xi(x, v) = v$ that trivializes the tautological bundle. We want to compute the (negative of the) curvature form
$$
\psi = -\frac i2 \partial \bar\partial \log h(\xi, \xi).
$$
The difference with the case of projective space is that there we had a flat Hermitian metric $h$ (one that did not depend on the base variable $x$), while here the metric has curvature. We also have to take derivatives in the base and fiber directions, instead of only in the fiber directions. (This is thankfully easy as the bundle splits holomorphically locally.)

So let $(\alpha, s)$ and $(\beta, t)$ be local holomorphic sections of $T_{U \times H_\lambda}$. We have
$$
\bar\partial_{(\beta, t)} \log h(\xi,\xi)
= \frac{h(\xi, \nabla_\beta \xi + t)}{h(\xi,\xi)},
$$
where $\nabla$ is the Chern connection of $h$. Then
\begin{align*}
\psi((\alpha,s), (\beta,t))
&= -\frac{h(\nabla_\alpha \xi + s, \nabla_\beta \xi + t)}{h(\xi,\xi)}
- \frac{h(\xi, \bar\partial_\alpha \nabla_\beta \xi)}{h(\xi,\xi)}
+ \frac{h(\nabla_\alpha \xi + s, \xi)}{h(\xi,\xi)}
\frac{h(\xi, \nabla_\beta \xi + t)}{h(\xi,\xi)}
\\
&= -\omega(\nabla_\alpha \xi + s, \nabla_\beta \xi + t)
- \frac{h(\xi, F^\nabla_{\alpha\beta}\xi)}{h(\xi,\xi)},
\end{align*}
where $\omega$ is the Fubini--Study metric on $\kk P(V)$ and we recall that on holomorphic sections we have $\bar\partial \nabla \xi = F^\nabla \xi$ and $F^\nabla$ is anti-Hermitian.

This curvature form does usually not have a definite sign. The Fubini--Study part can equal zero without the tangent fields in question being zero, and in general pretty much anything can appear in the curvature term of $h$. One exception is when $(E,h)$ is Griffiths negative. In that case, the curvature form $\psi$ is positive-definite, which implies that $E^*$ is ample.\footnote{We should be looking at $\kk P(E^*)$ like algebraic geometers. If we did, Griffiths positive would imply $E$ ample (that is, $\cc O(1) \to \kk P(E^*)$ would be ample).}




SKETCH:

$$
\bar\partial \log h(\xi,\xi)
= \frac{h(\xi, \nabla \xi)}{h(\xi,\xi)}
+ \frac{h(\xi, \id_H)}{h(\xi,\xi)}
$$
and
$$
\displaylines{
\partial\bar\partial \log h(\xi,\xi)
% Term 1, base derivatives
= \frac{h(\nabla \xi, \nabla \xi)}{h(\xi,\xi)}
+ \frac{h(\xi, \bar\partial_U \nabla \xi)}{h(\xi,\xi)}
- \frac{h(\nabla \xi, \xi)}{h(\xi,\xi)}
\wedge \frac{h(\xi, \nabla \xi)}{h(\xi,\xi)}
\hfill\cr\hfill{}
% Term 1, fiber derivatives
+ \frac{h(\id_H, \nabla \xi)}{h(\xi,\xi)}
- \frac{h(\id_H, \xi)}{h(\xi,\xi)}
\wedge \frac{h(\xi, \nabla \xi)}{h(\xi,\xi)}
\cr\hfill{}
% Term 2, base derivatives
+ \frac{h(\nabla \xi, \id_H)}{h(\xi,\xi)}
- \frac{h(\nabla \xi, \xi)}{h(\xi, \xi)}
\wedge \frac{h(\xi, \id_H)}{h(\xi,\xi)}
\cr\hfill{}
% Term 2, fiber derivatives
+ \frac{h(\id_H, \id_H)}{h(\xi,\xi)}
- \frac{h(\id_H, \xi)}{h(\xi,\xi)}
\wedge \frac{h(\xi, \id_H)}{h(\xi,\xi)}
}
$$
That is as a $(1,1)$-form. The associated sesquilinear form evaluated on $s \oplus \alpha$ and $t \oplus \beta$ is
$$
\psi(s \oplus \alpha, t \oplus \beta)
= -h_{\mathrm{FS}}(\nabla_\alpha \xi + s, \nabla_\beta \xi + t)
+ \frac{h(\frac i2\Theta_{\alpha\beta}\xi, \xi)}{h(\xi,\xi)}.
$$
I'd like to be able to say that the Fubini--Study metric doesn't depend on the Hermitian inner product used to define it, so even though $h$ is used in the definition of $h_{\mathrm{FS}}$, that inner product is equal to the pullback from the projective space factor. That is, it doesn't depend on the base variables. (So the only dependency on those comes from the vector fields we put in here.)


In any case, the curvature form of the tautological bundle has a definite sign when restricted to the fiber directions. This points the way to showing that the projective bundle is a K\"ahler manifold, at least when the base is a compact K\"ahler manifold: We pull back a K\"ahler metric from the base and multiply it with a big enough constant to offset the potential negativity of the curvature form in non-fiber directions.


\subsection{Splitting the tangent bundle}

We have a short exact sequence of vector bundles
$$
0
\longrightarrow T_{\kk P(E) / X}
\stackrel{j}{\longrightarrow} T_{\kk P(E)}
\stackrel{\pi_*}{\longrightarrow} \pi^* T_X
\longrightarrow 0.
$$
We also have a real $(1,1)$-form $\psi$ on $\kk P(E)$ that is positive-definite on the subbundle $T_{\kk P(E)/X}$. If $\psi$ were positive-definite on the whole tangent bundle, we could use it to split the exact sequence smoothly. In fact, knowing only that it is positive-definite on the subbundle is enough to do this.

One way to define a splitting of the exact sequence is to construct a linear morphism $j^* : T_{\kk P(E)} \to T_{\kk P(E)/X}$ such that $j^* j = \id_{T_{\kk P(E)/X}}$; the splitting is then given by
$$
T_{\kk P(E)} \to T_{\kk P(E)/X} \oplus \pi^*T_X,
\quad
\Xi \mapsto j^*(\Xi) \oplus \pi_*(\Xi).
$$
We can view the $(1,1)$-form $\psi$ as a morphism $\psi : T_{\kk P(E)} \to \overline{T}_{\kk P(E)}^*$. By hypothesis, it is an isomorphism when restricted to $T_{\kk P(E)/X}$. We then set
$$
j^* := \psi^{-1} \circ \bar{j}^* \circ \psi.
$$
It helps to look at what this gives on given fields. Let $A$ and $B$ be tangent fields on $\kk P(E)$. Then
$$
\psi(A) = (B \mapsto \psi(A, B)),
$$
and
$$
\bar j^* \circ \psi(A) = (b \mapsto \psi(A, b)),
$$
where $b \in T_{\kk P(E)/X}$. This defines an element of $\overline T_{\kk P(E)/X}^*$, where $\psi$ is an isomorphism, so $\psi^{-1} \circ \bar j^* \circ \psi(A)$ is the unique element $a$ of $T_{\kk P(E)/X}$ such that
$$
\psi(a, b) = \psi(A, b)
$$
for all $b$ in $T_{\kk P(E)/X}$. Tracing back through this shows that $j^* \circ j = \id_{T_{\kk P(E)/X}}$.

To work out what this gives for $s \oplus \alpha$ for our curvature form, we have
$$
t \mapsto \psi(s \oplus \alpha, t) = h_{\mathrm{FS}}(D^h_\alpha \xi + s, t)
$$
so $j^*(s \oplus \alpha) = D^h_\alpha \xi + s$. In our local coordinates, the splitting is the smooth isomorphism
$$
\Phi: T_H \oplus T_U \to T_H \oplus T_U,
\quad
s \oplus \alpha
\mapsto
\begin{pmatrix}
  \id & D^h_\bullet \xi
  \\
  0 & \id
\end{pmatrix}
\begin{pmatrix}
  s \\ \alpha
\end{pmatrix}.
$$
The curvature form of the tautological bundle is the pullback under $\Phi$ of the direct sum of the Fubini--Study metric on $T_H$ and the sesquilinear form
$$
\theta(\alpha, \beta) = -\frac{h(\frac i2\Theta_{\alpha\beta}\xi, \xi)}{h(\xi,\xi)}
$$
on $\pi^*T_X$. For what it's worth, that form extends to all of $T_{\kk P(E)}$ by $\theta(A, B) = \theta(\pi_*A, \pi_*B)$.

Note that we can add the pullback of any form we want from the base without changing the smooth splitting of the tangent bundle.

\subsection{Attempts at the Chern connection}

If $D$ is the Chern connection of the Fubini--Study metric, this would give us
$$
\displaylines{
\partial_{\gamma \oplus z} \psi(\alpha \oplus s, \beta \oplus t)
% First term, base directions
% Wait, the FS connection should only differentiate in fiber directions
= h_{\mathrm{FS}}(\partial_\gamma(\nabla_\alpha \xi + s), \nabla_\beta \xi + t)
+ h_{\mathrm{FS}}(\nabla_\alpha \xi + s, \bar\partial_\gamma(\nabla_\beta \xi + t))
+ h_{\mathrm{FS}}(\nabla_\alpha \xi + s, \tfrac{i}{2} \Theta_{\gamma\beta} \xi)
% First term, fiber directions
\hfill\cr\hfill{}
+ h_{\mathrm{FS}}(\nabla_\alpha z + \partial s, \nabla_\beta \xi + t)
\cr\hfill{}
% Second term, base directions
+ \frac{h(\nabla_\gamma\frac i2\Theta_{\alpha\beta}\xi, \xi)}{h(\xi,\xi)}
- \frac{h(\nabla_\gamma\xi,\xi)}{h(\xi,\xi)}
\frac{h(\frac i2\Theta_{\alpha\beta}\xi, \xi)}{h(\xi,\xi)}
\cr\hfill{}
% Second term, fiber directions
% We can flip the curvature endomorphism here twice and get that his is equal to
% h_{FS}(z, \tfrac i2 \Theta_{\alpha\beta} \xi)
+ \frac{h(\frac i2\Theta_{\alpha\beta}z, \xi)}{h(\xi,\xi)}
- \frac{h(z, \xi)}{h(\xi,\xi)}
\frac{h(\frac i2\Theta_{\alpha\beta}\xi, \xi)}{h(\xi,\xi)}
}
$$


\subsection{A metric}

I'm not sure we need to go into the tangent bundle splitting. The curvature form is defined on all of the projective bundle, and we're adding the pullback of a metric on the base, which is also defined on the whole bundle. We have a smooth closed $(1,1)$-form
\[
\Phi((\alpha,s), \ov{(\beta,t)})
= \pi_1^* h_{\kk P(V)}(D_{E,\alpha}\xi + s, \ov{D_{E,\beta}\xi + t})
- \frac{R_E(\alpha,\ov\beta,\xi,\ov\xi)}{h_E(\xi,\ov\xi)}
+ e^\lambda \pi_2^* h_X(\alpha,\ov \beta),
\]
where $h_X$ is a K\"ahler metric on the base, $V$ is the fiber of $E$ at the center point, and $\lambda \in \kk R$ is a constant large enough that $\Phi$ is positive-definite on all of the projective bundle.

I've chosen my notation very poorly, as $s,t,\xi$ are sections of $E$ while $\alpha,\beta$ are sections of $T_X$. Oh well.

What normally makes ``the sum of connections isn't a connection'' true is that $(D_1 + D_2)(f \sigma) = 2df \otimes \sigma + f(D_1 + D_2)\sigma$. I wonder if we get around this somehow here because we kind of have two pullback connections, so together they'll only give one $df$?



\subsection{Jumping straight to Grassmannian bundles}

Let $p: (E,h) \to X$ be a holomorphic Hermitian vector bundle and let $\pi: \Gr(k,E) \to X$ be the Grassmannian bundle of $k$-planes in $E$. This is a submersion and we have a short exact sequence
\[
\begin{tikzcd}
0 \ar[r] &
T_{\Gr(k,E)/X} \ar[r] &
T_{\Gr(k,E)} \ar[r,"\pi_*"] &
\pi^* T_X \ar[r]
& 0
\end{tikzcd}
\]
that I don't think we're going to use.

We can mimic the proof that the metric on the Grassmannian is K\"ahler to construct a smooth closed $(1,1)$-form $\theta$ on $\Gr(k,E)$. We trivialize $E$ locally around $x_0$ and split $E_{x_0} = S_0 \oplus W_0$, where $W_0 = S_0^\perp$ with respect to the metric $h_E(x_0)$. We denote the Euler field on $\Hom(S_0,W_0)$ by $\upsilon$ and set
\begin{align*}
\theta
&=- \frac i2 \partial\bar\partial \log \det(\id_{S_0} + \upsilon^\dagger \upsilon)
\\
&= -\frac i2 \partial \frac{\tr((D'_{\Hom(S_0,W_0)}\upsilon)^\dagger \upsilon)}{\det(\id_{S_0} + \upsilon^\dagger \upsilon)}
\\
&=
-\frac i2
\frac{
\tr\bigl((
\bar\partial D'_{\Hom(S_0,W_0)}\upsilon)^\dagger \upsilon)
\bigr)
}{\det(\id_{S_0} + \upsilon^\dagger \upsilon)}
+
\frac i2
\frac{\tr\bigl(
(D'_{\Hom(S_0,W_0)}\upsilon)^\dagger \wedge (D'_{\Hom(S_0,W_0)}\upsilon)
\bigr)
}{\det(\id_{S_0} + \upsilon^\dagger \upsilon)}
\\
&\qquad
+ \frac i2
\frac{\tr\bigl(
\upsilon^\dagger (D'_{\Hom(S_0,W_0)}\upsilon)
\bigr)
}{\det(\id_{S_0} + \upsilon^\dagger \upsilon)}
\wedge
\frac{\tr\bigl(
(D'_{\Hom(S_0,W_0)}\upsilon)^\dagger \upsilon
\bigr)
}{\det(\id_{S_0} + \upsilon^\dagger \upsilon)}.
\end{align*}
The Hermitian form this defines on $T_{\Gr(k,E)}$ is then
\[
\theta(\alpha, \ov\beta)
= h_{\Hom(S,Q)}\bigl(
D_{\Hom(S_0,W_0), \alpha} \upsilon,
\ov{D_{\Hom(S_0,W_0), \beta} \upsilon}
\bigr)
- \frac{\tr\bigl(
(\Theta_{\Hom(S_0,W_0),\beta\ov\alpha}\upsilon)^\dagger \upsilon
\bigr)}{\det(\id_{S_0} + \upsilon^\dagger \upsilon)}
\]
on the open set $\Hom(S_0,W_0) \times U$, where $U \subset X$ is the open set that trivializes $E$.

In the case of the Grassmannian -- that is, when $X$ is a point -- the metric on $\Hom(S_0,W_0)$ is flat and $D_{\Hom(S_0,W_0),\alpha}\upsilon = \alpha$, so this reduces to the usual expression for the K\"ahler metric on the Grassmannian.

We want to find the curvature tensor of $\theta + e^\lambda \pi^* h_X$, where $h_X$ is a K\"ahler metric on $X$. This is tricky for at least two reasons:
\begin{itemize}
\item $\theta$ is degenerate, so it doesn't have a Chern connection.
\item Even if it did, the Chern connection is non-linear as a function of the metric. The Chern connection of a sum of metrics is not the sum of Chern connections (the latter isn't a connection, for one).
\end{itemize}
We do have
\[
\frac i2 \partial \bar \partial h(\alpha,\ov\beta)
= h(D\alpha, \ov{D\beta}) - h(\Theta \alpha, \ov\beta)
\]
for any Hermitian metric, and we have
\[
h(D\alpha, \ov{D\beta})
= \tr_h \bigl(
h(D\alpha, \ov{\bullet})
\wedge
h(\bullet, \ov{D\beta})
\bigr)
\]
where the right-hand side is a $(1,1)$-form, and
\[
\partial h(\alpha, \ov\bullet)
= h(D\alpha, \ov\bullet)
\]
and similar for $\bar\partial h$. I wonder if we can poke enough at this to compute $R$ without having to compute the Chern connection. This gives
\[
h(\Theta \alpha, \ov\beta)
= \frac i2 \tr_h \big(
\partial h(\alpha, \ov\bullet)
\wedge
\bar\partial h(\bullet, \ov\beta)
\bigr)
- \frac i2 \partial\bar\partial h(\alpha,\ov\beta)
\]
which seems like the invariant version of
\[
\Theta
= - \frac i2 H^{-1} \partial \bar \partial H
+ \frac i2 H^{-1}\partial H \wedge H^{-1} \bar\partial  H.
\]
The trick to using that formula is usually to pick a frame where we have $\partial H = 0$ at the point of interest. That should translate into a frame such that $D = 0$ at the point of interest for us. But then we wouldn't need the dance with the trace, we could conclude directly from the original $\partial\bar\partial$ equation.

We might be lucky because the metric we pull back is a metric on the base $X$. We assume it's K\"ahler, so there exist normal coordinates $(z_1, \ldots, z_n)$ centered at $x_0$. Using these, we fabricate a normal coordinate frame for $\Hom(S,Q)$ around a point $g_0$ that maps to $x_0$; this is a frame $(e_1, \ldots, e_{k(n-k)})$ so that $D_{\Hom} = 0$ at $g_0$. The trick is that both $D_{\Hom} = 0$ and $\pi^*D_X = 0$ at $g_0$ with this setup. At $g_0$, we should then have
\[
\Theta =
-\frac i2 \partial \bar \partial \theta
-\frac i2 \partial \bar \partial \pi^*h_X
= -\frac i2 \partial \bar \partial \theta
+ \pi^*\Theta_X
\]
and we can simplify the evaluation of $\partial \bar \partial \theta$ at $g_0$ because of the vanishing of $D_{\Hom}$ there. Because everything in sight is a tensor, we can conclude what $\Theta$ looks like globally.


Recall that
\[
\theta(\alpha, \ov\beta)
= h\bigl(
D_{\alpha} \upsilon,
\ov{D_{\beta} \upsilon}
\bigr)
- \frac{\tr\bigl(
(\Theta_{\beta\ov\alpha}\upsilon)^\dagger \upsilon
\bigr)}{\det(\id_{S_0} + \upsilon^\dagger \upsilon)}
\]
and that
\(
h(\alpha,\ov\beta) = \tr(\beta^\dagger \alpha)
\)
at $0$. We have
$$
\displaylines{
\bar\partial_{\ov\delta}\theta(\alpha,\ov\beta)
= h(\Theta_{\alpha\ov\delta} \upsilon, \ov{D_\beta \upsilon})
+ h(D_\alpha\upsilon, \ov{D_\delta D_\beta \upsilon})
\hfill\cr\hfill{}
- \frac{\tr\bigl(
(D_\delta\Theta_{\beta\ov\alpha}\upsilon)^\dagger \upsilon
\bigr)}{\det(\id_{S_0} + \upsilon^\dagger \upsilon)}
+
\frac{\tr\bigl(
(D_\delta \upsilon)^\dagger \upsilon
\bigr)}{\det(\id_{S_0} + \upsilon^\dagger \upsilon)}
\frac{\tr\bigl(
(\Theta_{\beta\ov\alpha}\upsilon)^\dagger \upsilon
\bigr)}
{\det(\id_{S_0} + \upsilon^\dagger \upsilon)}
}
$$
so, omitting terms that are inner products against $D\tau$ for some section $\tau$ as those are zero at $0$, we have
$$
\displaylines{
\partial_{\gamma}\bar\partial_{\ov\delta}\theta(\alpha,\ov\beta)
=
h(\Theta_{\alpha\ov\delta} \upsilon, \ov{\Theta_{\beta\ov\gamma} \upsilon})
+ h(D_\gamma D_\alpha \upsilon, \ov{D_\delta D_\beta} \upsilon)
\hfill\cr\hfill{}
- \frac{\tr\bigl(
(\bar\partial_{\ov\gamma}D_\delta\Theta_{\beta\ov\alpha}\upsilon)^\dagger \upsilon
\bigr)}{\det(\id_{S_0} + \upsilon^\dagger \upsilon)}
+
\frac{\tr\bigl(
(\Theta_{\delta\ov\gamma})^\dagger \upsilon
\bigr)}{\det(\id_{S_0} + \upsilon^\dagger \upsilon)}
\frac{\tr\bigl(
(\Theta_{\beta\ov\alpha}\upsilon)^\dagger \upsilon
\bigr)}
{\det(\id_{S_0} + \upsilon^\dagger \upsilon)}.
}
$$
Taking adjoints ($\upsilon$ is holomorphic, so $\Theta_{\gamma\ov\delta} \upsilon = \bar\partial_{\ov\delta}D_\gamma \upsilon$, which is what we have on the adjoint side above. We conclude that
\[
\partial_{\gamma}\bar\partial_{\ov\delta}\theta(\alpha,\ov\beta)
= h(\Theta_{\alpha\ov\delta} \upsilon,
\ov{\Theta_{\beta\ov\gamma} \upsilon})
+ h(D_\gamma D_\alpha \upsilon,
\ov{D_\delta D_\beta} \upsilon)
+ h(\Theta_{\gamma\ov\delta} \upsilon,
\ov{\Theta_{\beta\ov\alpha} \upsilon})
\]
at $0$, where I think the sign flip happens when taking adjoints.

The upshot is that the curvature tensor of the metric on $\Gr(k,E)$ is
$$
\displaylines{
R(\alpha,\ov\beta,\gamma,\ov\delta)
= h(\Theta_{\alpha\ov\delta} \upsilon,
\ov{\Theta_{\beta\ov\gamma} \upsilon})
+ h(D_\gamma D_\alpha \upsilon,
\ov{D_\delta D_\beta} \upsilon)
\hfill\cr\hfill{}
+ h(\Theta_{\gamma\ov\delta} \upsilon,
\ov{\Theta_{\beta\ov\alpha} \upsilon})
+ e^\lambda \pi^*R_X(\alpha,\ov\beta,\gamma,\ov\delta)
}
$$
at the point of interest. The sign of the holomorphic sectional curvature of this tensor is determined by
$$
R(\alpha,\ov\alpha,\alpha,\ov\alpha)
= 2|\Theta_{\alpha\ov\alpha} \upsilon|^2
+ |D_\alpha D_\alpha \upsilon|^2
+ e^\lambda \pi^*R_X(\alpha,\ov\alpha,\alpha,\ov\alpha),
$$
which is semipositive. A priori, it might be zero in directions tangent to the Grassmannian part. But in those directions it is bounded from below by the lower bound on the holomorphic sectional curvature of the Grassmannian, which is $2/k^2 > 0$, so the holomorphic sectional curvature of this metric is positive.

Later: I don't think this works this well. For one, my claim about normal frames is wrong. If $(s_1,\ldots,s_r)$ is a normal frame, then
\[
D(fs_1) = df \otimes s_1 + f D s_1
\]
which equals $df \otimes s_1$ at the center of the frame and not $0$. Maybe some of this can be patched.


\subsection{Trying again}

Let $p: (E,h) \to X$ be a holomorphic Hermitian vector bundle and let $\pi: \Gr(k,E) \to X$ be the Grassmannian bundle of $k$-planes in $E$. This is a submersion and we have a short exact sequence
\[
\begin{tikzcd}
0 \ar[r] &
T_{\Gr/X} \ar[r] &
T_{\Gr} \ar[r,"\pi_*"] &
\pi^* T_X \ar[r]
& 0.
\end{tikzcd}
\]
There is also a short exact sequence
\begin{equation}
\label{eq:grassmannian}
\begin{tikzcd}
0 \ar[r] &
S \ar[r,"j"] &
\pi^*E \ar[r,"q"] &
Q \ar[r]
& 0
\end{tikzcd}
\end{equation}
where $S \to \Gr(k,E)$ is the tautological bundle of $k$-planes in $\pi^*E$. That bundle is equipped with the Hermitian metric inherited from $\pi^*h$ on $\pi^*E$, and so is the quotient bundle $Q$.

\begin{prop}
The second fundamental form of $S$ gives a $\cc C^\infty$ isomorphism
\[
b : T_{\Gr/X} \longrightarrow \Hom(S,Q).
\]
of holomorphic vector bundles.
\end{prop}

\begin{proof}
The second fundamental form gives a smooth vector bundle morphism $b : T_{\Gr} \to \Hom(S,Q)$ by construction; our claim is only that it is an isomorphism when restricted to the relative tangent bundle.

To see this, pick a point $S_0 \in \Gr$ and pick a complement $W_0$. TODO.
\end{proof}


\begin{prop}
The smooth $(1,1)$-form
\[
\theta(\alpha, \ov\beta)
= \langle b(\alpha), \ov{b(\beta)} \rangle_{\Hom(S,Q)}
\]
on $\Gr(k,E)$ is closed, semipositive, and  positive on $T_{\Gr/X}$.
\end{prop}

\begin{proof}
The form is certainly semipositive, and positive on $T_{\Gr/X}$ as $b$ is an isomorphism there.

To show that it is closed, we calculate
\[
\partial_\gamma \theta(\alpha,\ov\beta)
= \langle D'_\gamma b(\alpha), \ov{b(\beta)} \rangle
+ \langle b(\alpha), \ov{\bar\partial_{\ov\gamma} b(\beta)} \rangle.
\]

Closed: TODO.
\end{proof}


\section{TODO: Flag manifold}
\label{sec:orga0ef6a0}

These are K\"ahler. There is a ``natural'' non-K\"ahler metric on them.

See \cite{lam1975formula} and \cite{papantonopoulou1983tangent} for a description of the tangent bundle of a flag manifold.

Let $V$ be a complex vector space of dimension $n$. A \emph{flag} in $V$ is an increasing chain
\[
S_1 \subset S_2 \subset \cdots \subset S_k \subset V
\]
of subspaces $S_j$ of $V$. We note the dimensions of the subspaces by
$\dim S_j = n_j$ and speak of a $(n_1, \ldots, n_k)$ flag in $V$. If
$n_j = j$ and $k = n$ the flag is \emph{complete}.

A \emph{flag manifold}  is the set
\[
F(n_1,\ldots,n_k; V)
:= \{
S_1 \subset \cdots \subset S_k \subset V
\mid
\text{$\dim S_j = n_j$}
\}
\]
of $(n_1,\ldots,n_k)$ flags in $V$. We've already seen two examples of
these, as $F(1;V)$ is the projective space of lines in $V$ and
$F(k;V)$ is the Grassmannian of $k$-planes in $V$. As we may suspect
from those examples, these are compact complex manifolds with a
transitive action of $\GL(V)$.

Given a flag, we can forget about one of its components and obtain a flag with fewer elements. This defines forgetful morphisms
\[
\pi_j : F(n_1,\ldots,n_k; V)
\to
F(n_1, \ldots, n_{j-1}, n_{j+1}, \ldots, n_k; V).
\]
They should be holomorphic submersions and their fibers should be Grassmannians
(if $U \subset V \subset W \mapsto U \subset W$ then $V/U \subset W/U$ so the fiber is
$\dim V-\dim U$ planes in $W/U$). Does this let us recursively compute
the curvature tensor of a flag manifold? We supposedly know what the
cohomology of a Grassmannian bundle is, so it should let us compute the
cohomology of a flag manifold.

Do we need to go through all of this to get the metric? A flag manifold can be realized as a quotient $\GL(V) / B$ by some subgroup $B \subset \GL(V)$. Is the Fr\"obenius inner product on $\End V$ invariant under the action of $\GL(V)$ or only $U(V,h)$? We have
\[
\langle f, \ov g \rangle
= \tr(h^{-1} \ov g^* h f).
\]
The action of $\GL(V)$ on itself is $a \cdot f = a f a^{-1}$, so $\ov{a \cdot g}^* = \ov{a g a^{-1}}^* = \ov{a^{-1}}^* \ov g^* \ov a^*$. Then
\[
\langle a \cdot f, \ov{a \cdot g} \rangle_h
= \tr(h^{-1} \ov{a^{-1}}^* \ov g^* \ov a^* h a f a^{-1})
= \tr((a \cdot h)^{-1} \ov g^* (a \cdot h) f)
= \langle f, \ov{g} \rangle_{a \cdot h}
\]
so $\langle a \cdot f, \ov{a \cdot g} \rangle = \langle f, \ov g \rangle$ for all $f,g$ if and only if $a \cdot h = h$, that is, if $a \in U(V,h)$.

On projective space, the Fubini--Study metric is induced by the metric on the sphere, which is more or less $SU(n)$. I guess this is similar. I'm not sure we can conclude much from the invariance of the metric? Does that let us prove it is K\"ahler?



\section{TODO: Direct image curvatures}
\label{sec:org5619e67}

Weil--Peterson, maybe.

Berndtsson \cite{berndtsson2009curvature} has papers from about ten years ago we could look at. Cao et al took that further recently, but that's probably too advanced for what we want to do.


\section{TODO: Intuitive explanation for curvature forms}
\label{sec:org2b43ecb}

Wikipedia has a handwavy explanation of curvature as what happens when we parallel transport a section along a parallelogram. Can we make this precise?



\section{TODO: Hom-metrics}

Somewhere you have a draft that talks about the space of functions $f : X \to Y$ and how we can put a metric on that when given metrics on $X$ or $Y$. I think we computed the curvature tensors we get from that.


\section{TODO: Moduli space of complex tori}

That space has a fairly explicit metric whose curvature tensor we can compute. Might be related to the Grassmannian?


\section{TODO: Hilbert scheme or cycle space}

Check Barlet and Magnusson's book. Maybe difficult.

\section{TODO: Curves in projective space}

Done in \cite{fischer1983differential}, apparently. Looks inscrutable.

\section{TODO: Hypersurfaces}

Done in \cite{vitter1974curvature} for projective space. Vitter cites Smyth~\cite{smyth1967differential} and Nomizu--Smyth~\cite{nomizu1968differential}.

Smyth proves that the only complete K\"ahler--Einstein hypersurfaces of the constant holomorphic sectional curvature spaces are those spaces themselves of lower dimension, or the quadric in projective space. He does this with a lot of Riemannian calculations. Can we simplify?

Nomizu--Smyth prove more things in this direction. They end by classifying curves in $\kk P^2$ according to their curvature. Maybe we can simplify this with our expression for the holomorphic Codazzi equations?



\section{TODO: Bochner--Kodaira--Nakano identities}
\label{sec:bochn-koda-nakano}

See
\hyperlink{https://mathoverflow.net/questions/289779/where-do-the-akizuki-nakano-identities-first-appear}{MathOverflow}
and
\hyperlink{https://en.wikipedia.org/wiki/Bochner\%E2\%80\%93Kodaira\%E2\%80\%93Nakano_identity}{Wikipedia}
for references. One of them is Demailly from his youth.

Demailly proved these identities for a Hermitian metric on a Hermitian holomorphic vector bundle. He used essentially the moving frames formulation. Can we write an intrinsic proof?



\section{Degenerate Hermitian geometry}
\label{sec:deglinalg}



\subsection*{Introduction}

Let $\pi : X \to S$ be a family of complex manifolds, and consider the associated short exact sequence
\[
\begin{tikzcd}
0 \ar[r] & T_{X/S} \ar[r] & T_X \ar[r] & \pi^*T_S \ar[r] & 0.
\end{tikzcd}
\]
We often find ourselves in the situation of having a metric $h_S$ on the base $S$ and a Hermitian form $h_{X/S}$ on the $X$ that is positive-definite on the relative tangent bundle $T_{X/S}$, and then construct a metric on $X$ as the sum $h_X := h_{X/S} + \pi^*h_S$. This happens when we consider a projective or Grassmannian bundle associated to a vector bundle; the blow-up of a point in the original proof of the Kodaira embedding theorem; or metrics on versal families of deformations.

In these situations, the metric on the base has a curvature tensor, and the relative form often has something that resembles such a tensor. As neither form is positive-definite on $X$, we can't use our common tools to calculate a curvature tensor for $h_X$ from this information.

If we had the equivalent of Chern connections and curvature forms for arbitrary Hermitian forms, we could try to prove a version of the Codazzi--Griffiths equations that relate the curvature form of sub- and quotient bundles to that of the ambient bundle. In the positive-definite case, those equations can be used to calculate the curvature tensor of the sum $h_1 + h_2$ of two metrics on a given bundle $E$ by considering the short exact sequence
\[
\begin{tikzcd}
0 \ar[r] & E \ar[r] & E\oplus E \ar[r] &  E \ar[r] & 0,
\end{tikzcd}
\]
where the metrics on the bundles are (in order) $h_1 + h_2$, $h_1 \oplus h_2$, and the induced metric on the ``quotient''.

In this note, we fill in the details of this sketch. In section~\ref{sec:degenerate-chern-connections} we show that a holomorphic vector bundle $E \to X$ with a smooth Hermitian form $b$ admits a Chern connection. Such a connection is not unique when the form is degenerate, but we can always get rid of the ambiguity between different choices of connections by applying the form $b$, as the difference of two such connections must take values in its kernel. This yields a unique curvature \emph{tensor} for such a form. We then prove the Codazzi--Griffiths equations hold in this setting, and turn to applications.

To do this, we require a degenerate analogue of many statements about adjoints, orthogonal subspaces and induced inner products that are classical in the positive-definite case. We do not know a reference for these results, which would make for fun exercises in a linear algebra course, and thus provide statements and proofs. We do this in section~\ref{sec:degenerate-linear-algebra} as we need to refer to those results for our work on connections and curvature.

We consider two applications of our degenerate results. The first is a dotting of i's and crossing of t's from works of Alvarez~\cite{alvarez2016positive} and Yang and Zheng~\cite{yang2019hirzebruch} on the curvature of metrics on projective bundles. They prove that such bundles admit metrics of positive holomorphic sectional curvature, and finish by saying the same arguments can prove that so do Grassmannian bundles. We treat the case of Grassmannian bundles in detail, and as a corollary conclude that bundles of flag manifolds admit metrics of positive holomorphic sectional curvature.

The second application is an until-now unfulfilled expectation of the author's Ph.D.~advisor. Consider a family $\pi : X \to S$ of compact K\"ahler manifolds over a smooth base. We can complexify the K\"ahler cone of each manifold and obtain a holomorphic fibration $p : C \to S$ of such cones. Hodge theory gives a natural smooth semipositive Hermitian form on $C$ that is positive-definite on the relative tangent bundle $T_{C/S}$. Given a metric on the base, we can thus construct a metric on $C$, and we use the tools developed here to calculate its curvature tensor. We pay special attention to the case of compact K\"ahler manifolds with zero first Chern class, whose base of versal deformations is smooth and admits a Weil--Petersson metric.

We finish this introduction by noting that we think the technical results developed here on general Hermitian forms are only interesting when applied to the study of honest Hermitian metrics, and do not have merit on their own. Some examples may explain this opinion:

As an extreme case, we can consider the zero form on any vector bundle. Any connection will be compatible with that form. As a less extreme example, consider a compact complex manifold of dimension $n$ that admits a nowhere-zero holomorphic one-form. That form gives us a smooth Hermitian form whose kernel forms a holomorphic subbundle of rank $n-1$ and whose quotient is the trivial line bundle. Any of our Chern connections for this form will yield a curvature form whose top exterior power vanishes; we conclude the top Chern class of the manifold is zero. However, a much easier way to conclude the same is to use the additivity property of total Chern classes. The moral is that the larger the kernel of a given form, the less information having a compatible connection gives us. We gain the most by considering the non-degenerate case.

Even in the non-degenerate case (where our results are not new as the usual proofs for everything work unchanged for non-degenerate forms), the most useful results are to be found when the form we consider is positive-definite. The reason is that the Laplacian of a non-degenerate form of mixed signature will not be strongly elliptic, so the Hodge isomorphism theorem fails, and with it go most of the tools and techniques developed over the last hundred years or so. We prefer to not throw this baby out with the bathwater.




\subsection{Linear algebra}
\label{sec:degenerate-linear-algebra}



All vector spaces are finite-dimensional and over the complex numbers. Let $V$ be a vector space and let $b$ be a Hermitian form on $V$. We want to study the geometry of the pair $(V,b)$.\footnote{Everything we say also applies to symmetric bilinear forms on finite-dimensional vector spaces over arbitrary fields. The reader who is interested in those can mentally erase all the conjugations of vectors or spaces.}

Our interest in the topic comes from complex differential geometry, where we sometimes construct an honest metric by adding together two degenerate Hermitian forms. We'll thus mostly be focused on constructing adjoints of morphisms and inducing forms in short exact sequences.

\begin{defi}
A \emph{Hermitian space} is a finite-dimensional vector space $V$ with a Hermitian form $b$.

A \emph{morphism} of Hermitian spaces is a linear morphism $f : V \to W$ such that
\[
b_W(fx, \ov{fy}) = b_V(x, \ov y)
\]
for all $x, y \in V$.
\end{defi}

The category this forms is the one that has Hermitian spaces for objects and the above morphisms between them. Given two objects in this category, there is not necessarily any morphism between them; if $V$ has a non-degenerate form while $W$ has the zero form, there is no morphism $V \to W$.

A Hermitian form can be seen as a linear morphism $b : V \to \ov V^*$. As such, it has a kernel. We clearly have
\[
\Ker b = \{ x \in V \mid b(x, \ov y) = 0 \text{ for all $y \in V$}\}.
\]

\begin{prop}
Let $f : V \to W$ be a morphism of Hermitian spaces. Then
\[
f^{-1}(\Ker b_W) \subset \Ker b_v.
\]
\end{prop}

\begin{proof}
If $x \in f^{-1}(\Ker b_W)$ then
\[
b_V(x, \ov y) = b_W(fx, \ov{fy}) = 0
\]
for all $y \in V$, so $x \in \Ker b_V$.
\end{proof}

One motivation for the study of these objects is that while this category doesn't have many morphisms, the category of vector spaces with Hermitian inner products has even fewer morphisms. All morphisms there must be injective, while there are applications where we would like to be able to relax that condition. If we start from vector spaces with inner products and take as morphisms arbitrary linear ones, we get semipositive degenerate Hermitian forms. A priori there doesn't seem to be a reason to consider only positive forms, so we look at arbitrary ones.



The results we want to prove are all known for spaces with non-degenerate form. Our main technical tool will be to reduce to that case by quotienting out the kernel of the form we have.

\begin{prop}
Let $V$ be a vector space. For every Hermitian form $b$ on $V$, there exists a unique non-degenerate Hermitian form $\hat b$ on $V / \Ker V$ such that $\pi : V \to V/\Ker b$ is a Hermitian morphism.
\end{prop}


\begin{proof}
We define
\[
\hat b(qx, \ov{qy})
= b(x, \ov y).
\]
This is well-defined, as any other element in $qx$ differs from the one we chose by an element of $\Ker q = \Ker b$. The form $\hat b$ is also non-degenerate, as if $qx \in \Ker \hat b$ then $x \in \Ker b$, so $qx = 0$. The form is also defined in such a way that $q$ is a morphism of Hermitian spaces.

If $b'$ is another non-degenerate Hermitian form on $V / \Ker b$ that makes $q$ into a Hermitian morphism, then
\[
b'(qx, \ov{qy})
= b(x, \ov{y})
= b(qx, \ov{qy})
\]
for all $qy \in V / \Ker b$. Thus $b(qx) = b'(qx)$ as $(0,1)$-forms for all $qx \in V / \Ker b$, so $b = b'$.
\end{proof}


If this operation should be called anything, it is clearly a \emph{purge} of the original space, as it removes all the degeneracy from the Hermitian form. As the first of many applications, we'll study adjoints of morphisms.


\begin{defi}
Let $(V, b_V)$ and $(W, b_W)$ be Hermitian spaces, and let $f : V \to W$ be a linear morphism. An \emph{adjoint} of $f$ is a linear morphism $f^\dagger : W \to V$ such that
\[
b_V(f^\dagger x, \ov y)
= b_W(x, \ov{f y})
\]
for all $x \in W$ and $y \in V$.
\end{defi}


If $b_V$ is non-degenerate, every morphism $f : V \to W$ admits a unique adjoint. It is defined by
\[
f^\dagger = b_V^{-1} \circ \ov f^* \circ b_W,
\]
where $\ov f^*$ is the conjugate dual morphism $f^* : \ov W^* \to \ov V^*$ that $f$ induces. In general, adjoints may not exist, but we can say when they do.


\begin{theo}
Let $f : V \to W$ be a linear morphism. The morphism $f$ admits an adjoint if and only if $f(\Ker b_V) \subset \Ker b_W$. If the set of adjoints of $f$ is not empty, it is a $\Hom(W, \Ker b_V)$ torsor.
\end{theo}

\begin{proof}
If $f$ admits an adjoint $f^\dagger$, then
\[
b_V(f^\dagger x, \ov y)
= b_W(x, \ov{fy})
\]
for all $x \in W$ and $y \in V$. Taking $y \in \Ker b$, we see that $b_W(x, \ov{fy}) = 0$ for all $x \in W$, so $f(y) \in \Ker b_W$.

Suppose then that $f(\Ker b_V) \subset \Ker b_W$. Then $f$ defines a linear morphism $\hat f : V / \Ker b_V \to W / \Ker b_W$. As the induced forms on those spaces are non-degenerate, we get a unique adjoint $\hat f^\dagger : W / \Ker b_W \to V / \Ker b_V$. It induces a morphism $W \to V / \Ker b_V$. Picking a lift $V / \Ker b_V \to V$ we then get an adjoint $f^\dagger$ of $f$.

If $g$ and $h$ are adjoints of $f^\dagger$, then
\[
b_V((g - h)x, \ov y)
= b_V(gx, \ov y) - b_V(hx, \ov y)
= b_W(x, \ov{fy}) - b_W(x, \ov{fy})
= 0
\]
for all $x \in W$ and $y \in V$, so the image of $g - h$ is contained in $\Ker b_V$.
\end{proof}


\begin{coro}
The subset of $\Hom(V,W)$ of morphisms that have adjoints is a linear subspace.  The zero morphism always has an adjoint.
\end{coro}



\begin{coro}
If $f : V \to W$ has an adjoint $f^\dagger$, then $f^\dagger$ also has an adjoint.
\end{coro}

\begin{proof}
If $x \in \Ker W$, then
\[
b_V(f^\dagger x, \ov y)
= b_W(x, \ov{fy})
= 0
\]
for all $y \in V$, so $f^\dagger x \in \Ker b_V$.
\end{proof}

The difference between a morphism $f$ and an adjoint of its adjoint is an element of $\Hom(V, \Ker b_V)$, so a double adjoint is unique and equal to $f$ if and only if $b_V$ is non-degenerate.


\begin{prop}
Let $j : S \to V$ be a linear morphism, let $b_V$ be a Hermitian form on $V$, and set $b_S = j^*b_V$. Then $j$ admits an adjoint.
\end{prop}

\begin{proof}
We have
\[
\Ker b_S = j^{-1}(\Ker b_V) \cup \Ker j
\]
so $j(\Ker b_S) = j(j^{-1}(\Ker b_V)) \subset \Ker b_V$.
\end{proof}



If $V$ and $W$ are vector spaces with Hermitian forms $b_V$ and $b_W$, we'd like to use those forms to define a Hermitian form on $\Hom(V,W)$. When the form $b_V$ is non-degenerate, every linear morphism $f : V \to W$ admits a unique adjoint $f^\dagger : W \to V$, and the Fr\"obenius form on $\Hom(V,W)$ is
\[
b(f, \ov g) = \tr(g^\dagger f).
\]
In general, not every morphism $f : V \to W$ admits an adjoint, and the ones that do may not admit unique adjoints. We could try to define a Hermitian form on the subspace of morphisms that admit adjoints by $\tr(g^\dagger f)$, but as adjoints may not be unique this is not well-defined.

The way around this difficulty is again to pass to the quotient spaces. A morphism $f : V \to W$ admits an adjoint if and only if $f(\Ker b_V) \subset \Ker b_W$. In the latter case, there exists a unique morphism $\hat f : V / \Ker b_V \to W / \Ker b_W$ such that the diagram
\[
\begin{tikzcd}
V \ar[d,"\pi_V"] \ar[r,"f"] & W \ar[d,"\pi_W"]
\\
V / \Ker b_V \ar[r,"\hat f"] & W / \Ker b_W
\end{tikzcd}
\]
commutes; it is defined by $\hat f([v]) = \pi_W(f(v))$.


\begin{defi}
Let $(V,b_V)$ and $(W,b_W)$ be Hermitian spaces. We define
\[
b_{\Hom(V,W)}(f, \ov g) := \tr((\hat g)^\dagger \hat f)
\]
on the subspace of $\Hom(V,W)$ of morphisms that admit adjoints.
\end{defi}

If we denote by $\Hom_b(V,W) \subset \Hom(V,W)$ the subspace of morphisms that admit adjoints, the definition says that we have a surjective morphism
\[
\Hom_b(V,W) \to \Hom(V/\Ker b_V, W/\Ker b_W)
\]
and that we define a Hermitian form upstairs by pullback of the one downstairs, which exists because the Hermitian forms there are non-degenerate.






\paragraph{Orthogonal complements}


The objective of this section is to show how to obtain a Hermitian form on a quotient space given one on the ambient space. It is well-known how to do this when the forms are non-degenerate (for example, take duals to embed the quotient dual in the ambient dual, restrict the dual metric, and invert) but less clear how this should work in the general case.


A Hermitian form can be seen as a linear morphism $b : V \to \ov V^*$. As such, it comes with a kernel
\[
\Ker b = \{ v \in V \mid b(v) = 0 \}.
\]
For any subspace $S \subset V$ we define its \emph{orthogonal complement} with respect to the form $b$ as
\[
S^\perp = \{ v \in V \mid b(v, \ov w) = 0 \text{ for all $w \in S$}\}.
\]
It is clearly a linear subspace of $V$.

\begin{prop}
\[
\Ker b \subset S^\perp.
\]
\end{prop}

\begin{proof}
Let $v \in \Ker b$. Then $b(v, \ov w) = 0$ for all $w \in S$.
\end{proof}


As before we have the quotient space $q : V \to V / \Ker b$, and the induced non-degenerate form $\hat b$ on $V / \Ker b$.

\begin{prop}
Let $S \subset V$. Then
\[
q(S^\perp) = q(S)^\perp,
\]
where the orthogonal complement on the right-hand side is with respect to $\hat b$.
\end{prop}

\begin{proof}
Let $v \in S^\perp$, so $b(v, \ov w) = 0$ for all $w \in S$. Then $\hat b(qv, \ov{qw}) = 0$ for all $w \in S$, which means that $\hat b(qv, \ov w) = 0$ for all $w \in q(S)$, so $qv \in q(S)^\perp$.

Suppose then that $v \in q(S)^\perp$, so $\hat b(v, \ov w) = 0$ for all $w \in q(S)$. If $x \in V$ is such that $qx = v$, then this means that $b(x, \ov w) = 0$ for all $w \in S$. Then $x \in S^\perp$, so $qx \in q(S^\perp)$.
\end{proof}



\begin{prop}
\[
S + S^\perp = V.
\]
\end{prop}

\begin{proof}
We have
\[
q(S + S^\perp)
= q(S) + q(S^\perp)
= q(S) + q(S)^\perp
= V / \Ker b
\]
as $\hat q$ is non-degenerate. As $q$ is surjective, this implies the result.
\end{proof}




\begin{prop}
\[
S \cap S^\perp = S \cap \Ker b.
\]
\end{prop}

\begin{proof}
We have
\[
q(S \cap S^\perp)
= q(S) \cap q(S^\perp)
= q(S) \cap q(S)^\perp
= 0
\]
as $\hat q$ is non-degenerate,
so $S \cap S^\perp \subset \Ker q = \Ker b$. Conversely, $\Ker b \subset S^\perp$, so $S \cap \Ker b \subset S \cap S^\perp$.
\end{proof}


If $f : S \to V$ is a linear morphism and $b_V$ is a Hermitian form on $V$, then we get an induced Hermitian form $b_S := f^*b_V$ on $S$. We can ask whether the same happens when we have a morphism $f : V \to Q$ and a form $b_V$. Clearly we can only expect something on the image of $f$, so we should take it to be surjective. When the form on $b_V$ is non-degenerate, we can define a form on $Q$ by using the injection $0 \to Q^* \to V^*$ and the dual Hermitian forms. In general, we can still get a Hermitian form on the quotient space by using orthogonal complements.


\begin{theo}
Let $q : V \to Q$ be a surjective morphism. A Hermitian form $b_V$ on $V$ induces a Hermitian form $b_Q$ on $Q$.
\end{theo}

\begin{proof}
Let $S = \Ker q$, and consider $S^\perp \subset V$. We have a short exact sequence
\[
\begin{tikzcd}
0 \ar[r] &
S \cap \Ker b_V \ar[r] &
S^\perp \ar[r,"q"] &
Q \ar[r] &
0
\end{tikzcd}
\]
because $S^\perp \cap \Ker q = S^\perp \cap S = S \cap \Ker b_V$, and $q$ restricted to $S^\perp$ is still surjective. We define
\[
b_Q(qx, \ov{qy})
= b_V(x, \ov y)
\]
for $x, y \in S^\perp$. This is well-defined by the above, as any other element $x'$ that maps to $qx$ satisfies $x' - x \in S \cap \Ker b_V \subset \Ker b_V$.
\end{proof}


\begin{prop}
Let $q : V \to Q$ be surjective, and let $b_V$ and $b_Q$ be as above. Then $q$ admits an adjoint $q^\dagger : Q \to V$.
\end{prop}

\begin{proof}
We need to show that $q(\Ker b_V) \subset \Ker b_Q$. If $x, y$ are elements of $V$, we have by definition
\[
b_Q(qx, \ov{qy}) = b_V(x, \ov y).
\]
If $x \in \Ker b_V$, this implies that $b_Q(qx, \ov{qy}) = b_V(x, \ov y) = 0$ for all $y \in V$, so $qx \in \Ker b_Q$.
\end{proof}


\subsection{Chern connections}
\label{sec:degenerate-chern-connections}



Let $E \to X$ be a holomorphic vector bundle with a smooth Hermitian form $b$. We say that a connection $D$ on $E$ is \emph{compatible with $b$} if
\[
d b(s, \ov t) = b(Ds, \ov t) + b(s, \ov{Dt})
\]
for all sections $s, t$ of $E$.


\begin{prop}
Let $E \to X$ be a holomorphic vector bundle and let $b$ be a smooth Hermitian form on $E$. Then there exists a connection $D$ on $E$ that's compatible with $b$ and has $D^{0,1} = \bar\partial$.
\end{prop}


\begin{proof}
The plan is to construct a connection locally on trivializing neighborhoods of $E$ and to glue the pieces together with a partition of unity.

Let $s$ be a smooth section of $E$ over a neighborhood $U$. We want to solve the equation
\begin{equation}
\label{eq:conn}
\partial_\alpha b(s, \ov t)
= b(A(\alpha), \ov t),
\end{equation}
where $\alpha \in \cc C^\infty(U, T_X)$, $t \in \cc C^\infty(U, E)$ and $A \in \cc A^{1,0}(U, E)$. Consider the short exact sequence
\[
\begin{tikzcd}
0 \ar[r] &
\Ker b \ar[r] &
E \ar[r,"b"] &
\im b \ar[r] &
0,
\end{tikzcd}
\]
and tensor it by the vector bundle $\Omega^1_X$ to obtain
\[
\begin{tikzcd}
0 \ar[r] &
\Omega^1_X \otimes \Ker b \ar[r] &
\Omega^1_X \otimes E \ar[r,"\id \otimes b"] &
\Omega^1_X \otimes \im b \ar[r] &
0.
\end{tikzcd}
\]
We claim that $\alpha \otimes t \mapsto \partial_\alpha b(s, \ov t)$ is a smooth section of $\Omega^1_X \otimes \im b$. By the above, this is true if it is zero on the sheaf generated by sections of the form $\alpha \otimes t$, where $t \in \Ker b$. But for such a section we have $b(s, \ov t) = 0$ on $U$, so $\partial_\alpha b(s, \ov t) = 0$. By taking a small enough neighborhood so the sheaf morphism above is surjective, or noting that the sheaves are ones of smooth modules which are fine, we obtain a section $A_s \in \cc C^\infty(U, \Omega_X^1 \otimes E)$ such that \eqref{eq:conn} holds for any $\alpha$ and $t$.

Suppose now that there exists a holomorphic frame $(e_1, \ldots, e_r)$ of $E$ over $U$. Such a frame exists over a small enough neighborhood around any point. Let $A_1, \ldots, A_r$ be the sections we just found for $e_1, \ldots, e_r$. Any section $s \in \cc C^\infty(U, E)$ can be written as $s = \sum_j f_j e_j$ for smooth functions $f_j$ on $U$. We define
\[
D s := \sum_{j=1}^r df_j \otimes e_j + f_j A_j
\]
on $U$. We claim that $D$ is a connection on $E$ over $U$ that is compatible with $b$ and whose $(0,1)$-part is $\bar\partial$.

This object $D$ is a $1$-form with values in $E$ by construction of the $A_j$. If $s$ is a section of $E$ and $f$ a smooth function, we write $s = \sum_j f_j e_j$. Then
\begin{align*}
D(fs)
&= \sum d(f f_j) \otimes e_j + f f_j A_j
\\
&= \sum df \otimes f_j e_j + f df \otimes e_j + f f_j A_j
= df \otimes s + f Ds,
\end{align*}
so $D$ is a connection on $E$. The forms $A_j$ are $(1,0)$-forms by construction, so the $(0,1)$-part of $Ds$ is $\sum_j \bar\partial f_j e_j = \bar\partial s$.

Finally, let $f$ and $g$ be smooth functions. We have
\begin{align*}
d b(fe_j, \ov{ge_k})
&= d\bigl(f \ov g b(e_j, \ov e_k) \bigr)
\\
&= df \cdot \ov g b(e_j, \ov e_k)
+ f dg \cdot b(e_j, \ov e_k)
+ \partial b(e_j, \ov e_k)
+ \ov{\partial b(e_k, \ov e_j)}
\\
&= b(df \otimes e_j, \ov{ge_k})
+ b(f e_j, \ov{dg \otimes e_k})
+ b(A_j, \ov e_k)
+ b(e_j, \ov{A_k})
\\
&= b(D(f e_j), \ov{ge_k}) + b(f e_j, \ov{D(g e_k)}).
\end{align*}
For smooth sections $s = \sum f_j e_j$ and $t = \sum g_k e_k$ we deduce from the above and linearity that $D$ is compatible with $b$ over the neighborhood $U$.

We now cover $X$ by neighborhoods $U_\mu$ as above and construct connections $D_\mu$ on each $U_\mu$. We then take a partition of unity $(\phi_\mu)$ relative to this covering, and define a connection $D$ on $E$ by setting
\[
D s = \sum_\mu D_\mu(\phi_\mu s).
\]
This defines a connection on $E$ that's compatible with $b$ and whose $(0,1)$-part is $\bar\partial$.
\end{proof}



We will call a connection $D$ that's compatible with $b$ and has $D^{0,1} = \bar\partial$ a \emph{Chern connection} of $b$. If $b$ has a nontrivial kernel, such a connection is not unique.


\begin{prop}
Let $D_1$ and $D_2$ be Chern connections of $b$. Then $D_1 - D_2$ is an element of $\cc A^{1,0}(\Hom(E, \Ker b))$.
\end{prop}

\begin{proof}
It is well-known that the difference of two connections is an element of $\cc A^1(\End E)$. Both connections here have the same $(0,1)$-part, so their difference is a $(1,0)$-form. If $s$ and $t$ are holomorphic sections, we have
\[
b((D'_1 - D'_2)s, \ov t)
= \partial b(s, \ov t) - \partial b(s, \ov t) = 0,
\]
so $D_1 - D_2 = D_1' - D_2'$ takes values in $\Ker b$.
\end{proof}


\begin{prop}
If $s$ is a section of $\Ker b$ and $D$ is a Chern connection of $E$, then $Ds$ takes values in $\Ker b$.
\end{prop}

\begin{proof}
If $s$ is a section of $\Ker b$, then
\[
0
= \partial b(s, \ov t)
= b(Ds, \ov t)
\]
for all holomorphic sections $t$ of $E$, so $D s \in \cc A^1(\Ker b)$.
\end{proof}


\begin{theo}
Let $(E, b) \to X$ be a holomorphic vector bundle with a Hermitian form, and let $D$ be a Chern connection of $E$. The curvature tensor
\[
R(\alpha,\ov\beta,s, \ov t)
= b\bigl(\tfrac i2 D^2_{\alpha\ov\beta}s, \ov t\bigr)
\]
is independent of the choice of $D$.
\end{theo}


\begin{proof}
Let $D_1$ and $D_2$ be Chern connections of $b$. Then $D_1 = D_2 + A$, where $A \in \Hom(E, \Ker b)$, and
\[
D_1^2
= (D_2 + A)^2
= D_2^2 + A D_2 + D_2 A + A^2.
\]
The difference $D_1^2 - D_2^2$ thus takes values in $\Ker b$.
\end{proof}



These Chern connections and curvature tensors behave very well with respect to
pull-backs of arbitrary bundle morphisms:


\begin{prop}
Let $E \to X$ and $F \to X$ be holomorphic vector bundles. Let $b_F$ be a
Hermitian form on $F$ with a Chern connection $D_F$ and curvature tensor $R_F$
and let $f : E \to F$ be a holomorphic vector bundle morphism.
Then $f^*D_F$ and $f^*R_F$ are a Chern connection and the curvature tensor of
the Hermitian form $f^*b_F$ on $E$.
\end{prop}

\begin{proof}
    TODO
\end{proof}


\subsection{A word on notation}


In a lot of situations we care about, we now only have equality up to things that take values in the kernel of a Hermitian form. For example, if $f : E \to F$ is a Hermitian morphism, it admits an adjoint $f^\dagger$ and we have
\[
b_E(f^\dagger f s, \ov t)
= b_F(f s, \ov{f t})
= b_E(s, \ov t)
\]
for all sections $s$ and $t$,
so $f^\dagger f - \id_E \in \Hom(E, \Ker b_E)$. When working on vector spaces we could deal with this by quotienting out the kernel. If we try the same trick here, we run into the fact that the rank of the sheaf $\Ker b \to X$ may not be constant. We want to do differential geometry to the sections of this sheaf, so this is a problem.

This difficulty is a mirage as it can be taken care of by notation. We define an equivalence relation on sections by saying that two sections are equivalent if their difference is in the kernel of a Hermitian form of interest:


\begin{defi}
If $(E, b) \to X$ is a holomorphic vector bundle with a Hermitian form $b$, and $s, t \in \cc A^k(E)$, we write
\(
s \sim t
\)
if $s - t \in \cc A^k(\Ker b)$.
\end{defi}

In general it's problematic to define differential forms with values in a sheaf, but here it's hopefully clear what $\cc A^k(\Ker b)$ means as $\Ker b$ is a subsheaf of the sheaf of sections of a vector bundle.

Using this notation we can rephrase some of our earlier results on degenerate forms:
\begin{itemize}
\item
If $f : E \to F$ is a Hermitian morphism that admits an adjoint $f^\dagger$, then $f^\dagger f \sim \id_E$.

\item
If $f : E \to F$ is a morphism and $f_1^\dagger$ and $f_2^\dagger$ are adjoints of $f$, then $f_1^\dagger \sim f_2^\dagger$.

\item
If $D$ is a Chern connection of $(E,b)$ and $s \sim t$ then $Ds \sim Dt$.

\item
If $D_1$ and $D_2$ are Chern connections of $(E,b)$, then $D_1s \sim D_2s$ for all sections $s$.
\end{itemize}
As adjoints and connections behave well with respect to this relation, we can
prove the result we want by essentially the same arguments as in the case where
the Hermitian forms are positive-definite; we simply replace ``$=$'' in those
proofs by ``$\sim$'' and note that everything else works as before. We will use
this to prove the Codazzi--Griffiths equations by the same arguments as always.



\subsection{Sub- and quotient bundles}

Let
\[
\begin{tikzcd}
0 \ar[r] &
S \ar[r,"j"] &
E \ar[r,"q"] &
Q \ar[r] &
0
\end{tikzcd}
\]
be a short exact sequence of holomorphic bundles over $X$, and let $b_E$ be a smooth Hermitian form on $E$. It induces Hermitian forms $b_S$ and $b_Q$ on $S$ and $Q$ as we have seen. Both morphisms $j$ and $q$ also admit adjoints $j^\dagger: E \to S$ and $q^\dagger : Q \to E$. We denote Chern connections of $S$, $E$ and $Q$ by $D_S$, $D_E$ and $D_Q$.



TODO: Rewrite

We need $Df^\dagger = (Df)^\dagger$, so we need induced forms on things like $\Hom$. Or: If $b_V$ and $b_W$ are forms on $V$ and $W$, then $b_{\Hom(V,W)}$ is only defined for $f$ that admit adjoints. Then it is defined as
\[
b_{\Hom(V,W)}(f, \ov g) = \tr (f g^\dagger).
\]
Is that well-defined? What if we pick a different adjoint?


\begin{defi}
The \emph{second fundamental form} of $S$ in $E$ is
\[
b(s) := q(D_E(js) - jD_S(s)).
\]
\end{defi}

\begin{prop}
The second fundamental form is an element of $\cc A^{1,0}(\Hom(S,Q))$.
We have
\[
b(s)
= q(D_{\Hom(S,E)}(j)(s))
= q(D_E(js))
\]
for sections $s$ of $S$.
\end{prop}

\begin{proof}
As $qj = 0$ it is clear that $b(s) = q(D_E(js))$. We also have
\[
D_E(js) = D_{\Hom(S,E)}(j)(s) + j(D_Ss),
\]
so $b(s) = q(D_{\Hom(S,E)}(j)(s))$.

As $j$ is holomorphic, we have $D_{\Hom(S,E)}j = D'_{\Hom(S,E)}j$ so $b$ has no $(0,1)$-part. It is also clearly $\cc C^\infty$-linear in its tensor field variable. If $f$ is a smooth function, we have
\[
D_E(j(fs))
= df \otimes js + f D_E(js)
\]
so $b(fs) = q(D_E(j(fs))) = q(f D_E(js)) = fb(s)$.
\end{proof}



Our goal is to calculate the curvature forms of the sub- and quotient bundles. The first stop on the way is a wildly useful list of formulas from Demailly~{{\cite[Theorem~14.3]{demailly-complex}}}.


\begin{prop}
\label{prop:seq-formulas}
\begin{alignat*}{2}
D'_{\Hom(S,E)}j &\sim q^\dagger \circ b,
&
\qquad
\bar\partial j &= 0,
\\
D'_{\Hom(E, Q)} q &\sim - b \circ j^\dagger,
&
\bar\partial q &= 0,
\\
D'_{\Hom(E,S)} j^\dagger &\sim 0,
&
\bar\partial j^\dagger &\sim b^\dagger \circ q,
\\
D'_{\Hom(Q,E)} q^\dagger &\sim 0,
&
\bar\partial q^\dagger &\sim - j \circ b^\dagger,
\\
D'_{\Hom(S,Q)} b &\sim 0,
&
\bar\partial b^\dagger &\sim 0.
\end{alignat*}
\end{prop}

\begin{proof}
Let $s$ be a section of $S$. Then
\[
j(D_S s) + q^\dagger b(s)
\sim D_E(js)
= D_{\Hom(S,E)}j (s) + j(D_S s),
\]
where the first equality is by definition of $D_S$ and $b$, so
\[
D_{\Hom(S,E)}j \sim q^\dagger \circ b.
\]
The morphism $j$ is holomorphic, so $\bar\partial j = 0$. This proves the first line. The third line follows from the first by taking adjoints.


For the second line, the morphism $q$ is also holomorphic, so $\bar\partial q = 0$. Taking adjoints, we get that $D'_{\Hom(Q,E)} q^\dagger \sim 0$.
We have $\id_E \sim j \circ j^\dagger + q^\dagger \circ q$, so
\begin{align*}
0
&= D'_{\End E} \id_E
\\
&\sim D'_{\Hom(S,E)} j \circ j^\dagger + j \circ D'_{\Hom(E,S)}j^\dagger
+ D'_{\Hom(Q,E)}q^\dagger \circ q + q^\dagger D'_{\Hom(E,Q)} q
\\
&\sim q^\dagger \circ b \circ j^\dagger + q^\dagger \circ D'_{\Hom(E,Q)}q.
\end{align*}
Applying $q$, we conclude that $D'_{\Hom(E,Q)}q \sim - b \circ j^\dagger$. This proves the second line. Taking adjoints we get the last piece of the fourth line.

Finally we note that
\begin{align*}
D'_{\Hom(S,Q)} b
&\sim D'_{\Hom(S,Q)} (q D'_{\Hom(S,E)}j)
\\
&\sim - b \circ j^\dagger \circ q^\dagger \circ b
+ q^\dagger (D'_{\Hom(S,Q)})^2 j
\sim 0
\end{align*}
as we're dealing with curvature tensors of Hermitian forms which are of type $(1,1)$. The statement about $\bar\partial b^\dagger$ follows by taking adjoints.
\end{proof}



\begin{prop}
Under the smooth splitting $E \to S \oplus Q$ defined by $s \mapsto j^\dagger s \oplus qs$ the Chern connection and curvature form of $E$ is
\begin{align*}
D_E s &\sim
\begin{pmatrix}
j^\dagger & q
\end{pmatrix}^\dagger
\begin{pmatrix}
D_S & - b^\dagger
\\
b & D_Q
\end{pmatrix}
\begin{pmatrix}
j^\dagger \\ q
\end{pmatrix}
(s),
\\
D_E^2 s &\sim
\begin{pmatrix}
j^\dagger & q
\end{pmatrix}^\dagger
\begin{pmatrix}
D^2_S - b^\dagger \wedge b & -D'_{\Hom(Q,S)} b^\dagger
\\
\bar\partial b & D^2_Q - b \wedge b^\dagger
\end{pmatrix}
\begin{pmatrix}
  j^\dagger \\ q
\end{pmatrix}(s).
\end{align*}
\end{prop}


\begin{proof}
Let $s$ be a section of $E$. We write
\[
s \sim j (j^\dagger s) + q^\dagger( qs).
\]
Then
\[
D_E s
\sim q^\dagger \circ b (j^\dagger s)
+ j \circ D_S( j^\dagger s)
- j \circ b^\dagger (qs)
+ q^\dagger \circ D_Q (qs).
\]
This proves the first statement.
For each of the terms here, we have
\begin{align*}
D_E(q^\dagger \circ b (j^\dagger s))
&\sim -j \circ b^\dagger \circ b (j^\dagger s)
+ q^\dagger \circ \bar\partial b (j^\dagger s)
- q^\dagger \circ b \circ D_S(j^\dagger s),
\\
D_E(j \circ D_S( j^\dagger s))
&\sim q^\dagger \circ b \circ D_S(j^\dagger s)
+ j \circ D_S^2 (j^\dagger s),
\\
D_E(-j \circ b^\dagger (qs))
&\sim - q^\dagger \circ b \circ b^\dagger (qs)
- j \circ D'_{\Hom(Q,S)}b^\dagger (qs)
+ j \circ b^\dagger \circ D_Q(qs),
\\
D_E(q^\dagger \circ D_Q (qs))
&\sim -j \circ b^\dagger \circ D_Q(qs)
+ q^\dagger \circ D_Q^2 (qs).
\end{align*}
Grouping these together by pre- and postfix morphisms, we get
\begin{align*}
D_E^2 s
&\sim j \bigl( D_S^2 - b^\dagger \circ b \bigr) j^\dagger s
- j \bigl( D'_{\Hom(Q,S)} b^\dagger \bigr) qs
\\
&\qquad
+ q^\dagger \bigl( \bar\partial b \bigr) j^\dagger s
+ q^\dagger \bigl( D^2_Q - b \circ b^\dagger \bigr) qs
\\
&=
\begin{pmatrix}
j^\dagger & q
\end{pmatrix}^\dagger
\begin{pmatrix}
D^2_S - b^\dagger \wedge b & -D'_{\Hom(Q,S)} b^\dagger
\\
\bar\partial b & D^2_Q - b \wedge b^\dagger
\end{pmatrix}
\begin{pmatrix}
  j^\dagger \\ q
\end{pmatrix}(s).
\end{align*}
\end{proof}




\begin{coro}[Codazzi--Griffiths equations]
The curvature tensors of $S$ and $Q$ satisfy
\begin{align*}
R_S(\alpha, \ov\beta, s, \ov t)
&= R_E(\alpha, \ov\beta, js, \ov{jt})
- \tfrac i2 h_Q(b(\alpha, s), \ov{b(\beta, t)}),
\\
R_Q(\alpha, \ov\beta, s, \ov t)
&= R_E(\alpha, \ov\beta, q^\dagger s, \ov{q^\dagger t})
+ \tfrac i2 h_S(b^\dagger(\ov\beta, s), \ov{b^\dagger(\ov\alpha, t)}).
\end{align*}
\end{coro}

\begin{proof}
The task here is to convince the reader that our signs are correct and that the arguments go in the right place. Recall that we have a $(1,0)$-form $b$ with values in $\Hom(S,Q)$. Then $b \circ b^\dagger$ and $b^\dagger \circ b$ are both $(1,1)$-forms, with values in $\End Q$ and $\End S$, respectively. Our first claims are that the $(1,1)$-forms
\[
\tfrac i2 h_Q(b \circ b^\dagger(s), \ov s)
\qandq
\tfrac i2 h_S(b^\dagger \circ b(s), \ov s)
\]
are real and seminegative and semipositive for all sections $s$ (of $S$ and $Q$ as appropriate), respectively. For the realness, we have
\begin{align*}
\overline{\rho(\alpha, \ov\beta)}
:= \overline{\tfrac i2 h_Q(b \circ b^\dagger(\alpha, \ov\beta, s), \ov s)}
&= -\tfrac i2 h_Q(s, \ov{b \circ b^\dagger(\alpha,\ov\beta, s)})
\\
&= -\tfrac i2 h_Q(b \circ b^\dagger(\ov\alpha, \beta, s), \ov{s})
\\
&= \tfrac i2 h_Q(b \circ b^\dagger(\beta, \ov\alpha, s), \ov{s})
= \rho(\beta, \ov\alpha),
\end{align*}
so the first form is real. The second is also real by a very similar calculation.

TODO: We'd like to show that
\[
\tfrac i2 b \wedge b^\dagger (\alpha, \ov\beta, s)
= b(\alpha, b^\dagger(\ov\beta, s)),
\]
where the left-hand side is the value of the $(1,1)$-form $\frac i2 b \wedge b^\dagger$. Maybe this needs to be interpreted as ``the $(1,1)$-form is the imaginary part of a Hermitian form with values in $\End Q$, which is the right-hand side''.

Now note that
\[
\tfrac i2 h_S(b \circ b^\dagger(\alpha, \ov\beta, s), \ov t)
= \tfrac i2 h_S(b(\alpha, b^\dagger(\ov \beta, s)), \ov t)
= \tfrac i2 h_S(b^\dagger(\ov \beta, s), \ov{b^\dagger(\ov \alpha, t)}),
\]
so the second form is semipositive. The first form looks very similar, but a careful inspection will reveal that if we were to write it out in local coordinates its terms would be of the form $\frac i2 d\bar z_k \wedge dz_j$. This explains why we have
\begin{align*}
\tfrac i2 h_Q(b^\dagger \circ b(\alpha, \ov\beta, s), \ov t)
&= -\tfrac i2 h_Q(b^\dagger \circ b(\ov\beta, \alpha, s), \ov t)
\\
&= -\tfrac i2 h_Q(b^\dagger(\ov\beta, b(\alpha, s)), \ov t)
= -\tfrac i2 h_Q(b(\alpha, s), \ov{b(\ov\beta, t)}),
\end{align*}
so the form is seminegative.

The announced formulas both follow from these calculations. The same calculations show that we have equality $R_S = R_E$ or $R_E = R_Q$ if and only if $|b|_{\Hom(S,Q)} = 0$ or $|b^\dagger|_{\Hom(Q,S)} = 0$, which happens if and only if the splitting $E = S \oplus Q$ is holomorphic.
\end{proof}






\bibliographystyle{plain}
\bibliography{sgct}

\end{document}
