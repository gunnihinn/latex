% Created 2021-02-09 Tue 14:04
% Intended LaTeX compiler: pdflatex
\documentclass[11pt]{article}
\usepackage[utf8]{inputenc}
\usepackage[T1]{fontenc}
\usepackage{graphicx}
\usepackage{grffile}
\usepackage{longtable}
\usepackage{wrapfig}
\usepackage{rotating}
\usepackage[normalem]{ulem}
\usepackage{amsmath}
\usepackage{textcomp}
\usepackage{amssymb}
\usepackage{capt-of}
\usepackage{hyperref}
\usepackage{amsthm}
\newcommand{\kk}[1]{\mathbb{#1}}
\DeclareMathOperator{\im}{Im}
\DeclareMathOperator{\Ker}{Ker}
\DeclareMathOperator{\id}{id}
\author{Gunnar Þór Magnússon}
\date{\today}
\title{The savage garden of curvature tensors}
\hypersetup{
 pdfauthor={Gunnar Þór Magnússon},
 pdftitle={The savage garden of curvature tensors},
 pdfkeywords={},
 pdfsubject={},
 pdfcreator={Emacs 27.1 (Org mode 9.3)}, 
 pdflang={English}}
\begin{document}

\maketitle
\tableofcontents



\section{Topics}
\label{sec:org093d592}

Discuss positivity concepts: Nakano, Griffits positive. Ricci positive. Bisectional curvature; holomorphic bisectional curvature.

Can prove that holomorphic bisectional curvature > 0 -> scalar curvature > 0.

\subsection{Conformal metrics}
\label{sec:org65fcbad}

Let \((E,h) \to X\) be a holomorphic Hermitian vector bundle, and let \(D\) be its Chern connection.

If \(f\) is a smooth real-valued function on \(X\), then \(h' := e^f h\) is again a Hermitian metric, \emph{conformal} to \(h\). The formulas for the Chern connection and curvature of conformal metrics are less complicated than the equivalent formulas for Riemannian metrics. We have
$$
h'(D_{h'} s, t) + h'(s, D_{h'} t)
= d h'(s, t)
= df \otimes h'(s, t) + h'(D_h s, t) + h'(s, D_h t).
$$
Writing \(df = \partial f + \bar\partial f\) we see that the Chern connection of \(h'\) is
$$
D_{h'} s = \partial f \otimes s + D_h s.
$$
Using the expressions for the covariant exterior derivative, we also have
\begin{align*}
D_{h'}^2 s
&= D_{h'}(\partial f \otimes s + D_h s)
\\
&= d(\partial f) \otimes s - \partial f \wedge D_{h'} s + D_{h'}(D_h s)
\\
&= -\partial\bar\partial f \otimes s - \partial f \wedge (\partial f \otimes s + D_h s) + \partial f \wedge D_h s + D_h^2 s
\\
&= -\partial\bar\partial f \otimes s + D_h^2 s.
\end{align*}

The Chern curvature of \(h'\) is thus
$$
D^2_{h'} = -\partial\bar\partial f \otimes \id_E + D^2_h.
$$

It's fun to work out what this gives for metrics conformal to a flat metric. We'll do that later when we study a particular metric on the \hyperref[sec:org8f5818e]{Hopf manifold}.

\subsubsection{Conformal to Kahler is not Kahler}
\label{sec:org7b1cfdf}

It's worth mentioning that if \(h\) is a Kahler metric on a manifold of dimension \(n > 1\) and \(f\) is non-constant, then \(h'\) is not a Kahler metric. The reason is that if \(\omega\) is the symplectic form associated to \(h\), then the symplectic form of \(h'\) is \(e^f \omega\) and
$$
d(e^f \omega) = \omega \wedge (e^f df)
$$
and the linear morphism from one- to three-forms defined by wedging with the symplectic form \(\omega\) is injective. (The proof reduces to linear algebra by a calculation in local coordinates, either Darbeaux ones or in holomorphic ones that are orthonormal at a point. The hard Lefschetz theorem generalizes this to cohomology.)

\subsection{Flat curvature}
\label{sec:org504b250}

Open set in C\textsuperscript{n}, complex torus. Gauss--Manin connection.

Mention variation of Hodge structures and the intersection form in middle cohomology? It's not positive-definite, but it is non-degenerate and that's enough to define a Chern connection and curvature form.

\subsection{Projective space}
\label{sec:orgcfabeed}

Might be fun to do in different ways. Local coordinates, more globally.

\subsection{Poincaré half-plane}
\label{sec:orgfbd87cb}

\subsection{Higher-dimensional hyperbolic metrics}
\label{sec:org21fa1aa}

\subsection{Grassmannian}
\label{sec:org34425b6}

\subsection{Flag manifold}
\label{sec:orga0ef6a0}

These are Kahler. There is a "natural" non-Kahler metric on them.

\subsection{Hopf manifold}
\label{sec:org8f5818e}

Let \(\lambda \in \kk C\) be a complex number such that \(0 < |\lambda| < 1\). The \emph{Hopf manifold} is the quotient
$$
X := (\kk C^n \setminus \{0\}) / \Gamma,
$$
where \(\Gamma \cong \kk Z\) is the group generated by \(\lambda\) that acts by
$$
\lambda \cdot (z_1, \ldots, z_n) = (\lambda z_1, \ldots, \lambda z_n).
$$
The Hopf manifold is compact and is diffeomorphic to \(S^{2n-1} \times S^1\). In particular,
$$
H^2(X, \kk C) \cong H^2(S^{2n-1}, \kk C) \oplus H^1(S^{2n-1}, \kk C) \otimes H^1(S^1, \kk C) = 0,
$$
so it is not Kahler.

Let \(\pi : \kk C^n \setminus \{0\} \to X\) be the projection. If \(\omega\) is a Hermitian metric on \(X\), then its pullback \(\pi^*\omega\) is a Hermitian metric on \(\kk C^n \setminus \{0\}\) that is invariant under the action of \(\Gamma\). If we write it as \(\pi^*\omega = \sum_{j,k} a_{jk}(z) \tfrac{i}{2} dz_j \wedge d\bar z_k\), then the smooth functions \(a_{jk}\) must satisfy
$$
a_{jk}(\lambda z) = \frac{1}{|\lambda|^2} a_{jk}(z).
$$
We can pick one such metric to inspect; we'll choose \(\omega = \frac{1}{\|z\|^2} \omega_{\mathrm{std}}\), that is, a metric that is conformal to the standard metric on \(\kk C^n \setminus \{0\}\). The Ph.D. thesis \cite{istrati:tel-02156198} has a nice discussion of the history of these metrics on the Hopf manifold.

As the metric is conformal to a Kahler metric, and the conformal factor is non-constant, the metric is not Kahler. (We already knew this because \emph{no} metric on the Hopf manifold is Kahler, but it's nice to check.)


\subsubsection{Curvature tensor}
\label{sec:org96d544d}

We've \hyperref[sec:org65fcbad]{already computed} the curvature of a conformal metric, so we know the curvature form of this metric is
$$
D^2 s = i\partial\bar\partial \log \|z\|^2 \otimes s.
$$
Let's compute this and express the curvature tensor of the metric. We have
$$
\bar\partial \log \|z\|^2
= \frac{1}{\|z\|^2} \sum_{k=1}^n z_k \bar\partial z_k
$$
and
$$
i\partial\bar\partial \log \|z\|^2
= - \frac{i}{\|z\|^4} \biggl(\sum_{j=1}^n \bar z_j \partial z_j \biggr) \wedge \biggl(\sum_{k=1}^n z_k \bar\partial z_k\biggr) + \frac{1}{\|z\|^2} \sum_{j=1}^n i\partial z_j \wedge \bar\partial z_j.
$$
We can try to write this in a more global way by looking at the Euler vector field
$$
\xi = \sum_{j=1}^n z_j \frac{\partial}{\partial z_j}.
$$
Then
$$
i\partial\bar\partial \log \|z\|^2
= -\frac{4}{\|z\|^4} \langle \cdot, \xi \rangle \wedge \langle \xi, \cdot \rangle + \frac{2}{\|z\|^2} \langle \cdot, \cdot \rangle.
$$
The curvature tensor of the metric \(h\) on the Hopf manifold is then
$$
R(\alpha,\beta,\gamma,\delta)
= 2 h(\alpha, \beta)\,h(\gamma, \delta) - 4 h(\alpha, \xi)\, h(\xi, \beta)\,h(\gamma, \delta).
$$
We note that it has the expected conjugate symmetries, that is, that \(R(\beta, \alpha, \delta, \gamma) = \overline{R(\alpha, \beta, \gamma, \delta)}\), but \(R(\gamma, \delta, \alpha, \beta) \not= R(\alpha, \beta, \gamma, \delta)\) like it would if this were the curvature tensor of a Kahler metric.


\subsubsection{Holomorphic sectional curvature}
\label{sec:org6471503}

The holomorphic sectional curvature of the Hopf manifold is
\begin{align*}
H(\alpha)
= \frac{1}{\|\alpha\|^4} R(\alpha, \alpha, \alpha, \alpha)
&= \frac{1}{\|\alpha\|^4} \bigl( 2 \|\alpha\|^4 - 4 |h(\alpha, \xi)|^2 \|\alpha\|^2 \bigr)
\\
&\geq \frac{1}{\|\alpha\|^4}\bigl( 2 \|\alpha\|^4 - 4 \|\alpha\|^4\bigr)
= -2,
\end{align*}
where the inequality is by Cauchy--Schwarz and that \(\|\xi\|^2 = 1\) for the metric \(h\). By dropping the negative term we get an easy upper bound as well, and see that
$$
-2 \leq H(\alpha) \leq 2.
$$
Both of these bounds are sharp, as one is acheived when \(\alpha\) is a multiple of \(\xi\) and the other when \(\alpha\) is orthogonal to \(\xi\). By the intermediate value theorem, the holomorphic sectional curvature also assumes any value in between the two.


\subsubsection{Ricci tensors}
\label{sec:org3942125}

The curvature tensor can be contracted in three ways to obtain a \((1,1)\)-form. On a Kahler manifold, all three ways give the same result; on a non-Kahler manifold they may not.

The easiest of these to compute for us is the one given by taking the traces of the endomorphisms in the curvature form. As those endomorphisms are the identity here, we find that
$$
r_1(\alpha, \beta)
= 2n \bigl( h(\alpha, \beta) - 2 h(\alpha, \xi) \, h(\xi, \beta) \bigr).
$$
This is the same as we obtain by contracting the curvature tensor along \(\delta\) and \(\gamma\). As before, Cauchy--Schwarz gives us the estimates
$$
-2n
\leq \frac{r_1(\alpha, \alpha)}{\|\alpha\|^2}
\leq 2n
$$
which are sharp under the same conditions as before. As expected, this form is neither positive- nor negative-definite: It is the curvature form of the anti-canonical bundle on \(X\) when equipped with the metric induced by \(\omega\). If it were positive or negative, \(X\) would be projective.

Our second contraction is along \(\alpha\) and \(\beta\). The only relevant part of the curvature tensor we don't know how to contract is \(h(\alpha, \xi)h(\xi, \beta)\). Let \((\zeta_1, \ldots, \zeta_n)\) be a local holomorphic frame that's orthonormal at a point \(z\) we care about. We have
$$
\sum_{j=1}^n h(\zeta_j, \xi) h(\xi, \zeta_j) = h(\xi, \xi) = 1
$$
as \(\xi = \sum_{j=1}^n h(\xi, \zeta_j) \zeta_j\) and \(h(\xi,\xi) = 1\).
Then
$$
r_2(\gamma, \delta)
= 2n h(\gamma, \delta) - 4 h(\gamma, \delta)
= 2(n-2) h(\gamma, \delta).
$$
This form is not only different from \(r_1\) but it has a sign: when \(n = 2\) it is identically zero, and when \(n > 2\) it is positive-definite.

The third contraction is along \(\beta\) and \(\gamma\). We let \((\zeta_1, \ldots, \zeta_n)\) be a local holomorphic frame that's orthonormal at a point \(z\) as before. We have
$$
\sum_{j=1}^n h(\alpha, \zeta_j) h(\zeta_j, \delta)
= h(\alpha, \delta).
$$
Also
$$
\sum_{j=1}^n h(\alpha, \xi) h(\xi, \zeta_j) h(\zeta_j, \delta)
= h(\alpha, \xi) \sum_{j=1}^n  h(\xi, \zeta_j) h(\zeta_j, \delta)
= h(\alpha, \xi) h(\xi, \delta).
$$
Together, we get
$$
r_3(\alpha, \delta)
= 2 h(\alpha, \delta) - 4 h(\alpha, \xi) h(\xi, \delta)
= \frac{1}{n} r_1(\alpha, \delta).
$$

\subsubsection{Scalar curvature}
\label{sec:orgf9212d2}

We can contract any of the Ricci-forms we got to obtain the scalar curvature of the Hopf manifold. Picking the first two, we get
$$
s = 2n(n-2),
$$
while picking the third gives \(1/n\) times that. We see that the Hopf surface has identically zero scalar curvature, while the higher-dimensional Hopf manifolds have positive constant scalar curvature.


\subsection{Iwasawa manifold}
\label{sec:orgd67c2ff}

\subsection{Direct image curvatures}
\label{sec:org5619e67}

Weil--Peterson, maybe.

Berndtsson has papers from about ten years ago we could look at. Cao et al took that further recently, but that's probably too advanced for what we want to do.

\subsection{Algebraic curvature tensors}
\label{sec:orga0722e9}

Dimension of their subspace.
Kobayashi-Nomitzu products.
Holomorphic bisectional curvature determines curvature tensor.


\subsection{Intuitive explanation for curvature forms}
\label{sec:org2b43ecb}

Wikipedia has a handwavy explanation of curvature as what happens when we parallel transport a section along a parallelogram. Can we make this precise?


\subsection{Riemann surfaces}
\label{sec:org776713b}

Every Riemann surface is Kahler. The curvature tensor collapses to the Ricci tensor (or the scalar curvature even?).


\subsection{Hermitian metrics}
\label{sec:orgad8dbb1}

Discuss extra symmetries we get when \(E = T_X\).
Talk about torsion tensor and how it's related to the Kahler form
Define sectional curvatures, Ricci tensors, scalar curvature.

Discuss Kahler metrics. Extra curvature symmetries. All Ricci forms are equal.


\bibliographystyle{plain}
\bibliography{sgct}
\end{document}
