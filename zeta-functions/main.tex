\documentclass[10pt]{article}

\usepackage{tgpagella}
\linespread{1.1}
\usepackage[utf8]{inputenc}
\usepackage[T1]{fontenc}

\usepackage[normalem]{ulem}
\usepackage{textcomp}
\usepackage[colorlinks=true, citecolor=red]{hyperref}
\usepackage{url}
\usepackage{tikz-cd}

\usepackage{amsmath}
\usepackage{amssymb}
\usepackage{amsthm}

\author{Alan David Thomas}
\date{\today}
\title{Zeta functions: an introduction\\to algebraic geometry}

\newtheorem{theo}{Theorem}[subsection]
\newtheorem{prop}[theo]{Proposition}
\newtheorem{lemm}[theo]{Lemma}
\newtheorem{coro}[theo]{Corollary}
\newtheorem{rh}[theo]{The Riemann Hypothesis}
\newtheorem{rhqff}[theo]{Riemann Hypothesis for quadratic function fields}
\theoremstyle{definition}
\newtheorem{defi}[theo]{Definition}
\newtheorem{exam}[theo]{Example}
\newtheorem{rema}[theo]{Remark}

\newcommand{\kk}[1]{\mathbb{#1}}
\newcommand{\cc}[1]{\mathcal{#1}}

\def\eps{\varepsilon}
\def\empty{\varnothing}
\def\qandq{\quad\text{and}\quad}

\def\ov#1{\overline{#1}}

\def\ldb{[\mkern-2mu[}
\def\rdb{]\mkern-2mu]}
\def\ldp{(\mkern-2.5mu(}
\def\rdp{)\mkern-2.5mu)}

\def\NN{\mathbf{N}}
\def\ZZ{\mathbf{Z}}
\def\QQ{\mathbf{Q}}
\def\RR{\mathbf{R}}
\def\CC{\mathbf{C}}
\def\HH{\mathbf{H}}
\def\FF{\mathbf{F}}

\DeclareMathOperator{\chr}{char}
\DeclareMathOperator{\Gal}{Gal}
\DeclareMathOperator{\re}{Re}
\newcommand{\tri}{\mathbin{\triangleleft}}

\def\qw#1{`#1'}

\def\fnon{See Cashwell and Everett~\cite{bib:31}, Gilmer~\cite{bib:63}.}
\def\fntw{See Sansone and Gerretsen, vol. 1~\cite{bib:162}, \S7.2.1, p.~353.}
\def\fnth{Ibid., \S7.1.3, p.~351.}
\def\fnfo{Ibid., \S7.1.2, p.~351.}
\def\fnfi{Two holomorphic functions defined in half-planes are \emph{equivalent} if they agree in some half-plane, and a \emph{germ} is an equivalence class.}
\def\fnsi{This may be proved directly, as in Hardy \& Wright~\cite{bib:86}, \S17.1, p.~245.}
\def\fnse{See Titchmarsh~\cite{bib:194}, \S9.42, p.~300.}
\def\fnei{See Titchmarsh~\cite{bib:195}, Theorem 2.1, p.~13.}
\def\fnni{Ibid., pp.~1,2, or Hardy \& Wright~\cite{bib:86}, Theorem 280, p.~246.}

\def\fnonze{See Sansone and Gerretsen, vol. 1 \cite{bib:162}, \S3.11.1, p.~156.}
\def\fnonon{In his proof of the prime number theorem (see Riemann~\cite{bib:155} p.~148), Riemann says of the function $\xi(t) = \Gamma(s) (s-1) \zeta(s) \pi^{-t/2}$ where $s = \frac12 + it$ ``Man findet nun in der That etwa so viel reele Wurzeln innerhalb dieser Genzen und es ist sehr wahrscheinlich dass alle Wurzeln reell sind. Hiervon w\"are allerdings eine strenge Beweis zu wanschen; ich habe indess die Aufsuchung desselben nach einigen fl\"uchtigen vergeblichen Veruschen vorl\"aufig bei Seite gelassen, da er f\"ur n\"achsten Zweck meiner Untersuchung entbehrlich sind.''}
\def\fnontw{See Hardy and Wright \cite{bib:86}, Chapter 17 for further details and examples.}
\def\fnonth{For the definition and properties of integral closure, see Atiyah and Macdonald \cite{bib:17}, Chapter 5, pp.~59--73.}
\def\fnonfo{See Cohn \cite{bib:36}, Theorem 3, p.~96 for example.}
\def\fnonfi{See Cohn \cite{bib:36}, Chapter 6, \S9, p.~110, or Niven and Zuckermann \cite{bib:148}, Chapter 7, in particular \S7.8, pp.~175--179.}
\def\fnonsi{See Atiyah and Macdonald \cite{bib:17}, Chapter 8, pp.~93--98.}
\def\fnonse{This zeta function was introduced by Dedekind in 1871 (see Dedekind \cite{bib:37}), who proved convergence in $H_1$. Landau proved that $\zeta$ extends to a meromorphic function in the half plane $H_a$ where $a = 1 - \deg(K / \QQ)^{-1}$ (see Landau \cite{bib:116}), and Hecke finally proved the extension to $\CC$, see Hecke \cite{bib:94}. See Narkiewicz \cite{bib:146}, Chapter VII, Prop. 7.1, p.~293 and Theorem 7.1, p.~296.}
\def\fnonei{See Narkiewicz \cite{bib:146}, Chapter VI, Prop. 7.1, p.~293. The class number relation was first proved by Dedekind (see Dedekind \cite{bib:37}).}
\def\fnonni{This was first proved by Hecke in Hecke \cite{bib:94}.}

\def\fntwze{See Ribenboim \cite{bib:154}, pp.~97--98.}
\def\fntwon{See Artin \cite{bib:6}.}
\def\fntwtw{It is easy to show from the equation $1 + x = (\sum \binom{1/2}{r} X^r)^2$.}
\def\fntwth{See Artin \cite{bib:6}, pp.~196--197.}
\def\fntwfo{Ibid., p.~180 (imaginary case) and p.~194 (real case).}
\def\fntwfi{See Sansone and Gerretsen \cite{bib:162}, \S4.1, pp.~175--176.}
\def\fntwsi{See Artin \cite{bib:6}, p.~207.}
\def\fntwse{Ibid., p.~209.}
\def\fntwei{Ibid., p.~216.}
\def\fntwni{Ibid., p.~2223}

\def\fnthze{See \S\ref{ch:4.7} and Armitage \cite{bib:5}.}
\def\fnthon{This follows from a simple calculation using the formula.}
\def\fnthtw{See Artin \cite{bib:6}, equation (5), p.~209.}
\def\fnthth{See example \ref{8.3.2}.}
\def\fnthfo{For the definition and properties of transcendence degree, see Winter \cite{bib:217}, p.~41.}
\def\fnthfi{See Weil \cite{bib:214}, Chapter III, \S2, Lemma 1, p.~48.}
\def\fnthsi{See Endler \cite{bib:58}, \S1.14 and \S1.15, p.~6.}
\def\fnthse{Ibid., \S2, p.~8.}
\def\fnthei{Ibid., \S1.5, p.~2.}
\def\fnthni{Immediate from the definition.}

\def\fnfoze{}
\def\fnfoon{}
\def\fnfotw{}
\def\fnfoth{}
\def\fnfofo{}
\def\fnfofi{}
\def\fnfosi{}
\def\fnfose{}
\def\fnfoei{}
\def\fnfoni{}

\def\fnfize{}
\def\fnfion{}
\def\fnfitw{}
\def\fnfith{}
\def\fnfifo{}
\def\fnfifi{}
\def\fnfisi{}
\def\fnfise{}
\def\fnfiei{}
\def\fnfini{}

\def\fnsize{}
\def\fnsion{}
\def\fnsitw{}
\def\fnsith{}
\def\fnsifo{}
\def\fnsifi{}
\def\fnsisi{}
\def\fnsise{}
\def\fnsiei{}
\def\fnsini{}


\begin{document}

\maketitle


\section*{Introduction}

The aim of this book is to explain the development of the statements and proofs of the Weil conjectures from the analogous results for the Riemann zeta functions, at the same time describing the transformation of classical algebraic number theory and algebraic geometry into the language of schemes.
In order to motivate the cohomological aspects and in particular the idea of an \'etale map, a certain amount of classical algebraic geometry (over the complex numbers) and associated complex manifold theory is included, not only to provide an insight to the more abstract methods, but also to be used in the various comparison techniques.

All these methods and results are contained in existing literature, which to the non-expert may seem both impenetrable and disjointed.
This text is designed to be read by any mathematician who is conversant with the basics of commutative algebra, as in Atiyah \& Macdonald~\cite{bib:17}, field and Galois theory, as in Winter~\cite{bib:217}, and who has some familiarity with manifolds and algebraic topology (see for example Matsushima~\cite{bib:138} and Spanier~\cite{bib:183}) and categories as in Mitchell~\cite{bib:141}.

It would be impossible to cover the contents of this book with full details in a small number of pages, and so footnotes have been included to give explicit references to the relevant information.
I have tried to choose the facts which have been included to give the reader the flavour of the subject.
Following these footnotes, the reader will soon realize the volume of mathematics which has been omitted:
I should like to think that the value of this book lies as much in what is omitted as in what is included.
The decision on the precise contents was a most difficult one.
Probably the most important omissions are the details and precise statement of the Riemann--Roch theorem and its implications in the context of these notes.
The reason for not including this is that, in order to convey the full weight of this theorem, one would have to discuss Jacobians in more detail, and the associated K-theories, characteristic classes and index theorems which would have been impossible in the short number of pages.
This possibly will be the contents of a companion volume.

We start with Riemann's zeta function and its properties, especially those which express number theoretic properties of the ring of integers (which after all was the reason Riemann introduced it).
This immediately generalises to the zeta-function of an algebraic number field, which contains similar information about the field, or more precisely, about its ring of (algebraic) integers.

Next we discuss the contents of Artin's thesis in which he generalises these classical ideas, replacing by the euclidean domain $\FF_p[X]$ of polynomials over the finite field of order $p$, and considering quadratic extensions of the associated field of fractions.
This analogous situation has much in common with the classical case and there is a natural definition of a zeta-function.
Artin observed in the many cases he calculated that the analogue of the Riemann hypothesis was true.

In order to present these two parallel theories in a unified way, we introduce (real) valuations, global fields and the associated functional analysis of ad\`eles and id\`eles.

In Chapter~\ref{ch:6} we show how a global field of non-zero characteristic can be interpreted geometrically as a curve over a finite field.
This then provides a link between number theory and geometry.
Replacing the finite field by the field of complex numbers we have the classical algebraic theory, and from this the analytic theory, of Riemann surfaces, which we discuss in Chapter~\ref{ch:7}.
The analytic theory of course depends on using the \qw{usual} or \qw{strong} topology on the field of complex numbers which is not defined algebraically.
We find ourselves in the position of being able to prove certain algebraic/geometric results analytically but not algebraically, which is unsatisfactory for two reasons.
Firstly for aesthetic reasons, algebraic results should have algebraic proofs, and secondly, such analytic methods do not generalise to the study of curves over other fields.
In this context, the Appendix~\ref{ch:7.8} is of great significance, as it contains the essential ideas behind the effectiveness and importance of the \'etale topology, and the importance of elliptic curves or more generally abelian and Jacobian varieties, which we discuss in Chapter~\ref{ch:8}.
This Chapter ends with Hasse's proof of the Riemann hypothesis for curves of genus $1$.

Historically, this places us at around 1936, when the curve is still identified with its field of functions and is of no geometrical significance.
In order to prove the Riemann hypothesis for global fields of non-zero characteristic, Weil introduced (geometrically) products of curves which lead on to the abstract theory of algebraic varieties over arbitrary fields.

Chapter~\ref{ch:9} starts with the definition of an affine variety, and the concept of smoothness is shown to be an appropriate generalisation of integral closure.
In order to piece together affine varieties to obtain the general variety, we introduce the language of sheaves, first used for this purpose by Serre~\cite{bib:166}.
This is unfortunately rather technical, but simultaneously enables us to define the concept of a manifold, complex manifold and variety (and, in Chapter~\ref{ch:12}, of a scheme).
In \S\ref{ch:9.11} we describe briefly the properties of cycles on a variety, whose importance becomes apparent in Chapter~\ref{ch:10}.
At this point we are in a position to state the Weil conjectures, all of which have now been proved using \'etale topology and cohomology (the fundamental properties of which are contained in the 1583 pages of SGA~4, see \cite{bib:14}).

Besides stating these conjectures, Weil suggested that they should admit a cohomological proof and outlined the lines such a proof might take.
Such a cohomology appeared around 1959/60 and gradually the various necessary properties were established.
Working analytically, Dwork proved the first of these conjectures (rationality) in 1960, and in 1962 Lubkin proved cohomologically all the conjectures (except the Riemann hypothesis) under certain restrictions, although his proof referred to other theorems which had themselves not been proved.

In order to motivate these cohomological attacks on the problems, we describe in Chapter~\ref{ch:10} how one can prove the analogous conjectures (except the Riemann hypothesis) for smooth complex varieties, by giving them the usual topology and then applying \v{C}ech cohomology methods.
The fundamental result is the Lefschetz fixed point theorem, which connects the number of fixed points of a continuous map with its effect in cohomology, and which is used along with properties of cycles and their cohomological invariants.

Chapter~\ref{ch:10} continues this analogy by showing that the statement and proof of the Riemann hypothesis is quite a different problem, needing the force of Hodge theory of harmonic integrals to finally overcome it.

Hopefully by now the reader has reached Chapter~\ref{ch:10}, in which the language of schemes, Grothendieck topologies and \'etale maps are introduced, motivated by the earlier chapters, and we finish with a brief account of the proofs of the Weil conjectures.

Looking back, the reader should then realise that the Riemann hypothesis and the Weil conjectures, apart from being of intrinsic interest and direct application, have generated through the attempts at their verification a considerable amount of other mathematics which we would perhaps otherwise not have discovered.

To some extent then, this book is like a jigsaw puzzle.
At first selected pieces fit together.
As one progresses one finds individual pieces which do not seem to belong anywhere, but the further one goes, the more important these pieces become, until finally they all fit together to form the overall picture.
Of course one might say this is not the way to present a mathematical topic; a mathematical treatise should be linearly ordered.
While this is possible even desirable for \qw{elementary} branches (i.e.,~those using a small vocabulary), it would seem to be impossible for the contents of these notes, if one wishes to provide the motivation and intuition.
The ordering here is loosely based on the chronology of the subject.
Certain questions may be partially answered in one section, only to be resurrected in a later section and tackled again with the aid of techniques developed in the meantime.
One final comment: because of this approach, this book should not be read once!


\subsection*{Notation}

The notation used in this book is either standard or explained in the text, in which case the reader is referred to the index.
The symbols $\NN, \ZZ, \QQ, \RR, \CC$ and $\HH$ denote (respectively) the natural, integral, rational, real, complex and quaternionic numbers, and $S^n$ denotes the $n$-sphere.

All rings are assumed to be commutative with an identity unless otherwise stated.
For any ring $R$ we write $R[X]$ and $R(X)$ for the rings of polynomial and rational functions of $X$ over $R$, and $R\ldb X \rdb$ and $R\ldp X \rdp)$ for the rings of power and (finite) Laurent series.
If $S$ is a subset of $R$, $\langle S \rangle$ denotes the ideal it generates.

For any field $K$ we write $\chr K$ for its characteristic, $K^a$ for its algebraic closure and $K^*$, $\mu(K)$, $\mu_n(K)$ for the groups of units, roots of unity, $n$-th roots of unity in $K$, and $\Gal L/K$ for the Galois group of the normal extension $L/K$.
$\FF_q$ denotes the finite field of order $q$.

Indexing sets, especially for $\sum$, $\prod$, $\varinjlim$ and $\varprojlim$ are often omitted when they are obvious or irrelevant from the context.



\section{Dirichlet series}
\label{ch:1}

\subsection{Dirichlet series}
\label{ch:1.1}

Let $\Omega$ denote the set of functions from $\NN$ to $\CC$.
If we define addition of such functions componentwise, and multiplication by the formula
\[
\alpha \cdot \beta(n)
= \sum_{ij=n} \alpha(i) \beta(j)
\quad
\text{for $\alpha,\beta \in \Omega$}
\]
then $\Omega$ is a commutative ring with $1$, which is called the \emph{ring of Dirichlet series} (over $\CC$).
(It is also called the ring of arithmetic functions; it is a unique factorisation domain, being isomorphic with a ring of power series.\footnote{\fnon})
If $\alpha$ is a Dirichlet series, we define its \emph{derivative} $\alpha' \in \Omega$ by the formula
\[
\alpha'(n) = -\alpha(n) \log n.
\]
The function $d : \Omega \to \Omega$, $\alpha \mapsto \alpha'$, is called \emph{differentiation}.

If $z$ is a complex number, let $\alpha_S(z)$ denote the series $\sum_n \alpha(n) / n^z$ and if this series is convergent, let $\widehat \alpha(z)$ denote its sum.

By abuse of notation, we shall identify $\alpha$ with $\alpha_S$.


\subsection{Convergence of Dirichlet series}
\label{ch:1.2}

For any real number $r$, let $H_r$ denote the open half-plane $\{ z \in \CC \mid \re z > r \}$.
Suppose that $\alpha$ is a Dirichlet series and $z_0$ is a complex number such that $\alpha_S(z_0)$ converges.
Let $r_0 = \re z_0$.
The series $\alpha_S(z)$ converges\footnote{\fntw} for any $z \in H_{r_0}$, and converges uniformly\footnote{\fnth} in any compact subspace of $H_{r_0}$.
Since the function $1/n^z$ is holomorphic in the complex plane, it follows that $\widehat\alpha(z)$ is holomorphic in $H_{r_0}$.
In, in addition, there is a complex number $z_1$ such that $\alpha_S(z_1)$ converges absolutely, then\footnote{\fnfo} $\alpha_S(z)$ converges absolutely for any $z \in H_{r_1}$ where $r_1 = \re z_1$.

Let $\Omega_a = \{ \alpha \in \Omega \mid \text{$\exists z \in \CC$ such that $\alpha_S(z)$ converges absolutely} \}$.
This is a subring of $\Omega$, called the \emph{ring of absolutely convergent Dirichlet series}.
Let $\Omega_0$ be the ring of (germs of)\footnote{\fnfi} holomorphic functions defined in some open half-plane, and let $j : \Omega_a \to \Omega_0$ be the function $j(\alpha) = \widehat\alpha$.
This function is a homomorphism, and it commutes with differentiation.
In fact it is a monomorphism, and the proof\footnote{\fnsi} depends on the following lemma\footnote{\fnse}:


\begin{lemm}
\label{1.2.1}
Let $x,c$ be positive real numbers.
Then
\[
\frac{1}{2\pi i} \int_{c - i \infty}^{c + i\infty} \frac{x^z}{z} dz
= \begin{cases}
1 & \text{if $x < 1$}
\\
0 & \text{if $x > 1$}.
\end{cases}
\]
\end{lemm}


Suppose then that $\widehat\alpha$ is absolutely convergent in $H_r$, and let $x,c$ be positive real numbers such that $c > r$ and $x$ is non-integral.
We see that
\[
\int_{c - i\infty}^{c + i\infty} \frac{\widehat\alpha(z) x^z}{z} dz
= \int_{c - i\infty}^{c + i\infty} \sum_n \frac{\widehat\alpha(n) x^z}{z n^z} dz
= \sum_n \alpha(n) \int_{c - i\infty}^{c + i\infty} \frac{(x/n)^z}{z} dz,
\]
the interchange of integration and summation being possible since the series $\alpha_S(z)$ is uniformly convergent.
Thus applying \ref{1.2.1} we obtain the equation
\[
\frac{1}{2\pi i} 
\int_{c-i\infty}^{c+i\infty} \frac{\widehat\alpha(z) x^z}{z} dz
= \sum_n^{[x]} \alpha(n),
\]
which is called \emph{Perron's formula}.
This formula shows that the holomorphic function $\widehat\alpha$ determines the arithmetic function $\alpha$, so that $j : \Omega_a \to \Omega_0$ is indeed injective.
If $f$ is a function holomorphic in some half plane and $f = j(\alpha)$ for some (unique) $\alpha$, then we say that $\alpha$ is the \emph{Dirichlet series of $f$}.


\subsection{Examples of Dirichlet series}
\label{ch:1.3}

The canonical example of a Dirichlet series is the Riemann zeta-function.
This has Dirichlet series $\sum 1/n^z$, and is denoted by $\zeta(z)$.
This Dirichlet series converges absolutely in $H_1$, but does not converge at $z = 1$.
It is well-known\footnote{\fnei} that the $\zeta$-function extends to a meromorphic function on the complex plane with a simple pole (residue 1) at $z = 1$.
It satisfies the \emph{functional equation}:
\begin{equation}
\label{1.3.1}
\zeta(z) = 2 (2\pi)^{z-1} \Gamma(1-z) \sin(\pi z/2) \zeta(1-z),
\end{equation}
which by means of the equations
\begin{align*}
\Gamma(z) \Gamma(1-z)
&= \pi \operatorname{cosec} (\pi z),
\\
\Gamma(z) \gamma(z + \tfrac 12) 
&= 2^{1-2z} \pi^{1/2} \Gamma(2z)
\end{align*}
may be rewritten
\[
\pi^{-z/2} \Gamma(z/2) \zeta(z)
= \pi^{-(1-z)/2} \Gamma((1-z)/2) \zeta(1-z).
\]
The following proposition is generic in these notes\footnote{\fnni}.


\begin{prop}
\label{1.3.2}
For $z \in H_1$, the infinite product
\begin{equation}
\label{1.3.2.1}
\prod_{p \in \NN,\text{ $p$ prime}}
\frac{1}{1-p^{-z}}
\end{equation}
converges to $\zeta(z)$.
Moreover, this infinite product converges absolutely in $H_1$, and uniformly on compact sets.
\end{prop}

The fact that the value of this infinite product is $\zeta(z)$ is essentially that (formally)
\[
\prod_{p \in \NN,\text{ $p$ prime}}
\frac{1}{1-p^{-z}}
= \prod_{p \in \NN,\text{ $p$ prime}}
\sum_{k=1}^\infty p^{-kz}
= \sum_{n=1}^\infty n^{-z}
\]
since every positive integer is uniquely expressible as a product of positive prime integers.
To convert this into an analytic argument, one has to work with finite sets of primes and take the limit.
This proposition is thus an analytic statement of unique factorisation in the ring of integers.


\begin{coro}
\label{1.3.3}
The set of prime integers is infinite.
\end{coro}

\begin{proof}
If there were only a finite set of primes, then the product \eqref{1.3.2.1} would be a finite product of functions, each holomorphic in $H_0$, and so would define a holomorphic continuation of the $\zeta$-function to $H_0$, which is impossible because of the pole at $z = 1$.
\end{proof}


Since $(1-p^{-z})^{-1}$ does not vanish in $H_1$, it follows from Hurwitz's theorem\footnote{\fnonze} and \ref{1.3.2} that the $\zeta$-function has no zeros in $H_1$.
From the functional equation~\eqref{1.3.1}, it follows that the only zeros of the $\zeta$-function in the region $\{ z \mid \re z < 0 \}$ are the zeros of $\sin(\pi z / 2)$ which occur at the points $z = -2k$ for $k \in \NN$.
These zeros are called the \emph{trivial zeros}.
The remaining zeros thus lie in the \qw{critical strip} $\{z \in \CC \mid 0 \leq \re z \leq 1 \}$, and are called the \qw{non-trivial zeros}.


\begin{rh}
\label{1.3.4}
The zeros in the critical strip all lie on the line $\re z = \frac12$.
\end{rh}


Since first stated by Riemann\footnote{\fnonon}, mathematicians throughout the world have been unable to prove or disprove the validity of this hypothesis.

Many important examples of Dirichlet series are obtained from the $\zeta$ function, and reflect various arithmetic properties of the integers, for example:
\begin{equation}
\label{1.3.5}
\zeta^{-1}(z) = \sum \mu(n) / n^z,
\end{equation}
where $\mu$ is the M\"oebius function, i.e., $\mu(1) = 1$, $\mu(n) = 0$ if $n$ is not square-free, and $\mu(n) = (-1)^k$ if $n$ is the product of $k$ distinct primes;
\begin{equation}
\label{1.3.6}
\zeta^2(s) = \sum d(n) / n^z,
\end{equation}
where $d$ is the divisor function, i.e., $d(n)$ is the number of positive divisors of $n$;
\begin{align}
\label{1.3.7}
\zeta'(z) &= \sum -\log n / n^z
\quad\text{(cf. \ref{ch:1.1})};
\\
\label{1.3.8}
\zeta'(z) / \zeta(z) &= \sum \Lambda(n) / n^z,
\end{align}
where $\Lambda(n) = \log n$ if $n$ is a power of the prime $p$ and $0$ otherwise;
\begin{equation}
\label{1.3.9}
\zeta(z-1)/\zeta(z) = \sum \phi(n) / n^z,
\end{equation}
where $\phi$ is the Euler function.

In all the above examples\footnote{\fnontw} except \eqref{1.3.9}, The Dirichlet series are absolutely convergent in $H_1$.
The Dirichlet series in \eqref{1.3.9} is absolutely convergent in $H_2$.





\section{Classical number theory}
\label{ch:2}

Let $\ZZ$ denote the integers, and $\QQ$ the rationals (its field of fractions).
If $K$ is a finite algebraic extension of $\QQ$, then the integral closure\footnote{\fnonth} of $\ZZ$ in $K$, denoted by $A$, is called the \emph{ring of integers} of $K$.
Such rings have many properties in common with the ring $\ZZ$, as we now indicate.
Various proofs and definitions have been omitted, but will be given in more generality.

We start by remarking that if $r = \dim K/\QQ$, then $A$ is a free abelian group of rank $r$.


\subsection{Units}
\label{ch:2.1}

Let $U$ denote the group of units in $A$.
The subgroup of elements of finite order is clearly $\mu(K)$, the group of roots of unity in $K$, which is a finite cyclic group.
The quotient $U / \mu(K)$ is a free abelian group.
Now there are precisely $r$ isomorphisms of $K$ into $\CC$, of which say $r_1$ map into $\RR$, and the remainder occur in $r_2$ conjugate pairs, so that $r = r_1 + 2 r_2$.
The rank of $U / \mu(K)$ is $r_1 + r_2 - 1$.
Putting these results together we have \emph{Dirichlet's Unit Theorem}.


\begin{theo}
\label{2.1.1}
\[
U \cong \ZZ^{r_1 + r_2 - 1} + \mu(K).
\]
\end{theo}

A unit which forms part of a basis for the free part of $U$ is called a \emph{fundamental unit}.

In the case that $K = \QQ(\sqrt d)$, where $d \in \ZZ$ is square-free, the group $U$ is easy to describe.
If the group of roots of unity in $\QQ(\sqrt d)$ has order $m$, then $\QQ(\sqrt d)$ contains a subfield isomorphic with $\QQ(\omega)$, where $\omega = \exp(2 \pi i / m)$.
Since $\det \QQ(\omega) / \QQ = \phi(m)$, where $\phi$ is the Euler function, and $\deg \QQ(\sqrt d) / \QQ = 2$, we must have $\phi(m) = 1$ or $2$, which can only happen if $m = 1,2,3,4$ or $6$.
Since $-1 \in U$, and $(-1)^2 = 1$, $m$ is necessarily even.
In fact one can show\footnote{\fnonfo} that $m = 2$ except when $d = 1$ ($m = 4$) or $d = -3$ ($m = 6$).

If $d > 0$, then $r_1 = 2, r_2 = 0$ so $U$ has rank $1$.
If $d < 0$, then $r_1 = 0, r_2 = 1$ so $U$ is finite and cyclic.


\begin{exam}
\label{2.1.2}
If $K = \QQ(\sqrt 2)$, then $\mu(K) = \{-1,1\}$ so $U \cong \ZZ \oplus \ZZ / 2 \ZZ$.
A fundamental unit is $1 + \sqrt 2$, so any unit is uniquely expressible in the form $\pm(1 + \sqrt 2)^n$ for some $n \in \ZZ$.
\end{exam}


\begin{exam}
\label{2.1.3}
If $K = \QQ(\sqrt{94})$, the only fundamental units are the numbers $\pm 2143295 \pm 221064 \sqrt{94}$.
\end{exam}

There is no formula for the fundamental units even in the quadratic case.
They may be obtained by means of continued fractions.\footnote{\fnonfi}


\begin{exam}
For $p$ a prime integer, let $\omega = \exp(2 \pi i / p)$, and $K = \QQ(\omega)$.
The ring of integers $A$ is $\ZZ[\omega]$ and $r_1 = 0, r_2 = (p-1)/2$ for $p \geq 5$.
The element $\xi_t = \omega^t + \omega^{-t} = 2\cos(2 \pi t / p)$ for $t \in \ZZ$ is a unit.
In general a fundamental set of units is not known.
\end{exam}


\subsection{Ideal theory}
\label{ch:2.2}

The ring $A$ is Noetherian, and every non-zero prime ideal is maximal.
Since $A$ is by definition integrally closed, $A$ is\footnote{\fnonsi} a \emph{Dedekind domain}, and so every non-zero ideal is uniquely expressible as a product of prime ideals.
This is the analogue of unique factorization.
However, in general not every ideal is principal, and the \emph{ideal class group} measures this deficiency.
This group is trivial if and only if $A$ is a principal ideal domain.
In general (and this is an important result) this group is finite.
Its order is usually denoted by $h = h(K)$, and is called the \emph{class-number} of $K$.

If $I$ is a non-zero ideal of $A$, then the quotient ring $A / I$ is finite.
Let $N(I)$ denote its order, which is called the \emph{norm} of $I$.
The norm is multiplicative, i.e., $N(IJ) = N(I) N(J)$.
Moreover, for any positive integer $n$, the set $\{ I \tri A \mid N(I) = n\}$ is finite.


\subsection[The zeta function of K]{The zeta function of $K$}
\label{ch:2.3}

Let $\zeta(z,K)$ denote the Dirichlet series
\begin{equation}
\label{2.3.1}
\sum_{I \tri A, I\not=0} \frac{1}{N(I)^z}
= \sum_n \biggl(
\sum_{I \tri A, N(I)=n} \frac{1}{n^z}
\biggr).
\end{equation}

By analogy with the classical case, there is the following proposition:\footnote{\fnonse}


\begin{prop}
\label{2.3.2}
The Dirichlet series $\zeta(z,K)$ converges absolutely in $H_1$, and extends to a meromorphic function on the complex plane with a simple pole at $z = 1$.
The infinite product
\begin{equation}
\label{2.3.2.1}
\prod_{P \in \max(A)} \frac{1}{1 - N(P)^{-z}}
= \prod_n\biggl(
\prod_{P \in \max(A), N(P)=n} \frac{1}{1-n^{-z}}
\biggr)
\end{equation}
converges in $H_1$ to $\zeta(z,K)$.
Moreover, this infinite product converges absolutely in $H_1$, uniformly on compact subsets.
\end{prop}


As in \ref{1.3.2} the proof of \ref{2.3.2} depends essentially on unique factorisation.


\begin{rema}
\label{2.3.3}
If $P$ is a maximal ideal of $A$, then $N(P)$ is a power of the prime $p$, where $p$ generates the ideal $\ZZ \cap P$ in $\ZZ$.
Thus the infinite product \eqref{2.3.2.1} is indexed by prime powers.
\end{rema}

The following theorem\footnote{\fnonei} relates the behaviour of $\zeta(z,K)$ near its pole to the field $K$.


\begin{theo}
\label{2.3.4}
The residue of $\zeta(z,K)$ at the pole $z = 1$ is
\begin{equation}
\label{2.3.4.1}
\frac{2^{r_1} (2\pi)^{r_2} h R}{|\mu(K)| \sqrt D}
\end{equation}
where $h$ is the class number, $r_1$ and $r_2$ are as defined in \ref{ch:2.1}, $D$ is the discriminant and $R$ the regulator of $K$.
\end{theo}

This theorem and the two previous propositions indicate how the analytic properties of $\zeta(z,K)$ capture algebraic properties of $K$ and its ring of integers.
Other Dirichlet series and meromorphic functions may be obtained from $\zeta(z,K)$ as in \S\ref{ch:1.3}.


\subsection{The functional equation}
\label{ch:2.4}

It may be shown that the $\zeta$-function of $K$ satisfies\footnote{\fnonni} the functional equation:
\begin{equation}
\label{2.4.1}
\zeta(z,K)
= |D|^{1/2 - z}
\biggl(
\frac{\pi^{z-1/2} \Gamma(1/2 - z/2)}{\Gamma(z/2)}
\biggr)^{r_1}
\biggl(
\frac{(2\pi)^{2z-1} \Gamma(1-z)}{\Gamma(z)}
\biggr)^{r_2}
\zeta(1-z, K).
\end{equation}

It follows from \ref{2.3.2} as for $\zeta(z)$ that $\zeta(z,K)$ does not vanish for $z \in H_1$.
Since $\Gamma(z)$ does not vanish anywhere, has simple poles at the non-positive integers but no other poles, we deduce:

\begin{lemm}
\label{2.4.2}
The zeros of $\zeta(z,K)$ in the region $\{z \mid \re z < 0 \}$ can be described as follows:
\begin{enumerate}
\item
if $r_2\not=0$ then for any $m \in \NN$, $z = -2m$ is a zero with multiplicity $r_1 + r_2$ and $z = 1-2m$ is a zero with multiplicity $r_2$;

\item
if $r_2 = 0$, then for any $m \in \NN$, $z = -2m$ is a zero with multiplicity $r_1$;

\item
there are no other zeros in this region.
\end{enumerate}
\end{lemm}



\begin{coro}
\label{2.4.3}
The zeta function of $K$ determines $r_1, r_2, \dim K/\QQ$ and $|D|$.
\end{coro}

\begin{proof}
From \ref{2.4.2} we see that $r_1$ and $r_2$ are determined by the multiplicities of the zeros of the zeta function in the region $\{ z \mid \re z < 0 \}$, and so $\dim K/\QQ = r_1 + 2r_2$ is determined.
Substituting these values in equation \eqref{2.4.1} enables $|D|$ to be determined.
\end{proof}


\begin{coro}
\label{2.4.4}
For quadratic fields $K$, the zeta function determines $K$.
\end{coro}

\begin{proof}
Suppose $K = \QQ(\sqrt d)$, where $d$ is a square free integer.
Then\footnote{\fntwze} we have $D = d$ if $d \equiv 1 \mod 4$, but $D = 4d$ if $d \not\equiv 1 \mod 4$, and so
\[
d = \begin{cases}
(-1)^{1+r_1} |D| & \text{if $|D| \equiv 1 \mod 4$},
\\
(-1)^{1+r_1} |D|/4 & \text{if $|D| \not\equiv 1 \mod 4$}.
\end{cases}
\qedhere
\]
\end{proof}





\section{Artin's thesis}
\label{ch:3}

E. Artin, in his thesis\footnote{\fntwon} in 1921, observing the analogy between the ring $\FF_p[X]$ of polynomials over the finite field $\FF_p$ with $p$ elements ($p$ is prime) has much in common with the ring $\ZZ$ of integers, examined the analogues of the quadratic function fields.
That is to say, let $f(X) \in \FF_p[X]$ be a polynomial of degree at least $1$ and which is square-free, and let $K$ be the splitting field of the polynomial $Y^2 - f(X)$ (in the variable $Y$) over the field $\FF_P(X)$ of rational functions of $X$ (which is the field of fractions of $\FF_p[X]$).
Let $A$ be the integral closure of $\FF_p[X]$ in $K$.
Artin investigated the properties of the ring $A$ by analogy with the classical case.

If we write $Y = \sqrt f$, then any element of $K$ can be expressed uniquely in the form $a + b \sqrt f$ where $a,b \in \FF_p(X)$.
If $p$ is odd, then $a + b \sqrt f \in A$ if and only if $a,b \in \FF_p[X]$.
(This is similar with the classical case, where the integers of $\QQ(\sqrt d)$ are of the form $1/2(a + b\sqrt d)$, where $a,b$ are rational integers.
Since $p\not=2$, we can divide by $2$ in $\FF_p[X]$.)
Thus if $p \not= 2$ (which we tacitly assume throughout this number) $A = \{a + b \sqrt f \in K \mid a, b \in \FF_p[X]\}$.
Clearly $A$ is a free $\FF_p[X]$-module of rank $2$.


\subsection[The units of A]{The units of $A$}
\label{ch:3.1}

The theory of units in the classical case depends on whether or not $\sqrt d$ is real.
Artin introduces the analogue of this as follows.
Let $\FF_p(X)_\infty$ be the field of finite Laurent series in $X^{-1}$.
An element of this field is a formal sum of the form $\sum_{n=-\infty}^{\infty} a_n X^n$ such that for some $m \in \ZZ$, $a_n = 0$ if $n \geq m$.
This field contains $\FF_p[X]$, and hence $\FF_p(X)$ as a subfield.


\begin{defi}
\label{3.1.1}
We say that $\sqrt f$ is \emph{real} if $Y^2 - f$ factorises over $\FF_p(X)_\infty$, i.e., if there exists a $g \in \FF_p(X)_\infty$ such that $g^2 = f$.
We say that $\sqrt f$ is \emph{imaginary} if it is not real.
\end{defi}

Thus $\sqrt f$ is real if and only if there is an embedding of $K$ into $\FF_p(X)_\infty$ which is the identity on $\FF_p(X)$.

\begin{lemm}
\label{3.1.2}
If $f(X) = a_n X^n + \cdots + a_0$, where $a_n \not= 0$, then $\sqrt f$ is real if and only~if%
\begin{enumerate}
\item
\label{3.1.2.1}
$\deg f = n$ is even,

\item
\label{3.1.2.2}
the leading coefficient $a_n$ has a square root in $\FF_p$.
\end{enumerate}
\end{lemm}

\begin{proof}
If $\sqrt f$ is real, clearly \eqref{3.1.2.1} and \eqref{3.1.2.2} are satisfied.
Conversely, if \eqref{3.1.2.1} and \eqref{3.1.2.2} are satisfied we can write $f = a^2 X^{2m}(1 + h)$, where $a^2 = a_n$, $2m = n$ and $h$ is a polynomial in $X^{-1}$ of degree $2m$ with $h(0) = 0$.
Since\footnote{\fntwtw} the binomial coefficient $\binom{1/2}{r}$ is a rational number of the form $a/2^m$ with $a\in\ZZ$ it can be considered as an element of $\FF_p$.
The power series $\sum_{r=1}^\infty \binom{1/2}{r} h^r$ converges in $\FF_p(X)_\infty$ to a square root of $1+h$, and so
\[
g = aX^m \sum_{r=1}^\infty \binom{1/2}{r} h^r
\]
is an element of $\FF_p(X)_\infty$ whose square is $f$.
\end{proof}

Let $U$ denote the group of units of $A$.
The subgroup of elements of finite order is $\mu(K) = \mu(\FF_p) = \FF_p^*$ which is a finite cyclic group of order $p-1$.
Artin shows\footnote{\fntwth} that $U / \FF_p^*$ is zero if $\sqrt f$ is imaginary, but is infinite cyclic if $\sqrt f$ is real, in the second case using continued fractions as in the classical case to obtain fundamental units.


\subsection{Ideal theory}
\label{ch:3.2}

The ring $A$ is Noetherian, every non-zero prime ideal is maximal and $A$is integrally closed, so again $A$ is a Dedekind domain, and hence every non-zero ideal is a unique product of prime ideals.
The ideal class group is again finite;\footnote{\fntwfo} its order is as usual denoted by $h$.

If $I$ is a non-zero ideal of $A$, then $A/I$ is finite; let $N(I)$ be its order.
For any $n \in \NN$ the set $\{I \tri A \mid N(I) = n\}$ is finite.
We define the $\zeta$-function of the extension $K/\FF_p(X)$ by the Dirichlet series
\begin{equation}
\label{3.2.1}
\zeta(z,K/\FF_p(X))
= \sum_{I \tri A, I\not=0} \frac{1}{N(I)^z}.
\end{equation}
Because of unique factorisation of ideals and the fact that $N$ is multiplicative, we could equally well define the $\zeta$ function as the infinite product
\begin{equation}
\label{3.2.2}
\zeta(z,K/\FF_p(X))
= \prod_{P \in \max A} \frac{1}{1 - N(P)^{-z}}.
\end{equation}
If $I$ is a non-zero ideal of $A$, then $A/I$ is a finite vector space over $\FF_p$, say of dimension $\deg I$, so that $N(I) = p^{\deg I}$.
We define the $Z$-function of $K / \FF_p(X)$ to be the power series/infinite product
\begin{equation}
\label{3.2.3}
Z(t, K/\FF_p(X))
= \sum_{I \tri A, I\not=0} t^{\deg I}
= \prod_{P \in \max A} \frac{1}{1 - t^{\deg P}}.
\end{equation}
The relation between the $\zeta$ function and the $Z$ function is that
\begin{align*}
\zeta(z, K/\FF_p(X))
&= Z(p^{-z}, K/\FF_p(X)),
\\
Z(t, K/\FF_p(X)) 
&= \zeta(-\log_p t, K/\FF_p(X)).
\end{align*}


\begin{prop}
\label{3.2.4}
The $\zeta$ function of $K/\FF_p(X)$ is absolutely convergent in $H_1$.
\end{prop}

\begin{proof}
Consider the infinite product representation \eqref{3.2.2} of $\zeta(z, K/\FF_p(X))$.
For a fixed value of $z$, this product is absolutely convergent if\footnote{\fntwfi} (and only if) the series $\sum_{P \in \max A} 1/N(P)^z$ is absolutely convergent.
If $P \in \max A$ then the ideal $P' = P \cap \FF_p[X]$ of $\FF_p[X]$ is prime, and is non-zero, so is maximal.
Since $A/P$ is an extension of $\FF_p[X] / P'$, we know that $N(P) \geq p^r$ where $r$ is the degree of any polynomial which generates $P'$.
Conversely, if $P_1$ is a maximal ideal of $\FF_p[X]$, there are\footnote{\fntwsi} at most $2$ maximal ideals $P'$ of $A$ such that $P' \cap \FF_p[X] = P_1$.
Hence we have
\[
\sum_{P \in \max A} \frac{1}{|N(P)^z|}
\leq 2 \sum_{k=1}^\infty \sum_{f \in \FF_p[X], \text{$f$ monic}, \deg f = k} \frac{1}{|p^{kz}|}
\leq 2 \sum_{k=1}^\infty \frac{p^k}{|p^{kz}|}
\]
and the series on the right of the inequalities, being a geometric series with ratio less than $1$, is convergent.
\end{proof}


\begin{coro}
\label{3.2.5}
The $Z$ function of $K / \FF_p(X)$ is absolutely convergent in the disc $\{ t \in \CC \mid |t| < p^{-1} \}$.
\end{coro}

Artin next showed that $Z(t, K/\FF_p(X)) = P(t, K/\FF_p(X)) / (1 - pt)$, where $P(t, K/\FF_p(X))$ is a polynomial in $t$, of degree less than the degree of $f$.
In fact he gave an explicit value of its degree, depending on whether or not $\sqrt f$ is real and whether $n$ is even\footnote{\fntwse} and these results can be etracted from \eqref{3.3.4} and \eqref{3.3.7} below.

Thus $Z$ and $\zeta$ extend to meromorphic functions on the complex plane.
The $Z$ function has a single simple pole at $t = 1/p$, and so the $\zeta$ function (being periodic with period $2\pi i / p$) has simple poles at $z = 1 + 2k \pi i / p$ for any $k \in \ZZ$.
The residues at these poles involve\footnote{\fntwei} the class number of $K$, and can be determined from \eqref{3.3.9}.


\subsection{The functional equation}
\label{ch:3.3}

Suppose that $f \in \FF_p[X]$ is square-free of degree $n$.
The $Z$ function satisfies the functional equation\footnote{\fntwni}
\begin{equation}
\label{3.3.1}
Z(1/pt, K/\FF_p(X))
= \biggl(
\frac{1-pt}{1-1/t}
\biggr)
\biggl(
\frac{1}{pt^2}
\biggr)^g
Z(t, K/\FF_p(X))
\end{equation}
if $\sqrt f$ is imaginary and $n$ is odd,
\begin{equation}
\label{3.3.2}
Z(1/pt, K/\FF_p(X))
= \biggl(
\frac{1-p^2t^2}{1-1/t^2}
\biggr)
\biggl(
\frac{1}{pt^2}
\biggr)^g
Z(t, K/\FF_p(X))
\end{equation}
if $f$ is imaginary, $n$ even,
\begin{equation}
\label{3.3.3}
Z(1/pt, K/\FF_p(X))
= \biggl(
\frac{1-pt}{1-1/t}
\biggr)^2
\biggl(
\frac{1}{pt^2}
\biggr)^g
Z(t, K/\FF_p(X))
\end{equation}
if $\sqrt f$ is real, where $g$ is the genus\footnote{\fnthze} of $K/\FF_p(X)$, which is\footnote{\fnthon} $(n-1)/2$ if $n$ is odd, but $n/2 - 1$ if $n$ is even.
We define the $Z$ function of $K$, $Z(t,K)$ to be
\begin{equation}
\label{3.3.4}
\begin{aligned}
Z(t,K/\FF_p(X)) / (1-t) & \quad \text{if $\sqrt f$ is imaginary, $n$ odd,}
\\
Z(t,K/\FF_p(X)) / (1-t^2) & \quad \text{if $\sqrt f$ is imaginary, $n$ even,}
\\
Z(t,K/\FF_p(X)) / (1-t)^2 & \quad \text{if $\sqrt f$ is real,}
\end{aligned}
\end{equation}
then the above functional equations \eqref{3.3.1}--\eqref{3.3.3} can be condensed into the single equation
\begin{equation}
\label{3.3.5}
Z(1/pt,K) = (pt^2)^{1-g} Z(t,K).
\end{equation}


\begin{rema}
We shall see in Chapter~\ref{ch:5} that, as the notation suggests, 
% TODO: remove line, check for overfull hbox
the function
$Z(t, K/\FF_p(X))$ depends on the extension $K/\FF_p(X)$, and thus on $A$, whereas $Z(t,K)$ depends only on the field $K$.
This situation does not arise in the classical case.
\end{rema}

As we have said, Artin showed\footnote{\fnthtw} essentially that
\begin{equation}
\label{3.3.7}
Z(t,K) = P(t,K)/(1-pt)(1-t)
\end{equation}
where $P(t,K)$ is a polynomial of degree $2g$, and satisfies the functional equation
\begin{equation}
\label{3.3.8}
P(1/pt, K) = (pt^2)^{-g} P(t,K).
\end{equation}

Artin's results about the residues of the $\zeta$ functions are summarized in the following:

\begin{theo}
\label{3.3.9}
The resiude of $\zeta(z,K)$ at $z = 1$ is
\[
\frac{hp^{1-g}}{(p-1)\log p}.
\]
\end{theo}


This result is equivalent to either of the statements (i) the residue of $Z(t,K)$ at $t=1$ is $h/p-1$ or (ii) $P(1) = h$.


\subsection{The Riemann hypothesis}
\label{ch:3.4}

Artin observed, by making specific calculations in many cases that the non-integral zeros of the $\zeta$ function appeared to lie on the line $\re z = \frac12$.
He conjectured (by virtue of the relation between the $\zeta$ function and the $Z$ function) the following.


\begin{rhqff}
\label{3.4.1}
The zeros of the $Z$ function of a quadratic extension of $\FF_p(X)$ are either trivial (i.e., $\pm1$) or lie on the circle $|t| = 1/\sqrt p$.
\end{rhqff}

It follows from \eqref{3.3.8} that $P(t,K) = \prod_{i=1}^{2g} (1-\alpha_i(t))$ where $\alpha_i \alpha_{g+i} = p$.
The Riemann Hypothesis is that $|\alpha_i|^2 = p$.


\subsection{The points at infinity}
\label{ch:3.5}

The ideas of Artin extend to the study of the integral closure of $\FF_p[X]$ in some finite separable extension $K$ of $\FF_p[X]$.
One can define exactly as in \eqref{3.2.2} and \eqref{3.2.3} the corresponding $\zeta$ and $Z$ functions, and ask the following questions.

% TODO: Check p.16 for counters that use \ref{ch:3.5}
\begin{enumerate}
\item
\label{3.5.1}
Are the $\zeta$ and $Z$ functions absolutely convergent at any points, i.e., are they holomorphic in some domain?

\item
\label{3.5.2}
Do they extend to meromorphic functions on the complex plane, and if so where are the zeros and poles and what are the residues?

\item
\label{3.5.3}
Do they satisfy a functional equation?

\item
\label{3.5.4}
Is the $Z$ function rational?

\item
\label{3.5.5}
Is the generalised Riemann Hypothesis true, i.e., do the zeros of the $Z$ function have absolute value $1/\sqrt p$?
\end{enumerate}

We have seen that even in the quadratic case that there seem to be three separate cases to consider (depending on whether $\sqrt f$ is real or imaginary).
But these distinctions are somewhat artificial as the next example shows.


\begin{exam}
\label{3.5.6}
Let $K$ be the field obtained by adjoining to $\FF_7(X)$ the square roots of $f(X) = 3X^4 + 1$.
We shall see later\footnote{\fnthth} that $Z(t,K) = (1+7t^2)/(1-t)(1-7t)$.
Now since $Y = \sqrt f$ is imaginary, we have $Z(t,K/\FF_7(X)) = (1+t)(1+7t^2)$.

Now consider the elements $X_1 = 1/X$ and $Y_1 = Y/X^2$ of $K$.
The field $K$ is obtained by adjoining to $\FF_7(X_1)$ the element $Y_1$ which is a square root of $X_1^4 + 3$, so that $Y_1$ is real, and so $Z(t, K/\FF_7(X_1)) = (1-t)(1+7t^2)/(1-7t)$.

If we put $X_2 = 1/X-2$ and $Y_2 = Y/(X-2)^2$, then $K$ is also obtained from $\FF_7(X_2)$ by adjoining $Y_2$ which is a square root of $5X_2^3 + 2X_2^2 + 3X_2 + 3$.
Again $Y_2$ is imaginary, but $Y_2^2$ has odd degree in $X_2$, so $Z(t, K/\FF_7(X_2)) = (1+7t^2)/(1-7t)$.

This example shows that the same field can occur as a real or imaginary quadratic extension of either type depending on the choice of subfield of rational functions.

Moreover, since the three $Z$ functions are different, the three corresponding rings of integers, the integral closures of $\FF_7[X]$, $\FF_7[X_1]$ and $\FF_7[X_2]$ are not isomorphic.
\end{exam}

If $K$ is a finite algebraic extension of $\FF_p(X)$, then $K$ is a finitely generated extension of a finite field of transcendence degree\footnote{\fnthfo} one.
Conversely\footnote{\fnthfi} any finitely generated extension of a finite field of transcendence degree one is a finite separable extension of $\FF_p(X)$.


\begin{defi}
\label{3.5.7}
A \emph{function field in $n$ variables over a field $F$} is a finitely generated extension of $F$ of transcendence degree $n$.
A function field in one variable over a finite field is called an \emph{algebraic function field}.
If $K$ is an algebraic function field, its \emph{field of constants} (or constant field) is the algebraic closure of the prime field in $K$.
A \emph{global field} (or \emph{$A$-field}) is either an algebraic number field, or an algebraic function field.
\end{defi}

We shall see in the next section that there is a close analogy between algebraic number fields and algebraic function fields, which justifies the introduction of a common label.

Suppose that $K$ is an algebraic function field, with field of constants $k$.
A choice of embedding $k(X) \to K$ (or equivalently a choice of element $x \in K$ transcendental over $k$) is called a model for $K$.
Each model has an associated ring of integers, which as example~\ref{3.5.6} shows depends on the model.
The explanation of this and the relation between various models is geometrical, and will be given in Chapter~\ref{ch:9}.
We give a simple sketch now.

\begin{figure}[h]
% TODO: figure, p. 18
\end{figure}

An algebraic function field $K$ corresponds to a \qw{smooth} curve $\Gamma$ in some projective space, and each model corresponds to a projection from $\Gamma$ onto a projective line.
The ring of integers of a model corresponds to that part of $\Gamma$ which projects onto the affine line, which consists of all but a finite set of points of $\Gamma$.
In other words, the model \qw{ignores information at infinity}.
However, since there is only a finite set of points at infinity, not too much is lost.
The equations \eqref{3.3.4} are essentially putting back the points at infinity.

Altough this situation does not arise in the classical case (because there is a unique embedding of $\QQ$ into any algebraic number field) it can still be useful to consider points at infinity.
In order to do this, one introduces valuations.



\section{Valuation theory}
\label{ch:4}

\subsection{Todo}
\label{ch:4.7}

\section{Global fields}
\label{ch:5}

\section{Algebraic curves}
\label{ch:6}

\section{Riemann surfaces}
\label{ch:7}

\subsection{Appendix: The fundamental group}
\label{ch:7.8}

\section{Elliptic curves}
\label{ch:8}

\begin{exam}
\label{8.3.2}
\end{exam}

\section{Varieties}
\label{ch:9}

\subsection{Cycles}
\label{ch:9.11}


\section{Complex manifolds}
\label{ch:10}

\section{Hodge theory}
\label{ch:11}

\section{Schemes}
\label{ch:12}





\bibliographystyle{plainurl}
\bibliography{main}


\end{document}
