\documentclass[10pt]{article}

\usepackage{tgpagella}
\linespread{1.1}
\usepackage[utf8]{inputenc}
\usepackage[T1]{fontenc}

\usepackage[normalem]{ulem}
\usepackage{textcomp}
\usepackage[colorlinks=true, citecolor=red]{hyperref}
\usepackage{url}
\usepackage{tikz-cd}

\usepackage{amsmath}
\usepackage{amssymb}
\usepackage{amsthm}

\author{Alan David Thomas}
\date{\today}
\title{Zeta functions: an introduction\\to algebraic geometry}

\newtheorem{theo}{Theorem}[subsection]
\newtheorem{prop}[theo]{Proposition}
\newtheorem{lemm}[theo]{Lemma}
\newtheorem{coro}[theo]{Corollary}
\newtheorem{rh}[theo]{The Riemann Hypothesis}
\theoremstyle{definition}
\newtheorem{defi}[theo]{Definition}
\newtheorem{exam}[theo]{Example}
\newtheorem{rema}[theo]{Remark}

\newcommand{\kk}[1]{\mathbb{#1}}
\newcommand{\cc}[1]{\mathcal{#1}}

\def\eps{\varepsilon}
\def\empty{\varnothing}

\def\ov#1{\overline{#1}}

\def\ldb{[\mkern-2mu[}
\def\rdb{]\mkern-2mu]}
\def\ldp{(\mkern-2.5mu(}
\def\rdp{)\mkern-2.5mu)}

\def\NN{\mathbf{N}}
\def\ZZ{\mathbf{Z}}
\def\QQ{\mathbf{Q}}
\def\RR{\mathbf{R}}
\def\CC{\mathbf{C}}
\def\HH{\mathbf{H}}
\def\FF{\mathbf{F}}

\DeclareMathOperator{\chr}{char}
\DeclareMathOperator{\Gal}{Gal}
\DeclareMathOperator{\re}{Re}

\def\qw#1{`#1'}

\def\fnon{See Cashwell and Everett~\cite{bib:31}, Gilmer~\cite{bib:63}.}
\def\fntw{See Sansone and Gerretsen, vol. 1~\cite{bib:162}, \S7.2.1, p.353.}
\def\fnth{Ibid., \S7.1.3, p.351.}
\def\fnfo{Ibid., \S7.1.2, p.351.}
\def\fnfi{Two holomorphic functions defined in half-planes are \emph{equivalent} if they agree in some half-plane, and a \emph{germ} is an equivalence class.}
\def\fnsi{This may be proved directly, as in Hardy \& Wright~\cite{bib:86}, \S17.1, p.245.}
\def\fnse{See Titchmarsh~\cite{bib:194}, \S9.42, p.300.}
\def\fnei{See Titchmarsh~\cite{bib:195}, Theorem 2.1, p.13.}
\def\fnni{Ibid., pp.1,2, or Hardy \& Wright~\cite{bib:86}, Theorem 280, p.246.}

\def\fnonze{See Sansone and Gerretsen, vol. 1 \cite{bib:162}, \S3.11.1, p.156.}
\def\fnonon{In his proof of the prime number theorem (see Riemann~\cite{bib:155} p.148), Riemann says of the function $\xi(t) = \Gamma(s) (s-1) \zeta(s) \pi^{-t/2}$ where $s = \frac12 + it$ ``Man findet nun in der That etwa so viel reele Wurzeln innerhalb dieser Genzen und es ist sehr wahrscheinlich dass alle Wurzeln reell sind. Hiervon w\"are allerdings eine strenge Beweis zu wanschen; ich habe indess die Aufsuchung desselben nach einigen fl\"uchtigen vergeblichen Veruschen vorl\"aufig bei Seite gelassen, da er f\"ur n\"achsten Zweck meiner Untersuchung entbehrlich sind.''}
\def\fnontw{See Hardy and Wright \cite{bib:86}, Chapter 17 for further details and examples.}
\def\fnonth{}
\def\fnonfo{}
\def\fnonfi{}
\def\fnonsi{}
\def\fnonse{}
\def\fnonei{}
\def\fnonni{}

\def\fntwze{}
\def\fntwon{}
\def\fntwtw{}
\def\fntwth{}
\def\fntwfo{}
\def\fntwfi{}
\def\fntwsi{}
\def\fntwse{}
\def\fntwei{}
\def\fntwni{}

\def\fnthze{}
\def\fnthon{}
\def\fnthtw{}
\def\fnthth{}
\def\fnthfo{}
\def\fnthfi{}
\def\fnthsi{}
\def\fnthse{}
\def\fnthei{}
\def\fnthni{}


\begin{document}

\maketitle


\section*{Introduction}

The aim of this book is to explain the development of the statements and proofs of the Weil conjectures from the analogous results for the Riemann zeta functions, at the same time describing the transformation of classical algebraic number theory and algebraic geometry into the language of schemes.
In order to motivate the cohomological aspects and in particular the idea of an \'etale map, a certain amount of classical algebraic geometry (over the complex numbers) and associated complex manifold theory is included, not only to provide an insight to the more abstract methods, but also to be used in the various comparison techniques.

All these methods and results are contained in existing literature, which to the non-expert may seem both impenetrable and disjointed.
This text is designed to be read by any mathematician who is conversant with the basics of commutative algebra, as in Atiyah \& Macdonald~\cite{bib:17}, field and Galois theory, as in Winter~\cite{bib:217}, and who has some familiarity with manifolds and algebraic topology (see for example Matsushima~\cite{bib:138} and Spanier~\cite{bib:183}) and categories as in Mitchell~\cite{bib:141}.

It would be impossible to cover the contents of this book with full details in a small number of pages, and so footnotes have been included to give explicit references to the relevant information.
I have tried to choose the facts which have been included to give the reader the flavour of the subject.
Following these footnotes, the reader will soon realize the volume of mathematics which has been omitted:
I should like to think that the value of this book lies as much in what is omitted as in what is included.
The decision on the precise contents was a most difficult one.
Probably the most important omissions are the details and precise statement of the Riemann--Roch theorem and its implications in the context of these notes.
The reason for not including this is that, in order to convey the full weight of this theorem, one would have to discuss Jacobians in more detail, and the associated K-theories, characteristic classes and index theorems which would have been impossible in the short number of pages.
This possibly will be the contents of a companion volume.

We start with Riemann's zeta function and its properties, especially those which express number theoretic properties of the ring of integers (which after all was the reason Riemann introduced it).
This immediately generalises to the zeta-function of an algebraic number field, which contains similar information about the field, or more precisely, about its ring of (algebraic) integers.

Next we discuss the contents of Artin's thesis in which he generalises these classical ideas, replacing by the euclidean domain $\FF_p[X]$ of polynomials over the finite field of order $p$, and considering quadratic extensions of the associated field of fractions.
This analogous situation has much in common with the classical case and there is a natural definition of a zeta-function.
Artin observed in the many cases he calculated that the analogue of the Riemann hypothesis was true.

In order to present these two parallel theories in a unified way, we introduce (real) valuations, global fields and the associated functional analysis of ad\`eles and id\`eles.

In Chapter~\ref{ch:6} we show how a global field of non-zero characteristic can be interpreted geometrically as a curve over a finite field.
This then provides a link between number theory and geometry.
Replacing the finite field by the field of complex numbers we have the classical algebraic theory, and from this the analytic theory, of Riemann surfaces, which we discuss in Chapter~\ref{ch:7}.
The analytic theory of course depends on using the \qw{usual} or \qw{strong} topology on the field of complex numbers which is not defined algebraically.
We find ourselves in the position of being able to prove certain algebraic/geometric results analytically but not algebraically, which is unsatisfactory for two reasons.
Firstly for aesthetic reasons, algebraic results should have algebraic proofs, and secondly, such analytic methods do not generalise to the study of curves over other fields.
In this context, the Appendix~\ref{ch:7.8} is of great significance, as it contains the essential ideas behind the effectiveness and importance of the \'etale topology, and the importance of elliptic curves or more generally abelian and Jacobian varieties, which we discuss in Chapter~\ref{ch:8}.
This Chapter ends with Hasse's proof of the Riemann hypothesis for curves of genus $1$.

Historically, this places us at around 1936, when the curve is still identified with its field of functions and is of no geometrical significance.
In order to prove the Riemann hypothesis for global fields of non-zero characteristic, Weil introduced (geometrically) products of curves which lead on to the abstract theory of algebraic varieties over arbitrary fields.

Chapter~\ref{ch:9} starts with the definition of an affine variety, and the concept of smoothness is shown to be an appropriate generalisation of integral closure.
In order to piece together affine varieties to obtain the general variety, we introduce the language of sheaves, first used for this purpose by Serre~\cite{bib:166}.
This is unfortunately rather technical, but simultaneously enables us to define the concept of a manifold, complex manifold and variety (and, in Chapter~\ref{ch:12}, of a scheme).
In \S\ref{ch:9.11} we describe briefly the properties of cycles on a variety, whose importance becomes apparent in Chapter~\ref{ch:10}.
At this point we are in a position to state the Weil conjectures, all of which have now been proved using \'etale topology and cohomology (the fundamental properties of which are contained in the 1583 pages of SGA~4, see \cite{bib:14}).

Besides stating these conjectures, Weil suggested that they should admit a cohomological proof and outlined the lines such a proof might take.
Such a cohomology appeared around 1959/60 and gradually the various necessary properties were established.
Working analytically, Dwork proved the first of these conjectures (rationality) in 1960, and in 1962 Lubkin proved cohomologically all the conjectures (except the Riemann hypothesis) under certain restrictions, although his proof referred to other theorems which had themselves not been proved.

In order to motivate these cohomological attacks on the problems, we describe in Chapter~\ref{ch:10} how one can prove the analogous conjectures (except the Riemann hypothesis) for smooth complex varieties, by giving them the usual topology and then applying \v{C}ech cohomology methods.
The fundamental result is the Lefschetz fixed point theorem, which connects the number of fixed points of a continuous map with its effect in cohomology, and which is used along with properties of cycles and their cohomological invariants.

Chapter~\ref{ch:10} continues this analogy by showing that the statement and proof of the Riemann hypothesis is quite a different problem, needing the force of Hodge theory of harmonic integrals to finally overcome it.

Hopefully by now the reader has reached Chapter~\ref{ch:10}, in which the language of schemes, Grothendieck topologies and \'etale maps are introduced, motivated by the earlier chapters, and we finish with a brief account of the proofs of the Weil conjectures.

Looking back, the reader should then realise that the Riemann hypothesis and the Weil conjectures, apart from being of intrinsic interest and direct application, have generated through the attempts at their verification a considerable amount of other mathematics which we would perhaps otherwise not have discovered.

To some extent then, this book is like a jigsaw puzzle.
At first selected pieces fit together.
As one progresses one finds individual pieces which do not seem to belong anywhere, but the further one goes, the more important these pieces become, until finally they all fit together to form the overall picture.
Of course one might say this is not the way to present a mathematical topic; a mathematical treatise should be linearly ordered.
While this is possible even desirable for \qw{elementary} branches (i.e.,~those using a small vocabulary), it would seem to be impossible for the contents of these notes, if one wishes to provide the motivation and intuition.
The ordering here is loosely based on the chronology of the subject.
Certain questions may be partially answered in one section, only to be resurrected in a later section and tackled again with the aid of techniques developed in the meantime.
One final comment: because of this approach, this book should not be read once!


\subsection*{Notation}

The notation used in this book is either standard or explained in the text, in which case the reader is referred to the index.
The symbols $\NN, \ZZ, \QQ, \RR, \CC$ and $\HH$ denote (respectively) the natural, integral, rational, real, complex and quaternionic numbers, and $S^n$ denotes the $n$-sphere.

All rings are assumed to be commutative with an identity unless otherwise stated.
For any ring $R$ we write $R[X]$ and $R(X)$ for the rings of polynomial and rational functions of $X$ over $R$, and $R\ldb X \rdb$ and $R\ldp X \rdp)$ for the rings of power and (finite) Laurent series.
If $S$ is a subset of $R$, $\langle S \rangle$ denotes the ideal it generates.

For any field $K$ we write $\chr K$ for its characteristic, $K^a$ for its algebraic closure and $K^*$, $\mu(K)$, $\mu_n(K)$ for the groups of units, roots of unity, $n$-th roots of unity in $K$, and $\Gal L/K$ for the Galois group of the normal extension $L/K$.
$\FF_q$ denotes the finite field of order $q$.

Indexing sets, especially for $\sum$, $\prod$, $\varinjlim$ and $\varprojlim$ are often omitted when they are obvious or irrelevant from the context.



\section{Dirichlet series}
\label{ch:1}

\subsection{Dirichlet series}
\label{ch:1.1}

Let $\Omega$ denote the set of functions from $\NN$ to $\CC$.
If we define addition of such functions componentwise, and multiplication by the formula
\[
\alpha \cdot \beta(n)
= \sum_{ij=n} \alpha(i) \beta(j)
\quad
\text{for $\alpha,\beta \in \Omega$}
\]
then $\Omega$ is a commutative ring with $1$, which is called the \emph{ring of Dirichlet series} (over $\CC$).
(It is also called the ring of arithmetic functions; it is a unique factorisation domain, being isomorphic with a ring of power series.\footnote{\fnon})
If $\alpha$ is a Dirichlet series, we define its \emph{derivative} $\alpha' \in \Omega$ by the formula
\[
\alpha'(n) = -\alpha(n) \log n.
\]
The function $d : \Omega \to \Omega$, $\alpha \mapsto \alpha'$, is called \emph{differentiation}.

If $z$ is a complex number, let $\alpha_S(z)$ denote the series $\sum_n \alpha(n) / n^z$ and if this series is convergent, let $\widehat \alpha(z)$ denote its sum.

By abuse of notation, we shall identify $\alpha$ with $\alpha_S$.


\subsection{Convergence of Dirichlet series}
\label{ch:1.2}

For any real number $r$, let $H_r$ denote the open half-plane $\{ z \in \CC \mid \re z > r \}$.
Suppose that $\alpha$ is a Dirichlet series and $z_0$ is a complex number such that $\alpha_S(z_0)$ converges.
Let $r_0 = \re z_0$.
The series $\alpha_S(z)$ converges\footnote{\fntw} for any $z \in H_{r_0}$, and converges uniformly\footnote{\fnth} in any compact subspace of $H_{r_0}$.
Since the function $1/n^z$ is holomorphic in the complex plane, it follows that $\widehat\alpha(z)$ is holomorphic in $H_{r_0}$.
In, in addition, there is a complex number $z_1$ such that $\alpha_S(z_1)$ converges absolutely, then\footnote{\fnfo} $\alpha_S(z)$ converges absolutely for any $z \in H_{r_1}$ where $r_1 = \re z_1$.

Let $\Omega_a = \{ \alpha \in \Omega \mid \text{$\exists z \in \CC$ such that $\alpha_S(z)$ converges absolutely} \}$.
This is a subring of $\Omega$, called the \emph{ring of absolutely convergent Dirichlet series}.
Let $\Omega_0$ be the ring of (germs of)\footnote{\fnfi} holomorphic functions defined in some open half-plane, and let $j : \Omega_a \to \Omega_0$ be the function $j(\alpha) = \widehat\alpha$.
This function is a homomorphism, and it commutes with differentiation.
In fact it is a monomorphism, and the proof\footnote{\fnsi} depends on the following lemma\footnote{\fnse}:


\begin{lemm}
\label{1.2.1}
Let $x,c$ be positive real numbers.
Then
\[
\frac{1}{2\pi i} \int_{c - i \infty}^{c + i\infty} \frac{x^z}{z} dz
= \begin{cases}
1 & \text{if $x < 1$}
\\
0 & \text{if $x > 1$}.
\end{cases}
\]
\end{lemm}


Suppose then that $\widehat\alpha$ is absolutely convergent in $H_r$, and let $x,c$ be positive real numbers such that $c > r$ and $x$ is non-integral.
We see that
\[
\int_{c - i\infty}^{c + i\infty} \frac{\widehat\alpha(z) x^z}{z} dz
= \int_{c - i\infty}^{c + i\infty} \sum_n \frac{\widehat\alpha(n) x^z}{z n^z} dz
= \sum_n \alpha(n) \int_{c - i\infty}^{c + i\infty} \frac{(x/n)^z}{z} dz,
\]
the interchange of integration and summation being possible since the series $\alpha_S(z)$ is uniformly convergent.
Thus applying \ref{1.2.1} we obtain the equation
\[
\frac{1}{2\pi i} 
\int_{c-i\infty}^{c+i\infty} \frac{\widehat\alpha(z) x^z}{z} dz
= \sum_n^{[x]} \alpha(n),
\]
which is called \emph{Perron's formula}.
This formula shows that the holomorphic function $\widehat\alpha$ determines the arithmetic function $\alpha$, so that $j : \Omega_a \to \Omega_0$ is indeed injective.
If $f$ is a function holomorphic in some half plane and $f = j(\alpha)$ for some (unique) $\alpha$, then we say that $\alpha$ is the \emph{Dirichlet series of $f$}.


\subsection{Examples of Dirichlet series}
\label{ch:1.3}

The canonical example of a Dirichlet series is the Riemann zeta-function.
This has Dirichlet series $\sum 1/n^z$, and is denoted by $\zeta(z)$.
This Dirichlet series converges absolutely in $H_1$, but does not converge at $z = 1$.
It is well-known\footnote{\fnei} that the $\zeta$-function extends to a meromorphic function on the complex plane with a simple pole (residue 1) at $z = 1$.
It satisfies the \emph{functional equation}:
\begin{equation}
\label{1.3.1}
\zeta(z) = 2 (2\pi)^{z-1} \Gamma(1-z) \sin(\pi z/2) \zeta(1-z),
\end{equation}
which by means of the equations
\begin{align*}
\Gamma(z) \Gamma(1-z)
&= \pi \operatorname{cosec} (\pi z),
\\
\Gamma(z) \gamma(z + \tfrac 12) 
&= 2^{1-2z} \pi^{1/2} \Gamma(2z)
\end{align*}
may be rewritten
\[
\pi^{-z/2} \Gamma(z/2) \zeta(z)
= \pi^{-(1-z)/2} \Gamma((1-z)/2) \zeta(1-z).
\]
The following proposition is generic in these notes\footnote{\fnni}.


\begin{prop}
\label{1.3.2}
For $z \in H_1$, the infinite product
\begin{equation}
\label{1.3.2.1}
\prod_{p \in \NN,\text{ $p$ prime}}
\frac{1}{1-p^{-z}}
\end{equation}
converges to $\zeta(z)$.
Moreover, this infinite product converges absolutely in $H_1$, and uniformly on compact sets.
\end{prop}

The fact that the value of this infinite product is $\zeta(z)$ is essentially that (formally)
\[
\prod_{p \in \NN,\text{ $p$ prime}}
\frac{1}{1-p^{-z}}
= \prod_{p \in \NN,\text{ $p$ prime}}
\sum_{k=1}^\infty p^{-kz}
= \sum_{n=1}^\infty n^{-z}
\]
since every positive integer is uniquely expressible as a product of positive prime integers.
To convert this into an analytic argument, one has to work with finite sets of primes and take the limit.
This proposition is thus an analytic statement of unique factorisation in the ring of integers.


\begin{coro}
\label{1.3.3}
The set of prime integers is infinite.
\end{coro}

\begin{proof}
If there were only a finite set of primes, then the product \eqref{1.3.2.1} would be a finite product of functions, each holomorphic in $H_0$, and so would define a holomorphic continuation of the $\zeta$-function to $H_0$, which is impossible because of the pole at $z = 1$.
\end{proof}


Since $(1-p^{-z})^{-1}$ does not vanish in $H_1$, it follows from Hurwitz's theorem\footnote{\fnonze} and \ref{1.3.2} that the $\zeta$-function has no zeros in $H_1$.
From the functional equation~\eqref{1.3.1}, it follows that the only zeros of the $\zeta$-function in the region $\{ z \mid \re z < 0 \}$ are the zeros of $\sin(\pi z / 2)$ which occur at the points $z = -2k$ for $k \in \NN$.
These zeros are called the \emph{trivial zeros}.
The remaining zeros thus lie in the \qw{critical strip} $\{z \in \CC \mid 0 \leq \re z \leq 1 \}$, and are called the \qw{non-trivial zeros}.


\begin{rh}
\label{1.3.4}
The zeros in the critical strip all lie on the line $\re z = \frac12$.
\end{rh}


Since first stated by Riemann\footnote{\fnonon}, mathematicians throughout the world have been unable to prove or disprove the validity of this hypothesis.

Many important examples of Dirichlet series are obtained from the $\zeta$ function, and reflect various arithmetic properties of the integers, for example:
\begin{equation}
\label{1.3.5}
\zeta^{-1}(z) = \sum \mu(n) / n^z,
\end{equation}
where $\mu$ is the M\"oebius function, i.e., $\mu(1) = 1$, $\mu(n) = 0$ if $n$ is not square-free, and $\mu(n) = (-1)^k$ if $n$ is the product of $k$ distinct primes;
\begin{equation}
\label{1.3.6}
\zeta^2(s) = \sum d(n) / n^z,
\end{equation}
where $d$ is the divisor function, i.e., $d(n)$ is the number of positive divisors of $n$;
\begin{align}
\label{1.3.7}
\zeta'(z) &= \sum -\log n / n^z
\quad\text{(cf. \ref{ch:1.1})};
\\
\label{1.3.8}
\zeta'(z) / \zeta(z) &= \sum \Lambda(n) / n^z,
\end{align}
where $\Lambda(n) = \log n$ if $n$ is a power of the prime $p$ and $0$ otherwise;
\begin{equation}
\label{1.3.9}
\zeta(z-1)/\zeta(z) = \sum \phi(n) / n^z,
\end{equation}
where $\phi$ is the Euler function.

In all the above examples\footnote{\fnontw} except \eqref{1.3.9}, The Dirichlet series are absolutely convergent in $H_1$.
The Dirichlet series in \eqref{1.3.9} is absolutely convergent in $H_2$.





\section{Classical number theory}
\label{ch:2}

\section{Artin's thesis}
\label{ch:3}

\section{Valuation theory}
\label{ch:4}

\section{Global fields}
\label{ch:5}

\section{Algebraic curves}
\label{ch:6}

\section{Riemann surfaces}
\label{ch:7}

\subsection{Appendix: The fundamental group}
\label{ch:7.8}

\section{Elliptic curves}
\label{ch:8}

\section{Varieties}
\label{ch:9}

\subsection{Cycles}
\label{ch:9.11}


\section{Complex manifolds}
\label{ch:10}

\section{Hodge theory}
\label{ch:11}

\section{Schemes}
\label{ch:12}





\bibliographystyle{plainurl}
\bibliography{main}


\end{document}
