\documentclass[10pt,leqno]{article}

\usepackage{lmodern}
\usepackage[utf8]{inputenc}
\usepackage[T1]{fontenc}

\usepackage[normalem]{ulem}
\usepackage{textcomp}
\usepackage[colorlinks=true, citecolor=red]{hyperref}
\usepackage{url}
\usepackage{tikz-cd}

\usepackage{amsmath}
\usepackage{amssymb}
\usepackage{amsthm}

\author{Alan David Thomas}
\date{\today}
\title{Zeta functions: an introduction\\to algebraic geometry}

\newtheorem{theo}{Theorem}[subsection]
\newtheorem{prop}[theo]{Proposition}
\newtheorem{lemm}[theo]{Lemma}
\newtheorem{coro}[theo]{Corollary}
\newtheorem{rh}[theo]{The Riemann Hypothesis}
\newtheorem{rhqff}[theo]{Riemann Hypothesis for quadratic function fields}
\theoremstyle{definition}
\newtheorem{defi}[theo]{Definition}
\newtheorem{exam}[theo]{Example}
\newtheorem{rema}[theo]{Remark}

\newcommand{\kk}[1]{\mathbb{#1}}
\newcommand{\cc}[1]{\mathcal{#1}}

\def\eps{\varepsilon}
\def\empty{\varnothing}
\def\qandq{\quad\text{and}\quad}

\def\ov#1{\overline{#1}}

\def\ldb{[\mkern-2mu[}
\def\rdb{]\mkern-2mu]}
\def\ldp{(\mkern-2.5mu(}
\def\rdp{)\mkern-2.5mu)}

\def\<{\langle}
\def\>{\rangle}

\def\NN{\mathbf{N}}
\def\ZZ{\mathbf{Z}}
\def\QQ{\mathbf{Q}}
\def\RR{\mathbf{R}}
\def\CC{\mathbf{C}}
\def\HH{\mathbf{H}}
\def\FF{\mathbf{F}}
\def\PP{\mathbf{P}}

\def\mm{\mathfrak{m}}

\DeclareMathOperator{\chr}{char}
\DeclareMathOperator{\Gal}{Gal}
\DeclareMathOperator{\re}{Re}
\DeclareMathOperator{\ord}{ord}
\DeclareMathOperator{\img}{Im}
\newcommand{\tri}{\mathbin{\triangleleft}}

\def\qw#1{`#1'}

\def\fnon{See Cashwell and Everett~\cite{bib:31}, Gilmer~\cite{bib:63}.}
\def\fntw{See Sansone and Gerretsen, vol. 1~\cite{bib:162}, \S7.2.1, p.~353.}
\def\fnth{Ibid., \S7.1.3, p.~351.}
\def\fnfo{Ibid., \S7.1.2, p.~351.}
\def\fnfi{Two holomorphic functions defined in half-planes are \emph{equivalent} if they agree in some half-plane, and a \emph{germ} is an equivalence class.}
\def\fnsi{This may be proved directly, as in Hardy \& Wright~\cite{bib:86}, \S17.1, p.~245.}
\def\fnse{See Titchmarsh~\cite{bib:194}, \S9.42, p.~300.}
\def\fnei{See Titchmarsh~\cite{bib:195}, Theorem 2.1, p.~13.}
\def\fnni{Ibid., pp.~1,2, or Hardy \& Wright~\cite{bib:86}, Theorem 280, p.~246.}

\def\fnonze{See Sansone and Gerretsen, vol. 1 \cite{bib:162}, \S3.11.1, p.~156.}
\def\fnonon{In his proof of the prime number theorem (see Riemann~\cite{bib:155} p.~148), Riemann says of the function $\xi(t) = \Gamma(s) (s-1) \zeta(s) \pi^{-t/2}$ where $s = \frac12 + it$ ``Man findet nun in der That etwa so viel reele Wurzeln innerhalb dieser Genzen und es ist sehr wahrscheinlich dass alle Wurzeln reell sind. Hiervon w\"are allerdings eine strenge Beweis zu wanschen; ich habe indess die Aufsuchung desselben nach einigen fl\"uchtigen vergeblichen Veruschen vorl\"aufig bei Seite gelassen, da er f\"ur n\"achsten Zweck meiner Untersuchung entbehrlich sind.''}
\def\fnontw{See Hardy and Wright \cite{bib:86}, Chapter 17 for further details and examples.}
\def\fnonth{For the definition and properties of integral closure, see Atiyah and Macdonald \cite{bib:17}, Chapter 5, pp.~59--73.}
\def\fnonfo{See Cohn \cite{bib:36}, Theorem 3, p.~96 for example.}
\def\fnonfi{See Cohn \cite{bib:36}, Chapter 6, \S9, p.~110, or Niven and Zuckermann \cite{bib:148}, Chapter 7, in particular \S7.8, pp.~175--179.}
\def\fnonsi{See Atiyah and Macdonald \cite{bib:17}, Chapter 8, pp.~93--98.}
\def\fnonse{This zeta function was introduced by Dedekind in 1871 (see Dedekind \cite{bib:37}), who proved convergence in $H_1$. Landau proved that $\zeta$ extends to a meromorphic function in the half plane $H_a$ where $a = 1 - \deg(K / \QQ)^{-1}$ (see Landau \cite{bib:116}), and Hecke finally proved the extension to $\CC$, see Hecke \cite{bib:94}. See Narkiewicz \cite{bib:146}, Chapter VII, Prop. 7.1, p.~293 and Theorem 7.1, p.~296.}
\def\fnonei{See Narkiewicz \cite{bib:146}, Chapter VI, Prop. 7.1, p.~293. The class number relation was first proved by Dedekind (see Dedekind \cite{bib:37}).}
\def\fnonni{This was first proved by Hecke in Hecke \cite{bib:94}.}

\def\fntwze{See Ribenboim \cite{bib:154}, pp.~97--98.}
\def\fntwon{See Artin \cite{bib:6}.}
\def\fntwtw{It is easy to show from the equation $1 + x = (\sum \binom{1/2}{r} X^r)^2$.}
\def\fntwth{See Artin \cite{bib:6}, pp.~196--197.}
\def\fntwfo{Ibid., p.~180 (imaginary case) and p.~194 (real case).}
\def\fntwfi{See Sansone and Gerretsen \cite{bib:162}, \S4.1, pp.~175--176.}
\def\fntwsi{See Artin \cite{bib:6}, p.~207.}
\def\fntwse{Ibid., p.~209.}
\def\fntwei{Ibid., p.~216.}
\def\fntwni{Ibid., p.~2223}

\def\fnthze{See \S\ref{ch:4.7} and Armitage \cite{bib:5}.}
\def\fnthon{This follows from a simple calculation using the formula.}
\def\fnthtw{See Artin \cite{bib:6}, equation (5), p.~209.}
\def\fnthth{See example \ref{8.3.2}.}
\def\fnthfo{For the definition and properties of transcendence degree, see Winter \cite{bib:217}, p.~41.}
\def\fnthfi{See Weil \cite{bib:214}, Chapter III, \S2, Lemma 1, p.~48.}
\def\fnthsi{See Endler \cite{bib:58}, \S1.14 and \S1.15, p.~6.}
\def\fnthse{Ibid., \S2, p.~8.}
\def\fnthei{Ibid., \S1.5, p.~2.}
\def\fnthni{Immediate from the definition.}

\def\fnfoze{}
\def\fnfoon{}
\def\fnfotw{}
\def\fnfoth{}
\def\fnfofo{}
\def\fnfofi{}
\def\fnfosi{}
\def\fnfose{}
\def\fnfoei{}
\def\fnfoni{}

\def\fnfize{}
\def\fnfion{}
\def\fnfitw{}
\def\fnfith{}
\def\fnfifo{}
\def\fnfifi{}
\def\fnfisi{}
\def\fnfise{}
\def\fnfiei{}
\def\fnfini{}

\def\fnsize{}
\def\fnsion{}
\def\fnsitw{}
\def\fnsith{}
\def\fnsifo{}
\def\fnsifi{}
\def\fnsisi{}
\def\fnsise{}
\def\fnsiei{}
\def\fnsini{}


\begin{document}

\maketitle


\section*{Introduction}

The aim of this book is to explain the development of the statements and proofs of the Weil conjectures from the analogous results for the Riemann zeta functions, at the same time describing the transformation of classical algebraic number theory and algebraic geometry into the language of schemes.
In order to motivate the cohomological aspects and in particular the idea of an \'etale map, a certain amount of classical algebraic geometry (over the complex numbers) and associated complex manifold theory is included, not only to provide an insight to the more abstract methods, but also to be used in the various comparison techniques.

All these methods and results are contained in existing literature, which to the non-expert may seem both impenetrable and disjointed.
This text is designed to be read by any mathematician who is conversant with the basics of commutative algebra, as in Atiyah \& Macdonald~\cite{bib:17}, field and Galois theory, as in Winter~\cite{bib:217}, and who has some familiarity with manifolds and algebraic topology (see for example Matsushima~\cite{bib:138} and Spanier~\cite{bib:183}) and categories as in Mitchell~\cite{bib:141}.

It would be impossible to cover the contents of this book with full details in a small number of pages, and so footnotes have been included to give explicit references to the relevant information.
I have tried to choose the facts which have been included to give the reader the flavour of the subject.
Following these footnotes, the reader will soon realize the volume of mathematics which has been omitted:
I should like to think that the value of this book lies as much in what is omitted as in what is included.
The decision on the precise contents was a most difficult one.
Probably the most important omissions are the details and precise statement of the Riemann--Roch theorem and its implications in the context of these notes.
The reason for not including this is that, in order to convey the full weight of this theorem, one would have to discuss Jacobians in more detail, and the associated K-theories, characteristic classes and index theorems which would have been impossible in the short number of pages.
This possibly will be the contents of a companion volume.

We start with Riemann's zeta function and its properties, especially those which express number theoretic properties of the ring of integers (which after all was the reason Riemann introduced it).
This immediately generalises to the zeta-function of an algebraic number field, which contains similar information about the field, or more precisely, about its ring of (algebraic) integers.

Next we discuss the contents of Artin's thesis in which he generalises these classical ideas, replacing by the euclidean domain $\FF_p[X]$ of polynomials over the finite field of order $p$, and considering quadratic extensions of the associated field of fractions.
This analogous situation has much in common with the classical case and there is a natural definition of a zeta-function.
Artin observed in the many cases he calculated that the analogue of the Riemann hypothesis was true.

In order to present these two parallel theories in a unified way, we introduce (real) valuations, global fields and the associated functional analysis of ad\`eles and id\`eles.

In Chapter~\ref{ch:6} we show how a global field of non-zero characteristic can be interpreted geometrically as a curve over a finite field.
This then provides a link between number theory and geometry.
Replacing the finite field by the field of complex numbers we have the classical algebraic theory, and from this the analytic theory, of Riemann surfaces, which we discuss in Chapter~\ref{ch:7}.
The analytic theory of course depends on using the \qw{usual} or \qw{strong} topology on the field of complex numbers which is not defined algebraically.
We find ourselves in the position of being able to prove certain algebraic/geometric results analytically but not algebraically, which is unsatisfactory for two reasons.
Firstly for aesthetic reasons, algebraic results should have algebraic proofs, and secondly, such analytic methods do not generalise to the study of curves over other fields.
In this context, the Appendix~\ref{ch:7.8} is of great significance, as it contains the essential ideas behind the effectiveness and importance of the \'etale topology, and the importance of elliptic curves or more generally abelian and Jacobian varieties, which we discuss in Chapter~\ref{ch:8}.
This Chapter ends with Hasse's proof of the Riemann hypothesis for curves of genus $1$.

Historically, this places us at around 1936, when the curve is still identified with its field of functions and is of no geometrical significance.
In order to prove the Riemann hypothesis for global fields of non-zero characteristic, Weil introduced (geometrically) products of curves which lead on to the abstract theory of algebraic varieties over arbitrary fields.

Chapter~\ref{ch:9} starts with the definition of an affine variety, and the concept of smoothness is shown to be an appropriate generalisation of integral closure.
In order to piece together affine varieties to obtain the general variety, we introduce the language of sheaves, first used for this purpose by Serre~\cite{bib:166}.
This is unfortunately rather technical, but simultaneously enables us to define the concept of a manifold, complex manifold and variety (and, in Chapter~\ref{ch:12}, of a scheme).
In \S\ref{ch:9.11} we describe briefly the properties of cycles on a variety, whose importance becomes apparent in Chapter~\ref{ch:10}.
At this point we are in a position to state the Weil conjectures, all of which have now been proved using \'etale topology and cohomology (the fundamental properties of which are contained in the 1583 pages of SGA~4, see \cite{bib:14}).

Besides stating these conjectures, Weil suggested that they should admit a cohomological proof and outlined the lines such a proof might take.
Such a cohomology appeared around 1959/60 and gradually the various necessary properties were established.
Working analytically, Dwork proved the first of these conjectures (rationality) in 1960, and in 1962 Lubkin proved cohomologically all the conjectures (except the Riemann hypothesis) under certain restrictions, although his proof referred to other theorems which had themselves not been proved.

In order to motivate these cohomological attacks on the problems, we describe in Chapter~\ref{ch:10} how one can prove the analogous conjectures (except the Riemann hypothesis) for smooth complex varieties, by giving them the usual topology and then applying \v{C}ech cohomology methods.
The fundamental result is the Lefschetz fixed point theorem, which connects the number of fixed points of a continuous map with its effect in cohomology, and which is used along with properties of cycles and their cohomological invariants.

Chapter~\ref{ch:10} continues this analogy by showing that the statement and proof of the Riemann hypothesis is quite a different problem, needing the force of Hodge theory of harmonic integrals to finally overcome it.

Hopefully by now the reader has reached Chapter~\ref{ch:10}, in which the language of schemes, Grothendieck topologies and \'etale maps are introduced, motivated by the earlier chapters, and we finish with a brief account of the proofs of the Weil conjectures.

Looking back, the reader should then realise that the Riemann hypothesis and the Weil conjectures, apart from being of intrinsic interest and direct application, have generated through the attempts at their verification a considerable amount of other mathematics which we would perhaps otherwise not have discovered.

To some extent then, this book is like a jigsaw puzzle.
At first selected pieces fit together.
As one progresses one finds individual pieces which do not seem to belong anywhere, but the further one goes, the more important these pieces become, until finally they all fit together to form the overall picture.
Of course one might say this is not the way to present a mathematical topic; a mathematical treatise should be linearly ordered.
While this is possible even desirable for \qw{elementary} branches (i.e.,~those using a small vocabulary), it would seem to be impossible for the contents of these notes, if one wishes to provide the motivation and intuition.
The ordering here is loosely based on the chronology of the subject.
Certain questions may be partially answered in one section, only to be resurrected in a later section and tackled again with the aid of techniques developed in the meantime.
One final comment: because of this approach, this book should not be read once!


\subsection*{Notation}

The notation used in this book is either standard or explained in the text, in which case the reader is referred to the index.
The symbols $\NN, \ZZ, \QQ, \RR, \CC$ and $\HH$ denote (respectively) the natural, integral, rational, real, complex and quaternionic numbers, and $S^n$ denotes the $n$-sphere.

All rings are assumed to be commutative with an identity unless otherwise stated.
For any ring $R$ we write $R[X]$ and $R(X)$ for the rings of polynomial and rational functions of $X$ over $R$, and $R\ldb X \rdb$ and $R\ldp X \rdp)$ for the rings of power and (finite) Laurent series.
If $S$ is a subset of $R$, $\langle S \rangle$ denotes the ideal it generates.

For any field $K$ we write $\chr K$ for its characteristic, $K^a$ for its algebraic closure and $K^*$, $\mu(K)$, $\mu_n(K)$ for the groups of units, roots of unity, $n$-th roots of unity in $K$, and $\Gal L/K$ for the Galois group of the normal extension $L/K$.
$\FF_q$ denotes the finite field of order $q$.

Indexing sets, especially for $\sum$, $\prod$, $\varinjlim$ and $\varprojlim$ are often omitted when they are obvious or irrelevant from the context.



\section{Dirichlet series}
\label{ch:1}

\subsection{Dirichlet series}
\label{ch:1.1}

Let $\Omega$ denote the set of functions from $\NN$ to $\CC$.
If we define addition of such functions componentwise, and multiplication by the formula
\[
\alpha \cdot \beta(n)
= \sum_{ij=n} \alpha(i) \beta(j)
\quad
\text{for $\alpha,\beta \in \Omega$}
\]
then $\Omega$ is a commutative ring with $1$, which is called the \emph{ring of Dirichlet series} (over $\CC$).
(It is also called the ring of arithmetic functions; it is a unique factorisation domain, being isomorphic with a ring of power series.\footnote{\fnon})
If $\alpha$ is a Dirichlet series, we define its \emph{derivative} $\alpha' \in \Omega$ by the formula
\[
\alpha'(n) = -\alpha(n) \log n.
\]
The function $d : \Omega \to \Omega$, $\alpha \mapsto \alpha'$, is called \emph{differentiation}.

If $z$ is a complex number, let $\alpha_S(z)$ denote the series $\sum_n \alpha(n) / n^z$ and if this series is convergent, let $\widehat \alpha(z)$ denote its sum.

By abuse of notation, we shall identify $\alpha$ with $\alpha_S$.


\subsection{Convergence of Dirichlet series}
\label{ch:1.2}

For any real number $r$, let $H_r$ denote the open half-plane $\{ z \in \CC \mid \re z > r \}$.
Suppose that $\alpha$ is a Dirichlet series and $z_0$ is a complex number such that $\alpha_S(z_0)$ converges.
Let $r_0 = \re z_0$.
The series $\alpha_S(z)$ converges\footnote{\fntw} for any $z \in H_{r_0}$, and converges uniformly\footnote{\fnth} in any compact subspace of $H_{r_0}$.
Since the function $1/n^z$ is holomorphic in the complex plane, it follows that $\widehat\alpha(z)$ is holomorphic in $H_{r_0}$.
In, in addition, there is a complex number $z_1$ such that $\alpha_S(z_1)$ converges absolutely, then\footnote{\fnfo} $\alpha_S(z)$ converges absolutely for any $z \in H_{r_1}$ where $r_1 = \re z_1$.

Let $\Omega_a = \{ \alpha \in \Omega \mid \text{$\exists z \in \CC$ such that $\alpha_S(z)$ converges absolutely} \}$.
This is a subring of $\Omega$, called the \emph{ring of absolutely convergent Dirichlet series}.
Let $\Omega_0$ be the ring of (germs of)\footnote{\fnfi} holomorphic functions defined in some open half-plane, and let $j : \Omega_a \to \Omega_0$ be the function $j(\alpha) = \widehat\alpha$.
This function is a homomorphism, and it commutes with differentiation.
In fact it is a monomorphism, and the proof\footnote{\fnsi} depends on the following lemma:\footnote{\fnse}


\begin{lemm}
\label{1.2.1}
Let $x,c$ be positive real numbers.
Then
\[
\frac{1}{2\pi i} \int_{c - i \infty}^{c + i\infty} \frac{x^z}{z} dz
= \begin{cases}
1 & \text{if $x < 1$}
\\
0 & \text{if $x > 1$}.
\end{cases}
\]
\end{lemm}


Suppose then that $\widehat\alpha$ is absolutely convergent in $H_r$, and let $x,c$ be positive real numbers such that $c > r$ and $x$ is non-integral.
We see that
\[
\int_{c - i\infty}^{c + i\infty} \frac{\widehat\alpha(z) x^z}{z} dz
= \int_{c - i\infty}^{c + i\infty} \sum_n \frac{\widehat\alpha(n) x^z}{z n^z} dz
= \sum_n \alpha(n) \int_{c - i\infty}^{c + i\infty} \frac{(x/n)^z}{z} dz,
\]
the interchange of integration and summation being possible since the series $\alpha_S(z)$ is uniformly convergent.
Thus applying \ref{1.2.1} we obtain the equation
\[
\frac{1}{2\pi i} 
\int_{c-i\infty}^{c+i\infty} \frac{\widehat\alpha(z) x^z}{z} dz
= \sum_n^{[x]} \alpha(n),
\]
which is called \emph{Perron's formula}.
This formula shows that the holomorphic function $\widehat\alpha$ determines the arithmetic function $\alpha$, so that $j : \Omega_a \to \Omega_0$ is indeed injective.
If $f$ is a function holomorphic in some half plane and $f = j(\alpha)$ for some (unique) $\alpha$, then we say that $\alpha$ is the \emph{Dirichlet series of $f$}.


\subsection{Examples of Dirichlet series}
\label{ch:1.3}

The canonical example of a Dirichlet series is the Riemann zeta-function.
This has Dirichlet series $\sum 1/n^z$, and is denoted by $\zeta(z)$.
This Dirichlet series converges absolutely in $H_1$, but does not converge at $z = 1$.
It is well-known\footnote{\fnei} that the $\zeta$-function extends to a meromorphic function on the complex plane with a simple pole (residue 1) at $z = 1$.
It satisfies the \emph{functional equation}:
\begin{equation}
\label{1.3.1}
\zeta(z) = 2 (2\pi)^{z-1} \Gamma(1-z) \sin(\pi z/2) \zeta(1-z),
\end{equation}
which by means of the equations
\begin{align*}
\Gamma(z) \Gamma(1-z)
&= \pi \operatorname{cosec} (\pi z),
\\
\Gamma(z) \gamma(z + \tfrac 12) 
&= 2^{1-2z} \pi^{1/2} \Gamma(2z)
\end{align*}
may be rewritten
\[
\pi^{-z/2} \Gamma(z/2) \zeta(z)
= \pi^{-(1-z)/2} \Gamma((1-z)/2) \zeta(1-z).
\]
The following proposition is generic in these notes.\footnote{\fnni}


\begin{prop}
\label{1.3.2}
For $z \in H_1$, the infinite product
\begin{equation}
\label{1.3.2.1}
\prod_{p \in \NN,\text{ $p$ prime}}
\frac{1}{1-p^{-z}}
\end{equation}
converges to $\zeta(z)$.
Moreover, this infinite product converges absolutely in $H_1$, and uniformly on compact sets.
\end{prop}

The fact that the value of this infinite product is $\zeta(z)$ is essentially that (formally)
\[
\prod_{p \in \NN,\text{ $p$ prime}}
\frac{1}{1-p^{-z}}
= \prod_{p \in \NN,\text{ $p$ prime}}
\sum_{k=1}^\infty p^{-kz}
= \sum_{n=1}^\infty n^{-z}
\]
since every positive integer is uniquely expressible as a product of positive prime integers.
To convert this into an analytic argument, one has to work with finite sets of primes and take the limit.
This proposition is thus an analytic statement of unique factorisation in the ring of integers.


\begin{coro}
\label{1.3.3}
The set of prime integers is infinite.
\end{coro}

\begin{proof}
If there were only a finite set of primes, then the product \eqref{1.3.2.1} would be a finite product of functions, each holomorphic in $H_0$, and so would define a holomorphic continuation of the $\zeta$-function to $H_0$, which is impossible because of the pole at $z = 1$.
\end{proof}


Since $(1-p^{-z})^{-1}$ does not vanish in $H_1$, it follows from Hurwitz's theorem\footnote{\fnonze} and \ref{1.3.2} that the $\zeta$-function has no zeros in $H_1$.
From the functional equation~\eqref{1.3.1}, it follows that the only zeros of the $\zeta$-function in the region $\{ z \mid \re z < 0 \}$ are the zeros of $\sin(\pi z / 2)$ which occur at the points $z = -2k$ for $k \in \NN$.
These zeros are called the \emph{trivial zeros}.
The remaining zeros thus lie in the \qw{critical strip} $\{z \in \CC \mid 0 \leq \re z \leq 1 \}$, and are called the \qw{non-trivial zeros}.


\begin{rh}
\label{1.3.4}
The zeros in the critical strip all lie on the line $\re z = \frac12$.
\end{rh}


Since first stated by Riemann,\footnote{\fnonon} mathematicians throughout the world have been unable to prove or disprove the validity of this hypothesis.

Many important examples of Dirichlet series are obtained from the $\zeta$ function, and reflect various arithmetic properties of the integers, for example:
\begin{equation}
\label{1.3.5}
\zeta^{-1}(z) = \sum \mu(n) / n^z,
\end{equation}
where $\mu$ is the M\"oebius function, i.e., $\mu(1) = 1$, $\mu(n) = 0$ if $n$ is not square-free, and $\mu(n) = (-1)^k$ if $n$ is the product of $k$ distinct primes;
\begin{equation}
\label{1.3.6}
\zeta^2(s) = \sum d(n) / n^z,
\end{equation}
where $d$ is the divisor function, i.e., $d(n)$ is the number of positive divisors of $n$;
\begin{align}
\label{1.3.7}
\zeta'(z) &= \sum -\log n / n^z
\quad\text{(cf. \ref{ch:1.1})};
\\
\label{1.3.8}
\zeta'(z) / \zeta(z) &= \sum \Lambda(n) / n^z,
\end{align}
where $\Lambda(n) = \log n$ if $n$ is a power of the prime $p$ and $0$ otherwise;
\begin{equation}
\label{1.3.9}
\zeta(z-1)/\zeta(z) = \sum \phi(n) / n^z,
\end{equation}
where $\phi$ is the Euler function.

In all the above examples\footnote{\fnontw} except \eqref{1.3.9}, The Dirichlet series are absolutely convergent in $H_1$.
The Dirichlet series in \eqref{1.3.9} is absolutely convergent in $H_2$.





\section{Classical number theory}
\label{ch:2}

Let $\ZZ$ denote the integers, and $\QQ$ the rationals (its field of fractions).
If $K$ is a finite algebraic extension of $\QQ$, then the integral closure\footnote{\fnonth} of $\ZZ$ in $K$, denoted by $A$, is called the \emph{ring of integers} of $K$.
Such rings have many properties in common with the ring $\ZZ$, as we now indicate.
Various proofs and definitions have been omitted, but will be given in more generality.

We start by remarking that if $r = \dim K/\QQ$, then $A$ is a free abelian group of rank $r$.


\subsection{Units}
\label{ch:2.1}

Let $U$ denote the group of units in $A$.
The subgroup of elements of finite order is clearly $\mu(K)$, the group of roots of unity in $K$, which is a finite cyclic group.
The quotient $U / \mu(K)$ is a free abelian group.
Now there are precisely $r$ isomorphisms of $K$ into $\CC$, of which say $r_1$ map into $\RR$, and the remainder occur in $r_2$ conjugate pairs, so that $r = r_1 + 2 r_2$.
The rank of $U / \mu(K)$ is $r_1 + r_2 - 1$.
Putting these results together we have \emph{Dirichlet's Unit Theorem}.


\begin{theo}
\label{2.1.1}
\[
U \cong \ZZ^{r_1 + r_2 - 1} + \mu(K).
\]
\end{theo}

A unit which forms part of a basis for the free part of $U$ is called a \emph{fundamental unit}.

In the case that $K = \QQ(\sqrt d)$, where $d \in \ZZ$ is square-free, the group $U$ is easy to describe.
If the group of roots of unity in $\QQ(\sqrt d)$ has order $m$, then $\QQ(\sqrt d)$ contains a subfield isomorphic with $\QQ(\omega)$, where $\omega = \exp(2 \pi i / m)$.
Since $\det \QQ(\omega) / \QQ = \phi(m)$, where $\phi$ is the Euler function, and $\deg \QQ(\sqrt d) / \QQ = 2$, we must have $\phi(m) = 1$ or $2$, which can only happen if $m = 1,2,3,4$ or $6$.
Since $-1 \in U$, and $(-1)^2 = 1$, $m$ is necessarily even.
In fact one can show\footnote{\fnonfo} that $m = 2$ except when $d = 1$ ($m = 4$) or $d = -3$ ($m = 6$).

If $d > 0$, then $r_1 = 2, r_2 = 0$ so $U$ has rank $1$.
If $d < 0$, then $r_1 = 0, r_2 = 1$ so $U$ is finite and cyclic.


\begin{exam}
\label{2.1.2}
If $K = \QQ(\sqrt 2)$, then $\mu(K) = \{-1,1\}$ so $U \cong \ZZ \oplus \ZZ / 2 \ZZ$.
A fundamental unit is $1 + \sqrt 2$, so any unit is uniquely expressible in the form $\pm(1 + \sqrt 2)^n$ for some $n \in \ZZ$.
\end{exam}


\begin{exam}
\label{2.1.3}
If $K = \QQ(\sqrt{94})$, the only fundamental units are the numbers $\pm 2143295 \pm 221064 \sqrt{94}$.
\end{exam}

There is no formula for the fundamental units even in the quadratic case.
They may be obtained by means of continued fractions.\footnote{\fnonfi}


\begin{exam}
For $p$ a prime integer, let $\omega = \exp(2 \pi i / p)$, and $K = \QQ(\omega)$.
The ring of integers $A$ is $\ZZ[\omega]$ and $r_1 = 0, r_2 = (p-1)/2$ for $p \geq 5$.
The element $\xi_t = \omega^t + \omega^{-t} = 2\cos(2 \pi t / p)$ for $t \in \ZZ$ is a unit.
In general a fundamental set of units is not known.
\end{exam}


\subsection{Ideal theory}
\label{ch:2.2}

The ring $A$ is Noetherian, and every non-zero prime ideal is maximal.
Since $A$ is by definition integrally closed, $A$ is\footnote{\fnonsi} a \emph{Dedekind domain}, and so every non-zero ideal is uniquely expressible as a product of prime ideals.
This is the analogue of unique factorization.
However, in general not every ideal is principal, and the \emph{ideal class group} measures this deficiency.
This group is trivial if and only if $A$ is a principal ideal domain.
In general (and this is an important result) this group is finite.
Its order is usually denoted by $h = h(K)$, and is called the \emph{class-number} of $K$.

If $I$ is a non-zero ideal of $A$, then the quotient ring $A / I$ is finite.
Let $N(I)$ denote its order, which is called the \emph{norm} of $I$.
The norm is multiplicative, i.e., $N(IJ) = N(I) N(J)$.
Moreover, for any positive integer $n$, the set $\{ I \tri A \mid N(I) = n\}$ is finite.


\subsection[The zeta function of K]{The zeta function of $K$}
\label{ch:2.3}

Let $\zeta(z,K)$ denote the Dirichlet series
\begin{equation}
\label{2.3.1}
\sum_{I \tri A, I\not=0} \frac{1}{N(I)^z}
= \sum_n \biggl(
\sum_{I \tri A, N(I)=n} \frac{1}{n^z}
\biggr).
\end{equation}

By analogy with the classical case, there is the following proposition:\footnote{\fnonse}


\begin{prop}
\label{2.3.2}
The Dirichlet series $\zeta(z,K)$ converges absolutely in $H_1$, and extends to a meromorphic function on the complex plane with a simple pole at $z = 1$.
The infinite product
\begin{equation}
\label{2.3.2.1}
\prod_{P \in \max(A)} \frac{1}{1 - N(P)^{-z}}
= \prod_n\biggl(
\prod_{P \in \max(A), N(P)=n} \frac{1}{1-n^{-z}}
\biggr)
\end{equation}
converges in $H_1$ to $\zeta(z,K)$.
Moreover, this infinite product converges absolutely in $H_1$, uniformly on compact subsets.
\end{prop}


As in \ref{1.3.2} the proof of \ref{2.3.2} depends essentially on unique factorisation.


\begin{rema}
\label{2.3.3}
If $P$ is a maximal ideal of $A$, then $N(P)$ is a power of the prime $p$, where $p$ generates the ideal $\ZZ \cap P$ in $\ZZ$.
Thus the infinite product \eqref{2.3.2.1} is indexed by prime powers.
\end{rema}

The following theorem\footnote{\fnonei} relates the behaviour of $\zeta(z,K)$ near its pole to the field $K$.


\begin{theo}
\label{2.3.4}
The residue of $\zeta(z,K)$ at the pole $z = 1$ is
\begin{equation}
\label{2.3.4.1}
\frac{2^{r_1} (2\pi)^{r_2} h R}{|\mu(K)| \sqrt D}
\end{equation}
where $h$ is the class number, $r_1$ and $r_2$ are as defined in \ref{ch:2.1}, $D$ is the discriminant and $R$ the regulator of $K$.
\end{theo}

This theorem and the two previous propositions indicate how the analytic properties of $\zeta(z,K)$ capture algebraic properties of $K$ and its ring of integers.
Other Dirichlet series and meromorphic functions may be obtained from $\zeta(z,K)$ as in \S\ref{ch:1.3}.


\subsection{The functional equation}
\label{ch:2.4}

It may be shown that the $\zeta$-function of $K$ satisfies\footnote{\fnonni} the functional equation:
\begin{equation}
\label{2.4.1}
\zeta(z,K)
= |D|^{\frac12 - z}
\biggl(
\frac{\pi^{z-\frac12} \Gamma(\frac{1-z}2)}{\Gamma(z/2)}
\biggr)^{\!r_1}
\!\!
\biggl(
\frac{(2\pi)^{2z-1} \Gamma(1-z)}{\Gamma(z)}
\biggr)^{\!r_2}
\!\!
\zeta(1-z, K).
\end{equation}

It follows from \ref{2.3.2} as for $\zeta(z)$ that $\zeta(z,K)$ does not vanish for $z \in H_1$.
Since $\Gamma(z)$ does not vanish anywhere, has simple poles at the non-positive integers but no other poles, we deduce:

\begin{lemm}
\label{2.4.2}
The zeros of $\zeta(z,K)$ in the region $\{z \mid \re z < 0 \}$ can be described as follows:
\begin{enumerate}
\item
if $r_2\not=0$ then for any $m \in \NN$, $z = -2m$ is a zero with multiplicity $r_1 + r_2$ and $z = 1-2m$ is a zero with multiplicity $r_2$;

\item
if $r_2 = 0$, then for any $m \in \NN$, $z = -2m$ is a zero with multiplicity $r_1$;

\item
there are no other zeros in this region.
\end{enumerate}
\end{lemm}



\begin{coro}
\label{2.4.3}
The zeta function of $K$ determines $r_1, r_2, \dim K/\QQ$ and $|D|$.
\end{coro}

\begin{proof}
From \ref{2.4.2} we see that $r_1$ and $r_2$ are determined by the multiplicities of the zeros of the zeta function in the region $\{ z \mid \re z < 0 \}$, and so $\dim K/\QQ = r_1 + 2r_2$ is determined.
Substituting these values in equation \eqref{2.4.1} enables $|D|$ to be determined.
\end{proof}


\begin{coro}
\label{2.4.4}
For quadratic fields $K$, the zeta function determines $K$.
\end{coro}

\begin{proof}
Suppose $K = \QQ(\sqrt d)$, where $d$ is a square free integer.
Then\footnote{\fntwze} we have $D = d$ if $d \equiv 1 \mod 4$, but $D = 4d$ if $d \not\equiv 1 \mod 4$, and so
\[
d = \begin{cases}
(-1)^{1+r_1} |D| & \text{if $|D| \equiv 1 \mod 4$},
\\
(-1)^{1+r_1} |D|/4 & \text{if $|D| \not\equiv 1 \mod 4$}.
\end{cases}
\qedhere
\]
\end{proof}





\section{Artin's thesis}
\label{ch:3}

E. Artin, in his thesis\footnote{\fntwon} in 1921, observing the analogy between the ring $\FF_p[X]$ of polynomials over the finite field $\FF_p$ with $p$ elements ($p$ is prime) has much in common with the ring $\ZZ$ of integers, examined the analogues of the quadratic function fields.
That is to say, let $f(X) \in \FF_p[X]$ be a polynomial of degree at least $1$ and which is square-free, and let $K$ be the splitting field of the polynomial $Y^2 - f(X)$ (in the variable $Y$) over the field $\FF_P(X)$ of rational functions of $X$ (which is the field of fractions of $\FF_p[X]$).
Let $A$ be the integral closure of $\FF_p[X]$ in $K$.
Artin investigated the properties of the ring $A$ by analogy with the classical case.

If we write $Y = \sqrt f$, then any element of $K$ can be expressed uniquely in the form $a + b \sqrt f$ where $a,b \in \FF_p(X)$.
If $p$ is odd, then $a + b \sqrt f \in A$ if and only if $a,b \in \FF_p[X]$.
(This is similar with the classical case, where the integers of $\QQ(\sqrt d)$ are of the form $1/2(a + b\sqrt d)$, where $a,b$ are rational integers.
Since $p\not=2$, we can divide by $2$ in $\FF_p[X]$.)
Thus if $p \not= 2$ (which we tacitly assume throughout this number) $A = \{a + b \sqrt f \in K \mid a, b \in \FF_p[X]\}$.
Clearly $A$ is a free $\FF_p[X]$-module of rank $2$.


\subsection[The units of A]{The units of $A$}
\label{ch:3.1}

The theory of units in the classical case depends on whether or not $\sqrt d$ is real.
Artin introduces the analogue of this as follows.
Let $\FF_p(X)_\infty$ be the field of finite Laurent series in $X^{-1}$.
An element of this field is a formal sum of the form $\sum_{n=-\infty}^{\infty} a_n X^n$ such that for some $m \in \ZZ$, $a_n = 0$ if $n \geq m$.
This field contains $\FF_p[X]$, and hence $\FF_p(X)$ as a subfield.


\begin{defi}
\label{3.1.1}
We say that $\sqrt f$ is \emph{real} if $Y^2 - f$ factorises over $\FF_p(X)_\infty$, i.e., if there exists a $g \in \FF_p(X)_\infty$ such that $g^2 = f$.
We say that $\sqrt f$ is \emph{imaginary} if it is not real.
\end{defi}

Thus $\sqrt f$ is real if and only if there is an embedding of $K$ into $\FF_p(X)_\infty$ which is the identity on $\FF_p(X)$.

\begin{lemm}
\label{3.1.2}
If $f(X) = a_n X^n + \cdots + a_0$, where $a_n \not= 0$, then $\sqrt f$ is real if and only~if%
\begin{enumerate}
\item
\label{3.1.2.1}
$\deg f = n$ is even,

\item
\label{3.1.2.2}
the leading coefficient $a_n$ has a square root in $\FF_p$.
\end{enumerate}
\end{lemm}

\begin{proof}
If $\sqrt f$ is real, clearly \eqref{3.1.2.1} and \eqref{3.1.2.2} are satisfied.
Conversely, if \eqref{3.1.2.1} and \eqref{3.1.2.2} are satisfied we can write $f = a^2 X^{2m}(1 + h)$, where $a^2 = a_n$, $2m = n$ and $h$ is a polynomial in $X^{-1}$ of degree $2m$ with $h(0) = 0$.
Since\footnote{\fntwtw} the binomial coefficient $\binom{1/2}{r}$ is a rational number of the form $a/2^m$ with $a\in\ZZ$ it can be considered as an element of $\FF_p$.
The power series $\sum_{r=1}^\infty \binom{1/2}{r} h^r$ converges in $\FF_p(X)_\infty$ to a square root of $1+h$, and so
\[
g = aX^m \sum_{r=1}^\infty \binom{1/2}{r} h^r
\]
is an element of $\FF_p(X)_\infty$ whose square is $f$.
\end{proof}

Let $U$ denote the group of units of $A$.
The subgroup of elements of finite order is $\mu(K) = \mu(\FF_p) = \FF_p^*$ which is a finite cyclic group of order $p-1$.
Artin shows\footnote{\fntwth} that $U / \FF_p^*$ is zero if $\sqrt f$ is imaginary, but is infinite cyclic if $\sqrt f$ is real, in the second case using continued fractions as in the classical case to obtain fundamental units.


\subsection{Ideal theory}
\label{ch:3.2}

The ring $A$ is Noetherian, every non-zero prime ideal is maximal and $A$is integrally closed, so again $A$ is a Dedekind domain, and hence every non-zero ideal is a unique product of prime ideals.
The ideal class group is again finite;\footnote{\fntwfo} its order is as usual denoted by $h$.

If $I$ is a non-zero ideal of $A$, then $A/I$ is finite; let $N(I)$ be its order.
For any $n \in \NN$ the set $\{I \tri A \mid N(I) = n\}$ is finite.
We define the $\zeta$-function of the extension $K/\FF_p(X)$ by the Dirichlet series
\begin{equation}
\label{3.2.1}
\zeta(z,K/\FF_p(X))
= \sum_{I \tri A, I\not=0} \frac{1}{N(I)^z}.
\end{equation}
Because of unique factorisation of ideals and the fact that $N$ is multiplicative, we could equally well define the $\zeta$ function as the infinite product
\begin{equation}
\label{3.2.2}
\zeta(z,K/\FF_p(X))
= \prod_{P \in \max A} \frac{1}{1 - N(P)^{-z}}.
\end{equation}
If $I$ is a non-zero ideal of $A$, then $A/I$ is a finite vector space over $\FF_p$, say of dimension $\deg I$, so that $N(I) = p^{\deg I}$.
We define the $Z$-function of $K / \FF_p(X)$ to be the power series/infinite product
\begin{equation}
\label{3.2.3}
Z(t, K/\FF_p(X))
= \sum_{I \tri A, I\not=0} t^{\deg I}
= \prod_{P \in \max A} \frac{1}{1 - t^{\deg P}}.
\end{equation}
The relation between the $\zeta$ function and the $Z$ function is that
\begin{align*}
\zeta(z, K/\FF_p(X))
&= Z(p^{-z}, K/\FF_p(X)),
\\
Z(t, K/\FF_p(X)) 
&= \zeta(-\log_p t, K/\FF_p(X)).
\end{align*}


\begin{prop}
\label{3.2.4}
The $\zeta$ function of $K/\FF_p(X)$ is absolutely convergent in $H_1$.
\end{prop}

\begin{proof}
Consider the infinite product representation \eqref{3.2.2} of $\zeta(z, K/\FF_p(X))$.
For a fixed value of $z$, this product is absolutely convergent if\footnote{\fntwfi} (and only if) the series $\sum_{P \in \max A} 1/N(P)^z$ is absolutely convergent.
If $P \in \max A$ then the ideal $P' = P \cap \FF_p[X]$ of $\FF_p[X]$ is prime, and is non-zero, so is maximal.
Since $A/P$ is an extension of $\FF_p[X] / P'$, we know that $N(P) \geq p^r$ where $r$ is the degree of any polynomial which generates $P'$.
Conversely, if $P_1$ is a maximal ideal of $\FF_p[X]$, there are\footnote{\fntwsi} at most $2$ maximal ideals $P'$ of $A$ such that $P' \cap \FF_p[X] = P_1$.
Hence we have
\[
\sum_{P \in \max A} \frac{1}{|N(P)^z|}
\leq 2 \sum_{k=1}^\infty \sum_{f \in \FF_p[X], \text{$f$ monic}, \deg f = k} \frac{1}{|p^{kz}|}
\leq 2 \sum_{k=1}^\infty \frac{p^k}{|p^{kz}|}
\]
and the series on the right of the inequalities, being a geometric series with ratio less than $1$, is convergent.
\end{proof}


\begin{coro}
\label{3.2.5}
The $Z$ function of $K / \FF_p(X)$ is absolutely convergent in the disc $\{ t \in \CC \mid |t| < p^{-1} \}$.
\end{coro}

Artin next showed that $Z(t, K/\FF_p(X)) = P(t, K/\FF_p(X)) / (1 - pt)$, where $P(t, K/\FF_p(X))$ is a polynomial in $t$, of degree less than the degree of $f$.
In fact he gave an explicit value of its degree, depending on whether or not $\sqrt f$ is real and whether $n$ is even\footnote{\fntwse} and these results can be etracted from \eqref{3.3.4} and \eqref{3.3.7} below.

Thus $Z$ and $\zeta$ extend to meromorphic functions on the complex plane.
The $Z$ function has a single simple pole at $t = 1/p$, and so the $\zeta$ function (being periodic with period $2\pi i / p$) has simple poles at $z = 1 + 2k \pi i / p$ for any $k \in \ZZ$.
The residues at these poles involve\footnote{\fntwei} the class number of $K$, and can be determined from \eqref{3.3.9}.


\subsection{The functional equation}
\label{ch:3.3}

Suppose that $f \in \FF_p[X]$ is square-free of degree $n$.
The $Z$ function satisfies the functional equation\footnote{\fntwni}
\begin{equation}
\label{3.3.1}
Z(1/pt, K/\FF_p(X))
= \biggl(
\frac{1-pt}{1-1/t}
\biggr)
\biggl(
\frac{1}{pt^2}
\biggr)^g
Z(t, K/\FF_p(X))
\end{equation}
if $\sqrt f$ is imaginary and $n$ is odd,
\begin{equation}
\label{3.3.2}
Z(1/pt, K/\FF_p(X))
= \biggl(
\frac{1-p^2t^2}{1-1/t^2}
\biggr)
\biggl(
\frac{1}{pt^2}
\biggr)^g
Z(t, K/\FF_p(X))
\end{equation}
if $f$ is imaginary, $n$ even,
\begin{equation}
\label{3.3.3}
Z(1/pt, K/\FF_p(X))
= \biggl(
\frac{1-pt}{1-1/t}
\biggr)^2
\biggl(
\frac{1}{pt^2}
\biggr)^g
Z(t, K/\FF_p(X))
\end{equation}
if $\sqrt f$ is real, where $g$ is the genus\footnote{\fnthze} of $K/\FF_p(X)$, which is\footnote{\fnthon} $(n-1)/2$ if $n$ is odd, but $n/2 - 1$ if $n$ is even.
We define the $Z$ function of $K$, $Z(t,K)$ to be
\begin{equation}
\label{3.3.4}
\begin{aligned}
Z(t,K/\FF_p(X)) / (1-t) & \quad \text{if $\sqrt f$ is imaginary, $n$ odd,}
\\
Z(t,K/\FF_p(X)) / (1-t^2) & \quad \text{if $\sqrt f$ is imaginary, $n$ even,}
\\
Z(t,K/\FF_p(X)) / (1-t)^2 & \quad \text{if $\sqrt f$ is real,}
\end{aligned}
\end{equation}
then the above functional equations \eqref{3.3.1}--\eqref{3.3.3} can be condensed into the single equation
\begin{equation}
\label{3.3.5}
Z(1/pt,K) = (pt^2)^{1-g} Z(t,K).
\end{equation}


\begin{rema}
We shall see in Chapter~\ref{ch:5} that, as the notation suggests, 
% TODO: remove line, check for overfull hbox
the function
$Z(t, K/\FF_p(X))$ depends on the extension $K/\FF_p(X)$, and thus on $A$, whereas $Z(t,K)$ depends only on the field $K$.
This situation does not arise in the classical case.
\end{rema}

As we have said, Artin showed\footnote{\fnthtw} essentially that
\begin{equation}
\label{3.3.7}
Z(t,K) = P(t,K)/(1-pt)(1-t)
\end{equation}
where $P(t,K)$ is a polynomial of degree $2g$, and satisfies the functional equation
\begin{equation}
\label{3.3.8}
P(1/pt, K) = (pt^2)^{-g} P(t,K).
\end{equation}

Artin's results about the residues of the $\zeta$ functions are summarized in the following:

\begin{theo}
\label{3.3.9}
The resiude of $\zeta(z,K)$ at $z = 1$ is
\[
\frac{hp^{1-g}}{(p-1)\log p}.
\]
\end{theo}


This result is equivalent to either of the statements (i) the residue of $Z(t,K)$ at $t=1$ is $h/p-1$ or (ii) $P(1) = h$.


\subsection{The Riemann hypothesis}
\label{ch:3.4}

Artin observed, by making specific calculations in many cases that the non-integral zeros of the $\zeta$ function appeared to lie on the line $\re z = \frac12$.
He conjectured (by virtue of the relation between the $\zeta$ function and the $Z$ function) the following.


\begin{rhqff}
\label{3.4.1}
The zeros of the $Z$ function of a quadratic extension of $\FF_p(X)$ are either trivial (i.e., $\pm1$) or lie on the circle $|t| = 1/\sqrt p$.
\end{rhqff}

It follows from \eqref{3.3.8} that $P(t,K) = \prod_{i=1}^{2g} (1-\alpha_i(t))$ where $\alpha_i \alpha_{g+i} = p$.
The Riemann Hypothesis is that $|\alpha_i|^2 = p$.


\subsection{The points at infinity}
\label{ch:3.5}

The ideas of Artin extend to the study of the integral closure of $\FF_p[X]$ in some finite separable extension $K$ of $\FF_p[X]$.
One can define exactly as in \eqref{3.2.2} and \eqref{3.2.3} the corresponding $\zeta$ and $Z$ functions, and ask the following questions.

% TODO: Check p.16 for counters that use \ref{ch:3.5}
\begin{enumerate}
\item
\label{3.5.1}
Are the $\zeta$ and $Z$ functions absolutely convergent at any points, i.e., are they holomorphic in some domain?

\item
\label{3.5.2}
Do they extend to meromorphic functions on the complex plane, and if so where are the zeros and poles and what are the residues?

\item
\label{3.5.3}
Do they satisfy a functional equation?

\item
\label{3.5.4}
Is the $Z$ function rational?

\item
\label{3.5.5}
Is the generalised Riemann Hypothesis true, i.e., do the zeros of the $Z$ function have absolute value $1/\sqrt p$?
\end{enumerate}

We have seen that even in the quadratic case that there seem to be three separate cases to consider (depending on whether $\sqrt f$ is real or imaginary).
But these distinctions are somewhat artificial as the next example shows.


\begin{exam}
\label{3.5.6}
Let $K$ be the field obtained by adjoining to $\FF_7(X)$ the square roots of $f(X) = 3X^4 + 1$.
We shall see later\footnote{\fnthth} that $Z(t,K) = (1+7t^2)/(1-t)(1-7t)$.
Now since $Y = \sqrt f$ is imaginary, we have $Z(t,K/\FF_7(X)) = (1+t)(1+7t^2)$.

Now consider the elements $X_1 = 1/X$ and $Y_1 = Y/X^2$ of $K$.
The field $K$ is obtained by adjoining to $\FF_7(X_1)$ the element $Y_1$ which is a square root of $X_1^4 + 3$, so that $Y_1$ is real, and so $Z(t, K/\FF_7(X_1)) = (1-t)(1+7t^2)/(1-7t)$.

If we put $X_2 = 1/X-2$ and $Y_2 = Y/(X-2)^2$, then $K$ is also obtained from $\FF_7(X_2)$ by adjoining $Y_2$ which is a square root of $5X_2^3 + 2X_2^2 + 3X_2 + 3$.
Again $Y_2$ is imaginary, but $Y_2^2$ has odd degree in $X_2$, so $Z(t, K/\FF_7(X_2)) = (1+7t^2)/(1-7t)$.

This example shows that the same field can occur as a real or imaginary quadratic extension of either type depending on the choice of subfield of rational functions.

Moreover, since the three $Z$ functions are different, the three corresponding rings of integers, the integral closures of $\FF_7[X]$, $\FF_7[X_1]$ and $\FF_7[X_2]$ are not isomorphic.
\end{exam}

If $K$ is a finite algebraic extension of $\FF_p(X)$, then $K$ is a finitely generated extension of a finite field of transcendence degree\footnote{\fnthfo} one.
Conversely\footnote{\fnthfi} any finitely generated extension of a finite field of transcendence degree one is a finite separable extension of $\FF_p(X)$.


\begin{defi}
\label{3.5.7}
A \emph{function field in $n$ variables over a field $F$} is a finitely generated extension of $F$ of transcendence degree $n$.
A function field in one variable over a finite field is called an \emph{algebraic function field}.
If $K$ is an algebraic function field, its \emph{field of constants} (or constant field) is the algebraic closure of the prime field in $K$.
A \emph{global field} (or \emph{$A$-field}) is either an algebraic number field, or an algebraic function field.
\end{defi}

We shall see in the next section that there is a close analogy between algebraic number fields and algebraic function fields, which justifies the introduction of a common label.

Suppose that $K$ is an algebraic function field, with field of constants $k$.
A choice of embedding $k(X) \to K$ (or equivalently a choice of element $x \in K$ transcendental over $k$) is called a model for $K$.
Each model has an associated ring of integers, which as example~\ref{3.5.6} shows depends on the model.
The explanation of this and the relation between various models is geometrical, and will be given in Chapter~\ref{ch:9}.
We give a simple sketch now.

\begin{figure}[h]
% TODO: figure, p. 18
\end{figure}

An algebraic function field $K$ corresponds to a \qw{smooth} curve $\Gamma$ in some projective space, and each model corresponds to a projection from $\Gamma$ onto a projective line.
The ring of integers of a model corresponds to that part of $\Gamma$ which projects onto the affine line, which consists of all but a finite set of points of $\Gamma$.
In other words, the model \qw{ignores information at infinity}.
However, since there is only a finite set of points at infinity, not too much is lost.
The equations \eqref{3.3.4} are essentially putting back the points at infinity.

Altough this situation does not arise in the classical case (because there is a unique embedding of $\QQ$ into any algebraic number field) it can still be useful to consider points at infinity.
In order to do this, one introduces valuations.



\section{Valuation theory}
\label{ch:4}


In this chapter we summarise the essential properties of valuations.
These are then used in Chapter~\ref{ch:5} to provide a unified theory for both algebraic number fields and algebraic function fields.


\subsection{Basic definitions}
\label{ch:4.1}

\begin{defi}
\label{4.1.1}
A \emph{valuation} on a field $K$ is a function $v : K \to \RR$ such that
% TODO: Fix labels, should be item 4.1.1.1 etc
% Or just change the cited locations, if any.
\begin{enumerate}
\item
\label{4.1.1.1}
$v(x) \geq 0$ for all $x \in K$, and $v(x) = 0$ if and only if $x = 0$,

\item
\label{4.1.1.2}
$v(x + y) \leq v(x) + v(y)$ for all $x,y \in K$,

\item
\label{4.1.1.3}
$v(xy) = v(x) v(y)$ for all $x,y \in K$.
\end{enumerate}
\end{defi}


The subgroup $v(K^*) \subset \RR^*$ is called the \emph{value group} of the valuation $v$, and if it is a discrete group, then $v$ is called a \emph{discrete} valuation.

It follows from the definition that $v$ is a metric on $K$, and that with the induced topology $K$ is a topological field.
If
\begin{equation}
\label{4.1.1.4}
v(x + y) \leq \max\{v(x), v(y) \}
\quad
\text{for all $x,y \in K$},
\end{equation}
then $v$ is said to be \emph{non-archimedian}, or \emph{ultrametric}.
If it is not non-archimedian, then it is called \emph{archimedian}.
It is easy to prove:\footnote{\fnthsi}


\begin{lemm}
\label{4.1.2}
If $v$ is an ultrametric valuation on the field $K$ and $x_1,\ldots,x_n \in K$ are such that $v(x_1) \geq v(x_i)$ for $i=2,\ldots,n$ then $v(x_1 + \cdots + x_n) = v(x_1)$.
\end{lemm}

A valuation $v : K \to \RR$ is \emph{complete} if every Cauchy sequence converges.
One can associate in the usual way\footnote{\fnthse} to any valuation $v$ on a field $K$ its completion $i_v : K \to K_v$ which is characterised by the properties that $K_v$ is complete, and that the image of $i_v$ is dense in $K_v$.
Of course, $K_v$ is itself a field.

Suppose that $v,w$ are two valuations on the field $K$.
The following conditions are equivalent:\footnote{\fnthei}
\begin{enumerate}
% TODO: something with these labels
\item
\label{4.1.3.1}
the topologies induced by $v$ and $w$ coincide,

\item
\label{4.1.3.2}
$v(x) < 1$ if and only if $w(x) < 1$,

\item
\label{4.1.3.3}
there exists $r \in \RR$ such that $v(x)^r = w(x)$ for all $x \in K$,

\item
\label{4.1.3.4}
there is a topological isomorphism $\phi : K_v \to K_w$ such that $\phi i_v = i_w$.
\end{enumerate}


\begin{defi}
\label{4.1.4}
Two valuations $v,w$ on $K$ which satisfy the conditions \eqref{4.1.3.1}--\eqref{4.1.3.4} are said to be \emph{equivalent}.
An equivalence class of valuations on $K$ is called a \emph{spot} on $K$.
\end{defi}


Various properties of valuations which are preseved by equivalence can thus be attributed to the spots they determine.
For example, a spot $S$ is non-archi\-median is some (and hence all) $v \in S$ is non-archimedian.
For most purposes, the notions of \qw{valuation} and \qw{spot} can be identified, and we write $K_S$ for $K_v$ when $S$ is the spot determined by $v$.


\begin{exam}
\label{4.1.5}
For any field $K$ the spot defined by the valuation $v(x) = 1$ if $x \not=0$, $v(0) = 0$ is called the \emph{trivial} spot.
It induces on $K$ the discrete topology, and is a complete and discrete spot.
If $K$ is finite, the trivial spot is the only one.
\end{exam}


\begin{exam}
\label{4.1.6}
The spot defined by \qw{absolute value} on $\QQ$ is called the \emph{infinite spot}, and is denoted by $\infty$.
It is not discrete, but it is archimedian.
Its completion is $\QQ \hookrightarrow \RR$.
\end{exam}


\begin{exam}
\label{4.1.7}
Let $A$ be a Dedekind domain, and $K$ its field of fractions.
Let $P$ be a maximal ideal of $A$.
For any $a \in A$, define $\ord_P a$ to be the largest $n$ such that $a \in P^n$ (if $a \not= 0)$, and $\ord_P 0 = 0$.

Hence if $a \in R$ is non-zero then $\prod_{P \in \max A} P^{\ord_P a}$ is the unique factorisation of the ideal generated by $a$.
For $x \in K$, define $\ord_P x = \ord_P a - \ord_P b$ if $x = a/b$.
Choose some constant $c > 1$, and let $v_P(x) = c^{\ord_P x}$.
The function $v_P$ is a valuation, called a \emph{$P$-adic valuation} and the spot it determines (which is independent of the choice of $c$) is called the \emph{$P$-adic spot}, and is denoted by $P$.
If $P$ and $P'$ are distinct maximal ideals of $A$, then clearly the $P$-adic and $P'$-adic spots are distinct.

For example the spot defined on the field of rational functions $F(X)$ over the field $F$ by the valuation
\[
v(f/g) = c^{\deg f - \deg g}
\quad
\text{where $c > 1$, $f,g \in F[X]$}
\]
is independent of the choice of the constant $c$, and is called the \emph{infinite spot}, denoted by $\infty$.
We observe that this valuation is the composite of the $X$-adic valuation on $F(X)$ with the automorphism of $F(X)$ which takes $X$ to $1/X$.
\end{exam}

The following two propositions show an analogy between the fields $\QQ$ and $F(X)$.


\begin{prop}
\label{4.1.8}
The non-trivial spots on $\QQ$ are
\begin{enumerate}
\item
the infinite spot,

\item
the $P$-adic spots, where $P$ is a maximal ideal in $\ZZ$.
\end{enumerate}
All these spots are distinct, and all but the infinite spot are non-archimedian.
All of the completions are locally compact.
\end{prop}


The proof of \ref{4.1.8} is essentially the same as for \ref{4.1.9} below.
We note that if $k$ is a subfield of $K$, then any spot on $K$ restricts to a spot on $k$.


\begin{prop}
\label{4.1.9}
Let $F$ be a field.
The non-trivial spots on $F(X)$ which are trivial on $F$ (i.e., all the non-trivial spots if $F$ is finite) are
\begin{enumerate}
\item
the infinite spot,

\item
the $P$-adic spots, where $P$ is a maximal ideal in $F[X]$.
\end{enumerate}
All these spots are distinct, and all are non-archimedian.
If $F$ is finite, then all the completions are locally compact.
\end{prop}

\begin{proof}
First we note that any spot on $F(X)$ is non-archimedian since\footnote{\fnthni} its restriction to $F$ is non-archimedian.
Let $S$ be a spot on $F(X)$, and $v \in S$ a representative valuation.

If $v(x) > 1$, then $V(X^n) = v(X)^n > v(X)$, and so $v(f) = v(X)^d$ if $f \in F[X]$ is a polynomial of degree $d$ (using \ref{4.1.2}) and hence it follows that $S$ is the infinite spot.

If $v(x) \leq 1$, then $v(f) \leq 1$ for any polynomial $f \in F[X]$.
Since $S$ is non-trivial, there is a $g \in F[X]$ such that $v(g) < 1$, and without loss of generality we can assume that $g$ is irreducible.
If $P$ denotes the prime ideal $\< g \>$ generated by $g$, then $S$ is the $P$-adic spot, as we now prove.

If $h \in F[x]$ and $g \nmid h$, then there exist $a,b \in F[X]$ such that $ag + bh = 1$, and so $v(ag + bh) = 1$.
Hence $v(h) = 1$, since otherwise we would have $v(ag + bh) \leq \max(v(ag), v(bh)) < 1$.
Now any element of $F(X)$ may be written in the form $g^nh/k$ where $h,k \in F[X]$ and $g \nmid h$, $g \nmid k$, and then $v(g^nh/k) = v(g)^n v(h) / v(k) = v(g)^n$ so that $S$ is the $P$-adic spot.

Let $F(X)_P$ denote the completion of $F(X)$ at the $P$-adic spot, and $F(X)_\infty$ the completion at the infinite spot, and suppose now that $F$ is finite.
The natural maps $\{ \rho_n : F[X] \to F[X]/P^n\}$ are continuous where $F[X]/P^n$ has the topology induced by the discrete metric, and so the natural map $\rho : F[X] \to \prod_{n} F[X] / P^n = B$ is also continuous.
Since $B$ is the product of finite (and hence compact) spaces it is compact, so it is a complete metric space, and $\rho$ extends (by universality of completion) to an embedding of the closure of $F[X]$, namely $\{ a \in F(X)_P \mid v(a) \leq 1\}$, into $B$.

The image of $\rho$ ($= \varprojlim F[X] / P^n$) is closed in $B$, hence compact.
The topology on $F(X)_P$ has as basis sets homeomorphic with the set $\{a \in F(X)_P \mid v(a) < 1 \}$, and so is locally compact.

It follows that the space $F(X)_\infty$ is locally compact since it is homeomorphic to $F(X)_{(X)}$.
\end{proof}


If $F$ is algebraically closed, then the non-trivial spots on $F(X)$ which are trivial on $F$ are thus in $1$-$1$ correspondence with the points of $F \cup \{\infty\} = \PP^1(F)$, the projective line over $F$.
If $F$ is not necessarily algebraically closed but is perfect, there is a modified relationship.

Suppose that $P$ is a maximal ideal in $F[X]$ which is generated by the minimum polynomial of the element $\alpha \in F^a$; in this case we write $F(X)_\alpha$ for $F(X)_P$.
The following is a special case of Hensel's lemma:\footnote{\fnfoze}


\begin{lemm}
\label{4.1.10}
Suppose that $F$ is a perfect field, and $P$ is a maximal ideal in $F[X]$ generated by the polynomial $f(X)$.
Let $F'$ denote the quotient field $F[X]/P$, and let $\alpha \in F'$ denote the image of $X$ under the natural map $F[X] \to F'$.
Then we have
\begin{enumerate}
\item
\label{4.1.9.1}
the completion of $F(X)$ at the $P$-adic spot is the composite
\[
F(X) \to F'(X) \to F'(X)_{\alpha},
\]
i.e., $F(X)_P$ is the field of finite Laurent series $F'\ldp X-\alpha \rdp$ in $X-\alpha$ over the field $F'$ (which is a finite separable extension of $F$),

\item
\label{4.1.9.2}
the algebraic closure of $F$ in $F(X)_P$ is $F'$.
\end{enumerate}
\end{lemm}


\begin{proof}
Since $F'(X)_{\alpha}$ is complete, it suffices to show that the image of $F(X)$ in $F'(X)_\alpha$ is dense, but since $F'(X)$ is dense in $F'(X)_\alpha$ it suffices to show that $F'$ is contained in $D$, the closure of $F[X]$, or (sine $\alpha$ generates $F'$ over $F$) that $\alpha \in D$.
To do this, we prove by induction that for each positive integer $n$ there exist $g_n(X) \in F[X]$ and $h_n(X) \in F'[X]$ such that
\begin{equation}
\label{4.1.10.1}
g_n(X) - \alpha = (X-\alpha)^i h_n(X)
\quad
\text{where $i \geq n$.}
\end{equation}
The induction starts with $n=1$, $g_1(X) = 1$, $h_1(X) = 1$.
Assume we have inductively constructed $g_n, h_n$ satisfying \eqref{4.1.10.1}.
If $i > n$ then put $g_{n+1} = g_n$ and $h_{n+1} = h_n$.
Otherwise, let $g_{n+1}$ be $g_n(X) + g(X)( f(X) )^n$, where $g \in F[X]$ is yet to be determined.
Now $g_{n+1}(X) - \alpha = (X-\alpha)^n (h_n(X) + g(X) r(X)^n)$ where $r(X) = f(X) / (X - \alpha)$ is a polynomial in $F'[X]$.
Since $F$ is perfect, $f$ has distinct roots and so $r(\alpha) \not= 0$; let $\beta = h_n(\alpha) / r(\alpha)^n$ so that $\beta \in F'$.
Now choose $g(X) \in F[X]$ such that $g(\alpha) = \beta$ and $\deg g > \deg h_n$ so that $h_n(X) + g(X) r(X)^n$ is not identically zero but has $\alpha$ as a root.
There is thus a polynomial $h_{n+1}(X) \in F'[X]$ such that $h_n(X) + g(X) r(X)^n = (X-\alpha)^j h_{n+1}(X)$ where $j \geq 1$ and $h_{n+1}(\alpha) \not= 0$, and so
\[
g_{n+1}(X) - \alpha = (X - \alpha)^{i+j} h_{n+1}(X).
\]
The proof of \ref{4.1.9.2} is now trivial since the algebraic closure of $F'$ in $F'\ldp t \rdp$ is $F'$.
\end{proof}


\begin{coro}
\label{4.1.11}
If $F$ is a perfect field, then the non-trivial spots on $F(X)$ which are trivial on $F$ are in $1$-$1$ correspondence with the points of $\PP^1(F^a)/G$ where $F^a/F$ is the algebraic closure of $F$, and $G$ is the Galois group of $F^a /F$.
\end{coro}


This result is analogous to the fact that there is a $1$-$1$ correspondence between the maximal ideals of $F[X]$ and the points of $F^a / G$.
We shall pursue this later, in Chapter~\ref{ch:6}.


\subsection{Extension of valuations and spots}
\label{ch:4.2}

For any field $F$, let $P(F)$ denote the set of non-trivial spots on $F$.
Suppose that $S \in P(F)$ and that $v$ is a valuation representing $S$.
If $E/F$ is an extension and $w$ is a valuation on $E$ representing a spot $T$ on $E$ whose restriction to $F$ is $v$, then we say that \emph{$w$ extends $v$} or that \emph{$T$ extends $S$}, and write $T \mid S$.
If $\sigma$ is an automorphism of the field $E$ over $F$ then $w\sigma : E \to \RR$ is also a valuation which extends $v$, so that if $E/F$ is a normal extension $\Gal E/F$ acts on the set of spots which extend $S$.
There are three important cases where one can describe precisely all such extensions.


\subsubsection[Case 1]{Case 1: $F$ is complete, and $E/F$ is a finite extension}
\label{4.2.1}

Since any two norms on a finite-dimensional vector space over a complete field are equivalent, there is at most one spot extending $S$.
There is such an extension\footnote{\fnfoon} determined by the valuation
\[
v(e) = v(N_{E/F}(e))^{1/n}
\quad
\text{for all $e \in E$}
\]
where $n = \deg E/F$ and $N_{E/F} : E \to F$ is the norm function.

Clearly $E$ is then complete, and is archimedian or non-archimedian according to whether $F$ is.
Since $E$ as a topological vector space over $F$ is isomorphic with $F^n$ with norm $|(x_1, \ldots, x_n)| = \max_i v(x_i)$ it follows that $B_E \cong B_F^n$ (where $B_E, B_F$ are the closed balls in $E$ and $F$) so that $E$ is locally compact if and only if $F$ is locally compact.


\subsubsection[Case 2]{Case 2: $F$ is complete, and $F^a/F$ is the algebraic closure}
\label{4.2.2}

By case 1 there is a unique spot on $F^a$ extending $S$ which is represented by the valuation
\[
v(e) = \bigl(
v(N_{F(e)/F}(e))
\bigr)^{1/\deg(F(e)/F)}
\quad
\text{for $e \in F^a$.}
\]
If $F^a/F$ is an infinite extension then $F^a$ may not be complete\footnote{\fnfotw} but its completion is algebraically closed.\footnote{\fnfoth}


\subsubsection[Case 3]{Case 3: $F$ is not necessarily complete, but $E/F$ is a finite separable extension}
\label{4.2.3}

In this case, if $n = \deg E/F$ then there is at least one spot and at most $n$ spots which extend $S$.

Let $\alpha \in E$ be a primitive element, and let $f(X) \in F[X]$ be its minimum polynomial.
Let $B$ be the splitting field of $f$ over $F_v$, which by case 1 has a unique spot extending $S$.
Let $K \subset B$ be the subfield generated over $F$ by the roots of $f$, i.e., $K$ is the splitting field of $f$ over $F$ and hence the normal closure of $E$.
By abuse of notation, let $v$ denote the valuations on $B$ and $K$ obtained by first extending $v$ on $F_v$ to $B$, and then restricting to $K$.
Since $E/F$ is separable, there are precisely $n$ embeddings $\phi : E/F \to K/F$ 
\[
\begin{tikzcd}
F \ar[d] \ar[r] & E \ar[r, dashed, "\phi"] & K \ar[d]
\\
F_v \ar[rr] & & B
\end{tikzcd}
\]
and each of these induces a valuation $v_\phi$ on $E$ by the formula $v_\phi(e) = v(\phi(e))$ for all $e \in E$, with a corresponding spot $S_\phi$.

Moreover, every spot extending $S$ arises in this way.
For if $w$ is a valuation on $E$ extending $v$, then $E_w$ is generated over $F_v$ by $\alpha \in E_w$ so there is an embedding $j : E_w/F_v \to B/F_v$ which is continuous by uniqueness of valuations on finite extension of $F_v$, and since $f(X)$ splits completely in $B$ there is an embedding $\phi : E/F \to K/F$ so that the composite 
\[
E \stackrel{\phi}{\longrightarrow} K \longrightarrow B
\]
is the same as 
\[
E \longrightarrow E_w \stackrel{j}{\longrightarrow} B.
\]
Since then $w = v_\phi$, the spot $S$ is equal to $S_\phi$.
\[
\begin{tikzcd}
F \ar[d] \ar[r] & E \ar[r, dashed, "\phi"] \ar[d] & K \ar[d]
\\
F_v \ar[r] & E_w \ar[r, dashed, "j"] & B
\end{tikzcd}
\]

Any two embeddings $E/F \to K/F$ differ by composition with an automorphism of $K/F$.


\begin{prop}
\label{4.2.4}
If $K/F$ is a finite normal extension and $S$ is a spot on $F$, then the Galois group of $K/F$ acts transitively on the set of spots extending $S$.
\end{prop}


On the other hand, if two embeddings of $E/F$ in $K/F \subset B/F$ differ by composition with an automorphism of $B/F_v$ they determine the same spot since $v(\sigma(b)) = v(b)$ for all $b \in B, \sigma \in \Gal B/F_v$.
The converse if also true:


\begin{lemm}
\label{4.2.5}
If two embeddings $\phi_1,\phi_2 : E/F \to K/F$ determine the same spot on $E$, then there is a $\sigma \in \Gal B/F_v$ such that $\phi_2 = \sigma\phi_1$.
\end{lemm}

\begin{proof}
If $\phi_1,\phi_2$ determine equivalent valuations $v_1,v_2$ then there is a (topological) isomorphism $\psi : E_{v_1} / E \to E_{v_2}/E$ such that $\phi_2 = \psi\phi_1$.
Restricted to $F$, $\phi_1$ and $\phi_2$ are both the natural map $F \to F_v$, so that $\psi$ is the identity on $F$, hence on $F_v$ by denseness, and so there exists a $\sigma \in \Gal B/F_v$ such that $\sigma_{E_{v_1}} = \psi$.
\end{proof}


\begin{coro}
\label{4.2.6}
If $f(X)$ is an irreducible separable polynomial over the field $F$ and $E = F[X] / \< f \>$, then there is a $1$-$1$ correspondence between the set of spots on $E$ which extend a given spot $S$ on $F$ and the set of non-constant irreducible factors of $f(X)$ over $F_S$.
\end{coro}


\begin{exam}
\label{4.2.7}
Consider the extension $\QQ(\alpha)/\QQ$ where $\alpha \in \RR$ satisfies $\alpha^3 = 2$, and let $S$ be the infinite spot on $\QQ$.
In this example, using the notation of \ref{4.2.3}, $B = \CC$ and $K = \QQ(\alpha,\omega)$, where $\omega = \exp(2 \pi i / 3)$.
The three embeddings of $\QQ(\alpha)/\QQ$ into $\QQ(\alpha,\omega)/\QQ$ are determined by the image of $\alpha$ which is either $\alpha, \alpha\omega$ or $\alpha\omega^2$, the last two being conjugate.
The minimum polynomial of $\alpha$ over $\QQ$ is $X^3-2$ which factorises over $\RR$ into $(X-\alpha)(X^2 + \alpha X + \alpha^2)$, both terms being irreducible.
Thus there are two valuations on $\QQ(\alpha)$ extending the infinite one on $\QQ$ given by the formulae
\begin{align*}
v_1(a + b\alpha + c\alpha^2) &= |a + b\alpha + c\alpha^2|
\\
v_2(a + b\alpha + c\alpha^2) &= |a + b\omega\alpha + c\omega^2\alpha^2|
\end{align*}
when $a,b,c \in \QQ$ and where $|\,\cdot\,|$ denotes the absolute value in $\CC$.
\end{exam}


\begin{exam}
\label{4.2.8}
Let $f(X) \in \FF_p[X]$ be a square-free polynomial with leading coefficient $a$, and let $E/F$ be the splitting field of $Y^2 - f$ over $F(X)$.
This is a separable extension if $p \not= 2$, and is normal.
Let $v$ be a valuation on $F(X)$ representing the infinite spot.
We consider separately the three cases as in Chapter~\ref{ch:3}.


\subsubsection[Case 1]{Case 1: $\sqrt f$ real}
\label{4.2.8.1}

There are two embeddings of $E$ into $\FF_p(X)_\infty$ corresponding to the two solutions of $Y^2 = f$ in $\FF_p(X)$ and so there are two spots extending $S$, each having completion $\FF_p(X)_\infty$.


\subsubsection[Case 2]{Case 2: $\sqrt f$ imaginary, $\deg f$ even}
\label{4.2.8.2}

Since $a$ is not a square in $\FF_p$, the field $\FF_p(\sqrt a)$ is $\FF_q$ where $q = p^2$.
Now $\sqrt{a^{-1}f}$ is real, and so there are elements $g_1,g_2 \in \FF_p(X)\infty$ such that $g_1^2 = g_2^2 = a^{-1}f$, $g_1 = -g_2$ and so $\sqrt{ag_1}$ and $\sqrt{ag_2}$ in $\FF_q(X)_\infty$ are two distinct square roots of $f$, and these are conjugate under the Galois group (whose non-zero element acts on a Laurent series as the Frobenius map on the coefficients), and so determine the same spot, that is the unique spot extending $S$; its completion is $\FF_q(X)_\infty$ in which the algebraic closure of the prime field is $\FF_q$ of order $p^2$.

\subsubsection[Case 3]{Case 3: $\sqrt f$ imaginary, $\deg f$ odd}
\label{4.2.8.3}

The splitting field of $Y^2 - f$ over $\FF_p(X)_\infty$ is $\FF_p(\sqrt{qX})_\infty$ since 
\[
f(X) = (aX) X^{2n} (1+h)
\]
where $h \in \FF_p(X)_\infty$ and $v(h) < 1$, and where $\deg f = 2n+1$, and so $\sqrt f = \pm (aX) X^n \sum \binom{1/2}{r} h^r \in \FF_p(\sqrt{aX})_\infty$.

The two corresponding embeddings of $E$ into $\FF_p(\sqrt{aX})_\infty$ are again conjugate under the Galois group (whose non-zero element takes $\sqrt{aX}$ to $-\sqrt{aX}$), and so determine the same spot, which is the unique spot extending $S$; its completion is $\FF_p(\sqrt{aX})_\infty$.

Now consider a non-infinite spot $S$ on $\FF_p(X)$ corresponding to an element $\alpha \in F^a$ (see \ref{4.1.10}).
The extensions of $S$ to $E$ depend on whether or not $\alpha$ is a root of $f(X)$.

If $f(\alpha) \not= 0$ then $f(X) = f(\alpha)(1+h)$ where $h(X) \in \FF_p^a[X]$ has $\alpha$ as a root, and so $\sqrt f = \pm \sqrt{f(\alpha)} \sum \binom{1/2}{r} h^r$ is an element of $\FF_p^a(X)_\alpha$.
These two square roots of $f$ lie in $\FF_p(X)_S$ and so there are two spots extending $S$.

If $f(\alpha) = 0$, then $f(X) = (X-\alpha)(1+h)$ where $h \in \FF_p^a[X]$ and $h(\alpha) = 0$.
As before $1+h$ has a square root in $\FF_p^a(X)_\alpha$, but $X-\alpha$ does not.
Thus there is a unique spot extending $S$ obtained from either of the (conjugate) embeddings of $\FF_p(X)$ into the field obtained by adjoining to $\FF_p^a(X)_\alpha$ the square roots of $X-\alpha$.
\end{exam}


\begin{prop}
\label{4.2.9}
If $E/F$ is a finite separable extension and $S \in P(F)$ then the natural map
\[
F_S \otimes_F E \to \bigoplus_{T \mid S} E_t
\]
is an isomorphism of (topological) $F_S$-algebras.
\end{prop}

\begin{proof}
Let $e \in E$ be a primitive element, and let $f(X) \in F[X]$ be its minimum polynomial, so that \qw{evaluation at $e$} induces an isomorphism $F[X] / \<f\> \to E$.
Let $m_i(X)$ for $i=1,\ldots,r$ be the distinct monic irreducible factors of $f(X)$ over $F_S$, and let $T_i$ denote the spot on $E$ corresponding to $m_i$ (see \ref{4.2.5} and \ref{4.2.6}), so that \qw{evaluation at $e$} induces an isomorphism $F_S[X]/\< m_i \> \to E_{T_i}$.
Chinese remainder theorem gives the diagram
\[
\begin{tikzcd}
F_S \otimes_F F[X]/\<f\> \ar[r,"\cong"] \ar[d,"\cong"] &
F_S \otimes_F E \ar[dd] 
\\
F_S[X] / \<f\> \ar[d,"\cong"] &
\\
\smash{\displaystyle\bigoplus_{i=1}^n F_S[X]} / \<m_i\> \ar[r, "\cong"] &
\smash{\displaystyle\bigoplus_{i=1}^n} \, E_{T_i}
\end{tikzcd}
\]
and hence the result.
\end{proof}


\begin{coro}
\label{4.2.10}
If $E/F$ is a finite separable extension and $S \in P(F)$ then
\[
N_{E/F}(e)
= \prod_{T \mid S} N_{E_T/F_S}(e)
\]
for all $e \in E$.
\end{coro}

\begin{proof}
We have
\begin{align*}
N_{E/F}(e) 
&= \det{}_{\!F}(e : E \to E)
\\
&= \det{}_{\!F_S} (e : F_S \otimes_F E \to F_S \otimes_F E)
\\
&= \prod_{T \mid S} \det{}_{\!F_S}(e : E_T \to E_T)
\qquad\qquad\text{(by \eqref{4.2.9})}
\\
&= \prod_{T \mid S} N_{E_T/F_S}(e).
\qedhere
\end{align*}
\end{proof}



\subsection{Valuation rings}
\label{ch:4.3}

Let $F$ be a field, and $S$ a non-archimedian spot on $F$ represented by the valuation $v \in S$.
The subset
\begin{equation}
\label{4.3.1}
R_S = R_S(F)
= \{ x \in F \mid v(x) \leq 1 \}
\end{equation}
is a subring of $F$, and is independent of the choice of $v \in S$, by \eqref{4.1.3.2}.
% TODO: Fix \eqref above to point to something that makes sense
It is called the \emph{valuation ring of $F$ at $S$}, or the \emph{ring of integers at $S$}, its elements being called \emph{integers at $S$}.


\begin{prop}
\label{4.3.2}
The ring $R_S$ is integrally closed in $F$.
\end{prop}

\begin{proof}
Suppose $x \in F$ is integral over $R_S$ and so satisfies an equation of the form
\[
x^n + a_{n-1} x^{n-1} + \cdots + a_0 = 0
\]
with $a_0, \ldots, a_{n-1} \in R_S$.
If $x$ were not in $R_S$ so that $v(x) > 1$, then we would have $v(x^n) > v(a_i x^i)$ for $i = 0, \ldots, n-1$ and hence
\[
v(0) = v(x^n + a_{n-1} x^{n-1} + \cdots + a_0) = v(x)^n > 1
\]
(using \ref{4.1.2}) which is a contradiction.
\end{proof}

The ring $R_S$ is a local ring with (unique) maximal ideal
\begin{equation}
\label{4.3.3}
\mm_S = \mm_S(F) = \{ x \in F \mid v(x) < 1 \}
\quad
\text{(the interior of $R_S$)}
\end{equation}
and group of units
\begin{equation}
\label{4.3.4}
U_S = U_S(F) = \{ x \in F \mid v(x) = 1 \}
\quad
\text{(the boundary of $R_S$).}
\end{equation}
The quotient field
\begin{equation}
\label{4.3.5}
F(S) = R_S / \mm_S
\end{equation}
is called the \emph{residue field} of $F$ at $S$.
The natural map from $R_S$ to $F(S)$ will be denoted by $\rho$ or $\rho_S$, and is called \qw{reduction mod $S$}.


\begin{exam}
\label{4.3.6}
If $A$ is a Dedekind domain with field of fractions $F$ and $S$ is the $P$-adic spot on $F$ where $P$ is some maximal ideal of $A$ (see \ref{4.1.6}) then the ring of integers at $S$ is precisely the localisation $A_P$ of the ring $A$ at the maximal ideal $P$, and the residue field is $A/P$.

If $i : F \to E$ is a field extension and $T$ is a spot on $E$ extending the spot $S$ on $F$ then by restriction $i$ maps $R_S$ into $R_T$.
Moreover $i^{-1}(\mm_T) = \mm_B$, that is to say $i$ is a \emph{local homomorphism}, and it induces a (natural) embedding $i : F(S) \to E(T)$.
Clearly we have $\deg E(T)/F(S) \leq \deg E/F$.
In particular, if $S$ is a non-archimedian spot on a global field $K$, then $K(S)$ is finite.
\end{exam}


The following lemma is quite straightforward.\footnote{\fnfofo}


\begin{lemm}
\label{4.3.7}
If $F$ is a field, and $S$ is a non-archimedian spot on $F$ with completion $F_S$, then
\begin{enumerate}
\item
\label{4.3.7.1}
the natural embedding $F(S) \to F_S(S)$ is an isomorphism,

\item
\label{4.3.7.2}
if $v$ is a valuation on $F_S$ representing the spot $S$ then
\[
\img(v : F_S \to \RR) = \img(v : F \to \RR).
\]
\end{enumerate}
\end{lemm}


Thus the residue field and value group are invariant under completion.
We now consider the effect of algebraic closure.


\begin{lemm}
\label{4.3.8}
Let $F$ be a field which is complete with respect to the non-archimedian spot $S$.
Let $S^a$ denote the unique spot on $F^a$ (the algebraic closure of $F$) which extends $S$ (see \eqref{4.2.2}).
Then $F^a(S^a)$ is the algbraic closure of $F(S)$.
\end{lemm}

\begin{proof}
Suppose $\bar f(X)$ is a monic polynomial over $F^a(S^a)$.
Let $f(X)$ be a monic polynomial over $R_{S^a}$ of the same degree such that $\rho f = \bar f$.
All the roots of $f$ lie in $F^a$, and hence in $R_{S^a}$ by \ref{4.3.2}.
Their images in $F(S)$ are all the roots of $\bar f(X)$, so $F^a(S^a)$ is algebraically closed.

Now suppose that $\bar x$ is an element of $F^a(S^a)$ and let $x \in \rho^{-1}(\bar x) \subset R_{S^a}$ representatives of $\bar x$.
Since $x$ is algebraic over $F$ we have $f(x) = 0$ for some polynomial $F(X) \in F[X]$.
By multiplying $f(X)$ by some element of $F$ if necessary we can assume that $f(X) \in R_S[X]$ and that $1$ is a coefficient of $f(X)$.
The reduction $\bar f(X)$ of $f(X) \mod \mm_S$ is thus non-trivial and has $\bar x$ as a root, that is to say each element of $F^a(S^a)$ is algebraic over $F(S)$.
\end{proof}



\subsection{Integral closure}
\label{ch:4.4}

The relation between integral closure and spots is contained in the following result.


\begin{prop}
\label{4.4.1}
If $E/F$ is a finite separable extension and $S$ a non-archi\-median spot on $F$, then the integral closure $A$ of $R_S$ in $E$ is $\bigcap_{T \mid S} R_T$.
\end{prop}

\begin{proof}
Since $\bigcap_{T \mid S} R_T$ is integrally closed by \ref{4.3.2} and contains $R_S$, it contains $A$.
Conversely, suppose that $x \in \bigcap_{T \mid S} R_T$, and let $m(X) = X^n + a_{n-1} X^{n-1} + \cdots + a_0$ be its minimum polynomial over $F$.
Let $B$ be the splitting field of $m(X)$ over $F_S$ and let $R$ be the ring of integers of $B$ (at the unique spot extending $S$ on $F_S$).
Let $x_1, \ldots, x_n$ be the roots of $m(X)$ in $B$; if $x_i \not\in R$ for some $i$, then the embedding $F(x) \to B$ which takes $x$ to $x_i$ would define a spot on $F(x)$ which would extend to a spot $T \in S^E$ with the property that $x \in R_T$.
Hence $x_i \in R$ for all $i$, and since $a_j$ is, up to sign, the $j$-th elementary symmetric polynomial in the roots we have $a_j \in R$ for all $j$, and so $x$ is integral over $R_S$.
\end{proof}


\subsection{Discrete spots}
\label{ch:4.5}

Suppose that $S$ is a discrete spot on the field $F$ represented by the valuation $v \in S$.
By definition the value group $v(F^*)$ is a discrete subgroup of $\RR^*$, and so\footnote{\fnfofi} is a free abelian group of rank 1 generated by $v(\eta)$ where $\eta$ is any element of $\mm_S \setminus \mm_S^2$.
Such an element $\eta$ is called a \emph{prime element} of $F$ at $S$.
Any other element $x \in F$ is also a prime element if and only if $x = u \eta$ for some $u \in U_S$, that is to say there is a unique prime element up to units.

For any $r \in \ZZ$, let $\mm_S^r$ denote the $R_S$-submodule of $F$ generated by $\eta^r$.
(This notation is consistent since fo r$r \in \NN$, the element $\eta^r$ does generate the ideal $\mm_S^r$ of $R_S$.)

The function $\ord_S : F \to \ZZ$ defined by
\begin{equation}
\label{4.5.1}
\ord_S(x) = \max \{ r \mid x \in \mm_S^r \}
\end{equation}
is a Euclidean algorithm on $R_S$, and for any constant $c > 0$ the function $x \mapsto c^{-\ord_S x}$ from $F$ to $\RR$ is a valuation of $F$ representing $S$.
This should be compared with \ref{4.1.8} since $R_S$ is a Dedekind domain with unique maximal ideal; in fact it is a principal ideal domain.
The completion $\widehat R_S$ of $R_S$ with respect to the $\mm_S$-adic topology, namely $\varprojlim_{n} R_S / \mm_S^n$, is precisely the ring of integers in the complete field $F_S$.


\begin{lemm}
\label{4.5.2}
Let $S$ be a discrete spot on the field $F$ such that $F$ is complete and the residue field $F(S)$ is perfect, and has characteristic $p \not= 0$.
There exists a unique function $j : F(S) \to R_S$ such that
\begin{enumerate}
\item
\label{4.5.2.1}
$\rho j = 1$ (where $\rho : R_S \to F(S)$ is reduction mod $S$).

\item
\label{4.5.2.2}
$j(xy) = j(x) j(y)$ for all $x,y \in F(S)$.
\end{enumerate}
If $\chr F = p$, then $j$ is a field homomorphism.

If $\widetilde U_S = \{x \in U_S \mid \text{$x^m = 1$ for some $m$ with $(m,p) = 1$} \}$, then $\rho : \widetilde U_S \to \mu(F(S))$ is an isomorphism with inverse $j$.
\end{lemm}

\begin{proof}
For $x \in F$ and $n$ a positive integer let $U_n(X) = (\rho^{-1}(x^{p^{-n}}))^{p^n} \subset F$.
The family $\{U_n(x) \mid n \in \NN\}$ is a Cauchy filter\footnote{\fnfosi} in $F$.
Define $j(x) = \lim_n U_n(x)$, then clearly $j$ satisfies \eqref{4.5.2.1} and \eqref{4.5.2.2}.
If $j' : F(S) \to R_S$ is any other function satisfying these conditions then $j'(x) \in U_n(x)$ for all $n$, so $j' = j$.
Clearly $j$ is additive if $\chr F = \chr F(S)$.

To prove the last part, it suffices to show that $\rho : \widetilde U_S \to \mu(F(S))$ is injective.
Suppose then for some $x \in \widetilde U_S$ of order $m \not= 1$ we have $\rho(x) = 1$.
For any positive integer $n$ there are integers $a_n, b_n$ such that $a_n m + b_n p^n = 1$, since $(m,p) = 1$, and so $x$ is the $p^n$-th power of $x^{b_n}$.
This means that $x$ and $1$ both belong to $U_n(1)$ for all $n$, hence $x = 1$.
\end{proof}


In particular, under the conditions of \ref{4.5.2} if $\chr F = \chr F(S)$ and $F(S)$ is finite, then the map $j : F(S) \to F$ is an isomorphism onto the algebraic closure of the prime field in $F$.


\begin{defi}
\label{4.5.3}
If $F$ is any field and $S$ is a spot on $F$ which is discrete, then a function $j : F(S) \to R_S$ satisfying \eqref{4.5.2.1} and \eqref{4.5.2.2} and the set $j(F(S))$ are both called a \emph{system of multiplicative representatives (SMR)} for $F(S)$.
If the SMR is unique, then we write $\operatorname{SMR}(S)$ for $j(F(S))$.
\end{defi}

Suppose now that $S$ is a discrete spot on $F$, and that $F$ is complete with finite residue field.
Let $\eta \in \mm_S \setminus \mm_S^2$ be a prime element.
If $x \in \mm_S^n$ then it is easy to see that there is a unique element $a_n \in \operatorname{SMR}(S)$ such that $x - a_n \eta^n \in \mm_S^{n+1}$, namely $a_n = j\rho(x\eta^{-n})$, and hence by induction there is for each ineger $\NN$ a unique finite Laurent series $x_N = \sum_{\ord_S x}^n a_i \eta^i$ with $a_i \in \operatorname{SMR}(S)$ such that $x - x_N \in \mm_S^{N+1}$ or equivalently $\ord_S(x - x_N) > N$.
The sequence $\{ x_N \}$ is Cauchy, and converges to $x$, so each element of $F$ has a unique \qw{Laurent series} representation
\begin{equation}
\label{4.5.4}
x = \sum_{\ord_S x}^\infty a_i \eta^i
\quad
\text{where $a_i \in \operatorname{SMR}(S)$.}
\end{equation}

Such a Laurent series is integral (i.e., belongs to $R_S$) if and only if it is a power series.
This result shows that if $\chr F = \chr F(S)$ then $F$ is isomorphic to the field of Laurent series over $F(S)$, while $R_S$ is isomorphic to the ring of power series over $F(S)$.

This also shows that if $F(S) = R_S / \mm_S$ is finite, then so is $R_S / \mm_S^n$ for any $n \geq 1$, and so $\widehat R_S$ is compact since $\widehat R_S = \varprojlim_n R_S / \mm_S^n$ is an inverse limit of finite, hence compact, spaces.

These methods admit straightforward generalisations to finite $R_S$-algebras.
Suppose that $S$ is a discrete spot on the field $F$, that $F$ is complete, and that $F(S)$ is perfect of characteristic $p \not= 0$.
If $A$ is a finite $R_S$-algebra, then\footnote{\fnfose} $A$ is complete with respect to the $\mm_S$-adic topology and there exists and $\operatorname{SMR}(j : A/\mm_S A \to A)$ as in \ref{4.5.4}.
This enables us to prove \emph{Hensel's Lemma}.\footnote{\fnfoei}


\begin{lemm}
\label{4.5.5}
Let $S$ be a discrete spot on the field $F$ such that $F$ is complete, $F(S)$ is perfect.
Let $f(X)$ be a monic polynomial in $R_S[X]$ such that $\rho(f) = g_1g_2$ where $g_1,g_2$ are monic polynomials over $F(S)$ and are relatively prime and $\deg \rho(f) = \deg f$.
Then there exist monic polynomials $h_1,h_2 \in R_S[X]$ such that
\begin{enumerate}
\item
$f = h_1 h_2$

\item
$\rho(h_i) = g_i$ for $i = 1,2$.
\end{enumerate}
\end{lemm}

\begin{proof}
Consider the finite $R_S$-algebra $A = R_S[X]/\<f\>$ and its residue algebra $A/\mm_S A = F(S)/\<\rho f\>$.
Since $(g_1,g_2) = 1$ there is a decomposition $A/\mm_S A \cong F(S)[X]/\< g_1 \> \oplus F(S)[X] / \<g_2\>$; let $e_1$ and $e_2$ be the idempotents in $A/\mm_S A$ corresponding to this decomposition to $(1,0)$ and $(0,1)$, and let $a_i = j(e_i) \in A$, where $j : A / \mm_S A \to A$ is an $\operatorname{SMR}$.
Since $j$ is multiplicative, each $a_i$ is idempotent and $a_1 a_2 = 0$, so there is a decomposition $A = A a_1 \oplus A a_2 \oplus A(1 - a_1 - a_2)$.
The subalgebra $A(1 - a_1 - a_2)$ reduced mod $M_S$ is zero, so is itself zero by Nakayama's lemma.\footnote{\fnfoni}
If $a \in A$ denotes the image of $X$ under the map $R_S[X] \to A$ and $h_i(X) \in R_S[X]$ is the characteristic polynomial of $a : A a_i \to A a_i$, then $h_i(X)$ is monic and $h_1(X) h_2(X)$ is the characteristic polynomial of $a : A \to A$, which is $f(X)$.
Moreover $\rho h_i(X)$ is the characteristic polynomial of $\rho(a) : F(S) e_i \to F(S) e_i$ which is $g_i(X)$.
\end{proof}


\subsection{Ramification}
\label{ch:4.6}

Let $E/F$ be a finite extension and $T$ a discrete spot on $E$ extending $S$ on $F$.
If $v$ is a valuation on $E$ representing $T$ then $v(F^*)$ is a subgroup of $v(E^*)$ which has finite index since both groups are free of rank 1.
The integer
\begin{equation}
\label{4.6.1}
e_{T \mid S} = e = | v(E^*) / v(F^*) |
\end{equation}
is independent of the choice of $v$ and is called the \emph{ramification index} of $T$ over $S$, and satisfies (or is equally defined  by) the conditions
\begin{align}
\label{4.6.2}
&\text{$e \ord_S x = \ord_T x$ for all $x \in F$;}
\\
\label{4.6.3}
&\text{if $\eta$ is a prime of $F$ at $S$ and $\xi$ of $E$ at $T$ then $\eta^{-1} \xi^e \in U_T$.}
\end{align}


\begin{defi}
\label{4.6.4}
If $S$ is a complete discrete spot on the field $F$ and $E/F$ is a finite extension with spot $T$ extending $S$, then $S$ is \emph{unramified over $T$} if the ramification index $e_{T \mid S}$ is 1, and the extension $E(T)/F(S)$ is separable.

If $S$ is a discrete spot on a field $F$ which is not necessarily complete and $E/F$ is a finite extension, then $S$ is \emph{unramified over $E$} if for each $T$ extending $S$ on $F$, the spot $S$ on $F_S$ is unramified over $E_T$.
\end{defi}


From \ref{4.3.7} we see that the ramification index is invariant under completion.
The integer
\begin{equation}
\label{4.6.5}
f_{T \mid S} = f = \deg E(T) / F(S)
\end{equation}
is called the \emph{relative degree} of $T$ over $S$.


\begin{prop}
\label{4.6.6}
If $E/F$ is a finite extension, $T,S$ are discrete spots on $E,F$ which are also complete and $T \mid S$ then
\[
e_{T\mid S} \cdot f_{T \mid S} = \deg E/F.
\]
\end{prop}

\begin{proof}
Let $a_1, \ldots, a_f$ be a basis for $E(T)/F(S)$ and let $j : E(T) \to E$ be an SMR for $T$, which exists by \ref{4.5.2}.
If $\xi$ is a prime element of $E$ at $T$ then we shall show that the set $D = \{ j(a_r) \xi^s \mid 1 \leq r \leq f, 0 \leq s < e\}$ is a basis for $E$ over $F$.

Let $V$ be the $F$-subspace of $E$ generated by $E$, which is complete, being a finite-dimensional vector space over a complete field.
If $x \in E$ and $\ord_T x = m$ then $x = u\xi^m = u' \eta^n \xi^s$ where $u,u' \in U_T$ and $\eta$ is a prime element of $F$ at $S$ and $m = ne + s$ with $0 \leq s < e$.
Let $x_1, \ldots, x_f$ be the elements of $F(S)$ such that $\rho(u') = \sum x_i a_i$ and let $y = \sum j(x_i) j(a_i) \in V$.
Now $x  -y \eta^n \xi^s \in \mm_T^{m+1}$ since $u' - y \in \mm_T$; so there exists an element $v_1 \in V$ such that $\ord_T(x - v_1) > \ord_T x$.
So by induction there is a Cauchy sequence $\{v_i\}$ in $V$ which converges to $x$, so $x \in V$ and $B$ generates $E$ over $F$.

Suppose there is a non-trivial linear relation between the elements of $D$ over $F$.
We can assume without loss of generality that it is of the form
\begin{equation}
\label{4.6.6.1}
\sum d_{r,s} j(a_r) \xi^s = 0
\end{equation}
where $d_{r,s} \in R_S$ and for some $r,s$ we have $\ord_S d_{r,s} = 0$.

Let $n$ be the least integer such that $\ord_S d_{m,n} = 0$ for some $m$.
For any $s \not= n$, if $d_{r,s} \not= 0$ then we have
\begin{equation}
\label{4.6.6.2}
\ord_T( d_{r,s} j(a_r) \xi^s ) = s + \ord_S d_{r,s} \cdot e > n.
\end{equation}
Multiplying \eqref{4.6.6.1} by $\xi^{-n}$ and applying $\rho$ we obtain $\sum_r \rho(d_{r,n}) a_r = 0$, so we get $\ord_S d_{r,n} > 0$ for all $r$, which is a contradiction, so the elements of $D$ are linearly independent over $F$.
\end{proof}


\begin{coro}
\label{4.6.7}
If $E/F$ is a finite separable extension and $S$ is a discrete spot on $F$ then
\[
\sum_{T \mid S} e_{T \mid S} \cdot f_{T \mid S} = \deg E/F.
\]
\end{coro}

\begin{proof}
Immediate from \ref{4.6.6} and \ref{4.2.10}.
\end{proof}


\begin{rema}
\label{4.6.8}
Under the conditions of \ref{4.6.6}, the subspace $K$ of $E$ generated by $\operatorname{SMR}(T)$ is clearly a subfield of $E$ whose residue field is also $E(T)$.
The ramification index of $K$ over $F$ is 1.
This is a basic ingredient and starting point for class-field theory,\footnote{\fnfize} along with the relation between the Galois groups of $K/F$ and $K(T)/F(S)$.
\end{rema}



\subsection{Todo}
\label{ch:4.7}

\section{Global fields}
\label{ch:5}

\section{Algebraic curves}
\label{ch:6}

\section{Riemann surfaces}
\label{ch:7}

\subsection{Appendix: The fundamental group}
\label{ch:7.8}

\section{Elliptic curves}
\label{ch:8}

\begin{exam}
\label{8.3.2}
\end{exam}

\section{Varieties}
\label{ch:9}

\subsection{Cycles}
\label{ch:9.11}


\section{Complex manifolds}
\label{ch:10}

\section{Hodge theory}
\label{ch:11}

\section{Schemes}
\label{ch:12}





\bibliographystyle{plainurl}
\bibliography{main}


\end{document}
