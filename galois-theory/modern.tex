\documentclass[11pt]{article}

\usepackage{lmodern}
%\linespread{1.1}
\usepackage[utf8]{inputenc}
\usepackage[T1]{fontenc}

\usepackage[normalem]{ulem}
\usepackage{textcomp}
\usepackage{hyperref}
\usepackage{tikz-cd}

\usepackage{amsmath}
\usepackage{amssymb}
\usepackage{amsthm}

\newtheorem{theo}{Theorem}[section]
\newtheorem{prop}[theo]{Proposition}
\newtheorem*{lemm}{Lemma}
\newtheorem{coro}[theo]{Corollary}
\newtheorem*{coro*}{Corollary}
\theoremstyle{definition}
\newtheorem{defi}[theo]{Definition}
\newtheorem{exam}[theo]{Example}
\newtheorem{exer}[theo]{Exercise}
\newtheorem*{rema}{Remark}

\newcommand{\kk}[1]{\mathbb{#1}}
\newcommand{\cc}[1]{\mathcal{#1}}

\def\eps{\varepsilon}
\def\empty{\varnothing}

\def\ov#1{\overline{#1}}

\def\CC{\mathbf{C}}
\def\EE{\mathcal{E}}
\def\FF{\mathcal{F}}
\def\NN{\mathbf{N}}
\def\RR{\mathbf{R}}
\def\QQ{\mathbf{Q}}
\def\ZZ{\mathbf{Z}}
\def\PP{\mathbf{P}}

\DeclareMathOperator{\Span}{Span}
\DeclareMathOperator{\Hom}{Hom}
\DeclareMathOperator{\End}{End}
\DeclareMathOperator{\Aut}{Aut}
\DeclareMathOperator{\Ker}{Ker}
\DeclareMathOperator{\Img}{Im}
\DeclareMathOperator{\sgn}{sgn}
\DeclareMathOperator{\GL}{GL}
\DeclareMathOperator{\ord}{ord}
\DeclareMathOperator{\len}{len}
\DeclareMathOperator{\id}{id}

\author{Gunnar Þór Magnússon}
\date{\today}
\title{Linear algebra done right}


\begin{document}

\maketitle

\begin{abstract}
We review the basics of (mostly) finite-dimensional linear algebra the right way.
\end{abstract}

\section{Vector spaces}


We'll fix a field $k$ to work over once and for all.
This could be something familiar like $\QQ$, $\RR$ or $\CC$, a finite field like $\ZZ/p \ZZ$ where $p$ is prime, the algebraic closure of $\QQ$, the field $k(X)$ of rational functions in one variable over a field $k$, or something else.
There are a lot of fields, some big, some small.


\begin{defi}
A \emph{vector space} over $k$ is an abelian group $V$ with an action of $k$.
This means we can multiply an element of $V$ by an element of $k$ and get
another element of $V$, and that for $v,w \in V$ and $x,y \in k$ we have
\begin{align*}
(x + y) v &= xv + yv,
          &
x(yv) &= (xy) v,
\\
x(v + w) &= xv + xw,
      &
1 v &= v.
\end{align*}
\end{defi}


Some things follow easily, like that $0v = 0$ for all $v \in V$:
We have $v = 1v = (0 + 1)v = 0v + 1v = 0V + v$, so $0v = 0$.


\begin{exam}
If $I$ is an index set then the set $k^I = \{ (x_i)_{i \in I} \mid x_i \in k\}$ is a vector space if we define addition and multiplication component by component.
When $I$ has size $n$ we just write $k^n$.
If $X$ is any set then the set $\Hom(X,k) = \{ f : X \to k \}$ of functions from $X$ to $k$ is a vector space over $k$.
This is of course the same example as $k^X$.
\end{exam}

\begin{exam}
If $k \subset \ell$ is a field extension then $\ell$ is a vector space over $k$.
Familiar examples are $\RR \subset \CC$ or $\QQ \subset \RR$.
\end{exam}

\begin{exam}
If $M$ is a smooth manifold the cohomology group $H^k(M, \RR)$ are vector spaces over the real numbers.
\end{exam}

\begin{exam}
If $X$ is a complex manifold and $\cc F$ a sheaf over $X$ then the cohomology groups $H^k(X, \cc F)$ are complex vector spaces over $X$.
By considering schemes over a field $k$ and sheaves over them we get vector spaces over that field $k$.
\end{exam}

\begin{exam}
If $R$ is a commutative ring with unit and $m \subset R$ is a maximal ideal then the quotient $R/m$ is a field.
For example, consider $k[X]$ and the ideal generated by an irreducible polynomial $p$.
\end{exam}


\begin{defi}
Let $V$ be a vector space.
If $S \subset V$ is again a vector space, we say $S$ is a \emph{subspace} of $V$.
\end{defi}

\begin{exer}
A set $S \subset V$ is a subspace if the following conditions all hold:
\begin{enumerate}
\item
$0 \in S$.

\item
If $v,w \in S$ then $v + w \in S$.

\item
If $x \in k$ and $v \in S$ then $xv \in S$.
\end{enumerate}
\end{exer}


\begin{exer}
Let $X$ be a topological space.
The space of continuous functions with values in $k$ is a subspace of the space of all functions from $X$ to $k$.
\end{exer}

\begin{exam}
Let $X$ be a complex manifold.
As every holomorphic $p$-form is $d$-closed, the Dolbeault cohomology group $H^0(X, \Omega^p_X)$ is a subspace of the de Rham cohomology group $H^p(X,\CC)$.
\end{exam}

\begin{exam}
Let $X$ be a K\"ahler manifold.
By Hodge theory, the Dolbeault cohomology groups $H^{p,q}(X,\CC)$ are subspaces of the de~Rham cohomology groups $H^{p+q}(X,\CC)$.
\end{exam}

\begin{exer}
The real numbers $\RR$ are a subspace of the complex numbers $\CC$ as vector spaces over the rational numbers $\QQ$.
\end{exer}

\begin{exer}
Let $V$ be a vector space and $S = \{x\} \subset V$ a set containing a single point.
Then $S$ is a subspace if and only if $x = 0$.
\end{exer}

\begin{exer}
Consider the plane $k^2$ and the union of the lines defined by the $x$ and $y$ axes.
This is not a subspace of $k^2$.
\end{exer}


\begin{exer}
Let $V$ be a vector space and $(S_i)_{i \in I}$ a family of subspaces of $V$.
Show that $\bigcap_{i \in I} S_i$ is again a subspace of $V$.
\end{exer}


\begin{exer}
Let $V$ be a vector space and $S_1, S_2$ subspaces of $V$.
Show that the set
\[
S_1 + S_2 := \{ s_1 + s_2 \mid s_1 \in S_1, s_2 \in S_2 \}
\]
is a subspace of $V$ that contains both $S_1$ and $S_2$.
Show that it is in fact the smallest such subspace.
\end{exer}


\begin{exer}
Let $V$ be a vector space and $X \subset V$ a subset (not a subspace!).
Define the \emph{span of $X$} to be
\[
\operatorname{Span} X 
= \biggl\{ \sum_{i=1}^n \lambda_i x_i \Bigm|
\lambda_1,\ldots,\lambda_n \in k, \ x_1, \ldots, x_n \in X \biggr\},
\]
that is, the set of all finite linear combinations of elements of $X$.
Show that $\operatorname{Span} X$ is a subspace of $V$, and is the smallest subspace of $V$ that contains $X$.
\end{exer}


\section{Linear morphisms}


\begin{defi}
A map $f : V \to W$ between vector spaces is a \emph{linear morphism} if for all $v,w \in V$ and $x \in k$ we have
\[
f(v + w) = f(v) + v(w)
\quad\text{and}\quad
f(xv) = xf(x).
\]
\end{defi}

\begin{exer}
Prove the following statements.
\begin{enumerate}
\item
If $V$ and $W$ are vector spaces then the constant zero map is linear.

\item
The identity map on any vector space is linear.

\item
If $f : U \to V$ and $g : V \to W$ are linear morphisms then $gf : U \to W$ is a linear morphism.

\item
If $f : V \to W$ is a bijective linear morphism then $f^{-1} : W \to V$ is a linear morphism.
\end{enumerate}
\end{exer}



\begin{prop}
Let $V$ and $W$ be vector spaces.
The set $\Hom(V,W)$ of linear morphisms $f : V \to W$ is again a vector space.
\end{prop}

\begin{proof}
The linear morphisms form an abelian group under addition of elements in $W$ with the constant zero map as their neutral element.
If $x \in k$ and $f \in \Hom(V,W)$ we define $(x f)(v) = x f(v)$ for $v \in V$.
This satisfies the vector space axioms by inspection on values.
\end{proof}



In general there are many linear morphisms between any two vector spaces.
However, if we don't know that much about the spaces it may be difficult to write down some morphisms between them.
Often we only care about some subset of morphisms that arise from the surrounding context.

\begin{exam}
Let $f : M \to N$ be a smooth map between smooth manifolds.
Then $f$ induces a linear morphism of cohomology groups $f^* : H^k(N, \RR) \to H^k(M, \RR)$.
\end{exam}

In this example we probably care much more about the maps that arise as pullbacks by maps between the manifolds than about an arbitrary linear map.


In one case it's easy to write down many linear morphisms between vector spaces:

\begin{exam}
Consider the vector spaces $k^n$ and $k^m$.
They are generated by the vectors $e_i$ that have 1 in the $i$-th position and 0 everywhere else.
Let $f : k^n \to k^m$ be a linear morphism.
For every $j = 1, \ldots, n$ we can write $f(e_j) = \sum_{i=1}^m a_{ij} e_i$ for some scalars $a_{ij}$.
These fit into the matrix
\[
\begin{pmatrix}
a_{11} & a_{12} & \ldots & a_{1n}
\\
a_{21} & a_{22} & \ldots & a_{2n}
\\
\vdots & \ddots & \ddots & \vdots
\\
a_{m1} & a_{m2} & \ldots & a_{mn}
\end{pmatrix}
\]
and the morphism $f$ is uniquely determined by this matrix.
If we prescribe such a matrix, we also get a linear morphism from $k^n$ to $k^m$ by defining $f(\sum_{j=1}^n x_j e_j) = \sum_{j=1}^{n} \sum_{i=1}^m x_j a_{ij} e_i$.
\end{exam}


\begin{prop}
Let $f : V \to W$ be a linear morphism.
\begin{enumerate}
\item
If $S \subset V$ is a subspace then $f(S) \subset W$ is a subspace.

\item
If $S \subset W$ is a subspace then $f^{-1}(S) \subset V$ is a subspace.
\end{enumerate}
\end{prop}

\begin{proof}
First let $S$ be a subspace of $V$.
Since $f(0) = f(0 v) = 0 f(v) = 0$ for any $v \in S$ we see that $0 \in f(S)$.
Let $v,w \in f(S)$.
Then there are $v',w' \in S$ such that $f(v') = v$ and $f(w') = w$, so $v + w = f(v') + f(w') = f(v' + w') \in f(S)$.
If $x\in k$ then $xv = xf(v') = f(xv') \in f(S)$, so $f(S)$ is a subspace.

Let now $S$ be a subspace of $W$.
As we saw $f(0) = 0$ so $0 \in f^{-1}(S)$.
Let $v,w \in f^{-1}(S)$.
Then $f(v + w) = f(v) + f(w) \in S$, so $v + w \in f^{-1}(S)$.
If $x \in k$ then $f(xv) = xf(v) \in S$, so $xv \in f^{-1}(S)$, and $f^{-1}(S)$ is a subspace.
\end{proof}


\begin{defi}
If $f : V \to W$ is a linear morphism, its \emph{kernel} is $\Ker f := f^{-1}(0) \subset V$ and its \emph{image} is $\Img f := f(V) \subset W$.
\end{defi}

The image and kernel of a morphism are subspaces by the above proposition.

\begin{prop}
A linear morphism $f : V \to W$ is injective if and only if $\Ker f = 0$.
\end{prop}

\begin{proof}
Suppose $f$ is injective and $f(v) = 0$.
We know that $f(0) = 0$, so by injectivity we have $v = 0$.

Suppose that $\Ker f = 0$ and that $f(v) = f(w)$.
Then $0 = f(v) - f(w) = f(v-w)$, so $v-w \in \Ker f$, but then $v = w$.
\end{proof}


\begin{defi}
A sequence of linear morphisms
\[
U \stackrel{f}{\longrightarrow} V \stackrel{g}{\longrightarrow} W
\]
is \emph{exact} if $\Img f = \Ker g$.
A diagram
\[
\cdots \longrightarrow V \longrightarrow W \longrightarrow \cdots
\]
of vector spaces is exact if it is exact at every step.
An exact sequence of the form
\[
0 \longrightarrow U \longrightarrow V \longrightarrow W \longrightarrow 0
\]
is called a \emph{short exact sequence}.
\end{defi}


\begin{prop}
Let $f : V \to W$ be a linear morphism.
\begin{enumerate}
\item
The morphism $f$ is injective if and only if the sequence $0 \to V \stackrel{f}{\to} W$ is exact.

\item
The morphism $f$ is surjective if and only if the sequence $V \stackrel{f}{\to} W \to 0$ is exact.
\end{enumerate}
\end{prop}

\begin{proof}
First suppose $f$ is injective.
The only morphism from $0$ is the zero map, whose image is $0 = \Ker f$, so the sequence is exact.
Conversely, if the sequence is exact we have $\Ker f = 0$, so $f$ is injective.

Now suppose $f$ is surjective.
The only morphism to $0$ is the one that maps everything to $0$, so its kernel is $W = \Img f$ and the sequence is exact.
Conversely, if the sequence is exact we have $\Img f = W$.
\end{proof}


\begin{prop}
Let $f : V \to W$ be a linear morphism.
Then the sequence
\[
\begin{tikzcd}
0 \ar[r] & \Ker f \ar[r] & V \ar[r,"f"] & \Img f \ar[r] & 0
\end{tikzcd}
\]
is exact.
\end{prop}

\begin{proof}
The inclusion $j: \Ker f \to V$ is injective and the map $f : V \to \Img f$ is surjective, so we ony have to check exactness in the middle.
But $\Img j = \Ker f$ by definition, so the sequence is exact.
\end{proof}




\section{New spaces from old ones}


Given vector spaces $V$ and $W$ the underlying sets have a product $V \times W$.
We can give this set the product of a vector space and get a new one.
Once we have an infinite number of vector spaces there turn out to be two ways of forming a product from them, depending on whether we want the result to be a source or target of linear morphisms.
This way we get products and sums (or coproducts) of vector spaces.
These agree when we only have a finite number of spaces, which is what we'll most often deal with, but it's not much trouble to construct these for an arbitrary number of spaces.


\begin{prop}[Products of vector spaces]
\label{prop:product}
Let $I$ be an index set and $(V_i)_{i\in I}$ a family of vector spaces.
There exists a vector space $\prod_{i\in I} V_i$ and linear morphisms $\pi_i : \prod_{i \in I} V_i \to V_i$ such that the following universal property holds:

If $S$ is a vector space and $f_i : S \to V_i$, $i \in I$, are linear morphisms, then there exists a unique linear morphism $f : S \to \prod_{i \in I} V_i$ such that the diagram
\[
\begin{tikzcd}
S \ar[dr,"f_i"] \ar[r,"f"] & \displaystyle\prod_{i \in I} V_i \ar[d,"\pi_i"]
\\
             & V_i
\end{tikzcd}
\]
commutes for every $i \in I$.

Furthermore, if $W$ is any other space that satisfies the universal property, it is isomorphic to $\prod_{i \in I} V_i$.
\end{prop}

\begin{proof}
To define the product we start with the set $\prod_{i \in I} V_i$ of families $(v_i)_{i \in I}$ where $v_i \in V_i$ for all $i \in I$.
These families can be added and multiplied by elements of the field, and thus form a vector space.
We define the projecions by $\pi_i((v_i)_{i \in I}) = v_i$.
These are clearly linear morphisms.

Now let $S$ be a vector space and $f_i : S \to V_i$ linear morphisms.
We define a map $f : S \to \prod_{i \in I} V_i$ by $f(s) = (f_i(s))_{i \in I}$.
This is a linear morphism, as the reader may check, and satisfies $\pi_i f = f_i$ for all $i$.
If $g$ is any other morphism that also satisfies $\pi_i g = f_i$ for all $i$, then $\Img(f - g) \subset \Ker \pi_i$ for all $i$.
But $\bigcap_{i \in I} \Ker \pi_i = 0$, so $g = f$.

Suppose $W$ is another vector space with morphisms $\rho_i : W \to V_i$ that satisfies the same universal property.
Applying the universal property twice we get morphisms
\begin{gather*}
f : \prod_{i \in I} V_i \to W,
\quad
\pi_i = \rho_i f \text{ for all $i \in I$,}
\\
g : W \to \prod_{i \in I} V_i
\quad
\rho_i = \pi_i g  \text{ for all $i \in I$.}
\end{gather*}
Then we get $gf : \prod_{i \in I} V_i \to \prod_{i \in I} V_i$ that satisfies $\pi_i gf = \rho_i f = \pi_i$ for all $i$.
But taking the product of the identity morphisms $V_i \to V_i$ we see that the identity morphism $\prod_{i \in I} V_i \to \prod_{i \in I} V_i$ also satisfies this property, so $gf = \id$.
Similarly $\rho_i fg = \pi_i g = \rho_i$ for all $i$, and the identity map $W \to W$ also satisfies that property, so $fg = \id$.
Therefore $g = f^{-1}$ and $f$ is an isomorphism.
\end{proof}


\begin{prop}[Sums of vector spaces]
\label{prop:coproduct}
Let $I$ be an index set and $(V_i)_{i\in I}$ a family of vector spaces.
There exists a vector space $\sum_{i\in I} V_i$ and linear morphisms $\iota_i : V_i \to \sum_{i \in I} V_i$ such that the following universal property holds:

If $Q$ is a vector space and $f_i : V_i \to Q$, $i \in I$, are linear morphisms, then there exists a unique linear morphism $f : \sum_{i \in I} V_i \to Q$ such that the diagram
\[
\begin{tikzcd}
V_i \ar[dr,"f_i"] \ar[r,"\iota_i"] & \displaystyle\sum_{i \in I} V_i \ar[d,"f"]
\\
             & Q
\end{tikzcd}
\]
commutes for every $i \in I$.

Furthermore, if $W$ is any other space that satisfies the universal property, it is isomorphic to $\sum_{i \in I} V_i$.
\end{prop}


\begin{proof}
Let $\sum_{i \in I} V_i$ be the subset of $\prod_{i \in I} V_i$ of sequences with finite support, that is, sequences $(v_i)_{i \in I}$ such that the set $\{ i \in I \mid v_i \not= 0 \}$ is finite.
The sum of two sequences with finite support again has finite support and multiplying such a sequence by a scalar does not change its support, so this is a subspace.
We define maps $\iota_i : V_i \to \sum_{i \in I} V_i$ by $\iota_i(v) = (v_j)_{j \in I}$ where $v_j = v$ if $j = i$ and $v_j = 0$ otherwise.
These are linear morphisms by inspection.

Let $Q$ be a vector space and $f_i : V_i \to Q$ linear morphisms.
We define a map 
\[
f : \sum_{i \in I} V_i \to Q,
\quad
f((v_i)_{i \in I}) = \sum_{i \in I} f_i(v_i).
\]
This is actually defined because the sum is finite for any element of $\sum_{i \in I} V_i$, and satisfies $f\iota_i(v) = f((v_j)_{j \in I}) = f_i(v)$ for any $i$.
If $g$ is any other linear map that satisfies this property, and $(v_i) \in \sum_{i \in I} V_i$, then
\[
g((v_i)_{i \in I})
= \sum_{i \in I} g \iota_i(v_i)
= \sum_{i \in I} f_i(v_i)
= f((v_i)_{i \in I}).
\]
The proof of uniqueness of the sum is very similar to the one of uniqueness of the product.
\end{proof}


When the index set $I$ is finite the product and sum of a family of vector
spaces are clearly the same space.
In this case the product (or sum) is written $\bigoplus_{i \in I} V_i$, or $V_1 \oplus V_2 \oplus \cdots \oplus V_n$.

When the $V_i$ are subspaces of a single space $V$ and $V_i \cap V_j = 0$ for $i \not= j$ people often write $V_1 \oplus \cdots \oplus V_n$ for the subspace $V_1 + \cdots + V_n$ of $V$.
Officially the sum (product) $V_1 \oplus \cdots \oplus V_n$ is of course not a subspace of $V$ but of $\bigoplus_{i=1}^n V$.
This can kind of be justified:
For any subspaces $S_1, S_2$ of $V$ we have an exact sequence
\[
0 \longrightarrow S_1 \cap S_2 \longrightarrow S_1 \oplus S_2 \longrightarrow S_1 + S_2 \longrightarrow 0
\]
where the first arrow is $v \mapsto (v,v)$ and the second is $(v,w) \mapsto v - w$.
If $S_1 \cap S_2 = 0$ the second arrow is an isomorphism, so $S_1 \oplus S_2$ embeds inside $V$.


This is unfortunate, but unfortunately rather useful as notation, and we'll trust the reader to infer from context what is being meant.
In almost all cases of a sum or product of disjoint subsets of a common ambient space, what is meant is the interpretation of the sum or product as a subspace of the same space.


\begin{prop}
Let $I$ be an index set, $(V_i)_{i\in I}$ a family of vector spaces, and $W$ a vector space.
Then
\[
\Hom\biggl(\coprod_{i\in I} V_i, W\biggr)
= \prod_{i\in I} \Hom(V_i, W).
\]
\end{prop}

\begin{proof}
Let $f : \coprod_{i\in I} V_i \to W$ be a linear morphism.
Then we get a family of morphisms $(f_i)_{i \in I}$ defined by $f_i = f \iota_i$, which gives an element of $\prod_{i\in I} \Hom(V_i, W)$.
This defines a map between the two spaces, which is linear by applications of the universal property of the product.

There is also a map going the other way:
Let $(f_i)_{i \in I}$ be a family of morphisms $f_i : V_i \to W$.
\end{proof}









\begin{prop}
\[
k^n \oplus k^m \cong k^{n+m}.
\]
\end{prop}

\begin{proof}
The space $k^n \oplus k^m$ is the set of pairs $(v,w)$ where $v = (v_1, \ldots, v_n) \in k^n$ and $w = (w_1, \ldots, w_m) \in k^m$.
We can define $f : k^n \oplus k^m \to k^{n+m}$ by $f(v,w) = (v_1, \ldots, v_n, w_1, \ldots, w_m)$.
This is clearly an injective map and readily seen to be surjective, so it is an isomorphism.
\end{proof}



\begin{prop}[Quotients]
\label{prop:quotient}
Let $V$ be a vector space and let $S \subset V$ be a subspace.
There exists a vector space $V/S$ and a linear morphism $\pi : V \to V/S$ such that the following universal property holds:

If $Q$ is a vector space and $f : V \to Q$ is a linear morphism such that $S \subset \Ker f$ then there is a unique linear morphism $g : V/S \to Q$ such that the diagram
\[
\begin{tikzcd}
V \ar[d,"\pi"] \ar[dr,"f"] &
\\
V/S \ar[r,"g"] & Q
\end{tikzcd}
\]
commutes.

If $W$ is any other vector space that satisfies the same universal property, then $W$ is isomorphic to $V/S$.
\end{prop}

\begin{proof}
We define an equivalence relation $\sim$ on $V$ by $v \sim w$ if and only if $v-w \in S$.
Let $V/S$ be the set of equivalence classes $V/\sim$ and $\pi : V \to V/S$ the map that sends an element to its equivalence class.
We define addition and multiplication on $V/S$ by demanding they satisfy
\[
\pi(v) + \pi(w) = \pi(v + w)
\quad\text{and}\quad
x \pi(v) = \pi(xv).
\]
These operations are well defined:
Let $v \in \pi(v)$ and $w \in \pi(w)$.
If $v - v' \in S$ and $w - w' \in S$ we want to see that $\pi(v' + w') = \pi(v + w)$.
But
\[
(v + w) - (v' + w')
= (v - v') + (w - w')
\in S
\]
so that holds.
Similarly we would like to see that $\pi(xv) = \pi(xv')$, and
\[
xv - xv' = x(v - v') \in S,
\]
so that is true.
These definitions also make $\pi$ into a linear morphism.

Now suppose $Q$ is a vector space and $f : V \to Q$ a morphism such that $S \subset \Ker f$.
We define a map $g : V/S \to Q$ by $g(\pi(v)) = f(v)$ for any $v \in V$.
If $\pi(v) = \pi(v')$ then $v - v' \in S$ and $f(v') = f(v)$ by hypothesis, so this is well defined.
It is also a linear morphism.

If $W$ is a vector space with a morphism $\rho : V \to W$ that satisfies the same universal property we get two linear morphisms $f : V/S \to W$ such that $f \pi = \rho$ and $g : W \to V/S$ such that $g \rho = \pi$.
Then $gf : V/S \to V/S$ is a linear morphism such that $gf\pi = g\rho = \pi$, but $\id : V/S \to V/S$ is also such a morphism so $gf = \id$.
Similarly we see that $fg = \id$, so $V/S \cong W$.
\end{proof}



\begin{coro}[First isomorphism theorem]
\label{prop:first_isomorphism_thm}
Let $f : V \to W$ be a linear morphism.
Then $V / \Ker f \cong \Img f$.
\end{coro}

\begin{proof}
By Proposition~\ref{prop:quotient} there exists a unique linear morphism such that the diagram
\[
\begin{tikzcd}
V \ar[d,"\pi"] \ar[dr,"f"] &
\\
V / \Ker f \ar[r,"g"] & \Img f
\end{tikzcd}
\]
commutes.
As $f$ is surjective and $f = g\pi$ then $g$ is surjective.
Suppose that $g(w) = 0$ and let $v \in V$ be such that $\pi(v) = w$.
Then $f(v) = 0$ so $v \in \Ker f$ and $w = \pi(v) = 0$, so $g$ is injective.
\end{proof}





\paragraph{Splitting exact sequences}


This paragraph contains some workhorse results we apply again and again.


\begin{defi}
A short exact sequence
\[
\begin{tikzcd}
0 \ar[r] & S \ar[r,"j"] & V \ar[r,"q"] & Q \ar[r] & 0
\end{tikzcd}
\]
\emph{splits} if there is an isomorphism $f : V \to S \oplus Q$ such that the diagram
\[
\begin{tikzcd}
0 \ar[r] & S \ar[r,"j"] \arrow[d,equal] & V \ar[r,"q"] \ar[d,"f"] & Q \ar[r] \ar[d,equal] & 0
\\
0 \ar[r] & S \ar[r,"\iota_1"] & S \oplus Q \ar[r,"\pi_2"] & Q \ar[r] & 0
\end{tikzcd}
\]
is commutative, where $\iota_1 : S \to S \oplus Q$ and $\pi_2 : S \oplus Q \to Q$ are the obvious injection and projection.
\end{defi}


\begin{prop}
The following are equivalent.
\begin{enumerate}
\item
The sequence
\[
\begin{tikzcd}
0 \ar[r] & S \ar[r,"j"] & V \ar[r,"q"] & Q \ar[r] & 0
\end{tikzcd}
\]
splits.

\item
There exists a left split $\ell : V \to S$, that is, a linear morphism such that $\ell j = \id_S$.

\item
There exists a right split $r : Q \to V$, that is, a linear morphism such that $qr = \id_Q$.
\end{enumerate}
\end{prop}


\begin{proof}
Suppose the sequence splits.
Then there is an isomorphism $f : V \to S \oplus Q$ that makes the necessary diagram commute.
We set $\ell = \pi_1 f$, where $\pi_1 : S \oplus Q \to S$ is the projection.
Then we have $\iota_1 = fj$ so $\id_S = \pi_1 \iota_1 = \pi_1 f j = \ell j$ and $\ell$ is a left split.

Suppose we have a left split $\ell : V \to S$.
Suppose that $q(v) = 0$ for some $v \in \Ker \ell$.
Then $v \in \Ker q = \Img j$ so there is an $s \in S$ such that $v = j(s)$.
But then $s = \ell j(s) = 0$, so $v = 0$ and $q : \Ker \ell \to Q$ is injective.
Let $w \in Q$ and pick $v \in V$ such that $q(v) = w$.
Then $v - j \ell (v)$ is in $\Ker \ell$ and $q(v - j \ell (v)) = q(v)$, so $q : \Ker \ell \to Q$ is an isomorphism.
We set $r = q^{-1} : Q \to \Ker \ell \hookrightarrow V$, and have $qr = \id_Q$ so $r$ is a right split.

Suppose finally that we have a right split $r : Q \to V$.
We define a map $g : S \oplus Q \to V$ by $s \oplus w \mapsto j(s) + r(w)$.
If $g(s \oplus w) = 0$ then $0 = qg(s \oplus w) = q(j(s) + r(w)) = w$, so $g(s \oplus w) = j(s)$, but $j$ is injective so we also have $s = 0$ and $g$ is injective.
Let $v \in V$.
Then $v - rq(v) \in \Ker q = \Img j$, so there is $s \in S$ such that $j(s) = v - rq(v)$.
But then $g(s \oplus q(v)) = v - rq(v) + rq(v) = v$, so $g$ is surjective.
Then $f = g^{-1}$ splits the exact sequence.
\end{proof}




\begin{prop}
Any short exact sequence
\[
\begin{tikzcd}
0 \ar[r] &
S \ar[r] &
V \ar[r,"q"] &
Q \ar[r] &
0
\end{tikzcd}
\]
of vector spaces splits.
\end{prop}

\begin{proof}
A splitting of the sequence is equivalent to the existence of a right split $f : Q \to V$.
We will construct one by considering the set $P$ of pairs $(W, g)$ where $W \subset Q$ and $g : Q \to V$ is a right split.

First note that $P$ is nonempty, because the zero space and zero map are in $P$.
Next let $(W_i, g_i)_{i \in I}$ be an ascending chain of elements of $P$.
Then $\bigcup_{i \in I} W_i$ is a subspace of $Q$:
If $v,w \in \bigcup_{i \in I} W_i$ then there is some $i$ such that $v,w \in W_i$ and therefore $xv \in W_i$ and $v + w$ in $W_i$ for all $x \in k$.
There is also a map $g : \bigcup_{i \in I} W_i \to V$ defined by $g(v) = g_i(v)$ for any $i$ such that $v \in V_i$ that satisfies $qg = \id$.

Applying Zorn's lemma we find a maximal element $(W,g)$ in $P$.
Suppose $W \not= Q$.
Pick $z \in Q \setminus W$ and let $L = k z$.
Then $W \cap L = 0$ so $W \oplus L \subset Q$.
Now find some $v \in V$ such that $q(v) = z$ and define
\[
\hat g(w \oplus kz) = g(w) + k v.
\]
Then
\[
q \hat g(w + kz) 
= q(g(w)) + k q(v)
= w + kz
\]
so $\hat g$ is again a lift of $g$, in contradiction of $(W,g)$ being maximal.
Therefore $W = Q$ and $g$ is the right split we want.
\end{proof}


\begin{rema}
Later we'll see the notion of dimension of a vector space.
It is possible to prove this result for spaces of finite dimension without using Zorn's lemma or other results equivalent to the axiom of choice.
The argument is essentially the same, and would in fact only require $Q$ to be of finite dimension.
\end{rema}




\section{Duality}


\begin{defi}
The \emph{dual} of a vector space $V$ is the space $V^* = \Hom(V,k)$.
\end{defi}

\begin{prop}
If $f : V \to W$ is a linear morphism, there is an associated linear morphism $f^* : W^* \to V^*$ defined by $f^*(\lambda)(v) = \lambda(f(v))$.
\end{prop}

\begin{proof}
If $x,y \in k$, $\lambda, \mu \in W^*$ and $v \in V$ we have
\begin{align*}
f^*(x \lambda + y \mu)(v)
= (x \lambda + y \mu)(f(v))
&= x \lambda(f(v)) + y \mu(f(v))
\\
&= x f^*(\lambda)(v) + y f^*(\mu)(v)
\end{align*}
so $f^*$ is linear.
\end{proof}


\begin{prop}
If $V$ is a vector space, then $\id_V^* = \id_{V^*}$.
\end{prop}

\begin{proof}
Take $\lambda \in V^*$. Then $(\id_V)^* \lambda(v) = \lambda(\id_V v) = \lambda(v)$ for all $v \in V$.
\end{proof}


\begin{prop}
If $f : V \to W$ and $g : W \to Z$ are linear morphisms, then $(gf)^* = f^* g^*$.
\end{prop}

\begin{proof}
Let $\lambda \in Z^*$ and $v \in Z$.
Then $(gf)^*(\lambda)(v) = \lambda(gf(v)) = \lambda(g(f(v))) = f^* g^* \lambda(v)$.
\end{proof}

\begin{coro}
If $f : V \to W$ is an isomorphism, then $(f^{-1})^* = (f^*)^{-1}$.
\end{coro}



\begin{prop}
If
\[
\begin{tikzcd}
V \ar[r,"f"] & W \ar[r,"g"] & Z
\end{tikzcd}
\]
is exact, then
\[
\begin{tikzcd}
Z^* \ar[r,"g^*"] & W^* \ar[r,"f^*"] & V^*
\end{tikzcd}
\]
is exact.
\end{prop}

\begin{proof}
If the sequence is exact then $gf = 0$, so $0 = f^* g^*$.
Therefore $\Img g^* \subset \Ker f^*$.

Showing that $\Ker f^* \subset \Img g^*$ is slightly more involved.
The basic idea is to note that this condition lets us define an element of $(\Img g)^*$ that would do the trick, and then extend that map to the whole of $Z$.

Let then $\lambda \in \Ker f^*$, so $\lambda(f(v)) = 0$ for all $v \in V$.
Then $\Ker g = \Img f \subset \Ker \lambda$.
By Proposition~\ref{prop:quotient} there exists a unique linear morphism $\hat \lambda : W/\Ker g \to k$ such that the diagram
\[
\begin{tikzcd}
W \ar[dr,"\lambda"] \ar[d,"\pi"] &
\\
W / \Ker g \ar[r,"\hat\lambda"] & k
\end{tikzcd}
\]
commutes.
By Corollary~\ref{prop:first_isomorphism_thm} there is also a unique linear morphism $\hat g : W / \Ker g \to \Img g$ such that
\[
\begin{tikzcd}
W \ar[dr,"g"] \ar[d,"\pi"] &
\\
W / \Ker g \ar[r,"\hat g"] & \Img g
\end{tikzcd}
\]
commutes, which is an isomorphism.
As the sequence
\[
\begin{tikzcd}
0 \ar[r] & \Img g \ar[r] & Z \ar[r] & Z / \Img g \ar[r] & 0
\end{tikzcd}
\]
is exact there exists a splitting $\theta : Z \to \Img g \oplus Z / \Img g$.
We define $\mu \in Z^*$ by following the diagram
\[
\begin{tikzcd}
Z \ar[r,"\theta"] &
\Img g \oplus Z / \Img g \ar[r,"\pi_1"] &
\Img g \ar[r,"{\hat g}^{-1}"] &
W / \Ker g \ar[r,"\hat \lambda"] &
k.
\end{tikzcd}
\]
Note that
\[
\begin{tikzcd}
W \ar[r,"g"] \ar[rrr,"g",bend right=20] &
Z \ar[r,"\theta"] &
\Img g \oplus Z / \Img g \ar[r,"\pi_1"] &
\Img g
\end{tikzcd}
\]
commutes and recall that $g = \hat g \pi$ so ${\hat g}^{-1} g = \pi$.
Then $g^* \mu = \hat \lambda \pi = \lambda$, so $\lambda \in \Img g^*$.
\end{proof}


\begin{coro}
Let $f : V \to W$ be a linear morphism.
\begin{enumerate}
\item
If $f$ is injective then $f^*$ is surjective.

\item
If $f$ is surjective then $f^*$ is injective.
\end{enumerate}
\end{coro}

\begin{proof}
The morphism $f$ is injective if and only if the sequence
\[
\begin{tikzcd}
0 \ar[r] & V \ar[r,"f"] & W
\end{tikzcd}
\]
is exact.
If so then the sequence
\[
\begin{tikzcd}
W^* \ar[r,"f^*"] & V^* \ar[r] & 0
\end{tikzcd}
\]
is also exact, so $f^*$ is surjective.
The proof of the other statement is similar.
\end{proof}



\begin{prop}
Let $(V_i)_{i \in I}$ be a family of vector spaces.
Then
\[
\biggl( \sum_{i \in I} V_i \biggr)^* \!\! =
\sum_{i \in I} V_i^*.
\]
\end{prop}

\begin{proof}
Let $(\lambda_i)_{i \in I}$ be a family of linear functionals, where $\lambda_i \in V_i^*$.
By Proposition~\ref{prop:coproduct} there exists a unique linear morphism $\lambda : \sum_{i \in I} V_i \to k$ such that $\lambda_i = \lambda \iota_i$ for all $i$.
This defines a linear morphism
\[
\phi : \sum_{i \in I} V_i^* \longrightarrow
\biggl( \sum_{i \in I} V_i \biggr)^*.
\]
If $\phi((\lambda_i)) = 0$ we consider a fixed $j$ and $v \in V_j$ and let $(v_i)_{i \in I}$ be a family such that $v_i = v$ if $i = j$ and $v_i = 0$ otherwise.
Then $\lambda_j(v) = \phi((\lambda_i))(\iota_j(v)) = 0$, which means $\lambda_j = 0$ for all $j$.
Therefore $\phi$ is injective.

Now let $\lambda : \sum_{i \in I} V_i \to k$ be a linear functional.
Setting $\lambda_i = \iota_i^*\lambda$ we get a family of linear functionals, where $\lambda_i \in V_i^*$.
Then $\phi((\lambda_i))$ is a linear functional such that $\lambda_i = \iota_i^* \phi((\lambda_i))$ for all $i$, but so is $\lambda$, so $\phi((\lambda_i)) = \lambda$ by uniqueness.
\end{proof}


\begin{exer}
Let $(V_i)_{i \in I}$ be a family of vector spaces.
Then
\[
\biggl( \prod_{i \in I} V_i \biggr)^* \!\! =
\prod_{i \in I} V_i^*.
\]
\end{exer}




\section{Dimension}



A \emph{flag} in a vector space $V$ is an ascending chain of subspaces $0 \not= V_1 \subset V_2 \subset \cdots \subset V_n \subset V$ where $V_j \not= V_{j+1}$ for $j = 1,\ldots,n-1$.
We write $(V_\bullet) \subset V$ for a flag in $V$.
The \emph{length} of a flag is the number $n$ of elements in it, denoted $\len(V_\bullet)$.

\begin{defi}
The \emph{dimension} of a vector space is
\[
\dim V := \sup \{ \len (V_\bullet) \mid (V_\bullet) \subset V \}.
\]
If $\dim V < \infty$ we say $V$ is finite-dimensional; otherwise it is infinite-dimens\-ional.
\end{defi}


\begin{prop}
Let $V$ and $W$ be vector spaces.
\begin{enumerate}
\item
If $W \subset V$ then $\dim W \leq \dim V$.

\item
If $V$ is finite-dimensional and $W \subsetneq V$ then $\dim W < \dim V$.
\end{enumerate}
\end{prop}

\begin{proof}
Any flag in $W$ is a flag in $V$, so the first point is immediate.
For the second, take a flag $(W_\bullet)$ in $W$ of length $\dim W$.
As $W \not= V$ we can extend the flag to one of length $\dim W+1$ by appending $V$ to its end.
Therefore $\dim W < \dim W + 1 \leq \dim V$.
\end{proof}


\begin{prop}
Let $V$ and $W$ be vector spaces.
\begin{enumerate}
\item
If there is an injective morphism $f : V \to W$ then $\dim V \leq \dim W$.

\item
If there is a surjective morphism $ : V \to W$ then $\dim V \geq \dim W$.
\end{enumerate}
\end{prop}

\begin{proof}
For the first point, let $(V_\bullet)$ be a flag in $V$.
If $f(V_i) = f(V_{i+1})$ for some $i$ then there exists $x \in V_i$ and $y \in V_{i+1} \setminus V_i$ such that $f(x) = f(y)$.
Injectivity entails that $x = y$, which is a contradiction.
Therefore $(f(V_\bullet))$ is a flag in $W$.

For the second point, let $(W_\bullet)$ be a flag in $W$.
For each $W_i$ let $V_i = f^{-1}(W_i)$.
There exists some $y \in W_{i+1} \setminus W_i$ by hypothesis, and thus $x \in V_{i+1}$ such that $f(x) = y$.
Then we cannot have $x \in V_i$, so $(V_\bullet)$ is a flag in $V$.
\end{proof}

\begin{coro}
If $V \cong W$ then $\dim V = \dim W$.
\end{coro}


\begin{theo}
Let
\[
0 \longrightarrow 
S \stackrel{j}{\longrightarrow} 
V \stackrel{q}{\longrightarrow} 
Q \longrightarrow 0
\]
be a short exact sequence of vector spaces.
Then
\[
\dim V = \dim S + \dim Q.
\]
\end{theo}


\begin{proof}
Let $(Q_j)$ be a flag in $Q$ of length $m$, and consider the flag
$(V_j)$ in $V$ where $V_j = q^{-1}(Q_j)$.
As $\Img j = \Ker q$ we have $S \subsetneq V_j$ for all $j$.
If $(S_i)$ is a flag in $S$ of length $n$ we can then prefix it to the flag $(V_j)$ and get a flag of length $n + m$.
Therefore $\dim S + \dim Q \leq \dim V$.

Let now $(V_i)$ be a flag in $V$.
We then get two sequences of subspaces $S_i = V_i \cap S$ and $Q_i = q(V_i)$ in $S$ and $Q$, respectively.
The inclusions at each step there may not be strict, but we claim that there is no $i$ such that $S_i = S_{i+1}$ and $Q_i = Q_{i+1}$:
Suppose there is one and consider the diagram
\[
\begin{tikzcd}
0 \ar[r] & S_i \ar[r] \ar[d,equal] & V_i \ar[r] \ar[d] & Q_i \ar[r] \ar[d,equal] & 0
\\
0 \ar[r] & S_{i+1} \ar[r] & V_{i+1} \ar[r] & Q_{i+1} \ar[r] & 0
\end{tikzcd}
\]
where the arrow $V_i \to V_{i+1}$ is the inclusion.
As the other maps are either the identity or induced by restricting $j$ or $q$ to a subspace the diagram is commutative.
We claim the inclusion above is then actually surjective and thus an isomorphism:\footnote{This follows from the short five lemma, which we didn't prove.}
Pick an element in $V_{i+1}$ and push it down to $Q_{i+1}$.
Move that to $Q_i$ and find an element in $V_i$ that maps to it.
Commutativity means the element we just picked maps to the one we started with modulo an element of $S_{i+1}$.
But $S_{i+1} = S_i$, so adjusting our element in $V_i$ we can map exactly to the original element, so $V_i = V_{i+1}$, in contradiction of $(V_i)$ being a flag.

Now define flags in $S$ and $Q$ by skipping entries in $(S_i)$ and $(Q_i)$ where the inclusions are equalities.
We get flags $(S_j)$ and $(Q_\ell)$ in $S$ and $Q$ that we will denote by $(S \cap V_\bullet)$ and $(q(V_\bullet))$.
By the above discussion we see that $\len(V_\bullet) \leq \len(S \cap V_\bullet) + \len(q(V_\bullet))$, because at every step where $S_i = S_{i+1}$ we have $Q_i \not= Q_{i+1}$ and vice versa.
Then
\begin{align*}
\dim V
&= \sup\{ \len(V_\bullet) \mid (V_\bullet) \subset V \}
\\
&\leq \sup\{ \len(S \cap V_\bullet) + \len(q(V_\bullet)) \mid (V_\bullet) \subset V \}
\\
&\leq \sup\{ \len(S \cap V_\bullet) \mid (V_\bullet) \subset V \} 
+ \sup \{\len(q(V_\bullet)) \mid (V_\bullet) \subset V \}
\\
&\leq \sup\{ \len(S_\bullet) \mid (S_\bullet) \subset S \} 
+ \sup \{\len(Q_\bullet) \mid (Q_\bullet) \subset Q \}
\\
&= \dim S + \dim Q
\end{align*}
so the conclusion holds.
\end{proof}


\begin{exer}
Let $f : V \to W$ be a linear morphism.
Show that
\[
\dim V = \dim \Ker f + \dim \Img f.
\]
\end{exer}

\begin{exer}
Let $U$ and $V$ be subspaces of a vector space $W$.
Show that 
\[
\dim(U + V) = \dim U + \dim V - \dim U \cap V.
\]
\end{exer}


\begin{prop}
If $k$ is a field then $\dim k^n = n$.
\end{prop}

\begin{proof}
The only subspaces of $k$ are $\{0\}$ and $k$ itself, so the only flag in $k$ is $(k)$.
Therefore $\dim k = 1$.
Now proceed by induction on $n$.
Supposing the result holds for integers up to $n-1$, we note we have an exact sequence
\[
0 \longrightarrow
k \longrightarrow
k^n \longrightarrow
k^{n-1} \longrightarrow
0
\]
where the nontrivial maps are 
\begin{align*}
x &\mapsto (x, 0, \ldots, 0),
\\
(x_1, \ldots, x_n) &\mapsto (x_2, \ldots, x_n).
\end{align*}
Then $\dim k^n = \dim k + \dim k^{n-1} = n$.
\end{proof}





\begin{coro}
If $V$ is a vector space over a field $k$ and $\dim V = n$, then $V \cong k^n$.
\end{coro}

\begin{proof}
Suppose $\dim V = 1$.
If $v \not= 0$ then $(kv)$ is a flag of length one, so $kv = V$.
Then $k \to V$, $x \mapsto xv$ is an isomorphism.

Suppose the result holds for spaces of dimension at most $n-1$ and let $\dim V = n$.
Pick some $v \not= 0$ and consider the line $L = kv \subset V$.
We have a short exact sequence
\[
0 \longrightarrow
L \longrightarrow
V \longrightarrow
V/L \longrightarrow
0
\]
and $L \cong k$ and $V/L \cong k^{n-1}$ by hypothesis.
Since the sequence splits, we conclude that $V \cong k^n$.
\end{proof}




\begin{prop}
Let $V$ be a vector space.
Then $\dim V \leq \dim V^*$, with equality if $V$ is finite-dimensional.
\end{prop}

\begin{proof}
Let $V_1 \subset \cdots \subset V_n \subset V$ be a flag in $V$.
Then $V^* \supset V_n^* \supset \cdots \supset V_1^*$ is a flag in $V^*$, so $\dim V \leq \dim V^*$.

If $\dim V = 1$ then $V \cong k$.
If $\lambda : k \to k$ is a linear map then $\lambda(x) = x \lambda (1)$, so $\lambda$ is determined by its value at $1$.
Then $k^* = k$, so $\dim V^* = 1$.
Suppose the result holds for spaces of dimension up to $n$ and let $\dim V = n$.
Pick a line $L \subset V$ and consider $0 \to L \to V \to V/L \to 0$.
Taking duals we get $0 \to (V/L)^* \to V^* \to L^* \to 0$ and therefore $\dim V^* = (n-1) + 1 = n = \dim V$ by induction.
\end{proof}


\begin{coro}
If $V$ is finite-dimensional then $V = V^{**}$.
\end{coro}

\begin{proof}
The evaluation map $\operatorname{ev}: V \to V^{**}$ defined by $\operatorname{ev}(x) = (\lambda \mapsto \lambda(x))$ is injective for any vector space $V$.
If $V$ is finite-dimensional then $\dim V = \dim V^{**}$ so the map is surjective as well.
\end{proof}



\begin{prop}
Let $V$ and $W$ be vector spaces and $S \subset V$.
Any linear morphism $f : S \to W$ can be extended to a linear morphism $\hat f : V \to W$.
\end{prop}

\begin{proof}
Find a splitting $g : V \to S \oplus V/S$.
Now pick any linear morphism $h : S/V \to W$ and define $\hat f(v) = (f \oplus h)(g(v))$.
\end{proof}


\begin{prop}
Let $V$ and $W$ be vector spaces.
Then
\[
\dim V \cdot \dim W \leq \dim \Hom(V, W).
\]
\end{prop}


\begin{proof}
Let $(V_i)_{i=1}^n$ and $(W_j)_{j=1}^m$ be flags in $V$ and $W$, respectively.
For any subspaces $S \subset V$ and $Q \subset W$ the set $H(S,Q) := \{ f : V \to W \mid f(S) \subset W \}$ is a subspace of $\Hom(V,W)$.
Note that if $Q \subset Q'$ then $H(S,Q) \subset H(S,Q')$.
We define a first flag in $\Hom(V,W)$ by setting $H_j = H(V_n, W_j)$ for $j=1,\ldots,m$.
We have $H_j \subsetneq H_{j+1}$ for all $j$:
Consider the exact sequence
\[
\begin{tikzcd}
0 \ar[r] &
W_j \ar[r] &
W_{j+1} \ar[r] &
W_{j+1} / W_j \ar[r] &
0
\end{tikzcd}
\]
where $W_{j+1} / W_j \not= 0$ by hypothesis.
By picking a splitting of this sequence we can find a morphism $f : V_n \to W_{j+1}$ such that $\Img f$ is not contained in $W_j$, so the inclusion is strict.

We now claim that there is a subflag $H_j = F_1 \subset F_2 \subset \cdots \subset F_n = H_{j+1}$ for every $j$; this will imply the result.
Set then $F_1 = H_j$.
\end{proof}



\section{Tensors}


\begin{defi}
Let $U$, $V$ and $W$ be vector spaces.
A \emph{bilinear map} $f : U \times V \to W$ is a map that is linear in each variable separately.
\end{defi}


\begin{prop}
Let $U$ and $V$ be vector spaces.
There exists a vector space $U \otimes V$ and a bilinear morphism $\phi : U \times V \to U \otimes V$ that satisfies the following universal property:

If $W$ is a vector space and $f : U \times V \to W$ is a bilinear map, then there exists a unique linear morphism $\hat f : U \otimes V \to W$ such that the diagram
\[
\begin{tikzcd}
U \times V \ar[r,"\phi"] \ar[dr,"f"] & U \otimes V \ar[d,"\hat f"]
\\
& W
\end{tikzcd}
\]
commutes.

The space $U \otimes V$ and morphism $\phi : U \times V \to U \otimes V$ are unique up to isomorphism.
\end{prop}

\begin{proof}
We start by considering $F(U \times V)$; the free vector space on $U \times V$.
We then consider the subsets
\begin{align*}
I &= \{ x (u, v) - (xu, v) \mid x \in k, u \in U, v \in V \}
\\
J &= \{ x (u, v) - (u, xv) \mid x \in k, u \in U, v \in V \}
\\
K &= \{ (u+u', v) - (u, v) - (u',v) \mid u, u' \in U, v \in V \}
\\
L &= \{ (u, v+v') - (u, v) - (u,v') \mid u \in U, v, v' \in V \}
\end{align*}
and let $S = \Span(I \cup J \cup K \cup L)$ be the subspace generated by all these elements.
We set $U \otimes V = F(U \times V) / S$ and write $u \otimes v$ for the image of $(u, v)$ in the quotient.
There is a map $U \times V \to F(U \times V)$ that sends each element $(u, v)$ to itself, and we define $\phi : U \times V \to U \otimes V$ by following this map to the quotient.
The resulting morphism is bilinear.

Now let $f: U \times V \to W$ be a bilinear morphism.
It extends to a linear map $\tilde f : F(U \times V) \to W$ by $\tilde f((u, v) = f(u, v)$.
We see that $S \subset \Ker \tilde f$ so it descends to a linear morphism $\hat f : U \otimes V \to W$, and it satisfies $f = \hat f \phi$ by construction.
If $g : U \otimes V \to W$ is another morphism such that $f = g \phi$ then $(\hat f - g) \phi = 0$.
But $\phi$ is surjective, so then $\hat f = g$.

Suppose now that $\psi : U \times V \to Z$ is another space and morphism that satisfy the universal property.
We then get two commutative diagrams
\[
\begin{tikzcd}
U \times V \ar[dr,"\psi"] \ar[r,"\phi"] & U \otimes V \ar[d,"\hat\psi"]
\\
                         & Z
\end{tikzcd}
\quad\text{and}\quad
\begin{tikzcd}
U \times V \ar[dr,"\phi"] \ar[r,"\psi"] & Z \ar[d,"\hat\phi"]
\\
                         & U \otimes V
\end{tikzcd}
\]
such that $\psi = \hat\psi \phi$ and $\phi = \hat\phi \psi$.
Then $\hat\phi \hat\psi : U \otimes V \to U \otimes V$ is a linear morphism such that $\hat\phi \hat\psi \phi = \hat\phi \psi = \phi$; but $\id$ is also such a morphism so $\hat\phi \hat\psi = \id$ by the universal property.
Similarly we see that $\hat\psi \hat\phi = \id$, so $Z \cong U \otimes V$ and $\phi \cong \psi$.
\end{proof}





\section{Structure of endomorphism ring}




\end{document}
