\documentclass[11pt]{article}
\usepackage[utf8]{inputenc}
\usepackage[T1]{fontenc}
\usepackage{graphicx}
\usepackage{grffile}
\usepackage{longtable}
\usepackage{wrapfig}
\usepackage{rotating}
\usepackage[normalem]{ulem}
\usepackage{amsmath}
\usepackage{textcomp}
\usepackage{amssymb}
\usepackage{capt-of}
\usepackage{hyperref}
\author{Gunnar Þór Magnússon}
\title{Sérhæfð þekking}
\date{}
\begin{document}

\maketitle
Gunnar býr yfir umfangsmikilli þekkingu á hönnun, útfærslu, viðhaldi og bestun bakenda og gagnagrunna. Við hjá Sendiráðinu stefnum á að stækka við umfang okkar í verkefnum sem þarfnast slíkrar þekkingar. Því hentar reynsla Gunnars markmiðum okkar vel.

Á Íslandi eru margir hæfir forritarar sem hafa svipaða reynslu. Það sem gerir reynslu og þekkingu Gunnars sérstaka er sú stærðargráða sem hann hefur unnið á. Síðustu fimm ár hefur hann unnið við dreifð kerfi hjá Booking.com sem taka við umferð frá öllum heiminum, og það eru ekki margir starfsstaðir sem takast á við þannig umferð á Íslandi eða margir forritarar hér sem hafa reynslu af slíku. Á meðan að við búumst ekki við að þjóna slíkri umferð, þá er ljóst að það að vinna við hana temur vana sem nýtast vel við alla vinnu.

Hvað ráðningarferlið varðar hófst það með ánægjulegri tilviljun. Við hjá Sendiráðinu höfum verið að leggja áætlanir um að byrja að ráðast í bakendafrekari verkefni, og Gunnar hafði samband við okkur varðandi atvinnu því hann þekkir einn starfsmann okkar. Framkvæmda- og tæknistjórar okkar héldu svo hefðbundin atvinnuviðtöl við Gunnar og ákváðu í framhaldi af því að bjóða honum vinnu.

Við erum til taks ef einhverra fleiri upplýsinga er þörf.


\bigskip
Orri Guðjónsson

Sendiráðið
\end{document}
