\documentclass[11pt]{article}

\usepackage{tgpagella}
\linespread{1.1}
\usepackage[utf8]{inputenc}
\usepackage[T1]{fontenc}

\usepackage[normalem]{ulem}
\usepackage{textcomp}
\usepackage{hyperref}
\usepackage{tikz-cd}

\usepackage{amsmath}
\usepackage{amssymb}
\usepackage{amsthm}

\newtheorem{theo}{Theorem}[section]
\newtheorem{prop}[theo]{Proposition}
\newtheorem{lemm}[theo]{Lemma}
\newtheorem{coro}[theo]{Corollary}
\theoremstyle{definition}
\newtheorem{defi}[theo]{Definition}
\newtheorem{exam}[theo]{Example}

\newcommand{\kk}[1]{\mathbb{#1}}
\newcommand{\cc}[1]{\mathcal{#1}}

\def\eps{\varepsilon}
\def\empty{\varnothing}

\def\oint#1#2{(#1, #2)}
\def\cint#1#2{[#1, #2]}
\def\coint#1#2{[#1, #2)}
\def\ocint#1#2{(#1, #2]}

\def\ov#1{\overline{#1}}

\def\grp{\mathsf{Grp}}
\def\set{\mathsf{Set}}
\def\CC{\mathbf{C}}
\def\EE{\mathcal{E}}
\def\FF{\mathcal{F}}
\def\NN{\mathbf{N}}
\def\RR{\mathbf{R}}
\def\QQ{\mathbf{Q}}
\def\ZZ{\mathbf{Z}}
\def\PP{\mathbf{P}}

\DeclareMathOperator{\aut}{Aut}
\DeclareMathOperator{\ke}{Ker}
\DeclareMathOperator{\im}{Im}
\DeclareMathOperator{\sgn}{sgn}
\DeclareMathOperator{\GL}{GL}
\DeclareMathOperator{\ord}{ord}

\author{Gunnar Þór Magnússon}
\date{\today}
\title{Topics on Riemann surfaces}

\begin{document}

\maketitle


A \emph{Riemann surface} is a complex manifold of dimension one.
Suppose we all remember the basics, like what sheaves and ringed spaces are, and
the results of basic complex analysis.
Can we talk about some basic topics on Riemann surfaces?

\begin{itemize}
\item
Uniformization of compact Riemann surfaces.

\item
A noncompact Riemann surface is Stein.

\item
Riemann--Roch theorem, obviously.

\item
Embedding into Abelian varieties.

\item
Hurwitz theorem on maps $C \to C'$.

\item
A compact Riemann surface is algebraic.

\item
A compact Riemann surface can be embedded into $\PP^3$.

\item
Existence of constant curvature metrics.

\item
Something with \sout{plasma} randomness?

\item
Weil--Petersson metric.
Just do it locally, don't have to construct the whole moduli space.
Then do Hurwitz spaces.
\end{itemize}


\begin{theo}
Let $S$ be a compact Riemann surface.
Then one of the following holds:

(a) $S$ is isomorphic to the projective line.

(b) $S$ is an elliptic curve and its universal cover is the complex plane $\CC$.

(c) The universal cover of $S$ is the unit disk.
\end{theo}


\begin{theo}
Let $S$ be a noncompact Riemann surface.
Then $S$ is Stein.
\end{theo}


\section{Basics}


\begin{defi}
A \emph{Riemann surface} is a topological space $X$ that satisfies the
following conditions:
\begin{itemize}
\item
The space is Hausdorff.

\item
There is a countable base for the topology on the space.

\item
Every point is contained in
a neighborhood $U$ that is homeomorphic to an open set in $\CC$.

\item
If $\phi : U \to \phi(U)$ and $\psi : V \to \psi(V)$ are two such
homeomorphisms and $U \cap V \not= \empty$, then the function
$\phi\circ\psi^{-1}: \psi(U \cap V) \to \phi(U \cap V)$ is holomorphic.
\end{itemize}
\end{defi}

The first three conditions mean that $X$ is a topological manifold, and the
fourth that the change of coordinate functions are holomorphic.
We like our spaces to be Hausdorff, because $\RR^n$ is Hausdorff and we want
something that looks like that when we zoom in.
The condition on countable base excludes some things like the long line from
being manifolds.

The definition doesn't mention it, but we will always take our Riemann surfaces
to be connected.
The statements of some theorems require this, and we won't mention it when they do.


\begin{exam}
Every nonempty open set $U \subset \CC$ is a Riemann surface.
\end{exam}


\begin{exam}[Projective line]
Let $\PP^1$ be the set of complex lines through the origin in $\CC^2$.
We call $\PP^1$ the projective line.
As a set
$$
\PP^1 
= (\CC^2 \setminus \{0\}) / \CC
$$
and we denote the quotient map by $\pi : \CC^2 \setminus \{0\} \to \PP^1$.
Let $U_0 = \{(z,w) \in \CC^2 \mid z\not=0\}$ and
$U_1 = \{(z,w) \in \CC^2 \mid w\not=0\}$.
These are open sets in $\CC^2\setminus\{0\}$, so they are open in the quotient
space.
They also cover $\PP^1$.

We have a continuous map $\phi_0 : U_0 \to \CC$ given by $(z,w) \mapsto w/z$
that is constant on the fibers of the quotient map $\pi : \CC^2\setminus\{0\}
\to \PP^1$, so it induces a continuous map $\psi_0 : \pi(U_0) \to \CC$.

We also have a continuous map $\theta_0 : \CC \to \CC^2\setminus\{0\}$ given by
$w \mapsto (1, w)$. 
We have 
\begin{align*}
\psi_0 \circ (\pi \circ \theta_0)(w) 
&= \phi_0(1, w)
= w,
\\
(\pi \circ \theta_0) \circ \psi_0([z,w]) 
&= (\pi \circ \theta_0)(w/z) = \pi(1, w/z) = [z,w]
\end{align*}
so $\pi \circ \theta_0$ is a continuous inverse of $\psi_0$, which makes it a
homeomorphism.

On $\psi_1(\pi(U_0 \cap U_1))$ we have
$$
\psi_0 \circ (\pi \circ \theta_1)(z)
= \phi_0 (z, 1)
= 1/z
$$
which is holomorphic because $z \not= 0$ on the intersection.
Therefore $\PP^1$ is a Riemann surface.
It is also compact, as it is the image of the compact space $S^3 \subset \CC^2
\setminus \{0\}$.

We very often write $[z:w]$ for points in $\PP^1$, and call that way of writing
points homogeneous coordinates on $\PP^1$. This should be understood to be the
image of $(z,w)$ under $\pi$.
\end{exam}



\begin{defi}
Let $X$ and $Y$ be Riemann surfaces.
A map $f : X \to Y$ is \emph{holomorphic} if the composition
$\psi \circ f \circ \phi^{-1}$ is holomorphic for all coordinate functions
$\phi : U \subset X \to \phi(U)$ and $\psi : V \subset Y \to \psi(V)$.
\end{defi}

\begin{prop}[Open map theorem]
Let $f : X \to Y$ be a holomorphic map between Riemann surfaces.
If $U \subset X$ is open then $f(U) \subset Y$ is open.
\end{prop}

\begin{proof}
The question of whether a set is open or not is local, and
we reduce the the case of a holomorphic function in $\CC$ by picking coordinate
neighborhoods around a point $x$ and $f(x)$.
\end{proof}




\begin{prop}[Identity theorem]
Let $f,g : X \to Y$ be holomorphic maps between Riemann surfaces.
If the set $\{ x \in X \mid f(x) = g(x) \}$ has an accumulation point
then $f = g$.
\end{prop}

\begin{proof}
By picking charts around an accumulation point and its image we can see that
the functions agree on the whole chart by the identity theorem in the complex plane.
Then they also agree on any chart that intersects our original one.
Expanding outwards we cover the whole of $X$ and see the functions agree there.
\end{proof}

This is clearly a place where we used the connectedness of $X$ without saying so.



\paragraph{Order of a function at a point}


Let $U \subset \CC$ be an open set.
If $f : U \to \CC$ is meromorphic and has at worst poles for singularities then
for any point $z_0 \in U$ there exists an integer $k$ and a holomorphic
function $g$ on a neighborhood around $z_0$ such that $g(z_0) \not= 0$ and
$$
f(z) = (z - z_0)^k g(z).
$$
This is because we can develop $f$ as a Laurent series
$$
f(z) = \sum_{j\in\ZZ} a_j (z - z_0)^j.
$$
We set $k = \inf\{ j \in \ZZ \mid a_j \not= 0 \}$ and then get
$$
f(z) = (z - z_0)^k \cdot \sum_{j \geq 0} a_{k+j} (z - z_0)^j.
$$
We define the \emph{order of $f$ at $z_0$} to be $k$ and write it as
$\ord_{z_0}(f) = k$.

This is well defined:
Suppose we can also write $f(z) = (z - z_0)^l h(z)$ for $l \in \ZZ$ and $h$
holomorphic and nonvanishing at $z_0$. Then
$$
\frac{g(z)}{h(z)} = \frac{(z - z_0)^k}{(z - z_0)^l}
$$
is holomorphic and nonvanishing at $z_0$, which is only possible if $l = k$.
We also conclude that $h = g$ around $z_0$ in this case, so the decomposition
above is unique.

If $f(z) = (z - z_0)^k f_1(z)$ and $g(z) = (z - z_0)^l g_1(z)$ then
$$
fg(z) = (z - z_0)^{k + l} f_1g_1(z)
$$
so $\ord_{z_0}(fg) = \ord_{z_0}(f) + \ord_{z_0}(g)$.

If $f$ has a zero at $z_0$ then $\ord_{z_0}(f) > 0$,
if it has a pole there we have $\ord_{z_0}(f) < 0$, and otherwise we have
$\ord_{z_0}(f) = 0$.


Let $X$ be a Riemann surface and $\cc M$ the sheaf of meromorphic functions on $X$.
It is a sheaf of fields, by definition, as $\cc M$ is the field of fractions of
$\cc O$. For any point $x_0$ on $X$ there is a morphism of groups
$$
\ord_{x_0} : \cc M_{x_0} \to \ZZ.
$$
I guess $\ord$ is a sheaf morphism from $\cc M$ to the sheaf that assigns
$\bigoplus_{x \in U} \ZZ x$ to any open $U \subset X$.



\section{Divisors}


\begin{defi}
A \emph{divisor} on a Riemann surface $X$ is an element of the free abelian group
on the points of $X$.
Explicitly, a divisor $D$ on $X$ is a finite sum
$$
D = \sum_{k=1}^n a_k x_k,
$$
where $a_k \in \ZZ$ and $x_k \in X$.
\end{defi}


A meromorphic function $f : X \to \PP^1$ defines a divisor by
$$
(f) = \sum_{x \in X} \ord_x(f) \, x.
$$
The sum is finite because $\ord_x(f) \not= 0$ only if $f$ has a zero or a pole
at $x$.
We have $(fg) = (f) + (g)$ and $(1) = 0$ so these divisors form a subgroup of
all divisors. They're called principal divisors.

If all the coefficients of a divisor $D$ are nonnegative we write $D \geq 0$,
and if they are all nonpositive we write $D \leq 0$.


\begin{defi}
Let $D$ be a divisor on $X$.
The sheaf of functions with poles along $D$ is
$$
\cc O(D)(U)
:= \{ f : U \to \PP^1 \mid (f) + D \geq 0 \}.
$$
\end{defi}


What does this mean?
By picking $U$ small enough we may assume that $D = n x$ for some $x \in X$.
A section $f$ of $\cc O(nx)(U)$ must then be nonzero on $U \setminus \{x\}$,
but it may have poles there.
We also must have
$$
\ord_x(f) + n \geq 0
$$
so if $n > 0$ the section may have a pole of order up to $n$ at $x$;
and if $n < 0$ the section must have a zero of order at least $n$ at $x$. 


Want to see that $\cc O(D + D') = \cc O(D) \otimes \cc O(D')$
and $\cc O(0) = \cc O$.
Then it's clear that $\cc O(D)$ is an invertible sheaf for any divisor,
and thus the sheaf of sections of some line bundle.



\begin{theo}[Riemann existence theorem]
Let $X$ be a compact Riemann surface.
For any point on $X$ there exists a meromorphic function $f : X \to \PP^1$
that only has a pole at the given point.
\end{theo}


\begin{proof}
Fix a point $x$ and take $n > 0$.
We have an inclusion $\cc O \to \cc O(nx)$  and thus a short exact sequence
$$
0 \longrightarrow \cc O 
\longrightarrow \cc O(nx)
\longrightarrow \cc O(nx) / \cc O
\longrightarrow 0.
$$
We claim that $\cc O(nx) / \cc O$ is the skyscraper sheaf $\CC^n_x$.
If $U$ does not contain $x$ then the quotient is zero on $U$, so the only
nonzero stalk of the quotient is at $x$.
There, by picking coordinates centered on $x$ and considering Laurent series,
we see that the quotient is the set of sums of the form
$$
\sum_{k = -n}^{-1} a_k z^k
$$
which is just $\CC^n$.
Taking cohomology we now get an exact sequence
$$
H^0(X, \cc O(nx)) \longrightarrow
\CC^n \longrightarrow 
H^1(X, \cc O).
$$
For $n > \dim H^1(X, \cc O)$ the second arrow will have a nontrivial kernel,
so we must have $H^0(X, \cc O(nx)) \not= 0$.
\end{proof}





\section{Classification of compact Riemann surfaces}


\begin{theo}
Let $X$ be a compact Riemann surface of genus $0$.
Then $X$ is isomorphic to $\PP^1$ and admits a metric of positive constant
curvature.
\end{theo}


\begin{theo}
Let $X$ be a compact Riemann surface of genus $1$.
Then $X$ is isomorphic to a complex torus and admits a flat metric.
\end{theo}


\begin{theo}
Let $X$ be a compact Riemann surface of genus $g > 1$.
Then the universal cover of $X$ is the unit disk and $X$ admits a metric
of negative constant curvature.
\end{theo}


Note that $H_1(X,\ZZ)$ is a lattice in $H^1(X, K_X)^*$:
First off it embeds in the dual space via the linear map
$$
[\gamma] \mapsto \biggl(s \mapsto \int_\gamma s\biggr),
$$
where $\gamma$ is any representative of $[\gamma]$.
Further it is an abelian group and has rank $2g$.
This gives the Jacobian $J(X) := H^0(X, K_X)^* / H_1(X, \ZZ)$.

We claim there is a holomorphic map $f : X \to J(X)$.




\end{document}
