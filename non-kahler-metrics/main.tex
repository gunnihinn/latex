\documentclass[12pt]{amsart}

\usepackage{tgpagella}
\linespread{1.1}
\usepackage[utf8]{inputenc}
\usepackage[T1]{fontenc}

\usepackage[normalem]{ulem}
\usepackage{textcomp}
\usepackage{hyperref}

\usepackage{amsmath}
\usepackage{amssymb}
\usepackage{amsthm}

\usepackage{tikz-cd}
\usepackage{color}

\newtheorem{theo}{Theorem}
\newtheorem{prop}[theo]{Proposition}
\newtheorem{lemm}[theo]{Lemma}
\newtheorem{coro}[theo]{Corollary}
\theoremstyle{definition}
\newtheorem*{e}{Exercise}
\newtheorem{defi}[theo]{Definition}
\newtheorem{ques}[theo]{Question}
\newtheorem{exam}[theo]{Example}
\newtheorem{cexam}[theo]{Counterexample}

\newcommand{\kk}[1]{\mathbb{#1}}
\newcommand{\cc}[1]{\mathcal{#1}}

\def\qandq{\quad\text{and}\quad}
\def\ov#1{\overline{#1}}
\def\empty{\varnothing}

\newcommand{\cat}[1]{\mathsf{#1}}
\def\psh{\cat{PSh}}
\def\sh{\cat{Sh}}

\def\NN{\mathbf{N}}
\def\ZZ{\mathbf{Z}}
\def\QQ{\mathbf{Q}}
\def\RR{\mathbf{R}}
\def\CC{\mathbf{C}}
\def\PP{\mathbf{P}}

\DeclareMathOperator{\et}{Et}
\DeclareMathOperator{\re}{Re}
\DeclareMathOperator{\Hess}{Hess}
\DeclareMathOperator{\pr}{pr}
\DeclareMathOperator{\Span}{span}
\DeclareMathOperator{\Gr}{Gr}
\DeclareMathOperator{\GL}{GL}
\DeclareMathOperator{\im}{Im}
\DeclareMathOperator{\Vol}{Vol}
\DeclareMathOperator{\Ker}{Ker}
\DeclareMathOperator{\Img}{Im}
\DeclareMathOperator{\End}{End}
\DeclareMathOperator{\Aut}{Aut}
\DeclareMathOperator{\Hom}{Hom}
\DeclareMathOperator{\Sym}{Sym}
\DeclareMathOperator{\id}{id}
\DeclareMathOperator{\tr}{tr}
\DeclareMathOperator{\adj}{adj}
\DeclareMathOperator{\Spec}{Spec}

% Projective spaces
\newcommand{\PC}[1]{\PP(\CC^{#1})}
\newcommand{\PV}[1]{\PP(#1)}

\def\<{\langle}
\def\>{\rangle}
\def\d{\mathrm{d}}

\newcommand{\ff}[1]{\mathfrak{#1}}


\author{Gunnar \TH\'or Magn\'usson}
\date{\today}
\title{Constructing some non-K\"ahler metrics}

\begin{document}

\maketitle




\section{Metrics induced by covering bundles}


\begin{prop}
Let $X$ be a K\"ahler manifold and let
\[
\begin{tikzcd}
0 \ar[r] &
S \ar[r,"j"] &
E \ar[r,"q"] &
T_X \ar[r] &
0
\end{tikzcd}
\]
a short exact sequence of vector bundles.
There exists a Hermitian metric on $E$ such that the induced metric on $X$ is K\"ahler if and only if $Dq = \theta \otimes f$, where $\theta$ is a smooth $(1,0)$-form and $f$ is a smooth section of $\Hom(E,T_X)$.
\end{prop}



\begin{proof}
Being a K\"ahler metric is a local property, so we fix a point $x \in X$ and consider a coordinate neighborhood of it.
By shrinking the neighborhood we can find a local frame $(s_1, \ldots, s_r)$ of $E$ that is normal at $x$ and such that the $q(s_j)$ span $T_X$.
Write
\[
q(s_j) = \sum_{k=1}^n q_{kj} \frac{\partial}{\partial z_k}
\]
for holomorphic functions $q_{kj}$.
For sections $s,t$ of $E$ we have
\[
\tau(q(s), q(t))
= q(D_{E,q(s)} t) - q(D_{E,q(t)} s) - [q(s), q(t)].
\]
Since we picked our frame to be normal we have $D_{E,q(s_j)} s_k = 0$ at $x$ for all $j,k$.
Therefore
\begin{align*}
\tau(q(s_j), q(s_k))
&= [q(s_j), q(s_k)]
\\
&= \sum_{l,m=1}^n 
\biggl[
q_{lj} \frac{\partial}{\partial z_l},
q_{mk} \frac{\partial}{\partial z_m}
\biggr]
\\
&= \sum_{l,m=1}^n 
q_{lj} \frac{\partial q_{km}}{\partial z_l} \frac{\partial}{\partial z_m}
- q_{mk} \frac{\partial q_{jl}}{\partial z_m} \frac{\partial}{\partial z_l}
+ q_{lj} q_{mk} 
\biggl[
\frac{\partial}{\partial z_l},
\frac{\partial}{\partial z_m}
\biggr]
\\
&= \sum_{l,m=1}^n 
\biggl(
q_{mj} \frac{\partial q_{lk}}{\partial z_m}
- q_{mk} \frac{\partial q_{lj}}{\partial z_m}
\biggr)
\frac{\partial}{\partial z_l}
\end{align*}
at $x$.
Then the metric on $X$ is K\"ahler at $x$ if and only if
\[
\sum_{m=1}^n q_{mj} \frac{\partial q_{lk}}{\partial z_m}
= \sum_{m=1}^n q_{mk} \frac{\partial q_{lj}}{\partial z_m}
\]
for all $j,k = 1,\ldots,r$ and $l = 1,\ldots,n$.
This means that
\[
D_{\Hom(E,T_X),q(s_j)} q = D_{\Hom(E,T_X),q(s_k)} q
\]
for all $j,k = 1,\ldots,r$.
That is, the image of 
\[
T_X \to \Hom(E,T_X),
\quad
\xi \mapsto D_\xi q
\]
is at most a line, which happens exactly when we can then write $Dq = \theta \otimes f$ for a $(1,0)$-form $\theta$ and a morphism $f : E \to T_X$.
\end{proof}


\begin{proof}[Another attempt]
For any section $s$ of $E$ we have 
\[
D_X q(s)
= D_{\Hom}q(s) + q(D_E s)
= - b j^\dagger(s) + q(D_E s).
\]
The metric on $X$ is K\"ahler at at point $x$ if and only if there exists normal coordinates $(z_1,\ldots,z_n)$ centered at $x$.
Given any coordinate system we can find a normal frame $(e_1,\ldots,e_r)$ for $E$ at $x$.
Writing $s = \sum s_l e_l$ we see that
\[
D_E s 
= \sum ds_l \otimes e_l + s_l \otimes D_E e_l
= \sum ds_l \otimes e_l
\]
at the center.
Write $q(e_k) = \sum_l q_{lk} \partial / \partial z_l$.
Then
\[
D_X q(e_k)
= \sum_l \partial q_{lk} \otimes \frac{\partial}{\partial z_l}
\]
and
\[
b j^\dagger(e_k)
= \sum_l \partial q_{lk} \otimes \frac{\partial}{\partial z_l}
\]
at $x$.
Let $s$ be in $S$ and write $j(s) = \sum s_k e_k$.
Then
\[
b(s)
= \sum_{k,l} s_k \partial q_{lk} \otimes \frac{\partial}{\partial z_l}
\]
at $x$, which is holomorphic there.

so
\[
\tau(\xi, \eta)
= - b_{\xi} j^\dagger(t)
+ b_{\eta} j^\dagger(s)
+ q(D_{E,\xi} t)
- q(D_{E,\eta} s)
- [\xi, \eta].
\]

Note that $bj^\dagger$ is a tensor in $s$.
So is $\tau$, so by linearity 
\[
\rho(s,t)
= q(D_{E,q(s)} t)
- q(D_{E,q(t)} s)
- [q(s), q(t)]
\]
is a tensor in $s$ and $t$, as the reader can also check directly for their amusement.
If $s \in \Ker q = S$ we get
\[
\rho(s, t)
= q(D_{E,q(s)} t)
- q(D_{E,q(t)} s)
- [q(s), q(t)]
= - q(D_{E,q(t)} s)
= - b_{q(t)} s,
\]
where $b \in \cc A^{1,0}(\Hom(S,T_X))$ is the second fundamental form.
If $t$ is also in $S$ we get $\rho(s,t) = 0$.
If $q^\dagger$ is the adjoint of $q$ we have
\[
\tau(\xi, \eta) = \rho(q^\dagger \xi, q^\dagger \eta).
\]
Then the metric on $X$ is K\"ahler if and only if $\rho$ decomposes as
\[
\rho = \begin{pmatrix}
0 & b
\\
-b & 0
\end{pmatrix}
\]
according to the smooth splitting $E = S \oplus T_X$.

The terms $D_{\Hom,\xi}q (t) - D_{\Hom,\eta}q (s)$ and $q(D_{E,\xi} t) - q(D_{E,\eta} s) - [\xi, \eta]$ are both tensors in $s$ and $t$, so we get
\[
\tau(\xi, \eta)
= D_{\Hom,\xi}q (t) 
- D_{\Hom,\eta}q (s) 
- [\xi, \eta]
\]
at the center of a normal coordinate frame on $E$.
The metric on $X$ is K\"ahler there if and only if
\[
D_{\Hom,\xi}q (t) 
- D_{\Hom,\eta}q (s) 
= [\xi, \eta]
\]
there.
If $t \in \Ker q = S$ we then get
\[
b_\xi(t) 
= - D_{\Hom, \xi} q(t)
= 0
\]
for any $\xi$, so $b(t) = 0$.
Then the second fundamental form $b = 0$, so the sequence splits.
\end{proof}


\section{Misc}


There is a smooth section $\eta$ of $T_X$ such that
\[
D_\xi q = \omega(\xi, \bar\eta) f
\]
for all $\xi$, where $\omega$ is the K\"ahler metric on $X$.
We can apply $\omega$ and write
\[
\omega \circ D_\xi q \in \cc A^{1,1}(S).
\]

We do have $Dq = - b \circ j^\dagger$, so this does say something about the second fundamental form.


\begin{exam}
It would have been neat if the induced metric being K\"ahler meant that the short exact sequence splits.
This is not true:
Consider an elliptic curve $C$ and extensions
\[
\begin{tikzcd}
0 \ar[r] &
\CC \ar[r] &
E \ar[r] &
T_C \ar[r] &
0
\end{tikzcd}
\]
of the trivial line bundle by $T_C$.
These are classified by 
\[
H^1(C, \Hom(\CC, T_C)) = H^{1,1}(C,\CC)
\]
which has dimension $1$, so there are nonsplit such extensions.
However any metric on $F$ induces a K\"ahler metric on $C$ for dimension reasons, so the fundamental form decomposes.

This is not an accident of complex dimension one:
If $X = \prod_n C$ and $\pi_j : X \to C$ are the projections we get an extension
\[
\begin{tikzcd}
0 \ar[r] &
\CC^n \ar[r] &
\prod_{j=1}^n \pi_j^*E \ar[r] &
T_X \ar[r] &
0
\end{tikzcd}
\]
and a metric on the middle term that induces a K\"ahler metric on $X$.
\end{exam}


Let $X$ be a K\"ahler manifold and suppose there is an exact sequence
\[
\begin{tikzcd}
0 \ar[r] &
S \ar[r,"j"] &
E \ar[r,"q"] &
T_X \ar[r] &
0
\end{tikzcd}
\]
where $(E,h)$ is flat.
Then the second fundamental form $b = Dq \circ j \in \cc A^{1,0}(\Hom(S,T_X))$ is holomorphic, and thus parallel.
Suppose also the induced metric on $X$ is K\"ahler.
Then $b = Dq \circ j = \theta \otimes fj$.




\begin{exam}
Let $X$ be a compact K\"ahler manifold and suppose there is an exact sequence
\[
\begin{tikzcd}
0 \ar[r] &
S \ar[r,"j"] &
V \otimes L \ar[r,"q"] &
T_X \ar[r] &
0,
\end{tikzcd}
\]
where $L \to X$ is a line bundle and $V$ is a fixed finite-dimensional vector space.
Fix some inner product on $V$ and a Hermitian metric on $L$.
Consider a local section $s$ of $V \otimes L$.
Since $V$ is flat we can find a parallel frame $(v_j)$ of it, and write $s = \sum v_j \otimes s_j$, where $s_j$ are local sections of $L$.
Then
\[
Ds = \sum v_j \otimes D_L s_j.
\]
\end{exam}


Maybe stop the local work here.
If the metric is K\"ahler the second fundamental form is $b = \theta \otimes f$.
Then what?



Its adjoint is $b^\dagger = \ov\theta \otimes f^\dagger$.
We get
\[
b \wedge b^\dagger
= \theta \wedge \ov\theta \otimes f f^\dagger
\]
and
\[
\tr b \wedge b^\dagger \wedge \omega^{n-1}
= |\theta|^2 |f|^2 \omega^n.
\]


\section{Some non-K\"ahler metrics on projective varieties}

Let $X$ be a projective manifold of dimension $\dim_{\CC} X = n$.
Let $L \to X$ be an ample line bundle and $h$ a positively curved metric on $L$.
Since $L$ is ample there is a $k_0$ such that $T_X \otimes kL$ is globally generated for $k \geq k_0$.
We may also suppose $kL$ is very ample for $k \geq k_0$.
Then there is a surjective map
\[
H^0(X, T_X \otimes kL) 
\longrightarrow T_X \otimes kL
\]
and thus a short exact sequence
\[
\begin{tikzcd}
0 \ar[r] &
S \ar[r] &
H^0(X, T_X \otimes kL) \otimes -kL \ar[r,"q"] &
T_X \ar[r] &
0.
\end{tikzcd}
\]
If we fix some Hermitian metric $\omega$ on $X$ we get an $L^2$-inner product on the global sections of $T_X \otimes kL$, so we have a Hermitian metric on the middle bundle.
It induces another Hermitian metric $\omega_{h,k}$ on $X$.



\begin{prop}
The metric $\omega_{h,k}$ is not K\"ahler.
\end{prop}

\begin{proof}
The bundle $T_X \otimes kL$ is globally generated.
Let $(e_1, \ldots, e_r)$ be a basis of $H^0(X,T_X \otimes kL)$.
The $L^2$-metric defines a flat Hermitian metric on the associated trivial bundle and the sections $e_j$ are parallel with respect to the Chern connection of the metric.

Let $U$ be a coordinate neighborhood that trivializes $kL$ by a section $s$, and $\tau$ be a section of $-kL$ such that $\tau(s) = 1$ on $U$.
Any local section of $H^0(X, T_X \otimes kL) \otimes -kL$ can we written as $v = \sum f_j e_j \otimes \tau$, where $f_j$ are holomorphic.
For these we have
\[
D_E\Bigl( \sum f_j e_j \otimes \tau \Bigr)
= \sum d f_j \otimes e_j \otimes \tau + f_j e_j \otimes D_{-kL} \tau.
\]
Generally if $f : E \to Q$ is a map we have
\[
D_Q(f(e))
= D_{\Hom}f(e) + f(D_E e).
\]
If $g$ is a smooth function we then get
\begin{align*}
D_{\Hom}f(ge)
&= D_Q(f(ge)) - f(D_E(ge))
\\
&= dg \otimes f(e) + g D_Q(f(e)) - gd \otimes f(e) - g f(D_E(e))
\\
&= g \bigl( D_Q(f(e)) - f(D_E(e)) \bigr)
= g D_{\Hom} f(e).
\end{align*}
Therefore
\[
D_{\Hom}q \biggl( \sum_j f_j e_j \otimes \tau \biggr)
= \sum_j f_j \bigl( D_X q(e_j \otimes \tau)
- q(e_j \otimes D_{-kL}\tau) \bigr).
\]
Note that we can write $D_{-kL}\tau = \sigma \otimes \tau$ for a $(1,0)$-form $\sigma$.
So in fact
\[
D_{\Hom}q \Bigl( \sum_j f_j e_j \otimes \tau \Bigr)
= \sum_j f_j D_X q(e_j \otimes \tau)
- \sigma \otimes q\Bigl(\sum_j f_j e_j \otimes \tau\Bigr).
\]
Taking our section to be in $\Ker q$ we then get
\[
b\Bigl( \sum_j f_j e_j \otimes \tau \Bigr)
= \sum_j f_j D_X q(e_j \otimes \tau).
\]



If $e_j \in H^0(X,T_X \otimes kL)$ we can write
\[
e_{j|U} = \sum_k g_{kj} \frac{\partial}{\partial z_k} \otimes s
\]
for holomorphic functions $g_{kj}$.
Once restricted to $U$ we then have
\[
q\Bigl(\sum_{j} f_j e_{j} \otimes \tau \Bigr) 
= \sum_{j,k} f_j g_{kj} \frac{\partial}{\partial z_k}.
\]
For the Chern connection of the metric on $X$ we then get
\[
D_X q\Bigl(\sum_{j} f_j e_{j} \otimes \tau \Bigr) 
= \sum_{j,k} g_{kj} df_j \frac{\partial}{\partial z_k}
+ \sum_{j,k} f_j dg_{kj} \frac{\partial}{\partial z_k}
+ \sum_{j,k} f_j g_{kj} D\frac{\partial}{\partial z_k}.
\]
Meanwhile,
\[
q D_E\Bigl( \sum f_j e_j \otimes \tau \Bigr)
= \sum_{j,k} g_{jk} d f_j \frac{\partial}{\partial z_k}
+ f_j g_{kj} (D \tau)(s) \frac{\partial}{\partial z_k} .
\]


Note that $\eta \mapsto \tau(D_\tau s)$ is a linear functional.
If $\eta$ is in its kernel we get
\[
D_{\Hom,\eta}q ( e_j \otimes \tau )
= \sum_{k} D_{X,\eta}\Bigl(g_{kj} \frac{\partial}{\partial z_k}\Bigr).
\]
If now $\nu$ is not in its kernel we have
\[
D_{\Hom,\nu}q ( e_j \otimes \tau )
= \sum_{k} D_{X,\nu}\Bigl(g_{kj} \frac{\partial}{\partial z_k}\Bigr)
+ g_{jk} \tau(D_\nu s) \frac{\partial}{\partial z_k}.
\]
For any $\xi$ we can then write
\[
\xi = \frac{\tau(D_\xi s)}{\tau(D_\nu s)} \nu
+ \biggl(\xi - \frac{\tau(D_\xi s)}{\tau(D_\nu s)} \nu \biggr)
\in \cc O \nu \oplus \Ker \tau(Ds).
\]
Then
\[
D_{\Hom,\xi} q(e_j \otimes \tau)
= \frac{\tau(D_\xi s)}{\tau(D_\nu s)} D_{\Hom,\nu} q(e_j \otimes \tau)
+
\]
If our metric is K\"ahler then these two are the same up to some multiple, which we may take to be $1$ after scaling $\eta$.
If so we have
\[
\sum_{k} D_{X,\nu-\eta}\Bigl(g_{kj} \frac{\partial}{\partial z_k}\Bigr)
+ g_{jk} \tau(D_\nu s) \frac{\partial}{\partial z_k}
= 0
\]
and so we also have
\[
D_{X,\eta} q(e_j \otimes \tau)
= \sum_{k} D_{X,\eta}\Bigl(g_{kj} \frac{\partial}{\partial z_k}\Bigr)
= 0
\]
for any $\eta$ in the kernel of $\tau(D s)$.

If $\xi \in T_X$ then $\xi = q(\sum_j f_j e_j \otimes \tau)$ for some holomorphic $f_j$.
Then
\[
D \xi 
= D \sum_j f_j q(e_j \otimes \tau)
= \sum_j df_j \otimes q(e_j \otimes \tau)
+ f_j D q(e_j \otimes \tau),
\]
so taking $\eta = q(\sum_j h_j e_j \otimes \tau)$ in the kernel we get
\[
0 
= \tau(\xi,\eta) 
= D_\eta \xi - D_\xi \eta - [\xi, \eta]
= \sum_{j=1}^r h_j D_\xi q(e_j \otimes \tau).
\]
I feel like once $r = \dim H^0(X, T_X \otimes kL) > 2n$ this gives too many equations for the $D_\xi q(e_j \otimes \tau)$ to satisfy, so it should force $D_\xi q(e_j \otimes t) = 0$ for all $j$.
Then our metric would have to be flat, which doesn't make sense unless we're on a torus.

If $\lambda$ is a linear form and $\lambda(v) \not= 0$ then for any other $w$ we can write
\[
w = \frac{\lambda(w)}{\lambda(v)} v
+ \biggl( w - \frac{\lambda(w)}{\lambda(v)} v \biggr)
\in \CC v \oplus \Ker \lambda.
\]
\end{proof}




\end{document}
