\documentclass[12pt]{amsart}

\usepackage{tgpagella}
\linespread{1.1}
\usepackage[utf8]{inputenc}
\usepackage[T1]{fontenc}

\usepackage[normalem]{ulem}
\usepackage{textcomp}
\usepackage{hyperref}

\usepackage{amsmath}
\usepackage{amssymb}
\usepackage{amsthm}

\usepackage{tikz-cd}
\usepackage{color}

\newtheorem{theo}{Theorem}
\newtheorem{prop}[theo]{Proposition}
\newtheorem{lemm}[theo]{Lemma}
\newtheorem{coro}[theo]{Corollary}
\theoremstyle{definition}
\newtheorem*{e}{Exercise}
\newtheorem{defi}[theo]{Definition}
\newtheorem{ques}[theo]{Question}
\newtheorem{exam}[theo]{Example}
\newtheorem{cexam}[theo]{Counterexample}

\newcommand{\kk}[1]{\mathbb{#1}}
\newcommand{\cc}[1]{\mathcal{#1}}

\def\qandq{\quad\text{and}\quad}
\def\ov#1{\overline{#1}}
\def\empty{\varnothing}

\newcommand{\cat}[1]{\mathsf{#1}}
\def\psh{\cat{PSh}}
\def\sh{\cat{Sh}}

\def\NN{\mathbf{N}}
\def\ZZ{\mathbf{Z}}
\def\QQ{\mathbf{Q}}
\def\RR{\mathbf{R}}
\def\CC{\mathbf{C}}
\def\PP{\mathbf{P}}

\DeclareMathOperator{\Alt}{Alt}
\DeclareMathOperator{\et}{Et}
\DeclareMathOperator{\re}{Re}
\DeclareMathOperator{\Hess}{Hess}
\DeclareMathOperator{\pr}{pr}
\DeclareMathOperator{\Span}{span}
\DeclareMathOperator{\Gr}{Gr}
\DeclareMathOperator{\GL}{GL}
\DeclareMathOperator{\im}{Im}
\DeclareMathOperator{\Vol}{Vol}
\DeclareMathOperator{\Ker}{Ker}
\DeclareMathOperator{\Img}{Im}
\DeclareMathOperator{\End}{End}
\DeclareMathOperator{\Aut}{Aut}
\DeclareMathOperator{\Hom}{Hom}
\DeclareMathOperator{\Sym}{Sym}
\DeclareMathOperator{\id}{id}
\DeclareMathOperator{\tr}{tr}
\DeclareMathOperator{\adj}{adj}
\DeclareMathOperator{\Spec}{Spec}

% Projective spaces
\newcommand{\PC}[1]{\PP(\CC^{#1})}
\newcommand{\PV}[1]{\PP(#1)}

\def\<{\langle}
\def\>{\rangle}
\def\d{\mathrm{d}}

\newcommand{\ff}[1]{\mathfrak{#1}}


\author{Gunnar \TH\'or Magn\'usson}
\date{\today}
\title{Constructing some non-K\"ahler metrics}

\begin{document}

\maketitle


\section{Subbundles}


\begin{prop}
Let $X$ be a complex manifold and $E \to X$ a holomorphic vector bundle with a Hermitian metric $h$.
Suppose $f : T_X \to E$ is an injective bundle map.
Then the induced metric on $X$ is K\"ahler if and only if the second fundamental form of $T_X$ in $E$ is symmetric.
\end{prop}

\begin{proof}
Let
\[
\begin{tikzcd}
0 \ar[r] &
T_X \ar[r,"f"] &
E \ar[r,"q"] &
Q \ar[r] &
0
\end{tikzcd}
\]
be the short exact sequence associated to $f$.
The second fundamental form is
\[
b(\xi, \eta)
= q(D_{E,\xi} f \eta).
\]
The connection of the metric on $X$ satisfies 
\[
D_E f\xi = f D_X \xi - q^\dagger b(\xi)
\]
so
\[
D_{E,\xi} f\eta
- D_{E,\eta} f\xi 
= f D_{X,\xi} \eta
- f D_{X,\eta} \xi 
- q^\dagger b(\xi,\eta)
+ q^\dagger b(\eta,\xi)
\]
and thus
\[
d^E f(\xi,\eta)
= f \tau(\xi,\eta)
- q^\dagger b(\xi,\eta)
+ q^\dagger b(\eta,\xi).
\]
\end{proof}


What is $d^E f$?
Take coordinates $(z_1,\ldots,z_n)$ around $x$ and a frame $(e_1,\ldots,e_r)$ normal at $x$.
Write $f(\partial / \partial z_j) = \sum_k f_{jk} e_k$.
Then
\[
D_{E,\xi_k} f(\xi_j)
= \sum_{l} d_{\xi_k}f_{jl} \otimes e_l + f_{jl} D_{\xi_k} e_l
\]
so
\begin{align*}
d^E f(\xi_j, \xi_k)
&= 
D_{E,\xi_k} f(\xi_j)
- D_{E,\xi_k} f(\xi_j)
- f([\xi_j, \xi_k])
\\
&= 
\sum_{l} d_{\xi_k}f_{jl} \otimes e_l 
+ f_{jl} D_{\xi_k} e_l
\\
& \qquad\qquad
- \sum_{l} d_{\xi_j}f_{kl} \otimes e_l 
+ f_{kl} D_{\xi_j} e_l
- f([\xi_j, \xi_k])
\\
&= 
\sum_{l}\bigl(
d_{\xi_k}f_{jl}
- d_{\xi_j}f_{kl} 
\bigr ) \otimes e_l 
- f([\xi_j, \xi_k])
\\
&\qquad\qquad
+ 
\sum_{l}
f_{jl} D_{\xi_k} e_l
+ f_{kl} D_{\xi_j} e_l
\end{align*}




\section{Smooth}

Let $M$ be a manifold and $E \to M$ a vector bundle with an injective bundle morphism $f : T_M \to E$.
Suppose we are given:
\begin{itemize}
\item
A metric $h$ on $E$.

\item
A connection $D$ compatible with $h$.

\item
A connection $\nabla$ compatible with the induced metric $g = f^*h$.
\end{itemize}
The connections give a connection $D_{\Hom}$ on the bundle $\Hom(T_M, E)$.
Applying it to $f$ we get
\[
D_{\Hom}f
\in \cc A^1(\Hom(T_M, E))
\cong (T_M^*)^{\otimes 2} \otimes E.
\]
Recall also the exterior covariant derivative $d^D : \cc A^k(E) \to \cc A^{k+1}(E)$ associated a connection $D$ on $E$, and note that $\Hom(T_M, E) \cong T_M^* \otimes E = \cc A^1(E)$.

We write $\Alt : V \otimes V \to \bigwedge^2 V$ for the projection onto the quotient of alternating $2$-forms for a vector space $V$, that is, $\Alt v \otimes w = v \otimes w - w \otimes v$.\footnote{The projection onto the subspace of $2$-forms also involves dividing by $2!$.}


\begin{prop}
The torsion tensor $\tau$ of $g$ satisfies
\[
f \tau
= d^D f
- \Alt D_{\Hom}f.
\]
\end{prop}

\begin{proof}
We have
\[
D(f(\xi))
= D_{\Hom}f \xi + f(\nabla \xi)
\]
so
\begin{align*}
f \tau(\xi,\nu)
&= D_\xi f(\nu) 
- D_{\Hom,\xi} f(\nu)
- D_\nu f(\xi) 
+ D_{\Hom,\nu} f(\xi)
- f [\xi, \nu]
\\
&= 
d^D f(\xi, \nu)
- \Alt D_{\Hom}f (\xi, \nu).
\end{align*}
as claimed.
\end{proof}

Therefore $\tau = 0$ if and only if
\[
d^D f = D_{\Hom} f
\]
in $\bigwedge^2 T_X^* \otimes E$.
($D_{\Hom}f \in (T_X^*)^{\otimes 2} \otimes E = \operatorname{Sym}^2 T_X^* \otimes E \oplus \bigwedge^2 T_X^* \otimes E$.)



Because I forget: $V$ and $W$ vector spaces with bases $(v_j)$ and $(w_k)$ for $j=1,\ldots,n$ and $k=1,\ldots,m$.
A map $f : V \to W$ gives a matrix $(f_{kj})$.
Given inner products and taking $V \cong \CC^n$ and $W \cong \CC^m$ we get an adjoint $f^\dagger : W \to V = \ov f^t = (g_{jk})$, where $g_{jk} = \bar f_{kj}$.
Then
\[
f^\dagger f : V \to V
= \biggl(\sum_{l=1}^m g_{kl} f_{lj}\biggr)_{jk}
= \biggl(\sum_{l=1}^m \bar f_{lk} f_{lj}\biggr)_{jk}.
\]
I'm beginning to think there is no special condition here.
If our bundle is Hermitian and the manifold complex then we pick a point $x$, coordinates around it and a normal frame.
The matrix of the metric on $X$ in this setup is
\[
H = \ov F^t F + O(|z|^2),
\]
where $F$ is the matrix of $f : T_X \to E$.
The components of $H$ are then
\[
h_{jk} = \sum_{m=1}^r \bar f_{km} f_{jm} + O(|z|^2).
\]
Then
\[
\frac{\partial H}{\partial z_l}
= \frac{\partial \ov{F^t}}{\partial z_l} F
+ \ov{F^t} \frac{\partial F}{\partial z_l}
+ O(|z|).
\]
If $\ov F$ is holomorphic this simplifies to
\[
\frac{\partial H}{\partial z_l}
= \frac{\partial \ov F^t}{\partial z_l} F
+ O(|z|)
\]
so $H$ is K\"ahler at $x$ if and only if
\[
\sum_{m=1}^r \frac{\partial \bar f_{km}}{\partial z_l} f_{jm}
= \frac{\partial h_{jk}}{\partial z_l}
= \frac{\partial h_{lk}}{\partial z_j}
= \sum_{m=1}^r \frac{\partial \bar f_{km}}{\partial z_j} f_{lm}
\]
for all $j,k,l = 1, \ldots, n$ at $x$.
We could also say that
\[
\frac{\partial H}{\bar \partial z_m}
= \ov F^t \frac{\partial F}{\partial \bar z_m}
+ O(|z|)
\]
so $H$ is K\"ahler at $x$ if and only if
\[
\sum_{l=1}^r \bar f_{kl} \frac{\partial f_{jl}}{\partial \bar z_m}
= \frac{\partial h_{jk}}{\partial \bar z_m}
= \frac{\partial h_{jm}}{\partial \bar z_k}
= \sum_{l=1}^r \bar f_{ml} \frac{\partial f_{jl}}{\partial \bar z_k}
\]
for all $j,k,l = 1, \ldots, n$ at $x$.





\section{Metrics induced by covering bundles}



\begin{prop}
Let $X$ be a K\"ahler manifold and let
\[
\begin{tikzcd}
0 \ar[r] &
S \ar[r,"j"] &
E \ar[r,"q"] &
T_X \ar[r] &
0
\end{tikzcd}
\]
a short exact sequence of vector bundles.
If there exists a Hermitian metric on $E$ such that the induced metric on $X$ is K\"ahler then the second fundamental form of the sequence is holomorphic to order one.
\end{prop}



\begin{proof}
For any section $s$ of $E$ we have 
\[
D_X q(s)
= D_{\Hom}q(s) + q(D_E s)
= - b j^\dagger(s) + q(D_E s).
\]
The metric on $X$ is K\"ahler at at point $x$ if and only if there exists normal coordinates $(z_1,\ldots,z_n)$ centered at $x$.
Given any coordinate system we can find a normal frame $(e_1,\ldots,e_r)$ for $E$ at $x$.
Writing $s = \sum s_l e_l$ we see that
\[
D_E s 
= \sum ds_l \otimes e_l + s_l \otimes D_E e_l
= \sum ds_l \otimes e_l
\]
at the center.
Write $q(e_k) = \sum_l q_{lk} \partial / \partial z_l$ with $q_{lk}$ holomorphic.
Then
\[
D_X q(e_k)
= \sum_l \partial q_{lk} \otimes \frac{\partial}{\partial z_l}
\qandq
b j^\dagger(e_k)
= -\sum_l \partial q_{lk} \otimes \frac{\partial}{\partial z_l}
\]
at $x$.
Let $s$ be in $\cc O(S)$ and write $j(s) = \sum s_k e_k$ with $s_k$ holomorphic.
Then
\[
b(s)
= -\sum_{k,l} s_k \partial q_{lk} \otimes \frac{\partial}{\partial z_l}
\]
at $x$.
Hidden in this expression are non-holomorphic terms from $D_X$ that vanish at $x$, so $b$ is holomorphic to order one.
\end{proof}



\begin{exam}[Curve your enthusiasm]
It would have been neat if the induced metric being K\"ahler meant that the short exact sequence splits.
This is not true:
Consider a curve $C$ and extensions
\[
\begin{tikzcd}
0 \ar[r] &
\CC \ar[r] &
E \ar[r] &
T_C \ar[r] &
0
\end{tikzcd}
\]
of the trivial line bundle by $T_C$.
These are classified by 
\[
H^1(C, \Hom(T_C, \CC)) = H^{1,1}(C,\CC)
\]
which has dimension $1$, so there are nonsplit such extensions.
However any metric on $E$ induces a K\"ahler metric on $C$ for dimension reasons.
\end{exam}




\section{Some non-K\"ahler metrics on projective varieties}

Let $X$ be a projective manifold of dimension $\dim_{\CC} X = n$.
Let $L \to X$ be an ample line bundle and $h$ a positively curved metric on $L$.
Since $L$ is ample there is a $k_0$ such that $T_X \otimes kL$ is globally generated for $k \geq k_0$.
We may also suppose $kL$ is very ample for $k \geq k_0$.
Then there is a surjective map
\[
H^0(X, T_X \otimes kL) 
\longrightarrow T_X \otimes kL
\]
and thus a short exact sequence
\[
\begin{tikzcd}
0 \ar[r] &
S \ar[r] &
H^0(X, T_X \otimes kL) \otimes -kL \ar[r,"q"] &
T_X \ar[r] &
0.
\end{tikzcd}
\]
If we fix some Hermitian metric $\omega$ on $X$ we get an $L^2$-inner product on the global sections of $T_X \otimes kL$, so we have a Hermitian metric on the middle bundle.
It induces another Hermitian metric $\omega_{h,k}$ on $X$.

We've been ignoring the restriction map $H^0(X, T_X \otimes kL) \to T_X \otimes kL$ when calculating.
Is that all right?


Suppose first that $T_X$ is globally generated.
There is a surjective map $H^0(T_X) \otimes \cc O_X \to T_X$.
For any point $x$ there is a neighborhood $U$ of $x$ such that $r: H^0(X,T_X) \to T_X(U)$ is surjective.
Take orthogonal sections $(e_1, \ldots, e_r)$ that form a parallel frame of $H^0(T_X)$.
That frame is normal at any point.
Any section $\xi$ of $T_X(U)$ can be written $\xi = \sum_{j=1}^r a_j e_{j|U}$.
In particular we have $\partial / \partial z_k = \sum_{j=1}^r a_{jk} e_{j|U}$.
Write $r(e_j) = e_{j|U} = \sum_{k=1}^n r_{kj} \partial / \partial z_k$.
There are smooth $a_{jk}$ such that $r(a(\xi)) = \xi$ and
\[
\omega_{jk}
= \Bigl\<
\frac{\partial}{\partial z_j}, \ov{\frac{\partial}{\partial z_k}}
\Bigr\>
= \sum_{l,m} a_{jl} \bar a_{km} h(e_{l|U}, \ov e_{m|U})
= \sum_{l} a_{jl} \bar a_{kl}.
\]
Then
\[
\frac{\partial \omega_{jk}}{\partial z_m}
= \sum_{l=1}^r a_{jl} \frac{\partial \bar a_{kl}}{\partial z_m}
\]
because $a_{jl}$ is antiholomorphic.
The $a_{jk}$ satisfy $h(s, \ov{a(\nu)}) = 0$ for any $\nu \in T_X(U)$ and $s \in \Ker r$.
Writing $s = \sum s_l e_l$ we get
\[
\sum_{l} s_l \ov a_{kl} = 0
\]
so
\[
\sum_l \frac{\partial s_l}{\partial z_m} \ov a_{kl} 
+ s_l \frac{\partial \ov a_{kl}}{\partial z_m} = 0
\]
for any $s \in \Ker r$.



Switch gears:
Suppose $f : T_X \to E$ is injective and holomorphic.
Pick a frame $(e_1, \ldots, e_r)$ of $E$ that is normal at $x$.
The coefficients of the matrix of the metric on $X$ are
\[
h_{jk} 
= \sum_{l,m=1}^r f_{jl} \bar f_{km} \< e_l, \bar e_m \>
= \sum_{l=1}^r f_{jl} \bar f_{kl} + O(|z|^2)
\]
at $x$, so the metric is holomorphic there if
\[
\sum_{l=1}^r \frac{\partial f_{jl}}{\partial z_m} \bar f_{kl} 
= \sum_{l=1}^r \frac{\partial f_{ml}}{\partial z_j} \bar f_{kl} 
\]
for all $j,k,m$, which happens when
\[
\sum_{l=1}^r \biggl(
\frac{\partial f_{jl}}{\partial z_m}
- \frac{\partial f_{ml}}{\partial z_j}
\biggr) \bar f_{kl} 
= 0.
\]
Then the vectors
\[
v_{jm} = 
\biggl(
\frac{\partial f_{j1}}{\partial z_m}
- \frac{\partial f_{m1}}{\partial z_j},
\ldots,
\frac{\partial f_{jr}}{\partial z_m}
- \frac{\partial f_{mr}}{\partial z_j}
\biggr)
\]
are in the kernel of $\ov F^t : E_x \to T_{X,x}$.
There are $n^2$ of these vectors, but $v_{jj} = 0$ for all $j$, so we have $n(n-1)$ possibly nonzero vectors.
Since $f$ is injective $F^t$ is surjective, so $\dim \Ker \ov F^t = \dim E_x - \dim T_{X,x} = r - n$.
Alternatively, the kernel of that map is $f(T_X)^\perp$.



\begin{prop}
The metric $\omega_{h,k}$ is not K\"ahler.
\end{prop}

\begin{proof}
The bundle $T_X \otimes kL$ is globally generated.
Let $(e_1, \ldots, e_r)$ be a basis of $H^0(X,T_X \otimes kL)$.
The $L^2$-metric defines a flat Hermitian metric on the associated trivial bundle and the sections $e_j$ are parallel with respect to the Chern connection of the metric.

Let $U$ be a coordinate neighborhood that trivializes $kL$ by a section $s$, and $\tau$ be a section of $-kL$ such that $\tau(s) = 1$ on $U$.
We pick the section $s$ (and thus $\tau$) to define a normal frame at $x$.
Any local section of $H^0(X, T_X \otimes kL) \otimes -kL$ can we written as $v = \sum f_j e_j \otimes \tau$, where $f_j$ are holomorphic.
For these we have
\[
D_E\Bigl( \sum f_j e_j \otimes \tau \Bigr)
= \sum d f_j \otimes e_j \otimes \tau + f_j e_j \otimes D_{-kL} \tau.
\]
Generally if $f : E \to Q$ is a map we have
\[
D_Q(f(e))
= D_{\Hom}f(e) + f(D_E e).
\]
If $g$ is a smooth function we then get
\begin{align*}
D_{\Hom}f(ge)
&= D_Q(f(ge)) - f(D_E(ge))
\\
&= dg \otimes f(e) + g D_Q(f(e)) - gd \otimes f(e) - g f(D_E(e))
\\
&= g \bigl( D_Q(f(e)) - f(D_E(e)) \bigr)
= g D_{\Hom} f(e).
\end{align*}
Therefore
\[
D_{\Hom}q \biggl( \sum_j f_j e_j \otimes \tau \biggr)
= \sum_j f_j \bigl( D_X q(e_j \otimes \tau)
- q(e_j \otimes D_{-kL}\tau) \bigr).
\]
Note that we can write $D_{-kL}\tau = \sigma \otimes \tau$ for a $(1,0)$-form $\sigma$.
So in fact
\[
D_{\Hom}q \Bigl( \sum_j f_j e_j \otimes \tau \Bigr)
= \sum_j f_j D_X q(e_j \otimes \tau)
- \sigma \otimes q\Bigl(\sum_j f_j e_j \otimes \tau\Bigr).
\]
Taking our section to be in $\Ker q$ we then get
\[
b\Bigl( \sum_j f_j e_j \otimes \tau \Bigr)
= \sum_j f_j D_X q(e_j \otimes \tau).
\]



If $e_j \in H^0(X,T_X \otimes kL)$ we can write
\[
e_{j|U} = \sum_k g_{kj} \frac{\partial}{\partial z_k} \otimes s
\]
for holomorphic functions $g_{kj}$.
Once restricted to $U$ we then have
\[
q\Bigl(\sum_{j} f_j e_{j} \otimes \tau \Bigr) 
= \sum_{j,k} f_j g_{kj} \frac{\partial}{\partial z_k}
\]
so the matrix of $q$ is
\[
Q = (g_{kj})
\]
for $j = 1, \ldots, r$ and $k = 1, \ldots, n$.
The matrix of the metric on $X$ is then
\[
H = Q \ov Q^t + O(|z|^2)
= \biggl(
\sum_{m=1}^r g_{km} \ov g_{jm}
\biggr) + O(|z|^2)
\]
around $x$ so the metric is K\"ahler there if and only if
\[
\sum_{m=1}^r \frac{g_{km}}{\partial z_l} \ov g_{jm}
= \sum_{m=1}^r \frac{g_{km}}{\partial z_j} \ov g_{lm}
\]
for $j,k,l = 1,\ldots,r$.

Pick a frame $(e_1, \ldots, e_n)$ of $T_X \otimes kL$ that is normal at $x$.
Pick another frame $(s_1, \ldots, s_r)$ of $H^0(T_X \otimes kL) \otimes -kL$ such that $s_{j|U} = e_j$.
Can't choose $s_j$ parallel but probably want to, so get some matrix $R$ with linear combinations of $s_j$.
This is confused.





For the Chern connection of the metric on $X$ we then get
\[
D_X q\Bigl(\sum_{j} f_j e_{j} \otimes \tau \Bigr) 
= \sum_{j,k} g_{kj} df_j \frac{\partial}{\partial z_k}
+ \sum_{j,k} f_j dg_{kj} \frac{\partial}{\partial z_k}
+ \sum_{j,k} f_j g_{kj} D\frac{\partial}{\partial z_k}.
\]
Meanwhile,
\[
q D_E\Bigl( \sum f_j e_j \otimes \tau \Bigr)
= \sum_{j,k} g_{jk} d f_j \frac{\partial}{\partial z_k}
+ f_j g_{kj} (D \tau)(s) \frac{\partial}{\partial z_k} .
\]


Note that $\eta \mapsto \tau(D_\tau s)$ is a linear functional.
If $\eta$ is in its kernel we get
\[
D_{\Hom,\eta}q ( e_j \otimes \tau )
= \sum_{k} D_{X,\eta}\Bigl(g_{kj} \frac{\partial}{\partial z_k}\Bigr).
\]
If now $\nu$ is not in its kernel we have
\[
D_{\Hom,\nu}q ( e_j \otimes \tau )
= \sum_{k} D_{X,\nu}\Bigl(g_{kj} \frac{\partial}{\partial z_k}\Bigr)
+ g_{jk} \tau(D_\nu s) \frac{\partial}{\partial z_k}.
\]
For any $\xi$ we can then write
\[
\xi = \frac{\tau(D_\xi s)}{\tau(D_\nu s)} \nu
+ \biggl(\xi - \frac{\tau(D_\xi s)}{\tau(D_\nu s)} \nu \biggr)
\in \cc O \nu \oplus \Ker \tau(Ds).
\]
Then
\[
D_{\Hom,\xi} q(e_j \otimes \tau)
= \frac{\tau(D_\xi s)}{\tau(D_\nu s)} D_{\Hom,\nu} q(e_j \otimes \tau)
+
\]
If our metric is K\"ahler then these two are the same up to some multiple, which we may take to be $1$ after scaling $\eta$.
If so we have
\[
\sum_{k} D_{X,\nu-\eta}\Bigl(g_{kj} \frac{\partial}{\partial z_k}\Bigr)
+ g_{jk} \tau(D_\nu s) \frac{\partial}{\partial z_k}
= 0
\]
and so we also have
\[
D_{X,\eta} q(e_j \otimes \tau)
= \sum_{k} D_{X,\eta}\Bigl(g_{kj} \frac{\partial}{\partial z_k}\Bigr)
= 0
\]
for any $\eta$ in the kernel of $\tau(D s)$.

If $\xi \in T_X$ then $\xi = q(\sum_j f_j e_j \otimes \tau)$ for some holomorphic $f_j$.
Then
\[
D \xi 
= D \sum_j f_j q(e_j \otimes \tau)
= \sum_j df_j \otimes q(e_j \otimes \tau)
+ f_j D q(e_j \otimes \tau),
\]
so taking $\eta = q(\sum_j h_j e_j \otimes \tau)$ in the kernel we get
\[
0 
= \tau(\xi,\eta) 
= D_\eta \xi - D_\xi \eta - [\xi, \eta]
= \sum_{j=1}^r h_j D_\xi q(e_j \otimes \tau).
\]
I feel like once $r = \dim H^0(X, T_X \otimes kL) > 2n$ this gives too many equations for the $D_\xi q(e_j \otimes \tau)$ to satisfy, so it should force $D_\xi q(e_j \otimes t) = 0$ for all $j$.
Then our metric would have to be flat, which doesn't make sense unless we're on a torus.

If $\lambda$ is a linear form and $\lambda(v) \not= 0$ then for any other $w$ we can write
\[
w = \frac{\lambda(w)}{\lambda(v)} v
+ \biggl( w - \frac{\lambda(w)}{\lambda(v)} v \biggr)
\in \CC v \oplus \Ker \lambda.
\]
\end{proof}




\end{document}
